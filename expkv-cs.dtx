% \iffalse meta-comment
%
% File: expkv-cs.dtx Copyright (C) 2020-2021 Jonathan P. Spratte
%
% This work  may be  distributed and/or  modified under  the conditions  of the
% LaTeX Project Public License (LPPL),  either version 1.3c  of this license or
% (at your option) any later version.  The latest version of this license is in
% the file:
%
%   http://www.latex-project.org/lppl.txt
%
% ------------------------------------------------------------------------------
%
%<*driver>^^A>>=
\def\nameofplainTeX{plain}
\ifx\fmtname\nameofplainTeX\else
  \expandafter\begingroup
\fi
\input l3docstrip.tex
\askforoverwritefalse
\preamble

--------------------------------------------------------------
expkv-cs -- define expandable key=val macros using expkv
E-mail: jspratte@yahoo.de
Released under the LaTeX Project Public License v1.3c or later
See http://www.latex-project.org/lppl.txt
--------------------------------------------------------------

Copyright (C) 2020-2021 Jonathan P. Spratte

This  work may be  distributed and/or  modified under  the conditions  of the
LaTeX Project Public License (LPPL),  either version 1.3c  of this license or
(at your option) any later version.  The latest version of this license is in
the file:

  http://www.latex-project.org/lppl.txt

This work is "maintained" (as per LPPL maintenance status) by
  Jonathan P. Spratte.

This work consists of the file  expkv-cs.dtx
and the derived files           expkv-cs.pdf
                                expkv-cs.sty
                                expkv-cs.tex

\endpreamble
% stop docstrip adding \endinput
\postamble
\endpostamble
\generate{\file{expkv-cs.sty}{\from{expkv-cs.dtx}{pkg}}}
\generate{\file{expkv-cs.tex}{\from{expkv-cs.dtx}{tex}}}
\ifx\fmtname\nameofplainTeX
  \expandafter\endbatchfile
\else
  \expandafter\endgroup
\fi
%
\IfFileExists{expkv-cs.tex}{% \iffalse meta-comment
%
% File: expkv-cs.dtx Copyright (C) 2020-2021 Jonathan P. Spratte
%
% This work  may be  distributed and/or  modified under  the conditions  of the
% LaTeX Project Public License (LPPL),  either version 1.3c  of this license or
% (at your option) any later version.  The latest version of this license is in
% the file:
%
%   http://www.latex-project.org/lppl.txt
%
% ------------------------------------------------------------------------------
%
%<*driver>^^A>>=
\def\nameofplainTeX{plain}
\ifx\fmtname\nameofplainTeX\else
  \expandafter\begingroup
\fi
\input l3docstrip.tex
\askforoverwritefalse
\preamble

--------------------------------------------------------------
expkv-cs -- define expandable key=val macros using expkv
E-mail: jspratte@yahoo.de
Released under the LaTeX Project Public License v1.3c or later
See http://www.latex-project.org/lppl.txt
--------------------------------------------------------------

Copyright (C) 2020-2021 Jonathan P. Spratte

This  work may be  distributed and/or  modified under  the conditions  of the
LaTeX Project Public License (LPPL),  either version 1.3c  of this license or
(at your option) any later version.  The latest version of this license is in
the file:

  http://www.latex-project.org/lppl.txt

This work is "maintained" (as per LPPL maintenance status) by
  Jonathan P. Spratte.

This work consists of the file  expkv-cs.dtx
and the derived files           expkv-cs.pdf
                                expkv-cs.sty
                                expkv-cs.tex

\endpreamble
% stop docstrip adding \endinput
\postamble
\endpostamble
\generate{\file{expkv-cs.sty}{\from{expkv-cs.dtx}{pkg}}}
\generate{\file{expkv-cs.tex}{\from{expkv-cs.dtx}{tex}}}
\ifx\fmtname\nameofplainTeX
  \expandafter\endbatchfile
\else
  \expandafter\endgroup
\fi
%
\IfFileExists{expkv-cs.tex}{% \iffalse meta-comment
%
% File: expkv-cs.dtx Copyright (C) 2020-2021 Jonathan P. Spratte
%
% This work  may be  distributed and/or  modified under  the conditions  of the
% LaTeX Project Public License (LPPL),  either version 1.3c  of this license or
% (at your option) any later version.  The latest version of this license is in
% the file:
%
%   http://www.latex-project.org/lppl.txt
%
% ------------------------------------------------------------------------------
%
%<*driver>^^A>>=
\def\nameofplainTeX{plain}
\ifx\fmtname\nameofplainTeX\else
  \expandafter\begingroup
\fi
\input l3docstrip.tex
\askforoverwritefalse
\preamble

--------------------------------------------------------------
expkv-cs -- define expandable key=val macros using expkv
E-mail: jspratte@yahoo.de
Released under the LaTeX Project Public License v1.3c or later
See http://www.latex-project.org/lppl.txt
--------------------------------------------------------------

Copyright (C) 2020-2021 Jonathan P. Spratte

This  work may be  distributed and/or  modified under  the conditions  of the
LaTeX Project Public License (LPPL),  either version 1.3c  of this license or
(at your option) any later version.  The latest version of this license is in
the file:

  http://www.latex-project.org/lppl.txt

This work is "maintained" (as per LPPL maintenance status) by
  Jonathan P. Spratte.

This work consists of the file  expkv-cs.dtx
and the derived files           expkv-cs.pdf
                                expkv-cs.sty
                                expkv-cs.tex

\endpreamble
% stop docstrip adding \endinput
\postamble
\endpostamble
\generate{\file{expkv-cs.sty}{\from{expkv-cs.dtx}{pkg}}}
\generate{\file{expkv-cs.tex}{\from{expkv-cs.dtx}{tex}}}
\ifx\fmtname\nameofplainTeX
  \expandafter\endbatchfile
\else
  \expandafter\endgroup
\fi
%
\IfFileExists{expkv-cs.tex}{% \iffalse meta-comment
%
% File: expkv-cs.dtx Copyright (C) 2020-2021 Jonathan P. Spratte
%
% This work  may be  distributed and/or  modified under  the conditions  of the
% LaTeX Project Public License (LPPL),  either version 1.3c  of this license or
% (at your option) any later version.  The latest version of this license is in
% the file:
%
%   http://www.latex-project.org/lppl.txt
%
% ------------------------------------------------------------------------------
%
%<*driver>^^A>>=
\def\nameofplainTeX{plain}
\ifx\fmtname\nameofplainTeX\else
  \expandafter\begingroup
\fi
\input l3docstrip.tex
\askforoverwritefalse
\preamble

--------------------------------------------------------------
expkv-cs -- define expandable key=val macros using expkv
E-mail: jspratte@yahoo.de
Released under the LaTeX Project Public License v1.3c or later
See http://www.latex-project.org/lppl.txt
--------------------------------------------------------------

Copyright (C) 2020-2021 Jonathan P. Spratte

This  work may be  distributed and/or  modified under  the conditions  of the
LaTeX Project Public License (LPPL),  either version 1.3c  of this license or
(at your option) any later version.  The latest version of this license is in
the file:

  http://www.latex-project.org/lppl.txt

This work is "maintained" (as per LPPL maintenance status) by
  Jonathan P. Spratte.

This work consists of the file  expkv-cs.dtx
and the derived files           expkv-cs.pdf
                                expkv-cs.sty
                                expkv-cs.tex

\endpreamble
% stop docstrip adding \endinput
\postamble
\endpostamble
\generate{\file{expkv-cs.sty}{\from{expkv-cs.dtx}{pkg}}}
\generate{\file{expkv-cs.tex}{\from{expkv-cs.dtx}{tex}}}
\ifx\fmtname\nameofplainTeX
  \expandafter\endbatchfile
\else
  \expandafter\endgroup
\fi
%
\IfFileExists{expkv-cs.tex}{\input{expkv-cs.tex}}{}
\ProvidesFile{expkv-cs.dtx}
  [\csname ekvcDate\endcsname\ define expandable key=val macros using expkv]
\PassOptionsToPackage{full}{textcomp}
\documentclass{l3doc}
\RequirePackage[oldstylenums,nott]{kpfonts}
\input{glyphtounicode}
\pdfgentounicode=1
\RequirePackage{tcolorbox}
\newtcolorbox{exresult}[2][]
  {colback=expkvgrey!10!white,colframe=expkvgrey,fontupper=\small,width={#2},#1}
\newtcbox\exres[1][]
  {
     colback=expkvgrey!10!white
    ,colframe=expkvgrey
    ,size=small
    ,nobeforeafter
    ,tcbox raise base
    ,fontupper=\small
    ,#1
  }
\RequirePackage{listings}
\RequirePackage{booktabs}
\RequirePackage{array}
\RequirePackage{collcell}
\RequirePackage{siunitx}
\RequirePackage{xcolor}
\RequirePackage{caption}
\RequirePackage{microtype}
\RequirePackage{accsupp}
\lstdefinelanguage{expkvc}[primitive]{TeX}
  {
    ,moretexcs=[4]^^A e-TeX
      {
        expanded,
        numexpr,
        protected,
      }
    ,moretexcs=[5]^^A plain/LaTeX
      {
        approx,
        begin,
        empty,
        item,
        LaTeX,
        makeatletter,makeatother,multicolumn,
        newcommand,newdimen,
        RequirePackage,
        rule,
        TeX,
        textit,texttt,
        usepackage,
      }
    ,moretexcs=[6]^^A used packages
      {
        ^^A xfp
        fpeval,
        ^^A keyval
        setkeys,
        ^^A yax
        defactiveparameter,storevalue,setparameterlist
      }
    ,moretexcs=[2]^^A expkv-cs macros
      {
        ekvcSplit,ekvcSplitAndForward,ekvcSplitAndUse,
        ekvcHash,ekvcHashAndForward,ekvcHashAndUse,
        ekvcValue,ekvcValueFast,ekvcValueSplit,ekvcValueSplitFast,
        ekvcSecondaryKeys,
        ekvcChange,
        ekvcFlagNew,
        ekvcFlagHeight,
        ekvcFlagRaise,
        ekvcFlagSetTrue,ekvcFlagSetFalse,
        ekvcFlagIf,ekvcFlagIfRaised,
        ekvcFlagReset,
        ekvcFlagGetHeight,ekvcFlagGetHeights,
        ekvcPass
      }
    ,moretexcs=[3]^^A expkv-pkg macros
      {
        ^^A expkv
        ekvdef,ekvdefNoVal,^^A
        ekvlet,ekvletNoVal,ekvletkv,ekvletkvNoVal,^^A
        ekvdefunknown,ekvdefunknownNoVal,^^A
        ekvredirectunknown,ekvredirectunknownNoVal,^^A
        ekvset,ekvsetSneaked,^^A
        ekvsetdef,ekvsetSneakeddef,ekvsetdefSneaked,^^A
        ekvparse,^^A
        ekvVersion,ekvDate,^^A
        ekvifdefined,ekvifdefinedNoVal,ekvifdefinedset,^^A
        ekvbreak,ekvbreakPreSneak,ekvbreakPostSneak,^^A
        ekvsneak,ekvsneakPre,^^A
        ekvchangeset,^^A
        ekvoptarg,ekvoptargTF,^^A
        ekverr,
        ^^A expkv-def
        ekvdefinekeys
      }
  }
\colorlet{codeparam}{cyan!65!black}
\def\mylstwd{.55em}
\lstset
  {
    ,language=expkvc
    ,flexiblecolumns=false
    ,basewidth=\mylstwd
    ,gobble=2
    ,basicstyle=\fontfamily{jkp}\itshape
    ,texcsstyle=*[1]{\bfseries\color{expkvgrey}} ^^A primitives
    ,texcsstyle=*[2]{\bfseries\color{expkvred}}  ^^A expkv
    ,texcsstyle=*[3]{\color{expkvred}}           ^^A expkv-pkg
    ,texcsstyle=*[4]{\bfseries\color{expkvgrey}} ^^A e-TeX
    ,texcsstyle=*[5]{\bfseries\color{expkvgrey}} ^^A plain/LaTeX macros
    ,texcsstyle=*[6]{}                           ^^A macros of other packages
    ,commentstyle=\color[gray]{0.4}
    ,literate=
              {\{} {{\CodeUpBf\{}}{1}
              {\}} {{\CodeUpBf\}}}{1}
              {$}  {{\CodeUpBf\$}}{1}
              {[}  {{\CodeUp[}}{1}
              {]}  {{\CodeUp]}}{1}
              {(}  {{\CodeUp(}}{1}
              {)}  {{\CodeUp)}}{1}
              {*}  {{$*$}}{1}
              {1}  {{\CodeUp{1}}}{1}
              {2}  {{\CodeUp{2}}}{1}
              {3}  {{\CodeUp{3}}}{1}
              {4}  {{\CodeUp{4}}}{1}
              {5}  {{\CodeUp{5}}}{1}
              {6}  {{\CodeUp{6}}}{1}
              {7}  {{\CodeUp{7}}}{1}
              {8}  {{\CodeUp{8}}}{1}
              {9}  {{\CodeUp{9}}}{1}
              {0}  {{\CodeUp{0}}}{1}
              {##} {{\CodeColored{codeparam}{1}{\#}}}{1}
              {##1}{{\CodeColored{codeparam}{2}{\#1}}}{2}
              {##2}{{\CodeColored{codeparam}{2}{\#2}}}{2}
              {##3}{{\CodeColored{codeparam}{2}{\#3}}}{2}
              {##4}{{\CodeColored{codeparam}{2}{\#4}}}{2}
              {##5}{{\CodeColored{codeparam}{2}{\#5}}}{2}
              {##6}{{\CodeColored{codeparam}{2}{\#6}}}{2}
              {##7}{{\CodeColored{codeparam}{2}{\#7}}}{2}
              {##8}{{\CodeColored{codeparam}{2}{\#8}}}{2}
              {##9}{{\CodeColored{codeparam}{2}{\#9}}}{2}
              {<key>}{{$\langle$}key{$\rangle$}}{5}
              {<set>}{{$\langle$}set{$\rangle$}}{5}
              {<set1>}{{$\langle$}set1{$\rangle$}}{6}
              {<set2>}{{$\langle$}set2{$\rangle$}}{6}
    ^^A,literate=*{<key>}{\key}{4}{<set>}{\set}{4}
  }
\newcommand*\CodeColored[3]{\textcolor{#1}{\makebox[\dimexpr\mylstwd*#2]{#3}}}
\newcommand*\CodeUpBf[1]{\makebox[\mylstwd]{\textup{\textbf{#1}}}}
\newcommand*\CodeUp[1]{\makebox[\mylstwd]{\textup{#1}}}
\RequirePackage{randtext}
\let\metaORIG\meta
\protected\def\meta #1{\texttt{\metaORIG{#1}}}
\renewcommand*\thefootnote{\fnsymbol{footnote}}
\definecolor{expkvred}{HTML}{9F393D}
\colorlet{expkvgrey}{black!75}
\makeatletter
\newcommand*\example{\par\smallskip\noindent\textit{Example:} \ignorespaces}
\newcommand*\expkv
  {^^A
    \texorpdfstring
      {^^A
        \mbox
          {^^A
            \BeginAccSupp{ActualText=expkv}^^A
            \href{https://github.com/Skillmon/tex_expkv}
              {^^A
                \rmfamily
                \bfseries
                {\color{expkvgrey}e\kern-.05em x\kern-.05em}^^A
                \lower.493ex
                  \hbox{{\color{expkvgrey}P}\kern-.1em{\color{expkvred}k}}^^A
                \kern-.18em{\color{expkvred}v}^^A
              }^^A
            \EndAccSupp{}^^A
          }^^A
      }
      {expkv}^^A
  }
\newcommand*\expkvpkg[1]
  {^^A
    \texorpdfstring
      {^^A
        \mbox
          {^^A
            \BeginAccSupp{ActualText=expkv-#1}^^A
            \href{https://github.com/Skillmon/tex_expkv-#1}
              {^^A
                \rmfamily
                \bfseries
                {\color{expkvgrey}e\kern-.05em x\kern-.05em}^^A
                \lower.493ex
                  \hbox{{\color{expkvgrey}P}\kern-.1em{\color{expkvred}k}}^^A
                \kern-.18em{\color{expkvred}v}^^A
                {\color{expkvgrey}^^A
                  \kern.05em\rule[-.1ex]{.08em}{1.2ex}\kern.05em\textsc{#1}^^A
                }^^A
              }^^A
            \EndAccSupp{}^^A
          }^^A
      }
      {expkv-#1}^^A
  }
\newcommand*\expkvd{\expkvpkg{def}}
\newcommand*\expkvc{\expkvpkg{cs}}
\newcommand\kv{\meta{key}=\meta{value}}
\newcommand\key{\meta{key}}
\newcommand\val{\meta{value}}
\newcommand\set{\meta{set}}
\newcommand\enflong{\textcolor{black}{long}}
\newcommand\alllong{\textcolor{gray}{long}}
\newcommand\notlong{\textcolor{red!80!black}{long}}
\newcommand\prefixes[1]
  {^^A
    \hfill
    \ifcase\numexpr#1\relax\or
      \enflong\or
      \alllong\or
      \notlong\fi
  }
\hypersetup{linkcolor=red!80!black,urlcolor=purple!80!black}
\DoNotIndex{\def,\edef,\,,\=,\begingroup,\catcode,\chardef,\csname,\endcsname}
\DoNotIndex{\endgroup,\endinput,\errmessage,\expandafter,\input,\let,\long,\z@}
\DoNotIndex{\protected,\ProvidesFile,\ProvidesPackage,\relax,\space,\advance}
\DoNotIndex{\@,\unexpanded,\string,\expanded,\dimexpr,\global,\glueexpr,\hbox}
\DoNotIndex{\numexpr,\RequirePackage,\setbox,\the,\unless,\xdef,\gdef,\newcount}
\DoNotIndex{\number}
\DoNotIndex{\^,\@firstofone,\@firstoftwo,\@gobble,\@secondoftwo}
\DoNotIndex{\ifcsname}
\DoNotIndex{\ifx}
\DoNotIndex{\ifodd}
\DoNotIndex{\ifnum}
\DoNotIndex{\ifdefined}
\DoNotIndex{\iffalse}
\DoNotIndex{\iftrue}
\DoNotIndex{\else}
\DoNotIndex{\fi}
\@gobble\fi ^^A ignoring \ifx, \ifcsname, etc., but only one \fi
\@gobble\fi ^^A ignoring \ifx, \ifcsname, etc., but only one \fi
\@gobble\fi ^^A ignoring \ifx, \ifcsname, etc., but only one \fi
\@gobble\fi ^^A ignoring \ifx, \ifcsname, etc., but only one \fi
\@gobble\fi ^^A ignoring \ifx, \ifcsname, etc., but only one \fi
\@gobble\fi ^^A ignoring \ifx, \ifcsname, etc., but only one \fi
\@ifdefinable\gobbledocstriptag{\def\gobbledocstriptag#1>{}}
\makeatother
\begin{document}
  \title
    {^^A
      \texorpdfstring
        {^^A
          \huge\expkvc
          \\[\medskipamount]
          \Large define expandable \kv\ macros using \expkv
        }{expkv-cs - define expandable key=val macros using expkv}^^A
    }
  \date{\ekvcDate\space v\ekvcVersion}
  \author{Jonathan P. Spratte\thanks{\protect\randomize{jspratte@yahoo.de}}}
  \DocInput{expkv-cs.dtx}
\end{document}
%</driver>^^A=<<
% \fi
%
% \maketitle
% \renewcommand*\thefootnote{\arabic{footnote}}
%
% \begin{abstract}
% \noindent\parfillskip=0pt
% \expkvc\ provides two small interfaces to define expandable \kv\ macros using
% \expkv. It therefore lowers the entrance boundary to expandable \kv\ macros.
% The stylised name is \expkvc\ but the files use \file{expkv-cs}, this is due
% to CTAN-rules which don't allow \string| in package names since that is the
% pipe symbol in *nix shells.
% \end{abstract}
%
% \tableofcontents
%
% \begin{documentation}^^A>>=
%
% \section{Documentation}
%
% The \expkv\ package enables the new possibility of creating \kv\ macros which
% are fully expandable. The creation of such macros is however cumbersome for
% the average user. \expkvc\ tries to step in here. It provides interfaces to
% define \kv\ macros without worrying too much about the implementation. In case
% you're wondering now, the |cs| in \expkvc\ stands for control sequence,
% because |def| was already taken by \expkvd\ and ``control sequence'' is the
% term D.\,E.\,Knuth used in his \TeX book for named commands hence macros
% (though he also used the term ``macro''). So \expkvc\ defines control
% sequences for and with \expkv.
%
% There are two different approaches supported by this package. The first is
% splitting the keys up into individual arguments, the second is providing all
% the keys as a single argument to the underlying macro and getting an
% individual \val\ by using a hash. Well, actually there is no real hash, just
% some markers which are parsed, but this shouldn't be apparent to the user, the
% behaviour matches that of a hash-table.
%
% In addition to these two methods of defining a macro with primary keys a way
% to define secondary keys, which can reference the primary ones, is provided.
% These secondary keys don't correspond to an argument or an entry in the hash
% table directly but might come in handy for the average use case. Each macro
% has its own set of primary and secondary keys.
%
% A word of advice you should consider: If your macro doesn't have to be
% expandable (and often it doesn't) consider not using \expkvc. The interface
% has some overhead (though it still is fast -- check \autoref{sec:speed}) and
% the approach has its limits in versatility. If you don't need to be
% expandable, you should consider either defining your keys manually using
% \expkv\ or using \expkvd\ for convenience. Or you resort to another \kv\
% interface. Nevertheless setting up macros with \expkvc, especially with
% |\ekvcSplit|, is very convenient in my opinion, so if you just want to define
% a single macro with just a few keys this might be the way to go.
%
% \expkvc\ is usable as generic code and as a \LaTeX\ package. It'll
% automatically load \expkv\ in the same mode as well. To use it, just use one
% of
% \begin{lstlisting}
% \usepackage{expkv-cs} % LaTeX
% \input expkv-cs       % plainTeX
% \end{lstlisting}
%
%
% \subsection{Define Macros and Primary Keys}
%
% All macros defined with \expkvc\ have to be previously undefined or have the
% |\meaning| of |\relax|. This is necessary as there is no way to automatically
% undefine keys once they are set up (neither \expkv\ nor \expkvc\ keep track of
% defined keys) -- so to make sure there are no conflicts only new definitions
% are allowed (that's not the case for individual keys, only for frontend
% macros).
%
%
% \subsubsection{Primary Keys}\label{sec:primaries}
%
% In the following descriptions there will be one argument named \meta{primary
% keys}. This argument should be a \kv\ list where each \key\ will be one
% primary key and \val\ the associated initial value. By default all keys are
% defined short, but you can define long keys by prefixing \key\ with |long|
% (\emph{e.g.}, \texttt{long name=Jonathan P. Spratte}). You only need |long| if
% the key should be able to take a |\par| token. Note however that |long| keys
% are a microscopic grain faster (due to some internals of \expkvc). Only if at
% least one of the keys was |long| the \meta{cs} in the following defining
% macros will be |\long|. For obvious reasons there is no possibility to define
% a macro or key as |\protected|.
%
% At the moment \expkvc\ doesn't require any internal keys, but I can't foresee
% whether this will be the case in the future as well, as it might turn out that
% some features I deem useful can't be implemented without such internal keys.
% Because of this, please don't use key names starting with
% \texttt{EKVC\string|} as that should be the private name space.
%
%
% \subsubsection{Split}
%
% The split variants will provide the key values as separate arguments. This
% limits the number of keys for which this is truly useful.
%
% \begin{function}{\ekvcSplit}
%   \begin{syntax}
%     \cs{ekvcSplit}\meta{cs}\marg{primary keys}\marg{definition}
%   \end{syntax}
%   This defines \meta{cs} to be a macro taking one mandatory argument which
%   should contain a \kv\ list. The \meta{primary keys} will be defined for this
%   macro (see \autoref{sec:primaries}). The \meta{definition} is the code that
%   will be executed. You can access the \val\ of a \key\ by using a macro
%   parameter from |#1| to |#9|. The order of the macro parameters will be the
%   order provided in the \meta{primary keys} list (so |#1| is the \val\ of the
%   key defined first). With |\ekvcSplit| you can define macros using at most
%   nine primary keys.
% \end{function}
% \example The following defines a macro |\foo| that takes the keys |a| and |b|
% and outputs their values in a textual form:\\
% \begin{minipage}{.75\linewidth}
% \begin{lstlisting}
% \ekvcSplit\foo{a=a,b=b}{a is #1.\par b is #2.\par}
% \foo{}
% \foo{b=e}
% \end{lstlisting}
% \end{minipage}^^A
% \begin{exresult}[nobeforeafter,box align=center]{.25\linewidth}
%   \ekvcSplit\foo{a=a,b=b}{a is #1.\par b is #2.\par}
%   \foo{}
%   \foo{b=e}
% \end{exresult}
%
% \begin{function}{\ekvcSplitAndForward}
%   \begin{syntax}
%     \cs{ekvcSplitAndForward}\meta{cs}\marg{after}\marg{primary keys}
%   \end{syntax}
%   This defines \meta{cs} to be a macro taking one mandatory argument which
%   should contain a \kv\ list. You can use as many primary keys as you want
%   with this. The primary keys will be forwarded to \meta{after} as braced
%   arguments (as many as necessary for your primary keys). The order of the
%   braced arguments will be the order of your primary key definitions. In
%   \meta{after} you can use just a single control sequence, or some arbitrary
%   stuff which will be left in the input stream before your braced values (with
%   one set of braces stripped from \meta{after}), so both of the following
%   would be fine:
% \begin{lstlisting}
% \ekvcSplitAndForward\foo\foo@aux{keyA = A, keyB = B}
% \ekvcSplitAndForward\foo{\foo@aux{more args}}{keyA = A, keyB = B}
% \end{lstlisting}
% \end{function}
% In the first case |\foo@aux| should take at least two arguments (|keyA| and
% |keyB|), in the second case at least three (|more args|, |keyA|, and |keyB|).
%
% \begin{function}{\ekvcSplitAndUse}
%   \begin{syntax}
%     \cs{ekvcSplitAndUse}\meta{cs}\marg{primary keys}
%   \end{syntax}
%   This will roughly do the same as |\ekvcSplitAndForward|, but instead of
%   specifying what will be used after splitting the keys, \meta{cs} will use
%   what follows the \kv\ list. So its syntax will be
% \end{function}
%   \begin{syntax}
%     \meta{cs}\{\key=\val, \ldots\}\marg{after}
%   \end{syntax}
% and the code in |after| should expect at least as many arguments as the number
% of keys defined for \meta{cs}.
%
% \subsubsection{Hash}
%
% The hash variants will provide the key values as a single argument in which
% you can access specific values using a special macro. The implementation might
% be more convenient and scale better, \emph{but} it is much slower (for a
% primitive macro with a single key benchmarking was almost $1.7$ times slower,
% the root of which being the key access with |\ekvcValue|, not the parsing, and
% for a key access using |\ekvcValueFast| it was still about $1.2$ times
% slower). So if your macro uses less than ten primary keys, you should consider
% using the split approach.
%
% \begin{function}{\ekvcHash}
%   \begin{syntax}
%     \cs{ekvcHash}\meta{cs}\marg{primary keys}\marg{definition}
%   \end{syntax}
%   This defines \meta{cs} to be a macro taking one mandatory argument which
%   should contain a \kv\ list. You can use as many primary keys as you want.
%   The primary keys will be forwarded as a single argument containing every key
%   to the underlying macro. The underlying macro is defined as
%   \meta{definition}, in which you can access the \val\ of a \key\ by using
%   \texttt{\cs[no-index]{ekvcValue}\{\key\}\{\#1\}}.
% \end{function}
% \example This defines an equivalent macro like the |\foo| defined with
% |\ekvcSplit| earlier:\\
% \begin{minipage}{.75\linewidth}
% \begin{lstlisting}
% \ekvcHash\foo{a=a,b=b}{a is \ekvcValue{a}{#1}.\par
%                        b is \ekvcValue{b}{#1}.\par}
% \foo{}
% \foo{b=e}
% \end{lstlisting}
% \end{minipage}^^A
% \begin{exresult}[nobeforeafter,box align=center]{.25\linewidth}
%   \ekvcHash\foo{a=a,b=b}{a is \ekvcValue{a}{#1}.\par
%                          b is \ekvcValue{b}{#1}.\par}
%   \foo{}
%   \foo{b=e}
% \end{exresult}
%
% \begin{function}{\ekvcHashAndForward}
%   \begin{syntax}
%     \cs{ekvcHashAndForward}\meta{cs}\marg{after}\marg{primary keys}
%   \end{syntax}
%   This defines \meta{cs} to be a macro taking one mandatory argument which
%   should contain a \kv\ list. You can use as many primary keys as you want.
%   The primary keys will be forwarded as a single argument containing every key
%   to \meta{after}. You can use a single macro for \meta{after} or use some
%   arbitrary stuff, which will be left in the input stream before the hashed
%   \kv\ list with one set of braces stripped. In the
%   macro called in \meta{after} you can access the \val\ of a \key\ by using
%   \texttt{\cs[no-index]{ekvcValue}\{\key\}\{\#1\}} (or whichever argument the
%   hashed \kv\ list will be).
% \end{function}
%
% \begin{function}{\ekvcHashAndUse}
%   \begin{syntax}
%     \cs{ekvcHashAndUse}\meta{cs}\marg{primary keys}
%   \end{syntax}
%   This will roughly do the same as |\ekvcHashAndForward|, but instead of
%   specifying what will be used after hashing the keys, \meta{cs} will use what
%   follows the \kv\ list. So its syntax will be
% \end{function}
%   \begin{syntax}
%     \meta{cs}\{\key=\val, \ldots\}\marg{after}
%   \end{syntax}
%
% \begin{function}{\ekvcValue}
%   \begin{syntax}
%     \cs{ekvcValue}\{\key\}\marg{key list}
%   \end{syntax}
%   This is a safe way to access your keys in a hash variant. \key\
%   is the key which's \val\ you want to use out of the \meta{key list}.
%   \meta{key list} should be the key list argument forwarded to your underlying
%   macro by |\ekvcHash|, |\ekvcHashAndForward|, or |\ekvcHashAndUse|. It will
%   be tested whether the hash function to access that \key\ exists, the \key\
%   argument is not empty, and that the \meta{key list} really contains a \val\
%   of that \key. This macro needs exactly two steps of expansion and if used
%   inside of an |\edef| or |\expanded| context will protect the \val\ from
%   further expanding.
% \end{function}
%
% \begin{function}{\ekvcValueFast}
%   \begin{syntax}
%     \cs{ekvcValueFast}\{\key\}\marg{key list}
%   \end{syntax}
%   This behaves just like |\ekvcValue|, but \emph{without any} safety tests. As
%   a result this is about $1.4$ times faster \emph{but} will throw low level
%   \TeX\ errors eventually if the hash function isn't defined or the \key\
%   isn't part of the \meta{key list} (\emph{e.g.}, because it was defined as a
%   key for another macro -- all macros share the same hash function per \key).
%   Use it if you know what you're doing. This macro needs exactly three steps
%   of expansion in the no-errors case.
% \end{function}
%
% \begin{function}{\ekvcValueSplit}
%   \begin{syntax}
%     \cs{ekvcValueSplit}\{\key\}\marg{key list}\marg{next}
%   \end{syntax}
%   If you need a specific \key\ from a \meta{key list} more than once, it'll
%   be a good idea to only extract it once and from then on keep it as a
%   separate argument. Hence the macro \cs{ekvcValueSplit} will extract one
%   specific \key's value from the list and forward the remainder of the list
%   as the first and the \key's value as the second argument to \meta{next}, so
%   the result of this will be \meta{next}\marg{key list'}\marg{value} with
%   \meta{key list'} the remaining list. This is almost as fast as |\ekvcValue|
%   and runs the same tests. Keep in mind that you can't fetch for the same
%   \key\ again from \meta{key list'} as it got removed.
% \end{function}
% \example The following defines a macro |\foo| which will take three keys.
% Since the next parsing step will need the value of one of the keys multiple
% times we split that key off the list:\\
% \begin{minipage}{.75\linewidth}
% \begin{lstlisting}
% \ekvcHash\foo{a=a,b=b,c=c}
%   {\ekvcValueSplit{a}{#1}\foobar}
% \newcommand*\foobar[2]{a is #2.\par
%                        b is \ekvcValue{b}{#1}.\par
%                        c is \ekvcValue{c}{#1}.\par}
% \foo{}
% \end{lstlisting}
% \end{minipage}^^A
% \begin{exresult}[nobeforeafter,box align=center]{.25\linewidth}
%   \ekvcHash\foo{a=a,b=b,c=c}
%     {\ekvcValueSplit{a}{#1}\foobar}
%   \newcommand*\foobar[2]{a is #2.\par
%                          b is \ekvcValue{b}{#1}.\par
%                          c is \ekvcValue{c}{#1}.\par}
%   \foo{}
% \end{exresult}
%
% \begin{function}{\ekvcValueSplitFast}
%   \begin{syntax}
%     \cs{ekvcValueSplitFast}\{\key\}\marg{key list}\marg{next}
%   \end{syntax}
%   This behaves just like |\ekvcValueSplit|, but it won't run the same tests,
%   hence it is faster but more error prone, just like the relation between
%   |\ekvcValue| and |\ekvcValueFast|.
% \end{function}
%
%
% \subsection{Secondary Keys}\label{sec:secondaries}
%
% To remove some of the limitations with the approach that each primary key
% matches an argument or hash entry, you can define secondary keys. Those have
% to be defined for each macro but it doesn't matter whether that macro was a
% split or a hash variant. If a secondary key references another key it doesn't
% matter whether that other key is primary or secondary.
%
% Secondary keys can have a prefix (like |long|) which are called |p|-type
% prefix and must have a type (like |meta|) which are called |t|-type prefix.
% Some types might require some |p|-prefixes, while others might forbid those.
%
% Please keep in mind that key names shouldn't start with \texttt{EKVC\string|}.
%
% \begin{function}{\ekvcSecondaryKeys}
%   \begin{syntax}
%     \cs{ekvcSecondaryKeys}\meta{cs}\{\kv, \ldots\}
%   \end{syntax}
%   This is the front facing macro to define secondary keys. For the macro
%   \meta{cs} define \key\ to have definition \val. The general syntax for \key\
%   should be
%   \begin{quote}\ttfamily\small
%     \meta{prefix} \meta{name}
%   \end{quote}
%   Where \meta{prefix} is a space separated list of optional |p|-type prefixes
%   followed by one |t|-type prefix. The syntax of \val\ is dependent on the
%   used |t|-prefix.
% \end{function}
%
%
% \subsubsection{\texttt{p}-type Prefixes}
%
% There is only one |p|-prefix available, which is |long|.
%
% \begin{function}{long}
%   The following key will be defined |\long|.
% \end{function}
%
%
% \subsubsection{\texttt{t}-type Prefixes}
%
% If you're familiar with \expkvd\ you'll notice that the |t|-type prefixes
% provided here are much fewer. The expansion only concept doesn't allow for
% great variety in the auto-defined keys.
%
% The syntax examples of the |t|-prefixes will show which |p|-prefix will be
% automatically used by printing those black (\texttt{\enflong}), which
% will be available in grey (\texttt{\alllong}), and which will be disallowed in
% red (\texttt{\notlong}). This will be put flush right next to the syntax line.
%
% \begin{function}{meta}
%   \begin{syntax}
%     meta \key\ = \{\kv, \ldots\} \prefixes2
%   \end{syntax}
%   With a |meta| key you can set other keys. Whenever \key\ is used the keys in
%   the \kv\ list will be set to the values given there. You can use the \val\
%   given to \key\ by using |#1| in the \kv\ list. The keys in the \kv\ list can
%   be primary and secondary ones.
% \end{function}
%
% \begin{function}{nmeta}
%   \begin{syntax}
%     nmeta \key\ = \{\kv, \ldots\} \prefixes3
%   \end{syntax}
%   An |nmeta| key is like a |meta| key, but it doesn't take a value, so the
%   \kv\ list is static.
% \end{function}
%
% \begin{function}{alias}
%   \begin{syntax}
%     alias \key\ = \meta{key_2} \prefixes3
%   \end{syntax}
%   This assigns the definition of \meta{key_2} to \key. As a result \key\ is an
%   alias for \meta{key_2} behaving just the same. Both the value taking and the
%   |NoVal| version (that's \expkv\ slang for a key not accepting a value) will
%   be copied if they are defined when |alias| is used. Of course, \meta{key_2}
%   has to be defined, be it as a primary or secondary one.
% \end{function}
%
% \begin{function}{default}
%   \begin{syntax}
%     default \key\ = \marg{default} \prefixes3
%   \end{syntax}
%   If \key\ is a defined value taking key, you can define a |NoVal| version
%   with this that will behave as if \key\ was given \meta{default} as its
%   \val. Note that this doesn't change the initial values of primary keys set
%   at definition time in |\ekvcSplit| and friends. \key\ can be a primary or
%   secondary key.
% \end{function}
%
% \begin{function}{flag-bool}
%   \begin{syntax}
%     flag-bool \key\ = \meta{cs} \prefixes3
%   \end{syntax}
%   This is a secondary key that doesn't directly involve any of the primary or
%   secondary keys.
%   This defines \key\ to take a value, which should be either |true| or
%   |false|, and set the flag called \meta{cs} accordingly as a boolean.
%   If \meta{cs} isn't defined yet it will be initialised as a flag.
%   Please also read \autoref{sec:flags}.
% \end{function}
%
% \begin{function}{flag-true,flag-false}
%   \begin{syntax}
%     flag-true \key\ = \meta{cs} \prefixes3
%   \end{syntax}
%   This is a secondary key that doesn't directly involve any of the primary or
%   secondary keys.
%   This defines \key\ to take no value and set the flag called \meta{cs} to
%   |true| or |false|, respectively.
%   If \meta{cs} isn't defined yet it will be initialised as a flag.
%   Please also read \autoref{sec:flags}.
% \end{function}
%
% \begin{function}{flag-raise}
%   \begin{syntax}
%     flag-raise \key\ = \meta{cs} \prefixes3
%   \end{syntax}
%   This is a secondary key that doesn't directly involve any of the primary or
%   secondary keys.
%   This defines \key\ to take no value and raise the flag called \meta{cs}.
%   If \meta{cs} isn't defined yet it will be initialised as a flag.
%   Please also read \autoref{sec:flags}.
% \end{function}
%
%
% \subsection{Changing the Initial Values}\label{sec:change}
%
% \begin{function}{\ekvcChange}
%   \begin{syntax}
%     \cs{ekvcChange}\meta{cs}\{\kv,\ldots\}
%   \end{syntax}
%   This processes the \kv\ list (just like it would be processed by \meta{cs})
%   to set new defaults for the macro \meta{cs} defined with \expkvc\ (meaning
%   the values used if you don't provide anything at use time). Inside the \kv\
%   list both primary and secondary keys can be used. If \meta{cs} was defined
%   |\long| earlier it will still be |\long|, every other \TeX\ prefix will be
%   stripped (but \expkvc\ doesn't support them anywhere else so that should be
%   fine). The resulting new defaults will be stored inside the \meta{cs}
%   locally (just as the original defaults were).
% \end{function}
%
% Consider the following example:\\
% \begin{minipage}[c]{.75\linewidth}
% \begin{lstlisting}
% \ekvcSplit\foo{a=a,b=b}{a is #1.\par b is #2.\par}
% \foo{} % initial defaults
% \ekvcChange\foo{b=B}
% \foo{} % new defaults
% \ekvcSecondaryKeys\foo{meta c={a={#1},b={#1}}}
% \ekvcChange\foo{c=c}
% \foo{} % newer defaults
% \end{lstlisting}
% \end{minipage}^^A
% \begin{exresult}[nobeforeafter,box align=center]{.25\linewidth}
%   \ekvcSplit\foo{a=a,b=b}{a is #1.\par b is #2.\par}
%   \foo{}
%   \ekvcChange\foo{b=B}
%   \foo{}
%   \ekvcSecondaryKeys\foo{meta c={a={#1},b={#1}}}
%   \ekvcChange\foo{c=c}
%   \foo{}
% \end{exresult}
%
% As a result with this the typical setup macro could be implemented:
% \begin{lstlisting}
% \ekvcHashAndUse\foo{key=a,key=b}
% \newcommand\foosetup{\ekvcChange\foo}
% \end{lstlisting}
% Of course the usage is limited to a single macro |\foo|, hence this might be
% not as powerful as similar macros used with other \kv\ interfaces. But at
% least a few similar macros could be grouped using the same key parsing macro
% internally.
%
%
% \subsection{Flags}\label{sec:flags}
%
% The idea of flags is taken from \pkg{expl3}. They provide a way to store
% numerical information expandably, however only incrementing and accessing
% works expandably, decrementing is unexpandable. A flag has a height, which
% is a numerical value, and which can be raised by $1$. Flags come at a high
% computational cost (accessing them is slow and they require more memory than
% normal \TeX\ data types like registers, both getting linearly worse with the
% height), so don't use them if not necessary.
%
% The state of flags is always changed locally to the current group, but not to
% the current macro, so if you're using one of the |t|-types involving flags
% bear in mind that they can affect other macros using the same flags at the
% current group level!
%
% \expkvc\ provides some macros to access, alter, and use flags. Flags of
% \expkvc\ don't share a name space with the flags of \pkg{expl3}.
%
% \begin{function}{\ekvcFlagNew}
%   \begin{syntax}
%     \cs{ekvcFlagNew}\meta{flag}
%   \end{syntax}
%   This initialises the macro \meta{flag} as a new flag. It isn't checked
%   whether the macro \meta{flag} is currently undefined.  A \meta{flag} will
%   expand to the flag's current height with a trailing space (so you can use it
%   directly with |\ifnum| for example and it will terminate the number scanning
%   itself).
% \end{function}
%
%  All other macros dealing with flags take as a parameter a macro defined as a
%  \meta{flag} with |\ekvcFlagNew|.
%
% \begin{function}{\ekvcFlagHeight}
%   \begin{syntax}
%     \cs{ekvcFlagHeight}\meta{flag}
%   \end{syntax}
%   This expands to the current height of \meta{flag} in a single step of
%   expansion (without a trailing space).
% \end{function}
%
% \begin{function}{\ekvcFlagRaise}
%   \begin{syntax}
%     \cs{ekvcFlagRaise}\meta{flag}
%   \end{syntax}
%   This expandably raises the height of \meta{flag} by $1$.
% \end{function}
%
% \begin{function}{\ekvcFlagSetTrue,\ekvcFlagSetFalse}
%   \begin{syntax}
%     \cs{ekvcFlagSetTrue}\meta{flag}
%   \end{syntax}
%   By interpreting an even value as false and an odd value as true we can use a
%   flag as a boolean. This expandably sets \meta{flag} to true or false,
%   respectively, by raising it if necessary.
% \end{function}
%
% \begin{function}{\ekvcFlagIf}
%   \begin{syntax}
%     \cs{ekvcFlagIf}\meta{flag}\marg{true}\marg{false}
%   \end{syntax}
%   This interprets a \meta{flag} as a boolean and expands to either \meta{true}
%   or \meta{false}.
% \end{function}
%
% \begin{function}{\ekvcFlagIfRaised}
%   \begin{syntax}
%     \cs{ekvcFlagIfRaised}\meta{flag}\marg{true}\marg{false}
%   \end{syntax}
%   This tests whether the \meta{flag} is raised, meaning it has a height
%   greater than zero, and if so expands to \meta{true} else to \meta{false}.
% \end{function}
%
% \begin{function}{\ekvcFlagReset}
%   \begin{syntax}
%     \cs{ekvcFlagReset}\meta{flag}
%   \end{syntax}
%   This resets a flag (so restores its height to $0$). This operation is
%   \emph{not} expandable and done locally. If you really intend to use flags
%   you can reset them every now and then to keep the performance hit low.
% \end{function}
%
% \begin{function}{\ekvcFlagGetHeight}
%   \begin{syntax}
%     \cs{ekvcFlagGetHeight}\meta{flag}\marg{next}
%   \end{syntax}
%   This retrieves the current height of the \meta{flag} and provides it as a
%   braced argument to \meta{next}, leaving \meta{next}\marg{height} in the
%   input stream.
% \end{function}
%
% \begin{function}{\ekvcFlagGetHeights}
%   \begin{syntax}
%     \cs{ekvcFlagGetHeights}\marg{flag-list}\marg{next}
%   \end{syntax}
%   This retrieves the current height of each \meta{flag} in the
%   \meta{flag-list} and provides them as a single argument to \meta{next}.
%   Inside that argument each height is enclosed in a set of braces
%   individually. The \meta{flag-list} is just a single argument containing the
%   \meta{flag}s.
%   So a usage like
%   \texttt
%     {^^A
%       \cs[no-index]{ekvcFlagGetHeights}\hskip0pt^^A
%       \{\cs[no-index]{myflagA}\cs[no-index]{myflagB}\}\hskip0pt^^A
%       \{\cs[no-index]{stuff}\}^^A
%     }
%   will expand to
%   \texttt{\cs[no-index]{stuff}\{\marg{height-A}\marg{height-B}\}}.
% \end{function}
%
%
% \subsection{Further Examples}
%
% How could a documentation be a good documentation without some basic examples?
% Say we want to define a small macro expanding to some character description
% (who knows why this has to be expandable?). A character description will not
% have too many items to it, so we use |\ekvcSplit| (the comments with the
% parameter numbers are of course not necessary and just ease reading the
% example).
% \begin{lstlisting}
% \ekvcSplit\character
%   {
%     name=John Doe,            % #1
%     age=any,                  % #2
%     nationality=the Universe, % #3
%     hobby=to exist,           % #4
%     type=Mister,              % #5
%     pronoun=He,               % #6
%     possessive=his,           % #7
%   }
%   {#1 is a #5 from #3. #6 is of #2 age and #7 hobby is #4.\par}
% \end{lstlisting}
% Also we want to give some short cuts so that it's easier to describe several
% persons.
% \begin{lstlisting}
% \ekvcSecondaryKeys\character
%   {
%     alias pro = pronoun,
%     alias pos = possessive,
%     nmeta me =
%       {
%         name=Jonathan,
%         age=a young,
%         nationality=Germany,
%         hobby=\TeX\ coding,
%       },
%     meta lady =
%       {type=Lady, pronoun=She, possessive=her, name=Jane Doe, #1},
%     nmeta paulo =
%       {
%         name=Paulo,
%         type=duck,
%         age=a young,
%         nationality=Brazil,
%         hobby=to quack,
%       }
%   }
% \end{lstlisting}
% Now we can describe people using
% \begin{lstlisting}
% \character{}
% \character{me}
% \character{paulo}
% \character
%   {lady={name=Evelyn,nationality=Ireland,age=the best,hobby=reading}}
% \character
%   {
%     name=Our sun, type=star, nationality=our solar system, pro=It,
%     age=an old, pos=its, hobby=shining
%   }
% \end{lstlisting}
% As one might see, the |lady| key could actually have been an |nmeta| key as
% well, as all that is done with the argument is using it as a \kv\ list.
%
% The result of only the first two usages would be:
% \begin{exresult}{\linewidth}
%   \ekvcSplit\character
%     {
%       name=John Doe,
%       age=any,
%       nationality=the Universe,
%       hobby=to exist,
%       type=Mister,
%       pronoun=He,
%       possessive=his,
%     }
%     {#1 is a #5 from #3. #6 is of #2 age and #7 hobby is #4.\par}
%   \ekvcSecondaryKeys\character
%     {
%       alias pro = pronoun,
%       alias pos = possessive,
%       nmeta me =
%         {
%           name=Jonathan,
%           age=a young,
%           nationality=Germany,
%           hobby=\TeX\ coding,
%         },
%       meta lady =
%         {type=Lady, pronoun=She, possessive=her, name=Jane Doe, #1},
%       nmeta paulo =
%         {
%           name=Paulo,
%           type=duck,
%           age=a young,
%           nationality=Brazil,
%           hobby=to quack,
%         }
%     }
%   \character{}
%   \character{me}
% \end{exresult}
%
% Using \pkg{xparse} or \expkv's \cs{ekvoptarg} or \cs{ekvoptargTF} and
% forwarding arguments one can easily define \kv\ macros with actual optional
% and mandatory arguments as well. A small nonsense example (which should
% perhaps use |\ekvcSplitAndForward| instead of |\ekvcHashAndForward| since it
% only uses four keys and one other argument -- and isn't expandable since it
% uses a \env{tabular} environment, so it would've been better to use a more
% feature rich \kv\ interface most likely, \emph{e.g.}, the one provided by
% \expkvd):
% \begin{lstlisting}
% \makeatletter
% \newcommand*\nonsense{\ekvoptarg\nonsense@a{}}
% \ekvcHashAndForward\nonsense@a\nonsense@b
%   {
%     keyA = A,
%     keyB = B,
%     keyC = c,
%     keyD = d,
%   }
% \newcommand*\nonsense@b[2]
%   {%
%     \begin{tabular}{lll}
%       key & A & \ekvcValue{keyA}{#1} \\
%           & B & \ekvcValue{keyB}{#1} \\
%           & C & \ekvcValue{keyC}{#1} \\
%           & D & \ekvcValue{keyD}{#1} \\
%       \multicolumn{2}{l}{mandatory} & #2 \\
%     \end{tabular}%
%   }
% \makeatother
% \end{lstlisting}
% And then we would be able to do some nonsense
% \begin{lstlisting}
% \nonsense{}
% \nonsense[keyA=hihi]{haha}
% \nonsense[keyA=hihi, keyB=A]{hehe}
% \nonsense[keyC=huhu, keyA=hihi, keyB=A]{haha}
% \end{lstlisting}
% resulting in
% \begin{exresult}{\linewidth}
%   \makeatletter
%   \newcommand*\nonsense{\ekvoptarg\nonsense@a{}}
%   \ekvcHashAndForward\nonsense@a\nonsense@b
%     {
%       keyA = A,
%       keyB = B,
%       keyC = c,
%       keyD = d,
%     }
%   \newcommand*\nonsense@b[2]
%     {%
%       \begin{tabular}{lll}
%         key & A & \ekvcValue{keyA}{#1} \\
%             & B & \ekvcValue{keyB}{#1} \\
%             & C & \ekvcValue{keyC}{#1} \\
%             & D & \ekvcValue{keyD}{#1} \\
%         \multicolumn{2}{l}{mandatory} & #2 \\
%       \end{tabular}%
%     }
%   \makeatother
%   \nonsense{}
%   \nonsense[keyA=hihi]{haha}
%   \nonsense[keyA=hihi, keyB=A]{hehe}
%   \nonsense[keyC=huhu, keyA=hihi, keyB=A]{haha}
% \end{exresult}
%
%
% \subsection{Freedom for Keys!}
%
% If this was the \TeX book this subsection would have a double bend sign. Not
% because it is overly complicated, but because it shows things which could
% break \expkvc's expandability and its alignment safety. This is for
% experienced users wanting to get the most flexibility and knowing what they
% are doing.
%
% In case you're wondering, it is possible to define other keys than the
% primaries and the secondary types listed in \autoref{sec:secondaries} for a
% macro defined with \expkvc\ by using the low-level interface of \expkv\ or
% even the interface provided by \expkvd. The set name used for \expkvc's keys
% is the macro name, including the leading backslash, or more precisely
% |\string|\meta{cs} is used. This can be exploited to define additional keys
% with arbitrary code. Consider the following \emph{bad} example:
% \begin{lstlisting}
% \ekvcSplit\foo{a=A,b=B}{a is #1.\par b is #2}
% \protected\ekvdef{\string\foo}{c}{\def\fooC{#1}}
% \end{lstlisting}
% This would define a key named |c| that will store its value inside a macro.
% The issue with this is that this can't be done expandably. As a result, the
% macro |\foo| isn't always expandable any more (not that bad if this was never
% required; killjoy if it was) and as soon as the key |c| is used, it is also no
% longer alignment safe\footnote{This means that the \kv-list can't contain
%   alignment markers that are not inside an additional set of braces if used
%   inside a \TeX\ alignment.}
% (might be bad depending on the usage).
%
% So why do I show you this? Because we could as well do something useful like
% create a key that pre-parses the input and after that passes the parsed value
% on. This parsing would have to be completely expandable though. For the
% pass-on part we can use the following function:
%
% \begin{function}{\ekvcPass}
%   \begin{syntax}
%     \cs{ekvcPass}\meta{cs}\marg{key}\marg{value}
%   \end{syntax}
%   This passes \meta{value} on to \meta{key} for the \expkvc-macro \meta{cs}.
%   It should be used inside the key parsing of a macro defined with \expkvc,
%   else this most likely results in a low level \TeX\ error.
% \end{function}
%
% With this we could for example split the value of a key at a hyphen and pass
% the parts to different keys:\\
% \begin{minipage}[c]{.75\linewidth}
% \begin{lstlisting}
% \ekvcSplit\foo{a=A,b=B}{a is #1.\par b is #2.\par}
% \ekvdef{\string\foo}{c}{\fooSplit#1\par}
% \def\fooSplit#1-#2\par
%   {\ekvcPass\foo{a}{#1}\ekvcPass\foo{b}{#2}}
% \foo{}
% \foo{c=1-2}
% \end{lstlisting}
% \end{minipage}
% \begin{exresult}[nobeforeafter,box align=center]{.25\linewidth}
%   \ekvcSplit\foo{a=A,b=B}{a is #1.\par b is #2.\par}
%   \ekvdef{\string\foo}{c}{\fooSplit#1\par}
%   \def\fooSplit#1-#2\par{\ekvcPass\foo{a}{#1}\ekvcPass\foo{b}{#2}}
%   \foo{}
%   \foo{c=1-2}
% \end{exresult}
%
%
% \subsection{Speed Considerations}\label{sec:speed}
%
% As already mentioned in the introduction there are some speed considerations
% implied if you choose to define macros via \expkvc. However the overhead isn't
% the factor which should hinder you to use \expkvc\ if you found a reasonable
% use case. The key-parsing is still faster than with most other \kv\ packages
% (see the ``Comparisons'' subsection in the \expkv\ documentation).
%
% The speed considerations in this subsection use the first example in this
% documentation as the benchmark. So we have seven keys and a short sentence
% which should be typeset. For comparisons I use the following equivalent
% \expkvd\ definitions. Each result is the average between changing no keys from
% their initial values and altering four. Furthermore I'll compare three
% variants of \expkvc\ with the \expkvd\ definitions, namely the split example
% from above, a hash variant using |\ekvcValue| and a hash variant using
% |\ekvcValueFast|.
% \begin{lstlisting}
% \usepackage{expkv-def}
% \ekvdefinekeys{keys}
%   {%
%     ,store   name         = \KEYSname
%     ,initial name         = John Doe
%     ,store   age          = \KEYSage
%     ,initial age          = any
%     ,store   nationality  = \KEYSnationality
%     ,initial nationality  = the Universe
%     ,store   hobby        = \KEYShobby
%     ,initial hobby        = to exist
%     ,store   type         = \KEYStype
%     ,initial type         = Mister
%     ,store   pronoun      = \KEYSpronoun
%     ,initial pronoun      = He
%     ,store   possessive   = \KEYSpossessive
%     ,initial possessive   = his
%   }
% \newcommand*\KEYS[1]
%   {%
%     \begingroup
%       \ekvset{keys}{#1}%
%       \KEYSname\ is a \KEYStype\ from \KEYSnationality. \KEYSpronoun\ is
%       of \KEYSage\ age and \KEYSpossessive\ hobby is \KEYShobby.%
%     \endgroup
%   }
% \end{lstlisting}
%
% The first comparison removes the typesetting part from all the definitions, so
% that only the key parsing is compared. In this comparison the |\ekvcValue|
% and |\ekvcValueFast| variants will not differ, as they are exactly the same
% until the key usage. We find that the split approach is $1.4$ times slower
% than the \expkvd\ setup and the hash variants end up in the middle at $1.17$
% times slower.
%
% Next we put the typesetting part back in. Every call of the macros will
% typeset the sentences into a box register in horizontal mode. With the
% typesetting part (which includes the accessing of values) the fastest remains
% the \expkvd\ definitions, but split is close at $1.16$ times slower, followed
% by the hash variant with fast accesses at $1.36$ times slower, and the safe
% hash access variant ranks in the slowest $1.8$ times slower than \expkvd.
%
% Just in case you're wondering now, a simple macro taking seven arguments is
% $30$ to $40$ times faster than any of those in the argument grabbing and \kv\
% parsing part and only $1.5$ to $2.8$ times faster if the typesetting part is
% factored in. So the real choke isn't the parsing.
%
% So to summarize this, if you have a reasonable use case for expandable \kv\
% parsing macros you should go on and define them using \expkvc. If you have a
% reasonable use case for \kv\ parsing macros but defining them expandable isn't
% necessary for your use you should take advantage of the greater flexibility of
% non-expandable \kv\ setups (but if you're after maximum speed there aren't
% that many \kv\ parsers beating \expkvc). And if you are after maximum
% performance maybe ditching the \kv\ interface altogether is a good idea, but
% depending on the number of arguments your interface might get convoluted.
%
%
% \subsection{Useless Macros}
%
% Perhaps these macros aren't completely useless, but I figured from a user's
% point of view I wouldn't know what I should do with these.
%
% \begin{function}{\ekvcDate,\ekvcVersion}
%   These two macros store the version and the date of the package/generic code.
% \end{function}
%
%
% \subsection{Bugs}
%
% Of course I don't think there are any bugs (who would knowingly distribute
% buggy software as long as he isn't a multi-million dollar corporation?). But
% if you find some please let me know. For this one might find my email address
% on the first page or file an issue on Github:
% \url{https://github.com/Skillmon/tex_expkv-cs}
%
%
% \subsection{License}
%
% Copyright \textcopyright\ 2020\unless\ifnum\year=2020--\the\year\fi\
% Jonathan P. Spratte
%
% \medskip\noindent
% This work may be distributed and/or modified under the conditions of the
% \LaTeX\ Project Public License (LPPL), either version 1.3c of this license or
% (at your option) any later version. The latest version of this license is in
% the file:
%
% \url{http://www.latex-project.org/lppl.txt}
%
% \noindent
% This work is ``maintained'' (as per LPPL maintenance status) by
%
% \mbox{Jonathan P. Spratte}.
%
% \end{documentation}^^A=<<
%
% \begin{implementation}^^A>>=
%
% \clearpage
%
% \section{Implementation}^^A>>=
%^^A the LaTeX package >>=
% \subsection{The \LaTeX\ Package}
% Just like for \expkv\ we provide a small \LaTeX\ package that sets up things
% such that we behave nicely on \LaTeX\ packages and files system. It'll
% |\input| the generic code which implements the functionality.
% \gobbledocstriptag
%<*pkg>
%    \begin{macrocode}
\RequirePackage{expkv}
\def\ekvc@tmp
  {%
    \ProvidesFile{expkv-cs.tex}%
      [%
        \ekvcDate\space v\ekvcVersion\space
        define expandable key=val macros using expkv%
      ]%
  }
\input{expkv-cs.tex}
\ProvidesPackage{expkv-cs}%
  [%
    \ekvcDate\space v\ekvcVersion\space
    define expandable key=val macros using expkv%
  ]
%    \end{macrocode}
% \gobbledocstriptag
%</pkg>
%^^A=<<
%^^A main file >>=
% \subsection{The Generic Code}
% The rest of this implementation will be the generic code.
% \gobbledocstriptag
%<*tex>
%
% Load \expkv\ if the package didn't already do so -- since \expkv\ has
% safeguards against being loaded twice this does no harm and the overhead
% isn't that big. Also we reuse some of the internals of \expkv\ to save us from
% retyping them.
%    \begin{macrocode}
\input expkv
%    \end{macrocode}
%
% We make sure that \file{expkv-cs.tex} is only input once:
%    \begin{macrocode}
\expandafter\ifx\csname ekvcVersion\endcsname\relax
\else
  \expandafter\endinput
\fi
%    \end{macrocode}
%
% \begin{macro}{\ekvcVersion,\ekvcDate}
% We're on our first input, so lets store the version and date in a macro.
%    \begin{macrocode}
\def\ekvcVersion{0.10}
\def\ekvcDate{2021-06-03}
%    \end{macrocode}
% \end{macro}
%
% If the \LaTeX\ format is loaded we want to be a good file and report back who
% we are, for this the package will have defined |\ekvc@tmp| to use
% |\ProvidesFile|, else this will expand to a |\relax| and do no harm.
%    \begin{macrocode}
\csname ekvc@tmp\endcsname
%    \end{macrocode}
%
% Store the category code of |@| to later be able to reset it and change it to
% 11 for now.
%    \begin{macrocode}
\expandafter\chardef\csname ekvc@tmp\endcsname=\catcode`\@
\catcode`\@=11
%    \end{macrocode}
% |\ekvc@tmp| will be reused later, but we don't need it to ever store
% information long-term after \expkvc\ was initialized.
%
% \begin{macro}[internal]{\ekvc@keycount}
%   We'll need to keep count how many keys must be defined for each macro in the
%   |split| variants.
%    \begin{macrocode}
\newcount\ekvc@keycount
%    \end{macrocode}
% \end{macro}
%
% \begin{macro}[internal]{\ekvc@long,\ekvc@any@long}
%   Some macros will have to be defined long. These two will be let to |\long|
%   when this should be the case.
%    \begin{macrocode}
\let\ekvc@long\ekv@empty
\let\ekvc@any@long\ekv@empty
%    \end{macrocode}
% \end{macro}
%
% \begin{macro}[internal]{\ekvc@ifdefined}
%   We want to test whether a macro is already defined. This test checks for a
%   defined macro that isn't |\relax|.
%    \begin{macrocode}
\long\def\ekvc@ifdefined#1%
  {%
    \ifdefined#1%
      \ifx\relax#1%
        \ekv@fi@gobble
      \fi
      \@firstofone
      \ekv@fi@firstoftwo
    \fi
    \@secondoftwo
  }
%    \end{macrocode}
% \end{macro}
%
% \begin{macro}[internal]
%   {
%     \ekvc@ekvset@pre@expander,
%     \ekvc@ekvset@pre@expander@a,
%     \ekvc@ekvset@pre@expander@b
%   }
%   This macro expands |\ekvset| twice so that the first two steps of expansion
%   don't have to be made every time the \expkvc\ macros are used. We have to do
%   a little magic trick to get the macro parameter |#1| for the macro
%   definition this is used in, even though we're calling |\unexpanded|. We do
%   that by splitting the expanded |\ekvset| at some marks and place |##1| in
%   between. At this spot we also add |\ekv@alignsafe| and |\ekv@endalignsafe|
%   to ensure that macros created with \expkvc\ are alignment safe.
%    \begin{macrocode}
\def\ekvc@ekvset@pre@expander#1%
  {%
    \expandafter\ekvc@ekvset@pre@expander@a\ekvset{#1}\ekvc@stop\ekvc@stop
  }
\def\ekvc@ekvset@pre@expander@a
  {%
    \expandafter\ekvc@ekvset@pre@expander@b
  }
\def\ekvc@ekvset@pre@expander@b#1\ekvc@stop#2\ekvc@stop
  {%
    \unexpanded\expandafter{\ekv@alignsafe}%
    \unexpanded{#1}##1\unexpanded{#2}%
    \unexpanded\expandafter{\ekv@endalignsafe}%
  }
%    \end{macrocode}
% \end{macro}
%
% \begin{macro}{\ekvcSplitAndUse}
%   The first user macro we want to set up can be reused for
%   \cs[no-index]{ekvcSplitAndForward} and \cs[no-index]{ekvcSplit}. We'll split
%   this one up so that the test whether the macro is already defined doesn't
%   run twice.
%    \begin{macrocode}
\protected\long\def\ekvcSplitAndUse#1#2%
  {%
    \let\ekvc@helpers@needed\@firstoftwo
    \ekvc@ifdefined#1%
      {\ekvc@err@already@defined#1}%
      {\ekvcSplitAndUse@#1{}{#2}}%
  }
%    \end{macrocode}
% \end{macro}
%
% \begin{macro}[internal]{\ekvcSplitAndUse@}
%   The actual macro setting up things. We need to set some variables, forward
%   the key list to |\ekvc@SetupSplitKeys|, and afterwards define the front
%   facing macro to call |\ekvset| and put the initials and the argument sorting
%   macro behind it. The internals |\ekvc@any@long|, |\ekvc@initials| and
%   |\ekvc@keycount| will be set correctly by |\ekvc@SetupSplitKeys|.
%    \begin{macrocode}
\protected\long\def\ekvcSplitAndUse@#1#2#3%
  {%
    \edef\ekvc@set{\string#1}%
    \ekvc@SetupSplitKeys{#3}%
    \ekvc@helpers@needed
      {%
        \ekvc@any@long\edef#1##1%
          {%
            \expandafter\ekvc@ekvset@pre@expander\expandafter{\ekvc@set}%
            \unexpanded\expandafter
              {\csname ekvc@split@\the\ekvc@keycount\endcsname}%
            \unexpanded\expandafter{\ekvc@initials{}#2}%
          }%
      }%
      {%
        \ekvc@any@long\edef#1##1%
          {%
            \expandafter\ekvc@ekvset@pre@expander\expandafter{\ekvc@set}%
            \unexpanded{#2}%
            \unexpanded\expandafter{\ekvc@initials}%
          }%
      }%
  }
%    \end{macrocode}
% \end{macro}
%
%
% \begin{macro}{\ekvcSplitAndForward}
%   This just reuses |\ekvcSplitAndUse@| with a non-empty second argument,
%   resulting in that argument to be called after the splitting.
%    \begin{macrocode}
\protected\long\def\ekvcSplitAndForward#1#2#3%
  {%
    \let\ekvc@helpers@needed\@firstoftwo
    \ekvc@ifdefined#1%
      {\ekvc@err@already@defined#1}%
      {\ekvcSplitAndUse@#1{{#2}}{#3}}%
  }
%    \end{macrocode}
% \end{macro}
%
% \begin{macro}{\ekvcSplit}
%   The first half is just |\ekvcSplitAndForward| then we define the macro to
%   which the parsed key list is forwarded. There we need to allow for up to
%   nine arguments.
%    \begin{macrocode}
\protected\long\def\ekvcSplit#1#2#3%
  {%
    \let\ekvc@helpers@needed\@secondoftwo
    \ekvc@ifdefined#1%
      {\ekvc@err@already@defined#1}%
      {%
        \expandafter
        \ekvcSplitAndUse@\expandafter#1\csname ekvc@\string#1\endcsname{#2}%
        \ifnum\ekvc@keycount<1
          \ekvc@any@long\expandafter\def\csname ekvc@\string#1\endcsname{#3}%
        \else
          \ifnum\ekvc@keycount>9
            \ekvc@err@toomany{#1}%
            \let#1\ekvc@undefined
          \else
            \ekvcSplit@build@argspec
            \ekvc@any@long\expandafter
            \def\csname ekvc@\string#1\expandafter\endcsname\ekvc@tmp{#3}%
          \fi
        \fi
      }%
  }
%    \end{macrocode}
% \end{macro}
%
% \begin{macro}[internal]{\ekvcSplit@build@argspec,\ekvcSplit@build@argspec@}
%    \begin{macrocode}
\protected\def\ekvcSplit@build@argspec
  {%
    \begingroup
      \edef\ekvc@tmp
        {\endgroup\def\unexpanded{\ekvc@tmp}{\ekvcSplit@build@argspec@{1}}}%
    \ekvc@tmp
  }
\def\ekvcSplit@build@argspec@#1%
  {%
    \ifnum#1>\ekvc@keycount
      \ekv@fi@gobble
    \fi
    \@firstofone
      {%
        \unexpanded\expandafter{\csname ekvc@splitmark@#1\endcsname####}#1%
        \expandafter\ekvcSplit@build@argspec@\expandafter{\the\numexpr#1+1}%
      }%
  }
%    \end{macrocode}
% \end{macro}
%
% \begin{macro}[internal]
%   {
%     \ekvc@SetupSplitKeys, \ekvc@SetupSplitKeys@a, \ekvc@SetupSplitKeys@b,
%     \ekvc@SetupSplitKeys@c
%   }
%   These macros parse the list of keys and set up the key macros. First we need
%   to initialise some macros and start |\ekvparse|.
%    \begin{macrocode}
\protected\long\def\ekvc@SetupSplitKeys
  {%
    \ekvc@keycount=0
    \let\ekvc@any@long\ekv@empty
    \let\ekvc@initials\ekv@empty
    \ekvparse\ekvc@err@value@required\ekvc@SetupSplitKeys@a
  }
%    \end{macrocode}
%   Then we need to step the key counter for each key. Also we have to check
%   whether this key has a |long| prefix so we initialise |\ekvc@long|.
%    \begin{macrocode}
\protected\def\ekvc@SetupSplitKeys@a#1%
  {%
    \advance\ekvc@keycount1
    \let\ekvc@long\ekv@empty
    \ekvc@ifspace{#1}%
      {\ekvc@SetupSplitKeys@b#1\ekvc@stop}%
      {\ekvc@SetupSplitKeys@c{#1}}%
  }
%    \end{macrocode}
%   If there was a space, there might be a prefix. If so call the prefix macro,
%   else call the next step |\ekvc@SetupSplitKeys@c| which will define the key
%   macro and add the key's value to the initials list.
%    \begin{macrocode}
\protected\def\ekvc@SetupSplitKeys@b#1 #2\ekvc@stop
  {%
    \ekv@ifdefined{ekvc@split@p@#1}%
      {\csname ekvc@split@p@#1\endcsname{#2}}%
      {\ekvc@SetupSplitKeys@c{#1 #2}}%
  }
%    \end{macrocode}
%   The inner definition is grouped, because we don't want to actually define
%   the marks we build with |\csname|. We have to append the value to the
%   |\ekvc@initials| list here with the correct split mark. The key macro will
%   read everything up to those split marks and change the value following it to
%   the value given to the key. Additionally we'll need a sorting macro for each
%   key count in use so we set it up with |\ekvc@setup@splitmacro|.
%    \begin{macrocode}
\protected\long\def\ekvc@SetupSplitKeys@c#1#2%
  {%
    \begingroup
    \edef\ekvc@tmp
      {%
        \endgroup
        \long\def\unexpanded{\ekvc@tmp}####1####2%
            \unexpanded\expandafter
            {\csname ekvc@splitmark@\the\ekvc@keycount\endcsname}####3%
          {%
            ####2%
            \unexpanded\expandafter
              {\csname ekvc@splitmark@\the\ekvc@keycount\endcsname}{####1}%
          }%
%    \end{macrocode}
%   The short variant needs a bit of special treatment. The key macro will be
%   short to throw the correct error, but since there might be long macros
%   somewhere the reordering of arguments needs to be long, so for short keys we
%   use a two step approach, first grabbing only the short argument, then
%   reordering.
%    \begin{macrocode}
        \unless\ifx\ekvc@long\long
          \let\unexpanded\expandafter
            {\csname ekvc@\ekvc@set(#1)\endcsname\ekvc@tmp}%
          \def\unexpanded{\ekvc@tmp}####1%
            {%
              \unexpanded\expandafter{\csname ekvc@\ekvc@set(#1)\endcsname}%
                {####1}%
            }%
        \fi
        \def\unexpanded{\ekvc@initials}%
          {%
            \unexpanded\expandafter{\ekvc@initials}%
            \unexpanded\expandafter
              {\csname ekvc@splitmark@\the\ekvc@keycount\endcsname{#2}}%
          }%
      }%
    \ekvc@tmp
    \ekvlet\ekvc@set{#1}\ekvc@tmp
    \ekvc@helpers@needed
      {\expandafter\ekvc@setup@splitmacro\expandafter{\the\ekvc@keycount}}%
      {}%
  }
%    \end{macrocode}
% \end{macro}
%
% \begin{macro}[internal]{\ekvc@split@p@long}
%   The |long| prefix lets the internals |\ekvc@long| and |\ekvc@any@long| to
%   |\long| so that the key macro will be long.
%    \begin{macrocode}
\protected\def\ekvc@split@p@long
  {%
    \let\ekvc@long\long
    \let\ekvc@any@long\long
    \ekvc@SetupSplitKeys@c
  }
%    \end{macrocode}
% \end{macro}
%
% \begin{macro}[internal]{\ekvc@defarggobbler}
%   This is needed to define a macro with 1-9 parameters programmatically.
%   \LaTeX's \cs[no-index]{newcommand} does something similar for example.
%    \begin{macrocode}
\protected\def\ekvc@defarggobbler#1{\def\ekvc@tmp##1#1##2##{##1#1}}
%    \end{macrocode}
% \end{macro}
%
% \begin{macro}[internal]
%   {
%     \ekvc@setup@splitmacro,
%     \ekvc@split@1, \ekvc@split@2, \ekvc@split@3,
%     \ekvc@split@4, \ekvc@split@5, \ekvc@split@6,
%     \ekvc@split@7
%   }
%   Since the first few split macros are different from the others we manually
%   set those up now. All the others will be defined as needed (always
%   globally).  The split macros just read up until the correct split mark, move
%   that argument into a list and reinsert the rest, calling the next split
%   macro afterwards.
%    \begin{macrocode}
\begingroup
\edef\ekvc@tmp
  {%
    \long\gdef\unexpanded\expandafter{\csname ekvc@split@1\endcsname}%
        \unexpanded\expandafter{\csname ekvc@splitmark@1\endcsname}%
        ##1##2##3%
      {##3{##1}##2}%
    \long\gdef\unexpanded\expandafter{\csname ekvc@split@2\endcsname}%
        \unexpanded\expandafter{\csname ekvc@splitmark@1\endcsname}##1%
        \unexpanded\expandafter{\csname ekvc@splitmark@2\endcsname}##2%
        ##3##4%
      {##4{##1}{##2}##3}%
    \long\gdef\unexpanded\expandafter{\csname ekvc@split@3\endcsname}%
        \unexpanded\expandafter{\csname ekvc@splitmark@1\endcsname}##1%
        \unexpanded\expandafter{\csname ekvc@splitmark@2\endcsname}##2%
        \unexpanded\expandafter{\csname ekvc@splitmark@3\endcsname}##3%
        ##4##5%
      {##5{##1}{##2}{##3}##4}%
    \long\gdef\unexpanded\expandafter{\csname ekvc@split@4\endcsname}%
        \unexpanded\expandafter{\csname ekvc@splitmark@1\endcsname}##1%
        \unexpanded\expandafter{\csname ekvc@splitmark@2\endcsname}##2%
        \unexpanded\expandafter{\csname ekvc@splitmark@3\endcsname}##3%
        \unexpanded\expandafter{\csname ekvc@splitmark@4\endcsname}##4%
        ##5##6%
      {##6{##1}{##2}{##3}{##4}##5}%
    \long\gdef\unexpanded\expandafter{\csname ekvc@split@5\endcsname}%
        \unexpanded\expandafter{\csname ekvc@splitmark@1\endcsname}##1%
        \unexpanded\expandafter{\csname ekvc@splitmark@2\endcsname}##2%
        \unexpanded\expandafter{\csname ekvc@splitmark@3\endcsname}##3%
        \unexpanded\expandafter{\csname ekvc@splitmark@4\endcsname}##4%
        \unexpanded\expandafter{\csname ekvc@splitmark@5\endcsname}##5%
        ##6##7%
      {##7{##1}{##2}{##3}{##4}{##5}##6}%
    \long\gdef\unexpanded\expandafter{\csname ekvc@split@6\endcsname}%
        \unexpanded\expandafter{\csname ekvc@splitmark@1\endcsname}##1%
        \unexpanded\expandafter{\csname ekvc@splitmark@2\endcsname}##2%
        \unexpanded\expandafter{\csname ekvc@splitmark@3\endcsname}##3%
        \unexpanded\expandafter{\csname ekvc@splitmark@4\endcsname}##4%
        \unexpanded\expandafter{\csname ekvc@splitmark@5\endcsname}##5%
        \unexpanded\expandafter{\csname ekvc@splitmark@6\endcsname}##6%
        ##7##8%
      {##8{##1}{##2}{##3}{##4}{##5}{##6}##7}%
    \long\gdef\unexpanded\expandafter{\csname ekvc@split@7\endcsname}%
        \unexpanded\expandafter{\csname ekvc@splitmark@1\endcsname}##1%
        \unexpanded\expandafter{\csname ekvc@splitmark@2\endcsname}##2%
        \unexpanded\expandafter{\csname ekvc@splitmark@3\endcsname}##3%
        \unexpanded\expandafter{\csname ekvc@splitmark@4\endcsname}##4%
        \unexpanded\expandafter{\csname ekvc@splitmark@5\endcsname}##5%
        \unexpanded\expandafter{\csname ekvc@splitmark@6\endcsname}##6%
        \unexpanded\expandafter{\csname ekvc@splitmark@7\endcsname}##7%
        ##8##9%
      {##9{##1}{##2}{##3}{##4}{##5}{##6}{##7}##8}%
  }
\ekvc@tmp
\endgroup
\protected\def\ekvc@setup@splitmacro#1%
  {%
    \ekv@ifdefined{ekvc@split@#1}{}%
      {%
        \begingroup
          \edef\ekvc@tmp
            {%
              \long\gdef
                  \unexpanded\expandafter{\csname ekvc@split@#1\endcsname}%
                  ####1%
                  \unexpanded\expandafter{\csname ekvc@splitmark@#1\endcsname}%
                  ####2####3%
                {%
                  \unexpanded\expandafter
                    {\csname ekvc@split@\the\numexpr#1-1\relax\endcsname}%
                  ####1{{####2}####3}%
                }%
            }%
          \ekvc@tmp
        \endgroup
      }%
  }
%    \end{macrocode}
% \end{macro}
%
% \begin{macro}{\ekvcHashAndUse}
%   |\ekvcHashAndUse| works just like |\ekvcSplitAndUse|.
%    \begin{macrocode}
\protected\long\def\ekvcHashAndUse#1#2%
  {%
    \let\ekvc@helpers@needed\@firstoftwo
    \ekvc@ifdefined#1%
      {\ekvc@err@already@defined#1}%
      {\ekvcHashAndUse@#1{}{#2}}%
  }
%    \end{macrocode}
% \end{macro}
%
% \begin{macro}[internal]{\ekvcHashAndUse@}
%   This is more or less the same as |\ekvcSplitAndUse@|. Instead of an empty
%   group we place a marker after the initials, we don't use the sorting macros
%   of |split|, but instead pack all the values in one argument.
%    \begin{macrocode}
\protected\long\def\ekvcHashAndUse@#1#2#3%
  {%
    \edef\ekvc@set{\string#1}%
    \ekvc@SetupHashKeys{#3}%
    \ekvc@helpers@needed
      {%
        \ekvc@any@long\edef#1##1%
          {%
            \expandafter\ekvc@ekvset@pre@expander\expandafter{\ekvc@set}%
            \unexpanded{\ekvc@hash@pack@argument}%
            \unexpanded\expandafter{\ekvc@initials\ekvc@stop#2}%
          }%
      }%
      {%
        \ekvc@any@long\edef#1##1%
          {%
            \expandafter\ekvc@ekvset@pre@expander\expandafter{\ekvc@set}%
            \unexpanded{#2}%
            \unexpanded\expandafter{\ekvc@initials\ekvc@stop}%
          }%
      }%
  }
%    \end{macrocode}
% \end{macro}
%
% \begin{macro}{\ekvcHashAndForward}
%   |\ekvcHashAndForward| works just like |\ekvcSplitAndForward|.
%    \begin{macrocode}
\protected\long\def\ekvcHashAndForward#1#2#3%
  {%
    \let\ekvc@helpers@needed\@firstoftwo
    \ekvc@ifdefined#1%
      {\ekvc@err@already@defined#1}%
      {\ekvcHashAndUse@#1{{#2}}{#3}}%
  }
%    \end{macrocode}
% \end{macro}
%
% \begin{macro}{\ekvcHash}
%   |\ekvcHash| does the same as |\ekvcSplit|, but has the advantage of not
%   needing to count arguments, so the definition of the internal macro is a bit
%   more straight forward.
%    \begin{macrocode}
\protected\long\def\ekvcHash#1#2#3%
  {%
    \let\ekvc@helpers@needed\@secondoftwo
    \ekvc@ifdefined#1%
      {\ekvc@err@already@defined#1}%
      {%
        \expandafter
        \ekvcHashAndUse@\expandafter#1\csname ekvc@\string#1\endcsname{#2}%
        \ekvc@any@long\expandafter\def\csname ekvc@\string#1\endcsname
            ##1\ekvc@stop
          {#3}%
      }%
  }
%    \end{macrocode}
% \end{macro}
%
% \begin{macro}[internal]{\ekvc@hash@pack@argument}
%   All this macro does is pack the values into one argument and forward that to
%   the next macro.
%    \begin{macrocode}
\long\def\ekvc@hash@pack@argument#1\ekvc@stop#2{#2{#1}}
%    \end{macrocode}
% \end{macro}
%
% \begin{macro}[internal]
%   {\ekvc@SetupHashKeys,\ekvc@SetupHashKeys@a,\ekvc@SetupHashKeys@b}
%   This should look awfully familiar as well, since it's just the same as for
%   the split keys with a few other names here and there.
%    \begin{macrocode}
\protected\long\def\ekvc@SetupHashKeys#1%
  {%
    \let\ekvc@any@long\ekv@empty
    \let\ekvc@initials\ekv@empty
    \ekvparse\ekvc@err@value@required\ekvc@SetupHashKeys@a{#1}%
  }
\protected\def\ekvc@SetupHashKeys@a#1%
  {%
    \let\ekvc@long\ekv@empty
    \ekvc@ifspace{#1}%
      {\ekvc@SetupHashKeys@b#1\ekvc@stop}%
      {\ekvc@SetupHashKeys@c{#1}}%
  }
\protected\def\ekvc@SetupHashKeys@b#1 #2\ekvc@stop
  {%
    \ekv@ifdefined{ekvc@hash@p@#1}%
      {\csname ekvc@hash@p@#1\endcsname{#2}}%
      {\ekvc@SetupHashKeys@c{#1 #2}}%
  }
%    \end{macrocode}
%   Yes, even the defining macro looks awfully familiar. Instead of numbered we
%   have named marks. Still the key macros grab everything up to their
%   respective mark and reorder the arguments. The same quirk is applied for
%   short keys. And instead of the |\ekvc@setup@splitmacro| we use
%   |\ekvc@setup@hashmacro|.
%    \begin{macrocode}
\protected\long\def\ekvc@SetupHashKeys@c#1#2%
  {%
    \begingroup
    \edef\ekvc@tmp
      {%
        \endgroup
        \long\def\unexpanded{\ekvc@tmp}####1####2%
            \unexpanded\expandafter{\csname ekvc@hashmark@#1\endcsname}####3%
          {%
            ####2%
            \unexpanded\expandafter{\csname ekvc@hashmark@#1\endcsname}{####1}%
          }%
        \unless\ifx\ekvc@long\long
          \let\unexpanded\expandafter
            {\csname ekvc@\ekvc@set(#1)\endcsname\ekvc@tmp}%
          \def\unexpanded{\ekvc@tmp}####1%
            {%
              \unexpanded\expandafter{\csname ekvc@\ekvc@set(#1)\endcsname}%
                {####1}%
            }%
        \fi
        \def\unexpanded{\ekvc@initials}%
          {%
            \unexpanded\expandafter{\ekvc@initials}%
            \unexpanded\expandafter{\csname ekvc@hashmark@#1\endcsname{#2}}%
          }%
      }%
    \ekvc@tmp
    \ekvlet\ekvc@set{#1}\ekvc@tmp
    \ekvc@setup@hashmacro{#1}%
  }
%    \end{macrocode}
% \end{macro}
%
% \begin{macro}[internal]{\ekvc@hash@p@long}
%   Nothing astonishing here either.
%    \begin{macrocode}
\protected\def\ekvc@hash@p@long
  {%
    \let\ekvc@long\long
    \let\ekvc@any@long\long
    \ekvc@SetupHashKeys@c
  }
%    \end{macrocode}
% \end{macro}
%
% \begin{macro}[internal]{\ekvc@setup@hashmacro}
%   The safe hash macros will be executed inside of an |\unexpanded| expansion
%   context, so they have to insert braces for that once they are done.
%   Most of the tests which have to be executed will already be done, but we
%   have to play safe if the hash doesn't show up in the hash list. Therefore we
%   use some |\ekvc@mark|s and |\ekvc@stop| to throw errors if the hash isn't
%   found in the right place. The fast variants have an easier life and just
%   return the correct value.
%    \begin{macrocode}
\protected\def\ekvc@setup@hashmacro#1%
  {%
    \ekv@ifdefined{ekvc@fasthash@#1}{}%
      {%
        \begingroup
          \edef\ekvc@tmp
            {%
              \long\gdef
                \unexpanded\expandafter{\csname ekvc@fasthash@#1\endcsname}%
                  ####1%
                  \unexpanded\expandafter{\csname ekvc@hashmark@#1\endcsname}%
                  ####2####3\unexpanded{\ekvc@stop}%
                {####2}%
              \long\gdef
                \unexpanded\expandafter{\csname ekvc@safehash@#1\endcsname}%
                  ####1%
                {%
                  \unexpanded\expandafter{\csname ekvc@@safehash@#1\endcsname}%
                  ####1\unexpanded{\ekvc@mark}{}%
                  \unexpanded\expandafter
                    {%
                      \csname ekvc@hashmark@#1\endcsname{}%
                      \ekvc@mark{\ekvc@err@missing@hash{#1}}\ekvc@stop
                    }%
                }%
              \long\gdef
                \unexpanded\expandafter{\csname ekvc@@safehash@#1\endcsname}%
                  ####1%
                  \unexpanded\expandafter{\csname ekvc@hashmark@#1\endcsname}%
                  ####2####3\unexpanded{\ekvc@mark}####4####5%
                  \unexpanded{\ekvc@stop}%
                {%
                  ####4{####2}%
                }%
              \long\gdef\unexpanded\expandafter
                {\csname ekvc@fastsplithash@#1\endcsname}%
                  ####1%
                  \unexpanded\expandafter{\csname ekvc@hashmark@#1\endcsname}%
                  ####2####3\unexpanded{\ekvc@stop}####4%
                {%
                  ####4{####1####3}{####2}%
                }%
              \long\gdef\unexpanded\expandafter
                {\csname ekvc@safesplithash@#1\endcsname}####1%
                {%
                  \unexpanded\expandafter
                    {\csname ekvc@@safesplithash@#1\endcsname}%
                  ####1\unexpanded{\ekvc@mark\ekvc@safe@found@hash}%
                  \unexpanded\expandafter
                    {%
                      \csname ekvc@hashmark@#1\endcsname{}%
                      \ekvc@mark{\ekvc@err@missing@hash{#1}\ekvc@safe@no@hash}%
                      \ekvc@stop
                    }%
                }%
              \long\gdef\unexpanded\expandafter
                {\csname ekvc@@safesplithash@#1\endcsname}%
                  ####1%
                  \unexpanded\expandafter{\csname ekvc@hashmark@#1\endcsname}%
                  ####2####3\unexpanded{\ekvc@mark}####4####5%
                  \unexpanded{\ekvc@stop}%
                {%
                  ####4{####2}####1####3\unexpanded{\ekvc@stop}%
                }%
            }%
          \ekvc@tmp
        \endgroup
      }%
  }
%    \end{macrocode}
% \end{macro}
%
% \begin{macro}{\ekvcValue}
%   All this does is a few consistency checks on the first argument (not empty,
%   hash macro exists) and then call that hash-grabbing macro that will also
%   test whether the hash is inside of |#2| or not.
%    \begin{macrocode}
\long\def\ekvcValue#1%
  {%
    \unexpanded
    \ekv@ifdefined{ekvc@safehash@#1}%
      {\csname ekvc@safehash@#1\endcsname}%
      {\ekvc@err@unknown@hash{#1}\@firstoftwo{{}}}%
  }
%    \end{macrocode}
% \end{macro}
%
% \begin{macro}{\ekvcValueFast}
%   To be as fast as possible, this doesn't test for anything, assuming the user
%   knows best.
%    \begin{macrocode}
\long\def\ekvcValueFast#1#2{\csname ekvc@fasthash@#1\endcsname#2\ekvc@stop}
%    \end{macrocode}
% \end{macro}
%
% \begin{macro}{\ekvcValueSplit,\ekvcValueSplit@recover}
%   This splits off a single value.
%    \begin{macrocode}
\long\def\ekvcValueSplit#1%
  {%
    \ekv@ifdefined{ekvc@safesplithash@#1}%
      {\csname ekvc@safesplithash@#1\endcsname}%
      {\ekvc@err@unknown@hash{#1}\ekvcValueSplit@recover}%
  }
\long\def\ekvcValueSplit@recover#1#2%
  {%
    #2{#1}{}%
  }
%    \end{macrocode}
% \end{macro}
%
% \begin{macro}[internal]{\ekvc@safe@found@hash, \ekvc@safe@no@hash}
%    \begin{macrocode}
\long\def\ekvc@safe@found@hash#1#2\ekvc@stop#3%
  {%
    #3{#2}{#1}%
  }
\long\def\ekvc@safe@no@hash#1#2\ekvc@mark\ekvc@safe@found@hash\ekvc@stop#3%
  {%
    #3{#2}{}%
  }
%    \end{macrocode}
% \end{macro}
%
% \begin{macro}{\ekvcValueSplitFast}
%   Again a fast approach which doesn't provide too many safety measurements.
%   This needs to build the hash function and expand it before passing the
%   results to the next control sequence. The first step only builds the control
%   sequence.
%    \begin{macrocode}
\long\def\ekvcValueSplitFast#1#2%
  {%
    \csname ekvc@fastsplithash@#1\endcsname#2\ekvc@stop
  }
%    \end{macrocode}
% \end{macro}
%
% \begin{macro}[internal]
%   {
%     \ekvc@safehash@,\ekvc@fasthash@,
%     \ekvc@safesplithash@,\ekvc@fastsplithash@
%   }
%   At least in the empty hash case we can provide a meaningful error message
%   without affecting performance by just defining the macro that would be build
%   in that case. There is of course a downside to this, the error will not be
%   thrown by |\ekvcValueFast| in three expansion steps. The safe hash variant
%   has to also stop the |\unexpanded| expansion.
%    \begin{macrocode}
\long\def\ekvc@safehash@#1{\ekvc@err@empty@hash{}}
\long\def\ekvc@fasthash@#1\ekvc@stop{\ekvc@err@empty@hash}
\long\def\ekvc@safesplithash@#1#2{\ekvc@err@empty@hash#2{#1}{}}
\long\def\ekvc@fastsplithash@#1\ekvc@stop#2{\ekvc@err@empty@hash#2{#1}{}}
%    \end{macrocode}
% \end{macro}
%
% \begin{macro}{\ekvcSecondaryKeys}
%   The secondary keys are defined pretty similar to the way the originals are,
%   but here we also introduce some key types (those have a |@t@| in their name)
%   additionally to the prefixes.
%    \begin{macrocode}
\protected\long\def\ekvcSecondaryKeys#1#2%
  {%
    \edef\ekvc@set{\string#1}%
    \ekvparse\ekvc@err@value@required\ekvcSecondaryKeys@a{#2}%
  }
\protected\def\ekvcSecondaryKeys@a#1%
  {%
    \let\ekvc@long\ekv@empty
    \ekvc@ifspace{#1}%
      {\ekvcSecondaryKeys@b#1\ekvc@stop}%
      {\ekvc@err@missing@type{#1}\@gobble}%
  }
\protected\def\ekvcSecondaryKeys@b#1 #2\ekvc@stop
  {%
    \ekv@ifdefined{ekvc@p@#1}%
      {\csname ekvc@p@#1\endcsname}%
      {%
        \ekv@ifdefined{ekvc@t@#1}%
          {\csname ekvc@t@#1\endcsname}%
          {\ekvc@err@unknown@keytype{#1}\@firstoftwo\@gobble}%
      }%
      {#2}%
  }
%    \end{macrocode}
% \end{macro}
%
% \begin{macro}{\ekvcChange}
%   This can be used to change the defaults of an \expkvc\ defined macro. It
%   checks whether there is a set with the correct name and that the macro is
%   defined. If both is true the real work is done by |\ekvc@change|.
%    \begin{macrocode}
\protected\long\def\ekvcChange#1%
  {%
    \ekvifdefinedset{\string#1}%
      {%
        \ekvc@ifdefined#1%
          {\ekvc@change#1}%
          {\ekvc@err@no@key@macro#1\@gobble}%
      }%
      {\ekvc@err@no@key@macro#1\@gobble}%
  }
%    \end{macrocode}
% \end{macro}
%
% \begin{macro}[internal]
%   {\ekvc@change,\ekvc@change@a,\ekvc@change@b,\ekvc@change@c}
%   First we need to see whether the macro is currently |\long|. For this we get
%   the meaning and will parse it. |#1| is the macro name in which we want to
%   change the defaults.
%    \begin{macrocode}
\protected\def\ekvc@change#1%
  {\expandafter\ekvc@change@a\meaning#1\ekv@stop#1}
%    \end{macrocode}
%   A temporary definition to get the stringified |macro:|. |##1| will be the
%   list of prefixes, we don't care for the exact contents of |##2| and |##3|.
%    \begin{macrocode}
\def\ekvc@change@a#1%
  {%
    \protected\def\ekvc@change@a##1#1##2->##3\ekv@stop
      {%
        \ekvc@change@iflong{##1}%
          {\ekvc@change@b{}}%
          {\ekvc@change@b{\long}}%
      }%
  }
\expandafter\ekvc@change@a\expandafter{\detokenize{macro:}}
%    \end{macrocode}
%   Next we expand the macro once to get its contents (including the current
%   default values with their markers). |#1| is either |\long| or empty, |#2| is
%   the macro.
%    \begin{macrocode}
\protected\def\ekvc@change@b#1#2%
  {\expandafter\ekvc@change@c\expandafter{#2{##1}}{#1}#2}
%    \end{macrocode}
%   Here we place an unbalanced closing brace after the expansion of the macro.
%   Then we just parse the \kv-list with |\ekvset|, that will exchange the
%   values behind the markers. Once those are changed we give control to
%   |\ekvc@change@d|. The |\ekvset| step might horribly fail if the user defined
%   some keys that don't behave nice. |#1| is the expansion of the macro, |#2|
%   is either |\long| or empty, |#3| is the macro, and |#4| is the \kv-list
%   containing the new defaults.
%    \begin{macrocode}
\ekv@exparg{\protected\long\def\ekvc@change@c#1#2#3#4}%
  {%
    \expandafter\iffalse\expandafter{\expandafter\fi
      \ekvset{\string#3}{#4}%
      \ekvc@change@d{#2}{#3}%
      #1%
    }%
  }
%    \end{macrocode}
%   The final step needs to put an unbalanced opening brace after |\def|. We do
%   that with the help of a temporary macro which stores everything necessary
%   for |\def| and expand an |\iffalse}\fi| construct to leave a single
%   opening brace. |#1| will be either empty or |\long| and |#2| is the macro.
%   Each of the macros defined with \expkvc\ takes exactly one parameter, so we
%   put that here as |####1|.
%    \begin{macrocode}
\protected\def\ekvc@change@d#1#2%
  {%
    \def\ekvc@tmp{#1\def#2####1}%
    \expandafter\ekvc@tmp\expandafter{\iffalse}\fi
  }
%    \end{macrocode}
% \end{macro}
%
% \begin{macro}[internal]{\ekvc@change@iflong,\ekvc@change@iflong@}
%   Checking whether a string contains the string representation of |\long| can
%   be done by gobbling everything up to the first |\long| and checking whether
%   the result is completely empty. We need a temporary macro to get the result
%   of |\string\long| inside the definitions.
%    \begin{macrocode}
\def\ekvc@change@iflong#1%
  {%
    \protected\def\ekvc@change@iflong##1%
      {\expandafter\ekv@ifempty\expandafter{\ekvc@change@iflong@##1#1}}%
    \def\ekvc@change@iflong@##1#1{}
  }
\expandafter\ekvc@change@iflong\expandafter{\string\long}
%    \end{macrocode}
% \end{macro}
%
% \begin{macro}{\ekvcPass}
%   This macro can be used to pass a value to a key of some macro (this way
%   more complicated key codes are possible that in the end pass processed
%   values on to some macro). The implemantation is pretty straight forward.
%    \begin{macrocode}
\long\def\ekvcPass#1#2%
  {%
    \ekvifdefined{\string#1}{#2}%
      {\csname\ekv@name{\string#1}{#2}\endcsname}%
      {\ekvc@err@unknown@key@or@macro{#1}{#2}\@gobble}%
  }
%    \end{macrocode}
% \end{macro}
%
%
% \subsubsection{Secondary Key Types}
%
% \begin{macro}[internal]{\ekvc@p@long,\ekvc@after@ptype}
%   The prefixes are pretty straight forward again. Just set |\ekvc@long| and
%   forward to the |@t@| type.
%    \begin{macrocode}
\protected\def\ekvc@p@long#1%
  {%
    \ekvc@ifspace{#1}%
      {%
        \let\ekvc@long\long
        \ekvc@after@ptype#1\ekvc@stop
      }%
      {\ekvc@err@missing@type{long #1}\@gobble}%
  }
\protected\def\ekvc@after@ptype#1 #2\ekvc@stop
  {%
    \ekv@ifdefined{ekvc@t@#1}%
      {\csname ekvc@t@#1\endcsname{#2}}%
      {\ekvc@err@unknown@keytype{#1}\@gobble}%
  }
%    \end{macrocode}
% \end{macro}
%
% \begin{macro}[internal]
%   {
%     \ekvc@t@meta, \ekvc@t@nmeta,
%     \ekvc@type@meta, \ekvc@type@meta@a, \ekvc@type@meta@b
%   }
%   The |meta| and |nmeta| key types use a nested |\ekvset| to set other keys in
%   the same macro's \set.
%    \begin{macrocode}
\protected\def\ekvc@t@meta
  {%
    \edef\ekvc@tmp{\ekvc@set}%
    \expandafter\ekvc@type@meta\expandafter{\ekvc@tmp}\ekvc@long{##1}\ekvlet
  }
\protected\def\ekvc@t@nmeta#1%
  {%
    \ekvc@assert@not@long{nmeta #1}%
    \edef\ekvc@tmp{\ekvc@set}%
    \expandafter\ekvc@type@meta\expandafter{\ekvc@tmp}{}{}\ekvletNoVal{#1}%
  }
\protected\long\def\ekvc@type@meta#1#2#3#4#5#6%
  {%
    \expandafter\ekvc@type@meta@a\expandafter{\ekvset{#1}{#6}}{#2}{#3}%
    #4\ekvc@set{#5}\ekvc@tmp
  }
\protected\def\ekvc@type@meta@a
  {%
    \expandafter\ekvc@type@meta@b\expandafter
  }
\protected\long\def\ekvc@type@meta@b#1#2#3%
  {%
    #2\def\ekvc@tmp#3{#1}%
  }
%    \end{macrocode}
% \end{macro}
%
% \begin{macro}[internal]{\ekvc@t@alias}
%   |alias| just checks whether there is a key and/or |NoVal| key defined with
%   the target name and |\let| the key to those.
%    \begin{macrocode}
\protected\def\ekvc@t@alias#1#2%
  {%
    \ekvc@assert@not@long{alias #1}%
    \let\ekvc@tmp\@firstofone
    \ekvifdefined\ekvc@set{#2}%
      {%
        \ekvletkv\ekvc@set{#1}\ekvc@set{#2}%
        \let\ekvc@tmp\@gobble
      }%
      {}%
    \ekvifdefinedNoVal\ekvc@set{#2}%
      {%
        \ekvletkvNoVal\ekvc@set{#1}\ekvc@set{#2}%
        \let\ekvc@tmp\@gobble
      }%
      {}%
    \ekvc@tmp{\ekvc@err@unknown@key{#2}}%
  }
%    \end{macrocode}
% \end{macro}
%
% \begin{macro}[internal]{\ekvc@t@default}
%   The |default| key can be used to set a |NoVal| key for an existing key. It
%   will just pass the \val\ to the key macro of that other key.
%    \begin{macrocode}
\protected\long\def\ekvc@t@default#1#2%
  {%
    \ekvifdefined\ekvc@set{#1}%
      {%
        \ekvc@assert@not@long{default #1}%
        \edef\ekvc@tmp
          {%
            \unexpanded\expandafter
              {\csname\ekv@name\ekvc@set{#1}\endcsname{#2}}%
          }%
        \ekvletNoVal\ekvc@set{#1}\ekvc@tmp
      }%
      {\ekvc@err@unknown@key{#1}}%
  }
%    \end{macrocode}
% \end{macro}
%
% \begin{macro}[internal]{\ekvc@t@flag-bool}
%    \begin{macrocode}
\protected\expandafter\def\csname ekvc@t@flag-bool\endcsname#1#2%
  {%
    \ekvc@assert@not@long{flag-bool #1}%
    \unless\ifdefined#2\ekvcFlagNew#2\fi
    \ekvdef\ekvc@set{#1}%
      {%
        \ekv@ifdefined{ekvc@flag@set@##1}%
          {%
            \csname ekvc@flag@set@##1\expandafter\endcsname
              \ekvcFlagHeight#2\ekv@stop#2%
          }%
          {\ekvc@err@invalid@bool{##1}}%
      }%
  }
%    \end{macrocode}
% \end{macro}
%
% \begin{macro}[internal]
%   {\ekvc@t@flag-true,\ekvc@t@flag-false,\ekvc@t@flag-raise,\ekvc@type@flag}
%    \begin{macrocode}
\protected\def\ekvc@type@flag#1#2#3#4%
  {%
    \ekvc@assert@not@long{flag-#1 #3}%
    \unless\ifdefined#4\ekvcFlagNew#4\fi
    \ekv@exparg{\ekvdefNoVal\ekvc@set{#3}}{#2#4}%
  }
\protected\expandafter\def\csname ekvc@t@flag-true\endcsname
  {\ekvc@type@flag{true}\ekvcFlagSetTrue}
\protected\expandafter\def\csname ekvc@t@flag-false\endcsname
  {\ekvc@type@flag{false}\ekvcFlagSetFalse}
\protected\expandafter\def\csname ekvc@t@flag-raise\endcsname
  {\ekvc@type@flag{raise}\ekvcFlagRaise}
%    \end{macrocode}
% \end{macro}
%
%
% \subsubsection{Flags}
%
% The basic idea of flags is to store information by the fact that \TeX\
% expandably assigns the meaning |\relax| to undefined control sequences which
% were built with |\csname|. This mechanism is borrowed from \pkg{expl3}.
%
% \begin{macro}[internal]{\ekvc@flag@name,\ekvc@flag@namescheme}
%   Flags follow a simple naming scheme which we define here. |\ekvc@flag@name|
%   will store the name of an internal function that is used to build names of
%   the second naming scheme defined by |\ekvc@flag@namescheme|.
%    \begin{macrocode}
\def\ekvc@flag@name{ekvcf\string}
\def\ekvc@flag@namescheme#1#2{ekvch#2#1}
%    \end{macrocode}
% \end{macro}
%
% \begin{macro}{\ekvcFlagHeight}
%   For semantic reasons we use |\number| with another name.
%    \begin{macrocode}
\let\ekvcFlagHeight\number
%    \end{macrocode}
% \end{macro}
%
% \begin{macro}{\ekvcFlagNew}
%   This macro defines a new flag. It stores the function build with the
%   |\ekvc@flag@name| naming scheme after the internal function
%   |\ekvc@flag@height| that'll determine the current flag height. It'll also
%   define the macro named via |\ekvc@flag@name| to build names according to
%   |\ekvc@flag@namescheme|.
%    \begin{macrocode}
\protected\def\ekvcFlagNew#1%
  {%
    \edef#1%
      {%
        \unexpanded{\ekvc@flag@height}%
          \unexpanded\expandafter{\csname\ekvc@flag@name#1\endcsname}%
      }%
    \ekv@expargtwice
      {\expandafter\def\csname\ekvc@flag@name#1\endcsname##1}%
      {\expandafter\ekvc@flag@namescheme\expandafter{\string#1}{##1}}%
  }
%    \end{macrocode}
% \end{macro}
%
% \begin{macro}[internal]{\ekvc@flag@height,\ekvc@flag@height@}
%   This macro gets the height of a flag by a simple loop. The first loop
%   iteration differs a bit from the following in that it doesn't have to get
%   the current iteration count. The space at the end of |\ekvc@flag@height|
%   ends the |\number| evaluation.
%    \begin{macrocode}
\def\ekvc@flag@height#1%
  {%
    \ifcsname#10\endcsname
      \ekvc@flag@height@1\ekv@stop#1%
    \fi
    \@firstofone{0} % leave this space
  }
\def\ekvc@flag@height@#1\ekv@stop#2\fi\@firstofone#3%
  {%
    \fi
    \ifcsname#2{#1}\endcsname
      \expandafter\ekvc@flag@height@\the\numexpr#1+1\relax\ekv@stop#2%
    \fi
    \@firstofone{#1}%
  }
%    \end{macrocode}
% \end{macro}
%
% \begin{macro}{\ekvcFlagRaise}
%   Raising a flag simply means letting the |\ekvc@flag@namescheme| macro for
%   the current height to relax. The result of raising a flag is that its height
%   is bigger by $1$.
%    \begin{macrocode}
\ekv@exparg{\def\ekvcFlagRaise#1}%
  {%
    \expandafter\expandafter\expandafter\@gobble\expandafter
      \csname\ekvc@flag@namescheme{\string#1}{\ekvcFlagHeight#1}\endcsname
  }
%    \end{macrocode}
% \end{macro}
%
% \begin{macro}{\ekvcFlagSetTrue,\ekvcFlagSetFalse}
% \begin{macro}[internal]{\ekvc@flag@set@true,\ekvc@flag@set@false}
%   A flag is considered true if its current height is odd, and as false if it
%   is even. Therefore |\ekvcFlagSetTrue| and |\ekvcFlagSetFalse| only need to
%   raise the flag if the opposing boolean value is the current one.
%    \begin{macrocode}
\def\ekvcFlagSetTrue#1%
  {\expandafter\ekvc@flag@set@true\ekvcFlagHeight#1\ekv@stop#1}
\def\ekvcFlagSetFalse#1%
  {\expandafter\ekvc@flag@set@false\ekvcFlagHeight#1\ekv@stop#1}
%    \end{macrocode}
%   We can expand |\ekvc@flag@namescheme| at definition time here, which is why
%   we're using a temporary definition to set up |\ekvc@flag@set@true| and
%   |\ekvc@flag@set@false|.
%    \begin{macrocode}
\def\ekvc@flag@set@true#1%
  {%
    \def\ekvc@flag@set@true##1\ekv@stop##2%
      {%
        \ifodd##1
          \ekv@fi@gobble
        \fi
        \@firstofone{\expandafter\@gobble\csname#1\endcsname}%
      }%
    \def\ekvc@flag@set@false##1\ekv@stop##2%
      {%
        \ifodd##1
          \ekv@fi@firstofone
        \fi
        \@gobble{\expandafter\@gobble\csname#1\endcsname}%
      }%
  }
\expandafter\ekvc@flag@set@true\expandafter
  {\ekvc@flag@namescheme{\string#2}{#1}}
%    \end{macrocode}
% \end{macro}
% \end{macro}
%
% \begin{macro}{\ekvcFlagIf}
%   As already explained, truthiness is defined as a flag's height is odd, so we
%   just branch accordingly here.
%    \begin{macrocode}
\def\ekvcFlagIf#1%
  {%
    \ifodd#1%
      \ekv@fi@firstoftwo
    \fi
    \@secondoftwo
  }
%    \end{macrocode}
% \end{macro}
%
% \begin{macro}{\ekvcFlagIfRaised}
%   This macro uses flags as a switch, if a flag's current height is bigger than
%   $0$ this test yields true.
%    \begin{macrocode}
\ekv@exparg{\def\ekvcFlagIfRaised#1}%
  {%
    \expandafter\ifcsname\ekvc@flag@namescheme{\string#1}0\endcsname
      \ekv@fi@firstoftwo
    \fi
    \@secondoftwo
  }
%    \end{macrocode}
% \end{macro}
%
% \begin{macro}{\ekvcFlagReset}
% \begin{macro}[internal]{\ekvc@flag@reset,\ekvc@flag@reset@}
%   Resetting works by locally letting all the defined internal macros named
%   after |\ekvc@flag@namescheme| to undefined.
%    \begin{macrocode}
\protected\def\ekvcFlagReset#1%
  {\expandafter\ekvc@flag@reset\csname\ekvc@flag@name#1\endcsname}
\protected\def\ekvc@flag@reset#1%
  {%
    \ifcsname#10\endcsname
      \expandafter\let\csname#10\endcsname\ekvc@undefined
      \ekvc@flag@reset@1\ekv@stop#1%
    \fi
  }
\protected\def\ekvc@flag@reset@#1\ekv@stop#2\fi
  {%
    \fi
    \ifcsname#2{#1}\endcsname
      \expandafter\let\csname#2{#1}\endcsname\ekvc@undefined
      \expandafter\ekvc@flag@reset@\the\numexpr#1+1\relax\ekv@stop#2%
    \fi
  }
%    \end{macrocode}
% \end{macro}
% \end{macro}
%
% \begin{macro}{\ekvcFlagGetHeight}
% \begin{macro}[internal]{\ekvc@flag@get@height@single}
%   These are just small helpers, first getting the height of the flag and then
%   passing it on to the user supplied code.
%    \begin{macrocode}
\def\ekvcFlagGetHeight#1%
  {\expandafter\ekvc@flag@get@height@single\ekvcFlagHeight#1\ekv@stop}
\long\def\ekvc@flag@get@height@single#1\ekv@stop#2{#2{#1}}
%    \end{macrocode}
% \end{macro}
% \end{macro}
%
% \begin{macro}{\ekvcFlagGetHeights}
% \begin{macro}[internal]
%   {\ekvc@flag@get@heights,\ekvc@flag@get@heights@,\ekvc@flag@get@heights@done}
%   This works by a simple loop that stops at |\ekv@stop|. As long as that
%   marker isn't hit, get the next flags height and put it into a list after
%   |\ekv@stop|. |\ekvc@flag@get@heights@| uses the same marker name for the
%   end of the height, which shouldn't clash in any case. Once we're done we
%   remove the remainder of the current iteration and leave the user supplied
%   code in the input stream with all the flags' heights as a single argument.
%    \begin{macrocode}
\def\ekvcFlagGetHeights#1%
  {%
    \ekvc@flag@get@heights#1\ekv@stop{}%
  }
\def\ekvc@flag@get@heights#1%
  {%
    \ekv@gobbleto@stop#1\ekvc@flag@get@heights@done\ekv@stop
    \expandafter\ekvc@flag@get@heights@\ekvcFlagHeight#1\ekv@stop
  }
\def\ekvc@flag@get@heights@#1\ekv@stop#2\ekv@stop#3%
  {\ekvc@flag@get@heights#2\ekv@stop{#3{#1}}}
\long\def\ekvc@flag@get@heights@done
    \ekv@stop
    \expandafter\ekvc@flag@get@heights@\ekvcFlagHeight\ekv@stop\ekv@stop#1#2%
  {#2{#1}}
%    \end{macrocode}
% \end{macro}
% \end{macro}
%
%
% \subsubsection{Helper Macros}
%
% \begin{macro}[internal]{\ekvc@ifspace,\ekvc@ifspace@}
% A test which can be reduced to an if-empty by gobbling everything up to the
% first space.
%    \begin{macrocode}
\long\def\ekvc@ifspace#1%
  {%
    \ekvc@ifspace@#1 \ekv@ifempty@B
      \ekv@ifempty@false\ekv@ifempty@A\ekv@ifempty@B\@firstoftwo
  }
\long\def\ekvc@ifspace@#1 % keep this space
  {%
    \ekv@ifempty@\ekv@ifempty@A
  }
%    \end{macrocode}
% \end{macro}
%
%
% \subsubsection{Assertions}
% \begin{macro}[internal]{\ekvc@assert@not@long}
%   Some keys don't want to be |long| and we have to educate the user, so let's
%   throw an error if someone wanted these to be long.
%    \begin{macrocode}
\long\def\ekvc@assert@not@long#1{\ifx\ekvc@long\long\ekvc@err@no@long{#1}\fi}
%    \end{macrocode}
% \end{macro}
%
%
% \subsubsection{Messages}
%
% \begin{macro}[internal]
%   {
%     \ekvc@err@toomany,\ekvc@err@value@required,\ekvc@err@missing@type,
%     \ekvc@err@already@defined,\ekvc@err@no@key@macro
%   }
%   Boring unexpandable error messages.
%    \begin{macrocode}
\protected\def\ekvc@err@toomany#1%
  {%
    \errmessage{expkv-cs Error: Too many keys for macro `\string#1'}%
  }
\protected\def\ekvc@err@value@required#1%
  {%
    \errmessage{expkv-cs Error: Missing value for key `\unexpanded{#1}'}%
  }
\protected\def\ekvc@err@missing@type#1%
  {%
    \errmessage
      {expkv-cs Error: Missing type for secondary key `\unexpanded{#1}'}%
  }
\protected\def\ekvc@err@no@long#1%
  {%
    \errmessage
      {expkv-cs Error: prefix `long' not accepted for `\unexpanded{#1}'}%
  }
\protected\def\ekvc@err@already@defined#1%
  {%
    \errmessage{expkv-cs Error: Macro `\string#1' already defined}%
  }
\protected\def\ekvc@err@unknown@keytype#1%
  {%
    \errmessage{expkv-cs Error: Unknown key type `\unexpanded{#1}'}%
  }
\protected\def\ekvc@err@unknown@key#1%
  {%
    \errmessage
      {expkv-cs Error: Unknown key `\unexpanded{#1}' for macro `\ekvc@set'}%
  }
\long\def\ekvc@err@no@key@macro#1%
  {\errmessage{expkv-cs Error: \string#1 is no key=val macro}}
%    \end{macrocode}
% \end{macro}
%
% \begin{macro}[internal]{\ekvc@err}
%   We need a way to throw error messages expandably in some contexts.
%    \begin{macrocode}
\ekv@exparg{\long\def\ekvc@err#1}{\ekverr{expkv-cs}{#1}}
%    \end{macrocode}
% \end{macro}
%
% \begin{macro}[internal]
%   {
%     \ekvc@err@unknown@hash,\ekvc@err@empty@hash,\ekvc@err@missing@hash,
%     \ekvc@err@invalid@bool
%   }
%   And here are the expandable error messages.
%    \begin{macrocode}
\long\def\ekvc@err@unknown@hash#1{\ekvc@err{unknown hash `#1'}}
\long\def\ekvc@err@missing@hash#1{\ekvc@err{hash `#1' not found}}
\long\def\ekvc@err@empty@hash{\ekvc@err{empty hash}}
\def\ekvc@err@invalid@bool#1{\ekvc@err{invalid boolean value `#1'}}
\long\def\ekvc@err@unknown@key@or@macro#1#2%
  {\ekvc@err{unknown key `#2' for macro #1}}
%    \end{macrocode}
% \end{macro}
%
%
% Now everything that's left is to reset the category code of |@|.
%    \begin{macrocode}
\catcode`\@=\ekvc@tmp
%    \end{macrocode}
%
% \gobbledocstriptag
%</tex>
%^^A=<<
%
%^^A=<<
%
% \end{implementation}^^A=<<
%
% \clearpage
% \PrintIndex
%
\endinput
%
^^A vim: ft=tex fdm=marker fmr=>>=,=<<
}{}
\ProvidesFile{expkv-cs.dtx}
  [\csname ekvcDate\endcsname\ define expandable key=val macros using expkv]
\PassOptionsToPackage{full}{textcomp}
\documentclass{l3doc}
\RequirePackage[oldstylenums,nott]{kpfonts}
\input{glyphtounicode}
\pdfgentounicode=1
\RequirePackage{tcolorbox}
\newtcolorbox{exresult}[2][]
  {colback=expkvgrey!10!white,colframe=expkvgrey,fontupper=\small,width={#2},#1}
\newtcbox\exres[1][]
  {
     colback=expkvgrey!10!white
    ,colframe=expkvgrey
    ,size=small
    ,nobeforeafter
    ,tcbox raise base
    ,fontupper=\small
    ,#1
  }
\RequirePackage{listings}
\RequirePackage{booktabs}
\RequirePackage{array}
\RequirePackage{collcell}
\RequirePackage{siunitx}
\RequirePackage{xcolor}
\RequirePackage{caption}
\RequirePackage{microtype}
\RequirePackage{accsupp}
\lstdefinelanguage{expkvc}[primitive]{TeX}
  {
    ,moretexcs=[4]^^A e-TeX
      {
        expanded,
        numexpr,
        protected,
      }
    ,moretexcs=[5]^^A plain/LaTeX
      {
        approx,
        begin,
        empty,
        item,
        LaTeX,
        makeatletter,makeatother,multicolumn,
        newcommand,newdimen,
        RequirePackage,
        rule,
        TeX,
        textit,texttt,
        usepackage,
      }
    ,moretexcs=[6]^^A used packages
      {
        ^^A xfp
        fpeval,
        ^^A keyval
        setkeys,
        ^^A yax
        defactiveparameter,storevalue,setparameterlist
      }
    ,moretexcs=[2]^^A expkv-cs macros
      {
        ekvcSplit,ekvcSplitAndForward,ekvcSplitAndUse,
        ekvcHash,ekvcHashAndForward,ekvcHashAndUse,
        ekvcValue,ekvcValueFast,ekvcValueSplit,ekvcValueSplitFast,
        ekvcSecondaryKeys,
        ekvcChange,
        ekvcFlagNew,
        ekvcFlagHeight,
        ekvcFlagRaise,
        ekvcFlagSetTrue,ekvcFlagSetFalse,
        ekvcFlagIf,ekvcFlagIfRaised,
        ekvcFlagReset,
        ekvcFlagGetHeight,ekvcFlagGetHeights,
        ekvcPass
      }
    ,moretexcs=[3]^^A expkv-pkg macros
      {
        ^^A expkv
        ekvdef,ekvdefNoVal,^^A
        ekvlet,ekvletNoVal,ekvletkv,ekvletkvNoVal,^^A
        ekvdefunknown,ekvdefunknownNoVal,^^A
        ekvredirectunknown,ekvredirectunknownNoVal,^^A
        ekvset,ekvsetSneaked,^^A
        ekvsetdef,ekvsetSneakeddef,ekvsetdefSneaked,^^A
        ekvparse,^^A
        ekvVersion,ekvDate,^^A
        ekvifdefined,ekvifdefinedNoVal,ekvifdefinedset,^^A
        ekvbreak,ekvbreakPreSneak,ekvbreakPostSneak,^^A
        ekvsneak,ekvsneakPre,^^A
        ekvchangeset,^^A
        ekvoptarg,ekvoptargTF,^^A
        ekverr,
        ^^A expkv-def
        ekvdefinekeys
      }
  }
\colorlet{codeparam}{cyan!65!black}
\def\mylstwd{.55em}
\lstset
  {
    ,language=expkvc
    ,flexiblecolumns=false
    ,basewidth=\mylstwd
    ,gobble=2
    ,basicstyle=\fontfamily{jkp}\itshape
    ,texcsstyle=*[1]{\bfseries\color{expkvgrey}} ^^A primitives
    ,texcsstyle=*[2]{\bfseries\color{expkvred}}  ^^A expkv
    ,texcsstyle=*[3]{\color{expkvred}}           ^^A expkv-pkg
    ,texcsstyle=*[4]{\bfseries\color{expkvgrey}} ^^A e-TeX
    ,texcsstyle=*[5]{\bfseries\color{expkvgrey}} ^^A plain/LaTeX macros
    ,texcsstyle=*[6]{}                           ^^A macros of other packages
    ,commentstyle=\color[gray]{0.4}
    ,literate=
              {\{} {{\CodeUpBf\{}}{1}
              {\}} {{\CodeUpBf\}}}{1}
              {$}  {{\CodeUpBf\$}}{1}
              {[}  {{\CodeUp[}}{1}
              {]}  {{\CodeUp]}}{1}
              {(}  {{\CodeUp(}}{1}
              {)}  {{\CodeUp)}}{1}
              {*}  {{$*$}}{1}
              {1}  {{\CodeUp{1}}}{1}
              {2}  {{\CodeUp{2}}}{1}
              {3}  {{\CodeUp{3}}}{1}
              {4}  {{\CodeUp{4}}}{1}
              {5}  {{\CodeUp{5}}}{1}
              {6}  {{\CodeUp{6}}}{1}
              {7}  {{\CodeUp{7}}}{1}
              {8}  {{\CodeUp{8}}}{1}
              {9}  {{\CodeUp{9}}}{1}
              {0}  {{\CodeUp{0}}}{1}
              {##} {{\CodeColored{codeparam}{1}{\#}}}{1}
              {##1}{{\CodeColored{codeparam}{2}{\#1}}}{2}
              {##2}{{\CodeColored{codeparam}{2}{\#2}}}{2}
              {##3}{{\CodeColored{codeparam}{2}{\#3}}}{2}
              {##4}{{\CodeColored{codeparam}{2}{\#4}}}{2}
              {##5}{{\CodeColored{codeparam}{2}{\#5}}}{2}
              {##6}{{\CodeColored{codeparam}{2}{\#6}}}{2}
              {##7}{{\CodeColored{codeparam}{2}{\#7}}}{2}
              {##8}{{\CodeColored{codeparam}{2}{\#8}}}{2}
              {##9}{{\CodeColored{codeparam}{2}{\#9}}}{2}
              {<key>}{{$\langle$}key{$\rangle$}}{5}
              {<set>}{{$\langle$}set{$\rangle$}}{5}
              {<set1>}{{$\langle$}set1{$\rangle$}}{6}
              {<set2>}{{$\langle$}set2{$\rangle$}}{6}
    ^^A,literate=*{<key>}{\key}{4}{<set>}{\set}{4}
  }
\newcommand*\CodeColored[3]{\textcolor{#1}{\makebox[\dimexpr\mylstwd*#2]{#3}}}
\newcommand*\CodeUpBf[1]{\makebox[\mylstwd]{\textup{\textbf{#1}}}}
\newcommand*\CodeUp[1]{\makebox[\mylstwd]{\textup{#1}}}
\RequirePackage{randtext}
\let\metaORIG\meta
\protected\def\meta #1{\texttt{\metaORIG{#1}}}
\renewcommand*\thefootnote{\fnsymbol{footnote}}
\definecolor{expkvred}{HTML}{9F393D}
\colorlet{expkvgrey}{black!75}
\makeatletter
\newcommand*\example{\par\smallskip\noindent\textit{Example:} \ignorespaces}
\newcommand*\expkv
  {^^A
    \texorpdfstring
      {^^A
        \mbox
          {^^A
            \BeginAccSupp{ActualText=expkv}^^A
            \href{https://github.com/Skillmon/tex_expkv}
              {^^A
                \rmfamily
                \bfseries
                {\color{expkvgrey}e\kern-.05em x\kern-.05em}^^A
                \lower.493ex
                  \hbox{{\color{expkvgrey}P}\kern-.1em{\color{expkvred}k}}^^A
                \kern-.18em{\color{expkvred}v}^^A
              }^^A
            \EndAccSupp{}^^A
          }^^A
      }
      {expkv}^^A
  }
\newcommand*\expkvpkg[1]
  {^^A
    \texorpdfstring
      {^^A
        \mbox
          {^^A
            \BeginAccSupp{ActualText=expkv-#1}^^A
            \href{https://github.com/Skillmon/tex_expkv-#1}
              {^^A
                \rmfamily
                \bfseries
                {\color{expkvgrey}e\kern-.05em x\kern-.05em}^^A
                \lower.493ex
                  \hbox{{\color{expkvgrey}P}\kern-.1em{\color{expkvred}k}}^^A
                \kern-.18em{\color{expkvred}v}^^A
                {\color{expkvgrey}^^A
                  \kern.05em\rule[-.1ex]{.08em}{1.2ex}\kern.05em\textsc{#1}^^A
                }^^A
              }^^A
            \EndAccSupp{}^^A
          }^^A
      }
      {expkv-#1}^^A
  }
\newcommand*\expkvd{\expkvpkg{def}}
\newcommand*\expkvc{\expkvpkg{cs}}
\newcommand\kv{\meta{key}=\meta{value}}
\newcommand\key{\meta{key}}
\newcommand\val{\meta{value}}
\newcommand\set{\meta{set}}
\newcommand\enflong{\textcolor{black}{long}}
\newcommand\alllong{\textcolor{gray}{long}}
\newcommand\notlong{\textcolor{red!80!black}{long}}
\newcommand\prefixes[1]
  {^^A
    \hfill
    \ifcase\numexpr#1\relax\or
      \enflong\or
      \alllong\or
      \notlong\fi
  }
\hypersetup{linkcolor=red!80!black,urlcolor=purple!80!black}
\DoNotIndex{\def,\edef,\,,\=,\begingroup,\catcode,\chardef,\csname,\endcsname}
\DoNotIndex{\endgroup,\endinput,\errmessage,\expandafter,\input,\let,\long,\z@}
\DoNotIndex{\protected,\ProvidesFile,\ProvidesPackage,\relax,\space,\advance}
\DoNotIndex{\@,\unexpanded,\string,\expanded,\dimexpr,\global,\glueexpr,\hbox}
\DoNotIndex{\numexpr,\RequirePackage,\setbox,\the,\unless,\xdef,\gdef,\newcount}
\DoNotIndex{\number}
\DoNotIndex{\^,\@firstofone,\@firstoftwo,\@gobble,\@secondoftwo}
\DoNotIndex{\ifcsname}
\DoNotIndex{\ifx}
\DoNotIndex{\ifodd}
\DoNotIndex{\ifnum}
\DoNotIndex{\ifdefined}
\DoNotIndex{\iffalse}
\DoNotIndex{\iftrue}
\DoNotIndex{\else}
\DoNotIndex{\fi}
\@gobble\fi ^^A ignoring \ifx, \ifcsname, etc., but only one \fi
\@gobble\fi ^^A ignoring \ifx, \ifcsname, etc., but only one \fi
\@gobble\fi ^^A ignoring \ifx, \ifcsname, etc., but only one \fi
\@gobble\fi ^^A ignoring \ifx, \ifcsname, etc., but only one \fi
\@gobble\fi ^^A ignoring \ifx, \ifcsname, etc., but only one \fi
\@gobble\fi ^^A ignoring \ifx, \ifcsname, etc., but only one \fi
\@ifdefinable\gobbledocstriptag{\def\gobbledocstriptag#1>{}}
\makeatother
\begin{document}
  \title
    {^^A
      \texorpdfstring
        {^^A
          \huge\expkvc
          \\[\medskipamount]
          \Large define expandable \kv\ macros using \expkv
        }{expkv-cs - define expandable key=val macros using expkv}^^A
    }
  \date{\ekvcDate\space v\ekvcVersion}
  \author{Jonathan P. Spratte\thanks{\protect\randomize{jspratte@yahoo.de}}}
  \DocInput{expkv-cs.dtx}
\end{document}
%</driver>^^A=<<
% \fi
%
% \maketitle
% \renewcommand*\thefootnote{\arabic{footnote}}
%
% \begin{abstract}
% \noindent\parfillskip=0pt
% \expkvc\ provides two small interfaces to define expandable \kv\ macros using
% \expkv. It therefore lowers the entrance boundary to expandable \kv\ macros.
% The stylised name is \expkvc\ but the files use \file{expkv-cs}, this is due
% to CTAN-rules which don't allow \string| in package names since that is the
% pipe symbol in *nix shells.
% \end{abstract}
%
% \tableofcontents
%
% \begin{documentation}^^A>>=
%
% \section{Documentation}
%
% The \expkv\ package enables the new possibility of creating \kv\ macros which
% are fully expandable. The creation of such macros is however cumbersome for
% the average user. \expkvc\ tries to step in here. It provides interfaces to
% define \kv\ macros without worrying too much about the implementation. In case
% you're wondering now, the |cs| in \expkvc\ stands for control sequence,
% because |def| was already taken by \expkvd\ and ``control sequence'' is the
% term D.\,E.\,Knuth used in his \TeX book for named commands hence macros
% (though he also used the term ``macro''). So \expkvc\ defines control
% sequences for and with \expkv.
%
% There are two different approaches supported by this package. The first is
% splitting the keys up into individual arguments, the second is providing all
% the keys as a single argument to the underlying macro and getting an
% individual \val\ by using a hash. Well, actually there is no real hash, just
% some markers which are parsed, but this shouldn't be apparent to the user, the
% behaviour matches that of a hash-table.
%
% In addition to these two methods of defining a macro with primary keys a way
% to define secondary keys, which can reference the primary ones, is provided.
% These secondary keys don't correspond to an argument or an entry in the hash
% table directly but might come in handy for the average use case. Each macro
% has its own set of primary and secondary keys.
%
% A word of advice you should consider: If your macro doesn't have to be
% expandable (and often it doesn't) consider not using \expkvc. The interface
% has some overhead (though it still is fast -- check \autoref{sec:speed}) and
% the approach has its limits in versatility. If you don't need to be
% expandable, you should consider either defining your keys manually using
% \expkv\ or using \expkvd\ for convenience. Or you resort to another \kv\
% interface. Nevertheless setting up macros with \expkvc, especially with
% |\ekvcSplit|, is very convenient in my opinion, so if you just want to define
% a single macro with just a few keys this might be the way to go.
%
% \expkvc\ is usable as generic code and as a \LaTeX\ package. It'll
% automatically load \expkv\ in the same mode as well. To use it, just use one
% of
% \begin{lstlisting}
% \usepackage{expkv-cs} % LaTeX
% \input expkv-cs       % plainTeX
% \end{lstlisting}
%
%
% \subsection{Define Macros and Primary Keys}
%
% All macros defined with \expkvc\ have to be previously undefined or have the
% |\meaning| of |\relax|. This is necessary as there is no way to automatically
% undefine keys once they are set up (neither \expkv\ nor \expkvc\ keep track of
% defined keys) -- so to make sure there are no conflicts only new definitions
% are allowed (that's not the case for individual keys, only for frontend
% macros).
%
%
% \subsubsection{Primary Keys}\label{sec:primaries}
%
% In the following descriptions there will be one argument named \meta{primary
% keys}. This argument should be a \kv\ list where each \key\ will be one
% primary key and \val\ the associated initial value. By default all keys are
% defined short, but you can define long keys by prefixing \key\ with |long|
% (\emph{e.g.}, \texttt{long name=Jonathan P. Spratte}). You only need |long| if
% the key should be able to take a |\par| token. Note however that |long| keys
% are a microscopic grain faster (due to some internals of \expkvc). Only if at
% least one of the keys was |long| the \meta{cs} in the following defining
% macros will be |\long|. For obvious reasons there is no possibility to define
% a macro or key as |\protected|.
%
% At the moment \expkvc\ doesn't require any internal keys, but I can't foresee
% whether this will be the case in the future as well, as it might turn out that
% some features I deem useful can't be implemented without such internal keys.
% Because of this, please don't use key names starting with
% \texttt{EKVC\string|} as that should be the private name space.
%
%
% \subsubsection{Split}
%
% The split variants will provide the key values as separate arguments. This
% limits the number of keys for which this is truly useful.
%
% \begin{function}{\ekvcSplit}
%   \begin{syntax}
%     \cs{ekvcSplit}\meta{cs}\marg{primary keys}\marg{definition}
%   \end{syntax}
%   This defines \meta{cs} to be a macro taking one mandatory argument which
%   should contain a \kv\ list. The \meta{primary keys} will be defined for this
%   macro (see \autoref{sec:primaries}). The \meta{definition} is the code that
%   will be executed. You can access the \val\ of a \key\ by using a macro
%   parameter from |#1| to |#9|. The order of the macro parameters will be the
%   order provided in the \meta{primary keys} list (so |#1| is the \val\ of the
%   key defined first). With |\ekvcSplit| you can define macros using at most
%   nine primary keys.
% \end{function}
% \example The following defines a macro |\foo| that takes the keys |a| and |b|
% and outputs their values in a textual form:\\
% \begin{minipage}{.75\linewidth}
% \begin{lstlisting}
% \ekvcSplit\foo{a=a,b=b}{a is #1.\par b is #2.\par}
% \foo{}
% \foo{b=e}
% \end{lstlisting}
% \end{minipage}^^A
% \begin{exresult}[nobeforeafter,box align=center]{.25\linewidth}
%   \ekvcSplit\foo{a=a,b=b}{a is #1.\par b is #2.\par}
%   \foo{}
%   \foo{b=e}
% \end{exresult}
%
% \begin{function}{\ekvcSplitAndForward}
%   \begin{syntax}
%     \cs{ekvcSplitAndForward}\meta{cs}\marg{after}\marg{primary keys}
%   \end{syntax}
%   This defines \meta{cs} to be a macro taking one mandatory argument which
%   should contain a \kv\ list. You can use as many primary keys as you want
%   with this. The primary keys will be forwarded to \meta{after} as braced
%   arguments (as many as necessary for your primary keys). The order of the
%   braced arguments will be the order of your primary key definitions. In
%   \meta{after} you can use just a single control sequence, or some arbitrary
%   stuff which will be left in the input stream before your braced values (with
%   one set of braces stripped from \meta{after}), so both of the following
%   would be fine:
% \begin{lstlisting}
% \ekvcSplitAndForward\foo\foo@aux{keyA = A, keyB = B}
% \ekvcSplitAndForward\foo{\foo@aux{more args}}{keyA = A, keyB = B}
% \end{lstlisting}
% \end{function}
% In the first case |\foo@aux| should take at least two arguments (|keyA| and
% |keyB|), in the second case at least three (|more args|, |keyA|, and |keyB|).
%
% \begin{function}{\ekvcSplitAndUse}
%   \begin{syntax}
%     \cs{ekvcSplitAndUse}\meta{cs}\marg{primary keys}
%   \end{syntax}
%   This will roughly do the same as |\ekvcSplitAndForward|, but instead of
%   specifying what will be used after splitting the keys, \meta{cs} will use
%   what follows the \kv\ list. So its syntax will be
% \end{function}
%   \begin{syntax}
%     \meta{cs}\{\key=\val, \ldots\}\marg{after}
%   \end{syntax}
% and the code in |after| should expect at least as many arguments as the number
% of keys defined for \meta{cs}.
%
% \subsubsection{Hash}
%
% The hash variants will provide the key values as a single argument in which
% you can access specific values using a special macro. The implementation might
% be more convenient and scale better, \emph{but} it is much slower (for a
% primitive macro with a single key benchmarking was almost $1.7$ times slower,
% the root of which being the key access with |\ekvcValue|, not the parsing, and
% for a key access using |\ekvcValueFast| it was still about $1.2$ times
% slower). So if your macro uses less than ten primary keys, you should consider
% using the split approach.
%
% \begin{function}{\ekvcHash}
%   \begin{syntax}
%     \cs{ekvcHash}\meta{cs}\marg{primary keys}\marg{definition}
%   \end{syntax}
%   This defines \meta{cs} to be a macro taking one mandatory argument which
%   should contain a \kv\ list. You can use as many primary keys as you want.
%   The primary keys will be forwarded as a single argument containing every key
%   to the underlying macro. The underlying macro is defined as
%   \meta{definition}, in which you can access the \val\ of a \key\ by using
%   \texttt{\cs[no-index]{ekvcValue}\{\key\}\{\#1\}}.
% \end{function}
% \example This defines an equivalent macro like the |\foo| defined with
% |\ekvcSplit| earlier:\\
% \begin{minipage}{.75\linewidth}
% \begin{lstlisting}
% \ekvcHash\foo{a=a,b=b}{a is \ekvcValue{a}{#1}.\par
%                        b is \ekvcValue{b}{#1}.\par}
% \foo{}
% \foo{b=e}
% \end{lstlisting}
% \end{minipage}^^A
% \begin{exresult}[nobeforeafter,box align=center]{.25\linewidth}
%   \ekvcHash\foo{a=a,b=b}{a is \ekvcValue{a}{#1}.\par
%                          b is \ekvcValue{b}{#1}.\par}
%   \foo{}
%   \foo{b=e}
% \end{exresult}
%
% \begin{function}{\ekvcHashAndForward}
%   \begin{syntax}
%     \cs{ekvcHashAndForward}\meta{cs}\marg{after}\marg{primary keys}
%   \end{syntax}
%   This defines \meta{cs} to be a macro taking one mandatory argument which
%   should contain a \kv\ list. You can use as many primary keys as you want.
%   The primary keys will be forwarded as a single argument containing every key
%   to \meta{after}. You can use a single macro for \meta{after} or use some
%   arbitrary stuff, which will be left in the input stream before the hashed
%   \kv\ list with one set of braces stripped. In the
%   macro called in \meta{after} you can access the \val\ of a \key\ by using
%   \texttt{\cs[no-index]{ekvcValue}\{\key\}\{\#1\}} (or whichever argument the
%   hashed \kv\ list will be).
% \end{function}
%
% \begin{function}{\ekvcHashAndUse}
%   \begin{syntax}
%     \cs{ekvcHashAndUse}\meta{cs}\marg{primary keys}
%   \end{syntax}
%   This will roughly do the same as |\ekvcHashAndForward|, but instead of
%   specifying what will be used after hashing the keys, \meta{cs} will use what
%   follows the \kv\ list. So its syntax will be
% \end{function}
%   \begin{syntax}
%     \meta{cs}\{\key=\val, \ldots\}\marg{after}
%   \end{syntax}
%
% \begin{function}{\ekvcValue}
%   \begin{syntax}
%     \cs{ekvcValue}\{\key\}\marg{key list}
%   \end{syntax}
%   This is a safe way to access your keys in a hash variant. \key\
%   is the key which's \val\ you want to use out of the \meta{key list}.
%   \meta{key list} should be the key list argument forwarded to your underlying
%   macro by |\ekvcHash|, |\ekvcHashAndForward|, or |\ekvcHashAndUse|. It will
%   be tested whether the hash function to access that \key\ exists, the \key\
%   argument is not empty, and that the \meta{key list} really contains a \val\
%   of that \key. This macro needs exactly two steps of expansion and if used
%   inside of an |\edef| or |\expanded| context will protect the \val\ from
%   further expanding.
% \end{function}
%
% \begin{function}{\ekvcValueFast}
%   \begin{syntax}
%     \cs{ekvcValueFast}\{\key\}\marg{key list}
%   \end{syntax}
%   This behaves just like |\ekvcValue|, but \emph{without any} safety tests. As
%   a result this is about $1.4$ times faster \emph{but} will throw low level
%   \TeX\ errors eventually if the hash function isn't defined or the \key\
%   isn't part of the \meta{key list} (\emph{e.g.}, because it was defined as a
%   key for another macro -- all macros share the same hash function per \key).
%   Use it if you know what you're doing. This macro needs exactly three steps
%   of expansion in the no-errors case.
% \end{function}
%
% \begin{function}{\ekvcValueSplit}
%   \begin{syntax}
%     \cs{ekvcValueSplit}\{\key\}\marg{key list}\marg{next}
%   \end{syntax}
%   If you need a specific \key\ from a \meta{key list} more than once, it'll
%   be a good idea to only extract it once and from then on keep it as a
%   separate argument. Hence the macro \cs{ekvcValueSplit} will extract one
%   specific \key's value from the list and forward the remainder of the list
%   as the first and the \key's value as the second argument to \meta{next}, so
%   the result of this will be \meta{next}\marg{key list'}\marg{value} with
%   \meta{key list'} the remaining list. This is almost as fast as |\ekvcValue|
%   and runs the same tests. Keep in mind that you can't fetch for the same
%   \key\ again from \meta{key list'} as it got removed.
% \end{function}
% \example The following defines a macro |\foo| which will take three keys.
% Since the next parsing step will need the value of one of the keys multiple
% times we split that key off the list:\\
% \begin{minipage}{.75\linewidth}
% \begin{lstlisting}
% \ekvcHash\foo{a=a,b=b,c=c}
%   {\ekvcValueSplit{a}{#1}\foobar}
% \newcommand*\foobar[2]{a is #2.\par
%                        b is \ekvcValue{b}{#1}.\par
%                        c is \ekvcValue{c}{#1}.\par}
% \foo{}
% \end{lstlisting}
% \end{minipage}^^A
% \begin{exresult}[nobeforeafter,box align=center]{.25\linewidth}
%   \ekvcHash\foo{a=a,b=b,c=c}
%     {\ekvcValueSplit{a}{#1}\foobar}
%   \newcommand*\foobar[2]{a is #2.\par
%                          b is \ekvcValue{b}{#1}.\par
%                          c is \ekvcValue{c}{#1}.\par}
%   \foo{}
% \end{exresult}
%
% \begin{function}{\ekvcValueSplitFast}
%   \begin{syntax}
%     \cs{ekvcValueSplitFast}\{\key\}\marg{key list}\marg{next}
%   \end{syntax}
%   This behaves just like |\ekvcValueSplit|, but it won't run the same tests,
%   hence it is faster but more error prone, just like the relation between
%   |\ekvcValue| and |\ekvcValueFast|.
% \end{function}
%
%
% \subsection{Secondary Keys}\label{sec:secondaries}
%
% To remove some of the limitations with the approach that each primary key
% matches an argument or hash entry, you can define secondary keys. Those have
% to be defined for each macro but it doesn't matter whether that macro was a
% split or a hash variant. If a secondary key references another key it doesn't
% matter whether that other key is primary or secondary.
%
% Secondary keys can have a prefix (like |long|) which are called |p|-type
% prefix and must have a type (like |meta|) which are called |t|-type prefix.
% Some types might require some |p|-prefixes, while others might forbid those.
%
% Please keep in mind that key names shouldn't start with \texttt{EKVC\string|}.
%
% \begin{function}{\ekvcSecondaryKeys}
%   \begin{syntax}
%     \cs{ekvcSecondaryKeys}\meta{cs}\{\kv, \ldots\}
%   \end{syntax}
%   This is the front facing macro to define secondary keys. For the macro
%   \meta{cs} define \key\ to have definition \val. The general syntax for \key\
%   should be
%   \begin{quote}\ttfamily\small
%     \meta{prefix} \meta{name}
%   \end{quote}
%   Where \meta{prefix} is a space separated list of optional |p|-type prefixes
%   followed by one |t|-type prefix. The syntax of \val\ is dependent on the
%   used |t|-prefix.
% \end{function}
%
%
% \subsubsection{\texttt{p}-type Prefixes}
%
% There is only one |p|-prefix available, which is |long|.
%
% \begin{function}{long}
%   The following key will be defined |\long|.
% \end{function}
%
%
% \subsubsection{\texttt{t}-type Prefixes}
%
% If you're familiar with \expkvd\ you'll notice that the |t|-type prefixes
% provided here are much fewer. The expansion only concept doesn't allow for
% great variety in the auto-defined keys.
%
% The syntax examples of the |t|-prefixes will show which |p|-prefix will be
% automatically used by printing those black (\texttt{\enflong}), which
% will be available in grey (\texttt{\alllong}), and which will be disallowed in
% red (\texttt{\notlong}). This will be put flush right next to the syntax line.
%
% \begin{function}{meta}
%   \begin{syntax}
%     meta \key\ = \{\kv, \ldots\} \prefixes2
%   \end{syntax}
%   With a |meta| key you can set other keys. Whenever \key\ is used the keys in
%   the \kv\ list will be set to the values given there. You can use the \val\
%   given to \key\ by using |#1| in the \kv\ list. The keys in the \kv\ list can
%   be primary and secondary ones.
% \end{function}
%
% \begin{function}{nmeta}
%   \begin{syntax}
%     nmeta \key\ = \{\kv, \ldots\} \prefixes3
%   \end{syntax}
%   An |nmeta| key is like a |meta| key, but it doesn't take a value, so the
%   \kv\ list is static.
% \end{function}
%
% \begin{function}{alias}
%   \begin{syntax}
%     alias \key\ = \meta{key_2} \prefixes3
%   \end{syntax}
%   This assigns the definition of \meta{key_2} to \key. As a result \key\ is an
%   alias for \meta{key_2} behaving just the same. Both the value taking and the
%   |NoVal| version (that's \expkv\ slang for a key not accepting a value) will
%   be copied if they are defined when |alias| is used. Of course, \meta{key_2}
%   has to be defined, be it as a primary or secondary one.
% \end{function}
%
% \begin{function}{default}
%   \begin{syntax}
%     default \key\ = \marg{default} \prefixes3
%   \end{syntax}
%   If \key\ is a defined value taking key, you can define a |NoVal| version
%   with this that will behave as if \key\ was given \meta{default} as its
%   \val. Note that this doesn't change the initial values of primary keys set
%   at definition time in |\ekvcSplit| and friends. \key\ can be a primary or
%   secondary key.
% \end{function}
%
% \begin{function}{flag-bool}
%   \begin{syntax}
%     flag-bool \key\ = \meta{cs} \prefixes3
%   \end{syntax}
%   This is a secondary key that doesn't directly involve any of the primary or
%   secondary keys.
%   This defines \key\ to take a value, which should be either |true| or
%   |false|, and set the flag called \meta{cs} accordingly as a boolean.
%   If \meta{cs} isn't defined yet it will be initialised as a flag.
%   Please also read \autoref{sec:flags}.
% \end{function}
%
% \begin{function}{flag-true,flag-false}
%   \begin{syntax}
%     flag-true \key\ = \meta{cs} \prefixes3
%   \end{syntax}
%   This is a secondary key that doesn't directly involve any of the primary or
%   secondary keys.
%   This defines \key\ to take no value and set the flag called \meta{cs} to
%   |true| or |false|, respectively.
%   If \meta{cs} isn't defined yet it will be initialised as a flag.
%   Please also read \autoref{sec:flags}.
% \end{function}
%
% \begin{function}{flag-raise}
%   \begin{syntax}
%     flag-raise \key\ = \meta{cs} \prefixes3
%   \end{syntax}
%   This is a secondary key that doesn't directly involve any of the primary or
%   secondary keys.
%   This defines \key\ to take no value and raise the flag called \meta{cs}.
%   If \meta{cs} isn't defined yet it will be initialised as a flag.
%   Please also read \autoref{sec:flags}.
% \end{function}
%
%
% \subsection{Changing the Initial Values}\label{sec:change}
%
% \begin{function}{\ekvcChange}
%   \begin{syntax}
%     \cs{ekvcChange}\meta{cs}\{\kv,\ldots\}
%   \end{syntax}
%   This processes the \kv\ list (just like it would be processed by \meta{cs})
%   to set new defaults for the macro \meta{cs} defined with \expkvc\ (meaning
%   the values used if you don't provide anything at use time). Inside the \kv\
%   list both primary and secondary keys can be used. If \meta{cs} was defined
%   |\long| earlier it will still be |\long|, every other \TeX\ prefix will be
%   stripped (but \expkvc\ doesn't support them anywhere else so that should be
%   fine). The resulting new defaults will be stored inside the \meta{cs}
%   locally (just as the original defaults were).
% \end{function}
%
% Consider the following example:\\
% \begin{minipage}[c]{.75\linewidth}
% \begin{lstlisting}
% \ekvcSplit\foo{a=a,b=b}{a is #1.\par b is #2.\par}
% \foo{} % initial defaults
% \ekvcChange\foo{b=B}
% \foo{} % new defaults
% \ekvcSecondaryKeys\foo{meta c={a={#1},b={#1}}}
% \ekvcChange\foo{c=c}
% \foo{} % newer defaults
% \end{lstlisting}
% \end{minipage}^^A
% \begin{exresult}[nobeforeafter,box align=center]{.25\linewidth}
%   \ekvcSplit\foo{a=a,b=b}{a is #1.\par b is #2.\par}
%   \foo{}
%   \ekvcChange\foo{b=B}
%   \foo{}
%   \ekvcSecondaryKeys\foo{meta c={a={#1},b={#1}}}
%   \ekvcChange\foo{c=c}
%   \foo{}
% \end{exresult}
%
% As a result with this the typical setup macro could be implemented:
% \begin{lstlisting}
% \ekvcHashAndUse\foo{key=a,key=b}
% \newcommand\foosetup{\ekvcChange\foo}
% \end{lstlisting}
% Of course the usage is limited to a single macro |\foo|, hence this might be
% not as powerful as similar macros used with other \kv\ interfaces. But at
% least a few similar macros could be grouped using the same key parsing macro
% internally.
%
%
% \subsection{Flags}\label{sec:flags}
%
% The idea of flags is taken from \pkg{expl3}. They provide a way to store
% numerical information expandably, however only incrementing and accessing
% works expandably, decrementing is unexpandable. A flag has a height, which
% is a numerical value, and which can be raised by $1$. Flags come at a high
% computational cost (accessing them is slow and they require more memory than
% normal \TeX\ data types like registers, both getting linearly worse with the
% height), so don't use them if not necessary.
%
% The state of flags is always changed locally to the current group, but not to
% the current macro, so if you're using one of the |t|-types involving flags
% bear in mind that they can affect other macros using the same flags at the
% current group level!
%
% \expkvc\ provides some macros to access, alter, and use flags. Flags of
% \expkvc\ don't share a name space with the flags of \pkg{expl3}.
%
% \begin{function}{\ekvcFlagNew}
%   \begin{syntax}
%     \cs{ekvcFlagNew}\meta{flag}
%   \end{syntax}
%   This initialises the macro \meta{flag} as a new flag. It isn't checked
%   whether the macro \meta{flag} is currently undefined.  A \meta{flag} will
%   expand to the flag's current height with a trailing space (so you can use it
%   directly with |\ifnum| for example and it will terminate the number scanning
%   itself).
% \end{function}
%
%  All other macros dealing with flags take as a parameter a macro defined as a
%  \meta{flag} with |\ekvcFlagNew|.
%
% \begin{function}{\ekvcFlagHeight}
%   \begin{syntax}
%     \cs{ekvcFlagHeight}\meta{flag}
%   \end{syntax}
%   This expands to the current height of \meta{flag} in a single step of
%   expansion (without a trailing space).
% \end{function}
%
% \begin{function}{\ekvcFlagRaise}
%   \begin{syntax}
%     \cs{ekvcFlagRaise}\meta{flag}
%   \end{syntax}
%   This expandably raises the height of \meta{flag} by $1$.
% \end{function}
%
% \begin{function}{\ekvcFlagSetTrue,\ekvcFlagSetFalse}
%   \begin{syntax}
%     \cs{ekvcFlagSetTrue}\meta{flag}
%   \end{syntax}
%   By interpreting an even value as false and an odd value as true we can use a
%   flag as a boolean. This expandably sets \meta{flag} to true or false,
%   respectively, by raising it if necessary.
% \end{function}
%
% \begin{function}{\ekvcFlagIf}
%   \begin{syntax}
%     \cs{ekvcFlagIf}\meta{flag}\marg{true}\marg{false}
%   \end{syntax}
%   This interprets a \meta{flag} as a boolean and expands to either \meta{true}
%   or \meta{false}.
% \end{function}
%
% \begin{function}{\ekvcFlagIfRaised}
%   \begin{syntax}
%     \cs{ekvcFlagIfRaised}\meta{flag}\marg{true}\marg{false}
%   \end{syntax}
%   This tests whether the \meta{flag} is raised, meaning it has a height
%   greater than zero, and if so expands to \meta{true} else to \meta{false}.
% \end{function}
%
% \begin{function}{\ekvcFlagReset}
%   \begin{syntax}
%     \cs{ekvcFlagReset}\meta{flag}
%   \end{syntax}
%   This resets a flag (so restores its height to $0$). This operation is
%   \emph{not} expandable and done locally. If you really intend to use flags
%   you can reset them every now and then to keep the performance hit low.
% \end{function}
%
% \begin{function}{\ekvcFlagGetHeight}
%   \begin{syntax}
%     \cs{ekvcFlagGetHeight}\meta{flag}\marg{next}
%   \end{syntax}
%   This retrieves the current height of the \meta{flag} and provides it as a
%   braced argument to \meta{next}, leaving \meta{next}\marg{height} in the
%   input stream.
% \end{function}
%
% \begin{function}{\ekvcFlagGetHeights}
%   \begin{syntax}
%     \cs{ekvcFlagGetHeights}\marg{flag-list}\marg{next}
%   \end{syntax}
%   This retrieves the current height of each \meta{flag} in the
%   \meta{flag-list} and provides them as a single argument to \meta{next}.
%   Inside that argument each height is enclosed in a set of braces
%   individually. The \meta{flag-list} is just a single argument containing the
%   \meta{flag}s.
%   So a usage like
%   \texttt
%     {^^A
%       \cs[no-index]{ekvcFlagGetHeights}\hskip0pt^^A
%       \{\cs[no-index]{myflagA}\cs[no-index]{myflagB}\}\hskip0pt^^A
%       \{\cs[no-index]{stuff}\}^^A
%     }
%   will expand to
%   \texttt{\cs[no-index]{stuff}\{\marg{height-A}\marg{height-B}\}}.
% \end{function}
%
%
% \subsection{Further Examples}
%
% How could a documentation be a good documentation without some basic examples?
% Say we want to define a small macro expanding to some character description
% (who knows why this has to be expandable?). A character description will not
% have too many items to it, so we use |\ekvcSplit| (the comments with the
% parameter numbers are of course not necessary and just ease reading the
% example).
% \begin{lstlisting}
% \ekvcSplit\character
%   {
%     name=John Doe,            % #1
%     age=any,                  % #2
%     nationality=the Universe, % #3
%     hobby=to exist,           % #4
%     type=Mister,              % #5
%     pronoun=He,               % #6
%     possessive=his,           % #7
%   }
%   {#1 is a #5 from #3. #6 is of #2 age and #7 hobby is #4.\par}
% \end{lstlisting}
% Also we want to give some short cuts so that it's easier to describe several
% persons.
% \begin{lstlisting}
% \ekvcSecondaryKeys\character
%   {
%     alias pro = pronoun,
%     alias pos = possessive,
%     nmeta me =
%       {
%         name=Jonathan,
%         age=a young,
%         nationality=Germany,
%         hobby=\TeX\ coding,
%       },
%     meta lady =
%       {type=Lady, pronoun=She, possessive=her, name=Jane Doe, #1},
%     nmeta paulo =
%       {
%         name=Paulo,
%         type=duck,
%         age=a young,
%         nationality=Brazil,
%         hobby=to quack,
%       }
%   }
% \end{lstlisting}
% Now we can describe people using
% \begin{lstlisting}
% \character{}
% \character{me}
% \character{paulo}
% \character
%   {lady={name=Evelyn,nationality=Ireland,age=the best,hobby=reading}}
% \character
%   {
%     name=Our sun, type=star, nationality=our solar system, pro=It,
%     age=an old, pos=its, hobby=shining
%   }
% \end{lstlisting}
% As one might see, the |lady| key could actually have been an |nmeta| key as
% well, as all that is done with the argument is using it as a \kv\ list.
%
% The result of only the first two usages would be:
% \begin{exresult}{\linewidth}
%   \ekvcSplit\character
%     {
%       name=John Doe,
%       age=any,
%       nationality=the Universe,
%       hobby=to exist,
%       type=Mister,
%       pronoun=He,
%       possessive=his,
%     }
%     {#1 is a #5 from #3. #6 is of #2 age and #7 hobby is #4.\par}
%   \ekvcSecondaryKeys\character
%     {
%       alias pro = pronoun,
%       alias pos = possessive,
%       nmeta me =
%         {
%           name=Jonathan,
%           age=a young,
%           nationality=Germany,
%           hobby=\TeX\ coding,
%         },
%       meta lady =
%         {type=Lady, pronoun=She, possessive=her, name=Jane Doe, #1},
%       nmeta paulo =
%         {
%           name=Paulo,
%           type=duck,
%           age=a young,
%           nationality=Brazil,
%           hobby=to quack,
%         }
%     }
%   \character{}
%   \character{me}
% \end{exresult}
%
% Using \pkg{xparse} or \expkv's \cs{ekvoptarg} or \cs{ekvoptargTF} and
% forwarding arguments one can easily define \kv\ macros with actual optional
% and mandatory arguments as well. A small nonsense example (which should
% perhaps use |\ekvcSplitAndForward| instead of |\ekvcHashAndForward| since it
% only uses four keys and one other argument -- and isn't expandable since it
% uses a \env{tabular} environment, so it would've been better to use a more
% feature rich \kv\ interface most likely, \emph{e.g.}, the one provided by
% \expkvd):
% \begin{lstlisting}
% \makeatletter
% \newcommand*\nonsense{\ekvoptarg\nonsense@a{}}
% \ekvcHashAndForward\nonsense@a\nonsense@b
%   {
%     keyA = A,
%     keyB = B,
%     keyC = c,
%     keyD = d,
%   }
% \newcommand*\nonsense@b[2]
%   {%
%     \begin{tabular}{lll}
%       key & A & \ekvcValue{keyA}{#1} \\
%           & B & \ekvcValue{keyB}{#1} \\
%           & C & \ekvcValue{keyC}{#1} \\
%           & D & \ekvcValue{keyD}{#1} \\
%       \multicolumn{2}{l}{mandatory} & #2 \\
%     \end{tabular}%
%   }
% \makeatother
% \end{lstlisting}
% And then we would be able to do some nonsense
% \begin{lstlisting}
% \nonsense{}
% \nonsense[keyA=hihi]{haha}
% \nonsense[keyA=hihi, keyB=A]{hehe}
% \nonsense[keyC=huhu, keyA=hihi, keyB=A]{haha}
% \end{lstlisting}
% resulting in
% \begin{exresult}{\linewidth}
%   \makeatletter
%   \newcommand*\nonsense{\ekvoptarg\nonsense@a{}}
%   \ekvcHashAndForward\nonsense@a\nonsense@b
%     {
%       keyA = A,
%       keyB = B,
%       keyC = c,
%       keyD = d,
%     }
%   \newcommand*\nonsense@b[2]
%     {%
%       \begin{tabular}{lll}
%         key & A & \ekvcValue{keyA}{#1} \\
%             & B & \ekvcValue{keyB}{#1} \\
%             & C & \ekvcValue{keyC}{#1} \\
%             & D & \ekvcValue{keyD}{#1} \\
%         \multicolumn{2}{l}{mandatory} & #2 \\
%       \end{tabular}%
%     }
%   \makeatother
%   \nonsense{}
%   \nonsense[keyA=hihi]{haha}
%   \nonsense[keyA=hihi, keyB=A]{hehe}
%   \nonsense[keyC=huhu, keyA=hihi, keyB=A]{haha}
% \end{exresult}
%
%
% \subsection{Freedom for Keys!}
%
% If this was the \TeX book this subsection would have a double bend sign. Not
% because it is overly complicated, but because it shows things which could
% break \expkvc's expandability and its alignment safety. This is for
% experienced users wanting to get the most flexibility and knowing what they
% are doing.
%
% In case you're wondering, it is possible to define other keys than the
% primaries and the secondary types listed in \autoref{sec:secondaries} for a
% macro defined with \expkvc\ by using the low-level interface of \expkv\ or
% even the interface provided by \expkvd. The set name used for \expkvc's keys
% is the macro name, including the leading backslash, or more precisely
% |\string|\meta{cs} is used. This can be exploited to define additional keys
% with arbitrary code. Consider the following \emph{bad} example:
% \begin{lstlisting}
% \ekvcSplit\foo{a=A,b=B}{a is #1.\par b is #2}
% \protected\ekvdef{\string\foo}{c}{\def\fooC{#1}}
% \end{lstlisting}
% This would define a key named |c| that will store its value inside a macro.
% The issue with this is that this can't be done expandably. As a result, the
% macro |\foo| isn't always expandable any more (not that bad if this was never
% required; killjoy if it was) and as soon as the key |c| is used, it is also no
% longer alignment safe\footnote{This means that the \kv-list can't contain
%   alignment markers that are not inside an additional set of braces if used
%   inside a \TeX\ alignment.}
% (might be bad depending on the usage).
%
% So why do I show you this? Because we could as well do something useful like
% create a key that pre-parses the input and after that passes the parsed value
% on. This parsing would have to be completely expandable though. For the
% pass-on part we can use the following function:
%
% \begin{function}{\ekvcPass}
%   \begin{syntax}
%     \cs{ekvcPass}\meta{cs}\marg{key}\marg{value}
%   \end{syntax}
%   This passes \meta{value} on to \meta{key} for the \expkvc-macro \meta{cs}.
%   It should be used inside the key parsing of a macro defined with \expkvc,
%   else this most likely results in a low level \TeX\ error.
% \end{function}
%
% With this we could for example split the value of a key at a hyphen and pass
% the parts to different keys:\\
% \begin{minipage}[c]{.75\linewidth}
% \begin{lstlisting}
% \ekvcSplit\foo{a=A,b=B}{a is #1.\par b is #2.\par}
% \ekvdef{\string\foo}{c}{\fooSplit#1\par}
% \def\fooSplit#1-#2\par
%   {\ekvcPass\foo{a}{#1}\ekvcPass\foo{b}{#2}}
% \foo{}
% \foo{c=1-2}
% \end{lstlisting}
% \end{minipage}
% \begin{exresult}[nobeforeafter,box align=center]{.25\linewidth}
%   \ekvcSplit\foo{a=A,b=B}{a is #1.\par b is #2.\par}
%   \ekvdef{\string\foo}{c}{\fooSplit#1\par}
%   \def\fooSplit#1-#2\par{\ekvcPass\foo{a}{#1}\ekvcPass\foo{b}{#2}}
%   \foo{}
%   \foo{c=1-2}
% \end{exresult}
%
%
% \subsection{Speed Considerations}\label{sec:speed}
%
% As already mentioned in the introduction there are some speed considerations
% implied if you choose to define macros via \expkvc. However the overhead isn't
% the factor which should hinder you to use \expkvc\ if you found a reasonable
% use case. The key-parsing is still faster than with most other \kv\ packages
% (see the ``Comparisons'' subsection in the \expkv\ documentation).
%
% The speed considerations in this subsection use the first example in this
% documentation as the benchmark. So we have seven keys and a short sentence
% which should be typeset. For comparisons I use the following equivalent
% \expkvd\ definitions. Each result is the average between changing no keys from
% their initial values and altering four. Furthermore I'll compare three
% variants of \expkvc\ with the \expkvd\ definitions, namely the split example
% from above, a hash variant using |\ekvcValue| and a hash variant using
% |\ekvcValueFast|.
% \begin{lstlisting}
% \usepackage{expkv-def}
% \ekvdefinekeys{keys}
%   {%
%     ,store   name         = \KEYSname
%     ,initial name         = John Doe
%     ,store   age          = \KEYSage
%     ,initial age          = any
%     ,store   nationality  = \KEYSnationality
%     ,initial nationality  = the Universe
%     ,store   hobby        = \KEYShobby
%     ,initial hobby        = to exist
%     ,store   type         = \KEYStype
%     ,initial type         = Mister
%     ,store   pronoun      = \KEYSpronoun
%     ,initial pronoun      = He
%     ,store   possessive   = \KEYSpossessive
%     ,initial possessive   = his
%   }
% \newcommand*\KEYS[1]
%   {%
%     \begingroup
%       \ekvset{keys}{#1}%
%       \KEYSname\ is a \KEYStype\ from \KEYSnationality. \KEYSpronoun\ is
%       of \KEYSage\ age and \KEYSpossessive\ hobby is \KEYShobby.%
%     \endgroup
%   }
% \end{lstlisting}
%
% The first comparison removes the typesetting part from all the definitions, so
% that only the key parsing is compared. In this comparison the |\ekvcValue|
% and |\ekvcValueFast| variants will not differ, as they are exactly the same
% until the key usage. We find that the split approach is $1.4$ times slower
% than the \expkvd\ setup and the hash variants end up in the middle at $1.17$
% times slower.
%
% Next we put the typesetting part back in. Every call of the macros will
% typeset the sentences into a box register in horizontal mode. With the
% typesetting part (which includes the accessing of values) the fastest remains
% the \expkvd\ definitions, but split is close at $1.16$ times slower, followed
% by the hash variant with fast accesses at $1.36$ times slower, and the safe
% hash access variant ranks in the slowest $1.8$ times slower than \expkvd.
%
% Just in case you're wondering now, a simple macro taking seven arguments is
% $30$ to $40$ times faster than any of those in the argument grabbing and \kv\
% parsing part and only $1.5$ to $2.8$ times faster if the typesetting part is
% factored in. So the real choke isn't the parsing.
%
% So to summarize this, if you have a reasonable use case for expandable \kv\
% parsing macros you should go on and define them using \expkvc. If you have a
% reasonable use case for \kv\ parsing macros but defining them expandable isn't
% necessary for your use you should take advantage of the greater flexibility of
% non-expandable \kv\ setups (but if you're after maximum speed there aren't
% that many \kv\ parsers beating \expkvc). And if you are after maximum
% performance maybe ditching the \kv\ interface altogether is a good idea, but
% depending on the number of arguments your interface might get convoluted.
%
%
% \subsection{Useless Macros}
%
% Perhaps these macros aren't completely useless, but I figured from a user's
% point of view I wouldn't know what I should do with these.
%
% \begin{function}{\ekvcDate,\ekvcVersion}
%   These two macros store the version and the date of the package/generic code.
% \end{function}
%
%
% \subsection{Bugs}
%
% Of course I don't think there are any bugs (who would knowingly distribute
% buggy software as long as he isn't a multi-million dollar corporation?). But
% if you find some please let me know. For this one might find my email address
% on the first page or file an issue on Github:
% \url{https://github.com/Skillmon/tex_expkv-cs}
%
%
% \subsection{License}
%
% Copyright \textcopyright\ 2020\unless\ifnum\year=2020--\the\year\fi\
% Jonathan P. Spratte
%
% \medskip\noindent
% This work may be distributed and/or modified under the conditions of the
% \LaTeX\ Project Public License (LPPL), either version 1.3c of this license or
% (at your option) any later version. The latest version of this license is in
% the file:
%
% \url{http://www.latex-project.org/lppl.txt}
%
% \noindent
% This work is ``maintained'' (as per LPPL maintenance status) by
%
% \mbox{Jonathan P. Spratte}.
%
% \end{documentation}^^A=<<
%
% \begin{implementation}^^A>>=
%
% \clearpage
%
% \section{Implementation}^^A>>=
%^^A the LaTeX package >>=
% \subsection{The \LaTeX\ Package}
% Just like for \expkv\ we provide a small \LaTeX\ package that sets up things
% such that we behave nicely on \LaTeX\ packages and files system. It'll
% |\input| the generic code which implements the functionality.
% \gobbledocstriptag
%<*pkg>
%    \begin{macrocode}
\RequirePackage{expkv}
\def\ekvc@tmp
  {%
    \ProvidesFile{expkv-cs.tex}%
      [%
        \ekvcDate\space v\ekvcVersion\space
        define expandable key=val macros using expkv%
      ]%
  }
% \iffalse meta-comment
%
% File: expkv-cs.dtx Copyright (C) 2020-2021 Jonathan P. Spratte
%
% This work  may be  distributed and/or  modified under  the conditions  of the
% LaTeX Project Public License (LPPL),  either version 1.3c  of this license or
% (at your option) any later version.  The latest version of this license is in
% the file:
%
%   http://www.latex-project.org/lppl.txt
%
% ------------------------------------------------------------------------------
%
%<*driver>^^A>>=
\def\nameofplainTeX{plain}
\ifx\fmtname\nameofplainTeX\else
  \expandafter\begingroup
\fi
\input l3docstrip.tex
\askforoverwritefalse
\preamble

--------------------------------------------------------------
expkv-cs -- define expandable key=val macros using expkv
E-mail: jspratte@yahoo.de
Released under the LaTeX Project Public License v1.3c or later
See http://www.latex-project.org/lppl.txt
--------------------------------------------------------------

Copyright (C) 2020-2021 Jonathan P. Spratte

This  work may be  distributed and/or  modified under  the conditions  of the
LaTeX Project Public License (LPPL),  either version 1.3c  of this license or
(at your option) any later version.  The latest version of this license is in
the file:

  http://www.latex-project.org/lppl.txt

This work is "maintained" (as per LPPL maintenance status) by
  Jonathan P. Spratte.

This work consists of the file  expkv-cs.dtx
and the derived files           expkv-cs.pdf
                                expkv-cs.sty
                                expkv-cs.tex

\endpreamble
% stop docstrip adding \endinput
\postamble
\endpostamble
\generate{\file{expkv-cs.sty}{\from{expkv-cs.dtx}{pkg}}}
\generate{\file{expkv-cs.tex}{\from{expkv-cs.dtx}{tex}}}
\ifx\fmtname\nameofplainTeX
  \expandafter\endbatchfile
\else
  \expandafter\endgroup
\fi
%
\IfFileExists{expkv-cs.tex}{\input{expkv-cs.tex}}{}
\ProvidesFile{expkv-cs.dtx}
  [\csname ekvcDate\endcsname\ define expandable key=val macros using expkv]
\PassOptionsToPackage{full}{textcomp}
\documentclass{l3doc}
\RequirePackage[oldstylenums,nott]{kpfonts}
\input{glyphtounicode}
\pdfgentounicode=1
\RequirePackage{tcolorbox}
\newtcolorbox{exresult}[2][]
  {colback=expkvgrey!10!white,colframe=expkvgrey,fontupper=\small,width={#2},#1}
\newtcbox\exres[1][]
  {
     colback=expkvgrey!10!white
    ,colframe=expkvgrey
    ,size=small
    ,nobeforeafter
    ,tcbox raise base
    ,fontupper=\small
    ,#1
  }
\RequirePackage{listings}
\RequirePackage{booktabs}
\RequirePackage{array}
\RequirePackage{collcell}
\RequirePackage{siunitx}
\RequirePackage{xcolor}
\RequirePackage{caption}
\RequirePackage{microtype}
\RequirePackage{accsupp}
\lstdefinelanguage{expkvc}[primitive]{TeX}
  {
    ,moretexcs=[4]^^A e-TeX
      {
        expanded,
        numexpr,
        protected,
      }
    ,moretexcs=[5]^^A plain/LaTeX
      {
        approx,
        begin,
        empty,
        item,
        LaTeX,
        makeatletter,makeatother,multicolumn,
        newcommand,newdimen,
        RequirePackage,
        rule,
        TeX,
        textit,texttt,
        usepackage,
      }
    ,moretexcs=[6]^^A used packages
      {
        ^^A xfp
        fpeval,
        ^^A keyval
        setkeys,
        ^^A yax
        defactiveparameter,storevalue,setparameterlist
      }
    ,moretexcs=[2]^^A expkv-cs macros
      {
        ekvcSplit,ekvcSplitAndForward,ekvcSplitAndUse,
        ekvcHash,ekvcHashAndForward,ekvcHashAndUse,
        ekvcValue,ekvcValueFast,ekvcValueSplit,ekvcValueSplitFast,
        ekvcSecondaryKeys,
        ekvcChange,
        ekvcFlagNew,
        ekvcFlagHeight,
        ekvcFlagRaise,
        ekvcFlagSetTrue,ekvcFlagSetFalse,
        ekvcFlagIf,ekvcFlagIfRaised,
        ekvcFlagReset,
        ekvcFlagGetHeight,ekvcFlagGetHeights,
        ekvcPass
      }
    ,moretexcs=[3]^^A expkv-pkg macros
      {
        ^^A expkv
        ekvdef,ekvdefNoVal,^^A
        ekvlet,ekvletNoVal,ekvletkv,ekvletkvNoVal,^^A
        ekvdefunknown,ekvdefunknownNoVal,^^A
        ekvredirectunknown,ekvredirectunknownNoVal,^^A
        ekvset,ekvsetSneaked,^^A
        ekvsetdef,ekvsetSneakeddef,ekvsetdefSneaked,^^A
        ekvparse,^^A
        ekvVersion,ekvDate,^^A
        ekvifdefined,ekvifdefinedNoVal,ekvifdefinedset,^^A
        ekvbreak,ekvbreakPreSneak,ekvbreakPostSneak,^^A
        ekvsneak,ekvsneakPre,^^A
        ekvchangeset,^^A
        ekvoptarg,ekvoptargTF,^^A
        ekverr,
        ^^A expkv-def
        ekvdefinekeys
      }
  }
\colorlet{codeparam}{cyan!65!black}
\def\mylstwd{.55em}
\lstset
  {
    ,language=expkvc
    ,flexiblecolumns=false
    ,basewidth=\mylstwd
    ,gobble=2
    ,basicstyle=\fontfamily{jkp}\itshape
    ,texcsstyle=*[1]{\bfseries\color{expkvgrey}} ^^A primitives
    ,texcsstyle=*[2]{\bfseries\color{expkvred}}  ^^A expkv
    ,texcsstyle=*[3]{\color{expkvred}}           ^^A expkv-pkg
    ,texcsstyle=*[4]{\bfseries\color{expkvgrey}} ^^A e-TeX
    ,texcsstyle=*[5]{\bfseries\color{expkvgrey}} ^^A plain/LaTeX macros
    ,texcsstyle=*[6]{}                           ^^A macros of other packages
    ,commentstyle=\color[gray]{0.4}
    ,literate=
              {\{} {{\CodeUpBf\{}}{1}
              {\}} {{\CodeUpBf\}}}{1}
              {$}  {{\CodeUpBf\$}}{1}
              {[}  {{\CodeUp[}}{1}
              {]}  {{\CodeUp]}}{1}
              {(}  {{\CodeUp(}}{1}
              {)}  {{\CodeUp)}}{1}
              {*}  {{$*$}}{1}
              {1}  {{\CodeUp{1}}}{1}
              {2}  {{\CodeUp{2}}}{1}
              {3}  {{\CodeUp{3}}}{1}
              {4}  {{\CodeUp{4}}}{1}
              {5}  {{\CodeUp{5}}}{1}
              {6}  {{\CodeUp{6}}}{1}
              {7}  {{\CodeUp{7}}}{1}
              {8}  {{\CodeUp{8}}}{1}
              {9}  {{\CodeUp{9}}}{1}
              {0}  {{\CodeUp{0}}}{1}
              {##} {{\CodeColored{codeparam}{1}{\#}}}{1}
              {##1}{{\CodeColored{codeparam}{2}{\#1}}}{2}
              {##2}{{\CodeColored{codeparam}{2}{\#2}}}{2}
              {##3}{{\CodeColored{codeparam}{2}{\#3}}}{2}
              {##4}{{\CodeColored{codeparam}{2}{\#4}}}{2}
              {##5}{{\CodeColored{codeparam}{2}{\#5}}}{2}
              {##6}{{\CodeColored{codeparam}{2}{\#6}}}{2}
              {##7}{{\CodeColored{codeparam}{2}{\#7}}}{2}
              {##8}{{\CodeColored{codeparam}{2}{\#8}}}{2}
              {##9}{{\CodeColored{codeparam}{2}{\#9}}}{2}
              {<key>}{{$\langle$}key{$\rangle$}}{5}
              {<set>}{{$\langle$}set{$\rangle$}}{5}
              {<set1>}{{$\langle$}set1{$\rangle$}}{6}
              {<set2>}{{$\langle$}set2{$\rangle$}}{6}
    ^^A,literate=*{<key>}{\key}{4}{<set>}{\set}{4}
  }
\newcommand*\CodeColored[3]{\textcolor{#1}{\makebox[\dimexpr\mylstwd*#2]{#3}}}
\newcommand*\CodeUpBf[1]{\makebox[\mylstwd]{\textup{\textbf{#1}}}}
\newcommand*\CodeUp[1]{\makebox[\mylstwd]{\textup{#1}}}
\RequirePackage{randtext}
\let\metaORIG\meta
\protected\def\meta #1{\texttt{\metaORIG{#1}}}
\renewcommand*\thefootnote{\fnsymbol{footnote}}
\definecolor{expkvred}{HTML}{9F393D}
\colorlet{expkvgrey}{black!75}
\makeatletter
\newcommand*\example{\par\smallskip\noindent\textit{Example:} \ignorespaces}
\newcommand*\expkv
  {^^A
    \texorpdfstring
      {^^A
        \mbox
          {^^A
            \BeginAccSupp{ActualText=expkv}^^A
            \href{https://github.com/Skillmon/tex_expkv}
              {^^A
                \rmfamily
                \bfseries
                {\color{expkvgrey}e\kern-.05em x\kern-.05em}^^A
                \lower.493ex
                  \hbox{{\color{expkvgrey}P}\kern-.1em{\color{expkvred}k}}^^A
                \kern-.18em{\color{expkvred}v}^^A
              }^^A
            \EndAccSupp{}^^A
          }^^A
      }
      {expkv}^^A
  }
\newcommand*\expkvpkg[1]
  {^^A
    \texorpdfstring
      {^^A
        \mbox
          {^^A
            \BeginAccSupp{ActualText=expkv-#1}^^A
            \href{https://github.com/Skillmon/tex_expkv-#1}
              {^^A
                \rmfamily
                \bfseries
                {\color{expkvgrey}e\kern-.05em x\kern-.05em}^^A
                \lower.493ex
                  \hbox{{\color{expkvgrey}P}\kern-.1em{\color{expkvred}k}}^^A
                \kern-.18em{\color{expkvred}v}^^A
                {\color{expkvgrey}^^A
                  \kern.05em\rule[-.1ex]{.08em}{1.2ex}\kern.05em\textsc{#1}^^A
                }^^A
              }^^A
            \EndAccSupp{}^^A
          }^^A
      }
      {expkv-#1}^^A
  }
\newcommand*\expkvd{\expkvpkg{def}}
\newcommand*\expkvc{\expkvpkg{cs}}
\newcommand\kv{\meta{key}=\meta{value}}
\newcommand\key{\meta{key}}
\newcommand\val{\meta{value}}
\newcommand\set{\meta{set}}
\newcommand\enflong{\textcolor{black}{long}}
\newcommand\alllong{\textcolor{gray}{long}}
\newcommand\notlong{\textcolor{red!80!black}{long}}
\newcommand\prefixes[1]
  {^^A
    \hfill
    \ifcase\numexpr#1\relax\or
      \enflong\or
      \alllong\or
      \notlong\fi
  }
\hypersetup{linkcolor=red!80!black,urlcolor=purple!80!black}
\DoNotIndex{\def,\edef,\,,\=,\begingroup,\catcode,\chardef,\csname,\endcsname}
\DoNotIndex{\endgroup,\endinput,\errmessage,\expandafter,\input,\let,\long,\z@}
\DoNotIndex{\protected,\ProvidesFile,\ProvidesPackage,\relax,\space,\advance}
\DoNotIndex{\@,\unexpanded,\string,\expanded,\dimexpr,\global,\glueexpr,\hbox}
\DoNotIndex{\numexpr,\RequirePackage,\setbox,\the,\unless,\xdef,\gdef,\newcount}
\DoNotIndex{\number}
\DoNotIndex{\^,\@firstofone,\@firstoftwo,\@gobble,\@secondoftwo}
\DoNotIndex{\ifcsname}
\DoNotIndex{\ifx}
\DoNotIndex{\ifodd}
\DoNotIndex{\ifnum}
\DoNotIndex{\ifdefined}
\DoNotIndex{\iffalse}
\DoNotIndex{\iftrue}
\DoNotIndex{\else}
\DoNotIndex{\fi}
\@gobble\fi ^^A ignoring \ifx, \ifcsname, etc., but only one \fi
\@gobble\fi ^^A ignoring \ifx, \ifcsname, etc., but only one \fi
\@gobble\fi ^^A ignoring \ifx, \ifcsname, etc., but only one \fi
\@gobble\fi ^^A ignoring \ifx, \ifcsname, etc., but only one \fi
\@gobble\fi ^^A ignoring \ifx, \ifcsname, etc., but only one \fi
\@gobble\fi ^^A ignoring \ifx, \ifcsname, etc., but only one \fi
\@ifdefinable\gobbledocstriptag{\def\gobbledocstriptag#1>{}}
\makeatother
\begin{document}
  \title
    {^^A
      \texorpdfstring
        {^^A
          \huge\expkvc
          \\[\medskipamount]
          \Large define expandable \kv\ macros using \expkv
        }{expkv-cs - define expandable key=val macros using expkv}^^A
    }
  \date{\ekvcDate\space v\ekvcVersion}
  \author{Jonathan P. Spratte\thanks{\protect\randomize{jspratte@yahoo.de}}}
  \DocInput{expkv-cs.dtx}
\end{document}
%</driver>^^A=<<
% \fi
%
% \maketitle
% \renewcommand*\thefootnote{\arabic{footnote}}
%
% \begin{abstract}
% \noindent\parfillskip=0pt
% \expkvc\ provides two small interfaces to define expandable \kv\ macros using
% \expkv. It therefore lowers the entrance boundary to expandable \kv\ macros.
% The stylised name is \expkvc\ but the files use \file{expkv-cs}, this is due
% to CTAN-rules which don't allow \string| in package names since that is the
% pipe symbol in *nix shells.
% \end{abstract}
%
% \tableofcontents
%
% \begin{documentation}^^A>>=
%
% \section{Documentation}
%
% The \expkv\ package enables the new possibility of creating \kv\ macros which
% are fully expandable. The creation of such macros is however cumbersome for
% the average user. \expkvc\ tries to step in here. It provides interfaces to
% define \kv\ macros without worrying too much about the implementation. In case
% you're wondering now, the |cs| in \expkvc\ stands for control sequence,
% because |def| was already taken by \expkvd\ and ``control sequence'' is the
% term D.\,E.\,Knuth used in his \TeX book for named commands hence macros
% (though he also used the term ``macro''). So \expkvc\ defines control
% sequences for and with \expkv.
%
% There are two different approaches supported by this package. The first is
% splitting the keys up into individual arguments, the second is providing all
% the keys as a single argument to the underlying macro and getting an
% individual \val\ by using a hash. Well, actually there is no real hash, just
% some markers which are parsed, but this shouldn't be apparent to the user, the
% behaviour matches that of a hash-table.
%
% In addition to these two methods of defining a macro with primary keys a way
% to define secondary keys, which can reference the primary ones, is provided.
% These secondary keys don't correspond to an argument or an entry in the hash
% table directly but might come in handy for the average use case. Each macro
% has its own set of primary and secondary keys.
%
% A word of advice you should consider: If your macro doesn't have to be
% expandable (and often it doesn't) consider not using \expkvc. The interface
% has some overhead (though it still is fast -- check \autoref{sec:speed}) and
% the approach has its limits in versatility. If you don't need to be
% expandable, you should consider either defining your keys manually using
% \expkv\ or using \expkvd\ for convenience. Or you resort to another \kv\
% interface. Nevertheless setting up macros with \expkvc, especially with
% |\ekvcSplit|, is very convenient in my opinion, so if you just want to define
% a single macro with just a few keys this might be the way to go.
%
% \expkvc\ is usable as generic code and as a \LaTeX\ package. It'll
% automatically load \expkv\ in the same mode as well. To use it, just use one
% of
% \begin{lstlisting}
% \usepackage{expkv-cs} % LaTeX
% \input expkv-cs       % plainTeX
% \end{lstlisting}
%
%
% \subsection{Define Macros and Primary Keys}
%
% All macros defined with \expkvc\ have to be previously undefined or have the
% |\meaning| of |\relax|. This is necessary as there is no way to automatically
% undefine keys once they are set up (neither \expkv\ nor \expkvc\ keep track of
% defined keys) -- so to make sure there are no conflicts only new definitions
% are allowed (that's not the case for individual keys, only for frontend
% macros).
%
%
% \subsubsection{Primary Keys}\label{sec:primaries}
%
% In the following descriptions there will be one argument named \meta{primary
% keys}. This argument should be a \kv\ list where each \key\ will be one
% primary key and \val\ the associated initial value. By default all keys are
% defined short, but you can define long keys by prefixing \key\ with |long|
% (\emph{e.g.}, \texttt{long name=Jonathan P. Spratte}). You only need |long| if
% the key should be able to take a |\par| token. Note however that |long| keys
% are a microscopic grain faster (due to some internals of \expkvc). Only if at
% least one of the keys was |long| the \meta{cs} in the following defining
% macros will be |\long|. For obvious reasons there is no possibility to define
% a macro or key as |\protected|.
%
% At the moment \expkvc\ doesn't require any internal keys, but I can't foresee
% whether this will be the case in the future as well, as it might turn out that
% some features I deem useful can't be implemented without such internal keys.
% Because of this, please don't use key names starting with
% \texttt{EKVC\string|} as that should be the private name space.
%
%
% \subsubsection{Split}
%
% The split variants will provide the key values as separate arguments. This
% limits the number of keys for which this is truly useful.
%
% \begin{function}{\ekvcSplit}
%   \begin{syntax}
%     \cs{ekvcSplit}\meta{cs}\marg{primary keys}\marg{definition}
%   \end{syntax}
%   This defines \meta{cs} to be a macro taking one mandatory argument which
%   should contain a \kv\ list. The \meta{primary keys} will be defined for this
%   macro (see \autoref{sec:primaries}). The \meta{definition} is the code that
%   will be executed. You can access the \val\ of a \key\ by using a macro
%   parameter from |#1| to |#9|. The order of the macro parameters will be the
%   order provided in the \meta{primary keys} list (so |#1| is the \val\ of the
%   key defined first). With |\ekvcSplit| you can define macros using at most
%   nine primary keys.
% \end{function}
% \example The following defines a macro |\foo| that takes the keys |a| and |b|
% and outputs their values in a textual form:\\
% \begin{minipage}{.75\linewidth}
% \begin{lstlisting}
% \ekvcSplit\foo{a=a,b=b}{a is #1.\par b is #2.\par}
% \foo{}
% \foo{b=e}
% \end{lstlisting}
% \end{minipage}^^A
% \begin{exresult}[nobeforeafter,box align=center]{.25\linewidth}
%   \ekvcSplit\foo{a=a,b=b}{a is #1.\par b is #2.\par}
%   \foo{}
%   \foo{b=e}
% \end{exresult}
%
% \begin{function}{\ekvcSplitAndForward}
%   \begin{syntax}
%     \cs{ekvcSplitAndForward}\meta{cs}\marg{after}\marg{primary keys}
%   \end{syntax}
%   This defines \meta{cs} to be a macro taking one mandatory argument which
%   should contain a \kv\ list. You can use as many primary keys as you want
%   with this. The primary keys will be forwarded to \meta{after} as braced
%   arguments (as many as necessary for your primary keys). The order of the
%   braced arguments will be the order of your primary key definitions. In
%   \meta{after} you can use just a single control sequence, or some arbitrary
%   stuff which will be left in the input stream before your braced values (with
%   one set of braces stripped from \meta{after}), so both of the following
%   would be fine:
% \begin{lstlisting}
% \ekvcSplitAndForward\foo\foo@aux{keyA = A, keyB = B}
% \ekvcSplitAndForward\foo{\foo@aux{more args}}{keyA = A, keyB = B}
% \end{lstlisting}
% \end{function}
% In the first case |\foo@aux| should take at least two arguments (|keyA| and
% |keyB|), in the second case at least three (|more args|, |keyA|, and |keyB|).
%
% \begin{function}{\ekvcSplitAndUse}
%   \begin{syntax}
%     \cs{ekvcSplitAndUse}\meta{cs}\marg{primary keys}
%   \end{syntax}
%   This will roughly do the same as |\ekvcSplitAndForward|, but instead of
%   specifying what will be used after splitting the keys, \meta{cs} will use
%   what follows the \kv\ list. So its syntax will be
% \end{function}
%   \begin{syntax}
%     \meta{cs}\{\key=\val, \ldots\}\marg{after}
%   \end{syntax}
% and the code in |after| should expect at least as many arguments as the number
% of keys defined for \meta{cs}.
%
% \subsubsection{Hash}
%
% The hash variants will provide the key values as a single argument in which
% you can access specific values using a special macro. The implementation might
% be more convenient and scale better, \emph{but} it is much slower (for a
% primitive macro with a single key benchmarking was almost $1.7$ times slower,
% the root of which being the key access with |\ekvcValue|, not the parsing, and
% for a key access using |\ekvcValueFast| it was still about $1.2$ times
% slower). So if your macro uses less than ten primary keys, you should consider
% using the split approach.
%
% \begin{function}{\ekvcHash}
%   \begin{syntax}
%     \cs{ekvcHash}\meta{cs}\marg{primary keys}\marg{definition}
%   \end{syntax}
%   This defines \meta{cs} to be a macro taking one mandatory argument which
%   should contain a \kv\ list. You can use as many primary keys as you want.
%   The primary keys will be forwarded as a single argument containing every key
%   to the underlying macro. The underlying macro is defined as
%   \meta{definition}, in which you can access the \val\ of a \key\ by using
%   \texttt{\cs[no-index]{ekvcValue}\{\key\}\{\#1\}}.
% \end{function}
% \example This defines an equivalent macro like the |\foo| defined with
% |\ekvcSplit| earlier:\\
% \begin{minipage}{.75\linewidth}
% \begin{lstlisting}
% \ekvcHash\foo{a=a,b=b}{a is \ekvcValue{a}{#1}.\par
%                        b is \ekvcValue{b}{#1}.\par}
% \foo{}
% \foo{b=e}
% \end{lstlisting}
% \end{minipage}^^A
% \begin{exresult}[nobeforeafter,box align=center]{.25\linewidth}
%   \ekvcHash\foo{a=a,b=b}{a is \ekvcValue{a}{#1}.\par
%                          b is \ekvcValue{b}{#1}.\par}
%   \foo{}
%   \foo{b=e}
% \end{exresult}
%
% \begin{function}{\ekvcHashAndForward}
%   \begin{syntax}
%     \cs{ekvcHashAndForward}\meta{cs}\marg{after}\marg{primary keys}
%   \end{syntax}
%   This defines \meta{cs} to be a macro taking one mandatory argument which
%   should contain a \kv\ list. You can use as many primary keys as you want.
%   The primary keys will be forwarded as a single argument containing every key
%   to \meta{after}. You can use a single macro for \meta{after} or use some
%   arbitrary stuff, which will be left in the input stream before the hashed
%   \kv\ list with one set of braces stripped. In the
%   macro called in \meta{after} you can access the \val\ of a \key\ by using
%   \texttt{\cs[no-index]{ekvcValue}\{\key\}\{\#1\}} (or whichever argument the
%   hashed \kv\ list will be).
% \end{function}
%
% \begin{function}{\ekvcHashAndUse}
%   \begin{syntax}
%     \cs{ekvcHashAndUse}\meta{cs}\marg{primary keys}
%   \end{syntax}
%   This will roughly do the same as |\ekvcHashAndForward|, but instead of
%   specifying what will be used after hashing the keys, \meta{cs} will use what
%   follows the \kv\ list. So its syntax will be
% \end{function}
%   \begin{syntax}
%     \meta{cs}\{\key=\val, \ldots\}\marg{after}
%   \end{syntax}
%
% \begin{function}{\ekvcValue}
%   \begin{syntax}
%     \cs{ekvcValue}\{\key\}\marg{key list}
%   \end{syntax}
%   This is a safe way to access your keys in a hash variant. \key\
%   is the key which's \val\ you want to use out of the \meta{key list}.
%   \meta{key list} should be the key list argument forwarded to your underlying
%   macro by |\ekvcHash|, |\ekvcHashAndForward|, or |\ekvcHashAndUse|. It will
%   be tested whether the hash function to access that \key\ exists, the \key\
%   argument is not empty, and that the \meta{key list} really contains a \val\
%   of that \key. This macro needs exactly two steps of expansion and if used
%   inside of an |\edef| or |\expanded| context will protect the \val\ from
%   further expanding.
% \end{function}
%
% \begin{function}{\ekvcValueFast}
%   \begin{syntax}
%     \cs{ekvcValueFast}\{\key\}\marg{key list}
%   \end{syntax}
%   This behaves just like |\ekvcValue|, but \emph{without any} safety tests. As
%   a result this is about $1.4$ times faster \emph{but} will throw low level
%   \TeX\ errors eventually if the hash function isn't defined or the \key\
%   isn't part of the \meta{key list} (\emph{e.g.}, because it was defined as a
%   key for another macro -- all macros share the same hash function per \key).
%   Use it if you know what you're doing. This macro needs exactly three steps
%   of expansion in the no-errors case.
% \end{function}
%
% \begin{function}{\ekvcValueSplit}
%   \begin{syntax}
%     \cs{ekvcValueSplit}\{\key\}\marg{key list}\marg{next}
%   \end{syntax}
%   If you need a specific \key\ from a \meta{key list} more than once, it'll
%   be a good idea to only extract it once and from then on keep it as a
%   separate argument. Hence the macro \cs{ekvcValueSplit} will extract one
%   specific \key's value from the list and forward the remainder of the list
%   as the first and the \key's value as the second argument to \meta{next}, so
%   the result of this will be \meta{next}\marg{key list'}\marg{value} with
%   \meta{key list'} the remaining list. This is almost as fast as |\ekvcValue|
%   and runs the same tests. Keep in mind that you can't fetch for the same
%   \key\ again from \meta{key list'} as it got removed.
% \end{function}
% \example The following defines a macro |\foo| which will take three keys.
% Since the next parsing step will need the value of one of the keys multiple
% times we split that key off the list:\\
% \begin{minipage}{.75\linewidth}
% \begin{lstlisting}
% \ekvcHash\foo{a=a,b=b,c=c}
%   {\ekvcValueSplit{a}{#1}\foobar}
% \newcommand*\foobar[2]{a is #2.\par
%                        b is \ekvcValue{b}{#1}.\par
%                        c is \ekvcValue{c}{#1}.\par}
% \foo{}
% \end{lstlisting}
% \end{minipage}^^A
% \begin{exresult}[nobeforeafter,box align=center]{.25\linewidth}
%   \ekvcHash\foo{a=a,b=b,c=c}
%     {\ekvcValueSplit{a}{#1}\foobar}
%   \newcommand*\foobar[2]{a is #2.\par
%                          b is \ekvcValue{b}{#1}.\par
%                          c is \ekvcValue{c}{#1}.\par}
%   \foo{}
% \end{exresult}
%
% \begin{function}{\ekvcValueSplitFast}
%   \begin{syntax}
%     \cs{ekvcValueSplitFast}\{\key\}\marg{key list}\marg{next}
%   \end{syntax}
%   This behaves just like |\ekvcValueSplit|, but it won't run the same tests,
%   hence it is faster but more error prone, just like the relation between
%   |\ekvcValue| and |\ekvcValueFast|.
% \end{function}
%
%
% \subsection{Secondary Keys}\label{sec:secondaries}
%
% To remove some of the limitations with the approach that each primary key
% matches an argument or hash entry, you can define secondary keys. Those have
% to be defined for each macro but it doesn't matter whether that macro was a
% split or a hash variant. If a secondary key references another key it doesn't
% matter whether that other key is primary or secondary.
%
% Secondary keys can have a prefix (like |long|) which are called |p|-type
% prefix and must have a type (like |meta|) which are called |t|-type prefix.
% Some types might require some |p|-prefixes, while others might forbid those.
%
% Please keep in mind that key names shouldn't start with \texttt{EKVC\string|}.
%
% \begin{function}{\ekvcSecondaryKeys}
%   \begin{syntax}
%     \cs{ekvcSecondaryKeys}\meta{cs}\{\kv, \ldots\}
%   \end{syntax}
%   This is the front facing macro to define secondary keys. For the macro
%   \meta{cs} define \key\ to have definition \val. The general syntax for \key\
%   should be
%   \begin{quote}\ttfamily\small
%     \meta{prefix} \meta{name}
%   \end{quote}
%   Where \meta{prefix} is a space separated list of optional |p|-type prefixes
%   followed by one |t|-type prefix. The syntax of \val\ is dependent on the
%   used |t|-prefix.
% \end{function}
%
%
% \subsubsection{\texttt{p}-type Prefixes}
%
% There is only one |p|-prefix available, which is |long|.
%
% \begin{function}{long}
%   The following key will be defined |\long|.
% \end{function}
%
%
% \subsubsection{\texttt{t}-type Prefixes}
%
% If you're familiar with \expkvd\ you'll notice that the |t|-type prefixes
% provided here are much fewer. The expansion only concept doesn't allow for
% great variety in the auto-defined keys.
%
% The syntax examples of the |t|-prefixes will show which |p|-prefix will be
% automatically used by printing those black (\texttt{\enflong}), which
% will be available in grey (\texttt{\alllong}), and which will be disallowed in
% red (\texttt{\notlong}). This will be put flush right next to the syntax line.
%
% \begin{function}{meta}
%   \begin{syntax}
%     meta \key\ = \{\kv, \ldots\} \prefixes2
%   \end{syntax}
%   With a |meta| key you can set other keys. Whenever \key\ is used the keys in
%   the \kv\ list will be set to the values given there. You can use the \val\
%   given to \key\ by using |#1| in the \kv\ list. The keys in the \kv\ list can
%   be primary and secondary ones.
% \end{function}
%
% \begin{function}{nmeta}
%   \begin{syntax}
%     nmeta \key\ = \{\kv, \ldots\} \prefixes3
%   \end{syntax}
%   An |nmeta| key is like a |meta| key, but it doesn't take a value, so the
%   \kv\ list is static.
% \end{function}
%
% \begin{function}{alias}
%   \begin{syntax}
%     alias \key\ = \meta{key_2} \prefixes3
%   \end{syntax}
%   This assigns the definition of \meta{key_2} to \key. As a result \key\ is an
%   alias for \meta{key_2} behaving just the same. Both the value taking and the
%   |NoVal| version (that's \expkv\ slang for a key not accepting a value) will
%   be copied if they are defined when |alias| is used. Of course, \meta{key_2}
%   has to be defined, be it as a primary or secondary one.
% \end{function}
%
% \begin{function}{default}
%   \begin{syntax}
%     default \key\ = \marg{default} \prefixes3
%   \end{syntax}
%   If \key\ is a defined value taking key, you can define a |NoVal| version
%   with this that will behave as if \key\ was given \meta{default} as its
%   \val. Note that this doesn't change the initial values of primary keys set
%   at definition time in |\ekvcSplit| and friends. \key\ can be a primary or
%   secondary key.
% \end{function}
%
% \begin{function}{flag-bool}
%   \begin{syntax}
%     flag-bool \key\ = \meta{cs} \prefixes3
%   \end{syntax}
%   This is a secondary key that doesn't directly involve any of the primary or
%   secondary keys.
%   This defines \key\ to take a value, which should be either |true| or
%   |false|, and set the flag called \meta{cs} accordingly as a boolean.
%   If \meta{cs} isn't defined yet it will be initialised as a flag.
%   Please also read \autoref{sec:flags}.
% \end{function}
%
% \begin{function}{flag-true,flag-false}
%   \begin{syntax}
%     flag-true \key\ = \meta{cs} \prefixes3
%   \end{syntax}
%   This is a secondary key that doesn't directly involve any of the primary or
%   secondary keys.
%   This defines \key\ to take no value and set the flag called \meta{cs} to
%   |true| or |false|, respectively.
%   If \meta{cs} isn't defined yet it will be initialised as a flag.
%   Please also read \autoref{sec:flags}.
% \end{function}
%
% \begin{function}{flag-raise}
%   \begin{syntax}
%     flag-raise \key\ = \meta{cs} \prefixes3
%   \end{syntax}
%   This is a secondary key that doesn't directly involve any of the primary or
%   secondary keys.
%   This defines \key\ to take no value and raise the flag called \meta{cs}.
%   If \meta{cs} isn't defined yet it will be initialised as a flag.
%   Please also read \autoref{sec:flags}.
% \end{function}
%
%
% \subsection{Changing the Initial Values}\label{sec:change}
%
% \begin{function}{\ekvcChange}
%   \begin{syntax}
%     \cs{ekvcChange}\meta{cs}\{\kv,\ldots\}
%   \end{syntax}
%   This processes the \kv\ list (just like it would be processed by \meta{cs})
%   to set new defaults for the macro \meta{cs} defined with \expkvc\ (meaning
%   the values used if you don't provide anything at use time). Inside the \kv\
%   list both primary and secondary keys can be used. If \meta{cs} was defined
%   |\long| earlier it will still be |\long|, every other \TeX\ prefix will be
%   stripped (but \expkvc\ doesn't support them anywhere else so that should be
%   fine). The resulting new defaults will be stored inside the \meta{cs}
%   locally (just as the original defaults were).
% \end{function}
%
% Consider the following example:\\
% \begin{minipage}[c]{.75\linewidth}
% \begin{lstlisting}
% \ekvcSplit\foo{a=a,b=b}{a is #1.\par b is #2.\par}
% \foo{} % initial defaults
% \ekvcChange\foo{b=B}
% \foo{} % new defaults
% \ekvcSecondaryKeys\foo{meta c={a={#1},b={#1}}}
% \ekvcChange\foo{c=c}
% \foo{} % newer defaults
% \end{lstlisting}
% \end{minipage}^^A
% \begin{exresult}[nobeforeafter,box align=center]{.25\linewidth}
%   \ekvcSplit\foo{a=a,b=b}{a is #1.\par b is #2.\par}
%   \foo{}
%   \ekvcChange\foo{b=B}
%   \foo{}
%   \ekvcSecondaryKeys\foo{meta c={a={#1},b={#1}}}
%   \ekvcChange\foo{c=c}
%   \foo{}
% \end{exresult}
%
% As a result with this the typical setup macro could be implemented:
% \begin{lstlisting}
% \ekvcHashAndUse\foo{key=a,key=b}
% \newcommand\foosetup{\ekvcChange\foo}
% \end{lstlisting}
% Of course the usage is limited to a single macro |\foo|, hence this might be
% not as powerful as similar macros used with other \kv\ interfaces. But at
% least a few similar macros could be grouped using the same key parsing macro
% internally.
%
%
% \subsection{Flags}\label{sec:flags}
%
% The idea of flags is taken from \pkg{expl3}. They provide a way to store
% numerical information expandably, however only incrementing and accessing
% works expandably, decrementing is unexpandable. A flag has a height, which
% is a numerical value, and which can be raised by $1$. Flags come at a high
% computational cost (accessing them is slow and they require more memory than
% normal \TeX\ data types like registers, both getting linearly worse with the
% height), so don't use them if not necessary.
%
% The state of flags is always changed locally to the current group, but not to
% the current macro, so if you're using one of the |t|-types involving flags
% bear in mind that they can affect other macros using the same flags at the
% current group level!
%
% \expkvc\ provides some macros to access, alter, and use flags. Flags of
% \expkvc\ don't share a name space with the flags of \pkg{expl3}.
%
% \begin{function}{\ekvcFlagNew}
%   \begin{syntax}
%     \cs{ekvcFlagNew}\meta{flag}
%   \end{syntax}
%   This initialises the macro \meta{flag} as a new flag. It isn't checked
%   whether the macro \meta{flag} is currently undefined.  A \meta{flag} will
%   expand to the flag's current height with a trailing space (so you can use it
%   directly with |\ifnum| for example and it will terminate the number scanning
%   itself).
% \end{function}
%
%  All other macros dealing with flags take as a parameter a macro defined as a
%  \meta{flag} with |\ekvcFlagNew|.
%
% \begin{function}{\ekvcFlagHeight}
%   \begin{syntax}
%     \cs{ekvcFlagHeight}\meta{flag}
%   \end{syntax}
%   This expands to the current height of \meta{flag} in a single step of
%   expansion (without a trailing space).
% \end{function}
%
% \begin{function}{\ekvcFlagRaise}
%   \begin{syntax}
%     \cs{ekvcFlagRaise}\meta{flag}
%   \end{syntax}
%   This expandably raises the height of \meta{flag} by $1$.
% \end{function}
%
% \begin{function}{\ekvcFlagSetTrue,\ekvcFlagSetFalse}
%   \begin{syntax}
%     \cs{ekvcFlagSetTrue}\meta{flag}
%   \end{syntax}
%   By interpreting an even value as false and an odd value as true we can use a
%   flag as a boolean. This expandably sets \meta{flag} to true or false,
%   respectively, by raising it if necessary.
% \end{function}
%
% \begin{function}{\ekvcFlagIf}
%   \begin{syntax}
%     \cs{ekvcFlagIf}\meta{flag}\marg{true}\marg{false}
%   \end{syntax}
%   This interprets a \meta{flag} as a boolean and expands to either \meta{true}
%   or \meta{false}.
% \end{function}
%
% \begin{function}{\ekvcFlagIfRaised}
%   \begin{syntax}
%     \cs{ekvcFlagIfRaised}\meta{flag}\marg{true}\marg{false}
%   \end{syntax}
%   This tests whether the \meta{flag} is raised, meaning it has a height
%   greater than zero, and if so expands to \meta{true} else to \meta{false}.
% \end{function}
%
% \begin{function}{\ekvcFlagReset}
%   \begin{syntax}
%     \cs{ekvcFlagReset}\meta{flag}
%   \end{syntax}
%   This resets a flag (so restores its height to $0$). This operation is
%   \emph{not} expandable and done locally. If you really intend to use flags
%   you can reset them every now and then to keep the performance hit low.
% \end{function}
%
% \begin{function}{\ekvcFlagGetHeight}
%   \begin{syntax}
%     \cs{ekvcFlagGetHeight}\meta{flag}\marg{next}
%   \end{syntax}
%   This retrieves the current height of the \meta{flag} and provides it as a
%   braced argument to \meta{next}, leaving \meta{next}\marg{height} in the
%   input stream.
% \end{function}
%
% \begin{function}{\ekvcFlagGetHeights}
%   \begin{syntax}
%     \cs{ekvcFlagGetHeights}\marg{flag-list}\marg{next}
%   \end{syntax}
%   This retrieves the current height of each \meta{flag} in the
%   \meta{flag-list} and provides them as a single argument to \meta{next}.
%   Inside that argument each height is enclosed in a set of braces
%   individually. The \meta{flag-list} is just a single argument containing the
%   \meta{flag}s.
%   So a usage like
%   \texttt
%     {^^A
%       \cs[no-index]{ekvcFlagGetHeights}\hskip0pt^^A
%       \{\cs[no-index]{myflagA}\cs[no-index]{myflagB}\}\hskip0pt^^A
%       \{\cs[no-index]{stuff}\}^^A
%     }
%   will expand to
%   \texttt{\cs[no-index]{stuff}\{\marg{height-A}\marg{height-B}\}}.
% \end{function}
%
%
% \subsection{Further Examples}
%
% How could a documentation be a good documentation without some basic examples?
% Say we want to define a small macro expanding to some character description
% (who knows why this has to be expandable?). A character description will not
% have too many items to it, so we use |\ekvcSplit| (the comments with the
% parameter numbers are of course not necessary and just ease reading the
% example).
% \begin{lstlisting}
% \ekvcSplit\character
%   {
%     name=John Doe,            % #1
%     age=any,                  % #2
%     nationality=the Universe, % #3
%     hobby=to exist,           % #4
%     type=Mister,              % #5
%     pronoun=He,               % #6
%     possessive=his,           % #7
%   }
%   {#1 is a #5 from #3. #6 is of #2 age and #7 hobby is #4.\par}
% \end{lstlisting}
% Also we want to give some short cuts so that it's easier to describe several
% persons.
% \begin{lstlisting}
% \ekvcSecondaryKeys\character
%   {
%     alias pro = pronoun,
%     alias pos = possessive,
%     nmeta me =
%       {
%         name=Jonathan,
%         age=a young,
%         nationality=Germany,
%         hobby=\TeX\ coding,
%       },
%     meta lady =
%       {type=Lady, pronoun=She, possessive=her, name=Jane Doe, #1},
%     nmeta paulo =
%       {
%         name=Paulo,
%         type=duck,
%         age=a young,
%         nationality=Brazil,
%         hobby=to quack,
%       }
%   }
% \end{lstlisting}
% Now we can describe people using
% \begin{lstlisting}
% \character{}
% \character{me}
% \character{paulo}
% \character
%   {lady={name=Evelyn,nationality=Ireland,age=the best,hobby=reading}}
% \character
%   {
%     name=Our sun, type=star, nationality=our solar system, pro=It,
%     age=an old, pos=its, hobby=shining
%   }
% \end{lstlisting}
% As one might see, the |lady| key could actually have been an |nmeta| key as
% well, as all that is done with the argument is using it as a \kv\ list.
%
% The result of only the first two usages would be:
% \begin{exresult}{\linewidth}
%   \ekvcSplit\character
%     {
%       name=John Doe,
%       age=any,
%       nationality=the Universe,
%       hobby=to exist,
%       type=Mister,
%       pronoun=He,
%       possessive=his,
%     }
%     {#1 is a #5 from #3. #6 is of #2 age and #7 hobby is #4.\par}
%   \ekvcSecondaryKeys\character
%     {
%       alias pro = pronoun,
%       alias pos = possessive,
%       nmeta me =
%         {
%           name=Jonathan,
%           age=a young,
%           nationality=Germany,
%           hobby=\TeX\ coding,
%         },
%       meta lady =
%         {type=Lady, pronoun=She, possessive=her, name=Jane Doe, #1},
%       nmeta paulo =
%         {
%           name=Paulo,
%           type=duck,
%           age=a young,
%           nationality=Brazil,
%           hobby=to quack,
%         }
%     }
%   \character{}
%   \character{me}
% \end{exresult}
%
% Using \pkg{xparse} or \expkv's \cs{ekvoptarg} or \cs{ekvoptargTF} and
% forwarding arguments one can easily define \kv\ macros with actual optional
% and mandatory arguments as well. A small nonsense example (which should
% perhaps use |\ekvcSplitAndForward| instead of |\ekvcHashAndForward| since it
% only uses four keys and one other argument -- and isn't expandable since it
% uses a \env{tabular} environment, so it would've been better to use a more
% feature rich \kv\ interface most likely, \emph{e.g.}, the one provided by
% \expkvd):
% \begin{lstlisting}
% \makeatletter
% \newcommand*\nonsense{\ekvoptarg\nonsense@a{}}
% \ekvcHashAndForward\nonsense@a\nonsense@b
%   {
%     keyA = A,
%     keyB = B,
%     keyC = c,
%     keyD = d,
%   }
% \newcommand*\nonsense@b[2]
%   {%
%     \begin{tabular}{lll}
%       key & A & \ekvcValue{keyA}{#1} \\
%           & B & \ekvcValue{keyB}{#1} \\
%           & C & \ekvcValue{keyC}{#1} \\
%           & D & \ekvcValue{keyD}{#1} \\
%       \multicolumn{2}{l}{mandatory} & #2 \\
%     \end{tabular}%
%   }
% \makeatother
% \end{lstlisting}
% And then we would be able to do some nonsense
% \begin{lstlisting}
% \nonsense{}
% \nonsense[keyA=hihi]{haha}
% \nonsense[keyA=hihi, keyB=A]{hehe}
% \nonsense[keyC=huhu, keyA=hihi, keyB=A]{haha}
% \end{lstlisting}
% resulting in
% \begin{exresult}{\linewidth}
%   \makeatletter
%   \newcommand*\nonsense{\ekvoptarg\nonsense@a{}}
%   \ekvcHashAndForward\nonsense@a\nonsense@b
%     {
%       keyA = A,
%       keyB = B,
%       keyC = c,
%       keyD = d,
%     }
%   \newcommand*\nonsense@b[2]
%     {%
%       \begin{tabular}{lll}
%         key & A & \ekvcValue{keyA}{#1} \\
%             & B & \ekvcValue{keyB}{#1} \\
%             & C & \ekvcValue{keyC}{#1} \\
%             & D & \ekvcValue{keyD}{#1} \\
%         \multicolumn{2}{l}{mandatory} & #2 \\
%       \end{tabular}%
%     }
%   \makeatother
%   \nonsense{}
%   \nonsense[keyA=hihi]{haha}
%   \nonsense[keyA=hihi, keyB=A]{hehe}
%   \nonsense[keyC=huhu, keyA=hihi, keyB=A]{haha}
% \end{exresult}
%
%
% \subsection{Freedom for Keys!}
%
% If this was the \TeX book this subsection would have a double bend sign. Not
% because it is overly complicated, but because it shows things which could
% break \expkvc's expandability and its alignment safety. This is for
% experienced users wanting to get the most flexibility and knowing what they
% are doing.
%
% In case you're wondering, it is possible to define other keys than the
% primaries and the secondary types listed in \autoref{sec:secondaries} for a
% macro defined with \expkvc\ by using the low-level interface of \expkv\ or
% even the interface provided by \expkvd. The set name used for \expkvc's keys
% is the macro name, including the leading backslash, or more precisely
% |\string|\meta{cs} is used. This can be exploited to define additional keys
% with arbitrary code. Consider the following \emph{bad} example:
% \begin{lstlisting}
% \ekvcSplit\foo{a=A,b=B}{a is #1.\par b is #2}
% \protected\ekvdef{\string\foo}{c}{\def\fooC{#1}}
% \end{lstlisting}
% This would define a key named |c| that will store its value inside a macro.
% The issue with this is that this can't be done expandably. As a result, the
% macro |\foo| isn't always expandable any more (not that bad if this was never
% required; killjoy if it was) and as soon as the key |c| is used, it is also no
% longer alignment safe\footnote{This means that the \kv-list can't contain
%   alignment markers that are not inside an additional set of braces if used
%   inside a \TeX\ alignment.}
% (might be bad depending on the usage).
%
% So why do I show you this? Because we could as well do something useful like
% create a key that pre-parses the input and after that passes the parsed value
% on. This parsing would have to be completely expandable though. For the
% pass-on part we can use the following function:
%
% \begin{function}{\ekvcPass}
%   \begin{syntax}
%     \cs{ekvcPass}\meta{cs}\marg{key}\marg{value}
%   \end{syntax}
%   This passes \meta{value} on to \meta{key} for the \expkvc-macro \meta{cs}.
%   It should be used inside the key parsing of a macro defined with \expkvc,
%   else this most likely results in a low level \TeX\ error.
% \end{function}
%
% With this we could for example split the value of a key at a hyphen and pass
% the parts to different keys:\\
% \begin{minipage}[c]{.75\linewidth}
% \begin{lstlisting}
% \ekvcSplit\foo{a=A,b=B}{a is #1.\par b is #2.\par}
% \ekvdef{\string\foo}{c}{\fooSplit#1\par}
% \def\fooSplit#1-#2\par
%   {\ekvcPass\foo{a}{#1}\ekvcPass\foo{b}{#2}}
% \foo{}
% \foo{c=1-2}
% \end{lstlisting}
% \end{minipage}
% \begin{exresult}[nobeforeafter,box align=center]{.25\linewidth}
%   \ekvcSplit\foo{a=A,b=B}{a is #1.\par b is #2.\par}
%   \ekvdef{\string\foo}{c}{\fooSplit#1\par}
%   \def\fooSplit#1-#2\par{\ekvcPass\foo{a}{#1}\ekvcPass\foo{b}{#2}}
%   \foo{}
%   \foo{c=1-2}
% \end{exresult}
%
%
% \subsection{Speed Considerations}\label{sec:speed}
%
% As already mentioned in the introduction there are some speed considerations
% implied if you choose to define macros via \expkvc. However the overhead isn't
% the factor which should hinder you to use \expkvc\ if you found a reasonable
% use case. The key-parsing is still faster than with most other \kv\ packages
% (see the ``Comparisons'' subsection in the \expkv\ documentation).
%
% The speed considerations in this subsection use the first example in this
% documentation as the benchmark. So we have seven keys and a short sentence
% which should be typeset. For comparisons I use the following equivalent
% \expkvd\ definitions. Each result is the average between changing no keys from
% their initial values and altering four. Furthermore I'll compare three
% variants of \expkvc\ with the \expkvd\ definitions, namely the split example
% from above, a hash variant using |\ekvcValue| and a hash variant using
% |\ekvcValueFast|.
% \begin{lstlisting}
% \usepackage{expkv-def}
% \ekvdefinekeys{keys}
%   {%
%     ,store   name         = \KEYSname
%     ,initial name         = John Doe
%     ,store   age          = \KEYSage
%     ,initial age          = any
%     ,store   nationality  = \KEYSnationality
%     ,initial nationality  = the Universe
%     ,store   hobby        = \KEYShobby
%     ,initial hobby        = to exist
%     ,store   type         = \KEYStype
%     ,initial type         = Mister
%     ,store   pronoun      = \KEYSpronoun
%     ,initial pronoun      = He
%     ,store   possessive   = \KEYSpossessive
%     ,initial possessive   = his
%   }
% \newcommand*\KEYS[1]
%   {%
%     \begingroup
%       \ekvset{keys}{#1}%
%       \KEYSname\ is a \KEYStype\ from \KEYSnationality. \KEYSpronoun\ is
%       of \KEYSage\ age and \KEYSpossessive\ hobby is \KEYShobby.%
%     \endgroup
%   }
% \end{lstlisting}
%
% The first comparison removes the typesetting part from all the definitions, so
% that only the key parsing is compared. In this comparison the |\ekvcValue|
% and |\ekvcValueFast| variants will not differ, as they are exactly the same
% until the key usage. We find that the split approach is $1.4$ times slower
% than the \expkvd\ setup and the hash variants end up in the middle at $1.17$
% times slower.
%
% Next we put the typesetting part back in. Every call of the macros will
% typeset the sentences into a box register in horizontal mode. With the
% typesetting part (which includes the accessing of values) the fastest remains
% the \expkvd\ definitions, but split is close at $1.16$ times slower, followed
% by the hash variant with fast accesses at $1.36$ times slower, and the safe
% hash access variant ranks in the slowest $1.8$ times slower than \expkvd.
%
% Just in case you're wondering now, a simple macro taking seven arguments is
% $30$ to $40$ times faster than any of those in the argument grabbing and \kv\
% parsing part and only $1.5$ to $2.8$ times faster if the typesetting part is
% factored in. So the real choke isn't the parsing.
%
% So to summarize this, if you have a reasonable use case for expandable \kv\
% parsing macros you should go on and define them using \expkvc. If you have a
% reasonable use case for \kv\ parsing macros but defining them expandable isn't
% necessary for your use you should take advantage of the greater flexibility of
% non-expandable \kv\ setups (but if you're after maximum speed there aren't
% that many \kv\ parsers beating \expkvc). And if you are after maximum
% performance maybe ditching the \kv\ interface altogether is a good idea, but
% depending on the number of arguments your interface might get convoluted.
%
%
% \subsection{Useless Macros}
%
% Perhaps these macros aren't completely useless, but I figured from a user's
% point of view I wouldn't know what I should do with these.
%
% \begin{function}{\ekvcDate,\ekvcVersion}
%   These two macros store the version and the date of the package/generic code.
% \end{function}
%
%
% \subsection{Bugs}
%
% Of course I don't think there are any bugs (who would knowingly distribute
% buggy software as long as he isn't a multi-million dollar corporation?). But
% if you find some please let me know. For this one might find my email address
% on the first page or file an issue on Github:
% \url{https://github.com/Skillmon/tex_expkv-cs}
%
%
% \subsection{License}
%
% Copyright \textcopyright\ 2020\unless\ifnum\year=2020--\the\year\fi\
% Jonathan P. Spratte
%
% \medskip\noindent
% This work may be distributed and/or modified under the conditions of the
% \LaTeX\ Project Public License (LPPL), either version 1.3c of this license or
% (at your option) any later version. The latest version of this license is in
% the file:
%
% \url{http://www.latex-project.org/lppl.txt}
%
% \noindent
% This work is ``maintained'' (as per LPPL maintenance status) by
%
% \mbox{Jonathan P. Spratte}.
%
% \end{documentation}^^A=<<
%
% \begin{implementation}^^A>>=
%
% \clearpage
%
% \section{Implementation}^^A>>=
%^^A the LaTeX package >>=
% \subsection{The \LaTeX\ Package}
% Just like for \expkv\ we provide a small \LaTeX\ package that sets up things
% such that we behave nicely on \LaTeX\ packages and files system. It'll
% |\input| the generic code which implements the functionality.
% \gobbledocstriptag
%<*pkg>
%    \begin{macrocode}
\RequirePackage{expkv}
\def\ekvc@tmp
  {%
    \ProvidesFile{expkv-cs.tex}%
      [%
        \ekvcDate\space v\ekvcVersion\space
        define expandable key=val macros using expkv%
      ]%
  }
\input{expkv-cs.tex}
\ProvidesPackage{expkv-cs}%
  [%
    \ekvcDate\space v\ekvcVersion\space
    define expandable key=val macros using expkv%
  ]
%    \end{macrocode}
% \gobbledocstriptag
%</pkg>
%^^A=<<
%^^A main file >>=
% \subsection{The Generic Code}
% The rest of this implementation will be the generic code.
% \gobbledocstriptag
%<*tex>
%
% Load \expkv\ if the package didn't already do so -- since \expkv\ has
% safeguards against being loaded twice this does no harm and the overhead
% isn't that big. Also we reuse some of the internals of \expkv\ to save us from
% retyping them.
%    \begin{macrocode}
\input expkv
%    \end{macrocode}
%
% We make sure that \file{expkv-cs.tex} is only input once:
%    \begin{macrocode}
\expandafter\ifx\csname ekvcVersion\endcsname\relax
\else
  \expandafter\endinput
\fi
%    \end{macrocode}
%
% \begin{macro}{\ekvcVersion,\ekvcDate}
% We're on our first input, so lets store the version and date in a macro.
%    \begin{macrocode}
\def\ekvcVersion{0.10}
\def\ekvcDate{2021-06-03}
%    \end{macrocode}
% \end{macro}
%
% If the \LaTeX\ format is loaded we want to be a good file and report back who
% we are, for this the package will have defined |\ekvc@tmp| to use
% |\ProvidesFile|, else this will expand to a |\relax| and do no harm.
%    \begin{macrocode}
\csname ekvc@tmp\endcsname
%    \end{macrocode}
%
% Store the category code of |@| to later be able to reset it and change it to
% 11 for now.
%    \begin{macrocode}
\expandafter\chardef\csname ekvc@tmp\endcsname=\catcode`\@
\catcode`\@=11
%    \end{macrocode}
% |\ekvc@tmp| will be reused later, but we don't need it to ever store
% information long-term after \expkvc\ was initialized.
%
% \begin{macro}[internal]{\ekvc@keycount}
%   We'll need to keep count how many keys must be defined for each macro in the
%   |split| variants.
%    \begin{macrocode}
\newcount\ekvc@keycount
%    \end{macrocode}
% \end{macro}
%
% \begin{macro}[internal]{\ekvc@long,\ekvc@any@long}
%   Some macros will have to be defined long. These two will be let to |\long|
%   when this should be the case.
%    \begin{macrocode}
\let\ekvc@long\ekv@empty
\let\ekvc@any@long\ekv@empty
%    \end{macrocode}
% \end{macro}
%
% \begin{macro}[internal]{\ekvc@ifdefined}
%   We want to test whether a macro is already defined. This test checks for a
%   defined macro that isn't |\relax|.
%    \begin{macrocode}
\long\def\ekvc@ifdefined#1%
  {%
    \ifdefined#1%
      \ifx\relax#1%
        \ekv@fi@gobble
      \fi
      \@firstofone
      \ekv@fi@firstoftwo
    \fi
    \@secondoftwo
  }
%    \end{macrocode}
% \end{macro}
%
% \begin{macro}[internal]
%   {
%     \ekvc@ekvset@pre@expander,
%     \ekvc@ekvset@pre@expander@a,
%     \ekvc@ekvset@pre@expander@b
%   }
%   This macro expands |\ekvset| twice so that the first two steps of expansion
%   don't have to be made every time the \expkvc\ macros are used. We have to do
%   a little magic trick to get the macro parameter |#1| for the macro
%   definition this is used in, even though we're calling |\unexpanded|. We do
%   that by splitting the expanded |\ekvset| at some marks and place |##1| in
%   between. At this spot we also add |\ekv@alignsafe| and |\ekv@endalignsafe|
%   to ensure that macros created with \expkvc\ are alignment safe.
%    \begin{macrocode}
\def\ekvc@ekvset@pre@expander#1%
  {%
    \expandafter\ekvc@ekvset@pre@expander@a\ekvset{#1}\ekvc@stop\ekvc@stop
  }
\def\ekvc@ekvset@pre@expander@a
  {%
    \expandafter\ekvc@ekvset@pre@expander@b
  }
\def\ekvc@ekvset@pre@expander@b#1\ekvc@stop#2\ekvc@stop
  {%
    \unexpanded\expandafter{\ekv@alignsafe}%
    \unexpanded{#1}##1\unexpanded{#2}%
    \unexpanded\expandafter{\ekv@endalignsafe}%
  }
%    \end{macrocode}
% \end{macro}
%
% \begin{macro}{\ekvcSplitAndUse}
%   The first user macro we want to set up can be reused for
%   \cs[no-index]{ekvcSplitAndForward} and \cs[no-index]{ekvcSplit}. We'll split
%   this one up so that the test whether the macro is already defined doesn't
%   run twice.
%    \begin{macrocode}
\protected\long\def\ekvcSplitAndUse#1#2%
  {%
    \let\ekvc@helpers@needed\@firstoftwo
    \ekvc@ifdefined#1%
      {\ekvc@err@already@defined#1}%
      {\ekvcSplitAndUse@#1{}{#2}}%
  }
%    \end{macrocode}
% \end{macro}
%
% \begin{macro}[internal]{\ekvcSplitAndUse@}
%   The actual macro setting up things. We need to set some variables, forward
%   the key list to |\ekvc@SetupSplitKeys|, and afterwards define the front
%   facing macro to call |\ekvset| and put the initials and the argument sorting
%   macro behind it. The internals |\ekvc@any@long|, |\ekvc@initials| and
%   |\ekvc@keycount| will be set correctly by |\ekvc@SetupSplitKeys|.
%    \begin{macrocode}
\protected\long\def\ekvcSplitAndUse@#1#2#3%
  {%
    \edef\ekvc@set{\string#1}%
    \ekvc@SetupSplitKeys{#3}%
    \ekvc@helpers@needed
      {%
        \ekvc@any@long\edef#1##1%
          {%
            \expandafter\ekvc@ekvset@pre@expander\expandafter{\ekvc@set}%
            \unexpanded\expandafter
              {\csname ekvc@split@\the\ekvc@keycount\endcsname}%
            \unexpanded\expandafter{\ekvc@initials{}#2}%
          }%
      }%
      {%
        \ekvc@any@long\edef#1##1%
          {%
            \expandafter\ekvc@ekvset@pre@expander\expandafter{\ekvc@set}%
            \unexpanded{#2}%
            \unexpanded\expandafter{\ekvc@initials}%
          }%
      }%
  }
%    \end{macrocode}
% \end{macro}
%
%
% \begin{macro}{\ekvcSplitAndForward}
%   This just reuses |\ekvcSplitAndUse@| with a non-empty second argument,
%   resulting in that argument to be called after the splitting.
%    \begin{macrocode}
\protected\long\def\ekvcSplitAndForward#1#2#3%
  {%
    \let\ekvc@helpers@needed\@firstoftwo
    \ekvc@ifdefined#1%
      {\ekvc@err@already@defined#1}%
      {\ekvcSplitAndUse@#1{{#2}}{#3}}%
  }
%    \end{macrocode}
% \end{macro}
%
% \begin{macro}{\ekvcSplit}
%   The first half is just |\ekvcSplitAndForward| then we define the macro to
%   which the parsed key list is forwarded. There we need to allow for up to
%   nine arguments.
%    \begin{macrocode}
\protected\long\def\ekvcSplit#1#2#3%
  {%
    \let\ekvc@helpers@needed\@secondoftwo
    \ekvc@ifdefined#1%
      {\ekvc@err@already@defined#1}%
      {%
        \expandafter
        \ekvcSplitAndUse@\expandafter#1\csname ekvc@\string#1\endcsname{#2}%
        \ifnum\ekvc@keycount<1
          \ekvc@any@long\expandafter\def\csname ekvc@\string#1\endcsname{#3}%
        \else
          \ifnum\ekvc@keycount>9
            \ekvc@err@toomany{#1}%
            \let#1\ekvc@undefined
          \else
            \ekvcSplit@build@argspec
            \ekvc@any@long\expandafter
            \def\csname ekvc@\string#1\expandafter\endcsname\ekvc@tmp{#3}%
          \fi
        \fi
      }%
  }
%    \end{macrocode}
% \end{macro}
%
% \begin{macro}[internal]{\ekvcSplit@build@argspec,\ekvcSplit@build@argspec@}
%    \begin{macrocode}
\protected\def\ekvcSplit@build@argspec
  {%
    \begingroup
      \edef\ekvc@tmp
        {\endgroup\def\unexpanded{\ekvc@tmp}{\ekvcSplit@build@argspec@{1}}}%
    \ekvc@tmp
  }
\def\ekvcSplit@build@argspec@#1%
  {%
    \ifnum#1>\ekvc@keycount
      \ekv@fi@gobble
    \fi
    \@firstofone
      {%
        \unexpanded\expandafter{\csname ekvc@splitmark@#1\endcsname####}#1%
        \expandafter\ekvcSplit@build@argspec@\expandafter{\the\numexpr#1+1}%
      }%
  }
%    \end{macrocode}
% \end{macro}
%
% \begin{macro}[internal]
%   {
%     \ekvc@SetupSplitKeys, \ekvc@SetupSplitKeys@a, \ekvc@SetupSplitKeys@b,
%     \ekvc@SetupSplitKeys@c
%   }
%   These macros parse the list of keys and set up the key macros. First we need
%   to initialise some macros and start |\ekvparse|.
%    \begin{macrocode}
\protected\long\def\ekvc@SetupSplitKeys
  {%
    \ekvc@keycount=0
    \let\ekvc@any@long\ekv@empty
    \let\ekvc@initials\ekv@empty
    \ekvparse\ekvc@err@value@required\ekvc@SetupSplitKeys@a
  }
%    \end{macrocode}
%   Then we need to step the key counter for each key. Also we have to check
%   whether this key has a |long| prefix so we initialise |\ekvc@long|.
%    \begin{macrocode}
\protected\def\ekvc@SetupSplitKeys@a#1%
  {%
    \advance\ekvc@keycount1
    \let\ekvc@long\ekv@empty
    \ekvc@ifspace{#1}%
      {\ekvc@SetupSplitKeys@b#1\ekvc@stop}%
      {\ekvc@SetupSplitKeys@c{#1}}%
  }
%    \end{macrocode}
%   If there was a space, there might be a prefix. If so call the prefix macro,
%   else call the next step |\ekvc@SetupSplitKeys@c| which will define the key
%   macro and add the key's value to the initials list.
%    \begin{macrocode}
\protected\def\ekvc@SetupSplitKeys@b#1 #2\ekvc@stop
  {%
    \ekv@ifdefined{ekvc@split@p@#1}%
      {\csname ekvc@split@p@#1\endcsname{#2}}%
      {\ekvc@SetupSplitKeys@c{#1 #2}}%
  }
%    \end{macrocode}
%   The inner definition is grouped, because we don't want to actually define
%   the marks we build with |\csname|. We have to append the value to the
%   |\ekvc@initials| list here with the correct split mark. The key macro will
%   read everything up to those split marks and change the value following it to
%   the value given to the key. Additionally we'll need a sorting macro for each
%   key count in use so we set it up with |\ekvc@setup@splitmacro|.
%    \begin{macrocode}
\protected\long\def\ekvc@SetupSplitKeys@c#1#2%
  {%
    \begingroup
    \edef\ekvc@tmp
      {%
        \endgroup
        \long\def\unexpanded{\ekvc@tmp}####1####2%
            \unexpanded\expandafter
            {\csname ekvc@splitmark@\the\ekvc@keycount\endcsname}####3%
          {%
            ####2%
            \unexpanded\expandafter
              {\csname ekvc@splitmark@\the\ekvc@keycount\endcsname}{####1}%
          }%
%    \end{macrocode}
%   The short variant needs a bit of special treatment. The key macro will be
%   short to throw the correct error, but since there might be long macros
%   somewhere the reordering of arguments needs to be long, so for short keys we
%   use a two step approach, first grabbing only the short argument, then
%   reordering.
%    \begin{macrocode}
        \unless\ifx\ekvc@long\long
          \let\unexpanded\expandafter
            {\csname ekvc@\ekvc@set(#1)\endcsname\ekvc@tmp}%
          \def\unexpanded{\ekvc@tmp}####1%
            {%
              \unexpanded\expandafter{\csname ekvc@\ekvc@set(#1)\endcsname}%
                {####1}%
            }%
        \fi
        \def\unexpanded{\ekvc@initials}%
          {%
            \unexpanded\expandafter{\ekvc@initials}%
            \unexpanded\expandafter
              {\csname ekvc@splitmark@\the\ekvc@keycount\endcsname{#2}}%
          }%
      }%
    \ekvc@tmp
    \ekvlet\ekvc@set{#1}\ekvc@tmp
    \ekvc@helpers@needed
      {\expandafter\ekvc@setup@splitmacro\expandafter{\the\ekvc@keycount}}%
      {}%
  }
%    \end{macrocode}
% \end{macro}
%
% \begin{macro}[internal]{\ekvc@split@p@long}
%   The |long| prefix lets the internals |\ekvc@long| and |\ekvc@any@long| to
%   |\long| so that the key macro will be long.
%    \begin{macrocode}
\protected\def\ekvc@split@p@long
  {%
    \let\ekvc@long\long
    \let\ekvc@any@long\long
    \ekvc@SetupSplitKeys@c
  }
%    \end{macrocode}
% \end{macro}
%
% \begin{macro}[internal]{\ekvc@defarggobbler}
%   This is needed to define a macro with 1-9 parameters programmatically.
%   \LaTeX's \cs[no-index]{newcommand} does something similar for example.
%    \begin{macrocode}
\protected\def\ekvc@defarggobbler#1{\def\ekvc@tmp##1#1##2##{##1#1}}
%    \end{macrocode}
% \end{macro}
%
% \begin{macro}[internal]
%   {
%     \ekvc@setup@splitmacro,
%     \ekvc@split@1, \ekvc@split@2, \ekvc@split@3,
%     \ekvc@split@4, \ekvc@split@5, \ekvc@split@6,
%     \ekvc@split@7
%   }
%   Since the first few split macros are different from the others we manually
%   set those up now. All the others will be defined as needed (always
%   globally).  The split macros just read up until the correct split mark, move
%   that argument into a list and reinsert the rest, calling the next split
%   macro afterwards.
%    \begin{macrocode}
\begingroup
\edef\ekvc@tmp
  {%
    \long\gdef\unexpanded\expandafter{\csname ekvc@split@1\endcsname}%
        \unexpanded\expandafter{\csname ekvc@splitmark@1\endcsname}%
        ##1##2##3%
      {##3{##1}##2}%
    \long\gdef\unexpanded\expandafter{\csname ekvc@split@2\endcsname}%
        \unexpanded\expandafter{\csname ekvc@splitmark@1\endcsname}##1%
        \unexpanded\expandafter{\csname ekvc@splitmark@2\endcsname}##2%
        ##3##4%
      {##4{##1}{##2}##3}%
    \long\gdef\unexpanded\expandafter{\csname ekvc@split@3\endcsname}%
        \unexpanded\expandafter{\csname ekvc@splitmark@1\endcsname}##1%
        \unexpanded\expandafter{\csname ekvc@splitmark@2\endcsname}##2%
        \unexpanded\expandafter{\csname ekvc@splitmark@3\endcsname}##3%
        ##4##5%
      {##5{##1}{##2}{##3}##4}%
    \long\gdef\unexpanded\expandafter{\csname ekvc@split@4\endcsname}%
        \unexpanded\expandafter{\csname ekvc@splitmark@1\endcsname}##1%
        \unexpanded\expandafter{\csname ekvc@splitmark@2\endcsname}##2%
        \unexpanded\expandafter{\csname ekvc@splitmark@3\endcsname}##3%
        \unexpanded\expandafter{\csname ekvc@splitmark@4\endcsname}##4%
        ##5##6%
      {##6{##1}{##2}{##3}{##4}##5}%
    \long\gdef\unexpanded\expandafter{\csname ekvc@split@5\endcsname}%
        \unexpanded\expandafter{\csname ekvc@splitmark@1\endcsname}##1%
        \unexpanded\expandafter{\csname ekvc@splitmark@2\endcsname}##2%
        \unexpanded\expandafter{\csname ekvc@splitmark@3\endcsname}##3%
        \unexpanded\expandafter{\csname ekvc@splitmark@4\endcsname}##4%
        \unexpanded\expandafter{\csname ekvc@splitmark@5\endcsname}##5%
        ##6##7%
      {##7{##1}{##2}{##3}{##4}{##5}##6}%
    \long\gdef\unexpanded\expandafter{\csname ekvc@split@6\endcsname}%
        \unexpanded\expandafter{\csname ekvc@splitmark@1\endcsname}##1%
        \unexpanded\expandafter{\csname ekvc@splitmark@2\endcsname}##2%
        \unexpanded\expandafter{\csname ekvc@splitmark@3\endcsname}##3%
        \unexpanded\expandafter{\csname ekvc@splitmark@4\endcsname}##4%
        \unexpanded\expandafter{\csname ekvc@splitmark@5\endcsname}##5%
        \unexpanded\expandafter{\csname ekvc@splitmark@6\endcsname}##6%
        ##7##8%
      {##8{##1}{##2}{##3}{##4}{##5}{##6}##7}%
    \long\gdef\unexpanded\expandafter{\csname ekvc@split@7\endcsname}%
        \unexpanded\expandafter{\csname ekvc@splitmark@1\endcsname}##1%
        \unexpanded\expandafter{\csname ekvc@splitmark@2\endcsname}##2%
        \unexpanded\expandafter{\csname ekvc@splitmark@3\endcsname}##3%
        \unexpanded\expandafter{\csname ekvc@splitmark@4\endcsname}##4%
        \unexpanded\expandafter{\csname ekvc@splitmark@5\endcsname}##5%
        \unexpanded\expandafter{\csname ekvc@splitmark@6\endcsname}##6%
        \unexpanded\expandafter{\csname ekvc@splitmark@7\endcsname}##7%
        ##8##9%
      {##9{##1}{##2}{##3}{##4}{##5}{##6}{##7}##8}%
  }
\ekvc@tmp
\endgroup
\protected\def\ekvc@setup@splitmacro#1%
  {%
    \ekv@ifdefined{ekvc@split@#1}{}%
      {%
        \begingroup
          \edef\ekvc@tmp
            {%
              \long\gdef
                  \unexpanded\expandafter{\csname ekvc@split@#1\endcsname}%
                  ####1%
                  \unexpanded\expandafter{\csname ekvc@splitmark@#1\endcsname}%
                  ####2####3%
                {%
                  \unexpanded\expandafter
                    {\csname ekvc@split@\the\numexpr#1-1\relax\endcsname}%
                  ####1{{####2}####3}%
                }%
            }%
          \ekvc@tmp
        \endgroup
      }%
  }
%    \end{macrocode}
% \end{macro}
%
% \begin{macro}{\ekvcHashAndUse}
%   |\ekvcHashAndUse| works just like |\ekvcSplitAndUse|.
%    \begin{macrocode}
\protected\long\def\ekvcHashAndUse#1#2%
  {%
    \let\ekvc@helpers@needed\@firstoftwo
    \ekvc@ifdefined#1%
      {\ekvc@err@already@defined#1}%
      {\ekvcHashAndUse@#1{}{#2}}%
  }
%    \end{macrocode}
% \end{macro}
%
% \begin{macro}[internal]{\ekvcHashAndUse@}
%   This is more or less the same as |\ekvcSplitAndUse@|. Instead of an empty
%   group we place a marker after the initials, we don't use the sorting macros
%   of |split|, but instead pack all the values in one argument.
%    \begin{macrocode}
\protected\long\def\ekvcHashAndUse@#1#2#3%
  {%
    \edef\ekvc@set{\string#1}%
    \ekvc@SetupHashKeys{#3}%
    \ekvc@helpers@needed
      {%
        \ekvc@any@long\edef#1##1%
          {%
            \expandafter\ekvc@ekvset@pre@expander\expandafter{\ekvc@set}%
            \unexpanded{\ekvc@hash@pack@argument}%
            \unexpanded\expandafter{\ekvc@initials\ekvc@stop#2}%
          }%
      }%
      {%
        \ekvc@any@long\edef#1##1%
          {%
            \expandafter\ekvc@ekvset@pre@expander\expandafter{\ekvc@set}%
            \unexpanded{#2}%
            \unexpanded\expandafter{\ekvc@initials\ekvc@stop}%
          }%
      }%
  }
%    \end{macrocode}
% \end{macro}
%
% \begin{macro}{\ekvcHashAndForward}
%   |\ekvcHashAndForward| works just like |\ekvcSplitAndForward|.
%    \begin{macrocode}
\protected\long\def\ekvcHashAndForward#1#2#3%
  {%
    \let\ekvc@helpers@needed\@firstoftwo
    \ekvc@ifdefined#1%
      {\ekvc@err@already@defined#1}%
      {\ekvcHashAndUse@#1{{#2}}{#3}}%
  }
%    \end{macrocode}
% \end{macro}
%
% \begin{macro}{\ekvcHash}
%   |\ekvcHash| does the same as |\ekvcSplit|, but has the advantage of not
%   needing to count arguments, so the definition of the internal macro is a bit
%   more straight forward.
%    \begin{macrocode}
\protected\long\def\ekvcHash#1#2#3%
  {%
    \let\ekvc@helpers@needed\@secondoftwo
    \ekvc@ifdefined#1%
      {\ekvc@err@already@defined#1}%
      {%
        \expandafter
        \ekvcHashAndUse@\expandafter#1\csname ekvc@\string#1\endcsname{#2}%
        \ekvc@any@long\expandafter\def\csname ekvc@\string#1\endcsname
            ##1\ekvc@stop
          {#3}%
      }%
  }
%    \end{macrocode}
% \end{macro}
%
% \begin{macro}[internal]{\ekvc@hash@pack@argument}
%   All this macro does is pack the values into one argument and forward that to
%   the next macro.
%    \begin{macrocode}
\long\def\ekvc@hash@pack@argument#1\ekvc@stop#2{#2{#1}}
%    \end{macrocode}
% \end{macro}
%
% \begin{macro}[internal]
%   {\ekvc@SetupHashKeys,\ekvc@SetupHashKeys@a,\ekvc@SetupHashKeys@b}
%   This should look awfully familiar as well, since it's just the same as for
%   the split keys with a few other names here and there.
%    \begin{macrocode}
\protected\long\def\ekvc@SetupHashKeys#1%
  {%
    \let\ekvc@any@long\ekv@empty
    \let\ekvc@initials\ekv@empty
    \ekvparse\ekvc@err@value@required\ekvc@SetupHashKeys@a{#1}%
  }
\protected\def\ekvc@SetupHashKeys@a#1%
  {%
    \let\ekvc@long\ekv@empty
    \ekvc@ifspace{#1}%
      {\ekvc@SetupHashKeys@b#1\ekvc@stop}%
      {\ekvc@SetupHashKeys@c{#1}}%
  }
\protected\def\ekvc@SetupHashKeys@b#1 #2\ekvc@stop
  {%
    \ekv@ifdefined{ekvc@hash@p@#1}%
      {\csname ekvc@hash@p@#1\endcsname{#2}}%
      {\ekvc@SetupHashKeys@c{#1 #2}}%
  }
%    \end{macrocode}
%   Yes, even the defining macro looks awfully familiar. Instead of numbered we
%   have named marks. Still the key macros grab everything up to their
%   respective mark and reorder the arguments. The same quirk is applied for
%   short keys. And instead of the |\ekvc@setup@splitmacro| we use
%   |\ekvc@setup@hashmacro|.
%    \begin{macrocode}
\protected\long\def\ekvc@SetupHashKeys@c#1#2%
  {%
    \begingroup
    \edef\ekvc@tmp
      {%
        \endgroup
        \long\def\unexpanded{\ekvc@tmp}####1####2%
            \unexpanded\expandafter{\csname ekvc@hashmark@#1\endcsname}####3%
          {%
            ####2%
            \unexpanded\expandafter{\csname ekvc@hashmark@#1\endcsname}{####1}%
          }%
        \unless\ifx\ekvc@long\long
          \let\unexpanded\expandafter
            {\csname ekvc@\ekvc@set(#1)\endcsname\ekvc@tmp}%
          \def\unexpanded{\ekvc@tmp}####1%
            {%
              \unexpanded\expandafter{\csname ekvc@\ekvc@set(#1)\endcsname}%
                {####1}%
            }%
        \fi
        \def\unexpanded{\ekvc@initials}%
          {%
            \unexpanded\expandafter{\ekvc@initials}%
            \unexpanded\expandafter{\csname ekvc@hashmark@#1\endcsname{#2}}%
          }%
      }%
    \ekvc@tmp
    \ekvlet\ekvc@set{#1}\ekvc@tmp
    \ekvc@setup@hashmacro{#1}%
  }
%    \end{macrocode}
% \end{macro}
%
% \begin{macro}[internal]{\ekvc@hash@p@long}
%   Nothing astonishing here either.
%    \begin{macrocode}
\protected\def\ekvc@hash@p@long
  {%
    \let\ekvc@long\long
    \let\ekvc@any@long\long
    \ekvc@SetupHashKeys@c
  }
%    \end{macrocode}
% \end{macro}
%
% \begin{macro}[internal]{\ekvc@setup@hashmacro}
%   The safe hash macros will be executed inside of an |\unexpanded| expansion
%   context, so they have to insert braces for that once they are done.
%   Most of the tests which have to be executed will already be done, but we
%   have to play safe if the hash doesn't show up in the hash list. Therefore we
%   use some |\ekvc@mark|s and |\ekvc@stop| to throw errors if the hash isn't
%   found in the right place. The fast variants have an easier life and just
%   return the correct value.
%    \begin{macrocode}
\protected\def\ekvc@setup@hashmacro#1%
  {%
    \ekv@ifdefined{ekvc@fasthash@#1}{}%
      {%
        \begingroup
          \edef\ekvc@tmp
            {%
              \long\gdef
                \unexpanded\expandafter{\csname ekvc@fasthash@#1\endcsname}%
                  ####1%
                  \unexpanded\expandafter{\csname ekvc@hashmark@#1\endcsname}%
                  ####2####3\unexpanded{\ekvc@stop}%
                {####2}%
              \long\gdef
                \unexpanded\expandafter{\csname ekvc@safehash@#1\endcsname}%
                  ####1%
                {%
                  \unexpanded\expandafter{\csname ekvc@@safehash@#1\endcsname}%
                  ####1\unexpanded{\ekvc@mark}{}%
                  \unexpanded\expandafter
                    {%
                      \csname ekvc@hashmark@#1\endcsname{}%
                      \ekvc@mark{\ekvc@err@missing@hash{#1}}\ekvc@stop
                    }%
                }%
              \long\gdef
                \unexpanded\expandafter{\csname ekvc@@safehash@#1\endcsname}%
                  ####1%
                  \unexpanded\expandafter{\csname ekvc@hashmark@#1\endcsname}%
                  ####2####3\unexpanded{\ekvc@mark}####4####5%
                  \unexpanded{\ekvc@stop}%
                {%
                  ####4{####2}%
                }%
              \long\gdef\unexpanded\expandafter
                {\csname ekvc@fastsplithash@#1\endcsname}%
                  ####1%
                  \unexpanded\expandafter{\csname ekvc@hashmark@#1\endcsname}%
                  ####2####3\unexpanded{\ekvc@stop}####4%
                {%
                  ####4{####1####3}{####2}%
                }%
              \long\gdef\unexpanded\expandafter
                {\csname ekvc@safesplithash@#1\endcsname}####1%
                {%
                  \unexpanded\expandafter
                    {\csname ekvc@@safesplithash@#1\endcsname}%
                  ####1\unexpanded{\ekvc@mark\ekvc@safe@found@hash}%
                  \unexpanded\expandafter
                    {%
                      \csname ekvc@hashmark@#1\endcsname{}%
                      \ekvc@mark{\ekvc@err@missing@hash{#1}\ekvc@safe@no@hash}%
                      \ekvc@stop
                    }%
                }%
              \long\gdef\unexpanded\expandafter
                {\csname ekvc@@safesplithash@#1\endcsname}%
                  ####1%
                  \unexpanded\expandafter{\csname ekvc@hashmark@#1\endcsname}%
                  ####2####3\unexpanded{\ekvc@mark}####4####5%
                  \unexpanded{\ekvc@stop}%
                {%
                  ####4{####2}####1####3\unexpanded{\ekvc@stop}%
                }%
            }%
          \ekvc@tmp
        \endgroup
      }%
  }
%    \end{macrocode}
% \end{macro}
%
% \begin{macro}{\ekvcValue}
%   All this does is a few consistency checks on the first argument (not empty,
%   hash macro exists) and then call that hash-grabbing macro that will also
%   test whether the hash is inside of |#2| or not.
%    \begin{macrocode}
\long\def\ekvcValue#1%
  {%
    \unexpanded
    \ekv@ifdefined{ekvc@safehash@#1}%
      {\csname ekvc@safehash@#1\endcsname}%
      {\ekvc@err@unknown@hash{#1}\@firstoftwo{{}}}%
  }
%    \end{macrocode}
% \end{macro}
%
% \begin{macro}{\ekvcValueFast}
%   To be as fast as possible, this doesn't test for anything, assuming the user
%   knows best.
%    \begin{macrocode}
\long\def\ekvcValueFast#1#2{\csname ekvc@fasthash@#1\endcsname#2\ekvc@stop}
%    \end{macrocode}
% \end{macro}
%
% \begin{macro}{\ekvcValueSplit,\ekvcValueSplit@recover}
%   This splits off a single value.
%    \begin{macrocode}
\long\def\ekvcValueSplit#1%
  {%
    \ekv@ifdefined{ekvc@safesplithash@#1}%
      {\csname ekvc@safesplithash@#1\endcsname}%
      {\ekvc@err@unknown@hash{#1}\ekvcValueSplit@recover}%
  }
\long\def\ekvcValueSplit@recover#1#2%
  {%
    #2{#1}{}%
  }
%    \end{macrocode}
% \end{macro}
%
% \begin{macro}[internal]{\ekvc@safe@found@hash, \ekvc@safe@no@hash}
%    \begin{macrocode}
\long\def\ekvc@safe@found@hash#1#2\ekvc@stop#3%
  {%
    #3{#2}{#1}%
  }
\long\def\ekvc@safe@no@hash#1#2\ekvc@mark\ekvc@safe@found@hash\ekvc@stop#3%
  {%
    #3{#2}{}%
  }
%    \end{macrocode}
% \end{macro}
%
% \begin{macro}{\ekvcValueSplitFast}
%   Again a fast approach which doesn't provide too many safety measurements.
%   This needs to build the hash function and expand it before passing the
%   results to the next control sequence. The first step only builds the control
%   sequence.
%    \begin{macrocode}
\long\def\ekvcValueSplitFast#1#2%
  {%
    \csname ekvc@fastsplithash@#1\endcsname#2\ekvc@stop
  }
%    \end{macrocode}
% \end{macro}
%
% \begin{macro}[internal]
%   {
%     \ekvc@safehash@,\ekvc@fasthash@,
%     \ekvc@safesplithash@,\ekvc@fastsplithash@
%   }
%   At least in the empty hash case we can provide a meaningful error message
%   without affecting performance by just defining the macro that would be build
%   in that case. There is of course a downside to this, the error will not be
%   thrown by |\ekvcValueFast| in three expansion steps. The safe hash variant
%   has to also stop the |\unexpanded| expansion.
%    \begin{macrocode}
\long\def\ekvc@safehash@#1{\ekvc@err@empty@hash{}}
\long\def\ekvc@fasthash@#1\ekvc@stop{\ekvc@err@empty@hash}
\long\def\ekvc@safesplithash@#1#2{\ekvc@err@empty@hash#2{#1}{}}
\long\def\ekvc@fastsplithash@#1\ekvc@stop#2{\ekvc@err@empty@hash#2{#1}{}}
%    \end{macrocode}
% \end{macro}
%
% \begin{macro}{\ekvcSecondaryKeys}
%   The secondary keys are defined pretty similar to the way the originals are,
%   but here we also introduce some key types (those have a |@t@| in their name)
%   additionally to the prefixes.
%    \begin{macrocode}
\protected\long\def\ekvcSecondaryKeys#1#2%
  {%
    \edef\ekvc@set{\string#1}%
    \ekvparse\ekvc@err@value@required\ekvcSecondaryKeys@a{#2}%
  }
\protected\def\ekvcSecondaryKeys@a#1%
  {%
    \let\ekvc@long\ekv@empty
    \ekvc@ifspace{#1}%
      {\ekvcSecondaryKeys@b#1\ekvc@stop}%
      {\ekvc@err@missing@type{#1}\@gobble}%
  }
\protected\def\ekvcSecondaryKeys@b#1 #2\ekvc@stop
  {%
    \ekv@ifdefined{ekvc@p@#1}%
      {\csname ekvc@p@#1\endcsname}%
      {%
        \ekv@ifdefined{ekvc@t@#1}%
          {\csname ekvc@t@#1\endcsname}%
          {\ekvc@err@unknown@keytype{#1}\@firstoftwo\@gobble}%
      }%
      {#2}%
  }
%    \end{macrocode}
% \end{macro}
%
% \begin{macro}{\ekvcChange}
%   This can be used to change the defaults of an \expkvc\ defined macro. It
%   checks whether there is a set with the correct name and that the macro is
%   defined. If both is true the real work is done by |\ekvc@change|.
%    \begin{macrocode}
\protected\long\def\ekvcChange#1%
  {%
    \ekvifdefinedset{\string#1}%
      {%
        \ekvc@ifdefined#1%
          {\ekvc@change#1}%
          {\ekvc@err@no@key@macro#1\@gobble}%
      }%
      {\ekvc@err@no@key@macro#1\@gobble}%
  }
%    \end{macrocode}
% \end{macro}
%
% \begin{macro}[internal]
%   {\ekvc@change,\ekvc@change@a,\ekvc@change@b,\ekvc@change@c}
%   First we need to see whether the macro is currently |\long|. For this we get
%   the meaning and will parse it. |#1| is the macro name in which we want to
%   change the defaults.
%    \begin{macrocode}
\protected\def\ekvc@change#1%
  {\expandafter\ekvc@change@a\meaning#1\ekv@stop#1}
%    \end{macrocode}
%   A temporary definition to get the stringified |macro:|. |##1| will be the
%   list of prefixes, we don't care for the exact contents of |##2| and |##3|.
%    \begin{macrocode}
\def\ekvc@change@a#1%
  {%
    \protected\def\ekvc@change@a##1#1##2->##3\ekv@stop
      {%
        \ekvc@change@iflong{##1}%
          {\ekvc@change@b{}}%
          {\ekvc@change@b{\long}}%
      }%
  }
\expandafter\ekvc@change@a\expandafter{\detokenize{macro:}}
%    \end{macrocode}
%   Next we expand the macro once to get its contents (including the current
%   default values with their markers). |#1| is either |\long| or empty, |#2| is
%   the macro.
%    \begin{macrocode}
\protected\def\ekvc@change@b#1#2%
  {\expandafter\ekvc@change@c\expandafter{#2{##1}}{#1}#2}
%    \end{macrocode}
%   Here we place an unbalanced closing brace after the expansion of the macro.
%   Then we just parse the \kv-list with |\ekvset|, that will exchange the
%   values behind the markers. Once those are changed we give control to
%   |\ekvc@change@d|. The |\ekvset| step might horribly fail if the user defined
%   some keys that don't behave nice. |#1| is the expansion of the macro, |#2|
%   is either |\long| or empty, |#3| is the macro, and |#4| is the \kv-list
%   containing the new defaults.
%    \begin{macrocode}
\ekv@exparg{\protected\long\def\ekvc@change@c#1#2#3#4}%
  {%
    \expandafter\iffalse\expandafter{\expandafter\fi
      \ekvset{\string#3}{#4}%
      \ekvc@change@d{#2}{#3}%
      #1%
    }%
  }
%    \end{macrocode}
%   The final step needs to put an unbalanced opening brace after |\def|. We do
%   that with the help of a temporary macro which stores everything necessary
%   for |\def| and expand an |\iffalse}\fi| construct to leave a single
%   opening brace. |#1| will be either empty or |\long| and |#2| is the macro.
%   Each of the macros defined with \expkvc\ takes exactly one parameter, so we
%   put that here as |####1|.
%    \begin{macrocode}
\protected\def\ekvc@change@d#1#2%
  {%
    \def\ekvc@tmp{#1\def#2####1}%
    \expandafter\ekvc@tmp\expandafter{\iffalse}\fi
  }
%    \end{macrocode}
% \end{macro}
%
% \begin{macro}[internal]{\ekvc@change@iflong,\ekvc@change@iflong@}
%   Checking whether a string contains the string representation of |\long| can
%   be done by gobbling everything up to the first |\long| and checking whether
%   the result is completely empty. We need a temporary macro to get the result
%   of |\string\long| inside the definitions.
%    \begin{macrocode}
\def\ekvc@change@iflong#1%
  {%
    \protected\def\ekvc@change@iflong##1%
      {\expandafter\ekv@ifempty\expandafter{\ekvc@change@iflong@##1#1}}%
    \def\ekvc@change@iflong@##1#1{}
  }
\expandafter\ekvc@change@iflong\expandafter{\string\long}
%    \end{macrocode}
% \end{macro}
%
% \begin{macro}{\ekvcPass}
%   This macro can be used to pass a value to a key of some macro (this way
%   more complicated key codes are possible that in the end pass processed
%   values on to some macro). The implemantation is pretty straight forward.
%    \begin{macrocode}
\long\def\ekvcPass#1#2%
  {%
    \ekvifdefined{\string#1}{#2}%
      {\csname\ekv@name{\string#1}{#2}\endcsname}%
      {\ekvc@err@unknown@key@or@macro{#1}{#2}\@gobble}%
  }
%    \end{macrocode}
% \end{macro}
%
%
% \subsubsection{Secondary Key Types}
%
% \begin{macro}[internal]{\ekvc@p@long,\ekvc@after@ptype}
%   The prefixes are pretty straight forward again. Just set |\ekvc@long| and
%   forward to the |@t@| type.
%    \begin{macrocode}
\protected\def\ekvc@p@long#1%
  {%
    \ekvc@ifspace{#1}%
      {%
        \let\ekvc@long\long
        \ekvc@after@ptype#1\ekvc@stop
      }%
      {\ekvc@err@missing@type{long #1}\@gobble}%
  }
\protected\def\ekvc@after@ptype#1 #2\ekvc@stop
  {%
    \ekv@ifdefined{ekvc@t@#1}%
      {\csname ekvc@t@#1\endcsname{#2}}%
      {\ekvc@err@unknown@keytype{#1}\@gobble}%
  }
%    \end{macrocode}
% \end{macro}
%
% \begin{macro}[internal]
%   {
%     \ekvc@t@meta, \ekvc@t@nmeta,
%     \ekvc@type@meta, \ekvc@type@meta@a, \ekvc@type@meta@b
%   }
%   The |meta| and |nmeta| key types use a nested |\ekvset| to set other keys in
%   the same macro's \set.
%    \begin{macrocode}
\protected\def\ekvc@t@meta
  {%
    \edef\ekvc@tmp{\ekvc@set}%
    \expandafter\ekvc@type@meta\expandafter{\ekvc@tmp}\ekvc@long{##1}\ekvlet
  }
\protected\def\ekvc@t@nmeta#1%
  {%
    \ekvc@assert@not@long{nmeta #1}%
    \edef\ekvc@tmp{\ekvc@set}%
    \expandafter\ekvc@type@meta\expandafter{\ekvc@tmp}{}{}\ekvletNoVal{#1}%
  }
\protected\long\def\ekvc@type@meta#1#2#3#4#5#6%
  {%
    \expandafter\ekvc@type@meta@a\expandafter{\ekvset{#1}{#6}}{#2}{#3}%
    #4\ekvc@set{#5}\ekvc@tmp
  }
\protected\def\ekvc@type@meta@a
  {%
    \expandafter\ekvc@type@meta@b\expandafter
  }
\protected\long\def\ekvc@type@meta@b#1#2#3%
  {%
    #2\def\ekvc@tmp#3{#1}%
  }
%    \end{macrocode}
% \end{macro}
%
% \begin{macro}[internal]{\ekvc@t@alias}
%   |alias| just checks whether there is a key and/or |NoVal| key defined with
%   the target name and |\let| the key to those.
%    \begin{macrocode}
\protected\def\ekvc@t@alias#1#2%
  {%
    \ekvc@assert@not@long{alias #1}%
    \let\ekvc@tmp\@firstofone
    \ekvifdefined\ekvc@set{#2}%
      {%
        \ekvletkv\ekvc@set{#1}\ekvc@set{#2}%
        \let\ekvc@tmp\@gobble
      }%
      {}%
    \ekvifdefinedNoVal\ekvc@set{#2}%
      {%
        \ekvletkvNoVal\ekvc@set{#1}\ekvc@set{#2}%
        \let\ekvc@tmp\@gobble
      }%
      {}%
    \ekvc@tmp{\ekvc@err@unknown@key{#2}}%
  }
%    \end{macrocode}
% \end{macro}
%
% \begin{macro}[internal]{\ekvc@t@default}
%   The |default| key can be used to set a |NoVal| key for an existing key. It
%   will just pass the \val\ to the key macro of that other key.
%    \begin{macrocode}
\protected\long\def\ekvc@t@default#1#2%
  {%
    \ekvifdefined\ekvc@set{#1}%
      {%
        \ekvc@assert@not@long{default #1}%
        \edef\ekvc@tmp
          {%
            \unexpanded\expandafter
              {\csname\ekv@name\ekvc@set{#1}\endcsname{#2}}%
          }%
        \ekvletNoVal\ekvc@set{#1}\ekvc@tmp
      }%
      {\ekvc@err@unknown@key{#1}}%
  }
%    \end{macrocode}
% \end{macro}
%
% \begin{macro}[internal]{\ekvc@t@flag-bool}
%    \begin{macrocode}
\protected\expandafter\def\csname ekvc@t@flag-bool\endcsname#1#2%
  {%
    \ekvc@assert@not@long{flag-bool #1}%
    \unless\ifdefined#2\ekvcFlagNew#2\fi
    \ekvdef\ekvc@set{#1}%
      {%
        \ekv@ifdefined{ekvc@flag@set@##1}%
          {%
            \csname ekvc@flag@set@##1\expandafter\endcsname
              \ekvcFlagHeight#2\ekv@stop#2%
          }%
          {\ekvc@err@invalid@bool{##1}}%
      }%
  }
%    \end{macrocode}
% \end{macro}
%
% \begin{macro}[internal]
%   {\ekvc@t@flag-true,\ekvc@t@flag-false,\ekvc@t@flag-raise,\ekvc@type@flag}
%    \begin{macrocode}
\protected\def\ekvc@type@flag#1#2#3#4%
  {%
    \ekvc@assert@not@long{flag-#1 #3}%
    \unless\ifdefined#4\ekvcFlagNew#4\fi
    \ekv@exparg{\ekvdefNoVal\ekvc@set{#3}}{#2#4}%
  }
\protected\expandafter\def\csname ekvc@t@flag-true\endcsname
  {\ekvc@type@flag{true}\ekvcFlagSetTrue}
\protected\expandafter\def\csname ekvc@t@flag-false\endcsname
  {\ekvc@type@flag{false}\ekvcFlagSetFalse}
\protected\expandafter\def\csname ekvc@t@flag-raise\endcsname
  {\ekvc@type@flag{raise}\ekvcFlagRaise}
%    \end{macrocode}
% \end{macro}
%
%
% \subsubsection{Flags}
%
% The basic idea of flags is to store information by the fact that \TeX\
% expandably assigns the meaning |\relax| to undefined control sequences which
% were built with |\csname|. This mechanism is borrowed from \pkg{expl3}.
%
% \begin{macro}[internal]{\ekvc@flag@name,\ekvc@flag@namescheme}
%   Flags follow a simple naming scheme which we define here. |\ekvc@flag@name|
%   will store the name of an internal function that is used to build names of
%   the second naming scheme defined by |\ekvc@flag@namescheme|.
%    \begin{macrocode}
\def\ekvc@flag@name{ekvcf\string}
\def\ekvc@flag@namescheme#1#2{ekvch#2#1}
%    \end{macrocode}
% \end{macro}
%
% \begin{macro}{\ekvcFlagHeight}
%   For semantic reasons we use |\number| with another name.
%    \begin{macrocode}
\let\ekvcFlagHeight\number
%    \end{macrocode}
% \end{macro}
%
% \begin{macro}{\ekvcFlagNew}
%   This macro defines a new flag. It stores the function build with the
%   |\ekvc@flag@name| naming scheme after the internal function
%   |\ekvc@flag@height| that'll determine the current flag height. It'll also
%   define the macro named via |\ekvc@flag@name| to build names according to
%   |\ekvc@flag@namescheme|.
%    \begin{macrocode}
\protected\def\ekvcFlagNew#1%
  {%
    \edef#1%
      {%
        \unexpanded{\ekvc@flag@height}%
          \unexpanded\expandafter{\csname\ekvc@flag@name#1\endcsname}%
      }%
    \ekv@expargtwice
      {\expandafter\def\csname\ekvc@flag@name#1\endcsname##1}%
      {\expandafter\ekvc@flag@namescheme\expandafter{\string#1}{##1}}%
  }
%    \end{macrocode}
% \end{macro}
%
% \begin{macro}[internal]{\ekvc@flag@height,\ekvc@flag@height@}
%   This macro gets the height of a flag by a simple loop. The first loop
%   iteration differs a bit from the following in that it doesn't have to get
%   the current iteration count. The space at the end of |\ekvc@flag@height|
%   ends the |\number| evaluation.
%    \begin{macrocode}
\def\ekvc@flag@height#1%
  {%
    \ifcsname#10\endcsname
      \ekvc@flag@height@1\ekv@stop#1%
    \fi
    \@firstofone{0} % leave this space
  }
\def\ekvc@flag@height@#1\ekv@stop#2\fi\@firstofone#3%
  {%
    \fi
    \ifcsname#2{#1}\endcsname
      \expandafter\ekvc@flag@height@\the\numexpr#1+1\relax\ekv@stop#2%
    \fi
    \@firstofone{#1}%
  }
%    \end{macrocode}
% \end{macro}
%
% \begin{macro}{\ekvcFlagRaise}
%   Raising a flag simply means letting the |\ekvc@flag@namescheme| macro for
%   the current height to relax. The result of raising a flag is that its height
%   is bigger by $1$.
%    \begin{macrocode}
\ekv@exparg{\def\ekvcFlagRaise#1}%
  {%
    \expandafter\expandafter\expandafter\@gobble\expandafter
      \csname\ekvc@flag@namescheme{\string#1}{\ekvcFlagHeight#1}\endcsname
  }
%    \end{macrocode}
% \end{macro}
%
% \begin{macro}{\ekvcFlagSetTrue,\ekvcFlagSetFalse}
% \begin{macro}[internal]{\ekvc@flag@set@true,\ekvc@flag@set@false}
%   A flag is considered true if its current height is odd, and as false if it
%   is even. Therefore |\ekvcFlagSetTrue| and |\ekvcFlagSetFalse| only need to
%   raise the flag if the opposing boolean value is the current one.
%    \begin{macrocode}
\def\ekvcFlagSetTrue#1%
  {\expandafter\ekvc@flag@set@true\ekvcFlagHeight#1\ekv@stop#1}
\def\ekvcFlagSetFalse#1%
  {\expandafter\ekvc@flag@set@false\ekvcFlagHeight#1\ekv@stop#1}
%    \end{macrocode}
%   We can expand |\ekvc@flag@namescheme| at definition time here, which is why
%   we're using a temporary definition to set up |\ekvc@flag@set@true| and
%   |\ekvc@flag@set@false|.
%    \begin{macrocode}
\def\ekvc@flag@set@true#1%
  {%
    \def\ekvc@flag@set@true##1\ekv@stop##2%
      {%
        \ifodd##1
          \ekv@fi@gobble
        \fi
        \@firstofone{\expandafter\@gobble\csname#1\endcsname}%
      }%
    \def\ekvc@flag@set@false##1\ekv@stop##2%
      {%
        \ifodd##1
          \ekv@fi@firstofone
        \fi
        \@gobble{\expandafter\@gobble\csname#1\endcsname}%
      }%
  }
\expandafter\ekvc@flag@set@true\expandafter
  {\ekvc@flag@namescheme{\string#2}{#1}}
%    \end{macrocode}
% \end{macro}
% \end{macro}
%
% \begin{macro}{\ekvcFlagIf}
%   As already explained, truthiness is defined as a flag's height is odd, so we
%   just branch accordingly here.
%    \begin{macrocode}
\def\ekvcFlagIf#1%
  {%
    \ifodd#1%
      \ekv@fi@firstoftwo
    \fi
    \@secondoftwo
  }
%    \end{macrocode}
% \end{macro}
%
% \begin{macro}{\ekvcFlagIfRaised}
%   This macro uses flags as a switch, if a flag's current height is bigger than
%   $0$ this test yields true.
%    \begin{macrocode}
\ekv@exparg{\def\ekvcFlagIfRaised#1}%
  {%
    \expandafter\ifcsname\ekvc@flag@namescheme{\string#1}0\endcsname
      \ekv@fi@firstoftwo
    \fi
    \@secondoftwo
  }
%    \end{macrocode}
% \end{macro}
%
% \begin{macro}{\ekvcFlagReset}
% \begin{macro}[internal]{\ekvc@flag@reset,\ekvc@flag@reset@}
%   Resetting works by locally letting all the defined internal macros named
%   after |\ekvc@flag@namescheme| to undefined.
%    \begin{macrocode}
\protected\def\ekvcFlagReset#1%
  {\expandafter\ekvc@flag@reset\csname\ekvc@flag@name#1\endcsname}
\protected\def\ekvc@flag@reset#1%
  {%
    \ifcsname#10\endcsname
      \expandafter\let\csname#10\endcsname\ekvc@undefined
      \ekvc@flag@reset@1\ekv@stop#1%
    \fi
  }
\protected\def\ekvc@flag@reset@#1\ekv@stop#2\fi
  {%
    \fi
    \ifcsname#2{#1}\endcsname
      \expandafter\let\csname#2{#1}\endcsname\ekvc@undefined
      \expandafter\ekvc@flag@reset@\the\numexpr#1+1\relax\ekv@stop#2%
    \fi
  }
%    \end{macrocode}
% \end{macro}
% \end{macro}
%
% \begin{macro}{\ekvcFlagGetHeight}
% \begin{macro}[internal]{\ekvc@flag@get@height@single}
%   These are just small helpers, first getting the height of the flag and then
%   passing it on to the user supplied code.
%    \begin{macrocode}
\def\ekvcFlagGetHeight#1%
  {\expandafter\ekvc@flag@get@height@single\ekvcFlagHeight#1\ekv@stop}
\long\def\ekvc@flag@get@height@single#1\ekv@stop#2{#2{#1}}
%    \end{macrocode}
% \end{macro}
% \end{macro}
%
% \begin{macro}{\ekvcFlagGetHeights}
% \begin{macro}[internal]
%   {\ekvc@flag@get@heights,\ekvc@flag@get@heights@,\ekvc@flag@get@heights@done}
%   This works by a simple loop that stops at |\ekv@stop|. As long as that
%   marker isn't hit, get the next flags height and put it into a list after
%   |\ekv@stop|. |\ekvc@flag@get@heights@| uses the same marker name for the
%   end of the height, which shouldn't clash in any case. Once we're done we
%   remove the remainder of the current iteration and leave the user supplied
%   code in the input stream with all the flags' heights as a single argument.
%    \begin{macrocode}
\def\ekvcFlagGetHeights#1%
  {%
    \ekvc@flag@get@heights#1\ekv@stop{}%
  }
\def\ekvc@flag@get@heights#1%
  {%
    \ekv@gobbleto@stop#1\ekvc@flag@get@heights@done\ekv@stop
    \expandafter\ekvc@flag@get@heights@\ekvcFlagHeight#1\ekv@stop
  }
\def\ekvc@flag@get@heights@#1\ekv@stop#2\ekv@stop#3%
  {\ekvc@flag@get@heights#2\ekv@stop{#3{#1}}}
\long\def\ekvc@flag@get@heights@done
    \ekv@stop
    \expandafter\ekvc@flag@get@heights@\ekvcFlagHeight\ekv@stop\ekv@stop#1#2%
  {#2{#1}}
%    \end{macrocode}
% \end{macro}
% \end{macro}
%
%
% \subsubsection{Helper Macros}
%
% \begin{macro}[internal]{\ekvc@ifspace,\ekvc@ifspace@}
% A test which can be reduced to an if-empty by gobbling everything up to the
% first space.
%    \begin{macrocode}
\long\def\ekvc@ifspace#1%
  {%
    \ekvc@ifspace@#1 \ekv@ifempty@B
      \ekv@ifempty@false\ekv@ifempty@A\ekv@ifempty@B\@firstoftwo
  }
\long\def\ekvc@ifspace@#1 % keep this space
  {%
    \ekv@ifempty@\ekv@ifempty@A
  }
%    \end{macrocode}
% \end{macro}
%
%
% \subsubsection{Assertions}
% \begin{macro}[internal]{\ekvc@assert@not@long}
%   Some keys don't want to be |long| and we have to educate the user, so let's
%   throw an error if someone wanted these to be long.
%    \begin{macrocode}
\long\def\ekvc@assert@not@long#1{\ifx\ekvc@long\long\ekvc@err@no@long{#1}\fi}
%    \end{macrocode}
% \end{macro}
%
%
% \subsubsection{Messages}
%
% \begin{macro}[internal]
%   {
%     \ekvc@err@toomany,\ekvc@err@value@required,\ekvc@err@missing@type,
%     \ekvc@err@already@defined,\ekvc@err@no@key@macro
%   }
%   Boring unexpandable error messages.
%    \begin{macrocode}
\protected\def\ekvc@err@toomany#1%
  {%
    \errmessage{expkv-cs Error: Too many keys for macro `\string#1'}%
  }
\protected\def\ekvc@err@value@required#1%
  {%
    \errmessage{expkv-cs Error: Missing value for key `\unexpanded{#1}'}%
  }
\protected\def\ekvc@err@missing@type#1%
  {%
    \errmessage
      {expkv-cs Error: Missing type for secondary key `\unexpanded{#1}'}%
  }
\protected\def\ekvc@err@no@long#1%
  {%
    \errmessage
      {expkv-cs Error: prefix `long' not accepted for `\unexpanded{#1}'}%
  }
\protected\def\ekvc@err@already@defined#1%
  {%
    \errmessage{expkv-cs Error: Macro `\string#1' already defined}%
  }
\protected\def\ekvc@err@unknown@keytype#1%
  {%
    \errmessage{expkv-cs Error: Unknown key type `\unexpanded{#1}'}%
  }
\protected\def\ekvc@err@unknown@key#1%
  {%
    \errmessage
      {expkv-cs Error: Unknown key `\unexpanded{#1}' for macro `\ekvc@set'}%
  }
\long\def\ekvc@err@no@key@macro#1%
  {\errmessage{expkv-cs Error: \string#1 is no key=val macro}}
%    \end{macrocode}
% \end{macro}
%
% \begin{macro}[internal]{\ekvc@err}
%   We need a way to throw error messages expandably in some contexts.
%    \begin{macrocode}
\ekv@exparg{\long\def\ekvc@err#1}{\ekverr{expkv-cs}{#1}}
%    \end{macrocode}
% \end{macro}
%
% \begin{macro}[internal]
%   {
%     \ekvc@err@unknown@hash,\ekvc@err@empty@hash,\ekvc@err@missing@hash,
%     \ekvc@err@invalid@bool
%   }
%   And here are the expandable error messages.
%    \begin{macrocode}
\long\def\ekvc@err@unknown@hash#1{\ekvc@err{unknown hash `#1'}}
\long\def\ekvc@err@missing@hash#1{\ekvc@err{hash `#1' not found}}
\long\def\ekvc@err@empty@hash{\ekvc@err{empty hash}}
\def\ekvc@err@invalid@bool#1{\ekvc@err{invalid boolean value `#1'}}
\long\def\ekvc@err@unknown@key@or@macro#1#2%
  {\ekvc@err{unknown key `#2' for macro #1}}
%    \end{macrocode}
% \end{macro}
%
%
% Now everything that's left is to reset the category code of |@|.
%    \begin{macrocode}
\catcode`\@=\ekvc@tmp
%    \end{macrocode}
%
% \gobbledocstriptag
%</tex>
%^^A=<<
%
%^^A=<<
%
% \end{implementation}^^A=<<
%
% \clearpage
% \PrintIndex
%
\endinput
%
^^A vim: ft=tex fdm=marker fmr=>>=,=<<

\ProvidesPackage{expkv-cs}%
  [%
    \ekvcDate\space v\ekvcVersion\space
    define expandable key=val macros using expkv%
  ]
%    \end{macrocode}
% \gobbledocstriptag
%</pkg>
%^^A=<<
%^^A main file >>=
% \subsection{The Generic Code}
% The rest of this implementation will be the generic code.
% \gobbledocstriptag
%<*tex>
%
% Load \expkv\ if the package didn't already do so -- since \expkv\ has
% safeguards against being loaded twice this does no harm and the overhead
% isn't that big. Also we reuse some of the internals of \expkv\ to save us from
% retyping them.
%    \begin{macrocode}
\input expkv
%    \end{macrocode}
%
% We make sure that \file{expkv-cs.tex} is only input once:
%    \begin{macrocode}
\expandafter\ifx\csname ekvcVersion\endcsname\relax
\else
  \expandafter\endinput
\fi
%    \end{macrocode}
%
% \begin{macro}{\ekvcVersion,\ekvcDate}
% We're on our first input, so lets store the version and date in a macro.
%    \begin{macrocode}
\def\ekvcVersion{0.10}
\def\ekvcDate{2021-06-03}
%    \end{macrocode}
% \end{macro}
%
% If the \LaTeX\ format is loaded we want to be a good file and report back who
% we are, for this the package will have defined |\ekvc@tmp| to use
% |\ProvidesFile|, else this will expand to a |\relax| and do no harm.
%    \begin{macrocode}
\csname ekvc@tmp\endcsname
%    \end{macrocode}
%
% Store the category code of |@| to later be able to reset it and change it to
% 11 for now.
%    \begin{macrocode}
\expandafter\chardef\csname ekvc@tmp\endcsname=\catcode`\@
\catcode`\@=11
%    \end{macrocode}
% |\ekvc@tmp| will be reused later, but we don't need it to ever store
% information long-term after \expkvc\ was initialized.
%
% \begin{macro}[internal]{\ekvc@keycount}
%   We'll need to keep count how many keys must be defined for each macro in the
%   |split| variants.
%    \begin{macrocode}
\newcount\ekvc@keycount
%    \end{macrocode}
% \end{macro}
%
% \begin{macro}[internal]{\ekvc@long,\ekvc@any@long}
%   Some macros will have to be defined long. These two will be let to |\long|
%   when this should be the case.
%    \begin{macrocode}
\let\ekvc@long\ekv@empty
\let\ekvc@any@long\ekv@empty
%    \end{macrocode}
% \end{macro}
%
% \begin{macro}[internal]{\ekvc@ifdefined}
%   We want to test whether a macro is already defined. This test checks for a
%   defined macro that isn't |\relax|.
%    \begin{macrocode}
\long\def\ekvc@ifdefined#1%
  {%
    \ifdefined#1%
      \ifx\relax#1%
        \ekv@fi@gobble
      \fi
      \@firstofone
      \ekv@fi@firstoftwo
    \fi
    \@secondoftwo
  }
%    \end{macrocode}
% \end{macro}
%
% \begin{macro}[internal]
%   {
%     \ekvc@ekvset@pre@expander,
%     \ekvc@ekvset@pre@expander@a,
%     \ekvc@ekvset@pre@expander@b
%   }
%   This macro expands |\ekvset| twice so that the first two steps of expansion
%   don't have to be made every time the \expkvc\ macros are used. We have to do
%   a little magic trick to get the macro parameter |#1| for the macro
%   definition this is used in, even though we're calling |\unexpanded|. We do
%   that by splitting the expanded |\ekvset| at some marks and place |##1| in
%   between. At this spot we also add |\ekv@alignsafe| and |\ekv@endalignsafe|
%   to ensure that macros created with \expkvc\ are alignment safe.
%    \begin{macrocode}
\def\ekvc@ekvset@pre@expander#1%
  {%
    \expandafter\ekvc@ekvset@pre@expander@a\ekvset{#1}\ekvc@stop\ekvc@stop
  }
\def\ekvc@ekvset@pre@expander@a
  {%
    \expandafter\ekvc@ekvset@pre@expander@b
  }
\def\ekvc@ekvset@pre@expander@b#1\ekvc@stop#2\ekvc@stop
  {%
    \unexpanded\expandafter{\ekv@alignsafe}%
    \unexpanded{#1}##1\unexpanded{#2}%
    \unexpanded\expandafter{\ekv@endalignsafe}%
  }
%    \end{macrocode}
% \end{macro}
%
% \begin{macro}{\ekvcSplitAndUse}
%   The first user macro we want to set up can be reused for
%   \cs[no-index]{ekvcSplitAndForward} and \cs[no-index]{ekvcSplit}. We'll split
%   this one up so that the test whether the macro is already defined doesn't
%   run twice.
%    \begin{macrocode}
\protected\long\def\ekvcSplitAndUse#1#2%
  {%
    \let\ekvc@helpers@needed\@firstoftwo
    \ekvc@ifdefined#1%
      {\ekvc@err@already@defined#1}%
      {\ekvcSplitAndUse@#1{}{#2}}%
  }
%    \end{macrocode}
% \end{macro}
%
% \begin{macro}[internal]{\ekvcSplitAndUse@}
%   The actual macro setting up things. We need to set some variables, forward
%   the key list to |\ekvc@SetupSplitKeys|, and afterwards define the front
%   facing macro to call |\ekvset| and put the initials and the argument sorting
%   macro behind it. The internals |\ekvc@any@long|, |\ekvc@initials| and
%   |\ekvc@keycount| will be set correctly by |\ekvc@SetupSplitKeys|.
%    \begin{macrocode}
\protected\long\def\ekvcSplitAndUse@#1#2#3%
  {%
    \edef\ekvc@set{\string#1}%
    \ekvc@SetupSplitKeys{#3}%
    \ekvc@helpers@needed
      {%
        \ekvc@any@long\edef#1##1%
          {%
            \expandafter\ekvc@ekvset@pre@expander\expandafter{\ekvc@set}%
            \unexpanded\expandafter
              {\csname ekvc@split@\the\ekvc@keycount\endcsname}%
            \unexpanded\expandafter{\ekvc@initials{}#2}%
          }%
      }%
      {%
        \ekvc@any@long\edef#1##1%
          {%
            \expandafter\ekvc@ekvset@pre@expander\expandafter{\ekvc@set}%
            \unexpanded{#2}%
            \unexpanded\expandafter{\ekvc@initials}%
          }%
      }%
  }
%    \end{macrocode}
% \end{macro}
%
%
% \begin{macro}{\ekvcSplitAndForward}
%   This just reuses |\ekvcSplitAndUse@| with a non-empty second argument,
%   resulting in that argument to be called after the splitting.
%    \begin{macrocode}
\protected\long\def\ekvcSplitAndForward#1#2#3%
  {%
    \let\ekvc@helpers@needed\@firstoftwo
    \ekvc@ifdefined#1%
      {\ekvc@err@already@defined#1}%
      {\ekvcSplitAndUse@#1{{#2}}{#3}}%
  }
%    \end{macrocode}
% \end{macro}
%
% \begin{macro}{\ekvcSplit}
%   The first half is just |\ekvcSplitAndForward| then we define the macro to
%   which the parsed key list is forwarded. There we need to allow for up to
%   nine arguments.
%    \begin{macrocode}
\protected\long\def\ekvcSplit#1#2#3%
  {%
    \let\ekvc@helpers@needed\@secondoftwo
    \ekvc@ifdefined#1%
      {\ekvc@err@already@defined#1}%
      {%
        \expandafter
        \ekvcSplitAndUse@\expandafter#1\csname ekvc@\string#1\endcsname{#2}%
        \ifnum\ekvc@keycount<1
          \ekvc@any@long\expandafter\def\csname ekvc@\string#1\endcsname{#3}%
        \else
          \ifnum\ekvc@keycount>9
            \ekvc@err@toomany{#1}%
            \let#1\ekvc@undefined
          \else
            \ekvcSplit@build@argspec
            \ekvc@any@long\expandafter
            \def\csname ekvc@\string#1\expandafter\endcsname\ekvc@tmp{#3}%
          \fi
        \fi
      }%
  }
%    \end{macrocode}
% \end{macro}
%
% \begin{macro}[internal]{\ekvcSplit@build@argspec,\ekvcSplit@build@argspec@}
%    \begin{macrocode}
\protected\def\ekvcSplit@build@argspec
  {%
    \begingroup
      \edef\ekvc@tmp
        {\endgroup\def\unexpanded{\ekvc@tmp}{\ekvcSplit@build@argspec@{1}}}%
    \ekvc@tmp
  }
\def\ekvcSplit@build@argspec@#1%
  {%
    \ifnum#1>\ekvc@keycount
      \ekv@fi@gobble
    \fi
    \@firstofone
      {%
        \unexpanded\expandafter{\csname ekvc@splitmark@#1\endcsname####}#1%
        \expandafter\ekvcSplit@build@argspec@\expandafter{\the\numexpr#1+1}%
      }%
  }
%    \end{macrocode}
% \end{macro}
%
% \begin{macro}[internal]
%   {
%     \ekvc@SetupSplitKeys, \ekvc@SetupSplitKeys@a, \ekvc@SetupSplitKeys@b,
%     \ekvc@SetupSplitKeys@c
%   }
%   These macros parse the list of keys and set up the key macros. First we need
%   to initialise some macros and start |\ekvparse|.
%    \begin{macrocode}
\protected\long\def\ekvc@SetupSplitKeys
  {%
    \ekvc@keycount=0
    \let\ekvc@any@long\ekv@empty
    \let\ekvc@initials\ekv@empty
    \ekvparse\ekvc@err@value@required\ekvc@SetupSplitKeys@a
  }
%    \end{macrocode}
%   Then we need to step the key counter for each key. Also we have to check
%   whether this key has a |long| prefix so we initialise |\ekvc@long|.
%    \begin{macrocode}
\protected\def\ekvc@SetupSplitKeys@a#1%
  {%
    \advance\ekvc@keycount1
    \let\ekvc@long\ekv@empty
    \ekvc@ifspace{#1}%
      {\ekvc@SetupSplitKeys@b#1\ekvc@stop}%
      {\ekvc@SetupSplitKeys@c{#1}}%
  }
%    \end{macrocode}
%   If there was a space, there might be a prefix. If so call the prefix macro,
%   else call the next step |\ekvc@SetupSplitKeys@c| which will define the key
%   macro and add the key's value to the initials list.
%    \begin{macrocode}
\protected\def\ekvc@SetupSplitKeys@b#1 #2\ekvc@stop
  {%
    \ekv@ifdefined{ekvc@split@p@#1}%
      {\csname ekvc@split@p@#1\endcsname{#2}}%
      {\ekvc@SetupSplitKeys@c{#1 #2}}%
  }
%    \end{macrocode}
%   The inner definition is grouped, because we don't want to actually define
%   the marks we build with |\csname|. We have to append the value to the
%   |\ekvc@initials| list here with the correct split mark. The key macro will
%   read everything up to those split marks and change the value following it to
%   the value given to the key. Additionally we'll need a sorting macro for each
%   key count in use so we set it up with |\ekvc@setup@splitmacro|.
%    \begin{macrocode}
\protected\long\def\ekvc@SetupSplitKeys@c#1#2%
  {%
    \begingroup
    \edef\ekvc@tmp
      {%
        \endgroup
        \long\def\unexpanded{\ekvc@tmp}####1####2%
            \unexpanded\expandafter
            {\csname ekvc@splitmark@\the\ekvc@keycount\endcsname}####3%
          {%
            ####2%
            \unexpanded\expandafter
              {\csname ekvc@splitmark@\the\ekvc@keycount\endcsname}{####1}%
          }%
%    \end{macrocode}
%   The short variant needs a bit of special treatment. The key macro will be
%   short to throw the correct error, but since there might be long macros
%   somewhere the reordering of arguments needs to be long, so for short keys we
%   use a two step approach, first grabbing only the short argument, then
%   reordering.
%    \begin{macrocode}
        \unless\ifx\ekvc@long\long
          \let\unexpanded\expandafter
            {\csname ekvc@\ekvc@set(#1)\endcsname\ekvc@tmp}%
          \def\unexpanded{\ekvc@tmp}####1%
            {%
              \unexpanded\expandafter{\csname ekvc@\ekvc@set(#1)\endcsname}%
                {####1}%
            }%
        \fi
        \def\unexpanded{\ekvc@initials}%
          {%
            \unexpanded\expandafter{\ekvc@initials}%
            \unexpanded\expandafter
              {\csname ekvc@splitmark@\the\ekvc@keycount\endcsname{#2}}%
          }%
      }%
    \ekvc@tmp
    \ekvlet\ekvc@set{#1}\ekvc@tmp
    \ekvc@helpers@needed
      {\expandafter\ekvc@setup@splitmacro\expandafter{\the\ekvc@keycount}}%
      {}%
  }
%    \end{macrocode}
% \end{macro}
%
% \begin{macro}[internal]{\ekvc@split@p@long}
%   The |long| prefix lets the internals |\ekvc@long| and |\ekvc@any@long| to
%   |\long| so that the key macro will be long.
%    \begin{macrocode}
\protected\def\ekvc@split@p@long
  {%
    \let\ekvc@long\long
    \let\ekvc@any@long\long
    \ekvc@SetupSplitKeys@c
  }
%    \end{macrocode}
% \end{macro}
%
% \begin{macro}[internal]{\ekvc@defarggobbler}
%   This is needed to define a macro with 1-9 parameters programmatically.
%   \LaTeX's \cs[no-index]{newcommand} does something similar for example.
%    \begin{macrocode}
\protected\def\ekvc@defarggobbler#1{\def\ekvc@tmp##1#1##2##{##1#1}}
%    \end{macrocode}
% \end{macro}
%
% \begin{macro}[internal]
%   {
%     \ekvc@setup@splitmacro,
%     \ekvc@split@1, \ekvc@split@2, \ekvc@split@3,
%     \ekvc@split@4, \ekvc@split@5, \ekvc@split@6,
%     \ekvc@split@7
%   }
%   Since the first few split macros are different from the others we manually
%   set those up now. All the others will be defined as needed (always
%   globally).  The split macros just read up until the correct split mark, move
%   that argument into a list and reinsert the rest, calling the next split
%   macro afterwards.
%    \begin{macrocode}
\begingroup
\edef\ekvc@tmp
  {%
    \long\gdef\unexpanded\expandafter{\csname ekvc@split@1\endcsname}%
        \unexpanded\expandafter{\csname ekvc@splitmark@1\endcsname}%
        ##1##2##3%
      {##3{##1}##2}%
    \long\gdef\unexpanded\expandafter{\csname ekvc@split@2\endcsname}%
        \unexpanded\expandafter{\csname ekvc@splitmark@1\endcsname}##1%
        \unexpanded\expandafter{\csname ekvc@splitmark@2\endcsname}##2%
        ##3##4%
      {##4{##1}{##2}##3}%
    \long\gdef\unexpanded\expandafter{\csname ekvc@split@3\endcsname}%
        \unexpanded\expandafter{\csname ekvc@splitmark@1\endcsname}##1%
        \unexpanded\expandafter{\csname ekvc@splitmark@2\endcsname}##2%
        \unexpanded\expandafter{\csname ekvc@splitmark@3\endcsname}##3%
        ##4##5%
      {##5{##1}{##2}{##3}##4}%
    \long\gdef\unexpanded\expandafter{\csname ekvc@split@4\endcsname}%
        \unexpanded\expandafter{\csname ekvc@splitmark@1\endcsname}##1%
        \unexpanded\expandafter{\csname ekvc@splitmark@2\endcsname}##2%
        \unexpanded\expandafter{\csname ekvc@splitmark@3\endcsname}##3%
        \unexpanded\expandafter{\csname ekvc@splitmark@4\endcsname}##4%
        ##5##6%
      {##6{##1}{##2}{##3}{##4}##5}%
    \long\gdef\unexpanded\expandafter{\csname ekvc@split@5\endcsname}%
        \unexpanded\expandafter{\csname ekvc@splitmark@1\endcsname}##1%
        \unexpanded\expandafter{\csname ekvc@splitmark@2\endcsname}##2%
        \unexpanded\expandafter{\csname ekvc@splitmark@3\endcsname}##3%
        \unexpanded\expandafter{\csname ekvc@splitmark@4\endcsname}##4%
        \unexpanded\expandafter{\csname ekvc@splitmark@5\endcsname}##5%
        ##6##7%
      {##7{##1}{##2}{##3}{##4}{##5}##6}%
    \long\gdef\unexpanded\expandafter{\csname ekvc@split@6\endcsname}%
        \unexpanded\expandafter{\csname ekvc@splitmark@1\endcsname}##1%
        \unexpanded\expandafter{\csname ekvc@splitmark@2\endcsname}##2%
        \unexpanded\expandafter{\csname ekvc@splitmark@3\endcsname}##3%
        \unexpanded\expandafter{\csname ekvc@splitmark@4\endcsname}##4%
        \unexpanded\expandafter{\csname ekvc@splitmark@5\endcsname}##5%
        \unexpanded\expandafter{\csname ekvc@splitmark@6\endcsname}##6%
        ##7##8%
      {##8{##1}{##2}{##3}{##4}{##5}{##6}##7}%
    \long\gdef\unexpanded\expandafter{\csname ekvc@split@7\endcsname}%
        \unexpanded\expandafter{\csname ekvc@splitmark@1\endcsname}##1%
        \unexpanded\expandafter{\csname ekvc@splitmark@2\endcsname}##2%
        \unexpanded\expandafter{\csname ekvc@splitmark@3\endcsname}##3%
        \unexpanded\expandafter{\csname ekvc@splitmark@4\endcsname}##4%
        \unexpanded\expandafter{\csname ekvc@splitmark@5\endcsname}##5%
        \unexpanded\expandafter{\csname ekvc@splitmark@6\endcsname}##6%
        \unexpanded\expandafter{\csname ekvc@splitmark@7\endcsname}##7%
        ##8##9%
      {##9{##1}{##2}{##3}{##4}{##5}{##6}{##7}##8}%
  }
\ekvc@tmp
\endgroup
\protected\def\ekvc@setup@splitmacro#1%
  {%
    \ekv@ifdefined{ekvc@split@#1}{}%
      {%
        \begingroup
          \edef\ekvc@tmp
            {%
              \long\gdef
                  \unexpanded\expandafter{\csname ekvc@split@#1\endcsname}%
                  ####1%
                  \unexpanded\expandafter{\csname ekvc@splitmark@#1\endcsname}%
                  ####2####3%
                {%
                  \unexpanded\expandafter
                    {\csname ekvc@split@\the\numexpr#1-1\relax\endcsname}%
                  ####1{{####2}####3}%
                }%
            }%
          \ekvc@tmp
        \endgroup
      }%
  }
%    \end{macrocode}
% \end{macro}
%
% \begin{macro}{\ekvcHashAndUse}
%   |\ekvcHashAndUse| works just like |\ekvcSplitAndUse|.
%    \begin{macrocode}
\protected\long\def\ekvcHashAndUse#1#2%
  {%
    \let\ekvc@helpers@needed\@firstoftwo
    \ekvc@ifdefined#1%
      {\ekvc@err@already@defined#1}%
      {\ekvcHashAndUse@#1{}{#2}}%
  }
%    \end{macrocode}
% \end{macro}
%
% \begin{macro}[internal]{\ekvcHashAndUse@}
%   This is more or less the same as |\ekvcSplitAndUse@|. Instead of an empty
%   group we place a marker after the initials, we don't use the sorting macros
%   of |split|, but instead pack all the values in one argument.
%    \begin{macrocode}
\protected\long\def\ekvcHashAndUse@#1#2#3%
  {%
    \edef\ekvc@set{\string#1}%
    \ekvc@SetupHashKeys{#3}%
    \ekvc@helpers@needed
      {%
        \ekvc@any@long\edef#1##1%
          {%
            \expandafter\ekvc@ekvset@pre@expander\expandafter{\ekvc@set}%
            \unexpanded{\ekvc@hash@pack@argument}%
            \unexpanded\expandafter{\ekvc@initials\ekvc@stop#2}%
          }%
      }%
      {%
        \ekvc@any@long\edef#1##1%
          {%
            \expandafter\ekvc@ekvset@pre@expander\expandafter{\ekvc@set}%
            \unexpanded{#2}%
            \unexpanded\expandafter{\ekvc@initials\ekvc@stop}%
          }%
      }%
  }
%    \end{macrocode}
% \end{macro}
%
% \begin{macro}{\ekvcHashAndForward}
%   |\ekvcHashAndForward| works just like |\ekvcSplitAndForward|.
%    \begin{macrocode}
\protected\long\def\ekvcHashAndForward#1#2#3%
  {%
    \let\ekvc@helpers@needed\@firstoftwo
    \ekvc@ifdefined#1%
      {\ekvc@err@already@defined#1}%
      {\ekvcHashAndUse@#1{{#2}}{#3}}%
  }
%    \end{macrocode}
% \end{macro}
%
% \begin{macro}{\ekvcHash}
%   |\ekvcHash| does the same as |\ekvcSplit|, but has the advantage of not
%   needing to count arguments, so the definition of the internal macro is a bit
%   more straight forward.
%    \begin{macrocode}
\protected\long\def\ekvcHash#1#2#3%
  {%
    \let\ekvc@helpers@needed\@secondoftwo
    \ekvc@ifdefined#1%
      {\ekvc@err@already@defined#1}%
      {%
        \expandafter
        \ekvcHashAndUse@\expandafter#1\csname ekvc@\string#1\endcsname{#2}%
        \ekvc@any@long\expandafter\def\csname ekvc@\string#1\endcsname
            ##1\ekvc@stop
          {#3}%
      }%
  }
%    \end{macrocode}
% \end{macro}
%
% \begin{macro}[internal]{\ekvc@hash@pack@argument}
%   All this macro does is pack the values into one argument and forward that to
%   the next macro.
%    \begin{macrocode}
\long\def\ekvc@hash@pack@argument#1\ekvc@stop#2{#2{#1}}
%    \end{macrocode}
% \end{macro}
%
% \begin{macro}[internal]
%   {\ekvc@SetupHashKeys,\ekvc@SetupHashKeys@a,\ekvc@SetupHashKeys@b}
%   This should look awfully familiar as well, since it's just the same as for
%   the split keys with a few other names here and there.
%    \begin{macrocode}
\protected\long\def\ekvc@SetupHashKeys#1%
  {%
    \let\ekvc@any@long\ekv@empty
    \let\ekvc@initials\ekv@empty
    \ekvparse\ekvc@err@value@required\ekvc@SetupHashKeys@a{#1}%
  }
\protected\def\ekvc@SetupHashKeys@a#1%
  {%
    \let\ekvc@long\ekv@empty
    \ekvc@ifspace{#1}%
      {\ekvc@SetupHashKeys@b#1\ekvc@stop}%
      {\ekvc@SetupHashKeys@c{#1}}%
  }
\protected\def\ekvc@SetupHashKeys@b#1 #2\ekvc@stop
  {%
    \ekv@ifdefined{ekvc@hash@p@#1}%
      {\csname ekvc@hash@p@#1\endcsname{#2}}%
      {\ekvc@SetupHashKeys@c{#1 #2}}%
  }
%    \end{macrocode}
%   Yes, even the defining macro looks awfully familiar. Instead of numbered we
%   have named marks. Still the key macros grab everything up to their
%   respective mark and reorder the arguments. The same quirk is applied for
%   short keys. And instead of the |\ekvc@setup@splitmacro| we use
%   |\ekvc@setup@hashmacro|.
%    \begin{macrocode}
\protected\long\def\ekvc@SetupHashKeys@c#1#2%
  {%
    \begingroup
    \edef\ekvc@tmp
      {%
        \endgroup
        \long\def\unexpanded{\ekvc@tmp}####1####2%
            \unexpanded\expandafter{\csname ekvc@hashmark@#1\endcsname}####3%
          {%
            ####2%
            \unexpanded\expandafter{\csname ekvc@hashmark@#1\endcsname}{####1}%
          }%
        \unless\ifx\ekvc@long\long
          \let\unexpanded\expandafter
            {\csname ekvc@\ekvc@set(#1)\endcsname\ekvc@tmp}%
          \def\unexpanded{\ekvc@tmp}####1%
            {%
              \unexpanded\expandafter{\csname ekvc@\ekvc@set(#1)\endcsname}%
                {####1}%
            }%
        \fi
        \def\unexpanded{\ekvc@initials}%
          {%
            \unexpanded\expandafter{\ekvc@initials}%
            \unexpanded\expandafter{\csname ekvc@hashmark@#1\endcsname{#2}}%
          }%
      }%
    \ekvc@tmp
    \ekvlet\ekvc@set{#1}\ekvc@tmp
    \ekvc@setup@hashmacro{#1}%
  }
%    \end{macrocode}
% \end{macro}
%
% \begin{macro}[internal]{\ekvc@hash@p@long}
%   Nothing astonishing here either.
%    \begin{macrocode}
\protected\def\ekvc@hash@p@long
  {%
    \let\ekvc@long\long
    \let\ekvc@any@long\long
    \ekvc@SetupHashKeys@c
  }
%    \end{macrocode}
% \end{macro}
%
% \begin{macro}[internal]{\ekvc@setup@hashmacro}
%   The safe hash macros will be executed inside of an |\unexpanded| expansion
%   context, so they have to insert braces for that once they are done.
%   Most of the tests which have to be executed will already be done, but we
%   have to play safe if the hash doesn't show up in the hash list. Therefore we
%   use some |\ekvc@mark|s and |\ekvc@stop| to throw errors if the hash isn't
%   found in the right place. The fast variants have an easier life and just
%   return the correct value.
%    \begin{macrocode}
\protected\def\ekvc@setup@hashmacro#1%
  {%
    \ekv@ifdefined{ekvc@fasthash@#1}{}%
      {%
        \begingroup
          \edef\ekvc@tmp
            {%
              \long\gdef
                \unexpanded\expandafter{\csname ekvc@fasthash@#1\endcsname}%
                  ####1%
                  \unexpanded\expandafter{\csname ekvc@hashmark@#1\endcsname}%
                  ####2####3\unexpanded{\ekvc@stop}%
                {####2}%
              \long\gdef
                \unexpanded\expandafter{\csname ekvc@safehash@#1\endcsname}%
                  ####1%
                {%
                  \unexpanded\expandafter{\csname ekvc@@safehash@#1\endcsname}%
                  ####1\unexpanded{\ekvc@mark}{}%
                  \unexpanded\expandafter
                    {%
                      \csname ekvc@hashmark@#1\endcsname{}%
                      \ekvc@mark{\ekvc@err@missing@hash{#1}}\ekvc@stop
                    }%
                }%
              \long\gdef
                \unexpanded\expandafter{\csname ekvc@@safehash@#1\endcsname}%
                  ####1%
                  \unexpanded\expandafter{\csname ekvc@hashmark@#1\endcsname}%
                  ####2####3\unexpanded{\ekvc@mark}####4####5%
                  \unexpanded{\ekvc@stop}%
                {%
                  ####4{####2}%
                }%
              \long\gdef\unexpanded\expandafter
                {\csname ekvc@fastsplithash@#1\endcsname}%
                  ####1%
                  \unexpanded\expandafter{\csname ekvc@hashmark@#1\endcsname}%
                  ####2####3\unexpanded{\ekvc@stop}####4%
                {%
                  ####4{####1####3}{####2}%
                }%
              \long\gdef\unexpanded\expandafter
                {\csname ekvc@safesplithash@#1\endcsname}####1%
                {%
                  \unexpanded\expandafter
                    {\csname ekvc@@safesplithash@#1\endcsname}%
                  ####1\unexpanded{\ekvc@mark\ekvc@safe@found@hash}%
                  \unexpanded\expandafter
                    {%
                      \csname ekvc@hashmark@#1\endcsname{}%
                      \ekvc@mark{\ekvc@err@missing@hash{#1}\ekvc@safe@no@hash}%
                      \ekvc@stop
                    }%
                }%
              \long\gdef\unexpanded\expandafter
                {\csname ekvc@@safesplithash@#1\endcsname}%
                  ####1%
                  \unexpanded\expandafter{\csname ekvc@hashmark@#1\endcsname}%
                  ####2####3\unexpanded{\ekvc@mark}####4####5%
                  \unexpanded{\ekvc@stop}%
                {%
                  ####4{####2}####1####3\unexpanded{\ekvc@stop}%
                }%
            }%
          \ekvc@tmp
        \endgroup
      }%
  }
%    \end{macrocode}
% \end{macro}
%
% \begin{macro}{\ekvcValue}
%   All this does is a few consistency checks on the first argument (not empty,
%   hash macro exists) and then call that hash-grabbing macro that will also
%   test whether the hash is inside of |#2| or not.
%    \begin{macrocode}
\long\def\ekvcValue#1%
  {%
    \unexpanded
    \ekv@ifdefined{ekvc@safehash@#1}%
      {\csname ekvc@safehash@#1\endcsname}%
      {\ekvc@err@unknown@hash{#1}\@firstoftwo{{}}}%
  }
%    \end{macrocode}
% \end{macro}
%
% \begin{macro}{\ekvcValueFast}
%   To be as fast as possible, this doesn't test for anything, assuming the user
%   knows best.
%    \begin{macrocode}
\long\def\ekvcValueFast#1#2{\csname ekvc@fasthash@#1\endcsname#2\ekvc@stop}
%    \end{macrocode}
% \end{macro}
%
% \begin{macro}{\ekvcValueSplit,\ekvcValueSplit@recover}
%   This splits off a single value.
%    \begin{macrocode}
\long\def\ekvcValueSplit#1%
  {%
    \ekv@ifdefined{ekvc@safesplithash@#1}%
      {\csname ekvc@safesplithash@#1\endcsname}%
      {\ekvc@err@unknown@hash{#1}\ekvcValueSplit@recover}%
  }
\long\def\ekvcValueSplit@recover#1#2%
  {%
    #2{#1}{}%
  }
%    \end{macrocode}
% \end{macro}
%
% \begin{macro}[internal]{\ekvc@safe@found@hash, \ekvc@safe@no@hash}
%    \begin{macrocode}
\long\def\ekvc@safe@found@hash#1#2\ekvc@stop#3%
  {%
    #3{#2}{#1}%
  }
\long\def\ekvc@safe@no@hash#1#2\ekvc@mark\ekvc@safe@found@hash\ekvc@stop#3%
  {%
    #3{#2}{}%
  }
%    \end{macrocode}
% \end{macro}
%
% \begin{macro}{\ekvcValueSplitFast}
%   Again a fast approach which doesn't provide too many safety measurements.
%   This needs to build the hash function and expand it before passing the
%   results to the next control sequence. The first step only builds the control
%   sequence.
%    \begin{macrocode}
\long\def\ekvcValueSplitFast#1#2%
  {%
    \csname ekvc@fastsplithash@#1\endcsname#2\ekvc@stop
  }
%    \end{macrocode}
% \end{macro}
%
% \begin{macro}[internal]
%   {
%     \ekvc@safehash@,\ekvc@fasthash@,
%     \ekvc@safesplithash@,\ekvc@fastsplithash@
%   }
%   At least in the empty hash case we can provide a meaningful error message
%   without affecting performance by just defining the macro that would be build
%   in that case. There is of course a downside to this, the error will not be
%   thrown by |\ekvcValueFast| in three expansion steps. The safe hash variant
%   has to also stop the |\unexpanded| expansion.
%    \begin{macrocode}
\long\def\ekvc@safehash@#1{\ekvc@err@empty@hash{}}
\long\def\ekvc@fasthash@#1\ekvc@stop{\ekvc@err@empty@hash}
\long\def\ekvc@safesplithash@#1#2{\ekvc@err@empty@hash#2{#1}{}}
\long\def\ekvc@fastsplithash@#1\ekvc@stop#2{\ekvc@err@empty@hash#2{#1}{}}
%    \end{macrocode}
% \end{macro}
%
% \begin{macro}{\ekvcSecondaryKeys}
%   The secondary keys are defined pretty similar to the way the originals are,
%   but here we also introduce some key types (those have a |@t@| in their name)
%   additionally to the prefixes.
%    \begin{macrocode}
\protected\long\def\ekvcSecondaryKeys#1#2%
  {%
    \edef\ekvc@set{\string#1}%
    \ekvparse\ekvc@err@value@required\ekvcSecondaryKeys@a{#2}%
  }
\protected\def\ekvcSecondaryKeys@a#1%
  {%
    \let\ekvc@long\ekv@empty
    \ekvc@ifspace{#1}%
      {\ekvcSecondaryKeys@b#1\ekvc@stop}%
      {\ekvc@err@missing@type{#1}\@gobble}%
  }
\protected\def\ekvcSecondaryKeys@b#1 #2\ekvc@stop
  {%
    \ekv@ifdefined{ekvc@p@#1}%
      {\csname ekvc@p@#1\endcsname}%
      {%
        \ekv@ifdefined{ekvc@t@#1}%
          {\csname ekvc@t@#1\endcsname}%
          {\ekvc@err@unknown@keytype{#1}\@firstoftwo\@gobble}%
      }%
      {#2}%
  }
%    \end{macrocode}
% \end{macro}
%
% \begin{macro}{\ekvcChange}
%   This can be used to change the defaults of an \expkvc\ defined macro. It
%   checks whether there is a set with the correct name and that the macro is
%   defined. If both is true the real work is done by |\ekvc@change|.
%    \begin{macrocode}
\protected\long\def\ekvcChange#1%
  {%
    \ekvifdefinedset{\string#1}%
      {%
        \ekvc@ifdefined#1%
          {\ekvc@change#1}%
          {\ekvc@err@no@key@macro#1\@gobble}%
      }%
      {\ekvc@err@no@key@macro#1\@gobble}%
  }
%    \end{macrocode}
% \end{macro}
%
% \begin{macro}[internal]
%   {\ekvc@change,\ekvc@change@a,\ekvc@change@b,\ekvc@change@c}
%   First we need to see whether the macro is currently |\long|. For this we get
%   the meaning and will parse it. |#1| is the macro name in which we want to
%   change the defaults.
%    \begin{macrocode}
\protected\def\ekvc@change#1%
  {\expandafter\ekvc@change@a\meaning#1\ekv@stop#1}
%    \end{macrocode}
%   A temporary definition to get the stringified |macro:|. |##1| will be the
%   list of prefixes, we don't care for the exact contents of |##2| and |##3|.
%    \begin{macrocode}
\def\ekvc@change@a#1%
  {%
    \protected\def\ekvc@change@a##1#1##2->##3\ekv@stop
      {%
        \ekvc@change@iflong{##1}%
          {\ekvc@change@b{}}%
          {\ekvc@change@b{\long}}%
      }%
  }
\expandafter\ekvc@change@a\expandafter{\detokenize{macro:}}
%    \end{macrocode}
%   Next we expand the macro once to get its contents (including the current
%   default values with their markers). |#1| is either |\long| or empty, |#2| is
%   the macro.
%    \begin{macrocode}
\protected\def\ekvc@change@b#1#2%
  {\expandafter\ekvc@change@c\expandafter{#2{##1}}{#1}#2}
%    \end{macrocode}
%   Here we place an unbalanced closing brace after the expansion of the macro.
%   Then we just parse the \kv-list with |\ekvset|, that will exchange the
%   values behind the markers. Once those are changed we give control to
%   |\ekvc@change@d|. The |\ekvset| step might horribly fail if the user defined
%   some keys that don't behave nice. |#1| is the expansion of the macro, |#2|
%   is either |\long| or empty, |#3| is the macro, and |#4| is the \kv-list
%   containing the new defaults.
%    \begin{macrocode}
\ekv@exparg{\protected\long\def\ekvc@change@c#1#2#3#4}%
  {%
    \expandafter\iffalse\expandafter{\expandafter\fi
      \ekvset{\string#3}{#4}%
      \ekvc@change@d{#2}{#3}%
      #1%
    }%
  }
%    \end{macrocode}
%   The final step needs to put an unbalanced opening brace after |\def|. We do
%   that with the help of a temporary macro which stores everything necessary
%   for |\def| and expand an |\iffalse}\fi| construct to leave a single
%   opening brace. |#1| will be either empty or |\long| and |#2| is the macro.
%   Each of the macros defined with \expkvc\ takes exactly one parameter, so we
%   put that here as |####1|.
%    \begin{macrocode}
\protected\def\ekvc@change@d#1#2%
  {%
    \def\ekvc@tmp{#1\def#2####1}%
    \expandafter\ekvc@tmp\expandafter{\iffalse}\fi
  }
%    \end{macrocode}
% \end{macro}
%
% \begin{macro}[internal]{\ekvc@change@iflong,\ekvc@change@iflong@}
%   Checking whether a string contains the string representation of |\long| can
%   be done by gobbling everything up to the first |\long| and checking whether
%   the result is completely empty. We need a temporary macro to get the result
%   of |\string\long| inside the definitions.
%    \begin{macrocode}
\def\ekvc@change@iflong#1%
  {%
    \protected\def\ekvc@change@iflong##1%
      {\expandafter\ekv@ifempty\expandafter{\ekvc@change@iflong@##1#1}}%
    \def\ekvc@change@iflong@##1#1{}
  }
\expandafter\ekvc@change@iflong\expandafter{\string\long}
%    \end{macrocode}
% \end{macro}
%
% \begin{macro}{\ekvcPass}
%   This macro can be used to pass a value to a key of some macro (this way
%   more complicated key codes are possible that in the end pass processed
%   values on to some macro). The implemantation is pretty straight forward.
%    \begin{macrocode}
\long\def\ekvcPass#1#2%
  {%
    \ekvifdefined{\string#1}{#2}%
      {\csname\ekv@name{\string#1}{#2}\endcsname}%
      {\ekvc@err@unknown@key@or@macro{#1}{#2}\@gobble}%
  }
%    \end{macrocode}
% \end{macro}
%
%
% \subsubsection{Secondary Key Types}
%
% \begin{macro}[internal]{\ekvc@p@long,\ekvc@after@ptype}
%   The prefixes are pretty straight forward again. Just set |\ekvc@long| and
%   forward to the |@t@| type.
%    \begin{macrocode}
\protected\def\ekvc@p@long#1%
  {%
    \ekvc@ifspace{#1}%
      {%
        \let\ekvc@long\long
        \ekvc@after@ptype#1\ekvc@stop
      }%
      {\ekvc@err@missing@type{long #1}\@gobble}%
  }
\protected\def\ekvc@after@ptype#1 #2\ekvc@stop
  {%
    \ekv@ifdefined{ekvc@t@#1}%
      {\csname ekvc@t@#1\endcsname{#2}}%
      {\ekvc@err@unknown@keytype{#1}\@gobble}%
  }
%    \end{macrocode}
% \end{macro}
%
% \begin{macro}[internal]
%   {
%     \ekvc@t@meta, \ekvc@t@nmeta,
%     \ekvc@type@meta, \ekvc@type@meta@a, \ekvc@type@meta@b
%   }
%   The |meta| and |nmeta| key types use a nested |\ekvset| to set other keys in
%   the same macro's \set.
%    \begin{macrocode}
\protected\def\ekvc@t@meta
  {%
    \edef\ekvc@tmp{\ekvc@set}%
    \expandafter\ekvc@type@meta\expandafter{\ekvc@tmp}\ekvc@long{##1}\ekvlet
  }
\protected\def\ekvc@t@nmeta#1%
  {%
    \ekvc@assert@not@long{nmeta #1}%
    \edef\ekvc@tmp{\ekvc@set}%
    \expandafter\ekvc@type@meta\expandafter{\ekvc@tmp}{}{}\ekvletNoVal{#1}%
  }
\protected\long\def\ekvc@type@meta#1#2#3#4#5#6%
  {%
    \expandafter\ekvc@type@meta@a\expandafter{\ekvset{#1}{#6}}{#2}{#3}%
    #4\ekvc@set{#5}\ekvc@tmp
  }
\protected\def\ekvc@type@meta@a
  {%
    \expandafter\ekvc@type@meta@b\expandafter
  }
\protected\long\def\ekvc@type@meta@b#1#2#3%
  {%
    #2\def\ekvc@tmp#3{#1}%
  }
%    \end{macrocode}
% \end{macro}
%
% \begin{macro}[internal]{\ekvc@t@alias}
%   |alias| just checks whether there is a key and/or |NoVal| key defined with
%   the target name and |\let| the key to those.
%    \begin{macrocode}
\protected\def\ekvc@t@alias#1#2%
  {%
    \ekvc@assert@not@long{alias #1}%
    \let\ekvc@tmp\@firstofone
    \ekvifdefined\ekvc@set{#2}%
      {%
        \ekvletkv\ekvc@set{#1}\ekvc@set{#2}%
        \let\ekvc@tmp\@gobble
      }%
      {}%
    \ekvifdefinedNoVal\ekvc@set{#2}%
      {%
        \ekvletkvNoVal\ekvc@set{#1}\ekvc@set{#2}%
        \let\ekvc@tmp\@gobble
      }%
      {}%
    \ekvc@tmp{\ekvc@err@unknown@key{#2}}%
  }
%    \end{macrocode}
% \end{macro}
%
% \begin{macro}[internal]{\ekvc@t@default}
%   The |default| key can be used to set a |NoVal| key for an existing key. It
%   will just pass the \val\ to the key macro of that other key.
%    \begin{macrocode}
\protected\long\def\ekvc@t@default#1#2%
  {%
    \ekvifdefined\ekvc@set{#1}%
      {%
        \ekvc@assert@not@long{default #1}%
        \edef\ekvc@tmp
          {%
            \unexpanded\expandafter
              {\csname\ekv@name\ekvc@set{#1}\endcsname{#2}}%
          }%
        \ekvletNoVal\ekvc@set{#1}\ekvc@tmp
      }%
      {\ekvc@err@unknown@key{#1}}%
  }
%    \end{macrocode}
% \end{macro}
%
% \begin{macro}[internal]{\ekvc@t@flag-bool}
%    \begin{macrocode}
\protected\expandafter\def\csname ekvc@t@flag-bool\endcsname#1#2%
  {%
    \ekvc@assert@not@long{flag-bool #1}%
    \unless\ifdefined#2\ekvcFlagNew#2\fi
    \ekvdef\ekvc@set{#1}%
      {%
        \ekv@ifdefined{ekvc@flag@set@##1}%
          {%
            \csname ekvc@flag@set@##1\expandafter\endcsname
              \ekvcFlagHeight#2\ekv@stop#2%
          }%
          {\ekvc@err@invalid@bool{##1}}%
      }%
  }
%    \end{macrocode}
% \end{macro}
%
% \begin{macro}[internal]
%   {\ekvc@t@flag-true,\ekvc@t@flag-false,\ekvc@t@flag-raise,\ekvc@type@flag}
%    \begin{macrocode}
\protected\def\ekvc@type@flag#1#2#3#4%
  {%
    \ekvc@assert@not@long{flag-#1 #3}%
    \unless\ifdefined#4\ekvcFlagNew#4\fi
    \ekv@exparg{\ekvdefNoVal\ekvc@set{#3}}{#2#4}%
  }
\protected\expandafter\def\csname ekvc@t@flag-true\endcsname
  {\ekvc@type@flag{true}\ekvcFlagSetTrue}
\protected\expandafter\def\csname ekvc@t@flag-false\endcsname
  {\ekvc@type@flag{false}\ekvcFlagSetFalse}
\protected\expandafter\def\csname ekvc@t@flag-raise\endcsname
  {\ekvc@type@flag{raise}\ekvcFlagRaise}
%    \end{macrocode}
% \end{macro}
%
%
% \subsubsection{Flags}
%
% The basic idea of flags is to store information by the fact that \TeX\
% expandably assigns the meaning |\relax| to undefined control sequences which
% were built with |\csname|. This mechanism is borrowed from \pkg{expl3}.
%
% \begin{macro}[internal]{\ekvc@flag@name,\ekvc@flag@namescheme}
%   Flags follow a simple naming scheme which we define here. |\ekvc@flag@name|
%   will store the name of an internal function that is used to build names of
%   the second naming scheme defined by |\ekvc@flag@namescheme|.
%    \begin{macrocode}
\def\ekvc@flag@name{ekvcf\string}
\def\ekvc@flag@namescheme#1#2{ekvch#2#1}
%    \end{macrocode}
% \end{macro}
%
% \begin{macro}{\ekvcFlagHeight}
%   For semantic reasons we use |\number| with another name.
%    \begin{macrocode}
\let\ekvcFlagHeight\number
%    \end{macrocode}
% \end{macro}
%
% \begin{macro}{\ekvcFlagNew}
%   This macro defines a new flag. It stores the function build with the
%   |\ekvc@flag@name| naming scheme after the internal function
%   |\ekvc@flag@height| that'll determine the current flag height. It'll also
%   define the macro named via |\ekvc@flag@name| to build names according to
%   |\ekvc@flag@namescheme|.
%    \begin{macrocode}
\protected\def\ekvcFlagNew#1%
  {%
    \edef#1%
      {%
        \unexpanded{\ekvc@flag@height}%
          \unexpanded\expandafter{\csname\ekvc@flag@name#1\endcsname}%
      }%
    \ekv@expargtwice
      {\expandafter\def\csname\ekvc@flag@name#1\endcsname##1}%
      {\expandafter\ekvc@flag@namescheme\expandafter{\string#1}{##1}}%
  }
%    \end{macrocode}
% \end{macro}
%
% \begin{macro}[internal]{\ekvc@flag@height,\ekvc@flag@height@}
%   This macro gets the height of a flag by a simple loop. The first loop
%   iteration differs a bit from the following in that it doesn't have to get
%   the current iteration count. The space at the end of |\ekvc@flag@height|
%   ends the |\number| evaluation.
%    \begin{macrocode}
\def\ekvc@flag@height#1%
  {%
    \ifcsname#10\endcsname
      \ekvc@flag@height@1\ekv@stop#1%
    \fi
    \@firstofone{0} % leave this space
  }
\def\ekvc@flag@height@#1\ekv@stop#2\fi\@firstofone#3%
  {%
    \fi
    \ifcsname#2{#1}\endcsname
      \expandafter\ekvc@flag@height@\the\numexpr#1+1\relax\ekv@stop#2%
    \fi
    \@firstofone{#1}%
  }
%    \end{macrocode}
% \end{macro}
%
% \begin{macro}{\ekvcFlagRaise}
%   Raising a flag simply means letting the |\ekvc@flag@namescheme| macro for
%   the current height to relax. The result of raising a flag is that its height
%   is bigger by $1$.
%    \begin{macrocode}
\ekv@exparg{\def\ekvcFlagRaise#1}%
  {%
    \expandafter\expandafter\expandafter\@gobble\expandafter
      \csname\ekvc@flag@namescheme{\string#1}{\ekvcFlagHeight#1}\endcsname
  }
%    \end{macrocode}
% \end{macro}
%
% \begin{macro}{\ekvcFlagSetTrue,\ekvcFlagSetFalse}
% \begin{macro}[internal]{\ekvc@flag@set@true,\ekvc@flag@set@false}
%   A flag is considered true if its current height is odd, and as false if it
%   is even. Therefore |\ekvcFlagSetTrue| and |\ekvcFlagSetFalse| only need to
%   raise the flag if the opposing boolean value is the current one.
%    \begin{macrocode}
\def\ekvcFlagSetTrue#1%
  {\expandafter\ekvc@flag@set@true\ekvcFlagHeight#1\ekv@stop#1}
\def\ekvcFlagSetFalse#1%
  {\expandafter\ekvc@flag@set@false\ekvcFlagHeight#1\ekv@stop#1}
%    \end{macrocode}
%   We can expand |\ekvc@flag@namescheme| at definition time here, which is why
%   we're using a temporary definition to set up |\ekvc@flag@set@true| and
%   |\ekvc@flag@set@false|.
%    \begin{macrocode}
\def\ekvc@flag@set@true#1%
  {%
    \def\ekvc@flag@set@true##1\ekv@stop##2%
      {%
        \ifodd##1
          \ekv@fi@gobble
        \fi
        \@firstofone{\expandafter\@gobble\csname#1\endcsname}%
      }%
    \def\ekvc@flag@set@false##1\ekv@stop##2%
      {%
        \ifodd##1
          \ekv@fi@firstofone
        \fi
        \@gobble{\expandafter\@gobble\csname#1\endcsname}%
      }%
  }
\expandafter\ekvc@flag@set@true\expandafter
  {\ekvc@flag@namescheme{\string#2}{#1}}
%    \end{macrocode}
% \end{macro}
% \end{macro}
%
% \begin{macro}{\ekvcFlagIf}
%   As already explained, truthiness is defined as a flag's height is odd, so we
%   just branch accordingly here.
%    \begin{macrocode}
\def\ekvcFlagIf#1%
  {%
    \ifodd#1%
      \ekv@fi@firstoftwo
    \fi
    \@secondoftwo
  }
%    \end{macrocode}
% \end{macro}
%
% \begin{macro}{\ekvcFlagIfRaised}
%   This macro uses flags as a switch, if a flag's current height is bigger than
%   $0$ this test yields true.
%    \begin{macrocode}
\ekv@exparg{\def\ekvcFlagIfRaised#1}%
  {%
    \expandafter\ifcsname\ekvc@flag@namescheme{\string#1}0\endcsname
      \ekv@fi@firstoftwo
    \fi
    \@secondoftwo
  }
%    \end{macrocode}
% \end{macro}
%
% \begin{macro}{\ekvcFlagReset}
% \begin{macro}[internal]{\ekvc@flag@reset,\ekvc@flag@reset@}
%   Resetting works by locally letting all the defined internal macros named
%   after |\ekvc@flag@namescheme| to undefined.
%    \begin{macrocode}
\protected\def\ekvcFlagReset#1%
  {\expandafter\ekvc@flag@reset\csname\ekvc@flag@name#1\endcsname}
\protected\def\ekvc@flag@reset#1%
  {%
    \ifcsname#10\endcsname
      \expandafter\let\csname#10\endcsname\ekvc@undefined
      \ekvc@flag@reset@1\ekv@stop#1%
    \fi
  }
\protected\def\ekvc@flag@reset@#1\ekv@stop#2\fi
  {%
    \fi
    \ifcsname#2{#1}\endcsname
      \expandafter\let\csname#2{#1}\endcsname\ekvc@undefined
      \expandafter\ekvc@flag@reset@\the\numexpr#1+1\relax\ekv@stop#2%
    \fi
  }
%    \end{macrocode}
% \end{macro}
% \end{macro}
%
% \begin{macro}{\ekvcFlagGetHeight}
% \begin{macro}[internal]{\ekvc@flag@get@height@single}
%   These are just small helpers, first getting the height of the flag and then
%   passing it on to the user supplied code.
%    \begin{macrocode}
\def\ekvcFlagGetHeight#1%
  {\expandafter\ekvc@flag@get@height@single\ekvcFlagHeight#1\ekv@stop}
\long\def\ekvc@flag@get@height@single#1\ekv@stop#2{#2{#1}}
%    \end{macrocode}
% \end{macro}
% \end{macro}
%
% \begin{macro}{\ekvcFlagGetHeights}
% \begin{macro}[internal]
%   {\ekvc@flag@get@heights,\ekvc@flag@get@heights@,\ekvc@flag@get@heights@done}
%   This works by a simple loop that stops at |\ekv@stop|. As long as that
%   marker isn't hit, get the next flags height and put it into a list after
%   |\ekv@stop|. |\ekvc@flag@get@heights@| uses the same marker name for the
%   end of the height, which shouldn't clash in any case. Once we're done we
%   remove the remainder of the current iteration and leave the user supplied
%   code in the input stream with all the flags' heights as a single argument.
%    \begin{macrocode}
\def\ekvcFlagGetHeights#1%
  {%
    \ekvc@flag@get@heights#1\ekv@stop{}%
  }
\def\ekvc@flag@get@heights#1%
  {%
    \ekv@gobbleto@stop#1\ekvc@flag@get@heights@done\ekv@stop
    \expandafter\ekvc@flag@get@heights@\ekvcFlagHeight#1\ekv@stop
  }
\def\ekvc@flag@get@heights@#1\ekv@stop#2\ekv@stop#3%
  {\ekvc@flag@get@heights#2\ekv@stop{#3{#1}}}
\long\def\ekvc@flag@get@heights@done
    \ekv@stop
    \expandafter\ekvc@flag@get@heights@\ekvcFlagHeight\ekv@stop\ekv@stop#1#2%
  {#2{#1}}
%    \end{macrocode}
% \end{macro}
% \end{macro}
%
%
% \subsubsection{Helper Macros}
%
% \begin{macro}[internal]{\ekvc@ifspace,\ekvc@ifspace@}
% A test which can be reduced to an if-empty by gobbling everything up to the
% first space.
%    \begin{macrocode}
\long\def\ekvc@ifspace#1%
  {%
    \ekvc@ifspace@#1 \ekv@ifempty@B
      \ekv@ifempty@false\ekv@ifempty@A\ekv@ifempty@B\@firstoftwo
  }
\long\def\ekvc@ifspace@#1 % keep this space
  {%
    \ekv@ifempty@\ekv@ifempty@A
  }
%    \end{macrocode}
% \end{macro}
%
%
% \subsubsection{Assertions}
% \begin{macro}[internal]{\ekvc@assert@not@long}
%   Some keys don't want to be |long| and we have to educate the user, so let's
%   throw an error if someone wanted these to be long.
%    \begin{macrocode}
\long\def\ekvc@assert@not@long#1{\ifx\ekvc@long\long\ekvc@err@no@long{#1}\fi}
%    \end{macrocode}
% \end{macro}
%
%
% \subsubsection{Messages}
%
% \begin{macro}[internal]
%   {
%     \ekvc@err@toomany,\ekvc@err@value@required,\ekvc@err@missing@type,
%     \ekvc@err@already@defined,\ekvc@err@no@key@macro
%   }
%   Boring unexpandable error messages.
%    \begin{macrocode}
\protected\def\ekvc@err@toomany#1%
  {%
    \errmessage{expkv-cs Error: Too many keys for macro `\string#1'}%
  }
\protected\def\ekvc@err@value@required#1%
  {%
    \errmessage{expkv-cs Error: Missing value for key `\unexpanded{#1}'}%
  }
\protected\def\ekvc@err@missing@type#1%
  {%
    \errmessage
      {expkv-cs Error: Missing type for secondary key `\unexpanded{#1}'}%
  }
\protected\def\ekvc@err@no@long#1%
  {%
    \errmessage
      {expkv-cs Error: prefix `long' not accepted for `\unexpanded{#1}'}%
  }
\protected\def\ekvc@err@already@defined#1%
  {%
    \errmessage{expkv-cs Error: Macro `\string#1' already defined}%
  }
\protected\def\ekvc@err@unknown@keytype#1%
  {%
    \errmessage{expkv-cs Error: Unknown key type `\unexpanded{#1}'}%
  }
\protected\def\ekvc@err@unknown@key#1%
  {%
    \errmessage
      {expkv-cs Error: Unknown key `\unexpanded{#1}' for macro `\ekvc@set'}%
  }
\long\def\ekvc@err@no@key@macro#1%
  {\errmessage{expkv-cs Error: \string#1 is no key=val macro}}
%    \end{macrocode}
% \end{macro}
%
% \begin{macro}[internal]{\ekvc@err}
%   We need a way to throw error messages expandably in some contexts.
%    \begin{macrocode}
\ekv@exparg{\long\def\ekvc@err#1}{\ekverr{expkv-cs}{#1}}
%    \end{macrocode}
% \end{macro}
%
% \begin{macro}[internal]
%   {
%     \ekvc@err@unknown@hash,\ekvc@err@empty@hash,\ekvc@err@missing@hash,
%     \ekvc@err@invalid@bool
%   }
%   And here are the expandable error messages.
%    \begin{macrocode}
\long\def\ekvc@err@unknown@hash#1{\ekvc@err{unknown hash `#1'}}
\long\def\ekvc@err@missing@hash#1{\ekvc@err{hash `#1' not found}}
\long\def\ekvc@err@empty@hash{\ekvc@err{empty hash}}
\def\ekvc@err@invalid@bool#1{\ekvc@err{invalid boolean value `#1'}}
\long\def\ekvc@err@unknown@key@or@macro#1#2%
  {\ekvc@err{unknown key `#2' for macro #1}}
%    \end{macrocode}
% \end{macro}
%
%
% Now everything that's left is to reset the category code of |@|.
%    \begin{macrocode}
\catcode`\@=\ekvc@tmp
%    \end{macrocode}
%
% \gobbledocstriptag
%</tex>
%^^A=<<
%
%^^A=<<
%
% \end{implementation}^^A=<<
%
% \clearpage
% \PrintIndex
%
\endinput
%
^^A vim: ft=tex fdm=marker fmr=>>=,=<<
}{}
\ProvidesFile{expkv-cs.dtx}
  [\csname ekvcDate\endcsname\ define expandable key=val macros using expkv]
\PassOptionsToPackage{full}{textcomp}
\documentclass{l3doc}
\RequirePackage[oldstylenums,nott]{kpfonts}
\input{glyphtounicode}
\pdfgentounicode=1
\RequirePackage{tcolorbox}
\newtcolorbox{exresult}[2][]
  {colback=expkvgrey!10!white,colframe=expkvgrey,fontupper=\small,width={#2},#1}
\newtcbox\exres[1][]
  {
     colback=expkvgrey!10!white
    ,colframe=expkvgrey
    ,size=small
    ,nobeforeafter
    ,tcbox raise base
    ,fontupper=\small
    ,#1
  }
\RequirePackage{listings}
\RequirePackage{booktabs}
\RequirePackage{array}
\RequirePackage{collcell}
\RequirePackage{siunitx}
\RequirePackage{xcolor}
\RequirePackage{caption}
\RequirePackage{microtype}
\RequirePackage{accsupp}
\lstdefinelanguage{expkvc}[primitive]{TeX}
  {
    ,moretexcs=[4]^^A e-TeX
      {
        expanded,
        numexpr,
        protected,
      }
    ,moretexcs=[5]^^A plain/LaTeX
      {
        approx,
        begin,
        empty,
        item,
        LaTeX,
        makeatletter,makeatother,multicolumn,
        newcommand,newdimen,
        RequirePackage,
        rule,
        TeX,
        textit,texttt,
        usepackage,
      }
    ,moretexcs=[6]^^A used packages
      {
        ^^A xfp
        fpeval,
        ^^A keyval
        setkeys,
        ^^A yax
        defactiveparameter,storevalue,setparameterlist
      }
    ,moretexcs=[2]^^A expkv-cs macros
      {
        ekvcSplit,ekvcSplitAndForward,ekvcSplitAndUse,
        ekvcHash,ekvcHashAndForward,ekvcHashAndUse,
        ekvcValue,ekvcValueFast,ekvcValueSplit,ekvcValueSplitFast,
        ekvcSecondaryKeys,
        ekvcChange,
        ekvcFlagNew,
        ekvcFlagHeight,
        ekvcFlagRaise,
        ekvcFlagSetTrue,ekvcFlagSetFalse,
        ekvcFlagIf,ekvcFlagIfRaised,
        ekvcFlagReset,
        ekvcFlagGetHeight,ekvcFlagGetHeights,
        ekvcPass
      }
    ,moretexcs=[3]^^A expkv-pkg macros
      {
        ^^A expkv
        ekvdef,ekvdefNoVal,^^A
        ekvlet,ekvletNoVal,ekvletkv,ekvletkvNoVal,^^A
        ekvdefunknown,ekvdefunknownNoVal,^^A
        ekvredirectunknown,ekvredirectunknownNoVal,^^A
        ekvset,ekvsetSneaked,^^A
        ekvsetdef,ekvsetSneakeddef,ekvsetdefSneaked,^^A
        ekvparse,^^A
        ekvVersion,ekvDate,^^A
        ekvifdefined,ekvifdefinedNoVal,ekvifdefinedset,^^A
        ekvbreak,ekvbreakPreSneak,ekvbreakPostSneak,^^A
        ekvsneak,ekvsneakPre,^^A
        ekvchangeset,^^A
        ekvoptarg,ekvoptargTF,^^A
        ekverr,
        ^^A expkv-def
        ekvdefinekeys
      }
  }
\colorlet{codeparam}{cyan!65!black}
\def\mylstwd{.55em}
\lstset
  {
    ,language=expkvc
    ,flexiblecolumns=false
    ,basewidth=\mylstwd
    ,gobble=2
    ,basicstyle=\fontfamily{jkp}\itshape
    ,texcsstyle=*[1]{\bfseries\color{expkvgrey}} ^^A primitives
    ,texcsstyle=*[2]{\bfseries\color{expkvred}}  ^^A expkv
    ,texcsstyle=*[3]{\color{expkvred}}           ^^A expkv-pkg
    ,texcsstyle=*[4]{\bfseries\color{expkvgrey}} ^^A e-TeX
    ,texcsstyle=*[5]{\bfseries\color{expkvgrey}} ^^A plain/LaTeX macros
    ,texcsstyle=*[6]{}                           ^^A macros of other packages
    ,commentstyle=\color[gray]{0.4}
    ,literate=
              {\{} {{\CodeUpBf\{}}{1}
              {\}} {{\CodeUpBf\}}}{1}
              {$}  {{\CodeUpBf\$}}{1}
              {[}  {{\CodeUp[}}{1}
              {]}  {{\CodeUp]}}{1}
              {(}  {{\CodeUp(}}{1}
              {)}  {{\CodeUp)}}{1}
              {*}  {{$*$}}{1}
              {1}  {{\CodeUp{1}}}{1}
              {2}  {{\CodeUp{2}}}{1}
              {3}  {{\CodeUp{3}}}{1}
              {4}  {{\CodeUp{4}}}{1}
              {5}  {{\CodeUp{5}}}{1}
              {6}  {{\CodeUp{6}}}{1}
              {7}  {{\CodeUp{7}}}{1}
              {8}  {{\CodeUp{8}}}{1}
              {9}  {{\CodeUp{9}}}{1}
              {0}  {{\CodeUp{0}}}{1}
              {##} {{\CodeColored{codeparam}{1}{\#}}}{1}
              {##1}{{\CodeColored{codeparam}{2}{\#1}}}{2}
              {##2}{{\CodeColored{codeparam}{2}{\#2}}}{2}
              {##3}{{\CodeColored{codeparam}{2}{\#3}}}{2}
              {##4}{{\CodeColored{codeparam}{2}{\#4}}}{2}
              {##5}{{\CodeColored{codeparam}{2}{\#5}}}{2}
              {##6}{{\CodeColored{codeparam}{2}{\#6}}}{2}
              {##7}{{\CodeColored{codeparam}{2}{\#7}}}{2}
              {##8}{{\CodeColored{codeparam}{2}{\#8}}}{2}
              {##9}{{\CodeColored{codeparam}{2}{\#9}}}{2}
              {<key>}{{$\langle$}key{$\rangle$}}{5}
              {<set>}{{$\langle$}set{$\rangle$}}{5}
              {<set1>}{{$\langle$}set1{$\rangle$}}{6}
              {<set2>}{{$\langle$}set2{$\rangle$}}{6}
    ^^A,literate=*{<key>}{\key}{4}{<set>}{\set}{4}
  }
\newcommand*\CodeColored[3]{\textcolor{#1}{\makebox[\dimexpr\mylstwd*#2]{#3}}}
\newcommand*\CodeUpBf[1]{\makebox[\mylstwd]{\textup{\textbf{#1}}}}
\newcommand*\CodeUp[1]{\makebox[\mylstwd]{\textup{#1}}}
\RequirePackage{randtext}
\let\metaORIG\meta
\protected\def\meta #1{\texttt{\metaORIG{#1}}}
\renewcommand*\thefootnote{\fnsymbol{footnote}}
\definecolor{expkvred}{HTML}{9F393D}
\colorlet{expkvgrey}{black!75}
\makeatletter
\newcommand*\example{\par\smallskip\noindent\textit{Example:} \ignorespaces}
\newcommand*\expkv
  {^^A
    \texorpdfstring
      {^^A
        \mbox
          {^^A
            \BeginAccSupp{ActualText=expkv}^^A
            \href{https://github.com/Skillmon/tex_expkv}
              {^^A
                \rmfamily
                \bfseries
                {\color{expkvgrey}e\kern-.05em x\kern-.05em}^^A
                \lower.493ex
                  \hbox{{\color{expkvgrey}P}\kern-.1em{\color{expkvred}k}}^^A
                \kern-.18em{\color{expkvred}v}^^A
              }^^A
            \EndAccSupp{}^^A
          }^^A
      }
      {expkv}^^A
  }
\newcommand*\expkvpkg[1]
  {^^A
    \texorpdfstring
      {^^A
        \mbox
          {^^A
            \BeginAccSupp{ActualText=expkv-#1}^^A
            \href{https://github.com/Skillmon/tex_expkv-#1}
              {^^A
                \rmfamily
                \bfseries
                {\color{expkvgrey}e\kern-.05em x\kern-.05em}^^A
                \lower.493ex
                  \hbox{{\color{expkvgrey}P}\kern-.1em{\color{expkvred}k}}^^A
                \kern-.18em{\color{expkvred}v}^^A
                {\color{expkvgrey}^^A
                  \kern.05em\rule[-.1ex]{.08em}{1.2ex}\kern.05em\textsc{#1}^^A
                }^^A
              }^^A
            \EndAccSupp{}^^A
          }^^A
      }
      {expkv-#1}^^A
  }
\newcommand*\expkvd{\expkvpkg{def}}
\newcommand*\expkvc{\expkvpkg{cs}}
\newcommand\kv{\meta{key}=\meta{value}}
\newcommand\key{\meta{key}}
\newcommand\val{\meta{value}}
\newcommand\set{\meta{set}}
\newcommand\enflong{\textcolor{black}{long}}
\newcommand\alllong{\textcolor{gray}{long}}
\newcommand\notlong{\textcolor{red!80!black}{long}}
\newcommand\prefixes[1]
  {^^A
    \hfill
    \ifcase\numexpr#1\relax\or
      \enflong\or
      \alllong\or
      \notlong\fi
  }
\hypersetup{linkcolor=red!80!black,urlcolor=purple!80!black}
\DoNotIndex{\def,\edef,\,,\=,\begingroup,\catcode,\chardef,\csname,\endcsname}
\DoNotIndex{\endgroup,\endinput,\errmessage,\expandafter,\input,\let,\long,\z@}
\DoNotIndex{\protected,\ProvidesFile,\ProvidesPackage,\relax,\space,\advance}
\DoNotIndex{\@,\unexpanded,\string,\expanded,\dimexpr,\global,\glueexpr,\hbox}
\DoNotIndex{\numexpr,\RequirePackage,\setbox,\the,\unless,\xdef,\gdef,\newcount}
\DoNotIndex{\number}
\DoNotIndex{\^,\@firstofone,\@firstoftwo,\@gobble,\@secondoftwo}
\DoNotIndex{\ifcsname}
\DoNotIndex{\ifx}
\DoNotIndex{\ifodd}
\DoNotIndex{\ifnum}
\DoNotIndex{\ifdefined}
\DoNotIndex{\iffalse}
\DoNotIndex{\iftrue}
\DoNotIndex{\else}
\DoNotIndex{\fi}
\@gobble\fi ^^A ignoring \ifx, \ifcsname, etc., but only one \fi
\@gobble\fi ^^A ignoring \ifx, \ifcsname, etc., but only one \fi
\@gobble\fi ^^A ignoring \ifx, \ifcsname, etc., but only one \fi
\@gobble\fi ^^A ignoring \ifx, \ifcsname, etc., but only one \fi
\@gobble\fi ^^A ignoring \ifx, \ifcsname, etc., but only one \fi
\@gobble\fi ^^A ignoring \ifx, \ifcsname, etc., but only one \fi
\@ifdefinable\gobbledocstriptag{\def\gobbledocstriptag#1>{}}
\makeatother
\begin{document}
  \title
    {^^A
      \texorpdfstring
        {^^A
          \huge\expkvc
          \\[\medskipamount]
          \Large define expandable \kv\ macros using \expkv
        }{expkv-cs - define expandable key=val macros using expkv}^^A
    }
  \date{\ekvcDate\space v\ekvcVersion}
  \author{Jonathan P. Spratte\thanks{\protect\randomize{jspratte@yahoo.de}}}
  \DocInput{expkv-cs.dtx}
\end{document}
%</driver>^^A=<<
% \fi
%
% \maketitle
% \renewcommand*\thefootnote{\arabic{footnote}}
%
% \begin{abstract}
% \noindent\parfillskip=0pt
% \expkvc\ provides two small interfaces to define expandable \kv\ macros using
% \expkv. It therefore lowers the entrance boundary to expandable \kv\ macros.
% The stylised name is \expkvc\ but the files use \file{expkv-cs}, this is due
% to CTAN-rules which don't allow \string| in package names since that is the
% pipe symbol in *nix shells.
% \end{abstract}
%
% \tableofcontents
%
% \begin{documentation}^^A>>=
%
% \section{Documentation}
%
% The \expkv\ package enables the new possibility of creating \kv\ macros which
% are fully expandable. The creation of such macros is however cumbersome for
% the average user. \expkvc\ tries to step in here. It provides interfaces to
% define \kv\ macros without worrying too much about the implementation. In case
% you're wondering now, the |cs| in \expkvc\ stands for control sequence,
% because |def| was already taken by \expkvd\ and ``control sequence'' is the
% term D.\,E.\,Knuth used in his \TeX book for named commands hence macros
% (though he also used the term ``macro''). So \expkvc\ defines control
% sequences for and with \expkv.
%
% There are two different approaches supported by this package. The first is
% splitting the keys up into individual arguments, the second is providing all
% the keys as a single argument to the underlying macro and getting an
% individual \val\ by using a hash. Well, actually there is no real hash, just
% some markers which are parsed, but this shouldn't be apparent to the user, the
% behaviour matches that of a hash-table.
%
% In addition to these two methods of defining a macro with primary keys a way
% to define secondary keys, which can reference the primary ones, is provided.
% These secondary keys don't correspond to an argument or an entry in the hash
% table directly but might come in handy for the average use case. Each macro
% has its own set of primary and secondary keys.
%
% A word of advice you should consider: If your macro doesn't have to be
% expandable (and often it doesn't) consider not using \expkvc. The interface
% has some overhead (though it still is fast -- check \autoref{sec:speed}) and
% the approach has its limits in versatility. If you don't need to be
% expandable, you should consider either defining your keys manually using
% \expkv\ or using \expkvd\ for convenience. Or you resort to another \kv\
% interface. Nevertheless setting up macros with \expkvc, especially with
% |\ekvcSplit|, is very convenient in my opinion, so if you just want to define
% a single macro with just a few keys this might be the way to go.
%
% \expkvc\ is usable as generic code and as a \LaTeX\ package. It'll
% automatically load \expkv\ in the same mode as well. To use it, just use one
% of
% \begin{lstlisting}
% \usepackage{expkv-cs} % LaTeX
% \input expkv-cs       % plainTeX
% \end{lstlisting}
%
%
% \subsection{Define Macros and Primary Keys}
%
% All macros defined with \expkvc\ have to be previously undefined or have the
% |\meaning| of |\relax|. This is necessary as there is no way to automatically
% undefine keys once they are set up (neither \expkv\ nor \expkvc\ keep track of
% defined keys) -- so to make sure there are no conflicts only new definitions
% are allowed (that's not the case for individual keys, only for frontend
% macros).
%
%
% \subsubsection{Primary Keys}\label{sec:primaries}
%
% In the following descriptions there will be one argument named \meta{primary
% keys}. This argument should be a \kv\ list where each \key\ will be one
% primary key and \val\ the associated initial value. By default all keys are
% defined short, but you can define long keys by prefixing \key\ with |long|
% (\emph{e.g.}, \texttt{long name=Jonathan P. Spratte}). You only need |long| if
% the key should be able to take a |\par| token. Note however that |long| keys
% are a microscopic grain faster (due to some internals of \expkvc). Only if at
% least one of the keys was |long| the \meta{cs} in the following defining
% macros will be |\long|. For obvious reasons there is no possibility to define
% a macro or key as |\protected|.
%
% At the moment \expkvc\ doesn't require any internal keys, but I can't foresee
% whether this will be the case in the future as well, as it might turn out that
% some features I deem useful can't be implemented without such internal keys.
% Because of this, please don't use key names starting with
% \texttt{EKVC\string|} as that should be the private name space.
%
%
% \subsubsection{Split}
%
% The split variants will provide the key values as separate arguments. This
% limits the number of keys for which this is truly useful.
%
% \begin{function}{\ekvcSplit}
%   \begin{syntax}
%     \cs{ekvcSplit}\meta{cs}\marg{primary keys}\marg{definition}
%   \end{syntax}
%   This defines \meta{cs} to be a macro taking one mandatory argument which
%   should contain a \kv\ list. The \meta{primary keys} will be defined for this
%   macro (see \autoref{sec:primaries}). The \meta{definition} is the code that
%   will be executed. You can access the \val\ of a \key\ by using a macro
%   parameter from |#1| to |#9|. The order of the macro parameters will be the
%   order provided in the \meta{primary keys} list (so |#1| is the \val\ of the
%   key defined first). With |\ekvcSplit| you can define macros using at most
%   nine primary keys.
% \end{function}
% \example The following defines a macro |\foo| that takes the keys |a| and |b|
% and outputs their values in a textual form:\\
% \begin{minipage}{.75\linewidth}
% \begin{lstlisting}
% \ekvcSplit\foo{a=a,b=b}{a is #1.\par b is #2.\par}
% \foo{}
% \foo{b=e}
% \end{lstlisting}
% \end{minipage}^^A
% \begin{exresult}[nobeforeafter,box align=center]{.25\linewidth}
%   \ekvcSplit\foo{a=a,b=b}{a is #1.\par b is #2.\par}
%   \foo{}
%   \foo{b=e}
% \end{exresult}
%
% \begin{function}{\ekvcSplitAndForward}
%   \begin{syntax}
%     \cs{ekvcSplitAndForward}\meta{cs}\marg{after}\marg{primary keys}
%   \end{syntax}
%   This defines \meta{cs} to be a macro taking one mandatory argument which
%   should contain a \kv\ list. You can use as many primary keys as you want
%   with this. The primary keys will be forwarded to \meta{after} as braced
%   arguments (as many as necessary for your primary keys). The order of the
%   braced arguments will be the order of your primary key definitions. In
%   \meta{after} you can use just a single control sequence, or some arbitrary
%   stuff which will be left in the input stream before your braced values (with
%   one set of braces stripped from \meta{after}), so both of the following
%   would be fine:
% \begin{lstlisting}
% \ekvcSplitAndForward\foo\foo@aux{keyA = A, keyB = B}
% \ekvcSplitAndForward\foo{\foo@aux{more args}}{keyA = A, keyB = B}
% \end{lstlisting}
% \end{function}
% In the first case |\foo@aux| should take at least two arguments (|keyA| and
% |keyB|), in the second case at least three (|more args|, |keyA|, and |keyB|).
%
% \begin{function}{\ekvcSplitAndUse}
%   \begin{syntax}
%     \cs{ekvcSplitAndUse}\meta{cs}\marg{primary keys}
%   \end{syntax}
%   This will roughly do the same as |\ekvcSplitAndForward|, but instead of
%   specifying what will be used after splitting the keys, \meta{cs} will use
%   what follows the \kv\ list. So its syntax will be
% \end{function}
%   \begin{syntax}
%     \meta{cs}\{\key=\val, \ldots\}\marg{after}
%   \end{syntax}
% and the code in |after| should expect at least as many arguments as the number
% of keys defined for \meta{cs}.
%
% \subsubsection{Hash}
%
% The hash variants will provide the key values as a single argument in which
% you can access specific values using a special macro. The implementation might
% be more convenient and scale better, \emph{but} it is much slower (for a
% primitive macro with a single key benchmarking was almost $1.7$ times slower,
% the root of which being the key access with |\ekvcValue|, not the parsing, and
% for a key access using |\ekvcValueFast| it was still about $1.2$ times
% slower). So if your macro uses less than ten primary keys, you should consider
% using the split approach.
%
% \begin{function}{\ekvcHash}
%   \begin{syntax}
%     \cs{ekvcHash}\meta{cs}\marg{primary keys}\marg{definition}
%   \end{syntax}
%   This defines \meta{cs} to be a macro taking one mandatory argument which
%   should contain a \kv\ list. You can use as many primary keys as you want.
%   The primary keys will be forwarded as a single argument containing every key
%   to the underlying macro. The underlying macro is defined as
%   \meta{definition}, in which you can access the \val\ of a \key\ by using
%   \texttt{\cs[no-index]{ekvcValue}\{\key\}\{\#1\}}.
% \end{function}
% \example This defines an equivalent macro like the |\foo| defined with
% |\ekvcSplit| earlier:\\
% \begin{minipage}{.75\linewidth}
% \begin{lstlisting}
% \ekvcHash\foo{a=a,b=b}{a is \ekvcValue{a}{#1}.\par
%                        b is \ekvcValue{b}{#1}.\par}
% \foo{}
% \foo{b=e}
% \end{lstlisting}
% \end{minipage}^^A
% \begin{exresult}[nobeforeafter,box align=center]{.25\linewidth}
%   \ekvcHash\foo{a=a,b=b}{a is \ekvcValue{a}{#1}.\par
%                          b is \ekvcValue{b}{#1}.\par}
%   \foo{}
%   \foo{b=e}
% \end{exresult}
%
% \begin{function}{\ekvcHashAndForward}
%   \begin{syntax}
%     \cs{ekvcHashAndForward}\meta{cs}\marg{after}\marg{primary keys}
%   \end{syntax}
%   This defines \meta{cs} to be a macro taking one mandatory argument which
%   should contain a \kv\ list. You can use as many primary keys as you want.
%   The primary keys will be forwarded as a single argument containing every key
%   to \meta{after}. You can use a single macro for \meta{after} or use some
%   arbitrary stuff, which will be left in the input stream before the hashed
%   \kv\ list with one set of braces stripped. In the
%   macro called in \meta{after} you can access the \val\ of a \key\ by using
%   \texttt{\cs[no-index]{ekvcValue}\{\key\}\{\#1\}} (or whichever argument the
%   hashed \kv\ list will be).
% \end{function}
%
% \begin{function}{\ekvcHashAndUse}
%   \begin{syntax}
%     \cs{ekvcHashAndUse}\meta{cs}\marg{primary keys}
%   \end{syntax}
%   This will roughly do the same as |\ekvcHashAndForward|, but instead of
%   specifying what will be used after hashing the keys, \meta{cs} will use what
%   follows the \kv\ list. So its syntax will be
% \end{function}
%   \begin{syntax}
%     \meta{cs}\{\key=\val, \ldots\}\marg{after}
%   \end{syntax}
%
% \begin{function}{\ekvcValue}
%   \begin{syntax}
%     \cs{ekvcValue}\{\key\}\marg{key list}
%   \end{syntax}
%   This is a safe way to access your keys in a hash variant. \key\
%   is the key which's \val\ you want to use out of the \meta{key list}.
%   \meta{key list} should be the key list argument forwarded to your underlying
%   macro by |\ekvcHash|, |\ekvcHashAndForward|, or |\ekvcHashAndUse|. It will
%   be tested whether the hash function to access that \key\ exists, the \key\
%   argument is not empty, and that the \meta{key list} really contains a \val\
%   of that \key. This macro needs exactly two steps of expansion and if used
%   inside of an |\edef| or |\expanded| context will protect the \val\ from
%   further expanding.
% \end{function}
%
% \begin{function}{\ekvcValueFast}
%   \begin{syntax}
%     \cs{ekvcValueFast}\{\key\}\marg{key list}
%   \end{syntax}
%   This behaves just like |\ekvcValue|, but \emph{without any} safety tests. As
%   a result this is about $1.4$ times faster \emph{but} will throw low level
%   \TeX\ errors eventually if the hash function isn't defined or the \key\
%   isn't part of the \meta{key list} (\emph{e.g.}, because it was defined as a
%   key for another macro -- all macros share the same hash function per \key).
%   Use it if you know what you're doing. This macro needs exactly three steps
%   of expansion in the no-errors case.
% \end{function}
%
% \begin{function}{\ekvcValueSplit}
%   \begin{syntax}
%     \cs{ekvcValueSplit}\{\key\}\marg{key list}\marg{next}
%   \end{syntax}
%   If you need a specific \key\ from a \meta{key list} more than once, it'll
%   be a good idea to only extract it once and from then on keep it as a
%   separate argument. Hence the macro \cs{ekvcValueSplit} will extract one
%   specific \key's value from the list and forward the remainder of the list
%   as the first and the \key's value as the second argument to \meta{next}, so
%   the result of this will be \meta{next}\marg{key list'}\marg{value} with
%   \meta{key list'} the remaining list. This is almost as fast as |\ekvcValue|
%   and runs the same tests. Keep in mind that you can't fetch for the same
%   \key\ again from \meta{key list'} as it got removed.
% \end{function}
% \example The following defines a macro |\foo| which will take three keys.
% Since the next parsing step will need the value of one of the keys multiple
% times we split that key off the list:\\
% \begin{minipage}{.75\linewidth}
% \begin{lstlisting}
% \ekvcHash\foo{a=a,b=b,c=c}
%   {\ekvcValueSplit{a}{#1}\foobar}
% \newcommand*\foobar[2]{a is #2.\par
%                        b is \ekvcValue{b}{#1}.\par
%                        c is \ekvcValue{c}{#1}.\par}
% \foo{}
% \end{lstlisting}
% \end{minipage}^^A
% \begin{exresult}[nobeforeafter,box align=center]{.25\linewidth}
%   \ekvcHash\foo{a=a,b=b,c=c}
%     {\ekvcValueSplit{a}{#1}\foobar}
%   \newcommand*\foobar[2]{a is #2.\par
%                          b is \ekvcValue{b}{#1}.\par
%                          c is \ekvcValue{c}{#1}.\par}
%   \foo{}
% \end{exresult}
%
% \begin{function}{\ekvcValueSplitFast}
%   \begin{syntax}
%     \cs{ekvcValueSplitFast}\{\key\}\marg{key list}\marg{next}
%   \end{syntax}
%   This behaves just like |\ekvcValueSplit|, but it won't run the same tests,
%   hence it is faster but more error prone, just like the relation between
%   |\ekvcValue| and |\ekvcValueFast|.
% \end{function}
%
%
% \subsection{Secondary Keys}\label{sec:secondaries}
%
% To remove some of the limitations with the approach that each primary key
% matches an argument or hash entry, you can define secondary keys. Those have
% to be defined for each macro but it doesn't matter whether that macro was a
% split or a hash variant. If a secondary key references another key it doesn't
% matter whether that other key is primary or secondary.
%
% Secondary keys can have a prefix (like |long|) which are called |p|-type
% prefix and must have a type (like |meta|) which are called |t|-type prefix.
% Some types might require some |p|-prefixes, while others might forbid those.
%
% Please keep in mind that key names shouldn't start with \texttt{EKVC\string|}.
%
% \begin{function}{\ekvcSecondaryKeys}
%   \begin{syntax}
%     \cs{ekvcSecondaryKeys}\meta{cs}\{\kv, \ldots\}
%   \end{syntax}
%   This is the front facing macro to define secondary keys. For the macro
%   \meta{cs} define \key\ to have definition \val. The general syntax for \key\
%   should be
%   \begin{quote}\ttfamily\small
%     \meta{prefix} \meta{name}
%   \end{quote}
%   Where \meta{prefix} is a space separated list of optional |p|-type prefixes
%   followed by one |t|-type prefix. The syntax of \val\ is dependent on the
%   used |t|-prefix.
% \end{function}
%
%
% \subsubsection{\texttt{p}-type Prefixes}
%
% There is only one |p|-prefix available, which is |long|.
%
% \begin{function}{long}
%   The following key will be defined |\long|.
% \end{function}
%
%
% \subsubsection{\texttt{t}-type Prefixes}
%
% If you're familiar with \expkvd\ you'll notice that the |t|-type prefixes
% provided here are much fewer. The expansion only concept doesn't allow for
% great variety in the auto-defined keys.
%
% The syntax examples of the |t|-prefixes will show which |p|-prefix will be
% automatically used by printing those black (\texttt{\enflong}), which
% will be available in grey (\texttt{\alllong}), and which will be disallowed in
% red (\texttt{\notlong}). This will be put flush right next to the syntax line.
%
% \begin{function}{meta}
%   \begin{syntax}
%     meta \key\ = \{\kv, \ldots\} \prefixes2
%   \end{syntax}
%   With a |meta| key you can set other keys. Whenever \key\ is used the keys in
%   the \kv\ list will be set to the values given there. You can use the \val\
%   given to \key\ by using |#1| in the \kv\ list. The keys in the \kv\ list can
%   be primary and secondary ones.
% \end{function}
%
% \begin{function}{nmeta}
%   \begin{syntax}
%     nmeta \key\ = \{\kv, \ldots\} \prefixes3
%   \end{syntax}
%   An |nmeta| key is like a |meta| key, but it doesn't take a value, so the
%   \kv\ list is static.
% \end{function}
%
% \begin{function}{alias}
%   \begin{syntax}
%     alias \key\ = \meta{key_2} \prefixes3
%   \end{syntax}
%   This assigns the definition of \meta{key_2} to \key. As a result \key\ is an
%   alias for \meta{key_2} behaving just the same. Both the value taking and the
%   |NoVal| version (that's \expkv\ slang for a key not accepting a value) will
%   be copied if they are defined when |alias| is used. Of course, \meta{key_2}
%   has to be defined, be it as a primary or secondary one.
% \end{function}
%
% \begin{function}{default}
%   \begin{syntax}
%     default \key\ = \marg{default} \prefixes3
%   \end{syntax}
%   If \key\ is a defined value taking key, you can define a |NoVal| version
%   with this that will behave as if \key\ was given \meta{default} as its
%   \val. Note that this doesn't change the initial values of primary keys set
%   at definition time in |\ekvcSplit| and friends. \key\ can be a primary or
%   secondary key.
% \end{function}
%
% \begin{function}{flag-bool}
%   \begin{syntax}
%     flag-bool \key\ = \meta{cs} \prefixes3
%   \end{syntax}
%   This is a secondary key that doesn't directly involve any of the primary or
%   secondary keys.
%   This defines \key\ to take a value, which should be either |true| or
%   |false|, and set the flag called \meta{cs} accordingly as a boolean.
%   If \meta{cs} isn't defined yet it will be initialised as a flag.
%   Please also read \autoref{sec:flags}.
% \end{function}
%
% \begin{function}{flag-true,flag-false}
%   \begin{syntax}
%     flag-true \key\ = \meta{cs} \prefixes3
%   \end{syntax}
%   This is a secondary key that doesn't directly involve any of the primary or
%   secondary keys.
%   This defines \key\ to take no value and set the flag called \meta{cs} to
%   |true| or |false|, respectively.
%   If \meta{cs} isn't defined yet it will be initialised as a flag.
%   Please also read \autoref{sec:flags}.
% \end{function}
%
% \begin{function}{flag-raise}
%   \begin{syntax}
%     flag-raise \key\ = \meta{cs} \prefixes3
%   \end{syntax}
%   This is a secondary key that doesn't directly involve any of the primary or
%   secondary keys.
%   This defines \key\ to take no value and raise the flag called \meta{cs}.
%   If \meta{cs} isn't defined yet it will be initialised as a flag.
%   Please also read \autoref{sec:flags}.
% \end{function}
%
%
% \subsection{Changing the Initial Values}\label{sec:change}
%
% \begin{function}{\ekvcChange}
%   \begin{syntax}
%     \cs{ekvcChange}\meta{cs}\{\kv,\ldots\}
%   \end{syntax}
%   This processes the \kv\ list (just like it would be processed by \meta{cs})
%   to set new defaults for the macro \meta{cs} defined with \expkvc\ (meaning
%   the values used if you don't provide anything at use time). Inside the \kv\
%   list both primary and secondary keys can be used. If \meta{cs} was defined
%   |\long| earlier it will still be |\long|, every other \TeX\ prefix will be
%   stripped (but \expkvc\ doesn't support them anywhere else so that should be
%   fine). The resulting new defaults will be stored inside the \meta{cs}
%   locally (just as the original defaults were).
% \end{function}
%
% Consider the following example:\\
% \begin{minipage}[c]{.75\linewidth}
% \begin{lstlisting}
% \ekvcSplit\foo{a=a,b=b}{a is #1.\par b is #2.\par}
% \foo{} % initial defaults
% \ekvcChange\foo{b=B}
% \foo{} % new defaults
% \ekvcSecondaryKeys\foo{meta c={a={#1},b={#1}}}
% \ekvcChange\foo{c=c}
% \foo{} % newer defaults
% \end{lstlisting}
% \end{minipage}^^A
% \begin{exresult}[nobeforeafter,box align=center]{.25\linewidth}
%   \ekvcSplit\foo{a=a,b=b}{a is #1.\par b is #2.\par}
%   \foo{}
%   \ekvcChange\foo{b=B}
%   \foo{}
%   \ekvcSecondaryKeys\foo{meta c={a={#1},b={#1}}}
%   \ekvcChange\foo{c=c}
%   \foo{}
% \end{exresult}
%
% As a result with this the typical setup macro could be implemented:
% \begin{lstlisting}
% \ekvcHashAndUse\foo{key=a,key=b}
% \newcommand\foosetup{\ekvcChange\foo}
% \end{lstlisting}
% Of course the usage is limited to a single macro |\foo|, hence this might be
% not as powerful as similar macros used with other \kv\ interfaces. But at
% least a few similar macros could be grouped using the same key parsing macro
% internally.
%
%
% \subsection{Flags}\label{sec:flags}
%
% The idea of flags is taken from \pkg{expl3}. They provide a way to store
% numerical information expandably, however only incrementing and accessing
% works expandably, decrementing is unexpandable. A flag has a height, which
% is a numerical value, and which can be raised by $1$. Flags come at a high
% computational cost (accessing them is slow and they require more memory than
% normal \TeX\ data types like registers, both getting linearly worse with the
% height), so don't use them if not necessary.
%
% The state of flags is always changed locally to the current group, but not to
% the current macro, so if you're using one of the |t|-types involving flags
% bear in mind that they can affect other macros using the same flags at the
% current group level!
%
% \expkvc\ provides some macros to access, alter, and use flags. Flags of
% \expkvc\ don't share a name space with the flags of \pkg{expl3}.
%
% \begin{function}{\ekvcFlagNew}
%   \begin{syntax}
%     \cs{ekvcFlagNew}\meta{flag}
%   \end{syntax}
%   This initialises the macro \meta{flag} as a new flag. It isn't checked
%   whether the macro \meta{flag} is currently undefined.  A \meta{flag} will
%   expand to the flag's current height with a trailing space (so you can use it
%   directly with |\ifnum| for example and it will terminate the number scanning
%   itself).
% \end{function}
%
%  All other macros dealing with flags take as a parameter a macro defined as a
%  \meta{flag} with |\ekvcFlagNew|.
%
% \begin{function}{\ekvcFlagHeight}
%   \begin{syntax}
%     \cs{ekvcFlagHeight}\meta{flag}
%   \end{syntax}
%   This expands to the current height of \meta{flag} in a single step of
%   expansion (without a trailing space).
% \end{function}
%
% \begin{function}{\ekvcFlagRaise}
%   \begin{syntax}
%     \cs{ekvcFlagRaise}\meta{flag}
%   \end{syntax}
%   This expandably raises the height of \meta{flag} by $1$.
% \end{function}
%
% \begin{function}{\ekvcFlagSetTrue,\ekvcFlagSetFalse}
%   \begin{syntax}
%     \cs{ekvcFlagSetTrue}\meta{flag}
%   \end{syntax}
%   By interpreting an even value as false and an odd value as true we can use a
%   flag as a boolean. This expandably sets \meta{flag} to true or false,
%   respectively, by raising it if necessary.
% \end{function}
%
% \begin{function}{\ekvcFlagIf}
%   \begin{syntax}
%     \cs{ekvcFlagIf}\meta{flag}\marg{true}\marg{false}
%   \end{syntax}
%   This interprets a \meta{flag} as a boolean and expands to either \meta{true}
%   or \meta{false}.
% \end{function}
%
% \begin{function}{\ekvcFlagIfRaised}
%   \begin{syntax}
%     \cs{ekvcFlagIfRaised}\meta{flag}\marg{true}\marg{false}
%   \end{syntax}
%   This tests whether the \meta{flag} is raised, meaning it has a height
%   greater than zero, and if so expands to \meta{true} else to \meta{false}.
% \end{function}
%
% \begin{function}{\ekvcFlagReset}
%   \begin{syntax}
%     \cs{ekvcFlagReset}\meta{flag}
%   \end{syntax}
%   This resets a flag (so restores its height to $0$). This operation is
%   \emph{not} expandable and done locally. If you really intend to use flags
%   you can reset them every now and then to keep the performance hit low.
% \end{function}
%
% \begin{function}{\ekvcFlagGetHeight}
%   \begin{syntax}
%     \cs{ekvcFlagGetHeight}\meta{flag}\marg{next}
%   \end{syntax}
%   This retrieves the current height of the \meta{flag} and provides it as a
%   braced argument to \meta{next}, leaving \meta{next}\marg{height} in the
%   input stream.
% \end{function}
%
% \begin{function}{\ekvcFlagGetHeights}
%   \begin{syntax}
%     \cs{ekvcFlagGetHeights}\marg{flag-list}\marg{next}
%   \end{syntax}
%   This retrieves the current height of each \meta{flag} in the
%   \meta{flag-list} and provides them as a single argument to \meta{next}.
%   Inside that argument each height is enclosed in a set of braces
%   individually. The \meta{flag-list} is just a single argument containing the
%   \meta{flag}s.
%   So a usage like
%   \texttt
%     {^^A
%       \cs[no-index]{ekvcFlagGetHeights}\hskip0pt^^A
%       \{\cs[no-index]{myflagA}\cs[no-index]{myflagB}\}\hskip0pt^^A
%       \{\cs[no-index]{stuff}\}^^A
%     }
%   will expand to
%   \texttt{\cs[no-index]{stuff}\{\marg{height-A}\marg{height-B}\}}.
% \end{function}
%
%
% \subsection{Further Examples}
%
% How could a documentation be a good documentation without some basic examples?
% Say we want to define a small macro expanding to some character description
% (who knows why this has to be expandable?). A character description will not
% have too many items to it, so we use |\ekvcSplit| (the comments with the
% parameter numbers are of course not necessary and just ease reading the
% example).
% \begin{lstlisting}
% \ekvcSplit\character
%   {
%     name=John Doe,            % #1
%     age=any,                  % #2
%     nationality=the Universe, % #3
%     hobby=to exist,           % #4
%     type=Mister,              % #5
%     pronoun=He,               % #6
%     possessive=his,           % #7
%   }
%   {#1 is a #5 from #3. #6 is of #2 age and #7 hobby is #4.\par}
% \end{lstlisting}
% Also we want to give some short cuts so that it's easier to describe several
% persons.
% \begin{lstlisting}
% \ekvcSecondaryKeys\character
%   {
%     alias pro = pronoun,
%     alias pos = possessive,
%     nmeta me =
%       {
%         name=Jonathan,
%         age=a young,
%         nationality=Germany,
%         hobby=\TeX\ coding,
%       },
%     meta lady =
%       {type=Lady, pronoun=She, possessive=her, name=Jane Doe, #1},
%     nmeta paulo =
%       {
%         name=Paulo,
%         type=duck,
%         age=a young,
%         nationality=Brazil,
%         hobby=to quack,
%       }
%   }
% \end{lstlisting}
% Now we can describe people using
% \begin{lstlisting}
% \character{}
% \character{me}
% \character{paulo}
% \character
%   {lady={name=Evelyn,nationality=Ireland,age=the best,hobby=reading}}
% \character
%   {
%     name=Our sun, type=star, nationality=our solar system, pro=It,
%     age=an old, pos=its, hobby=shining
%   }
% \end{lstlisting}
% As one might see, the |lady| key could actually have been an |nmeta| key as
% well, as all that is done with the argument is using it as a \kv\ list.
%
% The result of only the first two usages would be:
% \begin{exresult}{\linewidth}
%   \ekvcSplit\character
%     {
%       name=John Doe,
%       age=any,
%       nationality=the Universe,
%       hobby=to exist,
%       type=Mister,
%       pronoun=He,
%       possessive=his,
%     }
%     {#1 is a #5 from #3. #6 is of #2 age and #7 hobby is #4.\par}
%   \ekvcSecondaryKeys\character
%     {
%       alias pro = pronoun,
%       alias pos = possessive,
%       nmeta me =
%         {
%           name=Jonathan,
%           age=a young,
%           nationality=Germany,
%           hobby=\TeX\ coding,
%         },
%       meta lady =
%         {type=Lady, pronoun=She, possessive=her, name=Jane Doe, #1},
%       nmeta paulo =
%         {
%           name=Paulo,
%           type=duck,
%           age=a young,
%           nationality=Brazil,
%           hobby=to quack,
%         }
%     }
%   \character{}
%   \character{me}
% \end{exresult}
%
% Using \pkg{xparse} or \expkv's \cs{ekvoptarg} or \cs{ekvoptargTF} and
% forwarding arguments one can easily define \kv\ macros with actual optional
% and mandatory arguments as well. A small nonsense example (which should
% perhaps use |\ekvcSplitAndForward| instead of |\ekvcHashAndForward| since it
% only uses four keys and one other argument -- and isn't expandable since it
% uses a \env{tabular} environment, so it would've been better to use a more
% feature rich \kv\ interface most likely, \emph{e.g.}, the one provided by
% \expkvd):
% \begin{lstlisting}
% \makeatletter
% \newcommand*\nonsense{\ekvoptarg\nonsense@a{}}
% \ekvcHashAndForward\nonsense@a\nonsense@b
%   {
%     keyA = A,
%     keyB = B,
%     keyC = c,
%     keyD = d,
%   }
% \newcommand*\nonsense@b[2]
%   {%
%     \begin{tabular}{lll}
%       key & A & \ekvcValue{keyA}{#1} \\
%           & B & \ekvcValue{keyB}{#1} \\
%           & C & \ekvcValue{keyC}{#1} \\
%           & D & \ekvcValue{keyD}{#1} \\
%       \multicolumn{2}{l}{mandatory} & #2 \\
%     \end{tabular}%
%   }
% \makeatother
% \end{lstlisting}
% And then we would be able to do some nonsense
% \begin{lstlisting}
% \nonsense{}
% \nonsense[keyA=hihi]{haha}
% \nonsense[keyA=hihi, keyB=A]{hehe}
% \nonsense[keyC=huhu, keyA=hihi, keyB=A]{haha}
% \end{lstlisting}
% resulting in
% \begin{exresult}{\linewidth}
%   \makeatletter
%   \newcommand*\nonsense{\ekvoptarg\nonsense@a{}}
%   \ekvcHashAndForward\nonsense@a\nonsense@b
%     {
%       keyA = A,
%       keyB = B,
%       keyC = c,
%       keyD = d,
%     }
%   \newcommand*\nonsense@b[2]
%     {%
%       \begin{tabular}{lll}
%         key & A & \ekvcValue{keyA}{#1} \\
%             & B & \ekvcValue{keyB}{#1} \\
%             & C & \ekvcValue{keyC}{#1} \\
%             & D & \ekvcValue{keyD}{#1} \\
%         \multicolumn{2}{l}{mandatory} & #2 \\
%       \end{tabular}%
%     }
%   \makeatother
%   \nonsense{}
%   \nonsense[keyA=hihi]{haha}
%   \nonsense[keyA=hihi, keyB=A]{hehe}
%   \nonsense[keyC=huhu, keyA=hihi, keyB=A]{haha}
% \end{exresult}
%
%
% \subsection{Freedom for Keys!}
%
% If this was the \TeX book this subsection would have a double bend sign. Not
% because it is overly complicated, but because it shows things which could
% break \expkvc's expandability and its alignment safety. This is for
% experienced users wanting to get the most flexibility and knowing what they
% are doing.
%
% In case you're wondering, it is possible to define other keys than the
% primaries and the secondary types listed in \autoref{sec:secondaries} for a
% macro defined with \expkvc\ by using the low-level interface of \expkv\ or
% even the interface provided by \expkvd. The set name used for \expkvc's keys
% is the macro name, including the leading backslash, or more precisely
% |\string|\meta{cs} is used. This can be exploited to define additional keys
% with arbitrary code. Consider the following \emph{bad} example:
% \begin{lstlisting}
% \ekvcSplit\foo{a=A,b=B}{a is #1.\par b is #2}
% \protected\ekvdef{\string\foo}{c}{\def\fooC{#1}}
% \end{lstlisting}
% This would define a key named |c| that will store its value inside a macro.
% The issue with this is that this can't be done expandably. As a result, the
% macro |\foo| isn't always expandable any more (not that bad if this was never
% required; killjoy if it was) and as soon as the key |c| is used, it is also no
% longer alignment safe\footnote{This means that the \kv-list can't contain
%   alignment markers that are not inside an additional set of braces if used
%   inside a \TeX\ alignment.}
% (might be bad depending on the usage).
%
% So why do I show you this? Because we could as well do something useful like
% create a key that pre-parses the input and after that passes the parsed value
% on. This parsing would have to be completely expandable though. For the
% pass-on part we can use the following function:
%
% \begin{function}{\ekvcPass}
%   \begin{syntax}
%     \cs{ekvcPass}\meta{cs}\marg{key}\marg{value}
%   \end{syntax}
%   This passes \meta{value} on to \meta{key} for the \expkvc-macro \meta{cs}.
%   It should be used inside the key parsing of a macro defined with \expkvc,
%   else this most likely results in a low level \TeX\ error.
% \end{function}
%
% With this we could for example split the value of a key at a hyphen and pass
% the parts to different keys:\\
% \begin{minipage}[c]{.75\linewidth}
% \begin{lstlisting}
% \ekvcSplit\foo{a=A,b=B}{a is #1.\par b is #2.\par}
% \ekvdef{\string\foo}{c}{\fooSplit#1\par}
% \def\fooSplit#1-#2\par
%   {\ekvcPass\foo{a}{#1}\ekvcPass\foo{b}{#2}}
% \foo{}
% \foo{c=1-2}
% \end{lstlisting}
% \end{minipage}
% \begin{exresult}[nobeforeafter,box align=center]{.25\linewidth}
%   \ekvcSplit\foo{a=A,b=B}{a is #1.\par b is #2.\par}
%   \ekvdef{\string\foo}{c}{\fooSplit#1\par}
%   \def\fooSplit#1-#2\par{\ekvcPass\foo{a}{#1}\ekvcPass\foo{b}{#2}}
%   \foo{}
%   \foo{c=1-2}
% \end{exresult}
%
%
% \subsection{Speed Considerations}\label{sec:speed}
%
% As already mentioned in the introduction there are some speed considerations
% implied if you choose to define macros via \expkvc. However the overhead isn't
% the factor which should hinder you to use \expkvc\ if you found a reasonable
% use case. The key-parsing is still faster than with most other \kv\ packages
% (see the ``Comparisons'' subsection in the \expkv\ documentation).
%
% The speed considerations in this subsection use the first example in this
% documentation as the benchmark. So we have seven keys and a short sentence
% which should be typeset. For comparisons I use the following equivalent
% \expkvd\ definitions. Each result is the average between changing no keys from
% their initial values and altering four. Furthermore I'll compare three
% variants of \expkvc\ with the \expkvd\ definitions, namely the split example
% from above, a hash variant using |\ekvcValue| and a hash variant using
% |\ekvcValueFast|.
% \begin{lstlisting}
% \usepackage{expkv-def}
% \ekvdefinekeys{keys}
%   {%
%     ,store   name         = \KEYSname
%     ,initial name         = John Doe
%     ,store   age          = \KEYSage
%     ,initial age          = any
%     ,store   nationality  = \KEYSnationality
%     ,initial nationality  = the Universe
%     ,store   hobby        = \KEYShobby
%     ,initial hobby        = to exist
%     ,store   type         = \KEYStype
%     ,initial type         = Mister
%     ,store   pronoun      = \KEYSpronoun
%     ,initial pronoun      = He
%     ,store   possessive   = \KEYSpossessive
%     ,initial possessive   = his
%   }
% \newcommand*\KEYS[1]
%   {%
%     \begingroup
%       \ekvset{keys}{#1}%
%       \KEYSname\ is a \KEYStype\ from \KEYSnationality. \KEYSpronoun\ is
%       of \KEYSage\ age and \KEYSpossessive\ hobby is \KEYShobby.%
%     \endgroup
%   }
% \end{lstlisting}
%
% The first comparison removes the typesetting part from all the definitions, so
% that only the key parsing is compared. In this comparison the |\ekvcValue|
% and |\ekvcValueFast| variants will not differ, as they are exactly the same
% until the key usage. We find that the split approach is $1.4$ times slower
% than the \expkvd\ setup and the hash variants end up in the middle at $1.17$
% times slower.
%
% Next we put the typesetting part back in. Every call of the macros will
% typeset the sentences into a box register in horizontal mode. With the
% typesetting part (which includes the accessing of values) the fastest remains
% the \expkvd\ definitions, but split is close at $1.16$ times slower, followed
% by the hash variant with fast accesses at $1.36$ times slower, and the safe
% hash access variant ranks in the slowest $1.8$ times slower than \expkvd.
%
% Just in case you're wondering now, a simple macro taking seven arguments is
% $30$ to $40$ times faster than any of those in the argument grabbing and \kv\
% parsing part and only $1.5$ to $2.8$ times faster if the typesetting part is
% factored in. So the real choke isn't the parsing.
%
% So to summarize this, if you have a reasonable use case for expandable \kv\
% parsing macros you should go on and define them using \expkvc. If you have a
% reasonable use case for \kv\ parsing macros but defining them expandable isn't
% necessary for your use you should take advantage of the greater flexibility of
% non-expandable \kv\ setups (but if you're after maximum speed there aren't
% that many \kv\ parsers beating \expkvc). And if you are after maximum
% performance maybe ditching the \kv\ interface altogether is a good idea, but
% depending on the number of arguments your interface might get convoluted.
%
%
% \subsection{Useless Macros}
%
% Perhaps these macros aren't completely useless, but I figured from a user's
% point of view I wouldn't know what I should do with these.
%
% \begin{function}{\ekvcDate,\ekvcVersion}
%   These two macros store the version and the date of the package/generic code.
% \end{function}
%
%
% \subsection{Bugs}
%
% Of course I don't think there are any bugs (who would knowingly distribute
% buggy software as long as he isn't a multi-million dollar corporation?). But
% if you find some please let me know. For this one might find my email address
% on the first page or file an issue on Github:
% \url{https://github.com/Skillmon/tex_expkv-cs}
%
%
% \subsection{License}
%
% Copyright \textcopyright\ 2020\unless\ifnum\year=2020--\the\year\fi\
% Jonathan P. Spratte
%
% \medskip\noindent
% This work may be distributed and/or modified under the conditions of the
% \LaTeX\ Project Public License (LPPL), either version 1.3c of this license or
% (at your option) any later version. The latest version of this license is in
% the file:
%
% \url{http://www.latex-project.org/lppl.txt}
%
% \noindent
% This work is ``maintained'' (as per LPPL maintenance status) by
%
% \mbox{Jonathan P. Spratte}.
%
% \end{documentation}^^A=<<
%
% \begin{implementation}^^A>>=
%
% \clearpage
%
% \section{Implementation}^^A>>=
%^^A the LaTeX package >>=
% \subsection{The \LaTeX\ Package}
% Just like for \expkv\ we provide a small \LaTeX\ package that sets up things
% such that we behave nicely on \LaTeX\ packages and files system. It'll
% |\input| the generic code which implements the functionality.
% \gobbledocstriptag
%<*pkg>
%    \begin{macrocode}
\RequirePackage{expkv}
\def\ekvc@tmp
  {%
    \ProvidesFile{expkv-cs.tex}%
      [%
        \ekvcDate\space v\ekvcVersion\space
        define expandable key=val macros using expkv%
      ]%
  }
% \iffalse meta-comment
%
% File: expkv-cs.dtx Copyright (C) 2020-2021 Jonathan P. Spratte
%
% This work  may be  distributed and/or  modified under  the conditions  of the
% LaTeX Project Public License (LPPL),  either version 1.3c  of this license or
% (at your option) any later version.  The latest version of this license is in
% the file:
%
%   http://www.latex-project.org/lppl.txt
%
% ------------------------------------------------------------------------------
%
%<*driver>^^A>>=
\def\nameofplainTeX{plain}
\ifx\fmtname\nameofplainTeX\else
  \expandafter\begingroup
\fi
\input l3docstrip.tex
\askforoverwritefalse
\preamble

--------------------------------------------------------------
expkv-cs -- define expandable key=val macros using expkv
E-mail: jspratte@yahoo.de
Released under the LaTeX Project Public License v1.3c or later
See http://www.latex-project.org/lppl.txt
--------------------------------------------------------------

Copyright (C) 2020-2021 Jonathan P. Spratte

This  work may be  distributed and/or  modified under  the conditions  of the
LaTeX Project Public License (LPPL),  either version 1.3c  of this license or
(at your option) any later version.  The latest version of this license is in
the file:

  http://www.latex-project.org/lppl.txt

This work is "maintained" (as per LPPL maintenance status) by
  Jonathan P. Spratte.

This work consists of the file  expkv-cs.dtx
and the derived files           expkv-cs.pdf
                                expkv-cs.sty
                                expkv-cs.tex

\endpreamble
% stop docstrip adding \endinput
\postamble
\endpostamble
\generate{\file{expkv-cs.sty}{\from{expkv-cs.dtx}{pkg}}}
\generate{\file{expkv-cs.tex}{\from{expkv-cs.dtx}{tex}}}
\ifx\fmtname\nameofplainTeX
  \expandafter\endbatchfile
\else
  \expandafter\endgroup
\fi
%
\IfFileExists{expkv-cs.tex}{% \iffalse meta-comment
%
% File: expkv-cs.dtx Copyright (C) 2020-2021 Jonathan P. Spratte
%
% This work  may be  distributed and/or  modified under  the conditions  of the
% LaTeX Project Public License (LPPL),  either version 1.3c  of this license or
% (at your option) any later version.  The latest version of this license is in
% the file:
%
%   http://www.latex-project.org/lppl.txt
%
% ------------------------------------------------------------------------------
%
%<*driver>^^A>>=
\def\nameofplainTeX{plain}
\ifx\fmtname\nameofplainTeX\else
  \expandafter\begingroup
\fi
\input l3docstrip.tex
\askforoverwritefalse
\preamble

--------------------------------------------------------------
expkv-cs -- define expandable key=val macros using expkv
E-mail: jspratte@yahoo.de
Released under the LaTeX Project Public License v1.3c or later
See http://www.latex-project.org/lppl.txt
--------------------------------------------------------------

Copyright (C) 2020-2021 Jonathan P. Spratte

This  work may be  distributed and/or  modified under  the conditions  of the
LaTeX Project Public License (LPPL),  either version 1.3c  of this license or
(at your option) any later version.  The latest version of this license is in
the file:

  http://www.latex-project.org/lppl.txt

This work is "maintained" (as per LPPL maintenance status) by
  Jonathan P. Spratte.

This work consists of the file  expkv-cs.dtx
and the derived files           expkv-cs.pdf
                                expkv-cs.sty
                                expkv-cs.tex

\endpreamble
% stop docstrip adding \endinput
\postamble
\endpostamble
\generate{\file{expkv-cs.sty}{\from{expkv-cs.dtx}{pkg}}}
\generate{\file{expkv-cs.tex}{\from{expkv-cs.dtx}{tex}}}
\ifx\fmtname\nameofplainTeX
  \expandafter\endbatchfile
\else
  \expandafter\endgroup
\fi
%
\IfFileExists{expkv-cs.tex}{\input{expkv-cs.tex}}{}
\ProvidesFile{expkv-cs.dtx}
  [\csname ekvcDate\endcsname\ define expandable key=val macros using expkv]
\PassOptionsToPackage{full}{textcomp}
\documentclass{l3doc}
\RequirePackage[oldstylenums,nott]{kpfonts}
\input{glyphtounicode}
\pdfgentounicode=1
\RequirePackage{tcolorbox}
\newtcolorbox{exresult}[2][]
  {colback=expkvgrey!10!white,colframe=expkvgrey,fontupper=\small,width={#2},#1}
\newtcbox\exres[1][]
  {
     colback=expkvgrey!10!white
    ,colframe=expkvgrey
    ,size=small
    ,nobeforeafter
    ,tcbox raise base
    ,fontupper=\small
    ,#1
  }
\RequirePackage{listings}
\RequirePackage{booktabs}
\RequirePackage{array}
\RequirePackage{collcell}
\RequirePackage{siunitx}
\RequirePackage{xcolor}
\RequirePackage{caption}
\RequirePackage{microtype}
\RequirePackage{accsupp}
\lstdefinelanguage{expkvc}[primitive]{TeX}
  {
    ,moretexcs=[4]^^A e-TeX
      {
        expanded,
        numexpr,
        protected,
      }
    ,moretexcs=[5]^^A plain/LaTeX
      {
        approx,
        begin,
        empty,
        item,
        LaTeX,
        makeatletter,makeatother,multicolumn,
        newcommand,newdimen,
        RequirePackage,
        rule,
        TeX,
        textit,texttt,
        usepackage,
      }
    ,moretexcs=[6]^^A used packages
      {
        ^^A xfp
        fpeval,
        ^^A keyval
        setkeys,
        ^^A yax
        defactiveparameter,storevalue,setparameterlist
      }
    ,moretexcs=[2]^^A expkv-cs macros
      {
        ekvcSplit,ekvcSplitAndForward,ekvcSplitAndUse,
        ekvcHash,ekvcHashAndForward,ekvcHashAndUse,
        ekvcValue,ekvcValueFast,ekvcValueSplit,ekvcValueSplitFast,
        ekvcSecondaryKeys,
        ekvcChange,
        ekvcFlagNew,
        ekvcFlagHeight,
        ekvcFlagRaise,
        ekvcFlagSetTrue,ekvcFlagSetFalse,
        ekvcFlagIf,ekvcFlagIfRaised,
        ekvcFlagReset,
        ekvcFlagGetHeight,ekvcFlagGetHeights,
        ekvcPass
      }
    ,moretexcs=[3]^^A expkv-pkg macros
      {
        ^^A expkv
        ekvdef,ekvdefNoVal,^^A
        ekvlet,ekvletNoVal,ekvletkv,ekvletkvNoVal,^^A
        ekvdefunknown,ekvdefunknownNoVal,^^A
        ekvredirectunknown,ekvredirectunknownNoVal,^^A
        ekvset,ekvsetSneaked,^^A
        ekvsetdef,ekvsetSneakeddef,ekvsetdefSneaked,^^A
        ekvparse,^^A
        ekvVersion,ekvDate,^^A
        ekvifdefined,ekvifdefinedNoVal,ekvifdefinedset,^^A
        ekvbreak,ekvbreakPreSneak,ekvbreakPostSneak,^^A
        ekvsneak,ekvsneakPre,^^A
        ekvchangeset,^^A
        ekvoptarg,ekvoptargTF,^^A
        ekverr,
        ^^A expkv-def
        ekvdefinekeys
      }
  }
\colorlet{codeparam}{cyan!65!black}
\def\mylstwd{.55em}
\lstset
  {
    ,language=expkvc
    ,flexiblecolumns=false
    ,basewidth=\mylstwd
    ,gobble=2
    ,basicstyle=\fontfamily{jkp}\itshape
    ,texcsstyle=*[1]{\bfseries\color{expkvgrey}} ^^A primitives
    ,texcsstyle=*[2]{\bfseries\color{expkvred}}  ^^A expkv
    ,texcsstyle=*[3]{\color{expkvred}}           ^^A expkv-pkg
    ,texcsstyle=*[4]{\bfseries\color{expkvgrey}} ^^A e-TeX
    ,texcsstyle=*[5]{\bfseries\color{expkvgrey}} ^^A plain/LaTeX macros
    ,texcsstyle=*[6]{}                           ^^A macros of other packages
    ,commentstyle=\color[gray]{0.4}
    ,literate=
              {\{} {{\CodeUpBf\{}}{1}
              {\}} {{\CodeUpBf\}}}{1}
              {$}  {{\CodeUpBf\$}}{1}
              {[}  {{\CodeUp[}}{1}
              {]}  {{\CodeUp]}}{1}
              {(}  {{\CodeUp(}}{1}
              {)}  {{\CodeUp)}}{1}
              {*}  {{$*$}}{1}
              {1}  {{\CodeUp{1}}}{1}
              {2}  {{\CodeUp{2}}}{1}
              {3}  {{\CodeUp{3}}}{1}
              {4}  {{\CodeUp{4}}}{1}
              {5}  {{\CodeUp{5}}}{1}
              {6}  {{\CodeUp{6}}}{1}
              {7}  {{\CodeUp{7}}}{1}
              {8}  {{\CodeUp{8}}}{1}
              {9}  {{\CodeUp{9}}}{1}
              {0}  {{\CodeUp{0}}}{1}
              {##} {{\CodeColored{codeparam}{1}{\#}}}{1}
              {##1}{{\CodeColored{codeparam}{2}{\#1}}}{2}
              {##2}{{\CodeColored{codeparam}{2}{\#2}}}{2}
              {##3}{{\CodeColored{codeparam}{2}{\#3}}}{2}
              {##4}{{\CodeColored{codeparam}{2}{\#4}}}{2}
              {##5}{{\CodeColored{codeparam}{2}{\#5}}}{2}
              {##6}{{\CodeColored{codeparam}{2}{\#6}}}{2}
              {##7}{{\CodeColored{codeparam}{2}{\#7}}}{2}
              {##8}{{\CodeColored{codeparam}{2}{\#8}}}{2}
              {##9}{{\CodeColored{codeparam}{2}{\#9}}}{2}
              {<key>}{{$\langle$}key{$\rangle$}}{5}
              {<set>}{{$\langle$}set{$\rangle$}}{5}
              {<set1>}{{$\langle$}set1{$\rangle$}}{6}
              {<set2>}{{$\langle$}set2{$\rangle$}}{6}
    ^^A,literate=*{<key>}{\key}{4}{<set>}{\set}{4}
  }
\newcommand*\CodeColored[3]{\textcolor{#1}{\makebox[\dimexpr\mylstwd*#2]{#3}}}
\newcommand*\CodeUpBf[1]{\makebox[\mylstwd]{\textup{\textbf{#1}}}}
\newcommand*\CodeUp[1]{\makebox[\mylstwd]{\textup{#1}}}
\RequirePackage{randtext}
\let\metaORIG\meta
\protected\def\meta #1{\texttt{\metaORIG{#1}}}
\renewcommand*\thefootnote{\fnsymbol{footnote}}
\definecolor{expkvred}{HTML}{9F393D}
\colorlet{expkvgrey}{black!75}
\makeatletter
\newcommand*\example{\par\smallskip\noindent\textit{Example:} \ignorespaces}
\newcommand*\expkv
  {^^A
    \texorpdfstring
      {^^A
        \mbox
          {^^A
            \BeginAccSupp{ActualText=expkv}^^A
            \href{https://github.com/Skillmon/tex_expkv}
              {^^A
                \rmfamily
                \bfseries
                {\color{expkvgrey}e\kern-.05em x\kern-.05em}^^A
                \lower.493ex
                  \hbox{{\color{expkvgrey}P}\kern-.1em{\color{expkvred}k}}^^A
                \kern-.18em{\color{expkvred}v}^^A
              }^^A
            \EndAccSupp{}^^A
          }^^A
      }
      {expkv}^^A
  }
\newcommand*\expkvpkg[1]
  {^^A
    \texorpdfstring
      {^^A
        \mbox
          {^^A
            \BeginAccSupp{ActualText=expkv-#1}^^A
            \href{https://github.com/Skillmon/tex_expkv-#1}
              {^^A
                \rmfamily
                \bfseries
                {\color{expkvgrey}e\kern-.05em x\kern-.05em}^^A
                \lower.493ex
                  \hbox{{\color{expkvgrey}P}\kern-.1em{\color{expkvred}k}}^^A
                \kern-.18em{\color{expkvred}v}^^A
                {\color{expkvgrey}^^A
                  \kern.05em\rule[-.1ex]{.08em}{1.2ex}\kern.05em\textsc{#1}^^A
                }^^A
              }^^A
            \EndAccSupp{}^^A
          }^^A
      }
      {expkv-#1}^^A
  }
\newcommand*\expkvd{\expkvpkg{def}}
\newcommand*\expkvc{\expkvpkg{cs}}
\newcommand\kv{\meta{key}=\meta{value}}
\newcommand\key{\meta{key}}
\newcommand\val{\meta{value}}
\newcommand\set{\meta{set}}
\newcommand\enflong{\textcolor{black}{long}}
\newcommand\alllong{\textcolor{gray}{long}}
\newcommand\notlong{\textcolor{red!80!black}{long}}
\newcommand\prefixes[1]
  {^^A
    \hfill
    \ifcase\numexpr#1\relax\or
      \enflong\or
      \alllong\or
      \notlong\fi
  }
\hypersetup{linkcolor=red!80!black,urlcolor=purple!80!black}
\DoNotIndex{\def,\edef,\,,\=,\begingroup,\catcode,\chardef,\csname,\endcsname}
\DoNotIndex{\endgroup,\endinput,\errmessage,\expandafter,\input,\let,\long,\z@}
\DoNotIndex{\protected,\ProvidesFile,\ProvidesPackage,\relax,\space,\advance}
\DoNotIndex{\@,\unexpanded,\string,\expanded,\dimexpr,\global,\glueexpr,\hbox}
\DoNotIndex{\numexpr,\RequirePackage,\setbox,\the,\unless,\xdef,\gdef,\newcount}
\DoNotIndex{\number}
\DoNotIndex{\^,\@firstofone,\@firstoftwo,\@gobble,\@secondoftwo}
\DoNotIndex{\ifcsname}
\DoNotIndex{\ifx}
\DoNotIndex{\ifodd}
\DoNotIndex{\ifnum}
\DoNotIndex{\ifdefined}
\DoNotIndex{\iffalse}
\DoNotIndex{\iftrue}
\DoNotIndex{\else}
\DoNotIndex{\fi}
\@gobble\fi ^^A ignoring \ifx, \ifcsname, etc., but only one \fi
\@gobble\fi ^^A ignoring \ifx, \ifcsname, etc., but only one \fi
\@gobble\fi ^^A ignoring \ifx, \ifcsname, etc., but only one \fi
\@gobble\fi ^^A ignoring \ifx, \ifcsname, etc., but only one \fi
\@gobble\fi ^^A ignoring \ifx, \ifcsname, etc., but only one \fi
\@gobble\fi ^^A ignoring \ifx, \ifcsname, etc., but only one \fi
\@ifdefinable\gobbledocstriptag{\def\gobbledocstriptag#1>{}}
\makeatother
\begin{document}
  \title
    {^^A
      \texorpdfstring
        {^^A
          \huge\expkvc
          \\[\medskipamount]
          \Large define expandable \kv\ macros using \expkv
        }{expkv-cs - define expandable key=val macros using expkv}^^A
    }
  \date{\ekvcDate\space v\ekvcVersion}
  \author{Jonathan P. Spratte\thanks{\protect\randomize{jspratte@yahoo.de}}}
  \DocInput{expkv-cs.dtx}
\end{document}
%</driver>^^A=<<
% \fi
%
% \maketitle
% \renewcommand*\thefootnote{\arabic{footnote}}
%
% \begin{abstract}
% \noindent\parfillskip=0pt
% \expkvc\ provides two small interfaces to define expandable \kv\ macros using
% \expkv. It therefore lowers the entrance boundary to expandable \kv\ macros.
% The stylised name is \expkvc\ but the files use \file{expkv-cs}, this is due
% to CTAN-rules which don't allow \string| in package names since that is the
% pipe symbol in *nix shells.
% \end{abstract}
%
% \tableofcontents
%
% \begin{documentation}^^A>>=
%
% \section{Documentation}
%
% The \expkv\ package enables the new possibility of creating \kv\ macros which
% are fully expandable. The creation of such macros is however cumbersome for
% the average user. \expkvc\ tries to step in here. It provides interfaces to
% define \kv\ macros without worrying too much about the implementation. In case
% you're wondering now, the |cs| in \expkvc\ stands for control sequence,
% because |def| was already taken by \expkvd\ and ``control sequence'' is the
% term D.\,E.\,Knuth used in his \TeX book for named commands hence macros
% (though he also used the term ``macro''). So \expkvc\ defines control
% sequences for and with \expkv.
%
% There are two different approaches supported by this package. The first is
% splitting the keys up into individual arguments, the second is providing all
% the keys as a single argument to the underlying macro and getting an
% individual \val\ by using a hash. Well, actually there is no real hash, just
% some markers which are parsed, but this shouldn't be apparent to the user, the
% behaviour matches that of a hash-table.
%
% In addition to these two methods of defining a macro with primary keys a way
% to define secondary keys, which can reference the primary ones, is provided.
% These secondary keys don't correspond to an argument or an entry in the hash
% table directly but might come in handy for the average use case. Each macro
% has its own set of primary and secondary keys.
%
% A word of advice you should consider: If your macro doesn't have to be
% expandable (and often it doesn't) consider not using \expkvc. The interface
% has some overhead (though it still is fast -- check \autoref{sec:speed}) and
% the approach has its limits in versatility. If you don't need to be
% expandable, you should consider either defining your keys manually using
% \expkv\ or using \expkvd\ for convenience. Or you resort to another \kv\
% interface. Nevertheless setting up macros with \expkvc, especially with
% |\ekvcSplit|, is very convenient in my opinion, so if you just want to define
% a single macro with just a few keys this might be the way to go.
%
% \expkvc\ is usable as generic code and as a \LaTeX\ package. It'll
% automatically load \expkv\ in the same mode as well. To use it, just use one
% of
% \begin{lstlisting}
% \usepackage{expkv-cs} % LaTeX
% \input expkv-cs       % plainTeX
% \end{lstlisting}
%
%
% \subsection{Define Macros and Primary Keys}
%
% All macros defined with \expkvc\ have to be previously undefined or have the
% |\meaning| of |\relax|. This is necessary as there is no way to automatically
% undefine keys once they are set up (neither \expkv\ nor \expkvc\ keep track of
% defined keys) -- so to make sure there are no conflicts only new definitions
% are allowed (that's not the case for individual keys, only for frontend
% macros).
%
%
% \subsubsection{Primary Keys}\label{sec:primaries}
%
% In the following descriptions there will be one argument named \meta{primary
% keys}. This argument should be a \kv\ list where each \key\ will be one
% primary key and \val\ the associated initial value. By default all keys are
% defined short, but you can define long keys by prefixing \key\ with |long|
% (\emph{e.g.}, \texttt{long name=Jonathan P. Spratte}). You only need |long| if
% the key should be able to take a |\par| token. Note however that |long| keys
% are a microscopic grain faster (due to some internals of \expkvc). Only if at
% least one of the keys was |long| the \meta{cs} in the following defining
% macros will be |\long|. For obvious reasons there is no possibility to define
% a macro or key as |\protected|.
%
% At the moment \expkvc\ doesn't require any internal keys, but I can't foresee
% whether this will be the case in the future as well, as it might turn out that
% some features I deem useful can't be implemented without such internal keys.
% Because of this, please don't use key names starting with
% \texttt{EKVC\string|} as that should be the private name space.
%
%
% \subsubsection{Split}
%
% The split variants will provide the key values as separate arguments. This
% limits the number of keys for which this is truly useful.
%
% \begin{function}{\ekvcSplit}
%   \begin{syntax}
%     \cs{ekvcSplit}\meta{cs}\marg{primary keys}\marg{definition}
%   \end{syntax}
%   This defines \meta{cs} to be a macro taking one mandatory argument which
%   should contain a \kv\ list. The \meta{primary keys} will be defined for this
%   macro (see \autoref{sec:primaries}). The \meta{definition} is the code that
%   will be executed. You can access the \val\ of a \key\ by using a macro
%   parameter from |#1| to |#9|. The order of the macro parameters will be the
%   order provided in the \meta{primary keys} list (so |#1| is the \val\ of the
%   key defined first). With |\ekvcSplit| you can define macros using at most
%   nine primary keys.
% \end{function}
% \example The following defines a macro |\foo| that takes the keys |a| and |b|
% and outputs their values in a textual form:\\
% \begin{minipage}{.75\linewidth}
% \begin{lstlisting}
% \ekvcSplit\foo{a=a,b=b}{a is #1.\par b is #2.\par}
% \foo{}
% \foo{b=e}
% \end{lstlisting}
% \end{minipage}^^A
% \begin{exresult}[nobeforeafter,box align=center]{.25\linewidth}
%   \ekvcSplit\foo{a=a,b=b}{a is #1.\par b is #2.\par}
%   \foo{}
%   \foo{b=e}
% \end{exresult}
%
% \begin{function}{\ekvcSplitAndForward}
%   \begin{syntax}
%     \cs{ekvcSplitAndForward}\meta{cs}\marg{after}\marg{primary keys}
%   \end{syntax}
%   This defines \meta{cs} to be a macro taking one mandatory argument which
%   should contain a \kv\ list. You can use as many primary keys as you want
%   with this. The primary keys will be forwarded to \meta{after} as braced
%   arguments (as many as necessary for your primary keys). The order of the
%   braced arguments will be the order of your primary key definitions. In
%   \meta{after} you can use just a single control sequence, or some arbitrary
%   stuff which will be left in the input stream before your braced values (with
%   one set of braces stripped from \meta{after}), so both of the following
%   would be fine:
% \begin{lstlisting}
% \ekvcSplitAndForward\foo\foo@aux{keyA = A, keyB = B}
% \ekvcSplitAndForward\foo{\foo@aux{more args}}{keyA = A, keyB = B}
% \end{lstlisting}
% \end{function}
% In the first case |\foo@aux| should take at least two arguments (|keyA| and
% |keyB|), in the second case at least three (|more args|, |keyA|, and |keyB|).
%
% \begin{function}{\ekvcSplitAndUse}
%   \begin{syntax}
%     \cs{ekvcSplitAndUse}\meta{cs}\marg{primary keys}
%   \end{syntax}
%   This will roughly do the same as |\ekvcSplitAndForward|, but instead of
%   specifying what will be used after splitting the keys, \meta{cs} will use
%   what follows the \kv\ list. So its syntax will be
% \end{function}
%   \begin{syntax}
%     \meta{cs}\{\key=\val, \ldots\}\marg{after}
%   \end{syntax}
% and the code in |after| should expect at least as many arguments as the number
% of keys defined for \meta{cs}.
%
% \subsubsection{Hash}
%
% The hash variants will provide the key values as a single argument in which
% you can access specific values using a special macro. The implementation might
% be more convenient and scale better, \emph{but} it is much slower (for a
% primitive macro with a single key benchmarking was almost $1.7$ times slower,
% the root of which being the key access with |\ekvcValue|, not the parsing, and
% for a key access using |\ekvcValueFast| it was still about $1.2$ times
% slower). So if your macro uses less than ten primary keys, you should consider
% using the split approach.
%
% \begin{function}{\ekvcHash}
%   \begin{syntax}
%     \cs{ekvcHash}\meta{cs}\marg{primary keys}\marg{definition}
%   \end{syntax}
%   This defines \meta{cs} to be a macro taking one mandatory argument which
%   should contain a \kv\ list. You can use as many primary keys as you want.
%   The primary keys will be forwarded as a single argument containing every key
%   to the underlying macro. The underlying macro is defined as
%   \meta{definition}, in which you can access the \val\ of a \key\ by using
%   \texttt{\cs[no-index]{ekvcValue}\{\key\}\{\#1\}}.
% \end{function}
% \example This defines an equivalent macro like the |\foo| defined with
% |\ekvcSplit| earlier:\\
% \begin{minipage}{.75\linewidth}
% \begin{lstlisting}
% \ekvcHash\foo{a=a,b=b}{a is \ekvcValue{a}{#1}.\par
%                        b is \ekvcValue{b}{#1}.\par}
% \foo{}
% \foo{b=e}
% \end{lstlisting}
% \end{minipage}^^A
% \begin{exresult}[nobeforeafter,box align=center]{.25\linewidth}
%   \ekvcHash\foo{a=a,b=b}{a is \ekvcValue{a}{#1}.\par
%                          b is \ekvcValue{b}{#1}.\par}
%   \foo{}
%   \foo{b=e}
% \end{exresult}
%
% \begin{function}{\ekvcHashAndForward}
%   \begin{syntax}
%     \cs{ekvcHashAndForward}\meta{cs}\marg{after}\marg{primary keys}
%   \end{syntax}
%   This defines \meta{cs} to be a macro taking one mandatory argument which
%   should contain a \kv\ list. You can use as many primary keys as you want.
%   The primary keys will be forwarded as a single argument containing every key
%   to \meta{after}. You can use a single macro for \meta{after} or use some
%   arbitrary stuff, which will be left in the input stream before the hashed
%   \kv\ list with one set of braces stripped. In the
%   macro called in \meta{after} you can access the \val\ of a \key\ by using
%   \texttt{\cs[no-index]{ekvcValue}\{\key\}\{\#1\}} (or whichever argument the
%   hashed \kv\ list will be).
% \end{function}
%
% \begin{function}{\ekvcHashAndUse}
%   \begin{syntax}
%     \cs{ekvcHashAndUse}\meta{cs}\marg{primary keys}
%   \end{syntax}
%   This will roughly do the same as |\ekvcHashAndForward|, but instead of
%   specifying what will be used after hashing the keys, \meta{cs} will use what
%   follows the \kv\ list. So its syntax will be
% \end{function}
%   \begin{syntax}
%     \meta{cs}\{\key=\val, \ldots\}\marg{after}
%   \end{syntax}
%
% \begin{function}{\ekvcValue}
%   \begin{syntax}
%     \cs{ekvcValue}\{\key\}\marg{key list}
%   \end{syntax}
%   This is a safe way to access your keys in a hash variant. \key\
%   is the key which's \val\ you want to use out of the \meta{key list}.
%   \meta{key list} should be the key list argument forwarded to your underlying
%   macro by |\ekvcHash|, |\ekvcHashAndForward|, or |\ekvcHashAndUse|. It will
%   be tested whether the hash function to access that \key\ exists, the \key\
%   argument is not empty, and that the \meta{key list} really contains a \val\
%   of that \key. This macro needs exactly two steps of expansion and if used
%   inside of an |\edef| or |\expanded| context will protect the \val\ from
%   further expanding.
% \end{function}
%
% \begin{function}{\ekvcValueFast}
%   \begin{syntax}
%     \cs{ekvcValueFast}\{\key\}\marg{key list}
%   \end{syntax}
%   This behaves just like |\ekvcValue|, but \emph{without any} safety tests. As
%   a result this is about $1.4$ times faster \emph{but} will throw low level
%   \TeX\ errors eventually if the hash function isn't defined or the \key\
%   isn't part of the \meta{key list} (\emph{e.g.}, because it was defined as a
%   key for another macro -- all macros share the same hash function per \key).
%   Use it if you know what you're doing. This macro needs exactly three steps
%   of expansion in the no-errors case.
% \end{function}
%
% \begin{function}{\ekvcValueSplit}
%   \begin{syntax}
%     \cs{ekvcValueSplit}\{\key\}\marg{key list}\marg{next}
%   \end{syntax}
%   If you need a specific \key\ from a \meta{key list} more than once, it'll
%   be a good idea to only extract it once and from then on keep it as a
%   separate argument. Hence the macro \cs{ekvcValueSplit} will extract one
%   specific \key's value from the list and forward the remainder of the list
%   as the first and the \key's value as the second argument to \meta{next}, so
%   the result of this will be \meta{next}\marg{key list'}\marg{value} with
%   \meta{key list'} the remaining list. This is almost as fast as |\ekvcValue|
%   and runs the same tests. Keep in mind that you can't fetch for the same
%   \key\ again from \meta{key list'} as it got removed.
% \end{function}
% \example The following defines a macro |\foo| which will take three keys.
% Since the next parsing step will need the value of one of the keys multiple
% times we split that key off the list:\\
% \begin{minipage}{.75\linewidth}
% \begin{lstlisting}
% \ekvcHash\foo{a=a,b=b,c=c}
%   {\ekvcValueSplit{a}{#1}\foobar}
% \newcommand*\foobar[2]{a is #2.\par
%                        b is \ekvcValue{b}{#1}.\par
%                        c is \ekvcValue{c}{#1}.\par}
% \foo{}
% \end{lstlisting}
% \end{minipage}^^A
% \begin{exresult}[nobeforeafter,box align=center]{.25\linewidth}
%   \ekvcHash\foo{a=a,b=b,c=c}
%     {\ekvcValueSplit{a}{#1}\foobar}
%   \newcommand*\foobar[2]{a is #2.\par
%                          b is \ekvcValue{b}{#1}.\par
%                          c is \ekvcValue{c}{#1}.\par}
%   \foo{}
% \end{exresult}
%
% \begin{function}{\ekvcValueSplitFast}
%   \begin{syntax}
%     \cs{ekvcValueSplitFast}\{\key\}\marg{key list}\marg{next}
%   \end{syntax}
%   This behaves just like |\ekvcValueSplit|, but it won't run the same tests,
%   hence it is faster but more error prone, just like the relation between
%   |\ekvcValue| and |\ekvcValueFast|.
% \end{function}
%
%
% \subsection{Secondary Keys}\label{sec:secondaries}
%
% To remove some of the limitations with the approach that each primary key
% matches an argument or hash entry, you can define secondary keys. Those have
% to be defined for each macro but it doesn't matter whether that macro was a
% split or a hash variant. If a secondary key references another key it doesn't
% matter whether that other key is primary or secondary.
%
% Secondary keys can have a prefix (like |long|) which are called |p|-type
% prefix and must have a type (like |meta|) which are called |t|-type prefix.
% Some types might require some |p|-prefixes, while others might forbid those.
%
% Please keep in mind that key names shouldn't start with \texttt{EKVC\string|}.
%
% \begin{function}{\ekvcSecondaryKeys}
%   \begin{syntax}
%     \cs{ekvcSecondaryKeys}\meta{cs}\{\kv, \ldots\}
%   \end{syntax}
%   This is the front facing macro to define secondary keys. For the macro
%   \meta{cs} define \key\ to have definition \val. The general syntax for \key\
%   should be
%   \begin{quote}\ttfamily\small
%     \meta{prefix} \meta{name}
%   \end{quote}
%   Where \meta{prefix} is a space separated list of optional |p|-type prefixes
%   followed by one |t|-type prefix. The syntax of \val\ is dependent on the
%   used |t|-prefix.
% \end{function}
%
%
% \subsubsection{\texttt{p}-type Prefixes}
%
% There is only one |p|-prefix available, which is |long|.
%
% \begin{function}{long}
%   The following key will be defined |\long|.
% \end{function}
%
%
% \subsubsection{\texttt{t}-type Prefixes}
%
% If you're familiar with \expkvd\ you'll notice that the |t|-type prefixes
% provided here are much fewer. The expansion only concept doesn't allow for
% great variety in the auto-defined keys.
%
% The syntax examples of the |t|-prefixes will show which |p|-prefix will be
% automatically used by printing those black (\texttt{\enflong}), which
% will be available in grey (\texttt{\alllong}), and which will be disallowed in
% red (\texttt{\notlong}). This will be put flush right next to the syntax line.
%
% \begin{function}{meta}
%   \begin{syntax}
%     meta \key\ = \{\kv, \ldots\} \prefixes2
%   \end{syntax}
%   With a |meta| key you can set other keys. Whenever \key\ is used the keys in
%   the \kv\ list will be set to the values given there. You can use the \val\
%   given to \key\ by using |#1| in the \kv\ list. The keys in the \kv\ list can
%   be primary and secondary ones.
% \end{function}
%
% \begin{function}{nmeta}
%   \begin{syntax}
%     nmeta \key\ = \{\kv, \ldots\} \prefixes3
%   \end{syntax}
%   An |nmeta| key is like a |meta| key, but it doesn't take a value, so the
%   \kv\ list is static.
% \end{function}
%
% \begin{function}{alias}
%   \begin{syntax}
%     alias \key\ = \meta{key_2} \prefixes3
%   \end{syntax}
%   This assigns the definition of \meta{key_2} to \key. As a result \key\ is an
%   alias for \meta{key_2} behaving just the same. Both the value taking and the
%   |NoVal| version (that's \expkv\ slang for a key not accepting a value) will
%   be copied if they are defined when |alias| is used. Of course, \meta{key_2}
%   has to be defined, be it as a primary or secondary one.
% \end{function}
%
% \begin{function}{default}
%   \begin{syntax}
%     default \key\ = \marg{default} \prefixes3
%   \end{syntax}
%   If \key\ is a defined value taking key, you can define a |NoVal| version
%   with this that will behave as if \key\ was given \meta{default} as its
%   \val. Note that this doesn't change the initial values of primary keys set
%   at definition time in |\ekvcSplit| and friends. \key\ can be a primary or
%   secondary key.
% \end{function}
%
% \begin{function}{flag-bool}
%   \begin{syntax}
%     flag-bool \key\ = \meta{cs} \prefixes3
%   \end{syntax}
%   This is a secondary key that doesn't directly involve any of the primary or
%   secondary keys.
%   This defines \key\ to take a value, which should be either |true| or
%   |false|, and set the flag called \meta{cs} accordingly as a boolean.
%   If \meta{cs} isn't defined yet it will be initialised as a flag.
%   Please also read \autoref{sec:flags}.
% \end{function}
%
% \begin{function}{flag-true,flag-false}
%   \begin{syntax}
%     flag-true \key\ = \meta{cs} \prefixes3
%   \end{syntax}
%   This is a secondary key that doesn't directly involve any of the primary or
%   secondary keys.
%   This defines \key\ to take no value and set the flag called \meta{cs} to
%   |true| or |false|, respectively.
%   If \meta{cs} isn't defined yet it will be initialised as a flag.
%   Please also read \autoref{sec:flags}.
% \end{function}
%
% \begin{function}{flag-raise}
%   \begin{syntax}
%     flag-raise \key\ = \meta{cs} \prefixes3
%   \end{syntax}
%   This is a secondary key that doesn't directly involve any of the primary or
%   secondary keys.
%   This defines \key\ to take no value and raise the flag called \meta{cs}.
%   If \meta{cs} isn't defined yet it will be initialised as a flag.
%   Please also read \autoref{sec:flags}.
% \end{function}
%
%
% \subsection{Changing the Initial Values}\label{sec:change}
%
% \begin{function}{\ekvcChange}
%   \begin{syntax}
%     \cs{ekvcChange}\meta{cs}\{\kv,\ldots\}
%   \end{syntax}
%   This processes the \kv\ list (just like it would be processed by \meta{cs})
%   to set new defaults for the macro \meta{cs} defined with \expkvc\ (meaning
%   the values used if you don't provide anything at use time). Inside the \kv\
%   list both primary and secondary keys can be used. If \meta{cs} was defined
%   |\long| earlier it will still be |\long|, every other \TeX\ prefix will be
%   stripped (but \expkvc\ doesn't support them anywhere else so that should be
%   fine). The resulting new defaults will be stored inside the \meta{cs}
%   locally (just as the original defaults were).
% \end{function}
%
% Consider the following example:\\
% \begin{minipage}[c]{.75\linewidth}
% \begin{lstlisting}
% \ekvcSplit\foo{a=a,b=b}{a is #1.\par b is #2.\par}
% \foo{} % initial defaults
% \ekvcChange\foo{b=B}
% \foo{} % new defaults
% \ekvcSecondaryKeys\foo{meta c={a={#1},b={#1}}}
% \ekvcChange\foo{c=c}
% \foo{} % newer defaults
% \end{lstlisting}
% \end{minipage}^^A
% \begin{exresult}[nobeforeafter,box align=center]{.25\linewidth}
%   \ekvcSplit\foo{a=a,b=b}{a is #1.\par b is #2.\par}
%   \foo{}
%   \ekvcChange\foo{b=B}
%   \foo{}
%   \ekvcSecondaryKeys\foo{meta c={a={#1},b={#1}}}
%   \ekvcChange\foo{c=c}
%   \foo{}
% \end{exresult}
%
% As a result with this the typical setup macro could be implemented:
% \begin{lstlisting}
% \ekvcHashAndUse\foo{key=a,key=b}
% \newcommand\foosetup{\ekvcChange\foo}
% \end{lstlisting}
% Of course the usage is limited to a single macro |\foo|, hence this might be
% not as powerful as similar macros used with other \kv\ interfaces. But at
% least a few similar macros could be grouped using the same key parsing macro
% internally.
%
%
% \subsection{Flags}\label{sec:flags}
%
% The idea of flags is taken from \pkg{expl3}. They provide a way to store
% numerical information expandably, however only incrementing and accessing
% works expandably, decrementing is unexpandable. A flag has a height, which
% is a numerical value, and which can be raised by $1$. Flags come at a high
% computational cost (accessing them is slow and they require more memory than
% normal \TeX\ data types like registers, both getting linearly worse with the
% height), so don't use them if not necessary.
%
% The state of flags is always changed locally to the current group, but not to
% the current macro, so if you're using one of the |t|-types involving flags
% bear in mind that they can affect other macros using the same flags at the
% current group level!
%
% \expkvc\ provides some macros to access, alter, and use flags. Flags of
% \expkvc\ don't share a name space with the flags of \pkg{expl3}.
%
% \begin{function}{\ekvcFlagNew}
%   \begin{syntax}
%     \cs{ekvcFlagNew}\meta{flag}
%   \end{syntax}
%   This initialises the macro \meta{flag} as a new flag. It isn't checked
%   whether the macro \meta{flag} is currently undefined.  A \meta{flag} will
%   expand to the flag's current height with a trailing space (so you can use it
%   directly with |\ifnum| for example and it will terminate the number scanning
%   itself).
% \end{function}
%
%  All other macros dealing with flags take as a parameter a macro defined as a
%  \meta{flag} with |\ekvcFlagNew|.
%
% \begin{function}{\ekvcFlagHeight}
%   \begin{syntax}
%     \cs{ekvcFlagHeight}\meta{flag}
%   \end{syntax}
%   This expands to the current height of \meta{flag} in a single step of
%   expansion (without a trailing space).
% \end{function}
%
% \begin{function}{\ekvcFlagRaise}
%   \begin{syntax}
%     \cs{ekvcFlagRaise}\meta{flag}
%   \end{syntax}
%   This expandably raises the height of \meta{flag} by $1$.
% \end{function}
%
% \begin{function}{\ekvcFlagSetTrue,\ekvcFlagSetFalse}
%   \begin{syntax}
%     \cs{ekvcFlagSetTrue}\meta{flag}
%   \end{syntax}
%   By interpreting an even value as false and an odd value as true we can use a
%   flag as a boolean. This expandably sets \meta{flag} to true or false,
%   respectively, by raising it if necessary.
% \end{function}
%
% \begin{function}{\ekvcFlagIf}
%   \begin{syntax}
%     \cs{ekvcFlagIf}\meta{flag}\marg{true}\marg{false}
%   \end{syntax}
%   This interprets a \meta{flag} as a boolean and expands to either \meta{true}
%   or \meta{false}.
% \end{function}
%
% \begin{function}{\ekvcFlagIfRaised}
%   \begin{syntax}
%     \cs{ekvcFlagIfRaised}\meta{flag}\marg{true}\marg{false}
%   \end{syntax}
%   This tests whether the \meta{flag} is raised, meaning it has a height
%   greater than zero, and if so expands to \meta{true} else to \meta{false}.
% \end{function}
%
% \begin{function}{\ekvcFlagReset}
%   \begin{syntax}
%     \cs{ekvcFlagReset}\meta{flag}
%   \end{syntax}
%   This resets a flag (so restores its height to $0$). This operation is
%   \emph{not} expandable and done locally. If you really intend to use flags
%   you can reset them every now and then to keep the performance hit low.
% \end{function}
%
% \begin{function}{\ekvcFlagGetHeight}
%   \begin{syntax}
%     \cs{ekvcFlagGetHeight}\meta{flag}\marg{next}
%   \end{syntax}
%   This retrieves the current height of the \meta{flag} and provides it as a
%   braced argument to \meta{next}, leaving \meta{next}\marg{height} in the
%   input stream.
% \end{function}
%
% \begin{function}{\ekvcFlagGetHeights}
%   \begin{syntax}
%     \cs{ekvcFlagGetHeights}\marg{flag-list}\marg{next}
%   \end{syntax}
%   This retrieves the current height of each \meta{flag} in the
%   \meta{flag-list} and provides them as a single argument to \meta{next}.
%   Inside that argument each height is enclosed in a set of braces
%   individually. The \meta{flag-list} is just a single argument containing the
%   \meta{flag}s.
%   So a usage like
%   \texttt
%     {^^A
%       \cs[no-index]{ekvcFlagGetHeights}\hskip0pt^^A
%       \{\cs[no-index]{myflagA}\cs[no-index]{myflagB}\}\hskip0pt^^A
%       \{\cs[no-index]{stuff}\}^^A
%     }
%   will expand to
%   \texttt{\cs[no-index]{stuff}\{\marg{height-A}\marg{height-B}\}}.
% \end{function}
%
%
% \subsection{Further Examples}
%
% How could a documentation be a good documentation without some basic examples?
% Say we want to define a small macro expanding to some character description
% (who knows why this has to be expandable?). A character description will not
% have too many items to it, so we use |\ekvcSplit| (the comments with the
% parameter numbers are of course not necessary and just ease reading the
% example).
% \begin{lstlisting}
% \ekvcSplit\character
%   {
%     name=John Doe,            % #1
%     age=any,                  % #2
%     nationality=the Universe, % #3
%     hobby=to exist,           % #4
%     type=Mister,              % #5
%     pronoun=He,               % #6
%     possessive=his,           % #7
%   }
%   {#1 is a #5 from #3. #6 is of #2 age and #7 hobby is #4.\par}
% \end{lstlisting}
% Also we want to give some short cuts so that it's easier to describe several
% persons.
% \begin{lstlisting}
% \ekvcSecondaryKeys\character
%   {
%     alias pro = pronoun,
%     alias pos = possessive,
%     nmeta me =
%       {
%         name=Jonathan,
%         age=a young,
%         nationality=Germany,
%         hobby=\TeX\ coding,
%       },
%     meta lady =
%       {type=Lady, pronoun=She, possessive=her, name=Jane Doe, #1},
%     nmeta paulo =
%       {
%         name=Paulo,
%         type=duck,
%         age=a young,
%         nationality=Brazil,
%         hobby=to quack,
%       }
%   }
% \end{lstlisting}
% Now we can describe people using
% \begin{lstlisting}
% \character{}
% \character{me}
% \character{paulo}
% \character
%   {lady={name=Evelyn,nationality=Ireland,age=the best,hobby=reading}}
% \character
%   {
%     name=Our sun, type=star, nationality=our solar system, pro=It,
%     age=an old, pos=its, hobby=shining
%   }
% \end{lstlisting}
% As one might see, the |lady| key could actually have been an |nmeta| key as
% well, as all that is done with the argument is using it as a \kv\ list.
%
% The result of only the first two usages would be:
% \begin{exresult}{\linewidth}
%   \ekvcSplit\character
%     {
%       name=John Doe,
%       age=any,
%       nationality=the Universe,
%       hobby=to exist,
%       type=Mister,
%       pronoun=He,
%       possessive=his,
%     }
%     {#1 is a #5 from #3. #6 is of #2 age and #7 hobby is #4.\par}
%   \ekvcSecondaryKeys\character
%     {
%       alias pro = pronoun,
%       alias pos = possessive,
%       nmeta me =
%         {
%           name=Jonathan,
%           age=a young,
%           nationality=Germany,
%           hobby=\TeX\ coding,
%         },
%       meta lady =
%         {type=Lady, pronoun=She, possessive=her, name=Jane Doe, #1},
%       nmeta paulo =
%         {
%           name=Paulo,
%           type=duck,
%           age=a young,
%           nationality=Brazil,
%           hobby=to quack,
%         }
%     }
%   \character{}
%   \character{me}
% \end{exresult}
%
% Using \pkg{xparse} or \expkv's \cs{ekvoptarg} or \cs{ekvoptargTF} and
% forwarding arguments one can easily define \kv\ macros with actual optional
% and mandatory arguments as well. A small nonsense example (which should
% perhaps use |\ekvcSplitAndForward| instead of |\ekvcHashAndForward| since it
% only uses four keys and one other argument -- and isn't expandable since it
% uses a \env{tabular} environment, so it would've been better to use a more
% feature rich \kv\ interface most likely, \emph{e.g.}, the one provided by
% \expkvd):
% \begin{lstlisting}
% \makeatletter
% \newcommand*\nonsense{\ekvoptarg\nonsense@a{}}
% \ekvcHashAndForward\nonsense@a\nonsense@b
%   {
%     keyA = A,
%     keyB = B,
%     keyC = c,
%     keyD = d,
%   }
% \newcommand*\nonsense@b[2]
%   {%
%     \begin{tabular}{lll}
%       key & A & \ekvcValue{keyA}{#1} \\
%           & B & \ekvcValue{keyB}{#1} \\
%           & C & \ekvcValue{keyC}{#1} \\
%           & D & \ekvcValue{keyD}{#1} \\
%       \multicolumn{2}{l}{mandatory} & #2 \\
%     \end{tabular}%
%   }
% \makeatother
% \end{lstlisting}
% And then we would be able to do some nonsense
% \begin{lstlisting}
% \nonsense{}
% \nonsense[keyA=hihi]{haha}
% \nonsense[keyA=hihi, keyB=A]{hehe}
% \nonsense[keyC=huhu, keyA=hihi, keyB=A]{haha}
% \end{lstlisting}
% resulting in
% \begin{exresult}{\linewidth}
%   \makeatletter
%   \newcommand*\nonsense{\ekvoptarg\nonsense@a{}}
%   \ekvcHashAndForward\nonsense@a\nonsense@b
%     {
%       keyA = A,
%       keyB = B,
%       keyC = c,
%       keyD = d,
%     }
%   \newcommand*\nonsense@b[2]
%     {%
%       \begin{tabular}{lll}
%         key & A & \ekvcValue{keyA}{#1} \\
%             & B & \ekvcValue{keyB}{#1} \\
%             & C & \ekvcValue{keyC}{#1} \\
%             & D & \ekvcValue{keyD}{#1} \\
%         \multicolumn{2}{l}{mandatory} & #2 \\
%       \end{tabular}%
%     }
%   \makeatother
%   \nonsense{}
%   \nonsense[keyA=hihi]{haha}
%   \nonsense[keyA=hihi, keyB=A]{hehe}
%   \nonsense[keyC=huhu, keyA=hihi, keyB=A]{haha}
% \end{exresult}
%
%
% \subsection{Freedom for Keys!}
%
% If this was the \TeX book this subsection would have a double bend sign. Not
% because it is overly complicated, but because it shows things which could
% break \expkvc's expandability and its alignment safety. This is for
% experienced users wanting to get the most flexibility and knowing what they
% are doing.
%
% In case you're wondering, it is possible to define other keys than the
% primaries and the secondary types listed in \autoref{sec:secondaries} for a
% macro defined with \expkvc\ by using the low-level interface of \expkv\ or
% even the interface provided by \expkvd. The set name used for \expkvc's keys
% is the macro name, including the leading backslash, or more precisely
% |\string|\meta{cs} is used. This can be exploited to define additional keys
% with arbitrary code. Consider the following \emph{bad} example:
% \begin{lstlisting}
% \ekvcSplit\foo{a=A,b=B}{a is #1.\par b is #2}
% \protected\ekvdef{\string\foo}{c}{\def\fooC{#1}}
% \end{lstlisting}
% This would define a key named |c| that will store its value inside a macro.
% The issue with this is that this can't be done expandably. As a result, the
% macro |\foo| isn't always expandable any more (not that bad if this was never
% required; killjoy if it was) and as soon as the key |c| is used, it is also no
% longer alignment safe\footnote{This means that the \kv-list can't contain
%   alignment markers that are not inside an additional set of braces if used
%   inside a \TeX\ alignment.}
% (might be bad depending on the usage).
%
% So why do I show you this? Because we could as well do something useful like
% create a key that pre-parses the input and after that passes the parsed value
% on. This parsing would have to be completely expandable though. For the
% pass-on part we can use the following function:
%
% \begin{function}{\ekvcPass}
%   \begin{syntax}
%     \cs{ekvcPass}\meta{cs}\marg{key}\marg{value}
%   \end{syntax}
%   This passes \meta{value} on to \meta{key} for the \expkvc-macro \meta{cs}.
%   It should be used inside the key parsing of a macro defined with \expkvc,
%   else this most likely results in a low level \TeX\ error.
% \end{function}
%
% With this we could for example split the value of a key at a hyphen and pass
% the parts to different keys:\\
% \begin{minipage}[c]{.75\linewidth}
% \begin{lstlisting}
% \ekvcSplit\foo{a=A,b=B}{a is #1.\par b is #2.\par}
% \ekvdef{\string\foo}{c}{\fooSplit#1\par}
% \def\fooSplit#1-#2\par
%   {\ekvcPass\foo{a}{#1}\ekvcPass\foo{b}{#2}}
% \foo{}
% \foo{c=1-2}
% \end{lstlisting}
% \end{minipage}
% \begin{exresult}[nobeforeafter,box align=center]{.25\linewidth}
%   \ekvcSplit\foo{a=A,b=B}{a is #1.\par b is #2.\par}
%   \ekvdef{\string\foo}{c}{\fooSplit#1\par}
%   \def\fooSplit#1-#2\par{\ekvcPass\foo{a}{#1}\ekvcPass\foo{b}{#2}}
%   \foo{}
%   \foo{c=1-2}
% \end{exresult}
%
%
% \subsection{Speed Considerations}\label{sec:speed}
%
% As already mentioned in the introduction there are some speed considerations
% implied if you choose to define macros via \expkvc. However the overhead isn't
% the factor which should hinder you to use \expkvc\ if you found a reasonable
% use case. The key-parsing is still faster than with most other \kv\ packages
% (see the ``Comparisons'' subsection in the \expkv\ documentation).
%
% The speed considerations in this subsection use the first example in this
% documentation as the benchmark. So we have seven keys and a short sentence
% which should be typeset. For comparisons I use the following equivalent
% \expkvd\ definitions. Each result is the average between changing no keys from
% their initial values and altering four. Furthermore I'll compare three
% variants of \expkvc\ with the \expkvd\ definitions, namely the split example
% from above, a hash variant using |\ekvcValue| and a hash variant using
% |\ekvcValueFast|.
% \begin{lstlisting}
% \usepackage{expkv-def}
% \ekvdefinekeys{keys}
%   {%
%     ,store   name         = \KEYSname
%     ,initial name         = John Doe
%     ,store   age          = \KEYSage
%     ,initial age          = any
%     ,store   nationality  = \KEYSnationality
%     ,initial nationality  = the Universe
%     ,store   hobby        = \KEYShobby
%     ,initial hobby        = to exist
%     ,store   type         = \KEYStype
%     ,initial type         = Mister
%     ,store   pronoun      = \KEYSpronoun
%     ,initial pronoun      = He
%     ,store   possessive   = \KEYSpossessive
%     ,initial possessive   = his
%   }
% \newcommand*\KEYS[1]
%   {%
%     \begingroup
%       \ekvset{keys}{#1}%
%       \KEYSname\ is a \KEYStype\ from \KEYSnationality. \KEYSpronoun\ is
%       of \KEYSage\ age and \KEYSpossessive\ hobby is \KEYShobby.%
%     \endgroup
%   }
% \end{lstlisting}
%
% The first comparison removes the typesetting part from all the definitions, so
% that only the key parsing is compared. In this comparison the |\ekvcValue|
% and |\ekvcValueFast| variants will not differ, as they are exactly the same
% until the key usage. We find that the split approach is $1.4$ times slower
% than the \expkvd\ setup and the hash variants end up in the middle at $1.17$
% times slower.
%
% Next we put the typesetting part back in. Every call of the macros will
% typeset the sentences into a box register in horizontal mode. With the
% typesetting part (which includes the accessing of values) the fastest remains
% the \expkvd\ definitions, but split is close at $1.16$ times slower, followed
% by the hash variant with fast accesses at $1.36$ times slower, and the safe
% hash access variant ranks in the slowest $1.8$ times slower than \expkvd.
%
% Just in case you're wondering now, a simple macro taking seven arguments is
% $30$ to $40$ times faster than any of those in the argument grabbing and \kv\
% parsing part and only $1.5$ to $2.8$ times faster if the typesetting part is
% factored in. So the real choke isn't the parsing.
%
% So to summarize this, if you have a reasonable use case for expandable \kv\
% parsing macros you should go on and define them using \expkvc. If you have a
% reasonable use case for \kv\ parsing macros but defining them expandable isn't
% necessary for your use you should take advantage of the greater flexibility of
% non-expandable \kv\ setups (but if you're after maximum speed there aren't
% that many \kv\ parsers beating \expkvc). And if you are after maximum
% performance maybe ditching the \kv\ interface altogether is a good idea, but
% depending on the number of arguments your interface might get convoluted.
%
%
% \subsection{Useless Macros}
%
% Perhaps these macros aren't completely useless, but I figured from a user's
% point of view I wouldn't know what I should do with these.
%
% \begin{function}{\ekvcDate,\ekvcVersion}
%   These two macros store the version and the date of the package/generic code.
% \end{function}
%
%
% \subsection{Bugs}
%
% Of course I don't think there are any bugs (who would knowingly distribute
% buggy software as long as he isn't a multi-million dollar corporation?). But
% if you find some please let me know. For this one might find my email address
% on the first page or file an issue on Github:
% \url{https://github.com/Skillmon/tex_expkv-cs}
%
%
% \subsection{License}
%
% Copyright \textcopyright\ 2020\unless\ifnum\year=2020--\the\year\fi\
% Jonathan P. Spratte
%
% \medskip\noindent
% This work may be distributed and/or modified under the conditions of the
% \LaTeX\ Project Public License (LPPL), either version 1.3c of this license or
% (at your option) any later version. The latest version of this license is in
% the file:
%
% \url{http://www.latex-project.org/lppl.txt}
%
% \noindent
% This work is ``maintained'' (as per LPPL maintenance status) by
%
% \mbox{Jonathan P. Spratte}.
%
% \end{documentation}^^A=<<
%
% \begin{implementation}^^A>>=
%
% \clearpage
%
% \section{Implementation}^^A>>=
%^^A the LaTeX package >>=
% \subsection{The \LaTeX\ Package}
% Just like for \expkv\ we provide a small \LaTeX\ package that sets up things
% such that we behave nicely on \LaTeX\ packages and files system. It'll
% |\input| the generic code which implements the functionality.
% \gobbledocstriptag
%<*pkg>
%    \begin{macrocode}
\RequirePackage{expkv}
\def\ekvc@tmp
  {%
    \ProvidesFile{expkv-cs.tex}%
      [%
        \ekvcDate\space v\ekvcVersion\space
        define expandable key=val macros using expkv%
      ]%
  }
\input{expkv-cs.tex}
\ProvidesPackage{expkv-cs}%
  [%
    \ekvcDate\space v\ekvcVersion\space
    define expandable key=val macros using expkv%
  ]
%    \end{macrocode}
% \gobbledocstriptag
%</pkg>
%^^A=<<
%^^A main file >>=
% \subsection{The Generic Code}
% The rest of this implementation will be the generic code.
% \gobbledocstriptag
%<*tex>
%
% Load \expkv\ if the package didn't already do so -- since \expkv\ has
% safeguards against being loaded twice this does no harm and the overhead
% isn't that big. Also we reuse some of the internals of \expkv\ to save us from
% retyping them.
%    \begin{macrocode}
\input expkv
%    \end{macrocode}
%
% We make sure that \file{expkv-cs.tex} is only input once:
%    \begin{macrocode}
\expandafter\ifx\csname ekvcVersion\endcsname\relax
\else
  \expandafter\endinput
\fi
%    \end{macrocode}
%
% \begin{macro}{\ekvcVersion,\ekvcDate}
% We're on our first input, so lets store the version and date in a macro.
%    \begin{macrocode}
\def\ekvcVersion{0.10}
\def\ekvcDate{2021-06-03}
%    \end{macrocode}
% \end{macro}
%
% If the \LaTeX\ format is loaded we want to be a good file and report back who
% we are, for this the package will have defined |\ekvc@tmp| to use
% |\ProvidesFile|, else this will expand to a |\relax| and do no harm.
%    \begin{macrocode}
\csname ekvc@tmp\endcsname
%    \end{macrocode}
%
% Store the category code of |@| to later be able to reset it and change it to
% 11 for now.
%    \begin{macrocode}
\expandafter\chardef\csname ekvc@tmp\endcsname=\catcode`\@
\catcode`\@=11
%    \end{macrocode}
% |\ekvc@tmp| will be reused later, but we don't need it to ever store
% information long-term after \expkvc\ was initialized.
%
% \begin{macro}[internal]{\ekvc@keycount}
%   We'll need to keep count how many keys must be defined for each macro in the
%   |split| variants.
%    \begin{macrocode}
\newcount\ekvc@keycount
%    \end{macrocode}
% \end{macro}
%
% \begin{macro}[internal]{\ekvc@long,\ekvc@any@long}
%   Some macros will have to be defined long. These two will be let to |\long|
%   when this should be the case.
%    \begin{macrocode}
\let\ekvc@long\ekv@empty
\let\ekvc@any@long\ekv@empty
%    \end{macrocode}
% \end{macro}
%
% \begin{macro}[internal]{\ekvc@ifdefined}
%   We want to test whether a macro is already defined. This test checks for a
%   defined macro that isn't |\relax|.
%    \begin{macrocode}
\long\def\ekvc@ifdefined#1%
  {%
    \ifdefined#1%
      \ifx\relax#1%
        \ekv@fi@gobble
      \fi
      \@firstofone
      \ekv@fi@firstoftwo
    \fi
    \@secondoftwo
  }
%    \end{macrocode}
% \end{macro}
%
% \begin{macro}[internal]
%   {
%     \ekvc@ekvset@pre@expander,
%     \ekvc@ekvset@pre@expander@a,
%     \ekvc@ekvset@pre@expander@b
%   }
%   This macro expands |\ekvset| twice so that the first two steps of expansion
%   don't have to be made every time the \expkvc\ macros are used. We have to do
%   a little magic trick to get the macro parameter |#1| for the macro
%   definition this is used in, even though we're calling |\unexpanded|. We do
%   that by splitting the expanded |\ekvset| at some marks and place |##1| in
%   between. At this spot we also add |\ekv@alignsafe| and |\ekv@endalignsafe|
%   to ensure that macros created with \expkvc\ are alignment safe.
%    \begin{macrocode}
\def\ekvc@ekvset@pre@expander#1%
  {%
    \expandafter\ekvc@ekvset@pre@expander@a\ekvset{#1}\ekvc@stop\ekvc@stop
  }
\def\ekvc@ekvset@pre@expander@a
  {%
    \expandafter\ekvc@ekvset@pre@expander@b
  }
\def\ekvc@ekvset@pre@expander@b#1\ekvc@stop#2\ekvc@stop
  {%
    \unexpanded\expandafter{\ekv@alignsafe}%
    \unexpanded{#1}##1\unexpanded{#2}%
    \unexpanded\expandafter{\ekv@endalignsafe}%
  }
%    \end{macrocode}
% \end{macro}
%
% \begin{macro}{\ekvcSplitAndUse}
%   The first user macro we want to set up can be reused for
%   \cs[no-index]{ekvcSplitAndForward} and \cs[no-index]{ekvcSplit}. We'll split
%   this one up so that the test whether the macro is already defined doesn't
%   run twice.
%    \begin{macrocode}
\protected\long\def\ekvcSplitAndUse#1#2%
  {%
    \let\ekvc@helpers@needed\@firstoftwo
    \ekvc@ifdefined#1%
      {\ekvc@err@already@defined#1}%
      {\ekvcSplitAndUse@#1{}{#2}}%
  }
%    \end{macrocode}
% \end{macro}
%
% \begin{macro}[internal]{\ekvcSplitAndUse@}
%   The actual macro setting up things. We need to set some variables, forward
%   the key list to |\ekvc@SetupSplitKeys|, and afterwards define the front
%   facing macro to call |\ekvset| and put the initials and the argument sorting
%   macro behind it. The internals |\ekvc@any@long|, |\ekvc@initials| and
%   |\ekvc@keycount| will be set correctly by |\ekvc@SetupSplitKeys|.
%    \begin{macrocode}
\protected\long\def\ekvcSplitAndUse@#1#2#3%
  {%
    \edef\ekvc@set{\string#1}%
    \ekvc@SetupSplitKeys{#3}%
    \ekvc@helpers@needed
      {%
        \ekvc@any@long\edef#1##1%
          {%
            \expandafter\ekvc@ekvset@pre@expander\expandafter{\ekvc@set}%
            \unexpanded\expandafter
              {\csname ekvc@split@\the\ekvc@keycount\endcsname}%
            \unexpanded\expandafter{\ekvc@initials{}#2}%
          }%
      }%
      {%
        \ekvc@any@long\edef#1##1%
          {%
            \expandafter\ekvc@ekvset@pre@expander\expandafter{\ekvc@set}%
            \unexpanded{#2}%
            \unexpanded\expandafter{\ekvc@initials}%
          }%
      }%
  }
%    \end{macrocode}
% \end{macro}
%
%
% \begin{macro}{\ekvcSplitAndForward}
%   This just reuses |\ekvcSplitAndUse@| with a non-empty second argument,
%   resulting in that argument to be called after the splitting.
%    \begin{macrocode}
\protected\long\def\ekvcSplitAndForward#1#2#3%
  {%
    \let\ekvc@helpers@needed\@firstoftwo
    \ekvc@ifdefined#1%
      {\ekvc@err@already@defined#1}%
      {\ekvcSplitAndUse@#1{{#2}}{#3}}%
  }
%    \end{macrocode}
% \end{macro}
%
% \begin{macro}{\ekvcSplit}
%   The first half is just |\ekvcSplitAndForward| then we define the macro to
%   which the parsed key list is forwarded. There we need to allow for up to
%   nine arguments.
%    \begin{macrocode}
\protected\long\def\ekvcSplit#1#2#3%
  {%
    \let\ekvc@helpers@needed\@secondoftwo
    \ekvc@ifdefined#1%
      {\ekvc@err@already@defined#1}%
      {%
        \expandafter
        \ekvcSplitAndUse@\expandafter#1\csname ekvc@\string#1\endcsname{#2}%
        \ifnum\ekvc@keycount<1
          \ekvc@any@long\expandafter\def\csname ekvc@\string#1\endcsname{#3}%
        \else
          \ifnum\ekvc@keycount>9
            \ekvc@err@toomany{#1}%
            \let#1\ekvc@undefined
          \else
            \ekvcSplit@build@argspec
            \ekvc@any@long\expandafter
            \def\csname ekvc@\string#1\expandafter\endcsname\ekvc@tmp{#3}%
          \fi
        \fi
      }%
  }
%    \end{macrocode}
% \end{macro}
%
% \begin{macro}[internal]{\ekvcSplit@build@argspec,\ekvcSplit@build@argspec@}
%    \begin{macrocode}
\protected\def\ekvcSplit@build@argspec
  {%
    \begingroup
      \edef\ekvc@tmp
        {\endgroup\def\unexpanded{\ekvc@tmp}{\ekvcSplit@build@argspec@{1}}}%
    \ekvc@tmp
  }
\def\ekvcSplit@build@argspec@#1%
  {%
    \ifnum#1>\ekvc@keycount
      \ekv@fi@gobble
    \fi
    \@firstofone
      {%
        \unexpanded\expandafter{\csname ekvc@splitmark@#1\endcsname####}#1%
        \expandafter\ekvcSplit@build@argspec@\expandafter{\the\numexpr#1+1}%
      }%
  }
%    \end{macrocode}
% \end{macro}
%
% \begin{macro}[internal]
%   {
%     \ekvc@SetupSplitKeys, \ekvc@SetupSplitKeys@a, \ekvc@SetupSplitKeys@b,
%     \ekvc@SetupSplitKeys@c
%   }
%   These macros parse the list of keys and set up the key macros. First we need
%   to initialise some macros and start |\ekvparse|.
%    \begin{macrocode}
\protected\long\def\ekvc@SetupSplitKeys
  {%
    \ekvc@keycount=0
    \let\ekvc@any@long\ekv@empty
    \let\ekvc@initials\ekv@empty
    \ekvparse\ekvc@err@value@required\ekvc@SetupSplitKeys@a
  }
%    \end{macrocode}
%   Then we need to step the key counter for each key. Also we have to check
%   whether this key has a |long| prefix so we initialise |\ekvc@long|.
%    \begin{macrocode}
\protected\def\ekvc@SetupSplitKeys@a#1%
  {%
    \advance\ekvc@keycount1
    \let\ekvc@long\ekv@empty
    \ekvc@ifspace{#1}%
      {\ekvc@SetupSplitKeys@b#1\ekvc@stop}%
      {\ekvc@SetupSplitKeys@c{#1}}%
  }
%    \end{macrocode}
%   If there was a space, there might be a prefix. If so call the prefix macro,
%   else call the next step |\ekvc@SetupSplitKeys@c| which will define the key
%   macro and add the key's value to the initials list.
%    \begin{macrocode}
\protected\def\ekvc@SetupSplitKeys@b#1 #2\ekvc@stop
  {%
    \ekv@ifdefined{ekvc@split@p@#1}%
      {\csname ekvc@split@p@#1\endcsname{#2}}%
      {\ekvc@SetupSplitKeys@c{#1 #2}}%
  }
%    \end{macrocode}
%   The inner definition is grouped, because we don't want to actually define
%   the marks we build with |\csname|. We have to append the value to the
%   |\ekvc@initials| list here with the correct split mark. The key macro will
%   read everything up to those split marks and change the value following it to
%   the value given to the key. Additionally we'll need a sorting macro for each
%   key count in use so we set it up with |\ekvc@setup@splitmacro|.
%    \begin{macrocode}
\protected\long\def\ekvc@SetupSplitKeys@c#1#2%
  {%
    \begingroup
    \edef\ekvc@tmp
      {%
        \endgroup
        \long\def\unexpanded{\ekvc@tmp}####1####2%
            \unexpanded\expandafter
            {\csname ekvc@splitmark@\the\ekvc@keycount\endcsname}####3%
          {%
            ####2%
            \unexpanded\expandafter
              {\csname ekvc@splitmark@\the\ekvc@keycount\endcsname}{####1}%
          }%
%    \end{macrocode}
%   The short variant needs a bit of special treatment. The key macro will be
%   short to throw the correct error, but since there might be long macros
%   somewhere the reordering of arguments needs to be long, so for short keys we
%   use a two step approach, first grabbing only the short argument, then
%   reordering.
%    \begin{macrocode}
        \unless\ifx\ekvc@long\long
          \let\unexpanded\expandafter
            {\csname ekvc@\ekvc@set(#1)\endcsname\ekvc@tmp}%
          \def\unexpanded{\ekvc@tmp}####1%
            {%
              \unexpanded\expandafter{\csname ekvc@\ekvc@set(#1)\endcsname}%
                {####1}%
            }%
        \fi
        \def\unexpanded{\ekvc@initials}%
          {%
            \unexpanded\expandafter{\ekvc@initials}%
            \unexpanded\expandafter
              {\csname ekvc@splitmark@\the\ekvc@keycount\endcsname{#2}}%
          }%
      }%
    \ekvc@tmp
    \ekvlet\ekvc@set{#1}\ekvc@tmp
    \ekvc@helpers@needed
      {\expandafter\ekvc@setup@splitmacro\expandafter{\the\ekvc@keycount}}%
      {}%
  }
%    \end{macrocode}
% \end{macro}
%
% \begin{macro}[internal]{\ekvc@split@p@long}
%   The |long| prefix lets the internals |\ekvc@long| and |\ekvc@any@long| to
%   |\long| so that the key macro will be long.
%    \begin{macrocode}
\protected\def\ekvc@split@p@long
  {%
    \let\ekvc@long\long
    \let\ekvc@any@long\long
    \ekvc@SetupSplitKeys@c
  }
%    \end{macrocode}
% \end{macro}
%
% \begin{macro}[internal]{\ekvc@defarggobbler}
%   This is needed to define a macro with 1-9 parameters programmatically.
%   \LaTeX's \cs[no-index]{newcommand} does something similar for example.
%    \begin{macrocode}
\protected\def\ekvc@defarggobbler#1{\def\ekvc@tmp##1#1##2##{##1#1}}
%    \end{macrocode}
% \end{macro}
%
% \begin{macro}[internal]
%   {
%     \ekvc@setup@splitmacro,
%     \ekvc@split@1, \ekvc@split@2, \ekvc@split@3,
%     \ekvc@split@4, \ekvc@split@5, \ekvc@split@6,
%     \ekvc@split@7
%   }
%   Since the first few split macros are different from the others we manually
%   set those up now. All the others will be defined as needed (always
%   globally).  The split macros just read up until the correct split mark, move
%   that argument into a list and reinsert the rest, calling the next split
%   macro afterwards.
%    \begin{macrocode}
\begingroup
\edef\ekvc@tmp
  {%
    \long\gdef\unexpanded\expandafter{\csname ekvc@split@1\endcsname}%
        \unexpanded\expandafter{\csname ekvc@splitmark@1\endcsname}%
        ##1##2##3%
      {##3{##1}##2}%
    \long\gdef\unexpanded\expandafter{\csname ekvc@split@2\endcsname}%
        \unexpanded\expandafter{\csname ekvc@splitmark@1\endcsname}##1%
        \unexpanded\expandafter{\csname ekvc@splitmark@2\endcsname}##2%
        ##3##4%
      {##4{##1}{##2}##3}%
    \long\gdef\unexpanded\expandafter{\csname ekvc@split@3\endcsname}%
        \unexpanded\expandafter{\csname ekvc@splitmark@1\endcsname}##1%
        \unexpanded\expandafter{\csname ekvc@splitmark@2\endcsname}##2%
        \unexpanded\expandafter{\csname ekvc@splitmark@3\endcsname}##3%
        ##4##5%
      {##5{##1}{##2}{##3}##4}%
    \long\gdef\unexpanded\expandafter{\csname ekvc@split@4\endcsname}%
        \unexpanded\expandafter{\csname ekvc@splitmark@1\endcsname}##1%
        \unexpanded\expandafter{\csname ekvc@splitmark@2\endcsname}##2%
        \unexpanded\expandafter{\csname ekvc@splitmark@3\endcsname}##3%
        \unexpanded\expandafter{\csname ekvc@splitmark@4\endcsname}##4%
        ##5##6%
      {##6{##1}{##2}{##3}{##4}##5}%
    \long\gdef\unexpanded\expandafter{\csname ekvc@split@5\endcsname}%
        \unexpanded\expandafter{\csname ekvc@splitmark@1\endcsname}##1%
        \unexpanded\expandafter{\csname ekvc@splitmark@2\endcsname}##2%
        \unexpanded\expandafter{\csname ekvc@splitmark@3\endcsname}##3%
        \unexpanded\expandafter{\csname ekvc@splitmark@4\endcsname}##4%
        \unexpanded\expandafter{\csname ekvc@splitmark@5\endcsname}##5%
        ##6##7%
      {##7{##1}{##2}{##3}{##4}{##5}##6}%
    \long\gdef\unexpanded\expandafter{\csname ekvc@split@6\endcsname}%
        \unexpanded\expandafter{\csname ekvc@splitmark@1\endcsname}##1%
        \unexpanded\expandafter{\csname ekvc@splitmark@2\endcsname}##2%
        \unexpanded\expandafter{\csname ekvc@splitmark@3\endcsname}##3%
        \unexpanded\expandafter{\csname ekvc@splitmark@4\endcsname}##4%
        \unexpanded\expandafter{\csname ekvc@splitmark@5\endcsname}##5%
        \unexpanded\expandafter{\csname ekvc@splitmark@6\endcsname}##6%
        ##7##8%
      {##8{##1}{##2}{##3}{##4}{##5}{##6}##7}%
    \long\gdef\unexpanded\expandafter{\csname ekvc@split@7\endcsname}%
        \unexpanded\expandafter{\csname ekvc@splitmark@1\endcsname}##1%
        \unexpanded\expandafter{\csname ekvc@splitmark@2\endcsname}##2%
        \unexpanded\expandafter{\csname ekvc@splitmark@3\endcsname}##3%
        \unexpanded\expandafter{\csname ekvc@splitmark@4\endcsname}##4%
        \unexpanded\expandafter{\csname ekvc@splitmark@5\endcsname}##5%
        \unexpanded\expandafter{\csname ekvc@splitmark@6\endcsname}##6%
        \unexpanded\expandafter{\csname ekvc@splitmark@7\endcsname}##7%
        ##8##9%
      {##9{##1}{##2}{##3}{##4}{##5}{##6}{##7}##8}%
  }
\ekvc@tmp
\endgroup
\protected\def\ekvc@setup@splitmacro#1%
  {%
    \ekv@ifdefined{ekvc@split@#1}{}%
      {%
        \begingroup
          \edef\ekvc@tmp
            {%
              \long\gdef
                  \unexpanded\expandafter{\csname ekvc@split@#1\endcsname}%
                  ####1%
                  \unexpanded\expandafter{\csname ekvc@splitmark@#1\endcsname}%
                  ####2####3%
                {%
                  \unexpanded\expandafter
                    {\csname ekvc@split@\the\numexpr#1-1\relax\endcsname}%
                  ####1{{####2}####3}%
                }%
            }%
          \ekvc@tmp
        \endgroup
      }%
  }
%    \end{macrocode}
% \end{macro}
%
% \begin{macro}{\ekvcHashAndUse}
%   |\ekvcHashAndUse| works just like |\ekvcSplitAndUse|.
%    \begin{macrocode}
\protected\long\def\ekvcHashAndUse#1#2%
  {%
    \let\ekvc@helpers@needed\@firstoftwo
    \ekvc@ifdefined#1%
      {\ekvc@err@already@defined#1}%
      {\ekvcHashAndUse@#1{}{#2}}%
  }
%    \end{macrocode}
% \end{macro}
%
% \begin{macro}[internal]{\ekvcHashAndUse@}
%   This is more or less the same as |\ekvcSplitAndUse@|. Instead of an empty
%   group we place a marker after the initials, we don't use the sorting macros
%   of |split|, but instead pack all the values in one argument.
%    \begin{macrocode}
\protected\long\def\ekvcHashAndUse@#1#2#3%
  {%
    \edef\ekvc@set{\string#1}%
    \ekvc@SetupHashKeys{#3}%
    \ekvc@helpers@needed
      {%
        \ekvc@any@long\edef#1##1%
          {%
            \expandafter\ekvc@ekvset@pre@expander\expandafter{\ekvc@set}%
            \unexpanded{\ekvc@hash@pack@argument}%
            \unexpanded\expandafter{\ekvc@initials\ekvc@stop#2}%
          }%
      }%
      {%
        \ekvc@any@long\edef#1##1%
          {%
            \expandafter\ekvc@ekvset@pre@expander\expandafter{\ekvc@set}%
            \unexpanded{#2}%
            \unexpanded\expandafter{\ekvc@initials\ekvc@stop}%
          }%
      }%
  }
%    \end{macrocode}
% \end{macro}
%
% \begin{macro}{\ekvcHashAndForward}
%   |\ekvcHashAndForward| works just like |\ekvcSplitAndForward|.
%    \begin{macrocode}
\protected\long\def\ekvcHashAndForward#1#2#3%
  {%
    \let\ekvc@helpers@needed\@firstoftwo
    \ekvc@ifdefined#1%
      {\ekvc@err@already@defined#1}%
      {\ekvcHashAndUse@#1{{#2}}{#3}}%
  }
%    \end{macrocode}
% \end{macro}
%
% \begin{macro}{\ekvcHash}
%   |\ekvcHash| does the same as |\ekvcSplit|, but has the advantage of not
%   needing to count arguments, so the definition of the internal macro is a bit
%   more straight forward.
%    \begin{macrocode}
\protected\long\def\ekvcHash#1#2#3%
  {%
    \let\ekvc@helpers@needed\@secondoftwo
    \ekvc@ifdefined#1%
      {\ekvc@err@already@defined#1}%
      {%
        \expandafter
        \ekvcHashAndUse@\expandafter#1\csname ekvc@\string#1\endcsname{#2}%
        \ekvc@any@long\expandafter\def\csname ekvc@\string#1\endcsname
            ##1\ekvc@stop
          {#3}%
      }%
  }
%    \end{macrocode}
% \end{macro}
%
% \begin{macro}[internal]{\ekvc@hash@pack@argument}
%   All this macro does is pack the values into one argument and forward that to
%   the next macro.
%    \begin{macrocode}
\long\def\ekvc@hash@pack@argument#1\ekvc@stop#2{#2{#1}}
%    \end{macrocode}
% \end{macro}
%
% \begin{macro}[internal]
%   {\ekvc@SetupHashKeys,\ekvc@SetupHashKeys@a,\ekvc@SetupHashKeys@b}
%   This should look awfully familiar as well, since it's just the same as for
%   the split keys with a few other names here and there.
%    \begin{macrocode}
\protected\long\def\ekvc@SetupHashKeys#1%
  {%
    \let\ekvc@any@long\ekv@empty
    \let\ekvc@initials\ekv@empty
    \ekvparse\ekvc@err@value@required\ekvc@SetupHashKeys@a{#1}%
  }
\protected\def\ekvc@SetupHashKeys@a#1%
  {%
    \let\ekvc@long\ekv@empty
    \ekvc@ifspace{#1}%
      {\ekvc@SetupHashKeys@b#1\ekvc@stop}%
      {\ekvc@SetupHashKeys@c{#1}}%
  }
\protected\def\ekvc@SetupHashKeys@b#1 #2\ekvc@stop
  {%
    \ekv@ifdefined{ekvc@hash@p@#1}%
      {\csname ekvc@hash@p@#1\endcsname{#2}}%
      {\ekvc@SetupHashKeys@c{#1 #2}}%
  }
%    \end{macrocode}
%   Yes, even the defining macro looks awfully familiar. Instead of numbered we
%   have named marks. Still the key macros grab everything up to their
%   respective mark and reorder the arguments. The same quirk is applied for
%   short keys. And instead of the |\ekvc@setup@splitmacro| we use
%   |\ekvc@setup@hashmacro|.
%    \begin{macrocode}
\protected\long\def\ekvc@SetupHashKeys@c#1#2%
  {%
    \begingroup
    \edef\ekvc@tmp
      {%
        \endgroup
        \long\def\unexpanded{\ekvc@tmp}####1####2%
            \unexpanded\expandafter{\csname ekvc@hashmark@#1\endcsname}####3%
          {%
            ####2%
            \unexpanded\expandafter{\csname ekvc@hashmark@#1\endcsname}{####1}%
          }%
        \unless\ifx\ekvc@long\long
          \let\unexpanded\expandafter
            {\csname ekvc@\ekvc@set(#1)\endcsname\ekvc@tmp}%
          \def\unexpanded{\ekvc@tmp}####1%
            {%
              \unexpanded\expandafter{\csname ekvc@\ekvc@set(#1)\endcsname}%
                {####1}%
            }%
        \fi
        \def\unexpanded{\ekvc@initials}%
          {%
            \unexpanded\expandafter{\ekvc@initials}%
            \unexpanded\expandafter{\csname ekvc@hashmark@#1\endcsname{#2}}%
          }%
      }%
    \ekvc@tmp
    \ekvlet\ekvc@set{#1}\ekvc@tmp
    \ekvc@setup@hashmacro{#1}%
  }
%    \end{macrocode}
% \end{macro}
%
% \begin{macro}[internal]{\ekvc@hash@p@long}
%   Nothing astonishing here either.
%    \begin{macrocode}
\protected\def\ekvc@hash@p@long
  {%
    \let\ekvc@long\long
    \let\ekvc@any@long\long
    \ekvc@SetupHashKeys@c
  }
%    \end{macrocode}
% \end{macro}
%
% \begin{macro}[internal]{\ekvc@setup@hashmacro}
%   The safe hash macros will be executed inside of an |\unexpanded| expansion
%   context, so they have to insert braces for that once they are done.
%   Most of the tests which have to be executed will already be done, but we
%   have to play safe if the hash doesn't show up in the hash list. Therefore we
%   use some |\ekvc@mark|s and |\ekvc@stop| to throw errors if the hash isn't
%   found in the right place. The fast variants have an easier life and just
%   return the correct value.
%    \begin{macrocode}
\protected\def\ekvc@setup@hashmacro#1%
  {%
    \ekv@ifdefined{ekvc@fasthash@#1}{}%
      {%
        \begingroup
          \edef\ekvc@tmp
            {%
              \long\gdef
                \unexpanded\expandafter{\csname ekvc@fasthash@#1\endcsname}%
                  ####1%
                  \unexpanded\expandafter{\csname ekvc@hashmark@#1\endcsname}%
                  ####2####3\unexpanded{\ekvc@stop}%
                {####2}%
              \long\gdef
                \unexpanded\expandafter{\csname ekvc@safehash@#1\endcsname}%
                  ####1%
                {%
                  \unexpanded\expandafter{\csname ekvc@@safehash@#1\endcsname}%
                  ####1\unexpanded{\ekvc@mark}{}%
                  \unexpanded\expandafter
                    {%
                      \csname ekvc@hashmark@#1\endcsname{}%
                      \ekvc@mark{\ekvc@err@missing@hash{#1}}\ekvc@stop
                    }%
                }%
              \long\gdef
                \unexpanded\expandafter{\csname ekvc@@safehash@#1\endcsname}%
                  ####1%
                  \unexpanded\expandafter{\csname ekvc@hashmark@#1\endcsname}%
                  ####2####3\unexpanded{\ekvc@mark}####4####5%
                  \unexpanded{\ekvc@stop}%
                {%
                  ####4{####2}%
                }%
              \long\gdef\unexpanded\expandafter
                {\csname ekvc@fastsplithash@#1\endcsname}%
                  ####1%
                  \unexpanded\expandafter{\csname ekvc@hashmark@#1\endcsname}%
                  ####2####3\unexpanded{\ekvc@stop}####4%
                {%
                  ####4{####1####3}{####2}%
                }%
              \long\gdef\unexpanded\expandafter
                {\csname ekvc@safesplithash@#1\endcsname}####1%
                {%
                  \unexpanded\expandafter
                    {\csname ekvc@@safesplithash@#1\endcsname}%
                  ####1\unexpanded{\ekvc@mark\ekvc@safe@found@hash}%
                  \unexpanded\expandafter
                    {%
                      \csname ekvc@hashmark@#1\endcsname{}%
                      \ekvc@mark{\ekvc@err@missing@hash{#1}\ekvc@safe@no@hash}%
                      \ekvc@stop
                    }%
                }%
              \long\gdef\unexpanded\expandafter
                {\csname ekvc@@safesplithash@#1\endcsname}%
                  ####1%
                  \unexpanded\expandafter{\csname ekvc@hashmark@#1\endcsname}%
                  ####2####3\unexpanded{\ekvc@mark}####4####5%
                  \unexpanded{\ekvc@stop}%
                {%
                  ####4{####2}####1####3\unexpanded{\ekvc@stop}%
                }%
            }%
          \ekvc@tmp
        \endgroup
      }%
  }
%    \end{macrocode}
% \end{macro}
%
% \begin{macro}{\ekvcValue}
%   All this does is a few consistency checks on the first argument (not empty,
%   hash macro exists) and then call that hash-grabbing macro that will also
%   test whether the hash is inside of |#2| or not.
%    \begin{macrocode}
\long\def\ekvcValue#1%
  {%
    \unexpanded
    \ekv@ifdefined{ekvc@safehash@#1}%
      {\csname ekvc@safehash@#1\endcsname}%
      {\ekvc@err@unknown@hash{#1}\@firstoftwo{{}}}%
  }
%    \end{macrocode}
% \end{macro}
%
% \begin{macro}{\ekvcValueFast}
%   To be as fast as possible, this doesn't test for anything, assuming the user
%   knows best.
%    \begin{macrocode}
\long\def\ekvcValueFast#1#2{\csname ekvc@fasthash@#1\endcsname#2\ekvc@stop}
%    \end{macrocode}
% \end{macro}
%
% \begin{macro}{\ekvcValueSplit,\ekvcValueSplit@recover}
%   This splits off a single value.
%    \begin{macrocode}
\long\def\ekvcValueSplit#1%
  {%
    \ekv@ifdefined{ekvc@safesplithash@#1}%
      {\csname ekvc@safesplithash@#1\endcsname}%
      {\ekvc@err@unknown@hash{#1}\ekvcValueSplit@recover}%
  }
\long\def\ekvcValueSplit@recover#1#2%
  {%
    #2{#1}{}%
  }
%    \end{macrocode}
% \end{macro}
%
% \begin{macro}[internal]{\ekvc@safe@found@hash, \ekvc@safe@no@hash}
%    \begin{macrocode}
\long\def\ekvc@safe@found@hash#1#2\ekvc@stop#3%
  {%
    #3{#2}{#1}%
  }
\long\def\ekvc@safe@no@hash#1#2\ekvc@mark\ekvc@safe@found@hash\ekvc@stop#3%
  {%
    #3{#2}{}%
  }
%    \end{macrocode}
% \end{macro}
%
% \begin{macro}{\ekvcValueSplitFast}
%   Again a fast approach which doesn't provide too many safety measurements.
%   This needs to build the hash function and expand it before passing the
%   results to the next control sequence. The first step only builds the control
%   sequence.
%    \begin{macrocode}
\long\def\ekvcValueSplitFast#1#2%
  {%
    \csname ekvc@fastsplithash@#1\endcsname#2\ekvc@stop
  }
%    \end{macrocode}
% \end{macro}
%
% \begin{macro}[internal]
%   {
%     \ekvc@safehash@,\ekvc@fasthash@,
%     \ekvc@safesplithash@,\ekvc@fastsplithash@
%   }
%   At least in the empty hash case we can provide a meaningful error message
%   without affecting performance by just defining the macro that would be build
%   in that case. There is of course a downside to this, the error will not be
%   thrown by |\ekvcValueFast| in three expansion steps. The safe hash variant
%   has to also stop the |\unexpanded| expansion.
%    \begin{macrocode}
\long\def\ekvc@safehash@#1{\ekvc@err@empty@hash{}}
\long\def\ekvc@fasthash@#1\ekvc@stop{\ekvc@err@empty@hash}
\long\def\ekvc@safesplithash@#1#2{\ekvc@err@empty@hash#2{#1}{}}
\long\def\ekvc@fastsplithash@#1\ekvc@stop#2{\ekvc@err@empty@hash#2{#1}{}}
%    \end{macrocode}
% \end{macro}
%
% \begin{macro}{\ekvcSecondaryKeys}
%   The secondary keys are defined pretty similar to the way the originals are,
%   but here we also introduce some key types (those have a |@t@| in their name)
%   additionally to the prefixes.
%    \begin{macrocode}
\protected\long\def\ekvcSecondaryKeys#1#2%
  {%
    \edef\ekvc@set{\string#1}%
    \ekvparse\ekvc@err@value@required\ekvcSecondaryKeys@a{#2}%
  }
\protected\def\ekvcSecondaryKeys@a#1%
  {%
    \let\ekvc@long\ekv@empty
    \ekvc@ifspace{#1}%
      {\ekvcSecondaryKeys@b#1\ekvc@stop}%
      {\ekvc@err@missing@type{#1}\@gobble}%
  }
\protected\def\ekvcSecondaryKeys@b#1 #2\ekvc@stop
  {%
    \ekv@ifdefined{ekvc@p@#1}%
      {\csname ekvc@p@#1\endcsname}%
      {%
        \ekv@ifdefined{ekvc@t@#1}%
          {\csname ekvc@t@#1\endcsname}%
          {\ekvc@err@unknown@keytype{#1}\@firstoftwo\@gobble}%
      }%
      {#2}%
  }
%    \end{macrocode}
% \end{macro}
%
% \begin{macro}{\ekvcChange}
%   This can be used to change the defaults of an \expkvc\ defined macro. It
%   checks whether there is a set with the correct name and that the macro is
%   defined. If both is true the real work is done by |\ekvc@change|.
%    \begin{macrocode}
\protected\long\def\ekvcChange#1%
  {%
    \ekvifdefinedset{\string#1}%
      {%
        \ekvc@ifdefined#1%
          {\ekvc@change#1}%
          {\ekvc@err@no@key@macro#1\@gobble}%
      }%
      {\ekvc@err@no@key@macro#1\@gobble}%
  }
%    \end{macrocode}
% \end{macro}
%
% \begin{macro}[internal]
%   {\ekvc@change,\ekvc@change@a,\ekvc@change@b,\ekvc@change@c}
%   First we need to see whether the macro is currently |\long|. For this we get
%   the meaning and will parse it. |#1| is the macro name in which we want to
%   change the defaults.
%    \begin{macrocode}
\protected\def\ekvc@change#1%
  {\expandafter\ekvc@change@a\meaning#1\ekv@stop#1}
%    \end{macrocode}
%   A temporary definition to get the stringified |macro:|. |##1| will be the
%   list of prefixes, we don't care for the exact contents of |##2| and |##3|.
%    \begin{macrocode}
\def\ekvc@change@a#1%
  {%
    \protected\def\ekvc@change@a##1#1##2->##3\ekv@stop
      {%
        \ekvc@change@iflong{##1}%
          {\ekvc@change@b{}}%
          {\ekvc@change@b{\long}}%
      }%
  }
\expandafter\ekvc@change@a\expandafter{\detokenize{macro:}}
%    \end{macrocode}
%   Next we expand the macro once to get its contents (including the current
%   default values with their markers). |#1| is either |\long| or empty, |#2| is
%   the macro.
%    \begin{macrocode}
\protected\def\ekvc@change@b#1#2%
  {\expandafter\ekvc@change@c\expandafter{#2{##1}}{#1}#2}
%    \end{macrocode}
%   Here we place an unbalanced closing brace after the expansion of the macro.
%   Then we just parse the \kv-list with |\ekvset|, that will exchange the
%   values behind the markers. Once those are changed we give control to
%   |\ekvc@change@d|. The |\ekvset| step might horribly fail if the user defined
%   some keys that don't behave nice. |#1| is the expansion of the macro, |#2|
%   is either |\long| or empty, |#3| is the macro, and |#4| is the \kv-list
%   containing the new defaults.
%    \begin{macrocode}
\ekv@exparg{\protected\long\def\ekvc@change@c#1#2#3#4}%
  {%
    \expandafter\iffalse\expandafter{\expandafter\fi
      \ekvset{\string#3}{#4}%
      \ekvc@change@d{#2}{#3}%
      #1%
    }%
  }
%    \end{macrocode}
%   The final step needs to put an unbalanced opening brace after |\def|. We do
%   that with the help of a temporary macro which stores everything necessary
%   for |\def| and expand an |\iffalse}\fi| construct to leave a single
%   opening brace. |#1| will be either empty or |\long| and |#2| is the macro.
%   Each of the macros defined with \expkvc\ takes exactly one parameter, so we
%   put that here as |####1|.
%    \begin{macrocode}
\protected\def\ekvc@change@d#1#2%
  {%
    \def\ekvc@tmp{#1\def#2####1}%
    \expandafter\ekvc@tmp\expandafter{\iffalse}\fi
  }
%    \end{macrocode}
% \end{macro}
%
% \begin{macro}[internal]{\ekvc@change@iflong,\ekvc@change@iflong@}
%   Checking whether a string contains the string representation of |\long| can
%   be done by gobbling everything up to the first |\long| and checking whether
%   the result is completely empty. We need a temporary macro to get the result
%   of |\string\long| inside the definitions.
%    \begin{macrocode}
\def\ekvc@change@iflong#1%
  {%
    \protected\def\ekvc@change@iflong##1%
      {\expandafter\ekv@ifempty\expandafter{\ekvc@change@iflong@##1#1}}%
    \def\ekvc@change@iflong@##1#1{}
  }
\expandafter\ekvc@change@iflong\expandafter{\string\long}
%    \end{macrocode}
% \end{macro}
%
% \begin{macro}{\ekvcPass}
%   This macro can be used to pass a value to a key of some macro (this way
%   more complicated key codes are possible that in the end pass processed
%   values on to some macro). The implemantation is pretty straight forward.
%    \begin{macrocode}
\long\def\ekvcPass#1#2%
  {%
    \ekvifdefined{\string#1}{#2}%
      {\csname\ekv@name{\string#1}{#2}\endcsname}%
      {\ekvc@err@unknown@key@or@macro{#1}{#2}\@gobble}%
  }
%    \end{macrocode}
% \end{macro}
%
%
% \subsubsection{Secondary Key Types}
%
% \begin{macro}[internal]{\ekvc@p@long,\ekvc@after@ptype}
%   The prefixes are pretty straight forward again. Just set |\ekvc@long| and
%   forward to the |@t@| type.
%    \begin{macrocode}
\protected\def\ekvc@p@long#1%
  {%
    \ekvc@ifspace{#1}%
      {%
        \let\ekvc@long\long
        \ekvc@after@ptype#1\ekvc@stop
      }%
      {\ekvc@err@missing@type{long #1}\@gobble}%
  }
\protected\def\ekvc@after@ptype#1 #2\ekvc@stop
  {%
    \ekv@ifdefined{ekvc@t@#1}%
      {\csname ekvc@t@#1\endcsname{#2}}%
      {\ekvc@err@unknown@keytype{#1}\@gobble}%
  }
%    \end{macrocode}
% \end{macro}
%
% \begin{macro}[internal]
%   {
%     \ekvc@t@meta, \ekvc@t@nmeta,
%     \ekvc@type@meta, \ekvc@type@meta@a, \ekvc@type@meta@b
%   }
%   The |meta| and |nmeta| key types use a nested |\ekvset| to set other keys in
%   the same macro's \set.
%    \begin{macrocode}
\protected\def\ekvc@t@meta
  {%
    \edef\ekvc@tmp{\ekvc@set}%
    \expandafter\ekvc@type@meta\expandafter{\ekvc@tmp}\ekvc@long{##1}\ekvlet
  }
\protected\def\ekvc@t@nmeta#1%
  {%
    \ekvc@assert@not@long{nmeta #1}%
    \edef\ekvc@tmp{\ekvc@set}%
    \expandafter\ekvc@type@meta\expandafter{\ekvc@tmp}{}{}\ekvletNoVal{#1}%
  }
\protected\long\def\ekvc@type@meta#1#2#3#4#5#6%
  {%
    \expandafter\ekvc@type@meta@a\expandafter{\ekvset{#1}{#6}}{#2}{#3}%
    #4\ekvc@set{#5}\ekvc@tmp
  }
\protected\def\ekvc@type@meta@a
  {%
    \expandafter\ekvc@type@meta@b\expandafter
  }
\protected\long\def\ekvc@type@meta@b#1#2#3%
  {%
    #2\def\ekvc@tmp#3{#1}%
  }
%    \end{macrocode}
% \end{macro}
%
% \begin{macro}[internal]{\ekvc@t@alias}
%   |alias| just checks whether there is a key and/or |NoVal| key defined with
%   the target name and |\let| the key to those.
%    \begin{macrocode}
\protected\def\ekvc@t@alias#1#2%
  {%
    \ekvc@assert@not@long{alias #1}%
    \let\ekvc@tmp\@firstofone
    \ekvifdefined\ekvc@set{#2}%
      {%
        \ekvletkv\ekvc@set{#1}\ekvc@set{#2}%
        \let\ekvc@tmp\@gobble
      }%
      {}%
    \ekvifdefinedNoVal\ekvc@set{#2}%
      {%
        \ekvletkvNoVal\ekvc@set{#1}\ekvc@set{#2}%
        \let\ekvc@tmp\@gobble
      }%
      {}%
    \ekvc@tmp{\ekvc@err@unknown@key{#2}}%
  }
%    \end{macrocode}
% \end{macro}
%
% \begin{macro}[internal]{\ekvc@t@default}
%   The |default| key can be used to set a |NoVal| key for an existing key. It
%   will just pass the \val\ to the key macro of that other key.
%    \begin{macrocode}
\protected\long\def\ekvc@t@default#1#2%
  {%
    \ekvifdefined\ekvc@set{#1}%
      {%
        \ekvc@assert@not@long{default #1}%
        \edef\ekvc@tmp
          {%
            \unexpanded\expandafter
              {\csname\ekv@name\ekvc@set{#1}\endcsname{#2}}%
          }%
        \ekvletNoVal\ekvc@set{#1}\ekvc@tmp
      }%
      {\ekvc@err@unknown@key{#1}}%
  }
%    \end{macrocode}
% \end{macro}
%
% \begin{macro}[internal]{\ekvc@t@flag-bool}
%    \begin{macrocode}
\protected\expandafter\def\csname ekvc@t@flag-bool\endcsname#1#2%
  {%
    \ekvc@assert@not@long{flag-bool #1}%
    \unless\ifdefined#2\ekvcFlagNew#2\fi
    \ekvdef\ekvc@set{#1}%
      {%
        \ekv@ifdefined{ekvc@flag@set@##1}%
          {%
            \csname ekvc@flag@set@##1\expandafter\endcsname
              \ekvcFlagHeight#2\ekv@stop#2%
          }%
          {\ekvc@err@invalid@bool{##1}}%
      }%
  }
%    \end{macrocode}
% \end{macro}
%
% \begin{macro}[internal]
%   {\ekvc@t@flag-true,\ekvc@t@flag-false,\ekvc@t@flag-raise,\ekvc@type@flag}
%    \begin{macrocode}
\protected\def\ekvc@type@flag#1#2#3#4%
  {%
    \ekvc@assert@not@long{flag-#1 #3}%
    \unless\ifdefined#4\ekvcFlagNew#4\fi
    \ekv@exparg{\ekvdefNoVal\ekvc@set{#3}}{#2#4}%
  }
\protected\expandafter\def\csname ekvc@t@flag-true\endcsname
  {\ekvc@type@flag{true}\ekvcFlagSetTrue}
\protected\expandafter\def\csname ekvc@t@flag-false\endcsname
  {\ekvc@type@flag{false}\ekvcFlagSetFalse}
\protected\expandafter\def\csname ekvc@t@flag-raise\endcsname
  {\ekvc@type@flag{raise}\ekvcFlagRaise}
%    \end{macrocode}
% \end{macro}
%
%
% \subsubsection{Flags}
%
% The basic idea of flags is to store information by the fact that \TeX\
% expandably assigns the meaning |\relax| to undefined control sequences which
% were built with |\csname|. This mechanism is borrowed from \pkg{expl3}.
%
% \begin{macro}[internal]{\ekvc@flag@name,\ekvc@flag@namescheme}
%   Flags follow a simple naming scheme which we define here. |\ekvc@flag@name|
%   will store the name of an internal function that is used to build names of
%   the second naming scheme defined by |\ekvc@flag@namescheme|.
%    \begin{macrocode}
\def\ekvc@flag@name{ekvcf\string}
\def\ekvc@flag@namescheme#1#2{ekvch#2#1}
%    \end{macrocode}
% \end{macro}
%
% \begin{macro}{\ekvcFlagHeight}
%   For semantic reasons we use |\number| with another name.
%    \begin{macrocode}
\let\ekvcFlagHeight\number
%    \end{macrocode}
% \end{macro}
%
% \begin{macro}{\ekvcFlagNew}
%   This macro defines a new flag. It stores the function build with the
%   |\ekvc@flag@name| naming scheme after the internal function
%   |\ekvc@flag@height| that'll determine the current flag height. It'll also
%   define the macro named via |\ekvc@flag@name| to build names according to
%   |\ekvc@flag@namescheme|.
%    \begin{macrocode}
\protected\def\ekvcFlagNew#1%
  {%
    \edef#1%
      {%
        \unexpanded{\ekvc@flag@height}%
          \unexpanded\expandafter{\csname\ekvc@flag@name#1\endcsname}%
      }%
    \ekv@expargtwice
      {\expandafter\def\csname\ekvc@flag@name#1\endcsname##1}%
      {\expandafter\ekvc@flag@namescheme\expandafter{\string#1}{##1}}%
  }
%    \end{macrocode}
% \end{macro}
%
% \begin{macro}[internal]{\ekvc@flag@height,\ekvc@flag@height@}
%   This macro gets the height of a flag by a simple loop. The first loop
%   iteration differs a bit from the following in that it doesn't have to get
%   the current iteration count. The space at the end of |\ekvc@flag@height|
%   ends the |\number| evaluation.
%    \begin{macrocode}
\def\ekvc@flag@height#1%
  {%
    \ifcsname#10\endcsname
      \ekvc@flag@height@1\ekv@stop#1%
    \fi
    \@firstofone{0} % leave this space
  }
\def\ekvc@flag@height@#1\ekv@stop#2\fi\@firstofone#3%
  {%
    \fi
    \ifcsname#2{#1}\endcsname
      \expandafter\ekvc@flag@height@\the\numexpr#1+1\relax\ekv@stop#2%
    \fi
    \@firstofone{#1}%
  }
%    \end{macrocode}
% \end{macro}
%
% \begin{macro}{\ekvcFlagRaise}
%   Raising a flag simply means letting the |\ekvc@flag@namescheme| macro for
%   the current height to relax. The result of raising a flag is that its height
%   is bigger by $1$.
%    \begin{macrocode}
\ekv@exparg{\def\ekvcFlagRaise#1}%
  {%
    \expandafter\expandafter\expandafter\@gobble\expandafter
      \csname\ekvc@flag@namescheme{\string#1}{\ekvcFlagHeight#1}\endcsname
  }
%    \end{macrocode}
% \end{macro}
%
% \begin{macro}{\ekvcFlagSetTrue,\ekvcFlagSetFalse}
% \begin{macro}[internal]{\ekvc@flag@set@true,\ekvc@flag@set@false}
%   A flag is considered true if its current height is odd, and as false if it
%   is even. Therefore |\ekvcFlagSetTrue| and |\ekvcFlagSetFalse| only need to
%   raise the flag if the opposing boolean value is the current one.
%    \begin{macrocode}
\def\ekvcFlagSetTrue#1%
  {\expandafter\ekvc@flag@set@true\ekvcFlagHeight#1\ekv@stop#1}
\def\ekvcFlagSetFalse#1%
  {\expandafter\ekvc@flag@set@false\ekvcFlagHeight#1\ekv@stop#1}
%    \end{macrocode}
%   We can expand |\ekvc@flag@namescheme| at definition time here, which is why
%   we're using a temporary definition to set up |\ekvc@flag@set@true| and
%   |\ekvc@flag@set@false|.
%    \begin{macrocode}
\def\ekvc@flag@set@true#1%
  {%
    \def\ekvc@flag@set@true##1\ekv@stop##2%
      {%
        \ifodd##1
          \ekv@fi@gobble
        \fi
        \@firstofone{\expandafter\@gobble\csname#1\endcsname}%
      }%
    \def\ekvc@flag@set@false##1\ekv@stop##2%
      {%
        \ifodd##1
          \ekv@fi@firstofone
        \fi
        \@gobble{\expandafter\@gobble\csname#1\endcsname}%
      }%
  }
\expandafter\ekvc@flag@set@true\expandafter
  {\ekvc@flag@namescheme{\string#2}{#1}}
%    \end{macrocode}
% \end{macro}
% \end{macro}
%
% \begin{macro}{\ekvcFlagIf}
%   As already explained, truthiness is defined as a flag's height is odd, so we
%   just branch accordingly here.
%    \begin{macrocode}
\def\ekvcFlagIf#1%
  {%
    \ifodd#1%
      \ekv@fi@firstoftwo
    \fi
    \@secondoftwo
  }
%    \end{macrocode}
% \end{macro}
%
% \begin{macro}{\ekvcFlagIfRaised}
%   This macro uses flags as a switch, if a flag's current height is bigger than
%   $0$ this test yields true.
%    \begin{macrocode}
\ekv@exparg{\def\ekvcFlagIfRaised#1}%
  {%
    \expandafter\ifcsname\ekvc@flag@namescheme{\string#1}0\endcsname
      \ekv@fi@firstoftwo
    \fi
    \@secondoftwo
  }
%    \end{macrocode}
% \end{macro}
%
% \begin{macro}{\ekvcFlagReset}
% \begin{macro}[internal]{\ekvc@flag@reset,\ekvc@flag@reset@}
%   Resetting works by locally letting all the defined internal macros named
%   after |\ekvc@flag@namescheme| to undefined.
%    \begin{macrocode}
\protected\def\ekvcFlagReset#1%
  {\expandafter\ekvc@flag@reset\csname\ekvc@flag@name#1\endcsname}
\protected\def\ekvc@flag@reset#1%
  {%
    \ifcsname#10\endcsname
      \expandafter\let\csname#10\endcsname\ekvc@undefined
      \ekvc@flag@reset@1\ekv@stop#1%
    \fi
  }
\protected\def\ekvc@flag@reset@#1\ekv@stop#2\fi
  {%
    \fi
    \ifcsname#2{#1}\endcsname
      \expandafter\let\csname#2{#1}\endcsname\ekvc@undefined
      \expandafter\ekvc@flag@reset@\the\numexpr#1+1\relax\ekv@stop#2%
    \fi
  }
%    \end{macrocode}
% \end{macro}
% \end{macro}
%
% \begin{macro}{\ekvcFlagGetHeight}
% \begin{macro}[internal]{\ekvc@flag@get@height@single}
%   These are just small helpers, first getting the height of the flag and then
%   passing it on to the user supplied code.
%    \begin{macrocode}
\def\ekvcFlagGetHeight#1%
  {\expandafter\ekvc@flag@get@height@single\ekvcFlagHeight#1\ekv@stop}
\long\def\ekvc@flag@get@height@single#1\ekv@stop#2{#2{#1}}
%    \end{macrocode}
% \end{macro}
% \end{macro}
%
% \begin{macro}{\ekvcFlagGetHeights}
% \begin{macro}[internal]
%   {\ekvc@flag@get@heights,\ekvc@flag@get@heights@,\ekvc@flag@get@heights@done}
%   This works by a simple loop that stops at |\ekv@stop|. As long as that
%   marker isn't hit, get the next flags height and put it into a list after
%   |\ekv@stop|. |\ekvc@flag@get@heights@| uses the same marker name for the
%   end of the height, which shouldn't clash in any case. Once we're done we
%   remove the remainder of the current iteration and leave the user supplied
%   code in the input stream with all the flags' heights as a single argument.
%    \begin{macrocode}
\def\ekvcFlagGetHeights#1%
  {%
    \ekvc@flag@get@heights#1\ekv@stop{}%
  }
\def\ekvc@flag@get@heights#1%
  {%
    \ekv@gobbleto@stop#1\ekvc@flag@get@heights@done\ekv@stop
    \expandafter\ekvc@flag@get@heights@\ekvcFlagHeight#1\ekv@stop
  }
\def\ekvc@flag@get@heights@#1\ekv@stop#2\ekv@stop#3%
  {\ekvc@flag@get@heights#2\ekv@stop{#3{#1}}}
\long\def\ekvc@flag@get@heights@done
    \ekv@stop
    \expandafter\ekvc@flag@get@heights@\ekvcFlagHeight\ekv@stop\ekv@stop#1#2%
  {#2{#1}}
%    \end{macrocode}
% \end{macro}
% \end{macro}
%
%
% \subsubsection{Helper Macros}
%
% \begin{macro}[internal]{\ekvc@ifspace,\ekvc@ifspace@}
% A test which can be reduced to an if-empty by gobbling everything up to the
% first space.
%    \begin{macrocode}
\long\def\ekvc@ifspace#1%
  {%
    \ekvc@ifspace@#1 \ekv@ifempty@B
      \ekv@ifempty@false\ekv@ifempty@A\ekv@ifempty@B\@firstoftwo
  }
\long\def\ekvc@ifspace@#1 % keep this space
  {%
    \ekv@ifempty@\ekv@ifempty@A
  }
%    \end{macrocode}
% \end{macro}
%
%
% \subsubsection{Assertions}
% \begin{macro}[internal]{\ekvc@assert@not@long}
%   Some keys don't want to be |long| and we have to educate the user, so let's
%   throw an error if someone wanted these to be long.
%    \begin{macrocode}
\long\def\ekvc@assert@not@long#1{\ifx\ekvc@long\long\ekvc@err@no@long{#1}\fi}
%    \end{macrocode}
% \end{macro}
%
%
% \subsubsection{Messages}
%
% \begin{macro}[internal]
%   {
%     \ekvc@err@toomany,\ekvc@err@value@required,\ekvc@err@missing@type,
%     \ekvc@err@already@defined,\ekvc@err@no@key@macro
%   }
%   Boring unexpandable error messages.
%    \begin{macrocode}
\protected\def\ekvc@err@toomany#1%
  {%
    \errmessage{expkv-cs Error: Too many keys for macro `\string#1'}%
  }
\protected\def\ekvc@err@value@required#1%
  {%
    \errmessage{expkv-cs Error: Missing value for key `\unexpanded{#1}'}%
  }
\protected\def\ekvc@err@missing@type#1%
  {%
    \errmessage
      {expkv-cs Error: Missing type for secondary key `\unexpanded{#1}'}%
  }
\protected\def\ekvc@err@no@long#1%
  {%
    \errmessage
      {expkv-cs Error: prefix `long' not accepted for `\unexpanded{#1}'}%
  }
\protected\def\ekvc@err@already@defined#1%
  {%
    \errmessage{expkv-cs Error: Macro `\string#1' already defined}%
  }
\protected\def\ekvc@err@unknown@keytype#1%
  {%
    \errmessage{expkv-cs Error: Unknown key type `\unexpanded{#1}'}%
  }
\protected\def\ekvc@err@unknown@key#1%
  {%
    \errmessage
      {expkv-cs Error: Unknown key `\unexpanded{#1}' for macro `\ekvc@set'}%
  }
\long\def\ekvc@err@no@key@macro#1%
  {\errmessage{expkv-cs Error: \string#1 is no key=val macro}}
%    \end{macrocode}
% \end{macro}
%
% \begin{macro}[internal]{\ekvc@err}
%   We need a way to throw error messages expandably in some contexts.
%    \begin{macrocode}
\ekv@exparg{\long\def\ekvc@err#1}{\ekverr{expkv-cs}{#1}}
%    \end{macrocode}
% \end{macro}
%
% \begin{macro}[internal]
%   {
%     \ekvc@err@unknown@hash,\ekvc@err@empty@hash,\ekvc@err@missing@hash,
%     \ekvc@err@invalid@bool
%   }
%   And here are the expandable error messages.
%    \begin{macrocode}
\long\def\ekvc@err@unknown@hash#1{\ekvc@err{unknown hash `#1'}}
\long\def\ekvc@err@missing@hash#1{\ekvc@err{hash `#1' not found}}
\long\def\ekvc@err@empty@hash{\ekvc@err{empty hash}}
\def\ekvc@err@invalid@bool#1{\ekvc@err{invalid boolean value `#1'}}
\long\def\ekvc@err@unknown@key@or@macro#1#2%
  {\ekvc@err{unknown key `#2' for macro #1}}
%    \end{macrocode}
% \end{macro}
%
%
% Now everything that's left is to reset the category code of |@|.
%    \begin{macrocode}
\catcode`\@=\ekvc@tmp
%    \end{macrocode}
%
% \gobbledocstriptag
%</tex>
%^^A=<<
%
%^^A=<<
%
% \end{implementation}^^A=<<
%
% \clearpage
% \PrintIndex
%
\endinput
%
^^A vim: ft=tex fdm=marker fmr=>>=,=<<
}{}
\ProvidesFile{expkv-cs.dtx}
  [\csname ekvcDate\endcsname\ define expandable key=val macros using expkv]
\PassOptionsToPackage{full}{textcomp}
\documentclass{l3doc}
\RequirePackage[oldstylenums,nott]{kpfonts}
\input{glyphtounicode}
\pdfgentounicode=1
\RequirePackage{tcolorbox}
\newtcolorbox{exresult}[2][]
  {colback=expkvgrey!10!white,colframe=expkvgrey,fontupper=\small,width={#2},#1}
\newtcbox\exres[1][]
  {
     colback=expkvgrey!10!white
    ,colframe=expkvgrey
    ,size=small
    ,nobeforeafter
    ,tcbox raise base
    ,fontupper=\small
    ,#1
  }
\RequirePackage{listings}
\RequirePackage{booktabs}
\RequirePackage{array}
\RequirePackage{collcell}
\RequirePackage{siunitx}
\RequirePackage{xcolor}
\RequirePackage{caption}
\RequirePackage{microtype}
\RequirePackage{accsupp}
\lstdefinelanguage{expkvc}[primitive]{TeX}
  {
    ,moretexcs=[4]^^A e-TeX
      {
        expanded,
        numexpr,
        protected,
      }
    ,moretexcs=[5]^^A plain/LaTeX
      {
        approx,
        begin,
        empty,
        item,
        LaTeX,
        makeatletter,makeatother,multicolumn,
        newcommand,newdimen,
        RequirePackage,
        rule,
        TeX,
        textit,texttt,
        usepackage,
      }
    ,moretexcs=[6]^^A used packages
      {
        ^^A xfp
        fpeval,
        ^^A keyval
        setkeys,
        ^^A yax
        defactiveparameter,storevalue,setparameterlist
      }
    ,moretexcs=[2]^^A expkv-cs macros
      {
        ekvcSplit,ekvcSplitAndForward,ekvcSplitAndUse,
        ekvcHash,ekvcHashAndForward,ekvcHashAndUse,
        ekvcValue,ekvcValueFast,ekvcValueSplit,ekvcValueSplitFast,
        ekvcSecondaryKeys,
        ekvcChange,
        ekvcFlagNew,
        ekvcFlagHeight,
        ekvcFlagRaise,
        ekvcFlagSetTrue,ekvcFlagSetFalse,
        ekvcFlagIf,ekvcFlagIfRaised,
        ekvcFlagReset,
        ekvcFlagGetHeight,ekvcFlagGetHeights,
        ekvcPass
      }
    ,moretexcs=[3]^^A expkv-pkg macros
      {
        ^^A expkv
        ekvdef,ekvdefNoVal,^^A
        ekvlet,ekvletNoVal,ekvletkv,ekvletkvNoVal,^^A
        ekvdefunknown,ekvdefunknownNoVal,^^A
        ekvredirectunknown,ekvredirectunknownNoVal,^^A
        ekvset,ekvsetSneaked,^^A
        ekvsetdef,ekvsetSneakeddef,ekvsetdefSneaked,^^A
        ekvparse,^^A
        ekvVersion,ekvDate,^^A
        ekvifdefined,ekvifdefinedNoVal,ekvifdefinedset,^^A
        ekvbreak,ekvbreakPreSneak,ekvbreakPostSneak,^^A
        ekvsneak,ekvsneakPre,^^A
        ekvchangeset,^^A
        ekvoptarg,ekvoptargTF,^^A
        ekverr,
        ^^A expkv-def
        ekvdefinekeys
      }
  }
\colorlet{codeparam}{cyan!65!black}
\def\mylstwd{.55em}
\lstset
  {
    ,language=expkvc
    ,flexiblecolumns=false
    ,basewidth=\mylstwd
    ,gobble=2
    ,basicstyle=\fontfamily{jkp}\itshape
    ,texcsstyle=*[1]{\bfseries\color{expkvgrey}} ^^A primitives
    ,texcsstyle=*[2]{\bfseries\color{expkvred}}  ^^A expkv
    ,texcsstyle=*[3]{\color{expkvred}}           ^^A expkv-pkg
    ,texcsstyle=*[4]{\bfseries\color{expkvgrey}} ^^A e-TeX
    ,texcsstyle=*[5]{\bfseries\color{expkvgrey}} ^^A plain/LaTeX macros
    ,texcsstyle=*[6]{}                           ^^A macros of other packages
    ,commentstyle=\color[gray]{0.4}
    ,literate=
              {\{} {{\CodeUpBf\{}}{1}
              {\}} {{\CodeUpBf\}}}{1}
              {$}  {{\CodeUpBf\$}}{1}
              {[}  {{\CodeUp[}}{1}
              {]}  {{\CodeUp]}}{1}
              {(}  {{\CodeUp(}}{1}
              {)}  {{\CodeUp)}}{1}
              {*}  {{$*$}}{1}
              {1}  {{\CodeUp{1}}}{1}
              {2}  {{\CodeUp{2}}}{1}
              {3}  {{\CodeUp{3}}}{1}
              {4}  {{\CodeUp{4}}}{1}
              {5}  {{\CodeUp{5}}}{1}
              {6}  {{\CodeUp{6}}}{1}
              {7}  {{\CodeUp{7}}}{1}
              {8}  {{\CodeUp{8}}}{1}
              {9}  {{\CodeUp{9}}}{1}
              {0}  {{\CodeUp{0}}}{1}
              {##} {{\CodeColored{codeparam}{1}{\#}}}{1}
              {##1}{{\CodeColored{codeparam}{2}{\#1}}}{2}
              {##2}{{\CodeColored{codeparam}{2}{\#2}}}{2}
              {##3}{{\CodeColored{codeparam}{2}{\#3}}}{2}
              {##4}{{\CodeColored{codeparam}{2}{\#4}}}{2}
              {##5}{{\CodeColored{codeparam}{2}{\#5}}}{2}
              {##6}{{\CodeColored{codeparam}{2}{\#6}}}{2}
              {##7}{{\CodeColored{codeparam}{2}{\#7}}}{2}
              {##8}{{\CodeColored{codeparam}{2}{\#8}}}{2}
              {##9}{{\CodeColored{codeparam}{2}{\#9}}}{2}
              {<key>}{{$\langle$}key{$\rangle$}}{5}
              {<set>}{{$\langle$}set{$\rangle$}}{5}
              {<set1>}{{$\langle$}set1{$\rangle$}}{6}
              {<set2>}{{$\langle$}set2{$\rangle$}}{6}
    ^^A,literate=*{<key>}{\key}{4}{<set>}{\set}{4}
  }
\newcommand*\CodeColored[3]{\textcolor{#1}{\makebox[\dimexpr\mylstwd*#2]{#3}}}
\newcommand*\CodeUpBf[1]{\makebox[\mylstwd]{\textup{\textbf{#1}}}}
\newcommand*\CodeUp[1]{\makebox[\mylstwd]{\textup{#1}}}
\RequirePackage{randtext}
\let\metaORIG\meta
\protected\def\meta #1{\texttt{\metaORIG{#1}}}
\renewcommand*\thefootnote{\fnsymbol{footnote}}
\definecolor{expkvred}{HTML}{9F393D}
\colorlet{expkvgrey}{black!75}
\makeatletter
\newcommand*\example{\par\smallskip\noindent\textit{Example:} \ignorespaces}
\newcommand*\expkv
  {^^A
    \texorpdfstring
      {^^A
        \mbox
          {^^A
            \BeginAccSupp{ActualText=expkv}^^A
            \href{https://github.com/Skillmon/tex_expkv}
              {^^A
                \rmfamily
                \bfseries
                {\color{expkvgrey}e\kern-.05em x\kern-.05em}^^A
                \lower.493ex
                  \hbox{{\color{expkvgrey}P}\kern-.1em{\color{expkvred}k}}^^A
                \kern-.18em{\color{expkvred}v}^^A
              }^^A
            \EndAccSupp{}^^A
          }^^A
      }
      {expkv}^^A
  }
\newcommand*\expkvpkg[1]
  {^^A
    \texorpdfstring
      {^^A
        \mbox
          {^^A
            \BeginAccSupp{ActualText=expkv-#1}^^A
            \href{https://github.com/Skillmon/tex_expkv-#1}
              {^^A
                \rmfamily
                \bfseries
                {\color{expkvgrey}e\kern-.05em x\kern-.05em}^^A
                \lower.493ex
                  \hbox{{\color{expkvgrey}P}\kern-.1em{\color{expkvred}k}}^^A
                \kern-.18em{\color{expkvred}v}^^A
                {\color{expkvgrey}^^A
                  \kern.05em\rule[-.1ex]{.08em}{1.2ex}\kern.05em\textsc{#1}^^A
                }^^A
              }^^A
            \EndAccSupp{}^^A
          }^^A
      }
      {expkv-#1}^^A
  }
\newcommand*\expkvd{\expkvpkg{def}}
\newcommand*\expkvc{\expkvpkg{cs}}
\newcommand\kv{\meta{key}=\meta{value}}
\newcommand\key{\meta{key}}
\newcommand\val{\meta{value}}
\newcommand\set{\meta{set}}
\newcommand\enflong{\textcolor{black}{long}}
\newcommand\alllong{\textcolor{gray}{long}}
\newcommand\notlong{\textcolor{red!80!black}{long}}
\newcommand\prefixes[1]
  {^^A
    \hfill
    \ifcase\numexpr#1\relax\or
      \enflong\or
      \alllong\or
      \notlong\fi
  }
\hypersetup{linkcolor=red!80!black,urlcolor=purple!80!black}
\DoNotIndex{\def,\edef,\,,\=,\begingroup,\catcode,\chardef,\csname,\endcsname}
\DoNotIndex{\endgroup,\endinput,\errmessage,\expandafter,\input,\let,\long,\z@}
\DoNotIndex{\protected,\ProvidesFile,\ProvidesPackage,\relax,\space,\advance}
\DoNotIndex{\@,\unexpanded,\string,\expanded,\dimexpr,\global,\glueexpr,\hbox}
\DoNotIndex{\numexpr,\RequirePackage,\setbox,\the,\unless,\xdef,\gdef,\newcount}
\DoNotIndex{\number}
\DoNotIndex{\^,\@firstofone,\@firstoftwo,\@gobble,\@secondoftwo}
\DoNotIndex{\ifcsname}
\DoNotIndex{\ifx}
\DoNotIndex{\ifodd}
\DoNotIndex{\ifnum}
\DoNotIndex{\ifdefined}
\DoNotIndex{\iffalse}
\DoNotIndex{\iftrue}
\DoNotIndex{\else}
\DoNotIndex{\fi}
\@gobble\fi ^^A ignoring \ifx, \ifcsname, etc., but only one \fi
\@gobble\fi ^^A ignoring \ifx, \ifcsname, etc., but only one \fi
\@gobble\fi ^^A ignoring \ifx, \ifcsname, etc., but only one \fi
\@gobble\fi ^^A ignoring \ifx, \ifcsname, etc., but only one \fi
\@gobble\fi ^^A ignoring \ifx, \ifcsname, etc., but only one \fi
\@gobble\fi ^^A ignoring \ifx, \ifcsname, etc., but only one \fi
\@ifdefinable\gobbledocstriptag{\def\gobbledocstriptag#1>{}}
\makeatother
\begin{document}
  \title
    {^^A
      \texorpdfstring
        {^^A
          \huge\expkvc
          \\[\medskipamount]
          \Large define expandable \kv\ macros using \expkv
        }{expkv-cs - define expandable key=val macros using expkv}^^A
    }
  \date{\ekvcDate\space v\ekvcVersion}
  \author{Jonathan P. Spratte\thanks{\protect\randomize{jspratte@yahoo.de}}}
  \DocInput{expkv-cs.dtx}
\end{document}
%</driver>^^A=<<
% \fi
%
% \maketitle
% \renewcommand*\thefootnote{\arabic{footnote}}
%
% \begin{abstract}
% \noindent\parfillskip=0pt
% \expkvc\ provides two small interfaces to define expandable \kv\ macros using
% \expkv. It therefore lowers the entrance boundary to expandable \kv\ macros.
% The stylised name is \expkvc\ but the files use \file{expkv-cs}, this is due
% to CTAN-rules which don't allow \string| in package names since that is the
% pipe symbol in *nix shells.
% \end{abstract}
%
% \tableofcontents
%
% \begin{documentation}^^A>>=
%
% \section{Documentation}
%
% The \expkv\ package enables the new possibility of creating \kv\ macros which
% are fully expandable. The creation of such macros is however cumbersome for
% the average user. \expkvc\ tries to step in here. It provides interfaces to
% define \kv\ macros without worrying too much about the implementation. In case
% you're wondering now, the |cs| in \expkvc\ stands for control sequence,
% because |def| was already taken by \expkvd\ and ``control sequence'' is the
% term D.\,E.\,Knuth used in his \TeX book for named commands hence macros
% (though he also used the term ``macro''). So \expkvc\ defines control
% sequences for and with \expkv.
%
% There are two different approaches supported by this package. The first is
% splitting the keys up into individual arguments, the second is providing all
% the keys as a single argument to the underlying macro and getting an
% individual \val\ by using a hash. Well, actually there is no real hash, just
% some markers which are parsed, but this shouldn't be apparent to the user, the
% behaviour matches that of a hash-table.
%
% In addition to these two methods of defining a macro with primary keys a way
% to define secondary keys, which can reference the primary ones, is provided.
% These secondary keys don't correspond to an argument or an entry in the hash
% table directly but might come in handy for the average use case. Each macro
% has its own set of primary and secondary keys.
%
% A word of advice you should consider: If your macro doesn't have to be
% expandable (and often it doesn't) consider not using \expkvc. The interface
% has some overhead (though it still is fast -- check \autoref{sec:speed}) and
% the approach has its limits in versatility. If you don't need to be
% expandable, you should consider either defining your keys manually using
% \expkv\ or using \expkvd\ for convenience. Or you resort to another \kv\
% interface. Nevertheless setting up macros with \expkvc, especially with
% |\ekvcSplit|, is very convenient in my opinion, so if you just want to define
% a single macro with just a few keys this might be the way to go.
%
% \expkvc\ is usable as generic code and as a \LaTeX\ package. It'll
% automatically load \expkv\ in the same mode as well. To use it, just use one
% of
% \begin{lstlisting}
% \usepackage{expkv-cs} % LaTeX
% \input expkv-cs       % plainTeX
% \end{lstlisting}
%
%
% \subsection{Define Macros and Primary Keys}
%
% All macros defined with \expkvc\ have to be previously undefined or have the
% |\meaning| of |\relax|. This is necessary as there is no way to automatically
% undefine keys once they are set up (neither \expkv\ nor \expkvc\ keep track of
% defined keys) -- so to make sure there are no conflicts only new definitions
% are allowed (that's not the case for individual keys, only for frontend
% macros).
%
%
% \subsubsection{Primary Keys}\label{sec:primaries}
%
% In the following descriptions there will be one argument named \meta{primary
% keys}. This argument should be a \kv\ list where each \key\ will be one
% primary key and \val\ the associated initial value. By default all keys are
% defined short, but you can define long keys by prefixing \key\ with |long|
% (\emph{e.g.}, \texttt{long name=Jonathan P. Spratte}). You only need |long| if
% the key should be able to take a |\par| token. Note however that |long| keys
% are a microscopic grain faster (due to some internals of \expkvc). Only if at
% least one of the keys was |long| the \meta{cs} in the following defining
% macros will be |\long|. For obvious reasons there is no possibility to define
% a macro or key as |\protected|.
%
% At the moment \expkvc\ doesn't require any internal keys, but I can't foresee
% whether this will be the case in the future as well, as it might turn out that
% some features I deem useful can't be implemented without such internal keys.
% Because of this, please don't use key names starting with
% \texttt{EKVC\string|} as that should be the private name space.
%
%
% \subsubsection{Split}
%
% The split variants will provide the key values as separate arguments. This
% limits the number of keys for which this is truly useful.
%
% \begin{function}{\ekvcSplit}
%   \begin{syntax}
%     \cs{ekvcSplit}\meta{cs}\marg{primary keys}\marg{definition}
%   \end{syntax}
%   This defines \meta{cs} to be a macro taking one mandatory argument which
%   should contain a \kv\ list. The \meta{primary keys} will be defined for this
%   macro (see \autoref{sec:primaries}). The \meta{definition} is the code that
%   will be executed. You can access the \val\ of a \key\ by using a macro
%   parameter from |#1| to |#9|. The order of the macro parameters will be the
%   order provided in the \meta{primary keys} list (so |#1| is the \val\ of the
%   key defined first). With |\ekvcSplit| you can define macros using at most
%   nine primary keys.
% \end{function}
% \example The following defines a macro |\foo| that takes the keys |a| and |b|
% and outputs their values in a textual form:\\
% \begin{minipage}{.75\linewidth}
% \begin{lstlisting}
% \ekvcSplit\foo{a=a,b=b}{a is #1.\par b is #2.\par}
% \foo{}
% \foo{b=e}
% \end{lstlisting}
% \end{minipage}^^A
% \begin{exresult}[nobeforeafter,box align=center]{.25\linewidth}
%   \ekvcSplit\foo{a=a,b=b}{a is #1.\par b is #2.\par}
%   \foo{}
%   \foo{b=e}
% \end{exresult}
%
% \begin{function}{\ekvcSplitAndForward}
%   \begin{syntax}
%     \cs{ekvcSplitAndForward}\meta{cs}\marg{after}\marg{primary keys}
%   \end{syntax}
%   This defines \meta{cs} to be a macro taking one mandatory argument which
%   should contain a \kv\ list. You can use as many primary keys as you want
%   with this. The primary keys will be forwarded to \meta{after} as braced
%   arguments (as many as necessary for your primary keys). The order of the
%   braced arguments will be the order of your primary key definitions. In
%   \meta{after} you can use just a single control sequence, or some arbitrary
%   stuff which will be left in the input stream before your braced values (with
%   one set of braces stripped from \meta{after}), so both of the following
%   would be fine:
% \begin{lstlisting}
% \ekvcSplitAndForward\foo\foo@aux{keyA = A, keyB = B}
% \ekvcSplitAndForward\foo{\foo@aux{more args}}{keyA = A, keyB = B}
% \end{lstlisting}
% \end{function}
% In the first case |\foo@aux| should take at least two arguments (|keyA| and
% |keyB|), in the second case at least three (|more args|, |keyA|, and |keyB|).
%
% \begin{function}{\ekvcSplitAndUse}
%   \begin{syntax}
%     \cs{ekvcSplitAndUse}\meta{cs}\marg{primary keys}
%   \end{syntax}
%   This will roughly do the same as |\ekvcSplitAndForward|, but instead of
%   specifying what will be used after splitting the keys, \meta{cs} will use
%   what follows the \kv\ list. So its syntax will be
% \end{function}
%   \begin{syntax}
%     \meta{cs}\{\key=\val, \ldots\}\marg{after}
%   \end{syntax}
% and the code in |after| should expect at least as many arguments as the number
% of keys defined for \meta{cs}.
%
% \subsubsection{Hash}
%
% The hash variants will provide the key values as a single argument in which
% you can access specific values using a special macro. The implementation might
% be more convenient and scale better, \emph{but} it is much slower (for a
% primitive macro with a single key benchmarking was almost $1.7$ times slower,
% the root of which being the key access with |\ekvcValue|, not the parsing, and
% for a key access using |\ekvcValueFast| it was still about $1.2$ times
% slower). So if your macro uses less than ten primary keys, you should consider
% using the split approach.
%
% \begin{function}{\ekvcHash}
%   \begin{syntax}
%     \cs{ekvcHash}\meta{cs}\marg{primary keys}\marg{definition}
%   \end{syntax}
%   This defines \meta{cs} to be a macro taking one mandatory argument which
%   should contain a \kv\ list. You can use as many primary keys as you want.
%   The primary keys will be forwarded as a single argument containing every key
%   to the underlying macro. The underlying macro is defined as
%   \meta{definition}, in which you can access the \val\ of a \key\ by using
%   \texttt{\cs[no-index]{ekvcValue}\{\key\}\{\#1\}}.
% \end{function}
% \example This defines an equivalent macro like the |\foo| defined with
% |\ekvcSplit| earlier:\\
% \begin{minipage}{.75\linewidth}
% \begin{lstlisting}
% \ekvcHash\foo{a=a,b=b}{a is \ekvcValue{a}{#1}.\par
%                        b is \ekvcValue{b}{#1}.\par}
% \foo{}
% \foo{b=e}
% \end{lstlisting}
% \end{minipage}^^A
% \begin{exresult}[nobeforeafter,box align=center]{.25\linewidth}
%   \ekvcHash\foo{a=a,b=b}{a is \ekvcValue{a}{#1}.\par
%                          b is \ekvcValue{b}{#1}.\par}
%   \foo{}
%   \foo{b=e}
% \end{exresult}
%
% \begin{function}{\ekvcHashAndForward}
%   \begin{syntax}
%     \cs{ekvcHashAndForward}\meta{cs}\marg{after}\marg{primary keys}
%   \end{syntax}
%   This defines \meta{cs} to be a macro taking one mandatory argument which
%   should contain a \kv\ list. You can use as many primary keys as you want.
%   The primary keys will be forwarded as a single argument containing every key
%   to \meta{after}. You can use a single macro for \meta{after} or use some
%   arbitrary stuff, which will be left in the input stream before the hashed
%   \kv\ list with one set of braces stripped. In the
%   macro called in \meta{after} you can access the \val\ of a \key\ by using
%   \texttt{\cs[no-index]{ekvcValue}\{\key\}\{\#1\}} (or whichever argument the
%   hashed \kv\ list will be).
% \end{function}
%
% \begin{function}{\ekvcHashAndUse}
%   \begin{syntax}
%     \cs{ekvcHashAndUse}\meta{cs}\marg{primary keys}
%   \end{syntax}
%   This will roughly do the same as |\ekvcHashAndForward|, but instead of
%   specifying what will be used after hashing the keys, \meta{cs} will use what
%   follows the \kv\ list. So its syntax will be
% \end{function}
%   \begin{syntax}
%     \meta{cs}\{\key=\val, \ldots\}\marg{after}
%   \end{syntax}
%
% \begin{function}{\ekvcValue}
%   \begin{syntax}
%     \cs{ekvcValue}\{\key\}\marg{key list}
%   \end{syntax}
%   This is a safe way to access your keys in a hash variant. \key\
%   is the key which's \val\ you want to use out of the \meta{key list}.
%   \meta{key list} should be the key list argument forwarded to your underlying
%   macro by |\ekvcHash|, |\ekvcHashAndForward|, or |\ekvcHashAndUse|. It will
%   be tested whether the hash function to access that \key\ exists, the \key\
%   argument is not empty, and that the \meta{key list} really contains a \val\
%   of that \key. This macro needs exactly two steps of expansion and if used
%   inside of an |\edef| or |\expanded| context will protect the \val\ from
%   further expanding.
% \end{function}
%
% \begin{function}{\ekvcValueFast}
%   \begin{syntax}
%     \cs{ekvcValueFast}\{\key\}\marg{key list}
%   \end{syntax}
%   This behaves just like |\ekvcValue|, but \emph{without any} safety tests. As
%   a result this is about $1.4$ times faster \emph{but} will throw low level
%   \TeX\ errors eventually if the hash function isn't defined or the \key\
%   isn't part of the \meta{key list} (\emph{e.g.}, because it was defined as a
%   key for another macro -- all macros share the same hash function per \key).
%   Use it if you know what you're doing. This macro needs exactly three steps
%   of expansion in the no-errors case.
% \end{function}
%
% \begin{function}{\ekvcValueSplit}
%   \begin{syntax}
%     \cs{ekvcValueSplit}\{\key\}\marg{key list}\marg{next}
%   \end{syntax}
%   If you need a specific \key\ from a \meta{key list} more than once, it'll
%   be a good idea to only extract it once and from then on keep it as a
%   separate argument. Hence the macro \cs{ekvcValueSplit} will extract one
%   specific \key's value from the list and forward the remainder of the list
%   as the first and the \key's value as the second argument to \meta{next}, so
%   the result of this will be \meta{next}\marg{key list'}\marg{value} with
%   \meta{key list'} the remaining list. This is almost as fast as |\ekvcValue|
%   and runs the same tests. Keep in mind that you can't fetch for the same
%   \key\ again from \meta{key list'} as it got removed.
% \end{function}
% \example The following defines a macro |\foo| which will take three keys.
% Since the next parsing step will need the value of one of the keys multiple
% times we split that key off the list:\\
% \begin{minipage}{.75\linewidth}
% \begin{lstlisting}
% \ekvcHash\foo{a=a,b=b,c=c}
%   {\ekvcValueSplit{a}{#1}\foobar}
% \newcommand*\foobar[2]{a is #2.\par
%                        b is \ekvcValue{b}{#1}.\par
%                        c is \ekvcValue{c}{#1}.\par}
% \foo{}
% \end{lstlisting}
% \end{minipage}^^A
% \begin{exresult}[nobeforeafter,box align=center]{.25\linewidth}
%   \ekvcHash\foo{a=a,b=b,c=c}
%     {\ekvcValueSplit{a}{#1}\foobar}
%   \newcommand*\foobar[2]{a is #2.\par
%                          b is \ekvcValue{b}{#1}.\par
%                          c is \ekvcValue{c}{#1}.\par}
%   \foo{}
% \end{exresult}
%
% \begin{function}{\ekvcValueSplitFast}
%   \begin{syntax}
%     \cs{ekvcValueSplitFast}\{\key\}\marg{key list}\marg{next}
%   \end{syntax}
%   This behaves just like |\ekvcValueSplit|, but it won't run the same tests,
%   hence it is faster but more error prone, just like the relation between
%   |\ekvcValue| and |\ekvcValueFast|.
% \end{function}
%
%
% \subsection{Secondary Keys}\label{sec:secondaries}
%
% To remove some of the limitations with the approach that each primary key
% matches an argument or hash entry, you can define secondary keys. Those have
% to be defined for each macro but it doesn't matter whether that macro was a
% split or a hash variant. If a secondary key references another key it doesn't
% matter whether that other key is primary or secondary.
%
% Secondary keys can have a prefix (like |long|) which are called |p|-type
% prefix and must have a type (like |meta|) which are called |t|-type prefix.
% Some types might require some |p|-prefixes, while others might forbid those.
%
% Please keep in mind that key names shouldn't start with \texttt{EKVC\string|}.
%
% \begin{function}{\ekvcSecondaryKeys}
%   \begin{syntax}
%     \cs{ekvcSecondaryKeys}\meta{cs}\{\kv, \ldots\}
%   \end{syntax}
%   This is the front facing macro to define secondary keys. For the macro
%   \meta{cs} define \key\ to have definition \val. The general syntax for \key\
%   should be
%   \begin{quote}\ttfamily\small
%     \meta{prefix} \meta{name}
%   \end{quote}
%   Where \meta{prefix} is a space separated list of optional |p|-type prefixes
%   followed by one |t|-type prefix. The syntax of \val\ is dependent on the
%   used |t|-prefix.
% \end{function}
%
%
% \subsubsection{\texttt{p}-type Prefixes}
%
% There is only one |p|-prefix available, which is |long|.
%
% \begin{function}{long}
%   The following key will be defined |\long|.
% \end{function}
%
%
% \subsubsection{\texttt{t}-type Prefixes}
%
% If you're familiar with \expkvd\ you'll notice that the |t|-type prefixes
% provided here are much fewer. The expansion only concept doesn't allow for
% great variety in the auto-defined keys.
%
% The syntax examples of the |t|-prefixes will show which |p|-prefix will be
% automatically used by printing those black (\texttt{\enflong}), which
% will be available in grey (\texttt{\alllong}), and which will be disallowed in
% red (\texttt{\notlong}). This will be put flush right next to the syntax line.
%
% \begin{function}{meta}
%   \begin{syntax}
%     meta \key\ = \{\kv, \ldots\} \prefixes2
%   \end{syntax}
%   With a |meta| key you can set other keys. Whenever \key\ is used the keys in
%   the \kv\ list will be set to the values given there. You can use the \val\
%   given to \key\ by using |#1| in the \kv\ list. The keys in the \kv\ list can
%   be primary and secondary ones.
% \end{function}
%
% \begin{function}{nmeta}
%   \begin{syntax}
%     nmeta \key\ = \{\kv, \ldots\} \prefixes3
%   \end{syntax}
%   An |nmeta| key is like a |meta| key, but it doesn't take a value, so the
%   \kv\ list is static.
% \end{function}
%
% \begin{function}{alias}
%   \begin{syntax}
%     alias \key\ = \meta{key_2} \prefixes3
%   \end{syntax}
%   This assigns the definition of \meta{key_2} to \key. As a result \key\ is an
%   alias for \meta{key_2} behaving just the same. Both the value taking and the
%   |NoVal| version (that's \expkv\ slang for a key not accepting a value) will
%   be copied if they are defined when |alias| is used. Of course, \meta{key_2}
%   has to be defined, be it as a primary or secondary one.
% \end{function}
%
% \begin{function}{default}
%   \begin{syntax}
%     default \key\ = \marg{default} \prefixes3
%   \end{syntax}
%   If \key\ is a defined value taking key, you can define a |NoVal| version
%   with this that will behave as if \key\ was given \meta{default} as its
%   \val. Note that this doesn't change the initial values of primary keys set
%   at definition time in |\ekvcSplit| and friends. \key\ can be a primary or
%   secondary key.
% \end{function}
%
% \begin{function}{flag-bool}
%   \begin{syntax}
%     flag-bool \key\ = \meta{cs} \prefixes3
%   \end{syntax}
%   This is a secondary key that doesn't directly involve any of the primary or
%   secondary keys.
%   This defines \key\ to take a value, which should be either |true| or
%   |false|, and set the flag called \meta{cs} accordingly as a boolean.
%   If \meta{cs} isn't defined yet it will be initialised as a flag.
%   Please also read \autoref{sec:flags}.
% \end{function}
%
% \begin{function}{flag-true,flag-false}
%   \begin{syntax}
%     flag-true \key\ = \meta{cs} \prefixes3
%   \end{syntax}
%   This is a secondary key that doesn't directly involve any of the primary or
%   secondary keys.
%   This defines \key\ to take no value and set the flag called \meta{cs} to
%   |true| or |false|, respectively.
%   If \meta{cs} isn't defined yet it will be initialised as a flag.
%   Please also read \autoref{sec:flags}.
% \end{function}
%
% \begin{function}{flag-raise}
%   \begin{syntax}
%     flag-raise \key\ = \meta{cs} \prefixes3
%   \end{syntax}
%   This is a secondary key that doesn't directly involve any of the primary or
%   secondary keys.
%   This defines \key\ to take no value and raise the flag called \meta{cs}.
%   If \meta{cs} isn't defined yet it will be initialised as a flag.
%   Please also read \autoref{sec:flags}.
% \end{function}
%
%
% \subsection{Changing the Initial Values}\label{sec:change}
%
% \begin{function}{\ekvcChange}
%   \begin{syntax}
%     \cs{ekvcChange}\meta{cs}\{\kv,\ldots\}
%   \end{syntax}
%   This processes the \kv\ list (just like it would be processed by \meta{cs})
%   to set new defaults for the macro \meta{cs} defined with \expkvc\ (meaning
%   the values used if you don't provide anything at use time). Inside the \kv\
%   list both primary and secondary keys can be used. If \meta{cs} was defined
%   |\long| earlier it will still be |\long|, every other \TeX\ prefix will be
%   stripped (but \expkvc\ doesn't support them anywhere else so that should be
%   fine). The resulting new defaults will be stored inside the \meta{cs}
%   locally (just as the original defaults were).
% \end{function}
%
% Consider the following example:\\
% \begin{minipage}[c]{.75\linewidth}
% \begin{lstlisting}
% \ekvcSplit\foo{a=a,b=b}{a is #1.\par b is #2.\par}
% \foo{} % initial defaults
% \ekvcChange\foo{b=B}
% \foo{} % new defaults
% \ekvcSecondaryKeys\foo{meta c={a={#1},b={#1}}}
% \ekvcChange\foo{c=c}
% \foo{} % newer defaults
% \end{lstlisting}
% \end{minipage}^^A
% \begin{exresult}[nobeforeafter,box align=center]{.25\linewidth}
%   \ekvcSplit\foo{a=a,b=b}{a is #1.\par b is #2.\par}
%   \foo{}
%   \ekvcChange\foo{b=B}
%   \foo{}
%   \ekvcSecondaryKeys\foo{meta c={a={#1},b={#1}}}
%   \ekvcChange\foo{c=c}
%   \foo{}
% \end{exresult}
%
% As a result with this the typical setup macro could be implemented:
% \begin{lstlisting}
% \ekvcHashAndUse\foo{key=a,key=b}
% \newcommand\foosetup{\ekvcChange\foo}
% \end{lstlisting}
% Of course the usage is limited to a single macro |\foo|, hence this might be
% not as powerful as similar macros used with other \kv\ interfaces. But at
% least a few similar macros could be grouped using the same key parsing macro
% internally.
%
%
% \subsection{Flags}\label{sec:flags}
%
% The idea of flags is taken from \pkg{expl3}. They provide a way to store
% numerical information expandably, however only incrementing and accessing
% works expandably, decrementing is unexpandable. A flag has a height, which
% is a numerical value, and which can be raised by $1$. Flags come at a high
% computational cost (accessing them is slow and they require more memory than
% normal \TeX\ data types like registers, both getting linearly worse with the
% height), so don't use them if not necessary.
%
% The state of flags is always changed locally to the current group, but not to
% the current macro, so if you're using one of the |t|-types involving flags
% bear in mind that they can affect other macros using the same flags at the
% current group level!
%
% \expkvc\ provides some macros to access, alter, and use flags. Flags of
% \expkvc\ don't share a name space with the flags of \pkg{expl3}.
%
% \begin{function}{\ekvcFlagNew}
%   \begin{syntax}
%     \cs{ekvcFlagNew}\meta{flag}
%   \end{syntax}
%   This initialises the macro \meta{flag} as a new flag. It isn't checked
%   whether the macro \meta{flag} is currently undefined.  A \meta{flag} will
%   expand to the flag's current height with a trailing space (so you can use it
%   directly with |\ifnum| for example and it will terminate the number scanning
%   itself).
% \end{function}
%
%  All other macros dealing with flags take as a parameter a macro defined as a
%  \meta{flag} with |\ekvcFlagNew|.
%
% \begin{function}{\ekvcFlagHeight}
%   \begin{syntax}
%     \cs{ekvcFlagHeight}\meta{flag}
%   \end{syntax}
%   This expands to the current height of \meta{flag} in a single step of
%   expansion (without a trailing space).
% \end{function}
%
% \begin{function}{\ekvcFlagRaise}
%   \begin{syntax}
%     \cs{ekvcFlagRaise}\meta{flag}
%   \end{syntax}
%   This expandably raises the height of \meta{flag} by $1$.
% \end{function}
%
% \begin{function}{\ekvcFlagSetTrue,\ekvcFlagSetFalse}
%   \begin{syntax}
%     \cs{ekvcFlagSetTrue}\meta{flag}
%   \end{syntax}
%   By interpreting an even value as false and an odd value as true we can use a
%   flag as a boolean. This expandably sets \meta{flag} to true or false,
%   respectively, by raising it if necessary.
% \end{function}
%
% \begin{function}{\ekvcFlagIf}
%   \begin{syntax}
%     \cs{ekvcFlagIf}\meta{flag}\marg{true}\marg{false}
%   \end{syntax}
%   This interprets a \meta{flag} as a boolean and expands to either \meta{true}
%   or \meta{false}.
% \end{function}
%
% \begin{function}{\ekvcFlagIfRaised}
%   \begin{syntax}
%     \cs{ekvcFlagIfRaised}\meta{flag}\marg{true}\marg{false}
%   \end{syntax}
%   This tests whether the \meta{flag} is raised, meaning it has a height
%   greater than zero, and if so expands to \meta{true} else to \meta{false}.
% \end{function}
%
% \begin{function}{\ekvcFlagReset}
%   \begin{syntax}
%     \cs{ekvcFlagReset}\meta{flag}
%   \end{syntax}
%   This resets a flag (so restores its height to $0$). This operation is
%   \emph{not} expandable and done locally. If you really intend to use flags
%   you can reset them every now and then to keep the performance hit low.
% \end{function}
%
% \begin{function}{\ekvcFlagGetHeight}
%   \begin{syntax}
%     \cs{ekvcFlagGetHeight}\meta{flag}\marg{next}
%   \end{syntax}
%   This retrieves the current height of the \meta{flag} and provides it as a
%   braced argument to \meta{next}, leaving \meta{next}\marg{height} in the
%   input stream.
% \end{function}
%
% \begin{function}{\ekvcFlagGetHeights}
%   \begin{syntax}
%     \cs{ekvcFlagGetHeights}\marg{flag-list}\marg{next}
%   \end{syntax}
%   This retrieves the current height of each \meta{flag} in the
%   \meta{flag-list} and provides them as a single argument to \meta{next}.
%   Inside that argument each height is enclosed in a set of braces
%   individually. The \meta{flag-list} is just a single argument containing the
%   \meta{flag}s.
%   So a usage like
%   \texttt
%     {^^A
%       \cs[no-index]{ekvcFlagGetHeights}\hskip0pt^^A
%       \{\cs[no-index]{myflagA}\cs[no-index]{myflagB}\}\hskip0pt^^A
%       \{\cs[no-index]{stuff}\}^^A
%     }
%   will expand to
%   \texttt{\cs[no-index]{stuff}\{\marg{height-A}\marg{height-B}\}}.
% \end{function}
%
%
% \subsection{Further Examples}
%
% How could a documentation be a good documentation without some basic examples?
% Say we want to define a small macro expanding to some character description
% (who knows why this has to be expandable?). A character description will not
% have too many items to it, so we use |\ekvcSplit| (the comments with the
% parameter numbers are of course not necessary and just ease reading the
% example).
% \begin{lstlisting}
% \ekvcSplit\character
%   {
%     name=John Doe,            % #1
%     age=any,                  % #2
%     nationality=the Universe, % #3
%     hobby=to exist,           % #4
%     type=Mister,              % #5
%     pronoun=He,               % #6
%     possessive=his,           % #7
%   }
%   {#1 is a #5 from #3. #6 is of #2 age and #7 hobby is #4.\par}
% \end{lstlisting}
% Also we want to give some short cuts so that it's easier to describe several
% persons.
% \begin{lstlisting}
% \ekvcSecondaryKeys\character
%   {
%     alias pro = pronoun,
%     alias pos = possessive,
%     nmeta me =
%       {
%         name=Jonathan,
%         age=a young,
%         nationality=Germany,
%         hobby=\TeX\ coding,
%       },
%     meta lady =
%       {type=Lady, pronoun=She, possessive=her, name=Jane Doe, #1},
%     nmeta paulo =
%       {
%         name=Paulo,
%         type=duck,
%         age=a young,
%         nationality=Brazil,
%         hobby=to quack,
%       }
%   }
% \end{lstlisting}
% Now we can describe people using
% \begin{lstlisting}
% \character{}
% \character{me}
% \character{paulo}
% \character
%   {lady={name=Evelyn,nationality=Ireland,age=the best,hobby=reading}}
% \character
%   {
%     name=Our sun, type=star, nationality=our solar system, pro=It,
%     age=an old, pos=its, hobby=shining
%   }
% \end{lstlisting}
% As one might see, the |lady| key could actually have been an |nmeta| key as
% well, as all that is done with the argument is using it as a \kv\ list.
%
% The result of only the first two usages would be:
% \begin{exresult}{\linewidth}
%   \ekvcSplit\character
%     {
%       name=John Doe,
%       age=any,
%       nationality=the Universe,
%       hobby=to exist,
%       type=Mister,
%       pronoun=He,
%       possessive=his,
%     }
%     {#1 is a #5 from #3. #6 is of #2 age and #7 hobby is #4.\par}
%   \ekvcSecondaryKeys\character
%     {
%       alias pro = pronoun,
%       alias pos = possessive,
%       nmeta me =
%         {
%           name=Jonathan,
%           age=a young,
%           nationality=Germany,
%           hobby=\TeX\ coding,
%         },
%       meta lady =
%         {type=Lady, pronoun=She, possessive=her, name=Jane Doe, #1},
%       nmeta paulo =
%         {
%           name=Paulo,
%           type=duck,
%           age=a young,
%           nationality=Brazil,
%           hobby=to quack,
%         }
%     }
%   \character{}
%   \character{me}
% \end{exresult}
%
% Using \pkg{xparse} or \expkv's \cs{ekvoptarg} or \cs{ekvoptargTF} and
% forwarding arguments one can easily define \kv\ macros with actual optional
% and mandatory arguments as well. A small nonsense example (which should
% perhaps use |\ekvcSplitAndForward| instead of |\ekvcHashAndForward| since it
% only uses four keys and one other argument -- and isn't expandable since it
% uses a \env{tabular} environment, so it would've been better to use a more
% feature rich \kv\ interface most likely, \emph{e.g.}, the one provided by
% \expkvd):
% \begin{lstlisting}
% \makeatletter
% \newcommand*\nonsense{\ekvoptarg\nonsense@a{}}
% \ekvcHashAndForward\nonsense@a\nonsense@b
%   {
%     keyA = A,
%     keyB = B,
%     keyC = c,
%     keyD = d,
%   }
% \newcommand*\nonsense@b[2]
%   {%
%     \begin{tabular}{lll}
%       key & A & \ekvcValue{keyA}{#1} \\
%           & B & \ekvcValue{keyB}{#1} \\
%           & C & \ekvcValue{keyC}{#1} \\
%           & D & \ekvcValue{keyD}{#1} \\
%       \multicolumn{2}{l}{mandatory} & #2 \\
%     \end{tabular}%
%   }
% \makeatother
% \end{lstlisting}
% And then we would be able to do some nonsense
% \begin{lstlisting}
% \nonsense{}
% \nonsense[keyA=hihi]{haha}
% \nonsense[keyA=hihi, keyB=A]{hehe}
% \nonsense[keyC=huhu, keyA=hihi, keyB=A]{haha}
% \end{lstlisting}
% resulting in
% \begin{exresult}{\linewidth}
%   \makeatletter
%   \newcommand*\nonsense{\ekvoptarg\nonsense@a{}}
%   \ekvcHashAndForward\nonsense@a\nonsense@b
%     {
%       keyA = A,
%       keyB = B,
%       keyC = c,
%       keyD = d,
%     }
%   \newcommand*\nonsense@b[2]
%     {%
%       \begin{tabular}{lll}
%         key & A & \ekvcValue{keyA}{#1} \\
%             & B & \ekvcValue{keyB}{#1} \\
%             & C & \ekvcValue{keyC}{#1} \\
%             & D & \ekvcValue{keyD}{#1} \\
%         \multicolumn{2}{l}{mandatory} & #2 \\
%       \end{tabular}%
%     }
%   \makeatother
%   \nonsense{}
%   \nonsense[keyA=hihi]{haha}
%   \nonsense[keyA=hihi, keyB=A]{hehe}
%   \nonsense[keyC=huhu, keyA=hihi, keyB=A]{haha}
% \end{exresult}
%
%
% \subsection{Freedom for Keys!}
%
% If this was the \TeX book this subsection would have a double bend sign. Not
% because it is overly complicated, but because it shows things which could
% break \expkvc's expandability and its alignment safety. This is for
% experienced users wanting to get the most flexibility and knowing what they
% are doing.
%
% In case you're wondering, it is possible to define other keys than the
% primaries and the secondary types listed in \autoref{sec:secondaries} for a
% macro defined with \expkvc\ by using the low-level interface of \expkv\ or
% even the interface provided by \expkvd. The set name used for \expkvc's keys
% is the macro name, including the leading backslash, or more precisely
% |\string|\meta{cs} is used. This can be exploited to define additional keys
% with arbitrary code. Consider the following \emph{bad} example:
% \begin{lstlisting}
% \ekvcSplit\foo{a=A,b=B}{a is #1.\par b is #2}
% \protected\ekvdef{\string\foo}{c}{\def\fooC{#1}}
% \end{lstlisting}
% This would define a key named |c| that will store its value inside a macro.
% The issue with this is that this can't be done expandably. As a result, the
% macro |\foo| isn't always expandable any more (not that bad if this was never
% required; killjoy if it was) and as soon as the key |c| is used, it is also no
% longer alignment safe\footnote{This means that the \kv-list can't contain
%   alignment markers that are not inside an additional set of braces if used
%   inside a \TeX\ alignment.}
% (might be bad depending on the usage).
%
% So why do I show you this? Because we could as well do something useful like
% create a key that pre-parses the input and after that passes the parsed value
% on. This parsing would have to be completely expandable though. For the
% pass-on part we can use the following function:
%
% \begin{function}{\ekvcPass}
%   \begin{syntax}
%     \cs{ekvcPass}\meta{cs}\marg{key}\marg{value}
%   \end{syntax}
%   This passes \meta{value} on to \meta{key} for the \expkvc-macro \meta{cs}.
%   It should be used inside the key parsing of a macro defined with \expkvc,
%   else this most likely results in a low level \TeX\ error.
% \end{function}
%
% With this we could for example split the value of a key at a hyphen and pass
% the parts to different keys:\\
% \begin{minipage}[c]{.75\linewidth}
% \begin{lstlisting}
% \ekvcSplit\foo{a=A,b=B}{a is #1.\par b is #2.\par}
% \ekvdef{\string\foo}{c}{\fooSplit#1\par}
% \def\fooSplit#1-#2\par
%   {\ekvcPass\foo{a}{#1}\ekvcPass\foo{b}{#2}}
% \foo{}
% \foo{c=1-2}
% \end{lstlisting}
% \end{minipage}
% \begin{exresult}[nobeforeafter,box align=center]{.25\linewidth}
%   \ekvcSplit\foo{a=A,b=B}{a is #1.\par b is #2.\par}
%   \ekvdef{\string\foo}{c}{\fooSplit#1\par}
%   \def\fooSplit#1-#2\par{\ekvcPass\foo{a}{#1}\ekvcPass\foo{b}{#2}}
%   \foo{}
%   \foo{c=1-2}
% \end{exresult}
%
%
% \subsection{Speed Considerations}\label{sec:speed}
%
% As already mentioned in the introduction there are some speed considerations
% implied if you choose to define macros via \expkvc. However the overhead isn't
% the factor which should hinder you to use \expkvc\ if you found a reasonable
% use case. The key-parsing is still faster than with most other \kv\ packages
% (see the ``Comparisons'' subsection in the \expkv\ documentation).
%
% The speed considerations in this subsection use the first example in this
% documentation as the benchmark. So we have seven keys and a short sentence
% which should be typeset. For comparisons I use the following equivalent
% \expkvd\ definitions. Each result is the average between changing no keys from
% their initial values and altering four. Furthermore I'll compare three
% variants of \expkvc\ with the \expkvd\ definitions, namely the split example
% from above, a hash variant using |\ekvcValue| and a hash variant using
% |\ekvcValueFast|.
% \begin{lstlisting}
% \usepackage{expkv-def}
% \ekvdefinekeys{keys}
%   {%
%     ,store   name         = \KEYSname
%     ,initial name         = John Doe
%     ,store   age          = \KEYSage
%     ,initial age          = any
%     ,store   nationality  = \KEYSnationality
%     ,initial nationality  = the Universe
%     ,store   hobby        = \KEYShobby
%     ,initial hobby        = to exist
%     ,store   type         = \KEYStype
%     ,initial type         = Mister
%     ,store   pronoun      = \KEYSpronoun
%     ,initial pronoun      = He
%     ,store   possessive   = \KEYSpossessive
%     ,initial possessive   = his
%   }
% \newcommand*\KEYS[1]
%   {%
%     \begingroup
%       \ekvset{keys}{#1}%
%       \KEYSname\ is a \KEYStype\ from \KEYSnationality. \KEYSpronoun\ is
%       of \KEYSage\ age and \KEYSpossessive\ hobby is \KEYShobby.%
%     \endgroup
%   }
% \end{lstlisting}
%
% The first comparison removes the typesetting part from all the definitions, so
% that only the key parsing is compared. In this comparison the |\ekvcValue|
% and |\ekvcValueFast| variants will not differ, as they are exactly the same
% until the key usage. We find that the split approach is $1.4$ times slower
% than the \expkvd\ setup and the hash variants end up in the middle at $1.17$
% times slower.
%
% Next we put the typesetting part back in. Every call of the macros will
% typeset the sentences into a box register in horizontal mode. With the
% typesetting part (which includes the accessing of values) the fastest remains
% the \expkvd\ definitions, but split is close at $1.16$ times slower, followed
% by the hash variant with fast accesses at $1.36$ times slower, and the safe
% hash access variant ranks in the slowest $1.8$ times slower than \expkvd.
%
% Just in case you're wondering now, a simple macro taking seven arguments is
% $30$ to $40$ times faster than any of those in the argument grabbing and \kv\
% parsing part and only $1.5$ to $2.8$ times faster if the typesetting part is
% factored in. So the real choke isn't the parsing.
%
% So to summarize this, if you have a reasonable use case for expandable \kv\
% parsing macros you should go on and define them using \expkvc. If you have a
% reasonable use case for \kv\ parsing macros but defining them expandable isn't
% necessary for your use you should take advantage of the greater flexibility of
% non-expandable \kv\ setups (but if you're after maximum speed there aren't
% that many \kv\ parsers beating \expkvc). And if you are after maximum
% performance maybe ditching the \kv\ interface altogether is a good idea, but
% depending on the number of arguments your interface might get convoluted.
%
%
% \subsection{Useless Macros}
%
% Perhaps these macros aren't completely useless, but I figured from a user's
% point of view I wouldn't know what I should do with these.
%
% \begin{function}{\ekvcDate,\ekvcVersion}
%   These two macros store the version and the date of the package/generic code.
% \end{function}
%
%
% \subsection{Bugs}
%
% Of course I don't think there are any bugs (who would knowingly distribute
% buggy software as long as he isn't a multi-million dollar corporation?). But
% if you find some please let me know. For this one might find my email address
% on the first page or file an issue on Github:
% \url{https://github.com/Skillmon/tex_expkv-cs}
%
%
% \subsection{License}
%
% Copyright \textcopyright\ 2020\unless\ifnum\year=2020--\the\year\fi\
% Jonathan P. Spratte
%
% \medskip\noindent
% This work may be distributed and/or modified under the conditions of the
% \LaTeX\ Project Public License (LPPL), either version 1.3c of this license or
% (at your option) any later version. The latest version of this license is in
% the file:
%
% \url{http://www.latex-project.org/lppl.txt}
%
% \noindent
% This work is ``maintained'' (as per LPPL maintenance status) by
%
% \mbox{Jonathan P. Spratte}.
%
% \end{documentation}^^A=<<
%
% \begin{implementation}^^A>>=
%
% \clearpage
%
% \section{Implementation}^^A>>=
%^^A the LaTeX package >>=
% \subsection{The \LaTeX\ Package}
% Just like for \expkv\ we provide a small \LaTeX\ package that sets up things
% such that we behave nicely on \LaTeX\ packages and files system. It'll
% |\input| the generic code which implements the functionality.
% \gobbledocstriptag
%<*pkg>
%    \begin{macrocode}
\RequirePackage{expkv}
\def\ekvc@tmp
  {%
    \ProvidesFile{expkv-cs.tex}%
      [%
        \ekvcDate\space v\ekvcVersion\space
        define expandable key=val macros using expkv%
      ]%
  }
% \iffalse meta-comment
%
% File: expkv-cs.dtx Copyright (C) 2020-2021 Jonathan P. Spratte
%
% This work  may be  distributed and/or  modified under  the conditions  of the
% LaTeX Project Public License (LPPL),  either version 1.3c  of this license or
% (at your option) any later version.  The latest version of this license is in
% the file:
%
%   http://www.latex-project.org/lppl.txt
%
% ------------------------------------------------------------------------------
%
%<*driver>^^A>>=
\def\nameofplainTeX{plain}
\ifx\fmtname\nameofplainTeX\else
  \expandafter\begingroup
\fi
\input l3docstrip.tex
\askforoverwritefalse
\preamble

--------------------------------------------------------------
expkv-cs -- define expandable key=val macros using expkv
E-mail: jspratte@yahoo.de
Released under the LaTeX Project Public License v1.3c or later
See http://www.latex-project.org/lppl.txt
--------------------------------------------------------------

Copyright (C) 2020-2021 Jonathan P. Spratte

This  work may be  distributed and/or  modified under  the conditions  of the
LaTeX Project Public License (LPPL),  either version 1.3c  of this license or
(at your option) any later version.  The latest version of this license is in
the file:

  http://www.latex-project.org/lppl.txt

This work is "maintained" (as per LPPL maintenance status) by
  Jonathan P. Spratte.

This work consists of the file  expkv-cs.dtx
and the derived files           expkv-cs.pdf
                                expkv-cs.sty
                                expkv-cs.tex

\endpreamble
% stop docstrip adding \endinput
\postamble
\endpostamble
\generate{\file{expkv-cs.sty}{\from{expkv-cs.dtx}{pkg}}}
\generate{\file{expkv-cs.tex}{\from{expkv-cs.dtx}{tex}}}
\ifx\fmtname\nameofplainTeX
  \expandafter\endbatchfile
\else
  \expandafter\endgroup
\fi
%
\IfFileExists{expkv-cs.tex}{\input{expkv-cs.tex}}{}
\ProvidesFile{expkv-cs.dtx}
  [\csname ekvcDate\endcsname\ define expandable key=val macros using expkv]
\PassOptionsToPackage{full}{textcomp}
\documentclass{l3doc}
\RequirePackage[oldstylenums,nott]{kpfonts}
\input{glyphtounicode}
\pdfgentounicode=1
\RequirePackage{tcolorbox}
\newtcolorbox{exresult}[2][]
  {colback=expkvgrey!10!white,colframe=expkvgrey,fontupper=\small,width={#2},#1}
\newtcbox\exres[1][]
  {
     colback=expkvgrey!10!white
    ,colframe=expkvgrey
    ,size=small
    ,nobeforeafter
    ,tcbox raise base
    ,fontupper=\small
    ,#1
  }
\RequirePackage{listings}
\RequirePackage{booktabs}
\RequirePackage{array}
\RequirePackage{collcell}
\RequirePackage{siunitx}
\RequirePackage{xcolor}
\RequirePackage{caption}
\RequirePackage{microtype}
\RequirePackage{accsupp}
\lstdefinelanguage{expkvc}[primitive]{TeX}
  {
    ,moretexcs=[4]^^A e-TeX
      {
        expanded,
        numexpr,
        protected,
      }
    ,moretexcs=[5]^^A plain/LaTeX
      {
        approx,
        begin,
        empty,
        item,
        LaTeX,
        makeatletter,makeatother,multicolumn,
        newcommand,newdimen,
        RequirePackage,
        rule,
        TeX,
        textit,texttt,
        usepackage,
      }
    ,moretexcs=[6]^^A used packages
      {
        ^^A xfp
        fpeval,
        ^^A keyval
        setkeys,
        ^^A yax
        defactiveparameter,storevalue,setparameterlist
      }
    ,moretexcs=[2]^^A expkv-cs macros
      {
        ekvcSplit,ekvcSplitAndForward,ekvcSplitAndUse,
        ekvcHash,ekvcHashAndForward,ekvcHashAndUse,
        ekvcValue,ekvcValueFast,ekvcValueSplit,ekvcValueSplitFast,
        ekvcSecondaryKeys,
        ekvcChange,
        ekvcFlagNew,
        ekvcFlagHeight,
        ekvcFlagRaise,
        ekvcFlagSetTrue,ekvcFlagSetFalse,
        ekvcFlagIf,ekvcFlagIfRaised,
        ekvcFlagReset,
        ekvcFlagGetHeight,ekvcFlagGetHeights,
        ekvcPass
      }
    ,moretexcs=[3]^^A expkv-pkg macros
      {
        ^^A expkv
        ekvdef,ekvdefNoVal,^^A
        ekvlet,ekvletNoVal,ekvletkv,ekvletkvNoVal,^^A
        ekvdefunknown,ekvdefunknownNoVal,^^A
        ekvredirectunknown,ekvredirectunknownNoVal,^^A
        ekvset,ekvsetSneaked,^^A
        ekvsetdef,ekvsetSneakeddef,ekvsetdefSneaked,^^A
        ekvparse,^^A
        ekvVersion,ekvDate,^^A
        ekvifdefined,ekvifdefinedNoVal,ekvifdefinedset,^^A
        ekvbreak,ekvbreakPreSneak,ekvbreakPostSneak,^^A
        ekvsneak,ekvsneakPre,^^A
        ekvchangeset,^^A
        ekvoptarg,ekvoptargTF,^^A
        ekverr,
        ^^A expkv-def
        ekvdefinekeys
      }
  }
\colorlet{codeparam}{cyan!65!black}
\def\mylstwd{.55em}
\lstset
  {
    ,language=expkvc
    ,flexiblecolumns=false
    ,basewidth=\mylstwd
    ,gobble=2
    ,basicstyle=\fontfamily{jkp}\itshape
    ,texcsstyle=*[1]{\bfseries\color{expkvgrey}} ^^A primitives
    ,texcsstyle=*[2]{\bfseries\color{expkvred}}  ^^A expkv
    ,texcsstyle=*[3]{\color{expkvred}}           ^^A expkv-pkg
    ,texcsstyle=*[4]{\bfseries\color{expkvgrey}} ^^A e-TeX
    ,texcsstyle=*[5]{\bfseries\color{expkvgrey}} ^^A plain/LaTeX macros
    ,texcsstyle=*[6]{}                           ^^A macros of other packages
    ,commentstyle=\color[gray]{0.4}
    ,literate=
              {\{} {{\CodeUpBf\{}}{1}
              {\}} {{\CodeUpBf\}}}{1}
              {$}  {{\CodeUpBf\$}}{1}
              {[}  {{\CodeUp[}}{1}
              {]}  {{\CodeUp]}}{1}
              {(}  {{\CodeUp(}}{1}
              {)}  {{\CodeUp)}}{1}
              {*}  {{$*$}}{1}
              {1}  {{\CodeUp{1}}}{1}
              {2}  {{\CodeUp{2}}}{1}
              {3}  {{\CodeUp{3}}}{1}
              {4}  {{\CodeUp{4}}}{1}
              {5}  {{\CodeUp{5}}}{1}
              {6}  {{\CodeUp{6}}}{1}
              {7}  {{\CodeUp{7}}}{1}
              {8}  {{\CodeUp{8}}}{1}
              {9}  {{\CodeUp{9}}}{1}
              {0}  {{\CodeUp{0}}}{1}
              {##} {{\CodeColored{codeparam}{1}{\#}}}{1}
              {##1}{{\CodeColored{codeparam}{2}{\#1}}}{2}
              {##2}{{\CodeColored{codeparam}{2}{\#2}}}{2}
              {##3}{{\CodeColored{codeparam}{2}{\#3}}}{2}
              {##4}{{\CodeColored{codeparam}{2}{\#4}}}{2}
              {##5}{{\CodeColored{codeparam}{2}{\#5}}}{2}
              {##6}{{\CodeColored{codeparam}{2}{\#6}}}{2}
              {##7}{{\CodeColored{codeparam}{2}{\#7}}}{2}
              {##8}{{\CodeColored{codeparam}{2}{\#8}}}{2}
              {##9}{{\CodeColored{codeparam}{2}{\#9}}}{2}
              {<key>}{{$\langle$}key{$\rangle$}}{5}
              {<set>}{{$\langle$}set{$\rangle$}}{5}
              {<set1>}{{$\langle$}set1{$\rangle$}}{6}
              {<set2>}{{$\langle$}set2{$\rangle$}}{6}
    ^^A,literate=*{<key>}{\key}{4}{<set>}{\set}{4}
  }
\newcommand*\CodeColored[3]{\textcolor{#1}{\makebox[\dimexpr\mylstwd*#2]{#3}}}
\newcommand*\CodeUpBf[1]{\makebox[\mylstwd]{\textup{\textbf{#1}}}}
\newcommand*\CodeUp[1]{\makebox[\mylstwd]{\textup{#1}}}
\RequirePackage{randtext}
\let\metaORIG\meta
\protected\def\meta #1{\texttt{\metaORIG{#1}}}
\renewcommand*\thefootnote{\fnsymbol{footnote}}
\definecolor{expkvred}{HTML}{9F393D}
\colorlet{expkvgrey}{black!75}
\makeatletter
\newcommand*\example{\par\smallskip\noindent\textit{Example:} \ignorespaces}
\newcommand*\expkv
  {^^A
    \texorpdfstring
      {^^A
        \mbox
          {^^A
            \BeginAccSupp{ActualText=expkv}^^A
            \href{https://github.com/Skillmon/tex_expkv}
              {^^A
                \rmfamily
                \bfseries
                {\color{expkvgrey}e\kern-.05em x\kern-.05em}^^A
                \lower.493ex
                  \hbox{{\color{expkvgrey}P}\kern-.1em{\color{expkvred}k}}^^A
                \kern-.18em{\color{expkvred}v}^^A
              }^^A
            \EndAccSupp{}^^A
          }^^A
      }
      {expkv}^^A
  }
\newcommand*\expkvpkg[1]
  {^^A
    \texorpdfstring
      {^^A
        \mbox
          {^^A
            \BeginAccSupp{ActualText=expkv-#1}^^A
            \href{https://github.com/Skillmon/tex_expkv-#1}
              {^^A
                \rmfamily
                \bfseries
                {\color{expkvgrey}e\kern-.05em x\kern-.05em}^^A
                \lower.493ex
                  \hbox{{\color{expkvgrey}P}\kern-.1em{\color{expkvred}k}}^^A
                \kern-.18em{\color{expkvred}v}^^A
                {\color{expkvgrey}^^A
                  \kern.05em\rule[-.1ex]{.08em}{1.2ex}\kern.05em\textsc{#1}^^A
                }^^A
              }^^A
            \EndAccSupp{}^^A
          }^^A
      }
      {expkv-#1}^^A
  }
\newcommand*\expkvd{\expkvpkg{def}}
\newcommand*\expkvc{\expkvpkg{cs}}
\newcommand\kv{\meta{key}=\meta{value}}
\newcommand\key{\meta{key}}
\newcommand\val{\meta{value}}
\newcommand\set{\meta{set}}
\newcommand\enflong{\textcolor{black}{long}}
\newcommand\alllong{\textcolor{gray}{long}}
\newcommand\notlong{\textcolor{red!80!black}{long}}
\newcommand\prefixes[1]
  {^^A
    \hfill
    \ifcase\numexpr#1\relax\or
      \enflong\or
      \alllong\or
      \notlong\fi
  }
\hypersetup{linkcolor=red!80!black,urlcolor=purple!80!black}
\DoNotIndex{\def,\edef,\,,\=,\begingroup,\catcode,\chardef,\csname,\endcsname}
\DoNotIndex{\endgroup,\endinput,\errmessage,\expandafter,\input,\let,\long,\z@}
\DoNotIndex{\protected,\ProvidesFile,\ProvidesPackage,\relax,\space,\advance}
\DoNotIndex{\@,\unexpanded,\string,\expanded,\dimexpr,\global,\glueexpr,\hbox}
\DoNotIndex{\numexpr,\RequirePackage,\setbox,\the,\unless,\xdef,\gdef,\newcount}
\DoNotIndex{\number}
\DoNotIndex{\^,\@firstofone,\@firstoftwo,\@gobble,\@secondoftwo}
\DoNotIndex{\ifcsname}
\DoNotIndex{\ifx}
\DoNotIndex{\ifodd}
\DoNotIndex{\ifnum}
\DoNotIndex{\ifdefined}
\DoNotIndex{\iffalse}
\DoNotIndex{\iftrue}
\DoNotIndex{\else}
\DoNotIndex{\fi}
\@gobble\fi ^^A ignoring \ifx, \ifcsname, etc., but only one \fi
\@gobble\fi ^^A ignoring \ifx, \ifcsname, etc., but only one \fi
\@gobble\fi ^^A ignoring \ifx, \ifcsname, etc., but only one \fi
\@gobble\fi ^^A ignoring \ifx, \ifcsname, etc., but only one \fi
\@gobble\fi ^^A ignoring \ifx, \ifcsname, etc., but only one \fi
\@gobble\fi ^^A ignoring \ifx, \ifcsname, etc., but only one \fi
\@ifdefinable\gobbledocstriptag{\def\gobbledocstriptag#1>{}}
\makeatother
\begin{document}
  \title
    {^^A
      \texorpdfstring
        {^^A
          \huge\expkvc
          \\[\medskipamount]
          \Large define expandable \kv\ macros using \expkv
        }{expkv-cs - define expandable key=val macros using expkv}^^A
    }
  \date{\ekvcDate\space v\ekvcVersion}
  \author{Jonathan P. Spratte\thanks{\protect\randomize{jspratte@yahoo.de}}}
  \DocInput{expkv-cs.dtx}
\end{document}
%</driver>^^A=<<
% \fi
%
% \maketitle
% \renewcommand*\thefootnote{\arabic{footnote}}
%
% \begin{abstract}
% \noindent\parfillskip=0pt
% \expkvc\ provides two small interfaces to define expandable \kv\ macros using
% \expkv. It therefore lowers the entrance boundary to expandable \kv\ macros.
% The stylised name is \expkvc\ but the files use \file{expkv-cs}, this is due
% to CTAN-rules which don't allow \string| in package names since that is the
% pipe symbol in *nix shells.
% \end{abstract}
%
% \tableofcontents
%
% \begin{documentation}^^A>>=
%
% \section{Documentation}
%
% The \expkv\ package enables the new possibility of creating \kv\ macros which
% are fully expandable. The creation of such macros is however cumbersome for
% the average user. \expkvc\ tries to step in here. It provides interfaces to
% define \kv\ macros without worrying too much about the implementation. In case
% you're wondering now, the |cs| in \expkvc\ stands for control sequence,
% because |def| was already taken by \expkvd\ and ``control sequence'' is the
% term D.\,E.\,Knuth used in his \TeX book for named commands hence macros
% (though he also used the term ``macro''). So \expkvc\ defines control
% sequences for and with \expkv.
%
% There are two different approaches supported by this package. The first is
% splitting the keys up into individual arguments, the second is providing all
% the keys as a single argument to the underlying macro and getting an
% individual \val\ by using a hash. Well, actually there is no real hash, just
% some markers which are parsed, but this shouldn't be apparent to the user, the
% behaviour matches that of a hash-table.
%
% In addition to these two methods of defining a macro with primary keys a way
% to define secondary keys, which can reference the primary ones, is provided.
% These secondary keys don't correspond to an argument or an entry in the hash
% table directly but might come in handy for the average use case. Each macro
% has its own set of primary and secondary keys.
%
% A word of advice you should consider: If your macro doesn't have to be
% expandable (and often it doesn't) consider not using \expkvc. The interface
% has some overhead (though it still is fast -- check \autoref{sec:speed}) and
% the approach has its limits in versatility. If you don't need to be
% expandable, you should consider either defining your keys manually using
% \expkv\ or using \expkvd\ for convenience. Or you resort to another \kv\
% interface. Nevertheless setting up macros with \expkvc, especially with
% |\ekvcSplit|, is very convenient in my opinion, so if you just want to define
% a single macro with just a few keys this might be the way to go.
%
% \expkvc\ is usable as generic code and as a \LaTeX\ package. It'll
% automatically load \expkv\ in the same mode as well. To use it, just use one
% of
% \begin{lstlisting}
% \usepackage{expkv-cs} % LaTeX
% \input expkv-cs       % plainTeX
% \end{lstlisting}
%
%
% \subsection{Define Macros and Primary Keys}
%
% All macros defined with \expkvc\ have to be previously undefined or have the
% |\meaning| of |\relax|. This is necessary as there is no way to automatically
% undefine keys once they are set up (neither \expkv\ nor \expkvc\ keep track of
% defined keys) -- so to make sure there are no conflicts only new definitions
% are allowed (that's not the case for individual keys, only for frontend
% macros).
%
%
% \subsubsection{Primary Keys}\label{sec:primaries}
%
% In the following descriptions there will be one argument named \meta{primary
% keys}. This argument should be a \kv\ list where each \key\ will be one
% primary key and \val\ the associated initial value. By default all keys are
% defined short, but you can define long keys by prefixing \key\ with |long|
% (\emph{e.g.}, \texttt{long name=Jonathan P. Spratte}). You only need |long| if
% the key should be able to take a |\par| token. Note however that |long| keys
% are a microscopic grain faster (due to some internals of \expkvc). Only if at
% least one of the keys was |long| the \meta{cs} in the following defining
% macros will be |\long|. For obvious reasons there is no possibility to define
% a macro or key as |\protected|.
%
% At the moment \expkvc\ doesn't require any internal keys, but I can't foresee
% whether this will be the case in the future as well, as it might turn out that
% some features I deem useful can't be implemented without such internal keys.
% Because of this, please don't use key names starting with
% \texttt{EKVC\string|} as that should be the private name space.
%
%
% \subsubsection{Split}
%
% The split variants will provide the key values as separate arguments. This
% limits the number of keys for which this is truly useful.
%
% \begin{function}{\ekvcSplit}
%   \begin{syntax}
%     \cs{ekvcSplit}\meta{cs}\marg{primary keys}\marg{definition}
%   \end{syntax}
%   This defines \meta{cs} to be a macro taking one mandatory argument which
%   should contain a \kv\ list. The \meta{primary keys} will be defined for this
%   macro (see \autoref{sec:primaries}). The \meta{definition} is the code that
%   will be executed. You can access the \val\ of a \key\ by using a macro
%   parameter from |#1| to |#9|. The order of the macro parameters will be the
%   order provided in the \meta{primary keys} list (so |#1| is the \val\ of the
%   key defined first). With |\ekvcSplit| you can define macros using at most
%   nine primary keys.
% \end{function}
% \example The following defines a macro |\foo| that takes the keys |a| and |b|
% and outputs their values in a textual form:\\
% \begin{minipage}{.75\linewidth}
% \begin{lstlisting}
% \ekvcSplit\foo{a=a,b=b}{a is #1.\par b is #2.\par}
% \foo{}
% \foo{b=e}
% \end{lstlisting}
% \end{minipage}^^A
% \begin{exresult}[nobeforeafter,box align=center]{.25\linewidth}
%   \ekvcSplit\foo{a=a,b=b}{a is #1.\par b is #2.\par}
%   \foo{}
%   \foo{b=e}
% \end{exresult}
%
% \begin{function}{\ekvcSplitAndForward}
%   \begin{syntax}
%     \cs{ekvcSplitAndForward}\meta{cs}\marg{after}\marg{primary keys}
%   \end{syntax}
%   This defines \meta{cs} to be a macro taking one mandatory argument which
%   should contain a \kv\ list. You can use as many primary keys as you want
%   with this. The primary keys will be forwarded to \meta{after} as braced
%   arguments (as many as necessary for your primary keys). The order of the
%   braced arguments will be the order of your primary key definitions. In
%   \meta{after} you can use just a single control sequence, or some arbitrary
%   stuff which will be left in the input stream before your braced values (with
%   one set of braces stripped from \meta{after}), so both of the following
%   would be fine:
% \begin{lstlisting}
% \ekvcSplitAndForward\foo\foo@aux{keyA = A, keyB = B}
% \ekvcSplitAndForward\foo{\foo@aux{more args}}{keyA = A, keyB = B}
% \end{lstlisting}
% \end{function}
% In the first case |\foo@aux| should take at least two arguments (|keyA| and
% |keyB|), in the second case at least three (|more args|, |keyA|, and |keyB|).
%
% \begin{function}{\ekvcSplitAndUse}
%   \begin{syntax}
%     \cs{ekvcSplitAndUse}\meta{cs}\marg{primary keys}
%   \end{syntax}
%   This will roughly do the same as |\ekvcSplitAndForward|, but instead of
%   specifying what will be used after splitting the keys, \meta{cs} will use
%   what follows the \kv\ list. So its syntax will be
% \end{function}
%   \begin{syntax}
%     \meta{cs}\{\key=\val, \ldots\}\marg{after}
%   \end{syntax}
% and the code in |after| should expect at least as many arguments as the number
% of keys defined for \meta{cs}.
%
% \subsubsection{Hash}
%
% The hash variants will provide the key values as a single argument in which
% you can access specific values using a special macro. The implementation might
% be more convenient and scale better, \emph{but} it is much slower (for a
% primitive macro with a single key benchmarking was almost $1.7$ times slower,
% the root of which being the key access with |\ekvcValue|, not the parsing, and
% for a key access using |\ekvcValueFast| it was still about $1.2$ times
% slower). So if your macro uses less than ten primary keys, you should consider
% using the split approach.
%
% \begin{function}{\ekvcHash}
%   \begin{syntax}
%     \cs{ekvcHash}\meta{cs}\marg{primary keys}\marg{definition}
%   \end{syntax}
%   This defines \meta{cs} to be a macro taking one mandatory argument which
%   should contain a \kv\ list. You can use as many primary keys as you want.
%   The primary keys will be forwarded as a single argument containing every key
%   to the underlying macro. The underlying macro is defined as
%   \meta{definition}, in which you can access the \val\ of a \key\ by using
%   \texttt{\cs[no-index]{ekvcValue}\{\key\}\{\#1\}}.
% \end{function}
% \example This defines an equivalent macro like the |\foo| defined with
% |\ekvcSplit| earlier:\\
% \begin{minipage}{.75\linewidth}
% \begin{lstlisting}
% \ekvcHash\foo{a=a,b=b}{a is \ekvcValue{a}{#1}.\par
%                        b is \ekvcValue{b}{#1}.\par}
% \foo{}
% \foo{b=e}
% \end{lstlisting}
% \end{minipage}^^A
% \begin{exresult}[nobeforeafter,box align=center]{.25\linewidth}
%   \ekvcHash\foo{a=a,b=b}{a is \ekvcValue{a}{#1}.\par
%                          b is \ekvcValue{b}{#1}.\par}
%   \foo{}
%   \foo{b=e}
% \end{exresult}
%
% \begin{function}{\ekvcHashAndForward}
%   \begin{syntax}
%     \cs{ekvcHashAndForward}\meta{cs}\marg{after}\marg{primary keys}
%   \end{syntax}
%   This defines \meta{cs} to be a macro taking one mandatory argument which
%   should contain a \kv\ list. You can use as many primary keys as you want.
%   The primary keys will be forwarded as a single argument containing every key
%   to \meta{after}. You can use a single macro for \meta{after} or use some
%   arbitrary stuff, which will be left in the input stream before the hashed
%   \kv\ list with one set of braces stripped. In the
%   macro called in \meta{after} you can access the \val\ of a \key\ by using
%   \texttt{\cs[no-index]{ekvcValue}\{\key\}\{\#1\}} (or whichever argument the
%   hashed \kv\ list will be).
% \end{function}
%
% \begin{function}{\ekvcHashAndUse}
%   \begin{syntax}
%     \cs{ekvcHashAndUse}\meta{cs}\marg{primary keys}
%   \end{syntax}
%   This will roughly do the same as |\ekvcHashAndForward|, but instead of
%   specifying what will be used after hashing the keys, \meta{cs} will use what
%   follows the \kv\ list. So its syntax will be
% \end{function}
%   \begin{syntax}
%     \meta{cs}\{\key=\val, \ldots\}\marg{after}
%   \end{syntax}
%
% \begin{function}{\ekvcValue}
%   \begin{syntax}
%     \cs{ekvcValue}\{\key\}\marg{key list}
%   \end{syntax}
%   This is a safe way to access your keys in a hash variant. \key\
%   is the key which's \val\ you want to use out of the \meta{key list}.
%   \meta{key list} should be the key list argument forwarded to your underlying
%   macro by |\ekvcHash|, |\ekvcHashAndForward|, or |\ekvcHashAndUse|. It will
%   be tested whether the hash function to access that \key\ exists, the \key\
%   argument is not empty, and that the \meta{key list} really contains a \val\
%   of that \key. This macro needs exactly two steps of expansion and if used
%   inside of an |\edef| or |\expanded| context will protect the \val\ from
%   further expanding.
% \end{function}
%
% \begin{function}{\ekvcValueFast}
%   \begin{syntax}
%     \cs{ekvcValueFast}\{\key\}\marg{key list}
%   \end{syntax}
%   This behaves just like |\ekvcValue|, but \emph{without any} safety tests. As
%   a result this is about $1.4$ times faster \emph{but} will throw low level
%   \TeX\ errors eventually if the hash function isn't defined or the \key\
%   isn't part of the \meta{key list} (\emph{e.g.}, because it was defined as a
%   key for another macro -- all macros share the same hash function per \key).
%   Use it if you know what you're doing. This macro needs exactly three steps
%   of expansion in the no-errors case.
% \end{function}
%
% \begin{function}{\ekvcValueSplit}
%   \begin{syntax}
%     \cs{ekvcValueSplit}\{\key\}\marg{key list}\marg{next}
%   \end{syntax}
%   If you need a specific \key\ from a \meta{key list} more than once, it'll
%   be a good idea to only extract it once and from then on keep it as a
%   separate argument. Hence the macro \cs{ekvcValueSplit} will extract one
%   specific \key's value from the list and forward the remainder of the list
%   as the first and the \key's value as the second argument to \meta{next}, so
%   the result of this will be \meta{next}\marg{key list'}\marg{value} with
%   \meta{key list'} the remaining list. This is almost as fast as |\ekvcValue|
%   and runs the same tests. Keep in mind that you can't fetch for the same
%   \key\ again from \meta{key list'} as it got removed.
% \end{function}
% \example The following defines a macro |\foo| which will take three keys.
% Since the next parsing step will need the value of one of the keys multiple
% times we split that key off the list:\\
% \begin{minipage}{.75\linewidth}
% \begin{lstlisting}
% \ekvcHash\foo{a=a,b=b,c=c}
%   {\ekvcValueSplit{a}{#1}\foobar}
% \newcommand*\foobar[2]{a is #2.\par
%                        b is \ekvcValue{b}{#1}.\par
%                        c is \ekvcValue{c}{#1}.\par}
% \foo{}
% \end{lstlisting}
% \end{minipage}^^A
% \begin{exresult}[nobeforeafter,box align=center]{.25\linewidth}
%   \ekvcHash\foo{a=a,b=b,c=c}
%     {\ekvcValueSplit{a}{#1}\foobar}
%   \newcommand*\foobar[2]{a is #2.\par
%                          b is \ekvcValue{b}{#1}.\par
%                          c is \ekvcValue{c}{#1}.\par}
%   \foo{}
% \end{exresult}
%
% \begin{function}{\ekvcValueSplitFast}
%   \begin{syntax}
%     \cs{ekvcValueSplitFast}\{\key\}\marg{key list}\marg{next}
%   \end{syntax}
%   This behaves just like |\ekvcValueSplit|, but it won't run the same tests,
%   hence it is faster but more error prone, just like the relation between
%   |\ekvcValue| and |\ekvcValueFast|.
% \end{function}
%
%
% \subsection{Secondary Keys}\label{sec:secondaries}
%
% To remove some of the limitations with the approach that each primary key
% matches an argument or hash entry, you can define secondary keys. Those have
% to be defined for each macro but it doesn't matter whether that macro was a
% split or a hash variant. If a secondary key references another key it doesn't
% matter whether that other key is primary or secondary.
%
% Secondary keys can have a prefix (like |long|) which are called |p|-type
% prefix and must have a type (like |meta|) which are called |t|-type prefix.
% Some types might require some |p|-prefixes, while others might forbid those.
%
% Please keep in mind that key names shouldn't start with \texttt{EKVC\string|}.
%
% \begin{function}{\ekvcSecondaryKeys}
%   \begin{syntax}
%     \cs{ekvcSecondaryKeys}\meta{cs}\{\kv, \ldots\}
%   \end{syntax}
%   This is the front facing macro to define secondary keys. For the macro
%   \meta{cs} define \key\ to have definition \val. The general syntax for \key\
%   should be
%   \begin{quote}\ttfamily\small
%     \meta{prefix} \meta{name}
%   \end{quote}
%   Where \meta{prefix} is a space separated list of optional |p|-type prefixes
%   followed by one |t|-type prefix. The syntax of \val\ is dependent on the
%   used |t|-prefix.
% \end{function}
%
%
% \subsubsection{\texttt{p}-type Prefixes}
%
% There is only one |p|-prefix available, which is |long|.
%
% \begin{function}{long}
%   The following key will be defined |\long|.
% \end{function}
%
%
% \subsubsection{\texttt{t}-type Prefixes}
%
% If you're familiar with \expkvd\ you'll notice that the |t|-type prefixes
% provided here are much fewer. The expansion only concept doesn't allow for
% great variety in the auto-defined keys.
%
% The syntax examples of the |t|-prefixes will show which |p|-prefix will be
% automatically used by printing those black (\texttt{\enflong}), which
% will be available in grey (\texttt{\alllong}), and which will be disallowed in
% red (\texttt{\notlong}). This will be put flush right next to the syntax line.
%
% \begin{function}{meta}
%   \begin{syntax}
%     meta \key\ = \{\kv, \ldots\} \prefixes2
%   \end{syntax}
%   With a |meta| key you can set other keys. Whenever \key\ is used the keys in
%   the \kv\ list will be set to the values given there. You can use the \val\
%   given to \key\ by using |#1| in the \kv\ list. The keys in the \kv\ list can
%   be primary and secondary ones.
% \end{function}
%
% \begin{function}{nmeta}
%   \begin{syntax}
%     nmeta \key\ = \{\kv, \ldots\} \prefixes3
%   \end{syntax}
%   An |nmeta| key is like a |meta| key, but it doesn't take a value, so the
%   \kv\ list is static.
% \end{function}
%
% \begin{function}{alias}
%   \begin{syntax}
%     alias \key\ = \meta{key_2} \prefixes3
%   \end{syntax}
%   This assigns the definition of \meta{key_2} to \key. As a result \key\ is an
%   alias for \meta{key_2} behaving just the same. Both the value taking and the
%   |NoVal| version (that's \expkv\ slang for a key not accepting a value) will
%   be copied if they are defined when |alias| is used. Of course, \meta{key_2}
%   has to be defined, be it as a primary or secondary one.
% \end{function}
%
% \begin{function}{default}
%   \begin{syntax}
%     default \key\ = \marg{default} \prefixes3
%   \end{syntax}
%   If \key\ is a defined value taking key, you can define a |NoVal| version
%   with this that will behave as if \key\ was given \meta{default} as its
%   \val. Note that this doesn't change the initial values of primary keys set
%   at definition time in |\ekvcSplit| and friends. \key\ can be a primary or
%   secondary key.
% \end{function}
%
% \begin{function}{flag-bool}
%   \begin{syntax}
%     flag-bool \key\ = \meta{cs} \prefixes3
%   \end{syntax}
%   This is a secondary key that doesn't directly involve any of the primary or
%   secondary keys.
%   This defines \key\ to take a value, which should be either |true| or
%   |false|, and set the flag called \meta{cs} accordingly as a boolean.
%   If \meta{cs} isn't defined yet it will be initialised as a flag.
%   Please also read \autoref{sec:flags}.
% \end{function}
%
% \begin{function}{flag-true,flag-false}
%   \begin{syntax}
%     flag-true \key\ = \meta{cs} \prefixes3
%   \end{syntax}
%   This is a secondary key that doesn't directly involve any of the primary or
%   secondary keys.
%   This defines \key\ to take no value and set the flag called \meta{cs} to
%   |true| or |false|, respectively.
%   If \meta{cs} isn't defined yet it will be initialised as a flag.
%   Please also read \autoref{sec:flags}.
% \end{function}
%
% \begin{function}{flag-raise}
%   \begin{syntax}
%     flag-raise \key\ = \meta{cs} \prefixes3
%   \end{syntax}
%   This is a secondary key that doesn't directly involve any of the primary or
%   secondary keys.
%   This defines \key\ to take no value and raise the flag called \meta{cs}.
%   If \meta{cs} isn't defined yet it will be initialised as a flag.
%   Please also read \autoref{sec:flags}.
% \end{function}
%
%
% \subsection{Changing the Initial Values}\label{sec:change}
%
% \begin{function}{\ekvcChange}
%   \begin{syntax}
%     \cs{ekvcChange}\meta{cs}\{\kv,\ldots\}
%   \end{syntax}
%   This processes the \kv\ list (just like it would be processed by \meta{cs})
%   to set new defaults for the macro \meta{cs} defined with \expkvc\ (meaning
%   the values used if you don't provide anything at use time). Inside the \kv\
%   list both primary and secondary keys can be used. If \meta{cs} was defined
%   |\long| earlier it will still be |\long|, every other \TeX\ prefix will be
%   stripped (but \expkvc\ doesn't support them anywhere else so that should be
%   fine). The resulting new defaults will be stored inside the \meta{cs}
%   locally (just as the original defaults were).
% \end{function}
%
% Consider the following example:\\
% \begin{minipage}[c]{.75\linewidth}
% \begin{lstlisting}
% \ekvcSplit\foo{a=a,b=b}{a is #1.\par b is #2.\par}
% \foo{} % initial defaults
% \ekvcChange\foo{b=B}
% \foo{} % new defaults
% \ekvcSecondaryKeys\foo{meta c={a={#1},b={#1}}}
% \ekvcChange\foo{c=c}
% \foo{} % newer defaults
% \end{lstlisting}
% \end{minipage}^^A
% \begin{exresult}[nobeforeafter,box align=center]{.25\linewidth}
%   \ekvcSplit\foo{a=a,b=b}{a is #1.\par b is #2.\par}
%   \foo{}
%   \ekvcChange\foo{b=B}
%   \foo{}
%   \ekvcSecondaryKeys\foo{meta c={a={#1},b={#1}}}
%   \ekvcChange\foo{c=c}
%   \foo{}
% \end{exresult}
%
% As a result with this the typical setup macro could be implemented:
% \begin{lstlisting}
% \ekvcHashAndUse\foo{key=a,key=b}
% \newcommand\foosetup{\ekvcChange\foo}
% \end{lstlisting}
% Of course the usage is limited to a single macro |\foo|, hence this might be
% not as powerful as similar macros used with other \kv\ interfaces. But at
% least a few similar macros could be grouped using the same key parsing macro
% internally.
%
%
% \subsection{Flags}\label{sec:flags}
%
% The idea of flags is taken from \pkg{expl3}. They provide a way to store
% numerical information expandably, however only incrementing and accessing
% works expandably, decrementing is unexpandable. A flag has a height, which
% is a numerical value, and which can be raised by $1$. Flags come at a high
% computational cost (accessing them is slow and they require more memory than
% normal \TeX\ data types like registers, both getting linearly worse with the
% height), so don't use them if not necessary.
%
% The state of flags is always changed locally to the current group, but not to
% the current macro, so if you're using one of the |t|-types involving flags
% bear in mind that they can affect other macros using the same flags at the
% current group level!
%
% \expkvc\ provides some macros to access, alter, and use flags. Flags of
% \expkvc\ don't share a name space with the flags of \pkg{expl3}.
%
% \begin{function}{\ekvcFlagNew}
%   \begin{syntax}
%     \cs{ekvcFlagNew}\meta{flag}
%   \end{syntax}
%   This initialises the macro \meta{flag} as a new flag. It isn't checked
%   whether the macro \meta{flag} is currently undefined.  A \meta{flag} will
%   expand to the flag's current height with a trailing space (so you can use it
%   directly with |\ifnum| for example and it will terminate the number scanning
%   itself).
% \end{function}
%
%  All other macros dealing with flags take as a parameter a macro defined as a
%  \meta{flag} with |\ekvcFlagNew|.
%
% \begin{function}{\ekvcFlagHeight}
%   \begin{syntax}
%     \cs{ekvcFlagHeight}\meta{flag}
%   \end{syntax}
%   This expands to the current height of \meta{flag} in a single step of
%   expansion (without a trailing space).
% \end{function}
%
% \begin{function}{\ekvcFlagRaise}
%   \begin{syntax}
%     \cs{ekvcFlagRaise}\meta{flag}
%   \end{syntax}
%   This expandably raises the height of \meta{flag} by $1$.
% \end{function}
%
% \begin{function}{\ekvcFlagSetTrue,\ekvcFlagSetFalse}
%   \begin{syntax}
%     \cs{ekvcFlagSetTrue}\meta{flag}
%   \end{syntax}
%   By interpreting an even value as false and an odd value as true we can use a
%   flag as a boolean. This expandably sets \meta{flag} to true or false,
%   respectively, by raising it if necessary.
% \end{function}
%
% \begin{function}{\ekvcFlagIf}
%   \begin{syntax}
%     \cs{ekvcFlagIf}\meta{flag}\marg{true}\marg{false}
%   \end{syntax}
%   This interprets a \meta{flag} as a boolean and expands to either \meta{true}
%   or \meta{false}.
% \end{function}
%
% \begin{function}{\ekvcFlagIfRaised}
%   \begin{syntax}
%     \cs{ekvcFlagIfRaised}\meta{flag}\marg{true}\marg{false}
%   \end{syntax}
%   This tests whether the \meta{flag} is raised, meaning it has a height
%   greater than zero, and if so expands to \meta{true} else to \meta{false}.
% \end{function}
%
% \begin{function}{\ekvcFlagReset}
%   \begin{syntax}
%     \cs{ekvcFlagReset}\meta{flag}
%   \end{syntax}
%   This resets a flag (so restores its height to $0$). This operation is
%   \emph{not} expandable and done locally. If you really intend to use flags
%   you can reset them every now and then to keep the performance hit low.
% \end{function}
%
% \begin{function}{\ekvcFlagGetHeight}
%   \begin{syntax}
%     \cs{ekvcFlagGetHeight}\meta{flag}\marg{next}
%   \end{syntax}
%   This retrieves the current height of the \meta{flag} and provides it as a
%   braced argument to \meta{next}, leaving \meta{next}\marg{height} in the
%   input stream.
% \end{function}
%
% \begin{function}{\ekvcFlagGetHeights}
%   \begin{syntax}
%     \cs{ekvcFlagGetHeights}\marg{flag-list}\marg{next}
%   \end{syntax}
%   This retrieves the current height of each \meta{flag} in the
%   \meta{flag-list} and provides them as a single argument to \meta{next}.
%   Inside that argument each height is enclosed in a set of braces
%   individually. The \meta{flag-list} is just a single argument containing the
%   \meta{flag}s.
%   So a usage like
%   \texttt
%     {^^A
%       \cs[no-index]{ekvcFlagGetHeights}\hskip0pt^^A
%       \{\cs[no-index]{myflagA}\cs[no-index]{myflagB}\}\hskip0pt^^A
%       \{\cs[no-index]{stuff}\}^^A
%     }
%   will expand to
%   \texttt{\cs[no-index]{stuff}\{\marg{height-A}\marg{height-B}\}}.
% \end{function}
%
%
% \subsection{Further Examples}
%
% How could a documentation be a good documentation without some basic examples?
% Say we want to define a small macro expanding to some character description
% (who knows why this has to be expandable?). A character description will not
% have too many items to it, so we use |\ekvcSplit| (the comments with the
% parameter numbers are of course not necessary and just ease reading the
% example).
% \begin{lstlisting}
% \ekvcSplit\character
%   {
%     name=John Doe,            % #1
%     age=any,                  % #2
%     nationality=the Universe, % #3
%     hobby=to exist,           % #4
%     type=Mister,              % #5
%     pronoun=He,               % #6
%     possessive=his,           % #7
%   }
%   {#1 is a #5 from #3. #6 is of #2 age and #7 hobby is #4.\par}
% \end{lstlisting}
% Also we want to give some short cuts so that it's easier to describe several
% persons.
% \begin{lstlisting}
% \ekvcSecondaryKeys\character
%   {
%     alias pro = pronoun,
%     alias pos = possessive,
%     nmeta me =
%       {
%         name=Jonathan,
%         age=a young,
%         nationality=Germany,
%         hobby=\TeX\ coding,
%       },
%     meta lady =
%       {type=Lady, pronoun=She, possessive=her, name=Jane Doe, #1},
%     nmeta paulo =
%       {
%         name=Paulo,
%         type=duck,
%         age=a young,
%         nationality=Brazil,
%         hobby=to quack,
%       }
%   }
% \end{lstlisting}
% Now we can describe people using
% \begin{lstlisting}
% \character{}
% \character{me}
% \character{paulo}
% \character
%   {lady={name=Evelyn,nationality=Ireland,age=the best,hobby=reading}}
% \character
%   {
%     name=Our sun, type=star, nationality=our solar system, pro=It,
%     age=an old, pos=its, hobby=shining
%   }
% \end{lstlisting}
% As one might see, the |lady| key could actually have been an |nmeta| key as
% well, as all that is done with the argument is using it as a \kv\ list.
%
% The result of only the first two usages would be:
% \begin{exresult}{\linewidth}
%   \ekvcSplit\character
%     {
%       name=John Doe,
%       age=any,
%       nationality=the Universe,
%       hobby=to exist,
%       type=Mister,
%       pronoun=He,
%       possessive=his,
%     }
%     {#1 is a #5 from #3. #6 is of #2 age and #7 hobby is #4.\par}
%   \ekvcSecondaryKeys\character
%     {
%       alias pro = pronoun,
%       alias pos = possessive,
%       nmeta me =
%         {
%           name=Jonathan,
%           age=a young,
%           nationality=Germany,
%           hobby=\TeX\ coding,
%         },
%       meta lady =
%         {type=Lady, pronoun=She, possessive=her, name=Jane Doe, #1},
%       nmeta paulo =
%         {
%           name=Paulo,
%           type=duck,
%           age=a young,
%           nationality=Brazil,
%           hobby=to quack,
%         }
%     }
%   \character{}
%   \character{me}
% \end{exresult}
%
% Using \pkg{xparse} or \expkv's \cs{ekvoptarg} or \cs{ekvoptargTF} and
% forwarding arguments one can easily define \kv\ macros with actual optional
% and mandatory arguments as well. A small nonsense example (which should
% perhaps use |\ekvcSplitAndForward| instead of |\ekvcHashAndForward| since it
% only uses four keys and one other argument -- and isn't expandable since it
% uses a \env{tabular} environment, so it would've been better to use a more
% feature rich \kv\ interface most likely, \emph{e.g.}, the one provided by
% \expkvd):
% \begin{lstlisting}
% \makeatletter
% \newcommand*\nonsense{\ekvoptarg\nonsense@a{}}
% \ekvcHashAndForward\nonsense@a\nonsense@b
%   {
%     keyA = A,
%     keyB = B,
%     keyC = c,
%     keyD = d,
%   }
% \newcommand*\nonsense@b[2]
%   {%
%     \begin{tabular}{lll}
%       key & A & \ekvcValue{keyA}{#1} \\
%           & B & \ekvcValue{keyB}{#1} \\
%           & C & \ekvcValue{keyC}{#1} \\
%           & D & \ekvcValue{keyD}{#1} \\
%       \multicolumn{2}{l}{mandatory} & #2 \\
%     \end{tabular}%
%   }
% \makeatother
% \end{lstlisting}
% And then we would be able to do some nonsense
% \begin{lstlisting}
% \nonsense{}
% \nonsense[keyA=hihi]{haha}
% \nonsense[keyA=hihi, keyB=A]{hehe}
% \nonsense[keyC=huhu, keyA=hihi, keyB=A]{haha}
% \end{lstlisting}
% resulting in
% \begin{exresult}{\linewidth}
%   \makeatletter
%   \newcommand*\nonsense{\ekvoptarg\nonsense@a{}}
%   \ekvcHashAndForward\nonsense@a\nonsense@b
%     {
%       keyA = A,
%       keyB = B,
%       keyC = c,
%       keyD = d,
%     }
%   \newcommand*\nonsense@b[2]
%     {%
%       \begin{tabular}{lll}
%         key & A & \ekvcValue{keyA}{#1} \\
%             & B & \ekvcValue{keyB}{#1} \\
%             & C & \ekvcValue{keyC}{#1} \\
%             & D & \ekvcValue{keyD}{#1} \\
%         \multicolumn{2}{l}{mandatory} & #2 \\
%       \end{tabular}%
%     }
%   \makeatother
%   \nonsense{}
%   \nonsense[keyA=hihi]{haha}
%   \nonsense[keyA=hihi, keyB=A]{hehe}
%   \nonsense[keyC=huhu, keyA=hihi, keyB=A]{haha}
% \end{exresult}
%
%
% \subsection{Freedom for Keys!}
%
% If this was the \TeX book this subsection would have a double bend sign. Not
% because it is overly complicated, but because it shows things which could
% break \expkvc's expandability and its alignment safety. This is for
% experienced users wanting to get the most flexibility and knowing what they
% are doing.
%
% In case you're wondering, it is possible to define other keys than the
% primaries and the secondary types listed in \autoref{sec:secondaries} for a
% macro defined with \expkvc\ by using the low-level interface of \expkv\ or
% even the interface provided by \expkvd. The set name used for \expkvc's keys
% is the macro name, including the leading backslash, or more precisely
% |\string|\meta{cs} is used. This can be exploited to define additional keys
% with arbitrary code. Consider the following \emph{bad} example:
% \begin{lstlisting}
% \ekvcSplit\foo{a=A,b=B}{a is #1.\par b is #2}
% \protected\ekvdef{\string\foo}{c}{\def\fooC{#1}}
% \end{lstlisting}
% This would define a key named |c| that will store its value inside a macro.
% The issue with this is that this can't be done expandably. As a result, the
% macro |\foo| isn't always expandable any more (not that bad if this was never
% required; killjoy if it was) and as soon as the key |c| is used, it is also no
% longer alignment safe\footnote{This means that the \kv-list can't contain
%   alignment markers that are not inside an additional set of braces if used
%   inside a \TeX\ alignment.}
% (might be bad depending on the usage).
%
% So why do I show you this? Because we could as well do something useful like
% create a key that pre-parses the input and after that passes the parsed value
% on. This parsing would have to be completely expandable though. For the
% pass-on part we can use the following function:
%
% \begin{function}{\ekvcPass}
%   \begin{syntax}
%     \cs{ekvcPass}\meta{cs}\marg{key}\marg{value}
%   \end{syntax}
%   This passes \meta{value} on to \meta{key} for the \expkvc-macro \meta{cs}.
%   It should be used inside the key parsing of a macro defined with \expkvc,
%   else this most likely results in a low level \TeX\ error.
% \end{function}
%
% With this we could for example split the value of a key at a hyphen and pass
% the parts to different keys:\\
% \begin{minipage}[c]{.75\linewidth}
% \begin{lstlisting}
% \ekvcSplit\foo{a=A,b=B}{a is #1.\par b is #2.\par}
% \ekvdef{\string\foo}{c}{\fooSplit#1\par}
% \def\fooSplit#1-#2\par
%   {\ekvcPass\foo{a}{#1}\ekvcPass\foo{b}{#2}}
% \foo{}
% \foo{c=1-2}
% \end{lstlisting}
% \end{minipage}
% \begin{exresult}[nobeforeafter,box align=center]{.25\linewidth}
%   \ekvcSplit\foo{a=A,b=B}{a is #1.\par b is #2.\par}
%   \ekvdef{\string\foo}{c}{\fooSplit#1\par}
%   \def\fooSplit#1-#2\par{\ekvcPass\foo{a}{#1}\ekvcPass\foo{b}{#2}}
%   \foo{}
%   \foo{c=1-2}
% \end{exresult}
%
%
% \subsection{Speed Considerations}\label{sec:speed}
%
% As already mentioned in the introduction there are some speed considerations
% implied if you choose to define macros via \expkvc. However the overhead isn't
% the factor which should hinder you to use \expkvc\ if you found a reasonable
% use case. The key-parsing is still faster than with most other \kv\ packages
% (see the ``Comparisons'' subsection in the \expkv\ documentation).
%
% The speed considerations in this subsection use the first example in this
% documentation as the benchmark. So we have seven keys and a short sentence
% which should be typeset. For comparisons I use the following equivalent
% \expkvd\ definitions. Each result is the average between changing no keys from
% their initial values and altering four. Furthermore I'll compare three
% variants of \expkvc\ with the \expkvd\ definitions, namely the split example
% from above, a hash variant using |\ekvcValue| and a hash variant using
% |\ekvcValueFast|.
% \begin{lstlisting}
% \usepackage{expkv-def}
% \ekvdefinekeys{keys}
%   {%
%     ,store   name         = \KEYSname
%     ,initial name         = John Doe
%     ,store   age          = \KEYSage
%     ,initial age          = any
%     ,store   nationality  = \KEYSnationality
%     ,initial nationality  = the Universe
%     ,store   hobby        = \KEYShobby
%     ,initial hobby        = to exist
%     ,store   type         = \KEYStype
%     ,initial type         = Mister
%     ,store   pronoun      = \KEYSpronoun
%     ,initial pronoun      = He
%     ,store   possessive   = \KEYSpossessive
%     ,initial possessive   = his
%   }
% \newcommand*\KEYS[1]
%   {%
%     \begingroup
%       \ekvset{keys}{#1}%
%       \KEYSname\ is a \KEYStype\ from \KEYSnationality. \KEYSpronoun\ is
%       of \KEYSage\ age and \KEYSpossessive\ hobby is \KEYShobby.%
%     \endgroup
%   }
% \end{lstlisting}
%
% The first comparison removes the typesetting part from all the definitions, so
% that only the key parsing is compared. In this comparison the |\ekvcValue|
% and |\ekvcValueFast| variants will not differ, as they are exactly the same
% until the key usage. We find that the split approach is $1.4$ times slower
% than the \expkvd\ setup and the hash variants end up in the middle at $1.17$
% times slower.
%
% Next we put the typesetting part back in. Every call of the macros will
% typeset the sentences into a box register in horizontal mode. With the
% typesetting part (which includes the accessing of values) the fastest remains
% the \expkvd\ definitions, but split is close at $1.16$ times slower, followed
% by the hash variant with fast accesses at $1.36$ times slower, and the safe
% hash access variant ranks in the slowest $1.8$ times slower than \expkvd.
%
% Just in case you're wondering now, a simple macro taking seven arguments is
% $30$ to $40$ times faster than any of those in the argument grabbing and \kv\
% parsing part and only $1.5$ to $2.8$ times faster if the typesetting part is
% factored in. So the real choke isn't the parsing.
%
% So to summarize this, if you have a reasonable use case for expandable \kv\
% parsing macros you should go on and define them using \expkvc. If you have a
% reasonable use case for \kv\ parsing macros but defining them expandable isn't
% necessary for your use you should take advantage of the greater flexibility of
% non-expandable \kv\ setups (but if you're after maximum speed there aren't
% that many \kv\ parsers beating \expkvc). And if you are after maximum
% performance maybe ditching the \kv\ interface altogether is a good idea, but
% depending on the number of arguments your interface might get convoluted.
%
%
% \subsection{Useless Macros}
%
% Perhaps these macros aren't completely useless, but I figured from a user's
% point of view I wouldn't know what I should do with these.
%
% \begin{function}{\ekvcDate,\ekvcVersion}
%   These two macros store the version and the date of the package/generic code.
% \end{function}
%
%
% \subsection{Bugs}
%
% Of course I don't think there are any bugs (who would knowingly distribute
% buggy software as long as he isn't a multi-million dollar corporation?). But
% if you find some please let me know. For this one might find my email address
% on the first page or file an issue on Github:
% \url{https://github.com/Skillmon/tex_expkv-cs}
%
%
% \subsection{License}
%
% Copyright \textcopyright\ 2020\unless\ifnum\year=2020--\the\year\fi\
% Jonathan P. Spratte
%
% \medskip\noindent
% This work may be distributed and/or modified under the conditions of the
% \LaTeX\ Project Public License (LPPL), either version 1.3c of this license or
% (at your option) any later version. The latest version of this license is in
% the file:
%
% \url{http://www.latex-project.org/lppl.txt}
%
% \noindent
% This work is ``maintained'' (as per LPPL maintenance status) by
%
% \mbox{Jonathan P. Spratte}.
%
% \end{documentation}^^A=<<
%
% \begin{implementation}^^A>>=
%
% \clearpage
%
% \section{Implementation}^^A>>=
%^^A the LaTeX package >>=
% \subsection{The \LaTeX\ Package}
% Just like for \expkv\ we provide a small \LaTeX\ package that sets up things
% such that we behave nicely on \LaTeX\ packages and files system. It'll
% |\input| the generic code which implements the functionality.
% \gobbledocstriptag
%<*pkg>
%    \begin{macrocode}
\RequirePackage{expkv}
\def\ekvc@tmp
  {%
    \ProvidesFile{expkv-cs.tex}%
      [%
        \ekvcDate\space v\ekvcVersion\space
        define expandable key=val macros using expkv%
      ]%
  }
\input{expkv-cs.tex}
\ProvidesPackage{expkv-cs}%
  [%
    \ekvcDate\space v\ekvcVersion\space
    define expandable key=val macros using expkv%
  ]
%    \end{macrocode}
% \gobbledocstriptag
%</pkg>
%^^A=<<
%^^A main file >>=
% \subsection{The Generic Code}
% The rest of this implementation will be the generic code.
% \gobbledocstriptag
%<*tex>
%
% Load \expkv\ if the package didn't already do so -- since \expkv\ has
% safeguards against being loaded twice this does no harm and the overhead
% isn't that big. Also we reuse some of the internals of \expkv\ to save us from
% retyping them.
%    \begin{macrocode}
\input expkv
%    \end{macrocode}
%
% We make sure that \file{expkv-cs.tex} is only input once:
%    \begin{macrocode}
\expandafter\ifx\csname ekvcVersion\endcsname\relax
\else
  \expandafter\endinput
\fi
%    \end{macrocode}
%
% \begin{macro}{\ekvcVersion,\ekvcDate}
% We're on our first input, so lets store the version and date in a macro.
%    \begin{macrocode}
\def\ekvcVersion{0.10}
\def\ekvcDate{2021-06-03}
%    \end{macrocode}
% \end{macro}
%
% If the \LaTeX\ format is loaded we want to be a good file and report back who
% we are, for this the package will have defined |\ekvc@tmp| to use
% |\ProvidesFile|, else this will expand to a |\relax| and do no harm.
%    \begin{macrocode}
\csname ekvc@tmp\endcsname
%    \end{macrocode}
%
% Store the category code of |@| to later be able to reset it and change it to
% 11 for now.
%    \begin{macrocode}
\expandafter\chardef\csname ekvc@tmp\endcsname=\catcode`\@
\catcode`\@=11
%    \end{macrocode}
% |\ekvc@tmp| will be reused later, but we don't need it to ever store
% information long-term after \expkvc\ was initialized.
%
% \begin{macro}[internal]{\ekvc@keycount}
%   We'll need to keep count how many keys must be defined for each macro in the
%   |split| variants.
%    \begin{macrocode}
\newcount\ekvc@keycount
%    \end{macrocode}
% \end{macro}
%
% \begin{macro}[internal]{\ekvc@long,\ekvc@any@long}
%   Some macros will have to be defined long. These two will be let to |\long|
%   when this should be the case.
%    \begin{macrocode}
\let\ekvc@long\ekv@empty
\let\ekvc@any@long\ekv@empty
%    \end{macrocode}
% \end{macro}
%
% \begin{macro}[internal]{\ekvc@ifdefined}
%   We want to test whether a macro is already defined. This test checks for a
%   defined macro that isn't |\relax|.
%    \begin{macrocode}
\long\def\ekvc@ifdefined#1%
  {%
    \ifdefined#1%
      \ifx\relax#1%
        \ekv@fi@gobble
      \fi
      \@firstofone
      \ekv@fi@firstoftwo
    \fi
    \@secondoftwo
  }
%    \end{macrocode}
% \end{macro}
%
% \begin{macro}[internal]
%   {
%     \ekvc@ekvset@pre@expander,
%     \ekvc@ekvset@pre@expander@a,
%     \ekvc@ekvset@pre@expander@b
%   }
%   This macro expands |\ekvset| twice so that the first two steps of expansion
%   don't have to be made every time the \expkvc\ macros are used. We have to do
%   a little magic trick to get the macro parameter |#1| for the macro
%   definition this is used in, even though we're calling |\unexpanded|. We do
%   that by splitting the expanded |\ekvset| at some marks and place |##1| in
%   between. At this spot we also add |\ekv@alignsafe| and |\ekv@endalignsafe|
%   to ensure that macros created with \expkvc\ are alignment safe.
%    \begin{macrocode}
\def\ekvc@ekvset@pre@expander#1%
  {%
    \expandafter\ekvc@ekvset@pre@expander@a\ekvset{#1}\ekvc@stop\ekvc@stop
  }
\def\ekvc@ekvset@pre@expander@a
  {%
    \expandafter\ekvc@ekvset@pre@expander@b
  }
\def\ekvc@ekvset@pre@expander@b#1\ekvc@stop#2\ekvc@stop
  {%
    \unexpanded\expandafter{\ekv@alignsafe}%
    \unexpanded{#1}##1\unexpanded{#2}%
    \unexpanded\expandafter{\ekv@endalignsafe}%
  }
%    \end{macrocode}
% \end{macro}
%
% \begin{macro}{\ekvcSplitAndUse}
%   The first user macro we want to set up can be reused for
%   \cs[no-index]{ekvcSplitAndForward} and \cs[no-index]{ekvcSplit}. We'll split
%   this one up so that the test whether the macro is already defined doesn't
%   run twice.
%    \begin{macrocode}
\protected\long\def\ekvcSplitAndUse#1#2%
  {%
    \let\ekvc@helpers@needed\@firstoftwo
    \ekvc@ifdefined#1%
      {\ekvc@err@already@defined#1}%
      {\ekvcSplitAndUse@#1{}{#2}}%
  }
%    \end{macrocode}
% \end{macro}
%
% \begin{macro}[internal]{\ekvcSplitAndUse@}
%   The actual macro setting up things. We need to set some variables, forward
%   the key list to |\ekvc@SetupSplitKeys|, and afterwards define the front
%   facing macro to call |\ekvset| and put the initials and the argument sorting
%   macro behind it. The internals |\ekvc@any@long|, |\ekvc@initials| and
%   |\ekvc@keycount| will be set correctly by |\ekvc@SetupSplitKeys|.
%    \begin{macrocode}
\protected\long\def\ekvcSplitAndUse@#1#2#3%
  {%
    \edef\ekvc@set{\string#1}%
    \ekvc@SetupSplitKeys{#3}%
    \ekvc@helpers@needed
      {%
        \ekvc@any@long\edef#1##1%
          {%
            \expandafter\ekvc@ekvset@pre@expander\expandafter{\ekvc@set}%
            \unexpanded\expandafter
              {\csname ekvc@split@\the\ekvc@keycount\endcsname}%
            \unexpanded\expandafter{\ekvc@initials{}#2}%
          }%
      }%
      {%
        \ekvc@any@long\edef#1##1%
          {%
            \expandafter\ekvc@ekvset@pre@expander\expandafter{\ekvc@set}%
            \unexpanded{#2}%
            \unexpanded\expandafter{\ekvc@initials}%
          }%
      }%
  }
%    \end{macrocode}
% \end{macro}
%
%
% \begin{macro}{\ekvcSplitAndForward}
%   This just reuses |\ekvcSplitAndUse@| with a non-empty second argument,
%   resulting in that argument to be called after the splitting.
%    \begin{macrocode}
\protected\long\def\ekvcSplitAndForward#1#2#3%
  {%
    \let\ekvc@helpers@needed\@firstoftwo
    \ekvc@ifdefined#1%
      {\ekvc@err@already@defined#1}%
      {\ekvcSplitAndUse@#1{{#2}}{#3}}%
  }
%    \end{macrocode}
% \end{macro}
%
% \begin{macro}{\ekvcSplit}
%   The first half is just |\ekvcSplitAndForward| then we define the macro to
%   which the parsed key list is forwarded. There we need to allow for up to
%   nine arguments.
%    \begin{macrocode}
\protected\long\def\ekvcSplit#1#2#3%
  {%
    \let\ekvc@helpers@needed\@secondoftwo
    \ekvc@ifdefined#1%
      {\ekvc@err@already@defined#1}%
      {%
        \expandafter
        \ekvcSplitAndUse@\expandafter#1\csname ekvc@\string#1\endcsname{#2}%
        \ifnum\ekvc@keycount<1
          \ekvc@any@long\expandafter\def\csname ekvc@\string#1\endcsname{#3}%
        \else
          \ifnum\ekvc@keycount>9
            \ekvc@err@toomany{#1}%
            \let#1\ekvc@undefined
          \else
            \ekvcSplit@build@argspec
            \ekvc@any@long\expandafter
            \def\csname ekvc@\string#1\expandafter\endcsname\ekvc@tmp{#3}%
          \fi
        \fi
      }%
  }
%    \end{macrocode}
% \end{macro}
%
% \begin{macro}[internal]{\ekvcSplit@build@argspec,\ekvcSplit@build@argspec@}
%    \begin{macrocode}
\protected\def\ekvcSplit@build@argspec
  {%
    \begingroup
      \edef\ekvc@tmp
        {\endgroup\def\unexpanded{\ekvc@tmp}{\ekvcSplit@build@argspec@{1}}}%
    \ekvc@tmp
  }
\def\ekvcSplit@build@argspec@#1%
  {%
    \ifnum#1>\ekvc@keycount
      \ekv@fi@gobble
    \fi
    \@firstofone
      {%
        \unexpanded\expandafter{\csname ekvc@splitmark@#1\endcsname####}#1%
        \expandafter\ekvcSplit@build@argspec@\expandafter{\the\numexpr#1+1}%
      }%
  }
%    \end{macrocode}
% \end{macro}
%
% \begin{macro}[internal]
%   {
%     \ekvc@SetupSplitKeys, \ekvc@SetupSplitKeys@a, \ekvc@SetupSplitKeys@b,
%     \ekvc@SetupSplitKeys@c
%   }
%   These macros parse the list of keys and set up the key macros. First we need
%   to initialise some macros and start |\ekvparse|.
%    \begin{macrocode}
\protected\long\def\ekvc@SetupSplitKeys
  {%
    \ekvc@keycount=0
    \let\ekvc@any@long\ekv@empty
    \let\ekvc@initials\ekv@empty
    \ekvparse\ekvc@err@value@required\ekvc@SetupSplitKeys@a
  }
%    \end{macrocode}
%   Then we need to step the key counter for each key. Also we have to check
%   whether this key has a |long| prefix so we initialise |\ekvc@long|.
%    \begin{macrocode}
\protected\def\ekvc@SetupSplitKeys@a#1%
  {%
    \advance\ekvc@keycount1
    \let\ekvc@long\ekv@empty
    \ekvc@ifspace{#1}%
      {\ekvc@SetupSplitKeys@b#1\ekvc@stop}%
      {\ekvc@SetupSplitKeys@c{#1}}%
  }
%    \end{macrocode}
%   If there was a space, there might be a prefix. If so call the prefix macro,
%   else call the next step |\ekvc@SetupSplitKeys@c| which will define the key
%   macro and add the key's value to the initials list.
%    \begin{macrocode}
\protected\def\ekvc@SetupSplitKeys@b#1 #2\ekvc@stop
  {%
    \ekv@ifdefined{ekvc@split@p@#1}%
      {\csname ekvc@split@p@#1\endcsname{#2}}%
      {\ekvc@SetupSplitKeys@c{#1 #2}}%
  }
%    \end{macrocode}
%   The inner definition is grouped, because we don't want to actually define
%   the marks we build with |\csname|. We have to append the value to the
%   |\ekvc@initials| list here with the correct split mark. The key macro will
%   read everything up to those split marks and change the value following it to
%   the value given to the key. Additionally we'll need a sorting macro for each
%   key count in use so we set it up with |\ekvc@setup@splitmacro|.
%    \begin{macrocode}
\protected\long\def\ekvc@SetupSplitKeys@c#1#2%
  {%
    \begingroup
    \edef\ekvc@tmp
      {%
        \endgroup
        \long\def\unexpanded{\ekvc@tmp}####1####2%
            \unexpanded\expandafter
            {\csname ekvc@splitmark@\the\ekvc@keycount\endcsname}####3%
          {%
            ####2%
            \unexpanded\expandafter
              {\csname ekvc@splitmark@\the\ekvc@keycount\endcsname}{####1}%
          }%
%    \end{macrocode}
%   The short variant needs a bit of special treatment. The key macro will be
%   short to throw the correct error, but since there might be long macros
%   somewhere the reordering of arguments needs to be long, so for short keys we
%   use a two step approach, first grabbing only the short argument, then
%   reordering.
%    \begin{macrocode}
        \unless\ifx\ekvc@long\long
          \let\unexpanded\expandafter
            {\csname ekvc@\ekvc@set(#1)\endcsname\ekvc@tmp}%
          \def\unexpanded{\ekvc@tmp}####1%
            {%
              \unexpanded\expandafter{\csname ekvc@\ekvc@set(#1)\endcsname}%
                {####1}%
            }%
        \fi
        \def\unexpanded{\ekvc@initials}%
          {%
            \unexpanded\expandafter{\ekvc@initials}%
            \unexpanded\expandafter
              {\csname ekvc@splitmark@\the\ekvc@keycount\endcsname{#2}}%
          }%
      }%
    \ekvc@tmp
    \ekvlet\ekvc@set{#1}\ekvc@tmp
    \ekvc@helpers@needed
      {\expandafter\ekvc@setup@splitmacro\expandafter{\the\ekvc@keycount}}%
      {}%
  }
%    \end{macrocode}
% \end{macro}
%
% \begin{macro}[internal]{\ekvc@split@p@long}
%   The |long| prefix lets the internals |\ekvc@long| and |\ekvc@any@long| to
%   |\long| so that the key macro will be long.
%    \begin{macrocode}
\protected\def\ekvc@split@p@long
  {%
    \let\ekvc@long\long
    \let\ekvc@any@long\long
    \ekvc@SetupSplitKeys@c
  }
%    \end{macrocode}
% \end{macro}
%
% \begin{macro}[internal]{\ekvc@defarggobbler}
%   This is needed to define a macro with 1-9 parameters programmatically.
%   \LaTeX's \cs[no-index]{newcommand} does something similar for example.
%    \begin{macrocode}
\protected\def\ekvc@defarggobbler#1{\def\ekvc@tmp##1#1##2##{##1#1}}
%    \end{macrocode}
% \end{macro}
%
% \begin{macro}[internal]
%   {
%     \ekvc@setup@splitmacro,
%     \ekvc@split@1, \ekvc@split@2, \ekvc@split@3,
%     \ekvc@split@4, \ekvc@split@5, \ekvc@split@6,
%     \ekvc@split@7
%   }
%   Since the first few split macros are different from the others we manually
%   set those up now. All the others will be defined as needed (always
%   globally).  The split macros just read up until the correct split mark, move
%   that argument into a list and reinsert the rest, calling the next split
%   macro afterwards.
%    \begin{macrocode}
\begingroup
\edef\ekvc@tmp
  {%
    \long\gdef\unexpanded\expandafter{\csname ekvc@split@1\endcsname}%
        \unexpanded\expandafter{\csname ekvc@splitmark@1\endcsname}%
        ##1##2##3%
      {##3{##1}##2}%
    \long\gdef\unexpanded\expandafter{\csname ekvc@split@2\endcsname}%
        \unexpanded\expandafter{\csname ekvc@splitmark@1\endcsname}##1%
        \unexpanded\expandafter{\csname ekvc@splitmark@2\endcsname}##2%
        ##3##4%
      {##4{##1}{##2}##3}%
    \long\gdef\unexpanded\expandafter{\csname ekvc@split@3\endcsname}%
        \unexpanded\expandafter{\csname ekvc@splitmark@1\endcsname}##1%
        \unexpanded\expandafter{\csname ekvc@splitmark@2\endcsname}##2%
        \unexpanded\expandafter{\csname ekvc@splitmark@3\endcsname}##3%
        ##4##5%
      {##5{##1}{##2}{##3}##4}%
    \long\gdef\unexpanded\expandafter{\csname ekvc@split@4\endcsname}%
        \unexpanded\expandafter{\csname ekvc@splitmark@1\endcsname}##1%
        \unexpanded\expandafter{\csname ekvc@splitmark@2\endcsname}##2%
        \unexpanded\expandafter{\csname ekvc@splitmark@3\endcsname}##3%
        \unexpanded\expandafter{\csname ekvc@splitmark@4\endcsname}##4%
        ##5##6%
      {##6{##1}{##2}{##3}{##4}##5}%
    \long\gdef\unexpanded\expandafter{\csname ekvc@split@5\endcsname}%
        \unexpanded\expandafter{\csname ekvc@splitmark@1\endcsname}##1%
        \unexpanded\expandafter{\csname ekvc@splitmark@2\endcsname}##2%
        \unexpanded\expandafter{\csname ekvc@splitmark@3\endcsname}##3%
        \unexpanded\expandafter{\csname ekvc@splitmark@4\endcsname}##4%
        \unexpanded\expandafter{\csname ekvc@splitmark@5\endcsname}##5%
        ##6##7%
      {##7{##1}{##2}{##3}{##4}{##5}##6}%
    \long\gdef\unexpanded\expandafter{\csname ekvc@split@6\endcsname}%
        \unexpanded\expandafter{\csname ekvc@splitmark@1\endcsname}##1%
        \unexpanded\expandafter{\csname ekvc@splitmark@2\endcsname}##2%
        \unexpanded\expandafter{\csname ekvc@splitmark@3\endcsname}##3%
        \unexpanded\expandafter{\csname ekvc@splitmark@4\endcsname}##4%
        \unexpanded\expandafter{\csname ekvc@splitmark@5\endcsname}##5%
        \unexpanded\expandafter{\csname ekvc@splitmark@6\endcsname}##6%
        ##7##8%
      {##8{##1}{##2}{##3}{##4}{##5}{##6}##7}%
    \long\gdef\unexpanded\expandafter{\csname ekvc@split@7\endcsname}%
        \unexpanded\expandafter{\csname ekvc@splitmark@1\endcsname}##1%
        \unexpanded\expandafter{\csname ekvc@splitmark@2\endcsname}##2%
        \unexpanded\expandafter{\csname ekvc@splitmark@3\endcsname}##3%
        \unexpanded\expandafter{\csname ekvc@splitmark@4\endcsname}##4%
        \unexpanded\expandafter{\csname ekvc@splitmark@5\endcsname}##5%
        \unexpanded\expandafter{\csname ekvc@splitmark@6\endcsname}##6%
        \unexpanded\expandafter{\csname ekvc@splitmark@7\endcsname}##7%
        ##8##9%
      {##9{##1}{##2}{##3}{##4}{##5}{##6}{##7}##8}%
  }
\ekvc@tmp
\endgroup
\protected\def\ekvc@setup@splitmacro#1%
  {%
    \ekv@ifdefined{ekvc@split@#1}{}%
      {%
        \begingroup
          \edef\ekvc@tmp
            {%
              \long\gdef
                  \unexpanded\expandafter{\csname ekvc@split@#1\endcsname}%
                  ####1%
                  \unexpanded\expandafter{\csname ekvc@splitmark@#1\endcsname}%
                  ####2####3%
                {%
                  \unexpanded\expandafter
                    {\csname ekvc@split@\the\numexpr#1-1\relax\endcsname}%
                  ####1{{####2}####3}%
                }%
            }%
          \ekvc@tmp
        \endgroup
      }%
  }
%    \end{macrocode}
% \end{macro}
%
% \begin{macro}{\ekvcHashAndUse}
%   |\ekvcHashAndUse| works just like |\ekvcSplitAndUse|.
%    \begin{macrocode}
\protected\long\def\ekvcHashAndUse#1#2%
  {%
    \let\ekvc@helpers@needed\@firstoftwo
    \ekvc@ifdefined#1%
      {\ekvc@err@already@defined#1}%
      {\ekvcHashAndUse@#1{}{#2}}%
  }
%    \end{macrocode}
% \end{macro}
%
% \begin{macro}[internal]{\ekvcHashAndUse@}
%   This is more or less the same as |\ekvcSplitAndUse@|. Instead of an empty
%   group we place a marker after the initials, we don't use the sorting macros
%   of |split|, but instead pack all the values in one argument.
%    \begin{macrocode}
\protected\long\def\ekvcHashAndUse@#1#2#3%
  {%
    \edef\ekvc@set{\string#1}%
    \ekvc@SetupHashKeys{#3}%
    \ekvc@helpers@needed
      {%
        \ekvc@any@long\edef#1##1%
          {%
            \expandafter\ekvc@ekvset@pre@expander\expandafter{\ekvc@set}%
            \unexpanded{\ekvc@hash@pack@argument}%
            \unexpanded\expandafter{\ekvc@initials\ekvc@stop#2}%
          }%
      }%
      {%
        \ekvc@any@long\edef#1##1%
          {%
            \expandafter\ekvc@ekvset@pre@expander\expandafter{\ekvc@set}%
            \unexpanded{#2}%
            \unexpanded\expandafter{\ekvc@initials\ekvc@stop}%
          }%
      }%
  }
%    \end{macrocode}
% \end{macro}
%
% \begin{macro}{\ekvcHashAndForward}
%   |\ekvcHashAndForward| works just like |\ekvcSplitAndForward|.
%    \begin{macrocode}
\protected\long\def\ekvcHashAndForward#1#2#3%
  {%
    \let\ekvc@helpers@needed\@firstoftwo
    \ekvc@ifdefined#1%
      {\ekvc@err@already@defined#1}%
      {\ekvcHashAndUse@#1{{#2}}{#3}}%
  }
%    \end{macrocode}
% \end{macro}
%
% \begin{macro}{\ekvcHash}
%   |\ekvcHash| does the same as |\ekvcSplit|, but has the advantage of not
%   needing to count arguments, so the definition of the internal macro is a bit
%   more straight forward.
%    \begin{macrocode}
\protected\long\def\ekvcHash#1#2#3%
  {%
    \let\ekvc@helpers@needed\@secondoftwo
    \ekvc@ifdefined#1%
      {\ekvc@err@already@defined#1}%
      {%
        \expandafter
        \ekvcHashAndUse@\expandafter#1\csname ekvc@\string#1\endcsname{#2}%
        \ekvc@any@long\expandafter\def\csname ekvc@\string#1\endcsname
            ##1\ekvc@stop
          {#3}%
      }%
  }
%    \end{macrocode}
% \end{macro}
%
% \begin{macro}[internal]{\ekvc@hash@pack@argument}
%   All this macro does is pack the values into one argument and forward that to
%   the next macro.
%    \begin{macrocode}
\long\def\ekvc@hash@pack@argument#1\ekvc@stop#2{#2{#1}}
%    \end{macrocode}
% \end{macro}
%
% \begin{macro}[internal]
%   {\ekvc@SetupHashKeys,\ekvc@SetupHashKeys@a,\ekvc@SetupHashKeys@b}
%   This should look awfully familiar as well, since it's just the same as for
%   the split keys with a few other names here and there.
%    \begin{macrocode}
\protected\long\def\ekvc@SetupHashKeys#1%
  {%
    \let\ekvc@any@long\ekv@empty
    \let\ekvc@initials\ekv@empty
    \ekvparse\ekvc@err@value@required\ekvc@SetupHashKeys@a{#1}%
  }
\protected\def\ekvc@SetupHashKeys@a#1%
  {%
    \let\ekvc@long\ekv@empty
    \ekvc@ifspace{#1}%
      {\ekvc@SetupHashKeys@b#1\ekvc@stop}%
      {\ekvc@SetupHashKeys@c{#1}}%
  }
\protected\def\ekvc@SetupHashKeys@b#1 #2\ekvc@stop
  {%
    \ekv@ifdefined{ekvc@hash@p@#1}%
      {\csname ekvc@hash@p@#1\endcsname{#2}}%
      {\ekvc@SetupHashKeys@c{#1 #2}}%
  }
%    \end{macrocode}
%   Yes, even the defining macro looks awfully familiar. Instead of numbered we
%   have named marks. Still the key macros grab everything up to their
%   respective mark and reorder the arguments. The same quirk is applied for
%   short keys. And instead of the |\ekvc@setup@splitmacro| we use
%   |\ekvc@setup@hashmacro|.
%    \begin{macrocode}
\protected\long\def\ekvc@SetupHashKeys@c#1#2%
  {%
    \begingroup
    \edef\ekvc@tmp
      {%
        \endgroup
        \long\def\unexpanded{\ekvc@tmp}####1####2%
            \unexpanded\expandafter{\csname ekvc@hashmark@#1\endcsname}####3%
          {%
            ####2%
            \unexpanded\expandafter{\csname ekvc@hashmark@#1\endcsname}{####1}%
          }%
        \unless\ifx\ekvc@long\long
          \let\unexpanded\expandafter
            {\csname ekvc@\ekvc@set(#1)\endcsname\ekvc@tmp}%
          \def\unexpanded{\ekvc@tmp}####1%
            {%
              \unexpanded\expandafter{\csname ekvc@\ekvc@set(#1)\endcsname}%
                {####1}%
            }%
        \fi
        \def\unexpanded{\ekvc@initials}%
          {%
            \unexpanded\expandafter{\ekvc@initials}%
            \unexpanded\expandafter{\csname ekvc@hashmark@#1\endcsname{#2}}%
          }%
      }%
    \ekvc@tmp
    \ekvlet\ekvc@set{#1}\ekvc@tmp
    \ekvc@setup@hashmacro{#1}%
  }
%    \end{macrocode}
% \end{macro}
%
% \begin{macro}[internal]{\ekvc@hash@p@long}
%   Nothing astonishing here either.
%    \begin{macrocode}
\protected\def\ekvc@hash@p@long
  {%
    \let\ekvc@long\long
    \let\ekvc@any@long\long
    \ekvc@SetupHashKeys@c
  }
%    \end{macrocode}
% \end{macro}
%
% \begin{macro}[internal]{\ekvc@setup@hashmacro}
%   The safe hash macros will be executed inside of an |\unexpanded| expansion
%   context, so they have to insert braces for that once they are done.
%   Most of the tests which have to be executed will already be done, but we
%   have to play safe if the hash doesn't show up in the hash list. Therefore we
%   use some |\ekvc@mark|s and |\ekvc@stop| to throw errors if the hash isn't
%   found in the right place. The fast variants have an easier life and just
%   return the correct value.
%    \begin{macrocode}
\protected\def\ekvc@setup@hashmacro#1%
  {%
    \ekv@ifdefined{ekvc@fasthash@#1}{}%
      {%
        \begingroup
          \edef\ekvc@tmp
            {%
              \long\gdef
                \unexpanded\expandafter{\csname ekvc@fasthash@#1\endcsname}%
                  ####1%
                  \unexpanded\expandafter{\csname ekvc@hashmark@#1\endcsname}%
                  ####2####3\unexpanded{\ekvc@stop}%
                {####2}%
              \long\gdef
                \unexpanded\expandafter{\csname ekvc@safehash@#1\endcsname}%
                  ####1%
                {%
                  \unexpanded\expandafter{\csname ekvc@@safehash@#1\endcsname}%
                  ####1\unexpanded{\ekvc@mark}{}%
                  \unexpanded\expandafter
                    {%
                      \csname ekvc@hashmark@#1\endcsname{}%
                      \ekvc@mark{\ekvc@err@missing@hash{#1}}\ekvc@stop
                    }%
                }%
              \long\gdef
                \unexpanded\expandafter{\csname ekvc@@safehash@#1\endcsname}%
                  ####1%
                  \unexpanded\expandafter{\csname ekvc@hashmark@#1\endcsname}%
                  ####2####3\unexpanded{\ekvc@mark}####4####5%
                  \unexpanded{\ekvc@stop}%
                {%
                  ####4{####2}%
                }%
              \long\gdef\unexpanded\expandafter
                {\csname ekvc@fastsplithash@#1\endcsname}%
                  ####1%
                  \unexpanded\expandafter{\csname ekvc@hashmark@#1\endcsname}%
                  ####2####3\unexpanded{\ekvc@stop}####4%
                {%
                  ####4{####1####3}{####2}%
                }%
              \long\gdef\unexpanded\expandafter
                {\csname ekvc@safesplithash@#1\endcsname}####1%
                {%
                  \unexpanded\expandafter
                    {\csname ekvc@@safesplithash@#1\endcsname}%
                  ####1\unexpanded{\ekvc@mark\ekvc@safe@found@hash}%
                  \unexpanded\expandafter
                    {%
                      \csname ekvc@hashmark@#1\endcsname{}%
                      \ekvc@mark{\ekvc@err@missing@hash{#1}\ekvc@safe@no@hash}%
                      \ekvc@stop
                    }%
                }%
              \long\gdef\unexpanded\expandafter
                {\csname ekvc@@safesplithash@#1\endcsname}%
                  ####1%
                  \unexpanded\expandafter{\csname ekvc@hashmark@#1\endcsname}%
                  ####2####3\unexpanded{\ekvc@mark}####4####5%
                  \unexpanded{\ekvc@stop}%
                {%
                  ####4{####2}####1####3\unexpanded{\ekvc@stop}%
                }%
            }%
          \ekvc@tmp
        \endgroup
      }%
  }
%    \end{macrocode}
% \end{macro}
%
% \begin{macro}{\ekvcValue}
%   All this does is a few consistency checks on the first argument (not empty,
%   hash macro exists) and then call that hash-grabbing macro that will also
%   test whether the hash is inside of |#2| or not.
%    \begin{macrocode}
\long\def\ekvcValue#1%
  {%
    \unexpanded
    \ekv@ifdefined{ekvc@safehash@#1}%
      {\csname ekvc@safehash@#1\endcsname}%
      {\ekvc@err@unknown@hash{#1}\@firstoftwo{{}}}%
  }
%    \end{macrocode}
% \end{macro}
%
% \begin{macro}{\ekvcValueFast}
%   To be as fast as possible, this doesn't test for anything, assuming the user
%   knows best.
%    \begin{macrocode}
\long\def\ekvcValueFast#1#2{\csname ekvc@fasthash@#1\endcsname#2\ekvc@stop}
%    \end{macrocode}
% \end{macro}
%
% \begin{macro}{\ekvcValueSplit,\ekvcValueSplit@recover}
%   This splits off a single value.
%    \begin{macrocode}
\long\def\ekvcValueSplit#1%
  {%
    \ekv@ifdefined{ekvc@safesplithash@#1}%
      {\csname ekvc@safesplithash@#1\endcsname}%
      {\ekvc@err@unknown@hash{#1}\ekvcValueSplit@recover}%
  }
\long\def\ekvcValueSplit@recover#1#2%
  {%
    #2{#1}{}%
  }
%    \end{macrocode}
% \end{macro}
%
% \begin{macro}[internal]{\ekvc@safe@found@hash, \ekvc@safe@no@hash}
%    \begin{macrocode}
\long\def\ekvc@safe@found@hash#1#2\ekvc@stop#3%
  {%
    #3{#2}{#1}%
  }
\long\def\ekvc@safe@no@hash#1#2\ekvc@mark\ekvc@safe@found@hash\ekvc@stop#3%
  {%
    #3{#2}{}%
  }
%    \end{macrocode}
% \end{macro}
%
% \begin{macro}{\ekvcValueSplitFast}
%   Again a fast approach which doesn't provide too many safety measurements.
%   This needs to build the hash function and expand it before passing the
%   results to the next control sequence. The first step only builds the control
%   sequence.
%    \begin{macrocode}
\long\def\ekvcValueSplitFast#1#2%
  {%
    \csname ekvc@fastsplithash@#1\endcsname#2\ekvc@stop
  }
%    \end{macrocode}
% \end{macro}
%
% \begin{macro}[internal]
%   {
%     \ekvc@safehash@,\ekvc@fasthash@,
%     \ekvc@safesplithash@,\ekvc@fastsplithash@
%   }
%   At least in the empty hash case we can provide a meaningful error message
%   without affecting performance by just defining the macro that would be build
%   in that case. There is of course a downside to this, the error will not be
%   thrown by |\ekvcValueFast| in three expansion steps. The safe hash variant
%   has to also stop the |\unexpanded| expansion.
%    \begin{macrocode}
\long\def\ekvc@safehash@#1{\ekvc@err@empty@hash{}}
\long\def\ekvc@fasthash@#1\ekvc@stop{\ekvc@err@empty@hash}
\long\def\ekvc@safesplithash@#1#2{\ekvc@err@empty@hash#2{#1}{}}
\long\def\ekvc@fastsplithash@#1\ekvc@stop#2{\ekvc@err@empty@hash#2{#1}{}}
%    \end{macrocode}
% \end{macro}
%
% \begin{macro}{\ekvcSecondaryKeys}
%   The secondary keys are defined pretty similar to the way the originals are,
%   but here we also introduce some key types (those have a |@t@| in their name)
%   additionally to the prefixes.
%    \begin{macrocode}
\protected\long\def\ekvcSecondaryKeys#1#2%
  {%
    \edef\ekvc@set{\string#1}%
    \ekvparse\ekvc@err@value@required\ekvcSecondaryKeys@a{#2}%
  }
\protected\def\ekvcSecondaryKeys@a#1%
  {%
    \let\ekvc@long\ekv@empty
    \ekvc@ifspace{#1}%
      {\ekvcSecondaryKeys@b#1\ekvc@stop}%
      {\ekvc@err@missing@type{#1}\@gobble}%
  }
\protected\def\ekvcSecondaryKeys@b#1 #2\ekvc@stop
  {%
    \ekv@ifdefined{ekvc@p@#1}%
      {\csname ekvc@p@#1\endcsname}%
      {%
        \ekv@ifdefined{ekvc@t@#1}%
          {\csname ekvc@t@#1\endcsname}%
          {\ekvc@err@unknown@keytype{#1}\@firstoftwo\@gobble}%
      }%
      {#2}%
  }
%    \end{macrocode}
% \end{macro}
%
% \begin{macro}{\ekvcChange}
%   This can be used to change the defaults of an \expkvc\ defined macro. It
%   checks whether there is a set with the correct name and that the macro is
%   defined. If both is true the real work is done by |\ekvc@change|.
%    \begin{macrocode}
\protected\long\def\ekvcChange#1%
  {%
    \ekvifdefinedset{\string#1}%
      {%
        \ekvc@ifdefined#1%
          {\ekvc@change#1}%
          {\ekvc@err@no@key@macro#1\@gobble}%
      }%
      {\ekvc@err@no@key@macro#1\@gobble}%
  }
%    \end{macrocode}
% \end{macro}
%
% \begin{macro}[internal]
%   {\ekvc@change,\ekvc@change@a,\ekvc@change@b,\ekvc@change@c}
%   First we need to see whether the macro is currently |\long|. For this we get
%   the meaning and will parse it. |#1| is the macro name in which we want to
%   change the defaults.
%    \begin{macrocode}
\protected\def\ekvc@change#1%
  {\expandafter\ekvc@change@a\meaning#1\ekv@stop#1}
%    \end{macrocode}
%   A temporary definition to get the stringified |macro:|. |##1| will be the
%   list of prefixes, we don't care for the exact contents of |##2| and |##3|.
%    \begin{macrocode}
\def\ekvc@change@a#1%
  {%
    \protected\def\ekvc@change@a##1#1##2->##3\ekv@stop
      {%
        \ekvc@change@iflong{##1}%
          {\ekvc@change@b{}}%
          {\ekvc@change@b{\long}}%
      }%
  }
\expandafter\ekvc@change@a\expandafter{\detokenize{macro:}}
%    \end{macrocode}
%   Next we expand the macro once to get its contents (including the current
%   default values with their markers). |#1| is either |\long| or empty, |#2| is
%   the macro.
%    \begin{macrocode}
\protected\def\ekvc@change@b#1#2%
  {\expandafter\ekvc@change@c\expandafter{#2{##1}}{#1}#2}
%    \end{macrocode}
%   Here we place an unbalanced closing brace after the expansion of the macro.
%   Then we just parse the \kv-list with |\ekvset|, that will exchange the
%   values behind the markers. Once those are changed we give control to
%   |\ekvc@change@d|. The |\ekvset| step might horribly fail if the user defined
%   some keys that don't behave nice. |#1| is the expansion of the macro, |#2|
%   is either |\long| or empty, |#3| is the macro, and |#4| is the \kv-list
%   containing the new defaults.
%    \begin{macrocode}
\ekv@exparg{\protected\long\def\ekvc@change@c#1#2#3#4}%
  {%
    \expandafter\iffalse\expandafter{\expandafter\fi
      \ekvset{\string#3}{#4}%
      \ekvc@change@d{#2}{#3}%
      #1%
    }%
  }
%    \end{macrocode}
%   The final step needs to put an unbalanced opening brace after |\def|. We do
%   that with the help of a temporary macro which stores everything necessary
%   for |\def| and expand an |\iffalse}\fi| construct to leave a single
%   opening brace. |#1| will be either empty or |\long| and |#2| is the macro.
%   Each of the macros defined with \expkvc\ takes exactly one parameter, so we
%   put that here as |####1|.
%    \begin{macrocode}
\protected\def\ekvc@change@d#1#2%
  {%
    \def\ekvc@tmp{#1\def#2####1}%
    \expandafter\ekvc@tmp\expandafter{\iffalse}\fi
  }
%    \end{macrocode}
% \end{macro}
%
% \begin{macro}[internal]{\ekvc@change@iflong,\ekvc@change@iflong@}
%   Checking whether a string contains the string representation of |\long| can
%   be done by gobbling everything up to the first |\long| and checking whether
%   the result is completely empty. We need a temporary macro to get the result
%   of |\string\long| inside the definitions.
%    \begin{macrocode}
\def\ekvc@change@iflong#1%
  {%
    \protected\def\ekvc@change@iflong##1%
      {\expandafter\ekv@ifempty\expandafter{\ekvc@change@iflong@##1#1}}%
    \def\ekvc@change@iflong@##1#1{}
  }
\expandafter\ekvc@change@iflong\expandafter{\string\long}
%    \end{macrocode}
% \end{macro}
%
% \begin{macro}{\ekvcPass}
%   This macro can be used to pass a value to a key of some macro (this way
%   more complicated key codes are possible that in the end pass processed
%   values on to some macro). The implemantation is pretty straight forward.
%    \begin{macrocode}
\long\def\ekvcPass#1#2%
  {%
    \ekvifdefined{\string#1}{#2}%
      {\csname\ekv@name{\string#1}{#2}\endcsname}%
      {\ekvc@err@unknown@key@or@macro{#1}{#2}\@gobble}%
  }
%    \end{macrocode}
% \end{macro}
%
%
% \subsubsection{Secondary Key Types}
%
% \begin{macro}[internal]{\ekvc@p@long,\ekvc@after@ptype}
%   The prefixes are pretty straight forward again. Just set |\ekvc@long| and
%   forward to the |@t@| type.
%    \begin{macrocode}
\protected\def\ekvc@p@long#1%
  {%
    \ekvc@ifspace{#1}%
      {%
        \let\ekvc@long\long
        \ekvc@after@ptype#1\ekvc@stop
      }%
      {\ekvc@err@missing@type{long #1}\@gobble}%
  }
\protected\def\ekvc@after@ptype#1 #2\ekvc@stop
  {%
    \ekv@ifdefined{ekvc@t@#1}%
      {\csname ekvc@t@#1\endcsname{#2}}%
      {\ekvc@err@unknown@keytype{#1}\@gobble}%
  }
%    \end{macrocode}
% \end{macro}
%
% \begin{macro}[internal]
%   {
%     \ekvc@t@meta, \ekvc@t@nmeta,
%     \ekvc@type@meta, \ekvc@type@meta@a, \ekvc@type@meta@b
%   }
%   The |meta| and |nmeta| key types use a nested |\ekvset| to set other keys in
%   the same macro's \set.
%    \begin{macrocode}
\protected\def\ekvc@t@meta
  {%
    \edef\ekvc@tmp{\ekvc@set}%
    \expandafter\ekvc@type@meta\expandafter{\ekvc@tmp}\ekvc@long{##1}\ekvlet
  }
\protected\def\ekvc@t@nmeta#1%
  {%
    \ekvc@assert@not@long{nmeta #1}%
    \edef\ekvc@tmp{\ekvc@set}%
    \expandafter\ekvc@type@meta\expandafter{\ekvc@tmp}{}{}\ekvletNoVal{#1}%
  }
\protected\long\def\ekvc@type@meta#1#2#3#4#5#6%
  {%
    \expandafter\ekvc@type@meta@a\expandafter{\ekvset{#1}{#6}}{#2}{#3}%
    #4\ekvc@set{#5}\ekvc@tmp
  }
\protected\def\ekvc@type@meta@a
  {%
    \expandafter\ekvc@type@meta@b\expandafter
  }
\protected\long\def\ekvc@type@meta@b#1#2#3%
  {%
    #2\def\ekvc@tmp#3{#1}%
  }
%    \end{macrocode}
% \end{macro}
%
% \begin{macro}[internal]{\ekvc@t@alias}
%   |alias| just checks whether there is a key and/or |NoVal| key defined with
%   the target name and |\let| the key to those.
%    \begin{macrocode}
\protected\def\ekvc@t@alias#1#2%
  {%
    \ekvc@assert@not@long{alias #1}%
    \let\ekvc@tmp\@firstofone
    \ekvifdefined\ekvc@set{#2}%
      {%
        \ekvletkv\ekvc@set{#1}\ekvc@set{#2}%
        \let\ekvc@tmp\@gobble
      }%
      {}%
    \ekvifdefinedNoVal\ekvc@set{#2}%
      {%
        \ekvletkvNoVal\ekvc@set{#1}\ekvc@set{#2}%
        \let\ekvc@tmp\@gobble
      }%
      {}%
    \ekvc@tmp{\ekvc@err@unknown@key{#2}}%
  }
%    \end{macrocode}
% \end{macro}
%
% \begin{macro}[internal]{\ekvc@t@default}
%   The |default| key can be used to set a |NoVal| key for an existing key. It
%   will just pass the \val\ to the key macro of that other key.
%    \begin{macrocode}
\protected\long\def\ekvc@t@default#1#2%
  {%
    \ekvifdefined\ekvc@set{#1}%
      {%
        \ekvc@assert@not@long{default #1}%
        \edef\ekvc@tmp
          {%
            \unexpanded\expandafter
              {\csname\ekv@name\ekvc@set{#1}\endcsname{#2}}%
          }%
        \ekvletNoVal\ekvc@set{#1}\ekvc@tmp
      }%
      {\ekvc@err@unknown@key{#1}}%
  }
%    \end{macrocode}
% \end{macro}
%
% \begin{macro}[internal]{\ekvc@t@flag-bool}
%    \begin{macrocode}
\protected\expandafter\def\csname ekvc@t@flag-bool\endcsname#1#2%
  {%
    \ekvc@assert@not@long{flag-bool #1}%
    \unless\ifdefined#2\ekvcFlagNew#2\fi
    \ekvdef\ekvc@set{#1}%
      {%
        \ekv@ifdefined{ekvc@flag@set@##1}%
          {%
            \csname ekvc@flag@set@##1\expandafter\endcsname
              \ekvcFlagHeight#2\ekv@stop#2%
          }%
          {\ekvc@err@invalid@bool{##1}}%
      }%
  }
%    \end{macrocode}
% \end{macro}
%
% \begin{macro}[internal]
%   {\ekvc@t@flag-true,\ekvc@t@flag-false,\ekvc@t@flag-raise,\ekvc@type@flag}
%    \begin{macrocode}
\protected\def\ekvc@type@flag#1#2#3#4%
  {%
    \ekvc@assert@not@long{flag-#1 #3}%
    \unless\ifdefined#4\ekvcFlagNew#4\fi
    \ekv@exparg{\ekvdefNoVal\ekvc@set{#3}}{#2#4}%
  }
\protected\expandafter\def\csname ekvc@t@flag-true\endcsname
  {\ekvc@type@flag{true}\ekvcFlagSetTrue}
\protected\expandafter\def\csname ekvc@t@flag-false\endcsname
  {\ekvc@type@flag{false}\ekvcFlagSetFalse}
\protected\expandafter\def\csname ekvc@t@flag-raise\endcsname
  {\ekvc@type@flag{raise}\ekvcFlagRaise}
%    \end{macrocode}
% \end{macro}
%
%
% \subsubsection{Flags}
%
% The basic idea of flags is to store information by the fact that \TeX\
% expandably assigns the meaning |\relax| to undefined control sequences which
% were built with |\csname|. This mechanism is borrowed from \pkg{expl3}.
%
% \begin{macro}[internal]{\ekvc@flag@name,\ekvc@flag@namescheme}
%   Flags follow a simple naming scheme which we define here. |\ekvc@flag@name|
%   will store the name of an internal function that is used to build names of
%   the second naming scheme defined by |\ekvc@flag@namescheme|.
%    \begin{macrocode}
\def\ekvc@flag@name{ekvcf\string}
\def\ekvc@flag@namescheme#1#2{ekvch#2#1}
%    \end{macrocode}
% \end{macro}
%
% \begin{macro}{\ekvcFlagHeight}
%   For semantic reasons we use |\number| with another name.
%    \begin{macrocode}
\let\ekvcFlagHeight\number
%    \end{macrocode}
% \end{macro}
%
% \begin{macro}{\ekvcFlagNew}
%   This macro defines a new flag. It stores the function build with the
%   |\ekvc@flag@name| naming scheme after the internal function
%   |\ekvc@flag@height| that'll determine the current flag height. It'll also
%   define the macro named via |\ekvc@flag@name| to build names according to
%   |\ekvc@flag@namescheme|.
%    \begin{macrocode}
\protected\def\ekvcFlagNew#1%
  {%
    \edef#1%
      {%
        \unexpanded{\ekvc@flag@height}%
          \unexpanded\expandafter{\csname\ekvc@flag@name#1\endcsname}%
      }%
    \ekv@expargtwice
      {\expandafter\def\csname\ekvc@flag@name#1\endcsname##1}%
      {\expandafter\ekvc@flag@namescheme\expandafter{\string#1}{##1}}%
  }
%    \end{macrocode}
% \end{macro}
%
% \begin{macro}[internal]{\ekvc@flag@height,\ekvc@flag@height@}
%   This macro gets the height of a flag by a simple loop. The first loop
%   iteration differs a bit from the following in that it doesn't have to get
%   the current iteration count. The space at the end of |\ekvc@flag@height|
%   ends the |\number| evaluation.
%    \begin{macrocode}
\def\ekvc@flag@height#1%
  {%
    \ifcsname#10\endcsname
      \ekvc@flag@height@1\ekv@stop#1%
    \fi
    \@firstofone{0} % leave this space
  }
\def\ekvc@flag@height@#1\ekv@stop#2\fi\@firstofone#3%
  {%
    \fi
    \ifcsname#2{#1}\endcsname
      \expandafter\ekvc@flag@height@\the\numexpr#1+1\relax\ekv@stop#2%
    \fi
    \@firstofone{#1}%
  }
%    \end{macrocode}
% \end{macro}
%
% \begin{macro}{\ekvcFlagRaise}
%   Raising a flag simply means letting the |\ekvc@flag@namescheme| macro for
%   the current height to relax. The result of raising a flag is that its height
%   is bigger by $1$.
%    \begin{macrocode}
\ekv@exparg{\def\ekvcFlagRaise#1}%
  {%
    \expandafter\expandafter\expandafter\@gobble\expandafter
      \csname\ekvc@flag@namescheme{\string#1}{\ekvcFlagHeight#1}\endcsname
  }
%    \end{macrocode}
% \end{macro}
%
% \begin{macro}{\ekvcFlagSetTrue,\ekvcFlagSetFalse}
% \begin{macro}[internal]{\ekvc@flag@set@true,\ekvc@flag@set@false}
%   A flag is considered true if its current height is odd, and as false if it
%   is even. Therefore |\ekvcFlagSetTrue| and |\ekvcFlagSetFalse| only need to
%   raise the flag if the opposing boolean value is the current one.
%    \begin{macrocode}
\def\ekvcFlagSetTrue#1%
  {\expandafter\ekvc@flag@set@true\ekvcFlagHeight#1\ekv@stop#1}
\def\ekvcFlagSetFalse#1%
  {\expandafter\ekvc@flag@set@false\ekvcFlagHeight#1\ekv@stop#1}
%    \end{macrocode}
%   We can expand |\ekvc@flag@namescheme| at definition time here, which is why
%   we're using a temporary definition to set up |\ekvc@flag@set@true| and
%   |\ekvc@flag@set@false|.
%    \begin{macrocode}
\def\ekvc@flag@set@true#1%
  {%
    \def\ekvc@flag@set@true##1\ekv@stop##2%
      {%
        \ifodd##1
          \ekv@fi@gobble
        \fi
        \@firstofone{\expandafter\@gobble\csname#1\endcsname}%
      }%
    \def\ekvc@flag@set@false##1\ekv@stop##2%
      {%
        \ifodd##1
          \ekv@fi@firstofone
        \fi
        \@gobble{\expandafter\@gobble\csname#1\endcsname}%
      }%
  }
\expandafter\ekvc@flag@set@true\expandafter
  {\ekvc@flag@namescheme{\string#2}{#1}}
%    \end{macrocode}
% \end{macro}
% \end{macro}
%
% \begin{macro}{\ekvcFlagIf}
%   As already explained, truthiness is defined as a flag's height is odd, so we
%   just branch accordingly here.
%    \begin{macrocode}
\def\ekvcFlagIf#1%
  {%
    \ifodd#1%
      \ekv@fi@firstoftwo
    \fi
    \@secondoftwo
  }
%    \end{macrocode}
% \end{macro}
%
% \begin{macro}{\ekvcFlagIfRaised}
%   This macro uses flags as a switch, if a flag's current height is bigger than
%   $0$ this test yields true.
%    \begin{macrocode}
\ekv@exparg{\def\ekvcFlagIfRaised#1}%
  {%
    \expandafter\ifcsname\ekvc@flag@namescheme{\string#1}0\endcsname
      \ekv@fi@firstoftwo
    \fi
    \@secondoftwo
  }
%    \end{macrocode}
% \end{macro}
%
% \begin{macro}{\ekvcFlagReset}
% \begin{macro}[internal]{\ekvc@flag@reset,\ekvc@flag@reset@}
%   Resetting works by locally letting all the defined internal macros named
%   after |\ekvc@flag@namescheme| to undefined.
%    \begin{macrocode}
\protected\def\ekvcFlagReset#1%
  {\expandafter\ekvc@flag@reset\csname\ekvc@flag@name#1\endcsname}
\protected\def\ekvc@flag@reset#1%
  {%
    \ifcsname#10\endcsname
      \expandafter\let\csname#10\endcsname\ekvc@undefined
      \ekvc@flag@reset@1\ekv@stop#1%
    \fi
  }
\protected\def\ekvc@flag@reset@#1\ekv@stop#2\fi
  {%
    \fi
    \ifcsname#2{#1}\endcsname
      \expandafter\let\csname#2{#1}\endcsname\ekvc@undefined
      \expandafter\ekvc@flag@reset@\the\numexpr#1+1\relax\ekv@stop#2%
    \fi
  }
%    \end{macrocode}
% \end{macro}
% \end{macro}
%
% \begin{macro}{\ekvcFlagGetHeight}
% \begin{macro}[internal]{\ekvc@flag@get@height@single}
%   These are just small helpers, first getting the height of the flag and then
%   passing it on to the user supplied code.
%    \begin{macrocode}
\def\ekvcFlagGetHeight#1%
  {\expandafter\ekvc@flag@get@height@single\ekvcFlagHeight#1\ekv@stop}
\long\def\ekvc@flag@get@height@single#1\ekv@stop#2{#2{#1}}
%    \end{macrocode}
% \end{macro}
% \end{macro}
%
% \begin{macro}{\ekvcFlagGetHeights}
% \begin{macro}[internal]
%   {\ekvc@flag@get@heights,\ekvc@flag@get@heights@,\ekvc@flag@get@heights@done}
%   This works by a simple loop that stops at |\ekv@stop|. As long as that
%   marker isn't hit, get the next flags height and put it into a list after
%   |\ekv@stop|. |\ekvc@flag@get@heights@| uses the same marker name for the
%   end of the height, which shouldn't clash in any case. Once we're done we
%   remove the remainder of the current iteration and leave the user supplied
%   code in the input stream with all the flags' heights as a single argument.
%    \begin{macrocode}
\def\ekvcFlagGetHeights#1%
  {%
    \ekvc@flag@get@heights#1\ekv@stop{}%
  }
\def\ekvc@flag@get@heights#1%
  {%
    \ekv@gobbleto@stop#1\ekvc@flag@get@heights@done\ekv@stop
    \expandafter\ekvc@flag@get@heights@\ekvcFlagHeight#1\ekv@stop
  }
\def\ekvc@flag@get@heights@#1\ekv@stop#2\ekv@stop#3%
  {\ekvc@flag@get@heights#2\ekv@stop{#3{#1}}}
\long\def\ekvc@flag@get@heights@done
    \ekv@stop
    \expandafter\ekvc@flag@get@heights@\ekvcFlagHeight\ekv@stop\ekv@stop#1#2%
  {#2{#1}}
%    \end{macrocode}
% \end{macro}
% \end{macro}
%
%
% \subsubsection{Helper Macros}
%
% \begin{macro}[internal]{\ekvc@ifspace,\ekvc@ifspace@}
% A test which can be reduced to an if-empty by gobbling everything up to the
% first space.
%    \begin{macrocode}
\long\def\ekvc@ifspace#1%
  {%
    \ekvc@ifspace@#1 \ekv@ifempty@B
      \ekv@ifempty@false\ekv@ifempty@A\ekv@ifempty@B\@firstoftwo
  }
\long\def\ekvc@ifspace@#1 % keep this space
  {%
    \ekv@ifempty@\ekv@ifempty@A
  }
%    \end{macrocode}
% \end{macro}
%
%
% \subsubsection{Assertions}
% \begin{macro}[internal]{\ekvc@assert@not@long}
%   Some keys don't want to be |long| and we have to educate the user, so let's
%   throw an error if someone wanted these to be long.
%    \begin{macrocode}
\long\def\ekvc@assert@not@long#1{\ifx\ekvc@long\long\ekvc@err@no@long{#1}\fi}
%    \end{macrocode}
% \end{macro}
%
%
% \subsubsection{Messages}
%
% \begin{macro}[internal]
%   {
%     \ekvc@err@toomany,\ekvc@err@value@required,\ekvc@err@missing@type,
%     \ekvc@err@already@defined,\ekvc@err@no@key@macro
%   }
%   Boring unexpandable error messages.
%    \begin{macrocode}
\protected\def\ekvc@err@toomany#1%
  {%
    \errmessage{expkv-cs Error: Too many keys for macro `\string#1'}%
  }
\protected\def\ekvc@err@value@required#1%
  {%
    \errmessage{expkv-cs Error: Missing value for key `\unexpanded{#1}'}%
  }
\protected\def\ekvc@err@missing@type#1%
  {%
    \errmessage
      {expkv-cs Error: Missing type for secondary key `\unexpanded{#1}'}%
  }
\protected\def\ekvc@err@no@long#1%
  {%
    \errmessage
      {expkv-cs Error: prefix `long' not accepted for `\unexpanded{#1}'}%
  }
\protected\def\ekvc@err@already@defined#1%
  {%
    \errmessage{expkv-cs Error: Macro `\string#1' already defined}%
  }
\protected\def\ekvc@err@unknown@keytype#1%
  {%
    \errmessage{expkv-cs Error: Unknown key type `\unexpanded{#1}'}%
  }
\protected\def\ekvc@err@unknown@key#1%
  {%
    \errmessage
      {expkv-cs Error: Unknown key `\unexpanded{#1}' for macro `\ekvc@set'}%
  }
\long\def\ekvc@err@no@key@macro#1%
  {\errmessage{expkv-cs Error: \string#1 is no key=val macro}}
%    \end{macrocode}
% \end{macro}
%
% \begin{macro}[internal]{\ekvc@err}
%   We need a way to throw error messages expandably in some contexts.
%    \begin{macrocode}
\ekv@exparg{\long\def\ekvc@err#1}{\ekverr{expkv-cs}{#1}}
%    \end{macrocode}
% \end{macro}
%
% \begin{macro}[internal]
%   {
%     \ekvc@err@unknown@hash,\ekvc@err@empty@hash,\ekvc@err@missing@hash,
%     \ekvc@err@invalid@bool
%   }
%   And here are the expandable error messages.
%    \begin{macrocode}
\long\def\ekvc@err@unknown@hash#1{\ekvc@err{unknown hash `#1'}}
\long\def\ekvc@err@missing@hash#1{\ekvc@err{hash `#1' not found}}
\long\def\ekvc@err@empty@hash{\ekvc@err{empty hash}}
\def\ekvc@err@invalid@bool#1{\ekvc@err{invalid boolean value `#1'}}
\long\def\ekvc@err@unknown@key@or@macro#1#2%
  {\ekvc@err{unknown key `#2' for macro #1}}
%    \end{macrocode}
% \end{macro}
%
%
% Now everything that's left is to reset the category code of |@|.
%    \begin{macrocode}
\catcode`\@=\ekvc@tmp
%    \end{macrocode}
%
% \gobbledocstriptag
%</tex>
%^^A=<<
%
%^^A=<<
%
% \end{implementation}^^A=<<
%
% \clearpage
% \PrintIndex
%
\endinput
%
^^A vim: ft=tex fdm=marker fmr=>>=,=<<

\ProvidesPackage{expkv-cs}%
  [%
    \ekvcDate\space v\ekvcVersion\space
    define expandable key=val macros using expkv%
  ]
%    \end{macrocode}
% \gobbledocstriptag
%</pkg>
%^^A=<<
%^^A main file >>=
% \subsection{The Generic Code}
% The rest of this implementation will be the generic code.
% \gobbledocstriptag
%<*tex>
%
% Load \expkv\ if the package didn't already do so -- since \expkv\ has
% safeguards against being loaded twice this does no harm and the overhead
% isn't that big. Also we reuse some of the internals of \expkv\ to save us from
% retyping them.
%    \begin{macrocode}
\input expkv
%    \end{macrocode}
%
% We make sure that \file{expkv-cs.tex} is only input once:
%    \begin{macrocode}
\expandafter\ifx\csname ekvcVersion\endcsname\relax
\else
  \expandafter\endinput
\fi
%    \end{macrocode}
%
% \begin{macro}{\ekvcVersion,\ekvcDate}
% We're on our first input, so lets store the version and date in a macro.
%    \begin{macrocode}
\def\ekvcVersion{0.10}
\def\ekvcDate{2021-06-03}
%    \end{macrocode}
% \end{macro}
%
% If the \LaTeX\ format is loaded we want to be a good file and report back who
% we are, for this the package will have defined |\ekvc@tmp| to use
% |\ProvidesFile|, else this will expand to a |\relax| and do no harm.
%    \begin{macrocode}
\csname ekvc@tmp\endcsname
%    \end{macrocode}
%
% Store the category code of |@| to later be able to reset it and change it to
% 11 for now.
%    \begin{macrocode}
\expandafter\chardef\csname ekvc@tmp\endcsname=\catcode`\@
\catcode`\@=11
%    \end{macrocode}
% |\ekvc@tmp| will be reused later, but we don't need it to ever store
% information long-term after \expkvc\ was initialized.
%
% \begin{macro}[internal]{\ekvc@keycount}
%   We'll need to keep count how many keys must be defined for each macro in the
%   |split| variants.
%    \begin{macrocode}
\newcount\ekvc@keycount
%    \end{macrocode}
% \end{macro}
%
% \begin{macro}[internal]{\ekvc@long,\ekvc@any@long}
%   Some macros will have to be defined long. These two will be let to |\long|
%   when this should be the case.
%    \begin{macrocode}
\let\ekvc@long\ekv@empty
\let\ekvc@any@long\ekv@empty
%    \end{macrocode}
% \end{macro}
%
% \begin{macro}[internal]{\ekvc@ifdefined}
%   We want to test whether a macro is already defined. This test checks for a
%   defined macro that isn't |\relax|.
%    \begin{macrocode}
\long\def\ekvc@ifdefined#1%
  {%
    \ifdefined#1%
      \ifx\relax#1%
        \ekv@fi@gobble
      \fi
      \@firstofone
      \ekv@fi@firstoftwo
    \fi
    \@secondoftwo
  }
%    \end{macrocode}
% \end{macro}
%
% \begin{macro}[internal]
%   {
%     \ekvc@ekvset@pre@expander,
%     \ekvc@ekvset@pre@expander@a,
%     \ekvc@ekvset@pre@expander@b
%   }
%   This macro expands |\ekvset| twice so that the first two steps of expansion
%   don't have to be made every time the \expkvc\ macros are used. We have to do
%   a little magic trick to get the macro parameter |#1| for the macro
%   definition this is used in, even though we're calling |\unexpanded|. We do
%   that by splitting the expanded |\ekvset| at some marks and place |##1| in
%   between. At this spot we also add |\ekv@alignsafe| and |\ekv@endalignsafe|
%   to ensure that macros created with \expkvc\ are alignment safe.
%    \begin{macrocode}
\def\ekvc@ekvset@pre@expander#1%
  {%
    \expandafter\ekvc@ekvset@pre@expander@a\ekvset{#1}\ekvc@stop\ekvc@stop
  }
\def\ekvc@ekvset@pre@expander@a
  {%
    \expandafter\ekvc@ekvset@pre@expander@b
  }
\def\ekvc@ekvset@pre@expander@b#1\ekvc@stop#2\ekvc@stop
  {%
    \unexpanded\expandafter{\ekv@alignsafe}%
    \unexpanded{#1}##1\unexpanded{#2}%
    \unexpanded\expandafter{\ekv@endalignsafe}%
  }
%    \end{macrocode}
% \end{macro}
%
% \begin{macro}{\ekvcSplitAndUse}
%   The first user macro we want to set up can be reused for
%   \cs[no-index]{ekvcSplitAndForward} and \cs[no-index]{ekvcSplit}. We'll split
%   this one up so that the test whether the macro is already defined doesn't
%   run twice.
%    \begin{macrocode}
\protected\long\def\ekvcSplitAndUse#1#2%
  {%
    \let\ekvc@helpers@needed\@firstoftwo
    \ekvc@ifdefined#1%
      {\ekvc@err@already@defined#1}%
      {\ekvcSplitAndUse@#1{}{#2}}%
  }
%    \end{macrocode}
% \end{macro}
%
% \begin{macro}[internal]{\ekvcSplitAndUse@}
%   The actual macro setting up things. We need to set some variables, forward
%   the key list to |\ekvc@SetupSplitKeys|, and afterwards define the front
%   facing macro to call |\ekvset| and put the initials and the argument sorting
%   macro behind it. The internals |\ekvc@any@long|, |\ekvc@initials| and
%   |\ekvc@keycount| will be set correctly by |\ekvc@SetupSplitKeys|.
%    \begin{macrocode}
\protected\long\def\ekvcSplitAndUse@#1#2#3%
  {%
    \edef\ekvc@set{\string#1}%
    \ekvc@SetupSplitKeys{#3}%
    \ekvc@helpers@needed
      {%
        \ekvc@any@long\edef#1##1%
          {%
            \expandafter\ekvc@ekvset@pre@expander\expandafter{\ekvc@set}%
            \unexpanded\expandafter
              {\csname ekvc@split@\the\ekvc@keycount\endcsname}%
            \unexpanded\expandafter{\ekvc@initials{}#2}%
          }%
      }%
      {%
        \ekvc@any@long\edef#1##1%
          {%
            \expandafter\ekvc@ekvset@pre@expander\expandafter{\ekvc@set}%
            \unexpanded{#2}%
            \unexpanded\expandafter{\ekvc@initials}%
          }%
      }%
  }
%    \end{macrocode}
% \end{macro}
%
%
% \begin{macro}{\ekvcSplitAndForward}
%   This just reuses |\ekvcSplitAndUse@| with a non-empty second argument,
%   resulting in that argument to be called after the splitting.
%    \begin{macrocode}
\protected\long\def\ekvcSplitAndForward#1#2#3%
  {%
    \let\ekvc@helpers@needed\@firstoftwo
    \ekvc@ifdefined#1%
      {\ekvc@err@already@defined#1}%
      {\ekvcSplitAndUse@#1{{#2}}{#3}}%
  }
%    \end{macrocode}
% \end{macro}
%
% \begin{macro}{\ekvcSplit}
%   The first half is just |\ekvcSplitAndForward| then we define the macro to
%   which the parsed key list is forwarded. There we need to allow for up to
%   nine arguments.
%    \begin{macrocode}
\protected\long\def\ekvcSplit#1#2#3%
  {%
    \let\ekvc@helpers@needed\@secondoftwo
    \ekvc@ifdefined#1%
      {\ekvc@err@already@defined#1}%
      {%
        \expandafter
        \ekvcSplitAndUse@\expandafter#1\csname ekvc@\string#1\endcsname{#2}%
        \ifnum\ekvc@keycount<1
          \ekvc@any@long\expandafter\def\csname ekvc@\string#1\endcsname{#3}%
        \else
          \ifnum\ekvc@keycount>9
            \ekvc@err@toomany{#1}%
            \let#1\ekvc@undefined
          \else
            \ekvcSplit@build@argspec
            \ekvc@any@long\expandafter
            \def\csname ekvc@\string#1\expandafter\endcsname\ekvc@tmp{#3}%
          \fi
        \fi
      }%
  }
%    \end{macrocode}
% \end{macro}
%
% \begin{macro}[internal]{\ekvcSplit@build@argspec,\ekvcSplit@build@argspec@}
%    \begin{macrocode}
\protected\def\ekvcSplit@build@argspec
  {%
    \begingroup
      \edef\ekvc@tmp
        {\endgroup\def\unexpanded{\ekvc@tmp}{\ekvcSplit@build@argspec@{1}}}%
    \ekvc@tmp
  }
\def\ekvcSplit@build@argspec@#1%
  {%
    \ifnum#1>\ekvc@keycount
      \ekv@fi@gobble
    \fi
    \@firstofone
      {%
        \unexpanded\expandafter{\csname ekvc@splitmark@#1\endcsname####}#1%
        \expandafter\ekvcSplit@build@argspec@\expandafter{\the\numexpr#1+1}%
      }%
  }
%    \end{macrocode}
% \end{macro}
%
% \begin{macro}[internal]
%   {
%     \ekvc@SetupSplitKeys, \ekvc@SetupSplitKeys@a, \ekvc@SetupSplitKeys@b,
%     \ekvc@SetupSplitKeys@c
%   }
%   These macros parse the list of keys and set up the key macros. First we need
%   to initialise some macros and start |\ekvparse|.
%    \begin{macrocode}
\protected\long\def\ekvc@SetupSplitKeys
  {%
    \ekvc@keycount=0
    \let\ekvc@any@long\ekv@empty
    \let\ekvc@initials\ekv@empty
    \ekvparse\ekvc@err@value@required\ekvc@SetupSplitKeys@a
  }
%    \end{macrocode}
%   Then we need to step the key counter for each key. Also we have to check
%   whether this key has a |long| prefix so we initialise |\ekvc@long|.
%    \begin{macrocode}
\protected\def\ekvc@SetupSplitKeys@a#1%
  {%
    \advance\ekvc@keycount1
    \let\ekvc@long\ekv@empty
    \ekvc@ifspace{#1}%
      {\ekvc@SetupSplitKeys@b#1\ekvc@stop}%
      {\ekvc@SetupSplitKeys@c{#1}}%
  }
%    \end{macrocode}
%   If there was a space, there might be a prefix. If so call the prefix macro,
%   else call the next step |\ekvc@SetupSplitKeys@c| which will define the key
%   macro and add the key's value to the initials list.
%    \begin{macrocode}
\protected\def\ekvc@SetupSplitKeys@b#1 #2\ekvc@stop
  {%
    \ekv@ifdefined{ekvc@split@p@#1}%
      {\csname ekvc@split@p@#1\endcsname{#2}}%
      {\ekvc@SetupSplitKeys@c{#1 #2}}%
  }
%    \end{macrocode}
%   The inner definition is grouped, because we don't want to actually define
%   the marks we build with |\csname|. We have to append the value to the
%   |\ekvc@initials| list here with the correct split mark. The key macro will
%   read everything up to those split marks and change the value following it to
%   the value given to the key. Additionally we'll need a sorting macro for each
%   key count in use so we set it up with |\ekvc@setup@splitmacro|.
%    \begin{macrocode}
\protected\long\def\ekvc@SetupSplitKeys@c#1#2%
  {%
    \begingroup
    \edef\ekvc@tmp
      {%
        \endgroup
        \long\def\unexpanded{\ekvc@tmp}####1####2%
            \unexpanded\expandafter
            {\csname ekvc@splitmark@\the\ekvc@keycount\endcsname}####3%
          {%
            ####2%
            \unexpanded\expandafter
              {\csname ekvc@splitmark@\the\ekvc@keycount\endcsname}{####1}%
          }%
%    \end{macrocode}
%   The short variant needs a bit of special treatment. The key macro will be
%   short to throw the correct error, but since there might be long macros
%   somewhere the reordering of arguments needs to be long, so for short keys we
%   use a two step approach, first grabbing only the short argument, then
%   reordering.
%    \begin{macrocode}
        \unless\ifx\ekvc@long\long
          \let\unexpanded\expandafter
            {\csname ekvc@\ekvc@set(#1)\endcsname\ekvc@tmp}%
          \def\unexpanded{\ekvc@tmp}####1%
            {%
              \unexpanded\expandafter{\csname ekvc@\ekvc@set(#1)\endcsname}%
                {####1}%
            }%
        \fi
        \def\unexpanded{\ekvc@initials}%
          {%
            \unexpanded\expandafter{\ekvc@initials}%
            \unexpanded\expandafter
              {\csname ekvc@splitmark@\the\ekvc@keycount\endcsname{#2}}%
          }%
      }%
    \ekvc@tmp
    \ekvlet\ekvc@set{#1}\ekvc@tmp
    \ekvc@helpers@needed
      {\expandafter\ekvc@setup@splitmacro\expandafter{\the\ekvc@keycount}}%
      {}%
  }
%    \end{macrocode}
% \end{macro}
%
% \begin{macro}[internal]{\ekvc@split@p@long}
%   The |long| prefix lets the internals |\ekvc@long| and |\ekvc@any@long| to
%   |\long| so that the key macro will be long.
%    \begin{macrocode}
\protected\def\ekvc@split@p@long
  {%
    \let\ekvc@long\long
    \let\ekvc@any@long\long
    \ekvc@SetupSplitKeys@c
  }
%    \end{macrocode}
% \end{macro}
%
% \begin{macro}[internal]{\ekvc@defarggobbler}
%   This is needed to define a macro with 1-9 parameters programmatically.
%   \LaTeX's \cs[no-index]{newcommand} does something similar for example.
%    \begin{macrocode}
\protected\def\ekvc@defarggobbler#1{\def\ekvc@tmp##1#1##2##{##1#1}}
%    \end{macrocode}
% \end{macro}
%
% \begin{macro}[internal]
%   {
%     \ekvc@setup@splitmacro,
%     \ekvc@split@1, \ekvc@split@2, \ekvc@split@3,
%     \ekvc@split@4, \ekvc@split@5, \ekvc@split@6,
%     \ekvc@split@7
%   }
%   Since the first few split macros are different from the others we manually
%   set those up now. All the others will be defined as needed (always
%   globally).  The split macros just read up until the correct split mark, move
%   that argument into a list and reinsert the rest, calling the next split
%   macro afterwards.
%    \begin{macrocode}
\begingroup
\edef\ekvc@tmp
  {%
    \long\gdef\unexpanded\expandafter{\csname ekvc@split@1\endcsname}%
        \unexpanded\expandafter{\csname ekvc@splitmark@1\endcsname}%
        ##1##2##3%
      {##3{##1}##2}%
    \long\gdef\unexpanded\expandafter{\csname ekvc@split@2\endcsname}%
        \unexpanded\expandafter{\csname ekvc@splitmark@1\endcsname}##1%
        \unexpanded\expandafter{\csname ekvc@splitmark@2\endcsname}##2%
        ##3##4%
      {##4{##1}{##2}##3}%
    \long\gdef\unexpanded\expandafter{\csname ekvc@split@3\endcsname}%
        \unexpanded\expandafter{\csname ekvc@splitmark@1\endcsname}##1%
        \unexpanded\expandafter{\csname ekvc@splitmark@2\endcsname}##2%
        \unexpanded\expandafter{\csname ekvc@splitmark@3\endcsname}##3%
        ##4##5%
      {##5{##1}{##2}{##3}##4}%
    \long\gdef\unexpanded\expandafter{\csname ekvc@split@4\endcsname}%
        \unexpanded\expandafter{\csname ekvc@splitmark@1\endcsname}##1%
        \unexpanded\expandafter{\csname ekvc@splitmark@2\endcsname}##2%
        \unexpanded\expandafter{\csname ekvc@splitmark@3\endcsname}##3%
        \unexpanded\expandafter{\csname ekvc@splitmark@4\endcsname}##4%
        ##5##6%
      {##6{##1}{##2}{##3}{##4}##5}%
    \long\gdef\unexpanded\expandafter{\csname ekvc@split@5\endcsname}%
        \unexpanded\expandafter{\csname ekvc@splitmark@1\endcsname}##1%
        \unexpanded\expandafter{\csname ekvc@splitmark@2\endcsname}##2%
        \unexpanded\expandafter{\csname ekvc@splitmark@3\endcsname}##3%
        \unexpanded\expandafter{\csname ekvc@splitmark@4\endcsname}##4%
        \unexpanded\expandafter{\csname ekvc@splitmark@5\endcsname}##5%
        ##6##7%
      {##7{##1}{##2}{##3}{##4}{##5}##6}%
    \long\gdef\unexpanded\expandafter{\csname ekvc@split@6\endcsname}%
        \unexpanded\expandafter{\csname ekvc@splitmark@1\endcsname}##1%
        \unexpanded\expandafter{\csname ekvc@splitmark@2\endcsname}##2%
        \unexpanded\expandafter{\csname ekvc@splitmark@3\endcsname}##3%
        \unexpanded\expandafter{\csname ekvc@splitmark@4\endcsname}##4%
        \unexpanded\expandafter{\csname ekvc@splitmark@5\endcsname}##5%
        \unexpanded\expandafter{\csname ekvc@splitmark@6\endcsname}##6%
        ##7##8%
      {##8{##1}{##2}{##3}{##4}{##5}{##6}##7}%
    \long\gdef\unexpanded\expandafter{\csname ekvc@split@7\endcsname}%
        \unexpanded\expandafter{\csname ekvc@splitmark@1\endcsname}##1%
        \unexpanded\expandafter{\csname ekvc@splitmark@2\endcsname}##2%
        \unexpanded\expandafter{\csname ekvc@splitmark@3\endcsname}##3%
        \unexpanded\expandafter{\csname ekvc@splitmark@4\endcsname}##4%
        \unexpanded\expandafter{\csname ekvc@splitmark@5\endcsname}##5%
        \unexpanded\expandafter{\csname ekvc@splitmark@6\endcsname}##6%
        \unexpanded\expandafter{\csname ekvc@splitmark@7\endcsname}##7%
        ##8##9%
      {##9{##1}{##2}{##3}{##4}{##5}{##6}{##7}##8}%
  }
\ekvc@tmp
\endgroup
\protected\def\ekvc@setup@splitmacro#1%
  {%
    \ekv@ifdefined{ekvc@split@#1}{}%
      {%
        \begingroup
          \edef\ekvc@tmp
            {%
              \long\gdef
                  \unexpanded\expandafter{\csname ekvc@split@#1\endcsname}%
                  ####1%
                  \unexpanded\expandafter{\csname ekvc@splitmark@#1\endcsname}%
                  ####2####3%
                {%
                  \unexpanded\expandafter
                    {\csname ekvc@split@\the\numexpr#1-1\relax\endcsname}%
                  ####1{{####2}####3}%
                }%
            }%
          \ekvc@tmp
        \endgroup
      }%
  }
%    \end{macrocode}
% \end{macro}
%
% \begin{macro}{\ekvcHashAndUse}
%   |\ekvcHashAndUse| works just like |\ekvcSplitAndUse|.
%    \begin{macrocode}
\protected\long\def\ekvcHashAndUse#1#2%
  {%
    \let\ekvc@helpers@needed\@firstoftwo
    \ekvc@ifdefined#1%
      {\ekvc@err@already@defined#1}%
      {\ekvcHashAndUse@#1{}{#2}}%
  }
%    \end{macrocode}
% \end{macro}
%
% \begin{macro}[internal]{\ekvcHashAndUse@}
%   This is more or less the same as |\ekvcSplitAndUse@|. Instead of an empty
%   group we place a marker after the initials, we don't use the sorting macros
%   of |split|, but instead pack all the values in one argument.
%    \begin{macrocode}
\protected\long\def\ekvcHashAndUse@#1#2#3%
  {%
    \edef\ekvc@set{\string#1}%
    \ekvc@SetupHashKeys{#3}%
    \ekvc@helpers@needed
      {%
        \ekvc@any@long\edef#1##1%
          {%
            \expandafter\ekvc@ekvset@pre@expander\expandafter{\ekvc@set}%
            \unexpanded{\ekvc@hash@pack@argument}%
            \unexpanded\expandafter{\ekvc@initials\ekvc@stop#2}%
          }%
      }%
      {%
        \ekvc@any@long\edef#1##1%
          {%
            \expandafter\ekvc@ekvset@pre@expander\expandafter{\ekvc@set}%
            \unexpanded{#2}%
            \unexpanded\expandafter{\ekvc@initials\ekvc@stop}%
          }%
      }%
  }
%    \end{macrocode}
% \end{macro}
%
% \begin{macro}{\ekvcHashAndForward}
%   |\ekvcHashAndForward| works just like |\ekvcSplitAndForward|.
%    \begin{macrocode}
\protected\long\def\ekvcHashAndForward#1#2#3%
  {%
    \let\ekvc@helpers@needed\@firstoftwo
    \ekvc@ifdefined#1%
      {\ekvc@err@already@defined#1}%
      {\ekvcHashAndUse@#1{{#2}}{#3}}%
  }
%    \end{macrocode}
% \end{macro}
%
% \begin{macro}{\ekvcHash}
%   |\ekvcHash| does the same as |\ekvcSplit|, but has the advantage of not
%   needing to count arguments, so the definition of the internal macro is a bit
%   more straight forward.
%    \begin{macrocode}
\protected\long\def\ekvcHash#1#2#3%
  {%
    \let\ekvc@helpers@needed\@secondoftwo
    \ekvc@ifdefined#1%
      {\ekvc@err@already@defined#1}%
      {%
        \expandafter
        \ekvcHashAndUse@\expandafter#1\csname ekvc@\string#1\endcsname{#2}%
        \ekvc@any@long\expandafter\def\csname ekvc@\string#1\endcsname
            ##1\ekvc@stop
          {#3}%
      }%
  }
%    \end{macrocode}
% \end{macro}
%
% \begin{macro}[internal]{\ekvc@hash@pack@argument}
%   All this macro does is pack the values into one argument and forward that to
%   the next macro.
%    \begin{macrocode}
\long\def\ekvc@hash@pack@argument#1\ekvc@stop#2{#2{#1}}
%    \end{macrocode}
% \end{macro}
%
% \begin{macro}[internal]
%   {\ekvc@SetupHashKeys,\ekvc@SetupHashKeys@a,\ekvc@SetupHashKeys@b}
%   This should look awfully familiar as well, since it's just the same as for
%   the split keys with a few other names here and there.
%    \begin{macrocode}
\protected\long\def\ekvc@SetupHashKeys#1%
  {%
    \let\ekvc@any@long\ekv@empty
    \let\ekvc@initials\ekv@empty
    \ekvparse\ekvc@err@value@required\ekvc@SetupHashKeys@a{#1}%
  }
\protected\def\ekvc@SetupHashKeys@a#1%
  {%
    \let\ekvc@long\ekv@empty
    \ekvc@ifspace{#1}%
      {\ekvc@SetupHashKeys@b#1\ekvc@stop}%
      {\ekvc@SetupHashKeys@c{#1}}%
  }
\protected\def\ekvc@SetupHashKeys@b#1 #2\ekvc@stop
  {%
    \ekv@ifdefined{ekvc@hash@p@#1}%
      {\csname ekvc@hash@p@#1\endcsname{#2}}%
      {\ekvc@SetupHashKeys@c{#1 #2}}%
  }
%    \end{macrocode}
%   Yes, even the defining macro looks awfully familiar. Instead of numbered we
%   have named marks. Still the key macros grab everything up to their
%   respective mark and reorder the arguments. The same quirk is applied for
%   short keys. And instead of the |\ekvc@setup@splitmacro| we use
%   |\ekvc@setup@hashmacro|.
%    \begin{macrocode}
\protected\long\def\ekvc@SetupHashKeys@c#1#2%
  {%
    \begingroup
    \edef\ekvc@tmp
      {%
        \endgroup
        \long\def\unexpanded{\ekvc@tmp}####1####2%
            \unexpanded\expandafter{\csname ekvc@hashmark@#1\endcsname}####3%
          {%
            ####2%
            \unexpanded\expandafter{\csname ekvc@hashmark@#1\endcsname}{####1}%
          }%
        \unless\ifx\ekvc@long\long
          \let\unexpanded\expandafter
            {\csname ekvc@\ekvc@set(#1)\endcsname\ekvc@tmp}%
          \def\unexpanded{\ekvc@tmp}####1%
            {%
              \unexpanded\expandafter{\csname ekvc@\ekvc@set(#1)\endcsname}%
                {####1}%
            }%
        \fi
        \def\unexpanded{\ekvc@initials}%
          {%
            \unexpanded\expandafter{\ekvc@initials}%
            \unexpanded\expandafter{\csname ekvc@hashmark@#1\endcsname{#2}}%
          }%
      }%
    \ekvc@tmp
    \ekvlet\ekvc@set{#1}\ekvc@tmp
    \ekvc@setup@hashmacro{#1}%
  }
%    \end{macrocode}
% \end{macro}
%
% \begin{macro}[internal]{\ekvc@hash@p@long}
%   Nothing astonishing here either.
%    \begin{macrocode}
\protected\def\ekvc@hash@p@long
  {%
    \let\ekvc@long\long
    \let\ekvc@any@long\long
    \ekvc@SetupHashKeys@c
  }
%    \end{macrocode}
% \end{macro}
%
% \begin{macro}[internal]{\ekvc@setup@hashmacro}
%   The safe hash macros will be executed inside of an |\unexpanded| expansion
%   context, so they have to insert braces for that once they are done.
%   Most of the tests which have to be executed will already be done, but we
%   have to play safe if the hash doesn't show up in the hash list. Therefore we
%   use some |\ekvc@mark|s and |\ekvc@stop| to throw errors if the hash isn't
%   found in the right place. The fast variants have an easier life and just
%   return the correct value.
%    \begin{macrocode}
\protected\def\ekvc@setup@hashmacro#1%
  {%
    \ekv@ifdefined{ekvc@fasthash@#1}{}%
      {%
        \begingroup
          \edef\ekvc@tmp
            {%
              \long\gdef
                \unexpanded\expandafter{\csname ekvc@fasthash@#1\endcsname}%
                  ####1%
                  \unexpanded\expandafter{\csname ekvc@hashmark@#1\endcsname}%
                  ####2####3\unexpanded{\ekvc@stop}%
                {####2}%
              \long\gdef
                \unexpanded\expandafter{\csname ekvc@safehash@#1\endcsname}%
                  ####1%
                {%
                  \unexpanded\expandafter{\csname ekvc@@safehash@#1\endcsname}%
                  ####1\unexpanded{\ekvc@mark}{}%
                  \unexpanded\expandafter
                    {%
                      \csname ekvc@hashmark@#1\endcsname{}%
                      \ekvc@mark{\ekvc@err@missing@hash{#1}}\ekvc@stop
                    }%
                }%
              \long\gdef
                \unexpanded\expandafter{\csname ekvc@@safehash@#1\endcsname}%
                  ####1%
                  \unexpanded\expandafter{\csname ekvc@hashmark@#1\endcsname}%
                  ####2####3\unexpanded{\ekvc@mark}####4####5%
                  \unexpanded{\ekvc@stop}%
                {%
                  ####4{####2}%
                }%
              \long\gdef\unexpanded\expandafter
                {\csname ekvc@fastsplithash@#1\endcsname}%
                  ####1%
                  \unexpanded\expandafter{\csname ekvc@hashmark@#1\endcsname}%
                  ####2####3\unexpanded{\ekvc@stop}####4%
                {%
                  ####4{####1####3}{####2}%
                }%
              \long\gdef\unexpanded\expandafter
                {\csname ekvc@safesplithash@#1\endcsname}####1%
                {%
                  \unexpanded\expandafter
                    {\csname ekvc@@safesplithash@#1\endcsname}%
                  ####1\unexpanded{\ekvc@mark\ekvc@safe@found@hash}%
                  \unexpanded\expandafter
                    {%
                      \csname ekvc@hashmark@#1\endcsname{}%
                      \ekvc@mark{\ekvc@err@missing@hash{#1}\ekvc@safe@no@hash}%
                      \ekvc@stop
                    }%
                }%
              \long\gdef\unexpanded\expandafter
                {\csname ekvc@@safesplithash@#1\endcsname}%
                  ####1%
                  \unexpanded\expandafter{\csname ekvc@hashmark@#1\endcsname}%
                  ####2####3\unexpanded{\ekvc@mark}####4####5%
                  \unexpanded{\ekvc@stop}%
                {%
                  ####4{####2}####1####3\unexpanded{\ekvc@stop}%
                }%
            }%
          \ekvc@tmp
        \endgroup
      }%
  }
%    \end{macrocode}
% \end{macro}
%
% \begin{macro}{\ekvcValue}
%   All this does is a few consistency checks on the first argument (not empty,
%   hash macro exists) and then call that hash-grabbing macro that will also
%   test whether the hash is inside of |#2| or not.
%    \begin{macrocode}
\long\def\ekvcValue#1%
  {%
    \unexpanded
    \ekv@ifdefined{ekvc@safehash@#1}%
      {\csname ekvc@safehash@#1\endcsname}%
      {\ekvc@err@unknown@hash{#1}\@firstoftwo{{}}}%
  }
%    \end{macrocode}
% \end{macro}
%
% \begin{macro}{\ekvcValueFast}
%   To be as fast as possible, this doesn't test for anything, assuming the user
%   knows best.
%    \begin{macrocode}
\long\def\ekvcValueFast#1#2{\csname ekvc@fasthash@#1\endcsname#2\ekvc@stop}
%    \end{macrocode}
% \end{macro}
%
% \begin{macro}{\ekvcValueSplit,\ekvcValueSplit@recover}
%   This splits off a single value.
%    \begin{macrocode}
\long\def\ekvcValueSplit#1%
  {%
    \ekv@ifdefined{ekvc@safesplithash@#1}%
      {\csname ekvc@safesplithash@#1\endcsname}%
      {\ekvc@err@unknown@hash{#1}\ekvcValueSplit@recover}%
  }
\long\def\ekvcValueSplit@recover#1#2%
  {%
    #2{#1}{}%
  }
%    \end{macrocode}
% \end{macro}
%
% \begin{macro}[internal]{\ekvc@safe@found@hash, \ekvc@safe@no@hash}
%    \begin{macrocode}
\long\def\ekvc@safe@found@hash#1#2\ekvc@stop#3%
  {%
    #3{#2}{#1}%
  }
\long\def\ekvc@safe@no@hash#1#2\ekvc@mark\ekvc@safe@found@hash\ekvc@stop#3%
  {%
    #3{#2}{}%
  }
%    \end{macrocode}
% \end{macro}
%
% \begin{macro}{\ekvcValueSplitFast}
%   Again a fast approach which doesn't provide too many safety measurements.
%   This needs to build the hash function and expand it before passing the
%   results to the next control sequence. The first step only builds the control
%   sequence.
%    \begin{macrocode}
\long\def\ekvcValueSplitFast#1#2%
  {%
    \csname ekvc@fastsplithash@#1\endcsname#2\ekvc@stop
  }
%    \end{macrocode}
% \end{macro}
%
% \begin{macro}[internal]
%   {
%     \ekvc@safehash@,\ekvc@fasthash@,
%     \ekvc@safesplithash@,\ekvc@fastsplithash@
%   }
%   At least in the empty hash case we can provide a meaningful error message
%   without affecting performance by just defining the macro that would be build
%   in that case. There is of course a downside to this, the error will not be
%   thrown by |\ekvcValueFast| in three expansion steps. The safe hash variant
%   has to also stop the |\unexpanded| expansion.
%    \begin{macrocode}
\long\def\ekvc@safehash@#1{\ekvc@err@empty@hash{}}
\long\def\ekvc@fasthash@#1\ekvc@stop{\ekvc@err@empty@hash}
\long\def\ekvc@safesplithash@#1#2{\ekvc@err@empty@hash#2{#1}{}}
\long\def\ekvc@fastsplithash@#1\ekvc@stop#2{\ekvc@err@empty@hash#2{#1}{}}
%    \end{macrocode}
% \end{macro}
%
% \begin{macro}{\ekvcSecondaryKeys}
%   The secondary keys are defined pretty similar to the way the originals are,
%   but here we also introduce some key types (those have a |@t@| in their name)
%   additionally to the prefixes.
%    \begin{macrocode}
\protected\long\def\ekvcSecondaryKeys#1#2%
  {%
    \edef\ekvc@set{\string#1}%
    \ekvparse\ekvc@err@value@required\ekvcSecondaryKeys@a{#2}%
  }
\protected\def\ekvcSecondaryKeys@a#1%
  {%
    \let\ekvc@long\ekv@empty
    \ekvc@ifspace{#1}%
      {\ekvcSecondaryKeys@b#1\ekvc@stop}%
      {\ekvc@err@missing@type{#1}\@gobble}%
  }
\protected\def\ekvcSecondaryKeys@b#1 #2\ekvc@stop
  {%
    \ekv@ifdefined{ekvc@p@#1}%
      {\csname ekvc@p@#1\endcsname}%
      {%
        \ekv@ifdefined{ekvc@t@#1}%
          {\csname ekvc@t@#1\endcsname}%
          {\ekvc@err@unknown@keytype{#1}\@firstoftwo\@gobble}%
      }%
      {#2}%
  }
%    \end{macrocode}
% \end{macro}
%
% \begin{macro}{\ekvcChange}
%   This can be used to change the defaults of an \expkvc\ defined macro. It
%   checks whether there is a set with the correct name and that the macro is
%   defined. If both is true the real work is done by |\ekvc@change|.
%    \begin{macrocode}
\protected\long\def\ekvcChange#1%
  {%
    \ekvifdefinedset{\string#1}%
      {%
        \ekvc@ifdefined#1%
          {\ekvc@change#1}%
          {\ekvc@err@no@key@macro#1\@gobble}%
      }%
      {\ekvc@err@no@key@macro#1\@gobble}%
  }
%    \end{macrocode}
% \end{macro}
%
% \begin{macro}[internal]
%   {\ekvc@change,\ekvc@change@a,\ekvc@change@b,\ekvc@change@c}
%   First we need to see whether the macro is currently |\long|. For this we get
%   the meaning and will parse it. |#1| is the macro name in which we want to
%   change the defaults.
%    \begin{macrocode}
\protected\def\ekvc@change#1%
  {\expandafter\ekvc@change@a\meaning#1\ekv@stop#1}
%    \end{macrocode}
%   A temporary definition to get the stringified |macro:|. |##1| will be the
%   list of prefixes, we don't care for the exact contents of |##2| and |##3|.
%    \begin{macrocode}
\def\ekvc@change@a#1%
  {%
    \protected\def\ekvc@change@a##1#1##2->##3\ekv@stop
      {%
        \ekvc@change@iflong{##1}%
          {\ekvc@change@b{}}%
          {\ekvc@change@b{\long}}%
      }%
  }
\expandafter\ekvc@change@a\expandafter{\detokenize{macro:}}
%    \end{macrocode}
%   Next we expand the macro once to get its contents (including the current
%   default values with their markers). |#1| is either |\long| or empty, |#2| is
%   the macro.
%    \begin{macrocode}
\protected\def\ekvc@change@b#1#2%
  {\expandafter\ekvc@change@c\expandafter{#2{##1}}{#1}#2}
%    \end{macrocode}
%   Here we place an unbalanced closing brace after the expansion of the macro.
%   Then we just parse the \kv-list with |\ekvset|, that will exchange the
%   values behind the markers. Once those are changed we give control to
%   |\ekvc@change@d|. The |\ekvset| step might horribly fail if the user defined
%   some keys that don't behave nice. |#1| is the expansion of the macro, |#2|
%   is either |\long| or empty, |#3| is the macro, and |#4| is the \kv-list
%   containing the new defaults.
%    \begin{macrocode}
\ekv@exparg{\protected\long\def\ekvc@change@c#1#2#3#4}%
  {%
    \expandafter\iffalse\expandafter{\expandafter\fi
      \ekvset{\string#3}{#4}%
      \ekvc@change@d{#2}{#3}%
      #1%
    }%
  }
%    \end{macrocode}
%   The final step needs to put an unbalanced opening brace after |\def|. We do
%   that with the help of a temporary macro which stores everything necessary
%   for |\def| and expand an |\iffalse}\fi| construct to leave a single
%   opening brace. |#1| will be either empty or |\long| and |#2| is the macro.
%   Each of the macros defined with \expkvc\ takes exactly one parameter, so we
%   put that here as |####1|.
%    \begin{macrocode}
\protected\def\ekvc@change@d#1#2%
  {%
    \def\ekvc@tmp{#1\def#2####1}%
    \expandafter\ekvc@tmp\expandafter{\iffalse}\fi
  }
%    \end{macrocode}
% \end{macro}
%
% \begin{macro}[internal]{\ekvc@change@iflong,\ekvc@change@iflong@}
%   Checking whether a string contains the string representation of |\long| can
%   be done by gobbling everything up to the first |\long| and checking whether
%   the result is completely empty. We need a temporary macro to get the result
%   of |\string\long| inside the definitions.
%    \begin{macrocode}
\def\ekvc@change@iflong#1%
  {%
    \protected\def\ekvc@change@iflong##1%
      {\expandafter\ekv@ifempty\expandafter{\ekvc@change@iflong@##1#1}}%
    \def\ekvc@change@iflong@##1#1{}
  }
\expandafter\ekvc@change@iflong\expandafter{\string\long}
%    \end{macrocode}
% \end{macro}
%
% \begin{macro}{\ekvcPass}
%   This macro can be used to pass a value to a key of some macro (this way
%   more complicated key codes are possible that in the end pass processed
%   values on to some macro). The implemantation is pretty straight forward.
%    \begin{macrocode}
\long\def\ekvcPass#1#2%
  {%
    \ekvifdefined{\string#1}{#2}%
      {\csname\ekv@name{\string#1}{#2}\endcsname}%
      {\ekvc@err@unknown@key@or@macro{#1}{#2}\@gobble}%
  }
%    \end{macrocode}
% \end{macro}
%
%
% \subsubsection{Secondary Key Types}
%
% \begin{macro}[internal]{\ekvc@p@long,\ekvc@after@ptype}
%   The prefixes are pretty straight forward again. Just set |\ekvc@long| and
%   forward to the |@t@| type.
%    \begin{macrocode}
\protected\def\ekvc@p@long#1%
  {%
    \ekvc@ifspace{#1}%
      {%
        \let\ekvc@long\long
        \ekvc@after@ptype#1\ekvc@stop
      }%
      {\ekvc@err@missing@type{long #1}\@gobble}%
  }
\protected\def\ekvc@after@ptype#1 #2\ekvc@stop
  {%
    \ekv@ifdefined{ekvc@t@#1}%
      {\csname ekvc@t@#1\endcsname{#2}}%
      {\ekvc@err@unknown@keytype{#1}\@gobble}%
  }
%    \end{macrocode}
% \end{macro}
%
% \begin{macro}[internal]
%   {
%     \ekvc@t@meta, \ekvc@t@nmeta,
%     \ekvc@type@meta, \ekvc@type@meta@a, \ekvc@type@meta@b
%   }
%   The |meta| and |nmeta| key types use a nested |\ekvset| to set other keys in
%   the same macro's \set.
%    \begin{macrocode}
\protected\def\ekvc@t@meta
  {%
    \edef\ekvc@tmp{\ekvc@set}%
    \expandafter\ekvc@type@meta\expandafter{\ekvc@tmp}\ekvc@long{##1}\ekvlet
  }
\protected\def\ekvc@t@nmeta#1%
  {%
    \ekvc@assert@not@long{nmeta #1}%
    \edef\ekvc@tmp{\ekvc@set}%
    \expandafter\ekvc@type@meta\expandafter{\ekvc@tmp}{}{}\ekvletNoVal{#1}%
  }
\protected\long\def\ekvc@type@meta#1#2#3#4#5#6%
  {%
    \expandafter\ekvc@type@meta@a\expandafter{\ekvset{#1}{#6}}{#2}{#3}%
    #4\ekvc@set{#5}\ekvc@tmp
  }
\protected\def\ekvc@type@meta@a
  {%
    \expandafter\ekvc@type@meta@b\expandafter
  }
\protected\long\def\ekvc@type@meta@b#1#2#3%
  {%
    #2\def\ekvc@tmp#3{#1}%
  }
%    \end{macrocode}
% \end{macro}
%
% \begin{macro}[internal]{\ekvc@t@alias}
%   |alias| just checks whether there is a key and/or |NoVal| key defined with
%   the target name and |\let| the key to those.
%    \begin{macrocode}
\protected\def\ekvc@t@alias#1#2%
  {%
    \ekvc@assert@not@long{alias #1}%
    \let\ekvc@tmp\@firstofone
    \ekvifdefined\ekvc@set{#2}%
      {%
        \ekvletkv\ekvc@set{#1}\ekvc@set{#2}%
        \let\ekvc@tmp\@gobble
      }%
      {}%
    \ekvifdefinedNoVal\ekvc@set{#2}%
      {%
        \ekvletkvNoVal\ekvc@set{#1}\ekvc@set{#2}%
        \let\ekvc@tmp\@gobble
      }%
      {}%
    \ekvc@tmp{\ekvc@err@unknown@key{#2}}%
  }
%    \end{macrocode}
% \end{macro}
%
% \begin{macro}[internal]{\ekvc@t@default}
%   The |default| key can be used to set a |NoVal| key for an existing key. It
%   will just pass the \val\ to the key macro of that other key.
%    \begin{macrocode}
\protected\long\def\ekvc@t@default#1#2%
  {%
    \ekvifdefined\ekvc@set{#1}%
      {%
        \ekvc@assert@not@long{default #1}%
        \edef\ekvc@tmp
          {%
            \unexpanded\expandafter
              {\csname\ekv@name\ekvc@set{#1}\endcsname{#2}}%
          }%
        \ekvletNoVal\ekvc@set{#1}\ekvc@tmp
      }%
      {\ekvc@err@unknown@key{#1}}%
  }
%    \end{macrocode}
% \end{macro}
%
% \begin{macro}[internal]{\ekvc@t@flag-bool}
%    \begin{macrocode}
\protected\expandafter\def\csname ekvc@t@flag-bool\endcsname#1#2%
  {%
    \ekvc@assert@not@long{flag-bool #1}%
    \unless\ifdefined#2\ekvcFlagNew#2\fi
    \ekvdef\ekvc@set{#1}%
      {%
        \ekv@ifdefined{ekvc@flag@set@##1}%
          {%
            \csname ekvc@flag@set@##1\expandafter\endcsname
              \ekvcFlagHeight#2\ekv@stop#2%
          }%
          {\ekvc@err@invalid@bool{##1}}%
      }%
  }
%    \end{macrocode}
% \end{macro}
%
% \begin{macro}[internal]
%   {\ekvc@t@flag-true,\ekvc@t@flag-false,\ekvc@t@flag-raise,\ekvc@type@flag}
%    \begin{macrocode}
\protected\def\ekvc@type@flag#1#2#3#4%
  {%
    \ekvc@assert@not@long{flag-#1 #3}%
    \unless\ifdefined#4\ekvcFlagNew#4\fi
    \ekv@exparg{\ekvdefNoVal\ekvc@set{#3}}{#2#4}%
  }
\protected\expandafter\def\csname ekvc@t@flag-true\endcsname
  {\ekvc@type@flag{true}\ekvcFlagSetTrue}
\protected\expandafter\def\csname ekvc@t@flag-false\endcsname
  {\ekvc@type@flag{false}\ekvcFlagSetFalse}
\protected\expandafter\def\csname ekvc@t@flag-raise\endcsname
  {\ekvc@type@flag{raise}\ekvcFlagRaise}
%    \end{macrocode}
% \end{macro}
%
%
% \subsubsection{Flags}
%
% The basic idea of flags is to store information by the fact that \TeX\
% expandably assigns the meaning |\relax| to undefined control sequences which
% were built with |\csname|. This mechanism is borrowed from \pkg{expl3}.
%
% \begin{macro}[internal]{\ekvc@flag@name,\ekvc@flag@namescheme}
%   Flags follow a simple naming scheme which we define here. |\ekvc@flag@name|
%   will store the name of an internal function that is used to build names of
%   the second naming scheme defined by |\ekvc@flag@namescheme|.
%    \begin{macrocode}
\def\ekvc@flag@name{ekvcf\string}
\def\ekvc@flag@namescheme#1#2{ekvch#2#1}
%    \end{macrocode}
% \end{macro}
%
% \begin{macro}{\ekvcFlagHeight}
%   For semantic reasons we use |\number| with another name.
%    \begin{macrocode}
\let\ekvcFlagHeight\number
%    \end{macrocode}
% \end{macro}
%
% \begin{macro}{\ekvcFlagNew}
%   This macro defines a new flag. It stores the function build with the
%   |\ekvc@flag@name| naming scheme after the internal function
%   |\ekvc@flag@height| that'll determine the current flag height. It'll also
%   define the macro named via |\ekvc@flag@name| to build names according to
%   |\ekvc@flag@namescheme|.
%    \begin{macrocode}
\protected\def\ekvcFlagNew#1%
  {%
    \edef#1%
      {%
        \unexpanded{\ekvc@flag@height}%
          \unexpanded\expandafter{\csname\ekvc@flag@name#1\endcsname}%
      }%
    \ekv@expargtwice
      {\expandafter\def\csname\ekvc@flag@name#1\endcsname##1}%
      {\expandafter\ekvc@flag@namescheme\expandafter{\string#1}{##1}}%
  }
%    \end{macrocode}
% \end{macro}
%
% \begin{macro}[internal]{\ekvc@flag@height,\ekvc@flag@height@}
%   This macro gets the height of a flag by a simple loop. The first loop
%   iteration differs a bit from the following in that it doesn't have to get
%   the current iteration count. The space at the end of |\ekvc@flag@height|
%   ends the |\number| evaluation.
%    \begin{macrocode}
\def\ekvc@flag@height#1%
  {%
    \ifcsname#10\endcsname
      \ekvc@flag@height@1\ekv@stop#1%
    \fi
    \@firstofone{0} % leave this space
  }
\def\ekvc@flag@height@#1\ekv@stop#2\fi\@firstofone#3%
  {%
    \fi
    \ifcsname#2{#1}\endcsname
      \expandafter\ekvc@flag@height@\the\numexpr#1+1\relax\ekv@stop#2%
    \fi
    \@firstofone{#1}%
  }
%    \end{macrocode}
% \end{macro}
%
% \begin{macro}{\ekvcFlagRaise}
%   Raising a flag simply means letting the |\ekvc@flag@namescheme| macro for
%   the current height to relax. The result of raising a flag is that its height
%   is bigger by $1$.
%    \begin{macrocode}
\ekv@exparg{\def\ekvcFlagRaise#1}%
  {%
    \expandafter\expandafter\expandafter\@gobble\expandafter
      \csname\ekvc@flag@namescheme{\string#1}{\ekvcFlagHeight#1}\endcsname
  }
%    \end{macrocode}
% \end{macro}
%
% \begin{macro}{\ekvcFlagSetTrue,\ekvcFlagSetFalse}
% \begin{macro}[internal]{\ekvc@flag@set@true,\ekvc@flag@set@false}
%   A flag is considered true if its current height is odd, and as false if it
%   is even. Therefore |\ekvcFlagSetTrue| and |\ekvcFlagSetFalse| only need to
%   raise the flag if the opposing boolean value is the current one.
%    \begin{macrocode}
\def\ekvcFlagSetTrue#1%
  {\expandafter\ekvc@flag@set@true\ekvcFlagHeight#1\ekv@stop#1}
\def\ekvcFlagSetFalse#1%
  {\expandafter\ekvc@flag@set@false\ekvcFlagHeight#1\ekv@stop#1}
%    \end{macrocode}
%   We can expand |\ekvc@flag@namescheme| at definition time here, which is why
%   we're using a temporary definition to set up |\ekvc@flag@set@true| and
%   |\ekvc@flag@set@false|.
%    \begin{macrocode}
\def\ekvc@flag@set@true#1%
  {%
    \def\ekvc@flag@set@true##1\ekv@stop##2%
      {%
        \ifodd##1
          \ekv@fi@gobble
        \fi
        \@firstofone{\expandafter\@gobble\csname#1\endcsname}%
      }%
    \def\ekvc@flag@set@false##1\ekv@stop##2%
      {%
        \ifodd##1
          \ekv@fi@firstofone
        \fi
        \@gobble{\expandafter\@gobble\csname#1\endcsname}%
      }%
  }
\expandafter\ekvc@flag@set@true\expandafter
  {\ekvc@flag@namescheme{\string#2}{#1}}
%    \end{macrocode}
% \end{macro}
% \end{macro}
%
% \begin{macro}{\ekvcFlagIf}
%   As already explained, truthiness is defined as a flag's height is odd, so we
%   just branch accordingly here.
%    \begin{macrocode}
\def\ekvcFlagIf#1%
  {%
    \ifodd#1%
      \ekv@fi@firstoftwo
    \fi
    \@secondoftwo
  }
%    \end{macrocode}
% \end{macro}
%
% \begin{macro}{\ekvcFlagIfRaised}
%   This macro uses flags as a switch, if a flag's current height is bigger than
%   $0$ this test yields true.
%    \begin{macrocode}
\ekv@exparg{\def\ekvcFlagIfRaised#1}%
  {%
    \expandafter\ifcsname\ekvc@flag@namescheme{\string#1}0\endcsname
      \ekv@fi@firstoftwo
    \fi
    \@secondoftwo
  }
%    \end{macrocode}
% \end{macro}
%
% \begin{macro}{\ekvcFlagReset}
% \begin{macro}[internal]{\ekvc@flag@reset,\ekvc@flag@reset@}
%   Resetting works by locally letting all the defined internal macros named
%   after |\ekvc@flag@namescheme| to undefined.
%    \begin{macrocode}
\protected\def\ekvcFlagReset#1%
  {\expandafter\ekvc@flag@reset\csname\ekvc@flag@name#1\endcsname}
\protected\def\ekvc@flag@reset#1%
  {%
    \ifcsname#10\endcsname
      \expandafter\let\csname#10\endcsname\ekvc@undefined
      \ekvc@flag@reset@1\ekv@stop#1%
    \fi
  }
\protected\def\ekvc@flag@reset@#1\ekv@stop#2\fi
  {%
    \fi
    \ifcsname#2{#1}\endcsname
      \expandafter\let\csname#2{#1}\endcsname\ekvc@undefined
      \expandafter\ekvc@flag@reset@\the\numexpr#1+1\relax\ekv@stop#2%
    \fi
  }
%    \end{macrocode}
% \end{macro}
% \end{macro}
%
% \begin{macro}{\ekvcFlagGetHeight}
% \begin{macro}[internal]{\ekvc@flag@get@height@single}
%   These are just small helpers, first getting the height of the flag and then
%   passing it on to the user supplied code.
%    \begin{macrocode}
\def\ekvcFlagGetHeight#1%
  {\expandafter\ekvc@flag@get@height@single\ekvcFlagHeight#1\ekv@stop}
\long\def\ekvc@flag@get@height@single#1\ekv@stop#2{#2{#1}}
%    \end{macrocode}
% \end{macro}
% \end{macro}
%
% \begin{macro}{\ekvcFlagGetHeights}
% \begin{macro}[internal]
%   {\ekvc@flag@get@heights,\ekvc@flag@get@heights@,\ekvc@flag@get@heights@done}
%   This works by a simple loop that stops at |\ekv@stop|. As long as that
%   marker isn't hit, get the next flags height and put it into a list after
%   |\ekv@stop|. |\ekvc@flag@get@heights@| uses the same marker name for the
%   end of the height, which shouldn't clash in any case. Once we're done we
%   remove the remainder of the current iteration and leave the user supplied
%   code in the input stream with all the flags' heights as a single argument.
%    \begin{macrocode}
\def\ekvcFlagGetHeights#1%
  {%
    \ekvc@flag@get@heights#1\ekv@stop{}%
  }
\def\ekvc@flag@get@heights#1%
  {%
    \ekv@gobbleto@stop#1\ekvc@flag@get@heights@done\ekv@stop
    \expandafter\ekvc@flag@get@heights@\ekvcFlagHeight#1\ekv@stop
  }
\def\ekvc@flag@get@heights@#1\ekv@stop#2\ekv@stop#3%
  {\ekvc@flag@get@heights#2\ekv@stop{#3{#1}}}
\long\def\ekvc@flag@get@heights@done
    \ekv@stop
    \expandafter\ekvc@flag@get@heights@\ekvcFlagHeight\ekv@stop\ekv@stop#1#2%
  {#2{#1}}
%    \end{macrocode}
% \end{macro}
% \end{macro}
%
%
% \subsubsection{Helper Macros}
%
% \begin{macro}[internal]{\ekvc@ifspace,\ekvc@ifspace@}
% A test which can be reduced to an if-empty by gobbling everything up to the
% first space.
%    \begin{macrocode}
\long\def\ekvc@ifspace#1%
  {%
    \ekvc@ifspace@#1 \ekv@ifempty@B
      \ekv@ifempty@false\ekv@ifempty@A\ekv@ifempty@B\@firstoftwo
  }
\long\def\ekvc@ifspace@#1 % keep this space
  {%
    \ekv@ifempty@\ekv@ifempty@A
  }
%    \end{macrocode}
% \end{macro}
%
%
% \subsubsection{Assertions}
% \begin{macro}[internal]{\ekvc@assert@not@long}
%   Some keys don't want to be |long| and we have to educate the user, so let's
%   throw an error if someone wanted these to be long.
%    \begin{macrocode}
\long\def\ekvc@assert@not@long#1{\ifx\ekvc@long\long\ekvc@err@no@long{#1}\fi}
%    \end{macrocode}
% \end{macro}
%
%
% \subsubsection{Messages}
%
% \begin{macro}[internal]
%   {
%     \ekvc@err@toomany,\ekvc@err@value@required,\ekvc@err@missing@type,
%     \ekvc@err@already@defined,\ekvc@err@no@key@macro
%   }
%   Boring unexpandable error messages.
%    \begin{macrocode}
\protected\def\ekvc@err@toomany#1%
  {%
    \errmessage{expkv-cs Error: Too many keys for macro `\string#1'}%
  }
\protected\def\ekvc@err@value@required#1%
  {%
    \errmessage{expkv-cs Error: Missing value for key `\unexpanded{#1}'}%
  }
\protected\def\ekvc@err@missing@type#1%
  {%
    \errmessage
      {expkv-cs Error: Missing type for secondary key `\unexpanded{#1}'}%
  }
\protected\def\ekvc@err@no@long#1%
  {%
    \errmessage
      {expkv-cs Error: prefix `long' not accepted for `\unexpanded{#1}'}%
  }
\protected\def\ekvc@err@already@defined#1%
  {%
    \errmessage{expkv-cs Error: Macro `\string#1' already defined}%
  }
\protected\def\ekvc@err@unknown@keytype#1%
  {%
    \errmessage{expkv-cs Error: Unknown key type `\unexpanded{#1}'}%
  }
\protected\def\ekvc@err@unknown@key#1%
  {%
    \errmessage
      {expkv-cs Error: Unknown key `\unexpanded{#1}' for macro `\ekvc@set'}%
  }
\long\def\ekvc@err@no@key@macro#1%
  {\errmessage{expkv-cs Error: \string#1 is no key=val macro}}
%    \end{macrocode}
% \end{macro}
%
% \begin{macro}[internal]{\ekvc@err}
%   We need a way to throw error messages expandably in some contexts.
%    \begin{macrocode}
\ekv@exparg{\long\def\ekvc@err#1}{\ekverr{expkv-cs}{#1}}
%    \end{macrocode}
% \end{macro}
%
% \begin{macro}[internal]
%   {
%     \ekvc@err@unknown@hash,\ekvc@err@empty@hash,\ekvc@err@missing@hash,
%     \ekvc@err@invalid@bool
%   }
%   And here are the expandable error messages.
%    \begin{macrocode}
\long\def\ekvc@err@unknown@hash#1{\ekvc@err{unknown hash `#1'}}
\long\def\ekvc@err@missing@hash#1{\ekvc@err{hash `#1' not found}}
\long\def\ekvc@err@empty@hash{\ekvc@err{empty hash}}
\def\ekvc@err@invalid@bool#1{\ekvc@err{invalid boolean value `#1'}}
\long\def\ekvc@err@unknown@key@or@macro#1#2%
  {\ekvc@err{unknown key `#2' for macro #1}}
%    \end{macrocode}
% \end{macro}
%
%
% Now everything that's left is to reset the category code of |@|.
%    \begin{macrocode}
\catcode`\@=\ekvc@tmp
%    \end{macrocode}
%
% \gobbledocstriptag
%</tex>
%^^A=<<
%
%^^A=<<
%
% \end{implementation}^^A=<<
%
% \clearpage
% \PrintIndex
%
\endinput
%
^^A vim: ft=tex fdm=marker fmr=>>=,=<<

\ProvidesPackage{expkv-cs}%
  [%
    \ekvcDate\space v\ekvcVersion\space
    define expandable key=val macros using expkv%
  ]
%    \end{macrocode}
% \gobbledocstriptag
%</pkg>
%^^A=<<
%^^A main file >>=
% \subsection{The Generic Code}
% The rest of this implementation will be the generic code.
% \gobbledocstriptag
%<*tex>
%
% Load \expkv\ if the package didn't already do so -- since \expkv\ has
% safeguards against being loaded twice this does no harm and the overhead
% isn't that big. Also we reuse some of the internals of \expkv\ to save us from
% retyping them.
%    \begin{macrocode}
\input expkv
%    \end{macrocode}
%
% We make sure that \file{expkv-cs.tex} is only input once:
%    \begin{macrocode}
\expandafter\ifx\csname ekvcVersion\endcsname\relax
\else
  \expandafter\endinput
\fi
%    \end{macrocode}
%
% \begin{macro}{\ekvcVersion,\ekvcDate}
% We're on our first input, so lets store the version and date in a macro.
%    \begin{macrocode}
\def\ekvcVersion{0.10}
\def\ekvcDate{2021-06-03}
%    \end{macrocode}
% \end{macro}
%
% If the \LaTeX\ format is loaded we want to be a good file and report back who
% we are, for this the package will have defined |\ekvc@tmp| to use
% |\ProvidesFile|, else this will expand to a |\relax| and do no harm.
%    \begin{macrocode}
\csname ekvc@tmp\endcsname
%    \end{macrocode}
%
% Store the category code of |@| to later be able to reset it and change it to
% 11 for now.
%    \begin{macrocode}
\expandafter\chardef\csname ekvc@tmp\endcsname=\catcode`\@
\catcode`\@=11
%    \end{macrocode}
% |\ekvc@tmp| will be reused later, but we don't need it to ever store
% information long-term after \expkvc\ was initialized.
%
% \begin{macro}[internal]{\ekvc@keycount}
%   We'll need to keep count how many keys must be defined for each macro in the
%   |split| variants.
%    \begin{macrocode}
\newcount\ekvc@keycount
%    \end{macrocode}
% \end{macro}
%
% \begin{macro}[internal]{\ekvc@long,\ekvc@any@long}
%   Some macros will have to be defined long. These two will be let to |\long|
%   when this should be the case.
%    \begin{macrocode}
\let\ekvc@long\ekv@empty
\let\ekvc@any@long\ekv@empty
%    \end{macrocode}
% \end{macro}
%
% \begin{macro}[internal]{\ekvc@ifdefined}
%   We want to test whether a macro is already defined. This test checks for a
%   defined macro that isn't |\relax|.
%    \begin{macrocode}
\long\def\ekvc@ifdefined#1%
  {%
    \ifdefined#1%
      \ifx\relax#1%
        \ekv@fi@gobble
      \fi
      \@firstofone
      \ekv@fi@firstoftwo
    \fi
    \@secondoftwo
  }
%    \end{macrocode}
% \end{macro}
%
% \begin{macro}[internal]
%   {
%     \ekvc@ekvset@pre@expander,
%     \ekvc@ekvset@pre@expander@a,
%     \ekvc@ekvset@pre@expander@b
%   }
%   This macro expands |\ekvset| twice so that the first two steps of expansion
%   don't have to be made every time the \expkvc\ macros are used. We have to do
%   a little magic trick to get the macro parameter |#1| for the macro
%   definition this is used in, even though we're calling |\unexpanded|. We do
%   that by splitting the expanded |\ekvset| at some marks and place |##1| in
%   between. At this spot we also add |\ekv@alignsafe| and |\ekv@endalignsafe|
%   to ensure that macros created with \expkvc\ are alignment safe.
%    \begin{macrocode}
\def\ekvc@ekvset@pre@expander#1%
  {%
    \expandafter\ekvc@ekvset@pre@expander@a\ekvset{#1}\ekvc@stop\ekvc@stop
  }
\def\ekvc@ekvset@pre@expander@a
  {%
    \expandafter\ekvc@ekvset@pre@expander@b
  }
\def\ekvc@ekvset@pre@expander@b#1\ekvc@stop#2\ekvc@stop
  {%
    \unexpanded\expandafter{\ekv@alignsafe}%
    \unexpanded{#1}##1\unexpanded{#2}%
    \unexpanded\expandafter{\ekv@endalignsafe}%
  }
%    \end{macrocode}
% \end{macro}
%
% \begin{macro}{\ekvcSplitAndUse}
%   The first user macro we want to set up can be reused for
%   \cs[no-index]{ekvcSplitAndForward} and \cs[no-index]{ekvcSplit}. We'll split
%   this one up so that the test whether the macro is already defined doesn't
%   run twice.
%    \begin{macrocode}
\protected\long\def\ekvcSplitAndUse#1#2%
  {%
    \let\ekvc@helpers@needed\@firstoftwo
    \ekvc@ifdefined#1%
      {\ekvc@err@already@defined#1}%
      {\ekvcSplitAndUse@#1{}{#2}}%
  }
%    \end{macrocode}
% \end{macro}
%
% \begin{macro}[internal]{\ekvcSplitAndUse@}
%   The actual macro setting up things. We need to set some variables, forward
%   the key list to |\ekvc@SetupSplitKeys|, and afterwards define the front
%   facing macro to call |\ekvset| and put the initials and the argument sorting
%   macro behind it. The internals |\ekvc@any@long|, |\ekvc@initials| and
%   |\ekvc@keycount| will be set correctly by |\ekvc@SetupSplitKeys|.
%    \begin{macrocode}
\protected\long\def\ekvcSplitAndUse@#1#2#3%
  {%
    \edef\ekvc@set{\string#1}%
    \ekvc@SetupSplitKeys{#3}%
    \ekvc@helpers@needed
      {%
        \ekvc@any@long\edef#1##1%
          {%
            \expandafter\ekvc@ekvset@pre@expander\expandafter{\ekvc@set}%
            \unexpanded\expandafter
              {\csname ekvc@split@\the\ekvc@keycount\endcsname}%
            \unexpanded\expandafter{\ekvc@initials{}#2}%
          }%
      }%
      {%
        \ekvc@any@long\edef#1##1%
          {%
            \expandafter\ekvc@ekvset@pre@expander\expandafter{\ekvc@set}%
            \unexpanded{#2}%
            \unexpanded\expandafter{\ekvc@initials}%
          }%
      }%
  }
%    \end{macrocode}
% \end{macro}
%
%
% \begin{macro}{\ekvcSplitAndForward}
%   This just reuses |\ekvcSplitAndUse@| with a non-empty second argument,
%   resulting in that argument to be called after the splitting.
%    \begin{macrocode}
\protected\long\def\ekvcSplitAndForward#1#2#3%
  {%
    \let\ekvc@helpers@needed\@firstoftwo
    \ekvc@ifdefined#1%
      {\ekvc@err@already@defined#1}%
      {\ekvcSplitAndUse@#1{{#2}}{#3}}%
  }
%    \end{macrocode}
% \end{macro}
%
% \begin{macro}{\ekvcSplit}
%   The first half is just |\ekvcSplitAndForward| then we define the macro to
%   which the parsed key list is forwarded. There we need to allow for up to
%   nine arguments.
%    \begin{macrocode}
\protected\long\def\ekvcSplit#1#2#3%
  {%
    \let\ekvc@helpers@needed\@secondoftwo
    \ekvc@ifdefined#1%
      {\ekvc@err@already@defined#1}%
      {%
        \expandafter
        \ekvcSplitAndUse@\expandafter#1\csname ekvc@\string#1\endcsname{#2}%
        \ifnum\ekvc@keycount<1
          \ekvc@any@long\expandafter\def\csname ekvc@\string#1\endcsname{#3}%
        \else
          \ifnum\ekvc@keycount>9
            \ekvc@err@toomany{#1}%
            \let#1\ekvc@undefined
          \else
            \ekvcSplit@build@argspec
            \ekvc@any@long\expandafter
            \def\csname ekvc@\string#1\expandafter\endcsname\ekvc@tmp{#3}%
          \fi
        \fi
      }%
  }
%    \end{macrocode}
% \end{macro}
%
% \begin{macro}[internal]{\ekvcSplit@build@argspec,\ekvcSplit@build@argspec@}
%    \begin{macrocode}
\protected\def\ekvcSplit@build@argspec
  {%
    \begingroup
      \edef\ekvc@tmp
        {\endgroup\def\unexpanded{\ekvc@tmp}{\ekvcSplit@build@argspec@{1}}}%
    \ekvc@tmp
  }
\def\ekvcSplit@build@argspec@#1%
  {%
    \ifnum#1>\ekvc@keycount
      \ekv@fi@gobble
    \fi
    \@firstofone
      {%
        \unexpanded\expandafter{\csname ekvc@splitmark@#1\endcsname####}#1%
        \expandafter\ekvcSplit@build@argspec@\expandafter{\the\numexpr#1+1}%
      }%
  }
%    \end{macrocode}
% \end{macro}
%
% \begin{macro}[internal]
%   {
%     \ekvc@SetupSplitKeys, \ekvc@SetupSplitKeys@a, \ekvc@SetupSplitKeys@b,
%     \ekvc@SetupSplitKeys@c
%   }
%   These macros parse the list of keys and set up the key macros. First we need
%   to initialise some macros and start |\ekvparse|.
%    \begin{macrocode}
\protected\long\def\ekvc@SetupSplitKeys
  {%
    \ekvc@keycount=0
    \let\ekvc@any@long\ekv@empty
    \let\ekvc@initials\ekv@empty
    \ekvparse\ekvc@err@value@required\ekvc@SetupSplitKeys@a
  }
%    \end{macrocode}
%   Then we need to step the key counter for each key. Also we have to check
%   whether this key has a |long| prefix so we initialise |\ekvc@long|.
%    \begin{macrocode}
\protected\def\ekvc@SetupSplitKeys@a#1%
  {%
    \advance\ekvc@keycount1
    \let\ekvc@long\ekv@empty
    \ekvc@ifspace{#1}%
      {\ekvc@SetupSplitKeys@b#1\ekvc@stop}%
      {\ekvc@SetupSplitKeys@c{#1}}%
  }
%    \end{macrocode}
%   If there was a space, there might be a prefix. If so call the prefix macro,
%   else call the next step |\ekvc@SetupSplitKeys@c| which will define the key
%   macro and add the key's value to the initials list.
%    \begin{macrocode}
\protected\def\ekvc@SetupSplitKeys@b#1 #2\ekvc@stop
  {%
    \ekv@ifdefined{ekvc@split@p@#1}%
      {\csname ekvc@split@p@#1\endcsname{#2}}%
      {\ekvc@SetupSplitKeys@c{#1 #2}}%
  }
%    \end{macrocode}
%   The inner definition is grouped, because we don't want to actually define
%   the marks we build with |\csname|. We have to append the value to the
%   |\ekvc@initials| list here with the correct split mark. The key macro will
%   read everything up to those split marks and change the value following it to
%   the value given to the key. Additionally we'll need a sorting macro for each
%   key count in use so we set it up with |\ekvc@setup@splitmacro|.
%    \begin{macrocode}
\protected\long\def\ekvc@SetupSplitKeys@c#1#2%
  {%
    \begingroup
    \edef\ekvc@tmp
      {%
        \endgroup
        \long\def\unexpanded{\ekvc@tmp}####1####2%
            \unexpanded\expandafter
            {\csname ekvc@splitmark@\the\ekvc@keycount\endcsname}####3%
          {%
            ####2%
            \unexpanded\expandafter
              {\csname ekvc@splitmark@\the\ekvc@keycount\endcsname}{####1}%
          }%
%    \end{macrocode}
%   The short variant needs a bit of special treatment. The key macro will be
%   short to throw the correct error, but since there might be long macros
%   somewhere the reordering of arguments needs to be long, so for short keys we
%   use a two step approach, first grabbing only the short argument, then
%   reordering.
%    \begin{macrocode}
        \unless\ifx\ekvc@long\long
          \let\unexpanded\expandafter
            {\csname ekvc@\ekvc@set(#1)\endcsname\ekvc@tmp}%
          \def\unexpanded{\ekvc@tmp}####1%
            {%
              \unexpanded\expandafter{\csname ekvc@\ekvc@set(#1)\endcsname}%
                {####1}%
            }%
        \fi
        \def\unexpanded{\ekvc@initials}%
          {%
            \unexpanded\expandafter{\ekvc@initials}%
            \unexpanded\expandafter
              {\csname ekvc@splitmark@\the\ekvc@keycount\endcsname{#2}}%
          }%
      }%
    \ekvc@tmp
    \ekvlet\ekvc@set{#1}\ekvc@tmp
    \ekvc@helpers@needed
      {\expandafter\ekvc@setup@splitmacro\expandafter{\the\ekvc@keycount}}%
      {}%
  }
%    \end{macrocode}
% \end{macro}
%
% \begin{macro}[internal]{\ekvc@split@p@long}
%   The |long| prefix lets the internals |\ekvc@long| and |\ekvc@any@long| to
%   |\long| so that the key macro will be long.
%    \begin{macrocode}
\protected\def\ekvc@split@p@long
  {%
    \let\ekvc@long\long
    \let\ekvc@any@long\long
    \ekvc@SetupSplitKeys@c
  }
%    \end{macrocode}
% \end{macro}
%
% \begin{macro}[internal]{\ekvc@defarggobbler}
%   This is needed to define a macro with 1-9 parameters programmatically.
%   \LaTeX's \cs[no-index]{newcommand} does something similar for example.
%    \begin{macrocode}
\protected\def\ekvc@defarggobbler#1{\def\ekvc@tmp##1#1##2##{##1#1}}
%    \end{macrocode}
% \end{macro}
%
% \begin{macro}[internal]
%   {
%     \ekvc@setup@splitmacro,
%     \ekvc@split@1, \ekvc@split@2, \ekvc@split@3,
%     \ekvc@split@4, \ekvc@split@5, \ekvc@split@6,
%     \ekvc@split@7
%   }
%   Since the first few split macros are different from the others we manually
%   set those up now. All the others will be defined as needed (always
%   globally).  The split macros just read up until the correct split mark, move
%   that argument into a list and reinsert the rest, calling the next split
%   macro afterwards.
%    \begin{macrocode}
\begingroup
\edef\ekvc@tmp
  {%
    \long\gdef\unexpanded\expandafter{\csname ekvc@split@1\endcsname}%
        \unexpanded\expandafter{\csname ekvc@splitmark@1\endcsname}%
        ##1##2##3%
      {##3{##1}##2}%
    \long\gdef\unexpanded\expandafter{\csname ekvc@split@2\endcsname}%
        \unexpanded\expandafter{\csname ekvc@splitmark@1\endcsname}##1%
        \unexpanded\expandafter{\csname ekvc@splitmark@2\endcsname}##2%
        ##3##4%
      {##4{##1}{##2}##3}%
    \long\gdef\unexpanded\expandafter{\csname ekvc@split@3\endcsname}%
        \unexpanded\expandafter{\csname ekvc@splitmark@1\endcsname}##1%
        \unexpanded\expandafter{\csname ekvc@splitmark@2\endcsname}##2%
        \unexpanded\expandafter{\csname ekvc@splitmark@3\endcsname}##3%
        ##4##5%
      {##5{##1}{##2}{##3}##4}%
    \long\gdef\unexpanded\expandafter{\csname ekvc@split@4\endcsname}%
        \unexpanded\expandafter{\csname ekvc@splitmark@1\endcsname}##1%
        \unexpanded\expandafter{\csname ekvc@splitmark@2\endcsname}##2%
        \unexpanded\expandafter{\csname ekvc@splitmark@3\endcsname}##3%
        \unexpanded\expandafter{\csname ekvc@splitmark@4\endcsname}##4%
        ##5##6%
      {##6{##1}{##2}{##3}{##4}##5}%
    \long\gdef\unexpanded\expandafter{\csname ekvc@split@5\endcsname}%
        \unexpanded\expandafter{\csname ekvc@splitmark@1\endcsname}##1%
        \unexpanded\expandafter{\csname ekvc@splitmark@2\endcsname}##2%
        \unexpanded\expandafter{\csname ekvc@splitmark@3\endcsname}##3%
        \unexpanded\expandafter{\csname ekvc@splitmark@4\endcsname}##4%
        \unexpanded\expandafter{\csname ekvc@splitmark@5\endcsname}##5%
        ##6##7%
      {##7{##1}{##2}{##3}{##4}{##5}##6}%
    \long\gdef\unexpanded\expandafter{\csname ekvc@split@6\endcsname}%
        \unexpanded\expandafter{\csname ekvc@splitmark@1\endcsname}##1%
        \unexpanded\expandafter{\csname ekvc@splitmark@2\endcsname}##2%
        \unexpanded\expandafter{\csname ekvc@splitmark@3\endcsname}##3%
        \unexpanded\expandafter{\csname ekvc@splitmark@4\endcsname}##4%
        \unexpanded\expandafter{\csname ekvc@splitmark@5\endcsname}##5%
        \unexpanded\expandafter{\csname ekvc@splitmark@6\endcsname}##6%
        ##7##8%
      {##8{##1}{##2}{##3}{##4}{##5}{##6}##7}%
    \long\gdef\unexpanded\expandafter{\csname ekvc@split@7\endcsname}%
        \unexpanded\expandafter{\csname ekvc@splitmark@1\endcsname}##1%
        \unexpanded\expandafter{\csname ekvc@splitmark@2\endcsname}##2%
        \unexpanded\expandafter{\csname ekvc@splitmark@3\endcsname}##3%
        \unexpanded\expandafter{\csname ekvc@splitmark@4\endcsname}##4%
        \unexpanded\expandafter{\csname ekvc@splitmark@5\endcsname}##5%
        \unexpanded\expandafter{\csname ekvc@splitmark@6\endcsname}##6%
        \unexpanded\expandafter{\csname ekvc@splitmark@7\endcsname}##7%
        ##8##9%
      {##9{##1}{##2}{##3}{##4}{##5}{##6}{##7}##8}%
  }
\ekvc@tmp
\endgroup
\protected\def\ekvc@setup@splitmacro#1%
  {%
    \ekv@ifdefined{ekvc@split@#1}{}%
      {%
        \begingroup
          \edef\ekvc@tmp
            {%
              \long\gdef
                  \unexpanded\expandafter{\csname ekvc@split@#1\endcsname}%
                  ####1%
                  \unexpanded\expandafter{\csname ekvc@splitmark@#1\endcsname}%
                  ####2####3%
                {%
                  \unexpanded\expandafter
                    {\csname ekvc@split@\the\numexpr#1-1\relax\endcsname}%
                  ####1{{####2}####3}%
                }%
            }%
          \ekvc@tmp
        \endgroup
      }%
  }
%    \end{macrocode}
% \end{macro}
%
% \begin{macro}{\ekvcHashAndUse}
%   |\ekvcHashAndUse| works just like |\ekvcSplitAndUse|.
%    \begin{macrocode}
\protected\long\def\ekvcHashAndUse#1#2%
  {%
    \let\ekvc@helpers@needed\@firstoftwo
    \ekvc@ifdefined#1%
      {\ekvc@err@already@defined#1}%
      {\ekvcHashAndUse@#1{}{#2}}%
  }
%    \end{macrocode}
% \end{macro}
%
% \begin{macro}[internal]{\ekvcHashAndUse@}
%   This is more or less the same as |\ekvcSplitAndUse@|. Instead of an empty
%   group we place a marker after the initials, we don't use the sorting macros
%   of |split|, but instead pack all the values in one argument.
%    \begin{macrocode}
\protected\long\def\ekvcHashAndUse@#1#2#3%
  {%
    \edef\ekvc@set{\string#1}%
    \ekvc@SetupHashKeys{#3}%
    \ekvc@helpers@needed
      {%
        \ekvc@any@long\edef#1##1%
          {%
            \expandafter\ekvc@ekvset@pre@expander\expandafter{\ekvc@set}%
            \unexpanded{\ekvc@hash@pack@argument}%
            \unexpanded\expandafter{\ekvc@initials\ekvc@stop#2}%
          }%
      }%
      {%
        \ekvc@any@long\edef#1##1%
          {%
            \expandafter\ekvc@ekvset@pre@expander\expandafter{\ekvc@set}%
            \unexpanded{#2}%
            \unexpanded\expandafter{\ekvc@initials\ekvc@stop}%
          }%
      }%
  }
%    \end{macrocode}
% \end{macro}
%
% \begin{macro}{\ekvcHashAndForward}
%   |\ekvcHashAndForward| works just like |\ekvcSplitAndForward|.
%    \begin{macrocode}
\protected\long\def\ekvcHashAndForward#1#2#3%
  {%
    \let\ekvc@helpers@needed\@firstoftwo
    \ekvc@ifdefined#1%
      {\ekvc@err@already@defined#1}%
      {\ekvcHashAndUse@#1{{#2}}{#3}}%
  }
%    \end{macrocode}
% \end{macro}
%
% \begin{macro}{\ekvcHash}
%   |\ekvcHash| does the same as |\ekvcSplit|, but has the advantage of not
%   needing to count arguments, so the definition of the internal macro is a bit
%   more straight forward.
%    \begin{macrocode}
\protected\long\def\ekvcHash#1#2#3%
  {%
    \let\ekvc@helpers@needed\@secondoftwo
    \ekvc@ifdefined#1%
      {\ekvc@err@already@defined#1}%
      {%
        \expandafter
        \ekvcHashAndUse@\expandafter#1\csname ekvc@\string#1\endcsname{#2}%
        \ekvc@any@long\expandafter\def\csname ekvc@\string#1\endcsname
            ##1\ekvc@stop
          {#3}%
      }%
  }
%    \end{macrocode}
% \end{macro}
%
% \begin{macro}[internal]{\ekvc@hash@pack@argument}
%   All this macro does is pack the values into one argument and forward that to
%   the next macro.
%    \begin{macrocode}
\long\def\ekvc@hash@pack@argument#1\ekvc@stop#2{#2{#1}}
%    \end{macrocode}
% \end{macro}
%
% \begin{macro}[internal]
%   {\ekvc@SetupHashKeys,\ekvc@SetupHashKeys@a,\ekvc@SetupHashKeys@b}
%   This should look awfully familiar as well, since it's just the same as for
%   the split keys with a few other names here and there.
%    \begin{macrocode}
\protected\long\def\ekvc@SetupHashKeys#1%
  {%
    \let\ekvc@any@long\ekv@empty
    \let\ekvc@initials\ekv@empty
    \ekvparse\ekvc@err@value@required\ekvc@SetupHashKeys@a{#1}%
  }
\protected\def\ekvc@SetupHashKeys@a#1%
  {%
    \let\ekvc@long\ekv@empty
    \ekvc@ifspace{#1}%
      {\ekvc@SetupHashKeys@b#1\ekvc@stop}%
      {\ekvc@SetupHashKeys@c{#1}}%
  }
\protected\def\ekvc@SetupHashKeys@b#1 #2\ekvc@stop
  {%
    \ekv@ifdefined{ekvc@hash@p@#1}%
      {\csname ekvc@hash@p@#1\endcsname{#2}}%
      {\ekvc@SetupHashKeys@c{#1 #2}}%
  }
%    \end{macrocode}
%   Yes, even the defining macro looks awfully familiar. Instead of numbered we
%   have named marks. Still the key macros grab everything up to their
%   respective mark and reorder the arguments. The same quirk is applied for
%   short keys. And instead of the |\ekvc@setup@splitmacro| we use
%   |\ekvc@setup@hashmacro|.
%    \begin{macrocode}
\protected\long\def\ekvc@SetupHashKeys@c#1#2%
  {%
    \begingroup
    \edef\ekvc@tmp
      {%
        \endgroup
        \long\def\unexpanded{\ekvc@tmp}####1####2%
            \unexpanded\expandafter{\csname ekvc@hashmark@#1\endcsname}####3%
          {%
            ####2%
            \unexpanded\expandafter{\csname ekvc@hashmark@#1\endcsname}{####1}%
          }%
        \unless\ifx\ekvc@long\long
          \let\unexpanded\expandafter
            {\csname ekvc@\ekvc@set(#1)\endcsname\ekvc@tmp}%
          \def\unexpanded{\ekvc@tmp}####1%
            {%
              \unexpanded\expandafter{\csname ekvc@\ekvc@set(#1)\endcsname}%
                {####1}%
            }%
        \fi
        \def\unexpanded{\ekvc@initials}%
          {%
            \unexpanded\expandafter{\ekvc@initials}%
            \unexpanded\expandafter{\csname ekvc@hashmark@#1\endcsname{#2}}%
          }%
      }%
    \ekvc@tmp
    \ekvlet\ekvc@set{#1}\ekvc@tmp
    \ekvc@setup@hashmacro{#1}%
  }
%    \end{macrocode}
% \end{macro}
%
% \begin{macro}[internal]{\ekvc@hash@p@long}
%   Nothing astonishing here either.
%    \begin{macrocode}
\protected\def\ekvc@hash@p@long
  {%
    \let\ekvc@long\long
    \let\ekvc@any@long\long
    \ekvc@SetupHashKeys@c
  }
%    \end{macrocode}
% \end{macro}
%
% \begin{macro}[internal]{\ekvc@setup@hashmacro}
%   The safe hash macros will be executed inside of an |\unexpanded| expansion
%   context, so they have to insert braces for that once they are done.
%   Most of the tests which have to be executed will already be done, but we
%   have to play safe if the hash doesn't show up in the hash list. Therefore we
%   use some |\ekvc@mark|s and |\ekvc@stop| to throw errors if the hash isn't
%   found in the right place. The fast variants have an easier life and just
%   return the correct value.
%    \begin{macrocode}
\protected\def\ekvc@setup@hashmacro#1%
  {%
    \ekv@ifdefined{ekvc@fasthash@#1}{}%
      {%
        \begingroup
          \edef\ekvc@tmp
            {%
              \long\gdef
                \unexpanded\expandafter{\csname ekvc@fasthash@#1\endcsname}%
                  ####1%
                  \unexpanded\expandafter{\csname ekvc@hashmark@#1\endcsname}%
                  ####2####3\unexpanded{\ekvc@stop}%
                {####2}%
              \long\gdef
                \unexpanded\expandafter{\csname ekvc@safehash@#1\endcsname}%
                  ####1%
                {%
                  \unexpanded\expandafter{\csname ekvc@@safehash@#1\endcsname}%
                  ####1\unexpanded{\ekvc@mark}{}%
                  \unexpanded\expandafter
                    {%
                      \csname ekvc@hashmark@#1\endcsname{}%
                      \ekvc@mark{\ekvc@err@missing@hash{#1}}\ekvc@stop
                    }%
                }%
              \long\gdef
                \unexpanded\expandafter{\csname ekvc@@safehash@#1\endcsname}%
                  ####1%
                  \unexpanded\expandafter{\csname ekvc@hashmark@#1\endcsname}%
                  ####2####3\unexpanded{\ekvc@mark}####4####5%
                  \unexpanded{\ekvc@stop}%
                {%
                  ####4{####2}%
                }%
              \long\gdef\unexpanded\expandafter
                {\csname ekvc@fastsplithash@#1\endcsname}%
                  ####1%
                  \unexpanded\expandafter{\csname ekvc@hashmark@#1\endcsname}%
                  ####2####3\unexpanded{\ekvc@stop}####4%
                {%
                  ####4{####1####3}{####2}%
                }%
              \long\gdef\unexpanded\expandafter
                {\csname ekvc@safesplithash@#1\endcsname}####1%
                {%
                  \unexpanded\expandafter
                    {\csname ekvc@@safesplithash@#1\endcsname}%
                  ####1\unexpanded{\ekvc@mark\ekvc@safe@found@hash}%
                  \unexpanded\expandafter
                    {%
                      \csname ekvc@hashmark@#1\endcsname{}%
                      \ekvc@mark{\ekvc@err@missing@hash{#1}\ekvc@safe@no@hash}%
                      \ekvc@stop
                    }%
                }%
              \long\gdef\unexpanded\expandafter
                {\csname ekvc@@safesplithash@#1\endcsname}%
                  ####1%
                  \unexpanded\expandafter{\csname ekvc@hashmark@#1\endcsname}%
                  ####2####3\unexpanded{\ekvc@mark}####4####5%
                  \unexpanded{\ekvc@stop}%
                {%
                  ####4{####2}####1####3\unexpanded{\ekvc@stop}%
                }%
            }%
          \ekvc@tmp
        \endgroup
      }%
  }
%    \end{macrocode}
% \end{macro}
%
% \begin{macro}{\ekvcValue}
%   All this does is a few consistency checks on the first argument (not empty,
%   hash macro exists) and then call that hash-grabbing macro that will also
%   test whether the hash is inside of |#2| or not.
%    \begin{macrocode}
\long\def\ekvcValue#1%
  {%
    \unexpanded
    \ekv@ifdefined{ekvc@safehash@#1}%
      {\csname ekvc@safehash@#1\endcsname}%
      {\ekvc@err@unknown@hash{#1}\@firstoftwo{{}}}%
  }
%    \end{macrocode}
% \end{macro}
%
% \begin{macro}{\ekvcValueFast}
%   To be as fast as possible, this doesn't test for anything, assuming the user
%   knows best.
%    \begin{macrocode}
\long\def\ekvcValueFast#1#2{\csname ekvc@fasthash@#1\endcsname#2\ekvc@stop}
%    \end{macrocode}
% \end{macro}
%
% \begin{macro}{\ekvcValueSplit,\ekvcValueSplit@recover}
%   This splits off a single value.
%    \begin{macrocode}
\long\def\ekvcValueSplit#1%
  {%
    \ekv@ifdefined{ekvc@safesplithash@#1}%
      {\csname ekvc@safesplithash@#1\endcsname}%
      {\ekvc@err@unknown@hash{#1}\ekvcValueSplit@recover}%
  }
\long\def\ekvcValueSplit@recover#1#2%
  {%
    #2{#1}{}%
  }
%    \end{macrocode}
% \end{macro}
%
% \begin{macro}[internal]{\ekvc@safe@found@hash, \ekvc@safe@no@hash}
%    \begin{macrocode}
\long\def\ekvc@safe@found@hash#1#2\ekvc@stop#3%
  {%
    #3{#2}{#1}%
  }
\long\def\ekvc@safe@no@hash#1#2\ekvc@mark\ekvc@safe@found@hash\ekvc@stop#3%
  {%
    #3{#2}{}%
  }
%    \end{macrocode}
% \end{macro}
%
% \begin{macro}{\ekvcValueSplitFast}
%   Again a fast approach which doesn't provide too many safety measurements.
%   This needs to build the hash function and expand it before passing the
%   results to the next control sequence. The first step only builds the control
%   sequence.
%    \begin{macrocode}
\long\def\ekvcValueSplitFast#1#2%
  {%
    \csname ekvc@fastsplithash@#1\endcsname#2\ekvc@stop
  }
%    \end{macrocode}
% \end{macro}
%
% \begin{macro}[internal]
%   {
%     \ekvc@safehash@,\ekvc@fasthash@,
%     \ekvc@safesplithash@,\ekvc@fastsplithash@
%   }
%   At least in the empty hash case we can provide a meaningful error message
%   without affecting performance by just defining the macro that would be build
%   in that case. There is of course a downside to this, the error will not be
%   thrown by |\ekvcValueFast| in three expansion steps. The safe hash variant
%   has to also stop the |\unexpanded| expansion.
%    \begin{macrocode}
\long\def\ekvc@safehash@#1{\ekvc@err@empty@hash{}}
\long\def\ekvc@fasthash@#1\ekvc@stop{\ekvc@err@empty@hash}
\long\def\ekvc@safesplithash@#1#2{\ekvc@err@empty@hash#2{#1}{}}
\long\def\ekvc@fastsplithash@#1\ekvc@stop#2{\ekvc@err@empty@hash#2{#1}{}}
%    \end{macrocode}
% \end{macro}
%
% \begin{macro}{\ekvcSecondaryKeys}
%   The secondary keys are defined pretty similar to the way the originals are,
%   but here we also introduce some key types (those have a |@t@| in their name)
%   additionally to the prefixes.
%    \begin{macrocode}
\protected\long\def\ekvcSecondaryKeys#1#2%
  {%
    \edef\ekvc@set{\string#1}%
    \ekvparse\ekvc@err@value@required\ekvcSecondaryKeys@a{#2}%
  }
\protected\def\ekvcSecondaryKeys@a#1%
  {%
    \let\ekvc@long\ekv@empty
    \ekvc@ifspace{#1}%
      {\ekvcSecondaryKeys@b#1\ekvc@stop}%
      {\ekvc@err@missing@type{#1}\@gobble}%
  }
\protected\def\ekvcSecondaryKeys@b#1 #2\ekvc@stop
  {%
    \ekv@ifdefined{ekvc@p@#1}%
      {\csname ekvc@p@#1\endcsname}%
      {%
        \ekv@ifdefined{ekvc@t@#1}%
          {\csname ekvc@t@#1\endcsname}%
          {\ekvc@err@unknown@keytype{#1}\@firstoftwo\@gobble}%
      }%
      {#2}%
  }
%    \end{macrocode}
% \end{macro}
%
% \begin{macro}{\ekvcChange}
%   This can be used to change the defaults of an \expkvc\ defined macro. It
%   checks whether there is a set with the correct name and that the macro is
%   defined. If both is true the real work is done by |\ekvc@change|.
%    \begin{macrocode}
\protected\long\def\ekvcChange#1%
  {%
    \ekvifdefinedset{\string#1}%
      {%
        \ekvc@ifdefined#1%
          {\ekvc@change#1}%
          {\ekvc@err@no@key@macro#1\@gobble}%
      }%
      {\ekvc@err@no@key@macro#1\@gobble}%
  }
%    \end{macrocode}
% \end{macro}
%
% \begin{macro}[internal]
%   {\ekvc@change,\ekvc@change@a,\ekvc@change@b,\ekvc@change@c}
%   First we need to see whether the macro is currently |\long|. For this we get
%   the meaning and will parse it. |#1| is the macro name in which we want to
%   change the defaults.
%    \begin{macrocode}
\protected\def\ekvc@change#1%
  {\expandafter\ekvc@change@a\meaning#1\ekv@stop#1}
%    \end{macrocode}
%   A temporary definition to get the stringified |macro:|. |##1| will be the
%   list of prefixes, we don't care for the exact contents of |##2| and |##3|.
%    \begin{macrocode}
\def\ekvc@change@a#1%
  {%
    \protected\def\ekvc@change@a##1#1##2->##3\ekv@stop
      {%
        \ekvc@change@iflong{##1}%
          {\ekvc@change@b{}}%
          {\ekvc@change@b{\long}}%
      }%
  }
\expandafter\ekvc@change@a\expandafter{\detokenize{macro:}}
%    \end{macrocode}
%   Next we expand the macro once to get its contents (including the current
%   default values with their markers). |#1| is either |\long| or empty, |#2| is
%   the macro.
%    \begin{macrocode}
\protected\def\ekvc@change@b#1#2%
  {\expandafter\ekvc@change@c\expandafter{#2{##1}}{#1}#2}
%    \end{macrocode}
%   Here we place an unbalanced closing brace after the expansion of the macro.
%   Then we just parse the \kv-list with |\ekvset|, that will exchange the
%   values behind the markers. Once those are changed we give control to
%   |\ekvc@change@d|. The |\ekvset| step might horribly fail if the user defined
%   some keys that don't behave nice. |#1| is the expansion of the macro, |#2|
%   is either |\long| or empty, |#3| is the macro, and |#4| is the \kv-list
%   containing the new defaults.
%    \begin{macrocode}
\ekv@exparg{\protected\long\def\ekvc@change@c#1#2#3#4}%
  {%
    \expandafter\iffalse\expandafter{\expandafter\fi
      \ekvset{\string#3}{#4}%
      \ekvc@change@d{#2}{#3}%
      #1%
    }%
  }
%    \end{macrocode}
%   The final step needs to put an unbalanced opening brace after |\def|. We do
%   that with the help of a temporary macro which stores everything necessary
%   for |\def| and expand an |\iffalse}\fi| construct to leave a single
%   opening brace. |#1| will be either empty or |\long| and |#2| is the macro.
%   Each of the macros defined with \expkvc\ takes exactly one parameter, so we
%   put that here as |####1|.
%    \begin{macrocode}
\protected\def\ekvc@change@d#1#2%
  {%
    \def\ekvc@tmp{#1\def#2####1}%
    \expandafter\ekvc@tmp\expandafter{\iffalse}\fi
  }
%    \end{macrocode}
% \end{macro}
%
% \begin{macro}[internal]{\ekvc@change@iflong,\ekvc@change@iflong@}
%   Checking whether a string contains the string representation of |\long| can
%   be done by gobbling everything up to the first |\long| and checking whether
%   the result is completely empty. We need a temporary macro to get the result
%   of |\string\long| inside the definitions.
%    \begin{macrocode}
\def\ekvc@change@iflong#1%
  {%
    \protected\def\ekvc@change@iflong##1%
      {\expandafter\ekv@ifempty\expandafter{\ekvc@change@iflong@##1#1}}%
    \def\ekvc@change@iflong@##1#1{}
  }
\expandafter\ekvc@change@iflong\expandafter{\string\long}
%    \end{macrocode}
% \end{macro}
%
% \begin{macro}{\ekvcPass}
%   This macro can be used to pass a value to a key of some macro (this way
%   more complicated key codes are possible that in the end pass processed
%   values on to some macro). The implemantation is pretty straight forward.
%    \begin{macrocode}
\long\def\ekvcPass#1#2%
  {%
    \ekvifdefined{\string#1}{#2}%
      {\csname\ekv@name{\string#1}{#2}\endcsname}%
      {\ekvc@err@unknown@key@or@macro{#1}{#2}\@gobble}%
  }
%    \end{macrocode}
% \end{macro}
%
%
% \subsubsection{Secondary Key Types}
%
% \begin{macro}[internal]{\ekvc@p@long,\ekvc@after@ptype}
%   The prefixes are pretty straight forward again. Just set |\ekvc@long| and
%   forward to the |@t@| type.
%    \begin{macrocode}
\protected\def\ekvc@p@long#1%
  {%
    \ekvc@ifspace{#1}%
      {%
        \let\ekvc@long\long
        \ekvc@after@ptype#1\ekvc@stop
      }%
      {\ekvc@err@missing@type{long #1}\@gobble}%
  }
\protected\def\ekvc@after@ptype#1 #2\ekvc@stop
  {%
    \ekv@ifdefined{ekvc@t@#1}%
      {\csname ekvc@t@#1\endcsname{#2}}%
      {\ekvc@err@unknown@keytype{#1}\@gobble}%
  }
%    \end{macrocode}
% \end{macro}
%
% \begin{macro}[internal]
%   {
%     \ekvc@t@meta, \ekvc@t@nmeta,
%     \ekvc@type@meta, \ekvc@type@meta@a, \ekvc@type@meta@b
%   }
%   The |meta| and |nmeta| key types use a nested |\ekvset| to set other keys in
%   the same macro's \set.
%    \begin{macrocode}
\protected\def\ekvc@t@meta
  {%
    \edef\ekvc@tmp{\ekvc@set}%
    \expandafter\ekvc@type@meta\expandafter{\ekvc@tmp}\ekvc@long{##1}\ekvlet
  }
\protected\def\ekvc@t@nmeta#1%
  {%
    \ekvc@assert@not@long{nmeta #1}%
    \edef\ekvc@tmp{\ekvc@set}%
    \expandafter\ekvc@type@meta\expandafter{\ekvc@tmp}{}{}\ekvletNoVal{#1}%
  }
\protected\long\def\ekvc@type@meta#1#2#3#4#5#6%
  {%
    \expandafter\ekvc@type@meta@a\expandafter{\ekvset{#1}{#6}}{#2}{#3}%
    #4\ekvc@set{#5}\ekvc@tmp
  }
\protected\def\ekvc@type@meta@a
  {%
    \expandafter\ekvc@type@meta@b\expandafter
  }
\protected\long\def\ekvc@type@meta@b#1#2#3%
  {%
    #2\def\ekvc@tmp#3{#1}%
  }
%    \end{macrocode}
% \end{macro}
%
% \begin{macro}[internal]{\ekvc@t@alias}
%   |alias| just checks whether there is a key and/or |NoVal| key defined with
%   the target name and |\let| the key to those.
%    \begin{macrocode}
\protected\def\ekvc@t@alias#1#2%
  {%
    \ekvc@assert@not@long{alias #1}%
    \let\ekvc@tmp\@firstofone
    \ekvifdefined\ekvc@set{#2}%
      {%
        \ekvletkv\ekvc@set{#1}\ekvc@set{#2}%
        \let\ekvc@tmp\@gobble
      }%
      {}%
    \ekvifdefinedNoVal\ekvc@set{#2}%
      {%
        \ekvletkvNoVal\ekvc@set{#1}\ekvc@set{#2}%
        \let\ekvc@tmp\@gobble
      }%
      {}%
    \ekvc@tmp{\ekvc@err@unknown@key{#2}}%
  }
%    \end{macrocode}
% \end{macro}
%
% \begin{macro}[internal]{\ekvc@t@default}
%   The |default| key can be used to set a |NoVal| key for an existing key. It
%   will just pass the \val\ to the key macro of that other key.
%    \begin{macrocode}
\protected\long\def\ekvc@t@default#1#2%
  {%
    \ekvifdefined\ekvc@set{#1}%
      {%
        \ekvc@assert@not@long{default #1}%
        \edef\ekvc@tmp
          {%
            \unexpanded\expandafter
              {\csname\ekv@name\ekvc@set{#1}\endcsname{#2}}%
          }%
        \ekvletNoVal\ekvc@set{#1}\ekvc@tmp
      }%
      {\ekvc@err@unknown@key{#1}}%
  }
%    \end{macrocode}
% \end{macro}
%
% \begin{macro}[internal]{\ekvc@t@flag-bool}
%    \begin{macrocode}
\protected\expandafter\def\csname ekvc@t@flag-bool\endcsname#1#2%
  {%
    \ekvc@assert@not@long{flag-bool #1}%
    \unless\ifdefined#2\ekvcFlagNew#2\fi
    \ekvdef\ekvc@set{#1}%
      {%
        \ekv@ifdefined{ekvc@flag@set@##1}%
          {%
            \csname ekvc@flag@set@##1\expandafter\endcsname
              \ekvcFlagHeight#2\ekv@stop#2%
          }%
          {\ekvc@err@invalid@bool{##1}}%
      }%
  }
%    \end{macrocode}
% \end{macro}
%
% \begin{macro}[internal]
%   {\ekvc@t@flag-true,\ekvc@t@flag-false,\ekvc@t@flag-raise,\ekvc@type@flag}
%    \begin{macrocode}
\protected\def\ekvc@type@flag#1#2#3#4%
  {%
    \ekvc@assert@not@long{flag-#1 #3}%
    \unless\ifdefined#4\ekvcFlagNew#4\fi
    \ekv@exparg{\ekvdefNoVal\ekvc@set{#3}}{#2#4}%
  }
\protected\expandafter\def\csname ekvc@t@flag-true\endcsname
  {\ekvc@type@flag{true}\ekvcFlagSetTrue}
\protected\expandafter\def\csname ekvc@t@flag-false\endcsname
  {\ekvc@type@flag{false}\ekvcFlagSetFalse}
\protected\expandafter\def\csname ekvc@t@flag-raise\endcsname
  {\ekvc@type@flag{raise}\ekvcFlagRaise}
%    \end{macrocode}
% \end{macro}
%
%
% \subsubsection{Flags}
%
% The basic idea of flags is to store information by the fact that \TeX\
% expandably assigns the meaning |\relax| to undefined control sequences which
% were built with |\csname|. This mechanism is borrowed from \pkg{expl3}.
%
% \begin{macro}[internal]{\ekvc@flag@name,\ekvc@flag@namescheme}
%   Flags follow a simple naming scheme which we define here. |\ekvc@flag@name|
%   will store the name of an internal function that is used to build names of
%   the second naming scheme defined by |\ekvc@flag@namescheme|.
%    \begin{macrocode}
\def\ekvc@flag@name{ekvcf\string}
\def\ekvc@flag@namescheme#1#2{ekvch#2#1}
%    \end{macrocode}
% \end{macro}
%
% \begin{macro}{\ekvcFlagHeight}
%   For semantic reasons we use |\number| with another name.
%    \begin{macrocode}
\let\ekvcFlagHeight\number
%    \end{macrocode}
% \end{macro}
%
% \begin{macro}{\ekvcFlagNew}
%   This macro defines a new flag. It stores the function build with the
%   |\ekvc@flag@name| naming scheme after the internal function
%   |\ekvc@flag@height| that'll determine the current flag height. It'll also
%   define the macro named via |\ekvc@flag@name| to build names according to
%   |\ekvc@flag@namescheme|.
%    \begin{macrocode}
\protected\def\ekvcFlagNew#1%
  {%
    \edef#1%
      {%
        \unexpanded{\ekvc@flag@height}%
          \unexpanded\expandafter{\csname\ekvc@flag@name#1\endcsname}%
      }%
    \ekv@expargtwice
      {\expandafter\def\csname\ekvc@flag@name#1\endcsname##1}%
      {\expandafter\ekvc@flag@namescheme\expandafter{\string#1}{##1}}%
  }
%    \end{macrocode}
% \end{macro}
%
% \begin{macro}[internal]{\ekvc@flag@height,\ekvc@flag@height@}
%   This macro gets the height of a flag by a simple loop. The first loop
%   iteration differs a bit from the following in that it doesn't have to get
%   the current iteration count. The space at the end of |\ekvc@flag@height|
%   ends the |\number| evaluation.
%    \begin{macrocode}
\def\ekvc@flag@height#1%
  {%
    \ifcsname#10\endcsname
      \ekvc@flag@height@1\ekv@stop#1%
    \fi
    \@firstofone{0} % leave this space
  }
\def\ekvc@flag@height@#1\ekv@stop#2\fi\@firstofone#3%
  {%
    \fi
    \ifcsname#2{#1}\endcsname
      \expandafter\ekvc@flag@height@\the\numexpr#1+1\relax\ekv@stop#2%
    \fi
    \@firstofone{#1}%
  }
%    \end{macrocode}
% \end{macro}
%
% \begin{macro}{\ekvcFlagRaise}
%   Raising a flag simply means letting the |\ekvc@flag@namescheme| macro for
%   the current height to relax. The result of raising a flag is that its height
%   is bigger by $1$.
%    \begin{macrocode}
\ekv@exparg{\def\ekvcFlagRaise#1}%
  {%
    \expandafter\expandafter\expandafter\@gobble\expandafter
      \csname\ekvc@flag@namescheme{\string#1}{\ekvcFlagHeight#1}\endcsname
  }
%    \end{macrocode}
% \end{macro}
%
% \begin{macro}{\ekvcFlagSetTrue,\ekvcFlagSetFalse}
% \begin{macro}[internal]{\ekvc@flag@set@true,\ekvc@flag@set@false}
%   A flag is considered true if its current height is odd, and as false if it
%   is even. Therefore |\ekvcFlagSetTrue| and |\ekvcFlagSetFalse| only need to
%   raise the flag if the opposing boolean value is the current one.
%    \begin{macrocode}
\def\ekvcFlagSetTrue#1%
  {\expandafter\ekvc@flag@set@true\ekvcFlagHeight#1\ekv@stop#1}
\def\ekvcFlagSetFalse#1%
  {\expandafter\ekvc@flag@set@false\ekvcFlagHeight#1\ekv@stop#1}
%    \end{macrocode}
%   We can expand |\ekvc@flag@namescheme| at definition time here, which is why
%   we're using a temporary definition to set up |\ekvc@flag@set@true| and
%   |\ekvc@flag@set@false|.
%    \begin{macrocode}
\def\ekvc@flag@set@true#1%
  {%
    \def\ekvc@flag@set@true##1\ekv@stop##2%
      {%
        \ifodd##1
          \ekv@fi@gobble
        \fi
        \@firstofone{\expandafter\@gobble\csname#1\endcsname}%
      }%
    \def\ekvc@flag@set@false##1\ekv@stop##2%
      {%
        \ifodd##1
          \ekv@fi@firstofone
        \fi
        \@gobble{\expandafter\@gobble\csname#1\endcsname}%
      }%
  }
\expandafter\ekvc@flag@set@true\expandafter
  {\ekvc@flag@namescheme{\string#2}{#1}}
%    \end{macrocode}
% \end{macro}
% \end{macro}
%
% \begin{macro}{\ekvcFlagIf}
%   As already explained, truthiness is defined as a flag's height is odd, so we
%   just branch accordingly here.
%    \begin{macrocode}
\def\ekvcFlagIf#1%
  {%
    \ifodd#1%
      \ekv@fi@firstoftwo
    \fi
    \@secondoftwo
  }
%    \end{macrocode}
% \end{macro}
%
% \begin{macro}{\ekvcFlagIfRaised}
%   This macro uses flags as a switch, if a flag's current height is bigger than
%   $0$ this test yields true.
%    \begin{macrocode}
\ekv@exparg{\def\ekvcFlagIfRaised#1}%
  {%
    \expandafter\ifcsname\ekvc@flag@namescheme{\string#1}0\endcsname
      \ekv@fi@firstoftwo
    \fi
    \@secondoftwo
  }
%    \end{macrocode}
% \end{macro}
%
% \begin{macro}{\ekvcFlagReset}
% \begin{macro}[internal]{\ekvc@flag@reset,\ekvc@flag@reset@}
%   Resetting works by locally letting all the defined internal macros named
%   after |\ekvc@flag@namescheme| to undefined.
%    \begin{macrocode}
\protected\def\ekvcFlagReset#1%
  {\expandafter\ekvc@flag@reset\csname\ekvc@flag@name#1\endcsname}
\protected\def\ekvc@flag@reset#1%
  {%
    \ifcsname#10\endcsname
      \expandafter\let\csname#10\endcsname\ekvc@undefined
      \ekvc@flag@reset@1\ekv@stop#1%
    \fi
  }
\protected\def\ekvc@flag@reset@#1\ekv@stop#2\fi
  {%
    \fi
    \ifcsname#2{#1}\endcsname
      \expandafter\let\csname#2{#1}\endcsname\ekvc@undefined
      \expandafter\ekvc@flag@reset@\the\numexpr#1+1\relax\ekv@stop#2%
    \fi
  }
%    \end{macrocode}
% \end{macro}
% \end{macro}
%
% \begin{macro}{\ekvcFlagGetHeight}
% \begin{macro}[internal]{\ekvc@flag@get@height@single}
%   These are just small helpers, first getting the height of the flag and then
%   passing it on to the user supplied code.
%    \begin{macrocode}
\def\ekvcFlagGetHeight#1%
  {\expandafter\ekvc@flag@get@height@single\ekvcFlagHeight#1\ekv@stop}
\long\def\ekvc@flag@get@height@single#1\ekv@stop#2{#2{#1}}
%    \end{macrocode}
% \end{macro}
% \end{macro}
%
% \begin{macro}{\ekvcFlagGetHeights}
% \begin{macro}[internal]
%   {\ekvc@flag@get@heights,\ekvc@flag@get@heights@,\ekvc@flag@get@heights@done}
%   This works by a simple loop that stops at |\ekv@stop|. As long as that
%   marker isn't hit, get the next flags height and put it into a list after
%   |\ekv@stop|. |\ekvc@flag@get@heights@| uses the same marker name for the
%   end of the height, which shouldn't clash in any case. Once we're done we
%   remove the remainder of the current iteration and leave the user supplied
%   code in the input stream with all the flags' heights as a single argument.
%    \begin{macrocode}
\def\ekvcFlagGetHeights#1%
  {%
    \ekvc@flag@get@heights#1\ekv@stop{}%
  }
\def\ekvc@flag@get@heights#1%
  {%
    \ekv@gobbleto@stop#1\ekvc@flag@get@heights@done\ekv@stop
    \expandafter\ekvc@flag@get@heights@\ekvcFlagHeight#1\ekv@stop
  }
\def\ekvc@flag@get@heights@#1\ekv@stop#2\ekv@stop#3%
  {\ekvc@flag@get@heights#2\ekv@stop{#3{#1}}}
\long\def\ekvc@flag@get@heights@done
    \ekv@stop
    \expandafter\ekvc@flag@get@heights@\ekvcFlagHeight\ekv@stop\ekv@stop#1#2%
  {#2{#1}}
%    \end{macrocode}
% \end{macro}
% \end{macro}
%
%
% \subsubsection{Helper Macros}
%
% \begin{macro}[internal]{\ekvc@ifspace,\ekvc@ifspace@}
% A test which can be reduced to an if-empty by gobbling everything up to the
% first space.
%    \begin{macrocode}
\long\def\ekvc@ifspace#1%
  {%
    \ekvc@ifspace@#1 \ekv@ifempty@B
      \ekv@ifempty@false\ekv@ifempty@A\ekv@ifempty@B\@firstoftwo
  }
\long\def\ekvc@ifspace@#1 % keep this space
  {%
    \ekv@ifempty@\ekv@ifempty@A
  }
%    \end{macrocode}
% \end{macro}
%
%
% \subsubsection{Assertions}
% \begin{macro}[internal]{\ekvc@assert@not@long}
%   Some keys don't want to be |long| and we have to educate the user, so let's
%   throw an error if someone wanted these to be long.
%    \begin{macrocode}
\long\def\ekvc@assert@not@long#1{\ifx\ekvc@long\long\ekvc@err@no@long{#1}\fi}
%    \end{macrocode}
% \end{macro}
%
%
% \subsubsection{Messages}
%
% \begin{macro}[internal]
%   {
%     \ekvc@err@toomany,\ekvc@err@value@required,\ekvc@err@missing@type,
%     \ekvc@err@already@defined,\ekvc@err@no@key@macro
%   }
%   Boring unexpandable error messages.
%    \begin{macrocode}
\protected\def\ekvc@err@toomany#1%
  {%
    \errmessage{expkv-cs Error: Too many keys for macro `\string#1'}%
  }
\protected\def\ekvc@err@value@required#1%
  {%
    \errmessage{expkv-cs Error: Missing value for key `\unexpanded{#1}'}%
  }
\protected\def\ekvc@err@missing@type#1%
  {%
    \errmessage
      {expkv-cs Error: Missing type for secondary key `\unexpanded{#1}'}%
  }
\protected\def\ekvc@err@no@long#1%
  {%
    \errmessage
      {expkv-cs Error: prefix `long' not accepted for `\unexpanded{#1}'}%
  }
\protected\def\ekvc@err@already@defined#1%
  {%
    \errmessage{expkv-cs Error: Macro `\string#1' already defined}%
  }
\protected\def\ekvc@err@unknown@keytype#1%
  {%
    \errmessage{expkv-cs Error: Unknown key type `\unexpanded{#1}'}%
  }
\protected\def\ekvc@err@unknown@key#1%
  {%
    \errmessage
      {expkv-cs Error: Unknown key `\unexpanded{#1}' for macro `\ekvc@set'}%
  }
\long\def\ekvc@err@no@key@macro#1%
  {\errmessage{expkv-cs Error: \string#1 is no key=val macro}}
%    \end{macrocode}
% \end{macro}
%
% \begin{macro}[internal]{\ekvc@err}
%   We need a way to throw error messages expandably in some contexts.
%    \begin{macrocode}
\ekv@exparg{\long\def\ekvc@err#1}{\ekverr{expkv-cs}{#1}}
%    \end{macrocode}
% \end{macro}
%
% \begin{macro}[internal]
%   {
%     \ekvc@err@unknown@hash,\ekvc@err@empty@hash,\ekvc@err@missing@hash,
%     \ekvc@err@invalid@bool
%   }
%   And here are the expandable error messages.
%    \begin{macrocode}
\long\def\ekvc@err@unknown@hash#1{\ekvc@err{unknown hash `#1'}}
\long\def\ekvc@err@missing@hash#1{\ekvc@err{hash `#1' not found}}
\long\def\ekvc@err@empty@hash{\ekvc@err{empty hash}}
\def\ekvc@err@invalid@bool#1{\ekvc@err{invalid boolean value `#1'}}
\long\def\ekvc@err@unknown@key@or@macro#1#2%
  {\ekvc@err{unknown key `#2' for macro #1}}
%    \end{macrocode}
% \end{macro}
%
%
% Now everything that's left is to reset the category code of |@|.
%    \begin{macrocode}
\catcode`\@=\ekvc@tmp
%    \end{macrocode}
%
% \gobbledocstriptag
%</tex>
%^^A=<<
%
%^^A=<<
%
% \end{implementation}^^A=<<
%
% \clearpage
% \PrintIndex
%
\endinput
%
^^A vim: ft=tex fdm=marker fmr=>>=,=<<
}{}
\ProvidesFile{expkv-cs.dtx}
  [\csname ekvcDate\endcsname\ define expandable key=val macros using expkv]
\PassOptionsToPackage{full}{textcomp}
\documentclass{l3doc}
\RequirePackage[oldstylenums,nott]{kpfonts}
\input{glyphtounicode}
\pdfgentounicode=1
\RequirePackage{tcolorbox}
\newtcolorbox{exresult}[2][]
  {colback=expkvgrey!10!white,colframe=expkvgrey,fontupper=\small,width={#2},#1}
\newtcbox\exres[1][]
  {
     colback=expkvgrey!10!white
    ,colframe=expkvgrey
    ,size=small
    ,nobeforeafter
    ,tcbox raise base
    ,fontupper=\small
    ,#1
  }
\RequirePackage{listings}
\RequirePackage{booktabs}
\RequirePackage{array}
\RequirePackage{collcell}
\RequirePackage{siunitx}
\RequirePackage{xcolor}
\RequirePackage{caption}
\RequirePackage{microtype}
\RequirePackage{accsupp}
\lstdefinelanguage{expkvc}[primitive]{TeX}
  {
    ,moretexcs=[4]^^A e-TeX
      {
        expanded,
        numexpr,
        protected,
      }
    ,moretexcs=[5]^^A plain/LaTeX
      {
        approx,
        begin,
        empty,
        item,
        LaTeX,
        makeatletter,makeatother,multicolumn,
        newcommand,newdimen,
        RequirePackage,
        rule,
        TeX,
        textit,texttt,
        usepackage,
      }
    ,moretexcs=[6]^^A used packages
      {
        ^^A xfp
        fpeval,
        ^^A keyval
        setkeys,
        ^^A yax
        defactiveparameter,storevalue,setparameterlist
      }
    ,moretexcs=[2]^^A expkv-cs macros
      {
        ekvcSplit,ekvcSplitAndForward,ekvcSplitAndUse,
        ekvcHash,ekvcHashAndForward,ekvcHashAndUse,
        ekvcValue,ekvcValueFast,ekvcValueSplit,ekvcValueSplitFast,
        ekvcSecondaryKeys,
        ekvcChange,
        ekvcFlagNew,
        ekvcFlagHeight,
        ekvcFlagRaise,
        ekvcFlagSetTrue,ekvcFlagSetFalse,
        ekvcFlagIf,ekvcFlagIfRaised,
        ekvcFlagReset,
        ekvcFlagGetHeight,ekvcFlagGetHeights,
        ekvcPass
      }
    ,moretexcs=[3]^^A expkv-pkg macros
      {
        ^^A expkv
        ekvdef,ekvdefNoVal,^^A
        ekvlet,ekvletNoVal,ekvletkv,ekvletkvNoVal,^^A
        ekvdefunknown,ekvdefunknownNoVal,^^A
        ekvredirectunknown,ekvredirectunknownNoVal,^^A
        ekvset,ekvsetSneaked,^^A
        ekvsetdef,ekvsetSneakeddef,ekvsetdefSneaked,^^A
        ekvparse,^^A
        ekvVersion,ekvDate,^^A
        ekvifdefined,ekvifdefinedNoVal,ekvifdefinedset,^^A
        ekvbreak,ekvbreakPreSneak,ekvbreakPostSneak,^^A
        ekvsneak,ekvsneakPre,^^A
        ekvchangeset,^^A
        ekvoptarg,ekvoptargTF,^^A
        ekverr,
        ^^A expkv-def
        ekvdefinekeys
      }
  }
\colorlet{codeparam}{cyan!65!black}
\def\mylstwd{.55em}
\lstset
  {
    ,language=expkvc
    ,flexiblecolumns=false
    ,basewidth=\mylstwd
    ,gobble=2
    ,basicstyle=\fontfamily{jkp}\itshape
    ,texcsstyle=*[1]{\bfseries\color{expkvgrey}} ^^A primitives
    ,texcsstyle=*[2]{\bfseries\color{expkvred}}  ^^A expkv
    ,texcsstyle=*[3]{\color{expkvred}}           ^^A expkv-pkg
    ,texcsstyle=*[4]{\bfseries\color{expkvgrey}} ^^A e-TeX
    ,texcsstyle=*[5]{\bfseries\color{expkvgrey}} ^^A plain/LaTeX macros
    ,texcsstyle=*[6]{}                           ^^A macros of other packages
    ,commentstyle=\color[gray]{0.4}
    ,literate=
              {\{} {{\CodeUpBf\{}}{1}
              {\}} {{\CodeUpBf\}}}{1}
              {$}  {{\CodeUpBf\$}}{1}
              {[}  {{\CodeUp[}}{1}
              {]}  {{\CodeUp]}}{1}
              {(}  {{\CodeUp(}}{1}
              {)}  {{\CodeUp)}}{1}
              {*}  {{$*$}}{1}
              {1}  {{\CodeUp{1}}}{1}
              {2}  {{\CodeUp{2}}}{1}
              {3}  {{\CodeUp{3}}}{1}
              {4}  {{\CodeUp{4}}}{1}
              {5}  {{\CodeUp{5}}}{1}
              {6}  {{\CodeUp{6}}}{1}
              {7}  {{\CodeUp{7}}}{1}
              {8}  {{\CodeUp{8}}}{1}
              {9}  {{\CodeUp{9}}}{1}
              {0}  {{\CodeUp{0}}}{1}
              {##} {{\CodeColored{codeparam}{1}{\#}}}{1}
              {##1}{{\CodeColored{codeparam}{2}{\#1}}}{2}
              {##2}{{\CodeColored{codeparam}{2}{\#2}}}{2}
              {##3}{{\CodeColored{codeparam}{2}{\#3}}}{2}
              {##4}{{\CodeColored{codeparam}{2}{\#4}}}{2}
              {##5}{{\CodeColored{codeparam}{2}{\#5}}}{2}
              {##6}{{\CodeColored{codeparam}{2}{\#6}}}{2}
              {##7}{{\CodeColored{codeparam}{2}{\#7}}}{2}
              {##8}{{\CodeColored{codeparam}{2}{\#8}}}{2}
              {##9}{{\CodeColored{codeparam}{2}{\#9}}}{2}
              {<key>}{{$\langle$}key{$\rangle$}}{5}
              {<set>}{{$\langle$}set{$\rangle$}}{5}
              {<set1>}{{$\langle$}set1{$\rangle$}}{6}
              {<set2>}{{$\langle$}set2{$\rangle$}}{6}
    ^^A,literate=*{<key>}{\key}{4}{<set>}{\set}{4}
  }
\newcommand*\CodeColored[3]{\textcolor{#1}{\makebox[\dimexpr\mylstwd*#2]{#3}}}
\newcommand*\CodeUpBf[1]{\makebox[\mylstwd]{\textup{\textbf{#1}}}}
\newcommand*\CodeUp[1]{\makebox[\mylstwd]{\textup{#1}}}
\RequirePackage{randtext}
\let\metaORIG\meta
\protected\def\meta #1{\texttt{\metaORIG{#1}}}
\renewcommand*\thefootnote{\fnsymbol{footnote}}
\definecolor{expkvred}{HTML}{9F393D}
\colorlet{expkvgrey}{black!75}
\makeatletter
\newcommand*\example{\par\smallskip\noindent\textit{Example:} \ignorespaces}
\newcommand*\expkv
  {^^A
    \texorpdfstring
      {^^A
        \mbox
          {^^A
            \BeginAccSupp{ActualText=expkv}^^A
            \href{https://github.com/Skillmon/tex_expkv}
              {^^A
                \rmfamily
                \bfseries
                {\color{expkvgrey}e\kern-.05em x\kern-.05em}^^A
                \lower.493ex
                  \hbox{{\color{expkvgrey}P}\kern-.1em{\color{expkvred}k}}^^A
                \kern-.18em{\color{expkvred}v}^^A
              }^^A
            \EndAccSupp{}^^A
          }^^A
      }
      {expkv}^^A
  }
\newcommand*\expkvpkg[1]
  {^^A
    \texorpdfstring
      {^^A
        \mbox
          {^^A
            \BeginAccSupp{ActualText=expkv-#1}^^A
            \href{https://github.com/Skillmon/tex_expkv-#1}
              {^^A
                \rmfamily
                \bfseries
                {\color{expkvgrey}e\kern-.05em x\kern-.05em}^^A
                \lower.493ex
                  \hbox{{\color{expkvgrey}P}\kern-.1em{\color{expkvred}k}}^^A
                \kern-.18em{\color{expkvred}v}^^A
                {\color{expkvgrey}^^A
                  \kern.05em\rule[-.1ex]{.08em}{1.2ex}\kern.05em\textsc{#1}^^A
                }^^A
              }^^A
            \EndAccSupp{}^^A
          }^^A
      }
      {expkv-#1}^^A
  }
\newcommand*\expkvd{\expkvpkg{def}}
\newcommand*\expkvc{\expkvpkg{cs}}
\newcommand\kv{\meta{key}=\meta{value}}
\newcommand\key{\meta{key}}
\newcommand\val{\meta{value}}
\newcommand\set{\meta{set}}
\newcommand\enflong{\textcolor{black}{long}}
\newcommand\alllong{\textcolor{gray}{long}}
\newcommand\notlong{\textcolor{red!80!black}{long}}
\newcommand\prefixes[1]
  {^^A
    \hfill
    \ifcase\numexpr#1\relax\or
      \enflong\or
      \alllong\or
      \notlong\fi
  }
\hypersetup{linkcolor=red!80!black,urlcolor=purple!80!black}
\DoNotIndex{\def,\edef,\,,\=,\begingroup,\catcode,\chardef,\csname,\endcsname}
\DoNotIndex{\endgroup,\endinput,\errmessage,\expandafter,\input,\let,\long,\z@}
\DoNotIndex{\protected,\ProvidesFile,\ProvidesPackage,\relax,\space,\advance}
\DoNotIndex{\@,\unexpanded,\string,\expanded,\dimexpr,\global,\glueexpr,\hbox}
\DoNotIndex{\numexpr,\RequirePackage,\setbox,\the,\unless,\xdef,\gdef,\newcount}
\DoNotIndex{\number}
\DoNotIndex{\^,\@firstofone,\@firstoftwo,\@gobble,\@secondoftwo}
\DoNotIndex{\ifcsname}
\DoNotIndex{\ifx}
\DoNotIndex{\ifodd}
\DoNotIndex{\ifnum}
\DoNotIndex{\ifdefined}
\DoNotIndex{\iffalse}
\DoNotIndex{\iftrue}
\DoNotIndex{\else}
\DoNotIndex{\fi}
\@gobble\fi ^^A ignoring \ifx, \ifcsname, etc., but only one \fi
\@gobble\fi ^^A ignoring \ifx, \ifcsname, etc., but only one \fi
\@gobble\fi ^^A ignoring \ifx, \ifcsname, etc., but only one \fi
\@gobble\fi ^^A ignoring \ifx, \ifcsname, etc., but only one \fi
\@gobble\fi ^^A ignoring \ifx, \ifcsname, etc., but only one \fi
\@gobble\fi ^^A ignoring \ifx, \ifcsname, etc., but only one \fi
\@ifdefinable\gobbledocstriptag{\def\gobbledocstriptag#1>{}}
\makeatother
\begin{document}
  \title
    {^^A
      \texorpdfstring
        {^^A
          \huge\expkvc
          \\[\medskipamount]
          \Large define expandable \kv\ macros using \expkv
        }{expkv-cs - define expandable key=val macros using expkv}^^A
    }
  \date{\ekvcDate\space v\ekvcVersion}
  \author{Jonathan P. Spratte\thanks{\protect\randomize{jspratte@yahoo.de}}}
  \DocInput{expkv-cs.dtx}
\end{document}
%</driver>^^A=<<
% \fi
%
% \maketitle
% \renewcommand*\thefootnote{\arabic{footnote}}
%
% \begin{abstract}
% \noindent\parfillskip=0pt
% \expkvc\ provides two small interfaces to define expandable \kv\ macros using
% \expkv. It therefore lowers the entrance boundary to expandable \kv\ macros.
% The stylised name is \expkvc\ but the files use \file{expkv-cs}, this is due
% to CTAN-rules which don't allow \string| in package names since that is the
% pipe symbol in *nix shells.
% \end{abstract}
%
% \tableofcontents
%
% \begin{documentation}^^A>>=
%
% \section{Documentation}
%
% The \expkv\ package enables the new possibility of creating \kv\ macros which
% are fully expandable. The creation of such macros is however cumbersome for
% the average user. \expkvc\ tries to step in here. It provides interfaces to
% define \kv\ macros without worrying too much about the implementation. In case
% you're wondering now, the |cs| in \expkvc\ stands for control sequence,
% because |def| was already taken by \expkvd\ and ``control sequence'' is the
% term D.\,E.\,Knuth used in his \TeX book for named commands hence macros
% (though he also used the term ``macro''). So \expkvc\ defines control
% sequences for and with \expkv.
%
% There are two different approaches supported by this package. The first is
% splitting the keys up into individual arguments, the second is providing all
% the keys as a single argument to the underlying macro and getting an
% individual \val\ by using a hash. Well, actually there is no real hash, just
% some markers which are parsed, but this shouldn't be apparent to the user, the
% behaviour matches that of a hash-table.
%
% In addition to these two methods of defining a macro with primary keys a way
% to define secondary keys, which can reference the primary ones, is provided.
% These secondary keys don't correspond to an argument or an entry in the hash
% table directly but might come in handy for the average use case. Each macro
% has its own set of primary and secondary keys.
%
% A word of advice you should consider: If your macro doesn't have to be
% expandable (and often it doesn't) consider not using \expkvc. The interface
% has some overhead (though it still is fast -- check \autoref{sec:speed}) and
% the approach has its limits in versatility. If you don't need to be
% expandable, you should consider either defining your keys manually using
% \expkv\ or using \expkvd\ for convenience. Or you resort to another \kv\
% interface. Nevertheless setting up macros with \expkvc, especially with
% |\ekvcSplit|, is very convenient in my opinion, so if you just want to define
% a single macro with just a few keys this might be the way to go.
%
% \expkvc\ is usable as generic code and as a \LaTeX\ package. It'll
% automatically load \expkv\ in the same mode as well. To use it, just use one
% of
% \begin{lstlisting}
% \usepackage{expkv-cs} % LaTeX
% \input expkv-cs       % plainTeX
% \end{lstlisting}
%
%
% \subsection{Define Macros and Primary Keys}
%
% All macros defined with \expkvc\ have to be previously undefined or have the
% |\meaning| of |\relax|. This is necessary as there is no way to automatically
% undefine keys once they are set up (neither \expkv\ nor \expkvc\ keep track of
% defined keys) -- so to make sure there are no conflicts only new definitions
% are allowed (that's not the case for individual keys, only for frontend
% macros).
%
%
% \subsubsection{Primary Keys}\label{sec:primaries}
%
% In the following descriptions there will be one argument named \meta{primary
% keys}. This argument should be a \kv\ list where each \key\ will be one
% primary key and \val\ the associated initial value. By default all keys are
% defined short, but you can define long keys by prefixing \key\ with |long|
% (\emph{e.g.}, \texttt{long name=Jonathan P. Spratte}). You only need |long| if
% the key should be able to take a |\par| token. Note however that |long| keys
% are a microscopic grain faster (due to some internals of \expkvc). Only if at
% least one of the keys was |long| the \meta{cs} in the following defining
% macros will be |\long|. For obvious reasons there is no possibility to define
% a macro or key as |\protected|.
%
% At the moment \expkvc\ doesn't require any internal keys, but I can't foresee
% whether this will be the case in the future as well, as it might turn out that
% some features I deem useful can't be implemented without such internal keys.
% Because of this, please don't use key names starting with
% \texttt{EKVC\string|} as that should be the private name space.
%
%
% \subsubsection{Split}
%
% The split variants will provide the key values as separate arguments. This
% limits the number of keys for which this is truly useful.
%
% \begin{function}{\ekvcSplit}
%   \begin{syntax}
%     \cs{ekvcSplit}\meta{cs}\marg{primary keys}\marg{definition}
%   \end{syntax}
%   This defines \meta{cs} to be a macro taking one mandatory argument which
%   should contain a \kv\ list. The \meta{primary keys} will be defined for this
%   macro (see \autoref{sec:primaries}). The \meta{definition} is the code that
%   will be executed. You can access the \val\ of a \key\ by using a macro
%   parameter from |#1| to |#9|. The order of the macro parameters will be the
%   order provided in the \meta{primary keys} list (so |#1| is the \val\ of the
%   key defined first). With |\ekvcSplit| you can define macros using at most
%   nine primary keys.
% \end{function}
% \example The following defines a macro |\foo| that takes the keys |a| and |b|
% and outputs their values in a textual form:\\
% \begin{minipage}{.75\linewidth}
% \begin{lstlisting}
% \ekvcSplit\foo{a=a,b=b}{a is #1.\par b is #2.\par}
% \foo{}
% \foo{b=e}
% \end{lstlisting}
% \end{minipage}^^A
% \begin{exresult}[nobeforeafter,box align=center]{.25\linewidth}
%   \ekvcSplit\foo{a=a,b=b}{a is #1.\par b is #2.\par}
%   \foo{}
%   \foo{b=e}
% \end{exresult}
%
% \begin{function}{\ekvcSplitAndForward}
%   \begin{syntax}
%     \cs{ekvcSplitAndForward}\meta{cs}\marg{after}\marg{primary keys}
%   \end{syntax}
%   This defines \meta{cs} to be a macro taking one mandatory argument which
%   should contain a \kv\ list. You can use as many primary keys as you want
%   with this. The primary keys will be forwarded to \meta{after} as braced
%   arguments (as many as necessary for your primary keys). The order of the
%   braced arguments will be the order of your primary key definitions. In
%   \meta{after} you can use just a single control sequence, or some arbitrary
%   stuff which will be left in the input stream before your braced values (with
%   one set of braces stripped from \meta{after}), so both of the following
%   would be fine:
% \begin{lstlisting}
% \ekvcSplitAndForward\foo\foo@aux{keyA = A, keyB = B}
% \ekvcSplitAndForward\foo{\foo@aux{more args}}{keyA = A, keyB = B}
% \end{lstlisting}
% \end{function}
% In the first case |\foo@aux| should take at least two arguments (|keyA| and
% |keyB|), in the second case at least three (|more args|, |keyA|, and |keyB|).
%
% \begin{function}{\ekvcSplitAndUse}
%   \begin{syntax}
%     \cs{ekvcSplitAndUse}\meta{cs}\marg{primary keys}
%   \end{syntax}
%   This will roughly do the same as |\ekvcSplitAndForward|, but instead of
%   specifying what will be used after splitting the keys, \meta{cs} will use
%   what follows the \kv\ list. So its syntax will be
% \end{function}
%   \begin{syntax}
%     \meta{cs}\{\key=\val, \ldots\}\marg{after}
%   \end{syntax}
% and the code in |after| should expect at least as many arguments as the number
% of keys defined for \meta{cs}.
%
% \subsubsection{Hash}
%
% The hash variants will provide the key values as a single argument in which
% you can access specific values using a special macro. The implementation might
% be more convenient and scale better, \emph{but} it is much slower (for a
% primitive macro with a single key benchmarking was almost $1.7$ times slower,
% the root of which being the key access with |\ekvcValue|, not the parsing, and
% for a key access using |\ekvcValueFast| it was still about $1.2$ times
% slower). So if your macro uses less than ten primary keys, you should consider
% using the split approach.
%
% \begin{function}{\ekvcHash}
%   \begin{syntax}
%     \cs{ekvcHash}\meta{cs}\marg{primary keys}\marg{definition}
%   \end{syntax}
%   This defines \meta{cs} to be a macro taking one mandatory argument which
%   should contain a \kv\ list. You can use as many primary keys as you want.
%   The primary keys will be forwarded as a single argument containing every key
%   to the underlying macro. The underlying macro is defined as
%   \meta{definition}, in which you can access the \val\ of a \key\ by using
%   \texttt{\cs[no-index]{ekvcValue}\{\key\}\{\#1\}}.
% \end{function}
% \example This defines an equivalent macro like the |\foo| defined with
% |\ekvcSplit| earlier:\\
% \begin{minipage}{.75\linewidth}
% \begin{lstlisting}
% \ekvcHash\foo{a=a,b=b}{a is \ekvcValue{a}{#1}.\par
%                        b is \ekvcValue{b}{#1}.\par}
% \foo{}
% \foo{b=e}
% \end{lstlisting}
% \end{minipage}^^A
% \begin{exresult}[nobeforeafter,box align=center]{.25\linewidth}
%   \ekvcHash\foo{a=a,b=b}{a is \ekvcValue{a}{#1}.\par
%                          b is \ekvcValue{b}{#1}.\par}
%   \foo{}
%   \foo{b=e}
% \end{exresult}
%
% \begin{function}{\ekvcHashAndForward}
%   \begin{syntax}
%     \cs{ekvcHashAndForward}\meta{cs}\marg{after}\marg{primary keys}
%   \end{syntax}
%   This defines \meta{cs} to be a macro taking one mandatory argument which
%   should contain a \kv\ list. You can use as many primary keys as you want.
%   The primary keys will be forwarded as a single argument containing every key
%   to \meta{after}. You can use a single macro for \meta{after} or use some
%   arbitrary stuff, which will be left in the input stream before the hashed
%   \kv\ list with one set of braces stripped. In the
%   macro called in \meta{after} you can access the \val\ of a \key\ by using
%   \texttt{\cs[no-index]{ekvcValue}\{\key\}\{\#1\}} (or whichever argument the
%   hashed \kv\ list will be).
% \end{function}
%
% \begin{function}{\ekvcHashAndUse}
%   \begin{syntax}
%     \cs{ekvcHashAndUse}\meta{cs}\marg{primary keys}
%   \end{syntax}
%   This will roughly do the same as |\ekvcHashAndForward|, but instead of
%   specifying what will be used after hashing the keys, \meta{cs} will use what
%   follows the \kv\ list. So its syntax will be
% \end{function}
%   \begin{syntax}
%     \meta{cs}\{\key=\val, \ldots\}\marg{after}
%   \end{syntax}
%
% \begin{function}{\ekvcValue}
%   \begin{syntax}
%     \cs{ekvcValue}\{\key\}\marg{key list}
%   \end{syntax}
%   This is a safe way to access your keys in a hash variant. \key\
%   is the key which's \val\ you want to use out of the \meta{key list}.
%   \meta{key list} should be the key list argument forwarded to your underlying
%   macro by |\ekvcHash|, |\ekvcHashAndForward|, or |\ekvcHashAndUse|. It will
%   be tested whether the hash function to access that \key\ exists, the \key\
%   argument is not empty, and that the \meta{key list} really contains a \val\
%   of that \key. This macro needs exactly two steps of expansion and if used
%   inside of an |\edef| or |\expanded| context will protect the \val\ from
%   further expanding.
% \end{function}
%
% \begin{function}{\ekvcValueFast}
%   \begin{syntax}
%     \cs{ekvcValueFast}\{\key\}\marg{key list}
%   \end{syntax}
%   This behaves just like |\ekvcValue|, but \emph{without any} safety tests. As
%   a result this is about $1.4$ times faster \emph{but} will throw low level
%   \TeX\ errors eventually if the hash function isn't defined or the \key\
%   isn't part of the \meta{key list} (\emph{e.g.}, because it was defined as a
%   key for another macro -- all macros share the same hash function per \key).
%   Use it if you know what you're doing. This macro needs exactly three steps
%   of expansion in the no-errors case.
% \end{function}
%
% \begin{function}{\ekvcValueSplit}
%   \begin{syntax}
%     \cs{ekvcValueSplit}\{\key\}\marg{key list}\marg{next}
%   \end{syntax}
%   If you need a specific \key\ from a \meta{key list} more than once, it'll
%   be a good idea to only extract it once and from then on keep it as a
%   separate argument. Hence the macro \cs{ekvcValueSplit} will extract one
%   specific \key's value from the list and forward the remainder of the list
%   as the first and the \key's value as the second argument to \meta{next}, so
%   the result of this will be \meta{next}\marg{key list'}\marg{value} with
%   \meta{key list'} the remaining list. This is almost as fast as |\ekvcValue|
%   and runs the same tests. Keep in mind that you can't fetch for the same
%   \key\ again from \meta{key list'} as it got removed.
% \end{function}
% \example The following defines a macro |\foo| which will take three keys.
% Since the next parsing step will need the value of one of the keys multiple
% times we split that key off the list:\\
% \begin{minipage}{.75\linewidth}
% \begin{lstlisting}
% \ekvcHash\foo{a=a,b=b,c=c}
%   {\ekvcValueSplit{a}{#1}\foobar}
% \newcommand*\foobar[2]{a is #2.\par
%                        b is \ekvcValue{b}{#1}.\par
%                        c is \ekvcValue{c}{#1}.\par}
% \foo{}
% \end{lstlisting}
% \end{minipage}^^A
% \begin{exresult}[nobeforeafter,box align=center]{.25\linewidth}
%   \ekvcHash\foo{a=a,b=b,c=c}
%     {\ekvcValueSplit{a}{#1}\foobar}
%   \newcommand*\foobar[2]{a is #2.\par
%                          b is \ekvcValue{b}{#1}.\par
%                          c is \ekvcValue{c}{#1}.\par}
%   \foo{}
% \end{exresult}
%
% \begin{function}{\ekvcValueSplitFast}
%   \begin{syntax}
%     \cs{ekvcValueSplitFast}\{\key\}\marg{key list}\marg{next}
%   \end{syntax}
%   This behaves just like |\ekvcValueSplit|, but it won't run the same tests,
%   hence it is faster but more error prone, just like the relation between
%   |\ekvcValue| and |\ekvcValueFast|.
% \end{function}
%
%
% \subsection{Secondary Keys}\label{sec:secondaries}
%
% To remove some of the limitations with the approach that each primary key
% matches an argument or hash entry, you can define secondary keys. Those have
% to be defined for each macro but it doesn't matter whether that macro was a
% split or a hash variant. If a secondary key references another key it doesn't
% matter whether that other key is primary or secondary.
%
% Secondary keys can have a prefix (like |long|) which are called |p|-type
% prefix and must have a type (like |meta|) which are called |t|-type prefix.
% Some types might require some |p|-prefixes, while others might forbid those.
%
% Please keep in mind that key names shouldn't start with \texttt{EKVC\string|}.
%
% \begin{function}{\ekvcSecondaryKeys}
%   \begin{syntax}
%     \cs{ekvcSecondaryKeys}\meta{cs}\{\kv, \ldots\}
%   \end{syntax}
%   This is the front facing macro to define secondary keys. For the macro
%   \meta{cs} define \key\ to have definition \val. The general syntax for \key\
%   should be
%   \begin{quote}\ttfamily\small
%     \meta{prefix} \meta{name}
%   \end{quote}
%   Where \meta{prefix} is a space separated list of optional |p|-type prefixes
%   followed by one |t|-type prefix. The syntax of \val\ is dependent on the
%   used |t|-prefix.
% \end{function}
%
%
% \subsubsection{\texttt{p}-type Prefixes}
%
% There is only one |p|-prefix available, which is |long|.
%
% \begin{function}{long}
%   The following key will be defined |\long|.
% \end{function}
%
%
% \subsubsection{\texttt{t}-type Prefixes}
%
% If you're familiar with \expkvd\ you'll notice that the |t|-type prefixes
% provided here are much fewer. The expansion only concept doesn't allow for
% great variety in the auto-defined keys.
%
% The syntax examples of the |t|-prefixes will show which |p|-prefix will be
% automatically used by printing those black (\texttt{\enflong}), which
% will be available in grey (\texttt{\alllong}), and which will be disallowed in
% red (\texttt{\notlong}). This will be put flush right next to the syntax line.
%
% \begin{function}{meta}
%   \begin{syntax}
%     meta \key\ = \{\kv, \ldots\} \prefixes2
%   \end{syntax}
%   With a |meta| key you can set other keys. Whenever \key\ is used the keys in
%   the \kv\ list will be set to the values given there. You can use the \val\
%   given to \key\ by using |#1| in the \kv\ list. The keys in the \kv\ list can
%   be primary and secondary ones.
% \end{function}
%
% \begin{function}{nmeta}
%   \begin{syntax}
%     nmeta \key\ = \{\kv, \ldots\} \prefixes3
%   \end{syntax}
%   An |nmeta| key is like a |meta| key, but it doesn't take a value, so the
%   \kv\ list is static.
% \end{function}
%
% \begin{function}{alias}
%   \begin{syntax}
%     alias \key\ = \meta{key_2} \prefixes3
%   \end{syntax}
%   This assigns the definition of \meta{key_2} to \key. As a result \key\ is an
%   alias for \meta{key_2} behaving just the same. Both the value taking and the
%   |NoVal| version (that's \expkv\ slang for a key not accepting a value) will
%   be copied if they are defined when |alias| is used. Of course, \meta{key_2}
%   has to be defined, be it as a primary or secondary one.
% \end{function}
%
% \begin{function}{default}
%   \begin{syntax}
%     default \key\ = \marg{default} \prefixes3
%   \end{syntax}
%   If \key\ is a defined value taking key, you can define a |NoVal| version
%   with this that will behave as if \key\ was given \meta{default} as its
%   \val. Note that this doesn't change the initial values of primary keys set
%   at definition time in |\ekvcSplit| and friends. \key\ can be a primary or
%   secondary key.
% \end{function}
%
% \begin{function}{flag-bool}
%   \begin{syntax}
%     flag-bool \key\ = \meta{cs} \prefixes3
%   \end{syntax}
%   This is a secondary key that doesn't directly involve any of the primary or
%   secondary keys.
%   This defines \key\ to take a value, which should be either |true| or
%   |false|, and set the flag called \meta{cs} accordingly as a boolean.
%   If \meta{cs} isn't defined yet it will be initialised as a flag.
%   Please also read \autoref{sec:flags}.
% \end{function}
%
% \begin{function}{flag-true,flag-false}
%   \begin{syntax}
%     flag-true \key\ = \meta{cs} \prefixes3
%   \end{syntax}
%   This is a secondary key that doesn't directly involve any of the primary or
%   secondary keys.
%   This defines \key\ to take no value and set the flag called \meta{cs} to
%   |true| or |false|, respectively.
%   If \meta{cs} isn't defined yet it will be initialised as a flag.
%   Please also read \autoref{sec:flags}.
% \end{function}
%
% \begin{function}{flag-raise}
%   \begin{syntax}
%     flag-raise \key\ = \meta{cs} \prefixes3
%   \end{syntax}
%   This is a secondary key that doesn't directly involve any of the primary or
%   secondary keys.
%   This defines \key\ to take no value and raise the flag called \meta{cs}.
%   If \meta{cs} isn't defined yet it will be initialised as a flag.
%   Please also read \autoref{sec:flags}.
% \end{function}
%
%
% \subsection{Changing the Initial Values}\label{sec:change}
%
% \begin{function}{\ekvcChange}
%   \begin{syntax}
%     \cs{ekvcChange}\meta{cs}\{\kv,\ldots\}
%   \end{syntax}
%   This processes the \kv\ list (just like it would be processed by \meta{cs})
%   to set new defaults for the macro \meta{cs} defined with \expkvc\ (meaning
%   the values used if you don't provide anything at use time). Inside the \kv\
%   list both primary and secondary keys can be used. If \meta{cs} was defined
%   |\long| earlier it will still be |\long|, every other \TeX\ prefix will be
%   stripped (but \expkvc\ doesn't support them anywhere else so that should be
%   fine). The resulting new defaults will be stored inside the \meta{cs}
%   locally (just as the original defaults were).
% \end{function}
%
% Consider the following example:\\
% \begin{minipage}[c]{.75\linewidth}
% \begin{lstlisting}
% \ekvcSplit\foo{a=a,b=b}{a is #1.\par b is #2.\par}
% \foo{} % initial defaults
% \ekvcChange\foo{b=B}
% \foo{} % new defaults
% \ekvcSecondaryKeys\foo{meta c={a={#1},b={#1}}}
% \ekvcChange\foo{c=c}
% \foo{} % newer defaults
% \end{lstlisting}
% \end{minipage}^^A
% \begin{exresult}[nobeforeafter,box align=center]{.25\linewidth}
%   \ekvcSplit\foo{a=a,b=b}{a is #1.\par b is #2.\par}
%   \foo{}
%   \ekvcChange\foo{b=B}
%   \foo{}
%   \ekvcSecondaryKeys\foo{meta c={a={#1},b={#1}}}
%   \ekvcChange\foo{c=c}
%   \foo{}
% \end{exresult}
%
% As a result with this the typical setup macro could be implemented:
% \begin{lstlisting}
% \ekvcHashAndUse\foo{key=a,key=b}
% \newcommand\foosetup{\ekvcChange\foo}
% \end{lstlisting}
% Of course the usage is limited to a single macro |\foo|, hence this might be
% not as powerful as similar macros used with other \kv\ interfaces. But at
% least a few similar macros could be grouped using the same key parsing macro
% internally.
%
%
% \subsection{Flags}\label{sec:flags}
%
% The idea of flags is taken from \pkg{expl3}. They provide a way to store
% numerical information expandably, however only incrementing and accessing
% works expandably, decrementing is unexpandable. A flag has a height, which
% is a numerical value, and which can be raised by $1$. Flags come at a high
% computational cost (accessing them is slow and they require more memory than
% normal \TeX\ data types like registers, both getting linearly worse with the
% height), so don't use them if not necessary.
%
% The state of flags is always changed locally to the current group, but not to
% the current macro, so if you're using one of the |t|-types involving flags
% bear in mind that they can affect other macros using the same flags at the
% current group level!
%
% \expkvc\ provides some macros to access, alter, and use flags. Flags of
% \expkvc\ don't share a name space with the flags of \pkg{expl3}.
%
% \begin{function}{\ekvcFlagNew}
%   \begin{syntax}
%     \cs{ekvcFlagNew}\meta{flag}
%   \end{syntax}
%   This initialises the macro \meta{flag} as a new flag. It isn't checked
%   whether the macro \meta{flag} is currently undefined.  A \meta{flag} will
%   expand to the flag's current height with a trailing space (so you can use it
%   directly with |\ifnum| for example and it will terminate the number scanning
%   itself).
% \end{function}
%
%  All other macros dealing with flags take as a parameter a macro defined as a
%  \meta{flag} with |\ekvcFlagNew|.
%
% \begin{function}{\ekvcFlagHeight}
%   \begin{syntax}
%     \cs{ekvcFlagHeight}\meta{flag}
%   \end{syntax}
%   This expands to the current height of \meta{flag} in a single step of
%   expansion (without a trailing space).
% \end{function}
%
% \begin{function}{\ekvcFlagRaise}
%   \begin{syntax}
%     \cs{ekvcFlagRaise}\meta{flag}
%   \end{syntax}
%   This expandably raises the height of \meta{flag} by $1$.
% \end{function}
%
% \begin{function}{\ekvcFlagSetTrue,\ekvcFlagSetFalse}
%   \begin{syntax}
%     \cs{ekvcFlagSetTrue}\meta{flag}
%   \end{syntax}
%   By interpreting an even value as false and an odd value as true we can use a
%   flag as a boolean. This expandably sets \meta{flag} to true or false,
%   respectively, by raising it if necessary.
% \end{function}
%
% \begin{function}{\ekvcFlagIf}
%   \begin{syntax}
%     \cs{ekvcFlagIf}\meta{flag}\marg{true}\marg{false}
%   \end{syntax}
%   This interprets a \meta{flag} as a boolean and expands to either \meta{true}
%   or \meta{false}.
% \end{function}
%
% \begin{function}{\ekvcFlagIfRaised}
%   \begin{syntax}
%     \cs{ekvcFlagIfRaised}\meta{flag}\marg{true}\marg{false}
%   \end{syntax}
%   This tests whether the \meta{flag} is raised, meaning it has a height
%   greater than zero, and if so expands to \meta{true} else to \meta{false}.
% \end{function}
%
% \begin{function}{\ekvcFlagReset}
%   \begin{syntax}
%     \cs{ekvcFlagReset}\meta{flag}
%   \end{syntax}
%   This resets a flag (so restores its height to $0$). This operation is
%   \emph{not} expandable and done locally. If you really intend to use flags
%   you can reset them every now and then to keep the performance hit low.
% \end{function}
%
% \begin{function}{\ekvcFlagGetHeight}
%   \begin{syntax}
%     \cs{ekvcFlagGetHeight}\meta{flag}\marg{next}
%   \end{syntax}
%   This retrieves the current height of the \meta{flag} and provides it as a
%   braced argument to \meta{next}, leaving \meta{next}\marg{height} in the
%   input stream.
% \end{function}
%
% \begin{function}{\ekvcFlagGetHeights}
%   \begin{syntax}
%     \cs{ekvcFlagGetHeights}\marg{flag-list}\marg{next}
%   \end{syntax}
%   This retrieves the current height of each \meta{flag} in the
%   \meta{flag-list} and provides them as a single argument to \meta{next}.
%   Inside that argument each height is enclosed in a set of braces
%   individually. The \meta{flag-list} is just a single argument containing the
%   \meta{flag}s.
%   So a usage like
%   \texttt
%     {^^A
%       \cs[no-index]{ekvcFlagGetHeights}\hskip0pt^^A
%       \{\cs[no-index]{myflagA}\cs[no-index]{myflagB}\}\hskip0pt^^A
%       \{\cs[no-index]{stuff}\}^^A
%     }
%   will expand to
%   \texttt{\cs[no-index]{stuff}\{\marg{height-A}\marg{height-B}\}}.
% \end{function}
%
%
% \subsection{Further Examples}
%
% How could a documentation be a good documentation without some basic examples?
% Say we want to define a small macro expanding to some character description
% (who knows why this has to be expandable?). A character description will not
% have too many items to it, so we use |\ekvcSplit| (the comments with the
% parameter numbers are of course not necessary and just ease reading the
% example).
% \begin{lstlisting}
% \ekvcSplit\character
%   {
%     name=John Doe,            % #1
%     age=any,                  % #2
%     nationality=the Universe, % #3
%     hobby=to exist,           % #4
%     type=Mister,              % #5
%     pronoun=He,               % #6
%     possessive=his,           % #7
%   }
%   {#1 is a #5 from #3. #6 is of #2 age and #7 hobby is #4.\par}
% \end{lstlisting}
% Also we want to give some short cuts so that it's easier to describe several
% persons.
% \begin{lstlisting}
% \ekvcSecondaryKeys\character
%   {
%     alias pro = pronoun,
%     alias pos = possessive,
%     nmeta me =
%       {
%         name=Jonathan,
%         age=a young,
%         nationality=Germany,
%         hobby=\TeX\ coding,
%       },
%     meta lady =
%       {type=Lady, pronoun=She, possessive=her, name=Jane Doe, #1},
%     nmeta paulo =
%       {
%         name=Paulo,
%         type=duck,
%         age=a young,
%         nationality=Brazil,
%         hobby=to quack,
%       }
%   }
% \end{lstlisting}
% Now we can describe people using
% \begin{lstlisting}
% \character{}
% \character{me}
% \character{paulo}
% \character
%   {lady={name=Evelyn,nationality=Ireland,age=the best,hobby=reading}}
% \character
%   {
%     name=Our sun, type=star, nationality=our solar system, pro=It,
%     age=an old, pos=its, hobby=shining
%   }
% \end{lstlisting}
% As one might see, the |lady| key could actually have been an |nmeta| key as
% well, as all that is done with the argument is using it as a \kv\ list.
%
% The result of only the first two usages would be:
% \begin{exresult}{\linewidth}
%   \ekvcSplit\character
%     {
%       name=John Doe,
%       age=any,
%       nationality=the Universe,
%       hobby=to exist,
%       type=Mister,
%       pronoun=He,
%       possessive=his,
%     }
%     {#1 is a #5 from #3. #6 is of #2 age and #7 hobby is #4.\par}
%   \ekvcSecondaryKeys\character
%     {
%       alias pro = pronoun,
%       alias pos = possessive,
%       nmeta me =
%         {
%           name=Jonathan,
%           age=a young,
%           nationality=Germany,
%           hobby=\TeX\ coding,
%         },
%       meta lady =
%         {type=Lady, pronoun=She, possessive=her, name=Jane Doe, #1},
%       nmeta paulo =
%         {
%           name=Paulo,
%           type=duck,
%           age=a young,
%           nationality=Brazil,
%           hobby=to quack,
%         }
%     }
%   \character{}
%   \character{me}
% \end{exresult}
%
% Using \pkg{xparse} or \expkv's \cs{ekvoptarg} or \cs{ekvoptargTF} and
% forwarding arguments one can easily define \kv\ macros with actual optional
% and mandatory arguments as well. A small nonsense example (which should
% perhaps use |\ekvcSplitAndForward| instead of |\ekvcHashAndForward| since it
% only uses four keys and one other argument -- and isn't expandable since it
% uses a \env{tabular} environment, so it would've been better to use a more
% feature rich \kv\ interface most likely, \emph{e.g.}, the one provided by
% \expkvd):
% \begin{lstlisting}
% \makeatletter
% \newcommand*\nonsense{\ekvoptarg\nonsense@a{}}
% \ekvcHashAndForward\nonsense@a\nonsense@b
%   {
%     keyA = A,
%     keyB = B,
%     keyC = c,
%     keyD = d,
%   }
% \newcommand*\nonsense@b[2]
%   {%
%     \begin{tabular}{lll}
%       key & A & \ekvcValue{keyA}{#1} \\
%           & B & \ekvcValue{keyB}{#1} \\
%           & C & \ekvcValue{keyC}{#1} \\
%           & D & \ekvcValue{keyD}{#1} \\
%       \multicolumn{2}{l}{mandatory} & #2 \\
%     \end{tabular}%
%   }
% \makeatother
% \end{lstlisting}
% And then we would be able to do some nonsense
% \begin{lstlisting}
% \nonsense{}
% \nonsense[keyA=hihi]{haha}
% \nonsense[keyA=hihi, keyB=A]{hehe}
% \nonsense[keyC=huhu, keyA=hihi, keyB=A]{haha}
% \end{lstlisting}
% resulting in
% \begin{exresult}{\linewidth}
%   \makeatletter
%   \newcommand*\nonsense{\ekvoptarg\nonsense@a{}}
%   \ekvcHashAndForward\nonsense@a\nonsense@b
%     {
%       keyA = A,
%       keyB = B,
%       keyC = c,
%       keyD = d,
%     }
%   \newcommand*\nonsense@b[2]
%     {%
%       \begin{tabular}{lll}
%         key & A & \ekvcValue{keyA}{#1} \\
%             & B & \ekvcValue{keyB}{#1} \\
%             & C & \ekvcValue{keyC}{#1} \\
%             & D & \ekvcValue{keyD}{#1} \\
%         \multicolumn{2}{l}{mandatory} & #2 \\
%       \end{tabular}%
%     }
%   \makeatother
%   \nonsense{}
%   \nonsense[keyA=hihi]{haha}
%   \nonsense[keyA=hihi, keyB=A]{hehe}
%   \nonsense[keyC=huhu, keyA=hihi, keyB=A]{haha}
% \end{exresult}
%
%
% \subsection{Freedom for Keys!}
%
% If this was the \TeX book this subsection would have a double bend sign. Not
% because it is overly complicated, but because it shows things which could
% break \expkvc's expandability and its alignment safety. This is for
% experienced users wanting to get the most flexibility and knowing what they
% are doing.
%
% In case you're wondering, it is possible to define other keys than the
% primaries and the secondary types listed in \autoref{sec:secondaries} for a
% macro defined with \expkvc\ by using the low-level interface of \expkv\ or
% even the interface provided by \expkvd. The set name used for \expkvc's keys
% is the macro name, including the leading backslash, or more precisely
% |\string|\meta{cs} is used. This can be exploited to define additional keys
% with arbitrary code. Consider the following \emph{bad} example:
% \begin{lstlisting}
% \ekvcSplit\foo{a=A,b=B}{a is #1.\par b is #2}
% \protected\ekvdef{\string\foo}{c}{\def\fooC{#1}}
% \end{lstlisting}
% This would define a key named |c| that will store its value inside a macro.
% The issue with this is that this can't be done expandably. As a result, the
% macro |\foo| isn't always expandable any more (not that bad if this was never
% required; killjoy if it was) and as soon as the key |c| is used, it is also no
% longer alignment safe\footnote{This means that the \kv-list can't contain
%   alignment markers that are not inside an additional set of braces if used
%   inside a \TeX\ alignment.}
% (might be bad depending on the usage).
%
% So why do I show you this? Because we could as well do something useful like
% create a key that pre-parses the input and after that passes the parsed value
% on. This parsing would have to be completely expandable though. For the
% pass-on part we can use the following function:
%
% \begin{function}{\ekvcPass}
%   \begin{syntax}
%     \cs{ekvcPass}\meta{cs}\marg{key}\marg{value}
%   \end{syntax}
%   This passes \meta{value} on to \meta{key} for the \expkvc-macro \meta{cs}.
%   It should be used inside the key parsing of a macro defined with \expkvc,
%   else this most likely results in a low level \TeX\ error.
% \end{function}
%
% With this we could for example split the value of a key at a hyphen and pass
% the parts to different keys:\\
% \begin{minipage}[c]{.75\linewidth}
% \begin{lstlisting}
% \ekvcSplit\foo{a=A,b=B}{a is #1.\par b is #2.\par}
% \ekvdef{\string\foo}{c}{\fooSplit#1\par}
% \def\fooSplit#1-#2\par
%   {\ekvcPass\foo{a}{#1}\ekvcPass\foo{b}{#2}}
% \foo{}
% \foo{c=1-2}
% \end{lstlisting}
% \end{minipage}
% \begin{exresult}[nobeforeafter,box align=center]{.25\linewidth}
%   \ekvcSplit\foo{a=A,b=B}{a is #1.\par b is #2.\par}
%   \ekvdef{\string\foo}{c}{\fooSplit#1\par}
%   \def\fooSplit#1-#2\par{\ekvcPass\foo{a}{#1}\ekvcPass\foo{b}{#2}}
%   \foo{}
%   \foo{c=1-2}
% \end{exresult}
%
%
% \subsection{Speed Considerations}\label{sec:speed}
%
% As already mentioned in the introduction there are some speed considerations
% implied if you choose to define macros via \expkvc. However the overhead isn't
% the factor which should hinder you to use \expkvc\ if you found a reasonable
% use case. The key-parsing is still faster than with most other \kv\ packages
% (see the ``Comparisons'' subsection in the \expkv\ documentation).
%
% The speed considerations in this subsection use the first example in this
% documentation as the benchmark. So we have seven keys and a short sentence
% which should be typeset. For comparisons I use the following equivalent
% \expkvd\ definitions. Each result is the average between changing no keys from
% their initial values and altering four. Furthermore I'll compare three
% variants of \expkvc\ with the \expkvd\ definitions, namely the split example
% from above, a hash variant using |\ekvcValue| and a hash variant using
% |\ekvcValueFast|.
% \begin{lstlisting}
% \usepackage{expkv-def}
% \ekvdefinekeys{keys}
%   {%
%     ,store   name         = \KEYSname
%     ,initial name         = John Doe
%     ,store   age          = \KEYSage
%     ,initial age          = any
%     ,store   nationality  = \KEYSnationality
%     ,initial nationality  = the Universe
%     ,store   hobby        = \KEYShobby
%     ,initial hobby        = to exist
%     ,store   type         = \KEYStype
%     ,initial type         = Mister
%     ,store   pronoun      = \KEYSpronoun
%     ,initial pronoun      = He
%     ,store   possessive   = \KEYSpossessive
%     ,initial possessive   = his
%   }
% \newcommand*\KEYS[1]
%   {%
%     \begingroup
%       \ekvset{keys}{#1}%
%       \KEYSname\ is a \KEYStype\ from \KEYSnationality. \KEYSpronoun\ is
%       of \KEYSage\ age and \KEYSpossessive\ hobby is \KEYShobby.%
%     \endgroup
%   }
% \end{lstlisting}
%
% The first comparison removes the typesetting part from all the definitions, so
% that only the key parsing is compared. In this comparison the |\ekvcValue|
% and |\ekvcValueFast| variants will not differ, as they are exactly the same
% until the key usage. We find that the split approach is $1.4$ times slower
% than the \expkvd\ setup and the hash variants end up in the middle at $1.17$
% times slower.
%
% Next we put the typesetting part back in. Every call of the macros will
% typeset the sentences into a box register in horizontal mode. With the
% typesetting part (which includes the accessing of values) the fastest remains
% the \expkvd\ definitions, but split is close at $1.16$ times slower, followed
% by the hash variant with fast accesses at $1.36$ times slower, and the safe
% hash access variant ranks in the slowest $1.8$ times slower than \expkvd.
%
% Just in case you're wondering now, a simple macro taking seven arguments is
% $30$ to $40$ times faster than any of those in the argument grabbing and \kv\
% parsing part and only $1.5$ to $2.8$ times faster if the typesetting part is
% factored in. So the real choke isn't the parsing.
%
% So to summarize this, if you have a reasonable use case for expandable \kv\
% parsing macros you should go on and define them using \expkvc. If you have a
% reasonable use case for \kv\ parsing macros but defining them expandable isn't
% necessary for your use you should take advantage of the greater flexibility of
% non-expandable \kv\ setups (but if you're after maximum speed there aren't
% that many \kv\ parsers beating \expkvc). And if you are after maximum
% performance maybe ditching the \kv\ interface altogether is a good idea, but
% depending on the number of arguments your interface might get convoluted.
%
%
% \subsection{Useless Macros}
%
% Perhaps these macros aren't completely useless, but I figured from a user's
% point of view I wouldn't know what I should do with these.
%
% \begin{function}{\ekvcDate,\ekvcVersion}
%   These two macros store the version and the date of the package/generic code.
% \end{function}
%
%
% \subsection{Bugs}
%
% Of course I don't think there are any bugs (who would knowingly distribute
% buggy software as long as he isn't a multi-million dollar corporation?). But
% if you find some please let me know. For this one might find my email address
% on the first page or file an issue on Github:
% \url{https://github.com/Skillmon/tex_expkv-cs}
%
%
% \subsection{License}
%
% Copyright \textcopyright\ 2020\unless\ifnum\year=2020--\the\year\fi\
% Jonathan P. Spratte
%
% \medskip\noindent
% This work may be distributed and/or modified under the conditions of the
% \LaTeX\ Project Public License (LPPL), either version 1.3c of this license or
% (at your option) any later version. The latest version of this license is in
% the file:
%
% \url{http://www.latex-project.org/lppl.txt}
%
% \noindent
% This work is ``maintained'' (as per LPPL maintenance status) by
%
% \mbox{Jonathan P. Spratte}.
%
% \end{documentation}^^A=<<
%
% \begin{implementation}^^A>>=
%
% \clearpage
%
% \section{Implementation}^^A>>=
%^^A the LaTeX package >>=
% \subsection{The \LaTeX\ Package}
% Just like for \expkv\ we provide a small \LaTeX\ package that sets up things
% such that we behave nicely on \LaTeX\ packages and files system. It'll
% |\input| the generic code which implements the functionality.
% \gobbledocstriptag
%<*pkg>
%    \begin{macrocode}
\RequirePackage{expkv}
\def\ekvc@tmp
  {%
    \ProvidesFile{expkv-cs.tex}%
      [%
        \ekvcDate\space v\ekvcVersion\space
        define expandable key=val macros using expkv%
      ]%
  }
% \iffalse meta-comment
%
% File: expkv-cs.dtx Copyright (C) 2020-2021 Jonathan P. Spratte
%
% This work  may be  distributed and/or  modified under  the conditions  of the
% LaTeX Project Public License (LPPL),  either version 1.3c  of this license or
% (at your option) any later version.  The latest version of this license is in
% the file:
%
%   http://www.latex-project.org/lppl.txt
%
% ------------------------------------------------------------------------------
%
%<*driver>^^A>>=
\def\nameofplainTeX{plain}
\ifx\fmtname\nameofplainTeX\else
  \expandafter\begingroup
\fi
\input l3docstrip.tex
\askforoverwritefalse
\preamble

--------------------------------------------------------------
expkv-cs -- define expandable key=val macros using expkv
E-mail: jspratte@yahoo.de
Released under the LaTeX Project Public License v1.3c or later
See http://www.latex-project.org/lppl.txt
--------------------------------------------------------------

Copyright (C) 2020-2021 Jonathan P. Spratte

This  work may be  distributed and/or  modified under  the conditions  of the
LaTeX Project Public License (LPPL),  either version 1.3c  of this license or
(at your option) any later version.  The latest version of this license is in
the file:

  http://www.latex-project.org/lppl.txt

This work is "maintained" (as per LPPL maintenance status) by
  Jonathan P. Spratte.

This work consists of the file  expkv-cs.dtx
and the derived files           expkv-cs.pdf
                                expkv-cs.sty
                                expkv-cs.tex

\endpreamble
% stop docstrip adding \endinput
\postamble
\endpostamble
\generate{\file{expkv-cs.sty}{\from{expkv-cs.dtx}{pkg}}}
\generate{\file{expkv-cs.tex}{\from{expkv-cs.dtx}{tex}}}
\ifx\fmtname\nameofplainTeX
  \expandafter\endbatchfile
\else
  \expandafter\endgroup
\fi
%
\IfFileExists{expkv-cs.tex}{% \iffalse meta-comment
%
% File: expkv-cs.dtx Copyright (C) 2020-2021 Jonathan P. Spratte
%
% This work  may be  distributed and/or  modified under  the conditions  of the
% LaTeX Project Public License (LPPL),  either version 1.3c  of this license or
% (at your option) any later version.  The latest version of this license is in
% the file:
%
%   http://www.latex-project.org/lppl.txt
%
% ------------------------------------------------------------------------------
%
%<*driver>^^A>>=
\def\nameofplainTeX{plain}
\ifx\fmtname\nameofplainTeX\else
  \expandafter\begingroup
\fi
\input l3docstrip.tex
\askforoverwritefalse
\preamble

--------------------------------------------------------------
expkv-cs -- define expandable key=val macros using expkv
E-mail: jspratte@yahoo.de
Released under the LaTeX Project Public License v1.3c or later
See http://www.latex-project.org/lppl.txt
--------------------------------------------------------------

Copyright (C) 2020-2021 Jonathan P. Spratte

This  work may be  distributed and/or  modified under  the conditions  of the
LaTeX Project Public License (LPPL),  either version 1.3c  of this license or
(at your option) any later version.  The latest version of this license is in
the file:

  http://www.latex-project.org/lppl.txt

This work is "maintained" (as per LPPL maintenance status) by
  Jonathan P. Spratte.

This work consists of the file  expkv-cs.dtx
and the derived files           expkv-cs.pdf
                                expkv-cs.sty
                                expkv-cs.tex

\endpreamble
% stop docstrip adding \endinput
\postamble
\endpostamble
\generate{\file{expkv-cs.sty}{\from{expkv-cs.dtx}{pkg}}}
\generate{\file{expkv-cs.tex}{\from{expkv-cs.dtx}{tex}}}
\ifx\fmtname\nameofplainTeX
  \expandafter\endbatchfile
\else
  \expandafter\endgroup
\fi
%
\IfFileExists{expkv-cs.tex}{% \iffalse meta-comment
%
% File: expkv-cs.dtx Copyright (C) 2020-2021 Jonathan P. Spratte
%
% This work  may be  distributed and/or  modified under  the conditions  of the
% LaTeX Project Public License (LPPL),  either version 1.3c  of this license or
% (at your option) any later version.  The latest version of this license is in
% the file:
%
%   http://www.latex-project.org/lppl.txt
%
% ------------------------------------------------------------------------------
%
%<*driver>^^A>>=
\def\nameofplainTeX{plain}
\ifx\fmtname\nameofplainTeX\else
  \expandafter\begingroup
\fi
\input l3docstrip.tex
\askforoverwritefalse
\preamble

--------------------------------------------------------------
expkv-cs -- define expandable key=val macros using expkv
E-mail: jspratte@yahoo.de
Released under the LaTeX Project Public License v1.3c or later
See http://www.latex-project.org/lppl.txt
--------------------------------------------------------------

Copyright (C) 2020-2021 Jonathan P. Spratte

This  work may be  distributed and/or  modified under  the conditions  of the
LaTeX Project Public License (LPPL),  either version 1.3c  of this license or
(at your option) any later version.  The latest version of this license is in
the file:

  http://www.latex-project.org/lppl.txt

This work is "maintained" (as per LPPL maintenance status) by
  Jonathan P. Spratte.

This work consists of the file  expkv-cs.dtx
and the derived files           expkv-cs.pdf
                                expkv-cs.sty
                                expkv-cs.tex

\endpreamble
% stop docstrip adding \endinput
\postamble
\endpostamble
\generate{\file{expkv-cs.sty}{\from{expkv-cs.dtx}{pkg}}}
\generate{\file{expkv-cs.tex}{\from{expkv-cs.dtx}{tex}}}
\ifx\fmtname\nameofplainTeX
  \expandafter\endbatchfile
\else
  \expandafter\endgroup
\fi
%
\IfFileExists{expkv-cs.tex}{\input{expkv-cs.tex}}{}
\ProvidesFile{expkv-cs.dtx}
  [\csname ekvcDate\endcsname\ define expandable key=val macros using expkv]
\PassOptionsToPackage{full}{textcomp}
\documentclass{l3doc}
\RequirePackage[oldstylenums,nott]{kpfonts}
\input{glyphtounicode}
\pdfgentounicode=1
\RequirePackage{tcolorbox}
\newtcolorbox{exresult}[2][]
  {colback=expkvgrey!10!white,colframe=expkvgrey,fontupper=\small,width={#2},#1}
\newtcbox\exres[1][]
  {
     colback=expkvgrey!10!white
    ,colframe=expkvgrey
    ,size=small
    ,nobeforeafter
    ,tcbox raise base
    ,fontupper=\small
    ,#1
  }
\RequirePackage{listings}
\RequirePackage{booktabs}
\RequirePackage{array}
\RequirePackage{collcell}
\RequirePackage{siunitx}
\RequirePackage{xcolor}
\RequirePackage{caption}
\RequirePackage{microtype}
\RequirePackage{accsupp}
\lstdefinelanguage{expkvc}[primitive]{TeX}
  {
    ,moretexcs=[4]^^A e-TeX
      {
        expanded,
        numexpr,
        protected,
      }
    ,moretexcs=[5]^^A plain/LaTeX
      {
        approx,
        begin,
        empty,
        item,
        LaTeX,
        makeatletter,makeatother,multicolumn,
        newcommand,newdimen,
        RequirePackage,
        rule,
        TeX,
        textit,texttt,
        usepackage,
      }
    ,moretexcs=[6]^^A used packages
      {
        ^^A xfp
        fpeval,
        ^^A keyval
        setkeys,
        ^^A yax
        defactiveparameter,storevalue,setparameterlist
      }
    ,moretexcs=[2]^^A expkv-cs macros
      {
        ekvcSplit,ekvcSplitAndForward,ekvcSplitAndUse,
        ekvcHash,ekvcHashAndForward,ekvcHashAndUse,
        ekvcValue,ekvcValueFast,ekvcValueSplit,ekvcValueSplitFast,
        ekvcSecondaryKeys,
        ekvcChange,
        ekvcFlagNew,
        ekvcFlagHeight,
        ekvcFlagRaise,
        ekvcFlagSetTrue,ekvcFlagSetFalse,
        ekvcFlagIf,ekvcFlagIfRaised,
        ekvcFlagReset,
        ekvcFlagGetHeight,ekvcFlagGetHeights,
        ekvcPass
      }
    ,moretexcs=[3]^^A expkv-pkg macros
      {
        ^^A expkv
        ekvdef,ekvdefNoVal,^^A
        ekvlet,ekvletNoVal,ekvletkv,ekvletkvNoVal,^^A
        ekvdefunknown,ekvdefunknownNoVal,^^A
        ekvredirectunknown,ekvredirectunknownNoVal,^^A
        ekvset,ekvsetSneaked,^^A
        ekvsetdef,ekvsetSneakeddef,ekvsetdefSneaked,^^A
        ekvparse,^^A
        ekvVersion,ekvDate,^^A
        ekvifdefined,ekvifdefinedNoVal,ekvifdefinedset,^^A
        ekvbreak,ekvbreakPreSneak,ekvbreakPostSneak,^^A
        ekvsneak,ekvsneakPre,^^A
        ekvchangeset,^^A
        ekvoptarg,ekvoptargTF,^^A
        ekverr,
        ^^A expkv-def
        ekvdefinekeys
      }
  }
\colorlet{codeparam}{cyan!65!black}
\def\mylstwd{.55em}
\lstset
  {
    ,language=expkvc
    ,flexiblecolumns=false
    ,basewidth=\mylstwd
    ,gobble=2
    ,basicstyle=\fontfamily{jkp}\itshape
    ,texcsstyle=*[1]{\bfseries\color{expkvgrey}} ^^A primitives
    ,texcsstyle=*[2]{\bfseries\color{expkvred}}  ^^A expkv
    ,texcsstyle=*[3]{\color{expkvred}}           ^^A expkv-pkg
    ,texcsstyle=*[4]{\bfseries\color{expkvgrey}} ^^A e-TeX
    ,texcsstyle=*[5]{\bfseries\color{expkvgrey}} ^^A plain/LaTeX macros
    ,texcsstyle=*[6]{}                           ^^A macros of other packages
    ,commentstyle=\color[gray]{0.4}
    ,literate=
              {\{} {{\CodeUpBf\{}}{1}
              {\}} {{\CodeUpBf\}}}{1}
              {$}  {{\CodeUpBf\$}}{1}
              {[}  {{\CodeUp[}}{1}
              {]}  {{\CodeUp]}}{1}
              {(}  {{\CodeUp(}}{1}
              {)}  {{\CodeUp)}}{1}
              {*}  {{$*$}}{1}
              {1}  {{\CodeUp{1}}}{1}
              {2}  {{\CodeUp{2}}}{1}
              {3}  {{\CodeUp{3}}}{1}
              {4}  {{\CodeUp{4}}}{1}
              {5}  {{\CodeUp{5}}}{1}
              {6}  {{\CodeUp{6}}}{1}
              {7}  {{\CodeUp{7}}}{1}
              {8}  {{\CodeUp{8}}}{1}
              {9}  {{\CodeUp{9}}}{1}
              {0}  {{\CodeUp{0}}}{1}
              {##} {{\CodeColored{codeparam}{1}{\#}}}{1}
              {##1}{{\CodeColored{codeparam}{2}{\#1}}}{2}
              {##2}{{\CodeColored{codeparam}{2}{\#2}}}{2}
              {##3}{{\CodeColored{codeparam}{2}{\#3}}}{2}
              {##4}{{\CodeColored{codeparam}{2}{\#4}}}{2}
              {##5}{{\CodeColored{codeparam}{2}{\#5}}}{2}
              {##6}{{\CodeColored{codeparam}{2}{\#6}}}{2}
              {##7}{{\CodeColored{codeparam}{2}{\#7}}}{2}
              {##8}{{\CodeColored{codeparam}{2}{\#8}}}{2}
              {##9}{{\CodeColored{codeparam}{2}{\#9}}}{2}
              {<key>}{{$\langle$}key{$\rangle$}}{5}
              {<set>}{{$\langle$}set{$\rangle$}}{5}
              {<set1>}{{$\langle$}set1{$\rangle$}}{6}
              {<set2>}{{$\langle$}set2{$\rangle$}}{6}
    ^^A,literate=*{<key>}{\key}{4}{<set>}{\set}{4}
  }
\newcommand*\CodeColored[3]{\textcolor{#1}{\makebox[\dimexpr\mylstwd*#2]{#3}}}
\newcommand*\CodeUpBf[1]{\makebox[\mylstwd]{\textup{\textbf{#1}}}}
\newcommand*\CodeUp[1]{\makebox[\mylstwd]{\textup{#1}}}
\RequirePackage{randtext}
\let\metaORIG\meta
\protected\def\meta #1{\texttt{\metaORIG{#1}}}
\renewcommand*\thefootnote{\fnsymbol{footnote}}
\definecolor{expkvred}{HTML}{9F393D}
\colorlet{expkvgrey}{black!75}
\makeatletter
\newcommand*\example{\par\smallskip\noindent\textit{Example:} \ignorespaces}
\newcommand*\expkv
  {^^A
    \texorpdfstring
      {^^A
        \mbox
          {^^A
            \BeginAccSupp{ActualText=expkv}^^A
            \href{https://github.com/Skillmon/tex_expkv}
              {^^A
                \rmfamily
                \bfseries
                {\color{expkvgrey}e\kern-.05em x\kern-.05em}^^A
                \lower.493ex
                  \hbox{{\color{expkvgrey}P}\kern-.1em{\color{expkvred}k}}^^A
                \kern-.18em{\color{expkvred}v}^^A
              }^^A
            \EndAccSupp{}^^A
          }^^A
      }
      {expkv}^^A
  }
\newcommand*\expkvpkg[1]
  {^^A
    \texorpdfstring
      {^^A
        \mbox
          {^^A
            \BeginAccSupp{ActualText=expkv-#1}^^A
            \href{https://github.com/Skillmon/tex_expkv-#1}
              {^^A
                \rmfamily
                \bfseries
                {\color{expkvgrey}e\kern-.05em x\kern-.05em}^^A
                \lower.493ex
                  \hbox{{\color{expkvgrey}P}\kern-.1em{\color{expkvred}k}}^^A
                \kern-.18em{\color{expkvred}v}^^A
                {\color{expkvgrey}^^A
                  \kern.05em\rule[-.1ex]{.08em}{1.2ex}\kern.05em\textsc{#1}^^A
                }^^A
              }^^A
            \EndAccSupp{}^^A
          }^^A
      }
      {expkv-#1}^^A
  }
\newcommand*\expkvd{\expkvpkg{def}}
\newcommand*\expkvc{\expkvpkg{cs}}
\newcommand\kv{\meta{key}=\meta{value}}
\newcommand\key{\meta{key}}
\newcommand\val{\meta{value}}
\newcommand\set{\meta{set}}
\newcommand\enflong{\textcolor{black}{long}}
\newcommand\alllong{\textcolor{gray}{long}}
\newcommand\notlong{\textcolor{red!80!black}{long}}
\newcommand\prefixes[1]
  {^^A
    \hfill
    \ifcase\numexpr#1\relax\or
      \enflong\or
      \alllong\or
      \notlong\fi
  }
\hypersetup{linkcolor=red!80!black,urlcolor=purple!80!black}
\DoNotIndex{\def,\edef,\,,\=,\begingroup,\catcode,\chardef,\csname,\endcsname}
\DoNotIndex{\endgroup,\endinput,\errmessage,\expandafter,\input,\let,\long,\z@}
\DoNotIndex{\protected,\ProvidesFile,\ProvidesPackage,\relax,\space,\advance}
\DoNotIndex{\@,\unexpanded,\string,\expanded,\dimexpr,\global,\glueexpr,\hbox}
\DoNotIndex{\numexpr,\RequirePackage,\setbox,\the,\unless,\xdef,\gdef,\newcount}
\DoNotIndex{\number}
\DoNotIndex{\^,\@firstofone,\@firstoftwo,\@gobble,\@secondoftwo}
\DoNotIndex{\ifcsname}
\DoNotIndex{\ifx}
\DoNotIndex{\ifodd}
\DoNotIndex{\ifnum}
\DoNotIndex{\ifdefined}
\DoNotIndex{\iffalse}
\DoNotIndex{\iftrue}
\DoNotIndex{\else}
\DoNotIndex{\fi}
\@gobble\fi ^^A ignoring \ifx, \ifcsname, etc., but only one \fi
\@gobble\fi ^^A ignoring \ifx, \ifcsname, etc., but only one \fi
\@gobble\fi ^^A ignoring \ifx, \ifcsname, etc., but only one \fi
\@gobble\fi ^^A ignoring \ifx, \ifcsname, etc., but only one \fi
\@gobble\fi ^^A ignoring \ifx, \ifcsname, etc., but only one \fi
\@gobble\fi ^^A ignoring \ifx, \ifcsname, etc., but only one \fi
\@ifdefinable\gobbledocstriptag{\def\gobbledocstriptag#1>{}}
\makeatother
\begin{document}
  \title
    {^^A
      \texorpdfstring
        {^^A
          \huge\expkvc
          \\[\medskipamount]
          \Large define expandable \kv\ macros using \expkv
        }{expkv-cs - define expandable key=val macros using expkv}^^A
    }
  \date{\ekvcDate\space v\ekvcVersion}
  \author{Jonathan P. Spratte\thanks{\protect\randomize{jspratte@yahoo.de}}}
  \DocInput{expkv-cs.dtx}
\end{document}
%</driver>^^A=<<
% \fi
%
% \maketitle
% \renewcommand*\thefootnote{\arabic{footnote}}
%
% \begin{abstract}
% \noindent\parfillskip=0pt
% \expkvc\ provides two small interfaces to define expandable \kv\ macros using
% \expkv. It therefore lowers the entrance boundary to expandable \kv\ macros.
% The stylised name is \expkvc\ but the files use \file{expkv-cs}, this is due
% to CTAN-rules which don't allow \string| in package names since that is the
% pipe symbol in *nix shells.
% \end{abstract}
%
% \tableofcontents
%
% \begin{documentation}^^A>>=
%
% \section{Documentation}
%
% The \expkv\ package enables the new possibility of creating \kv\ macros which
% are fully expandable. The creation of such macros is however cumbersome for
% the average user. \expkvc\ tries to step in here. It provides interfaces to
% define \kv\ macros without worrying too much about the implementation. In case
% you're wondering now, the |cs| in \expkvc\ stands for control sequence,
% because |def| was already taken by \expkvd\ and ``control sequence'' is the
% term D.\,E.\,Knuth used in his \TeX book for named commands hence macros
% (though he also used the term ``macro''). So \expkvc\ defines control
% sequences for and with \expkv.
%
% There are two different approaches supported by this package. The first is
% splitting the keys up into individual arguments, the second is providing all
% the keys as a single argument to the underlying macro and getting an
% individual \val\ by using a hash. Well, actually there is no real hash, just
% some markers which are parsed, but this shouldn't be apparent to the user, the
% behaviour matches that of a hash-table.
%
% In addition to these two methods of defining a macro with primary keys a way
% to define secondary keys, which can reference the primary ones, is provided.
% These secondary keys don't correspond to an argument or an entry in the hash
% table directly but might come in handy for the average use case. Each macro
% has its own set of primary and secondary keys.
%
% A word of advice you should consider: If your macro doesn't have to be
% expandable (and often it doesn't) consider not using \expkvc. The interface
% has some overhead (though it still is fast -- check \autoref{sec:speed}) and
% the approach has its limits in versatility. If you don't need to be
% expandable, you should consider either defining your keys manually using
% \expkv\ or using \expkvd\ for convenience. Or you resort to another \kv\
% interface. Nevertheless setting up macros with \expkvc, especially with
% |\ekvcSplit|, is very convenient in my opinion, so if you just want to define
% a single macro with just a few keys this might be the way to go.
%
% \expkvc\ is usable as generic code and as a \LaTeX\ package. It'll
% automatically load \expkv\ in the same mode as well. To use it, just use one
% of
% \begin{lstlisting}
% \usepackage{expkv-cs} % LaTeX
% \input expkv-cs       % plainTeX
% \end{lstlisting}
%
%
% \subsection{Define Macros and Primary Keys}
%
% All macros defined with \expkvc\ have to be previously undefined or have the
% |\meaning| of |\relax|. This is necessary as there is no way to automatically
% undefine keys once they are set up (neither \expkv\ nor \expkvc\ keep track of
% defined keys) -- so to make sure there are no conflicts only new definitions
% are allowed (that's not the case for individual keys, only for frontend
% macros).
%
%
% \subsubsection{Primary Keys}\label{sec:primaries}
%
% In the following descriptions there will be one argument named \meta{primary
% keys}. This argument should be a \kv\ list where each \key\ will be one
% primary key and \val\ the associated initial value. By default all keys are
% defined short, but you can define long keys by prefixing \key\ with |long|
% (\emph{e.g.}, \texttt{long name=Jonathan P. Spratte}). You only need |long| if
% the key should be able to take a |\par| token. Note however that |long| keys
% are a microscopic grain faster (due to some internals of \expkvc). Only if at
% least one of the keys was |long| the \meta{cs} in the following defining
% macros will be |\long|. For obvious reasons there is no possibility to define
% a macro or key as |\protected|.
%
% At the moment \expkvc\ doesn't require any internal keys, but I can't foresee
% whether this will be the case in the future as well, as it might turn out that
% some features I deem useful can't be implemented without such internal keys.
% Because of this, please don't use key names starting with
% \texttt{EKVC\string|} as that should be the private name space.
%
%
% \subsubsection{Split}
%
% The split variants will provide the key values as separate arguments. This
% limits the number of keys for which this is truly useful.
%
% \begin{function}{\ekvcSplit}
%   \begin{syntax}
%     \cs{ekvcSplit}\meta{cs}\marg{primary keys}\marg{definition}
%   \end{syntax}
%   This defines \meta{cs} to be a macro taking one mandatory argument which
%   should contain a \kv\ list. The \meta{primary keys} will be defined for this
%   macro (see \autoref{sec:primaries}). The \meta{definition} is the code that
%   will be executed. You can access the \val\ of a \key\ by using a macro
%   parameter from |#1| to |#9|. The order of the macro parameters will be the
%   order provided in the \meta{primary keys} list (so |#1| is the \val\ of the
%   key defined first). With |\ekvcSplit| you can define macros using at most
%   nine primary keys.
% \end{function}
% \example The following defines a macro |\foo| that takes the keys |a| and |b|
% and outputs their values in a textual form:\\
% \begin{minipage}{.75\linewidth}
% \begin{lstlisting}
% \ekvcSplit\foo{a=a,b=b}{a is #1.\par b is #2.\par}
% \foo{}
% \foo{b=e}
% \end{lstlisting}
% \end{minipage}^^A
% \begin{exresult}[nobeforeafter,box align=center]{.25\linewidth}
%   \ekvcSplit\foo{a=a,b=b}{a is #1.\par b is #2.\par}
%   \foo{}
%   \foo{b=e}
% \end{exresult}
%
% \begin{function}{\ekvcSplitAndForward}
%   \begin{syntax}
%     \cs{ekvcSplitAndForward}\meta{cs}\marg{after}\marg{primary keys}
%   \end{syntax}
%   This defines \meta{cs} to be a macro taking one mandatory argument which
%   should contain a \kv\ list. You can use as many primary keys as you want
%   with this. The primary keys will be forwarded to \meta{after} as braced
%   arguments (as many as necessary for your primary keys). The order of the
%   braced arguments will be the order of your primary key definitions. In
%   \meta{after} you can use just a single control sequence, or some arbitrary
%   stuff which will be left in the input stream before your braced values (with
%   one set of braces stripped from \meta{after}), so both of the following
%   would be fine:
% \begin{lstlisting}
% \ekvcSplitAndForward\foo\foo@aux{keyA = A, keyB = B}
% \ekvcSplitAndForward\foo{\foo@aux{more args}}{keyA = A, keyB = B}
% \end{lstlisting}
% \end{function}
% In the first case |\foo@aux| should take at least two arguments (|keyA| and
% |keyB|), in the second case at least three (|more args|, |keyA|, and |keyB|).
%
% \begin{function}{\ekvcSplitAndUse}
%   \begin{syntax}
%     \cs{ekvcSplitAndUse}\meta{cs}\marg{primary keys}
%   \end{syntax}
%   This will roughly do the same as |\ekvcSplitAndForward|, but instead of
%   specifying what will be used after splitting the keys, \meta{cs} will use
%   what follows the \kv\ list. So its syntax will be
% \end{function}
%   \begin{syntax}
%     \meta{cs}\{\key=\val, \ldots\}\marg{after}
%   \end{syntax}
% and the code in |after| should expect at least as many arguments as the number
% of keys defined for \meta{cs}.
%
% \subsubsection{Hash}
%
% The hash variants will provide the key values as a single argument in which
% you can access specific values using a special macro. The implementation might
% be more convenient and scale better, \emph{but} it is much slower (for a
% primitive macro with a single key benchmarking was almost $1.7$ times slower,
% the root of which being the key access with |\ekvcValue|, not the parsing, and
% for a key access using |\ekvcValueFast| it was still about $1.2$ times
% slower). So if your macro uses less than ten primary keys, you should consider
% using the split approach.
%
% \begin{function}{\ekvcHash}
%   \begin{syntax}
%     \cs{ekvcHash}\meta{cs}\marg{primary keys}\marg{definition}
%   \end{syntax}
%   This defines \meta{cs} to be a macro taking one mandatory argument which
%   should contain a \kv\ list. You can use as many primary keys as you want.
%   The primary keys will be forwarded as a single argument containing every key
%   to the underlying macro. The underlying macro is defined as
%   \meta{definition}, in which you can access the \val\ of a \key\ by using
%   \texttt{\cs[no-index]{ekvcValue}\{\key\}\{\#1\}}.
% \end{function}
% \example This defines an equivalent macro like the |\foo| defined with
% |\ekvcSplit| earlier:\\
% \begin{minipage}{.75\linewidth}
% \begin{lstlisting}
% \ekvcHash\foo{a=a,b=b}{a is \ekvcValue{a}{#1}.\par
%                        b is \ekvcValue{b}{#1}.\par}
% \foo{}
% \foo{b=e}
% \end{lstlisting}
% \end{minipage}^^A
% \begin{exresult}[nobeforeafter,box align=center]{.25\linewidth}
%   \ekvcHash\foo{a=a,b=b}{a is \ekvcValue{a}{#1}.\par
%                          b is \ekvcValue{b}{#1}.\par}
%   \foo{}
%   \foo{b=e}
% \end{exresult}
%
% \begin{function}{\ekvcHashAndForward}
%   \begin{syntax}
%     \cs{ekvcHashAndForward}\meta{cs}\marg{after}\marg{primary keys}
%   \end{syntax}
%   This defines \meta{cs} to be a macro taking one mandatory argument which
%   should contain a \kv\ list. You can use as many primary keys as you want.
%   The primary keys will be forwarded as a single argument containing every key
%   to \meta{after}. You can use a single macro for \meta{after} or use some
%   arbitrary stuff, which will be left in the input stream before the hashed
%   \kv\ list with one set of braces stripped. In the
%   macro called in \meta{after} you can access the \val\ of a \key\ by using
%   \texttt{\cs[no-index]{ekvcValue}\{\key\}\{\#1\}} (or whichever argument the
%   hashed \kv\ list will be).
% \end{function}
%
% \begin{function}{\ekvcHashAndUse}
%   \begin{syntax}
%     \cs{ekvcHashAndUse}\meta{cs}\marg{primary keys}
%   \end{syntax}
%   This will roughly do the same as |\ekvcHashAndForward|, but instead of
%   specifying what will be used after hashing the keys, \meta{cs} will use what
%   follows the \kv\ list. So its syntax will be
% \end{function}
%   \begin{syntax}
%     \meta{cs}\{\key=\val, \ldots\}\marg{after}
%   \end{syntax}
%
% \begin{function}{\ekvcValue}
%   \begin{syntax}
%     \cs{ekvcValue}\{\key\}\marg{key list}
%   \end{syntax}
%   This is a safe way to access your keys in a hash variant. \key\
%   is the key which's \val\ you want to use out of the \meta{key list}.
%   \meta{key list} should be the key list argument forwarded to your underlying
%   macro by |\ekvcHash|, |\ekvcHashAndForward|, or |\ekvcHashAndUse|. It will
%   be tested whether the hash function to access that \key\ exists, the \key\
%   argument is not empty, and that the \meta{key list} really contains a \val\
%   of that \key. This macro needs exactly two steps of expansion and if used
%   inside of an |\edef| or |\expanded| context will protect the \val\ from
%   further expanding.
% \end{function}
%
% \begin{function}{\ekvcValueFast}
%   \begin{syntax}
%     \cs{ekvcValueFast}\{\key\}\marg{key list}
%   \end{syntax}
%   This behaves just like |\ekvcValue|, but \emph{without any} safety tests. As
%   a result this is about $1.4$ times faster \emph{but} will throw low level
%   \TeX\ errors eventually if the hash function isn't defined or the \key\
%   isn't part of the \meta{key list} (\emph{e.g.}, because it was defined as a
%   key for another macro -- all macros share the same hash function per \key).
%   Use it if you know what you're doing. This macro needs exactly three steps
%   of expansion in the no-errors case.
% \end{function}
%
% \begin{function}{\ekvcValueSplit}
%   \begin{syntax}
%     \cs{ekvcValueSplit}\{\key\}\marg{key list}\marg{next}
%   \end{syntax}
%   If you need a specific \key\ from a \meta{key list} more than once, it'll
%   be a good idea to only extract it once and from then on keep it as a
%   separate argument. Hence the macro \cs{ekvcValueSplit} will extract one
%   specific \key's value from the list and forward the remainder of the list
%   as the first and the \key's value as the second argument to \meta{next}, so
%   the result of this will be \meta{next}\marg{key list'}\marg{value} with
%   \meta{key list'} the remaining list. This is almost as fast as |\ekvcValue|
%   and runs the same tests. Keep in mind that you can't fetch for the same
%   \key\ again from \meta{key list'} as it got removed.
% \end{function}
% \example The following defines a macro |\foo| which will take three keys.
% Since the next parsing step will need the value of one of the keys multiple
% times we split that key off the list:\\
% \begin{minipage}{.75\linewidth}
% \begin{lstlisting}
% \ekvcHash\foo{a=a,b=b,c=c}
%   {\ekvcValueSplit{a}{#1}\foobar}
% \newcommand*\foobar[2]{a is #2.\par
%                        b is \ekvcValue{b}{#1}.\par
%                        c is \ekvcValue{c}{#1}.\par}
% \foo{}
% \end{lstlisting}
% \end{minipage}^^A
% \begin{exresult}[nobeforeafter,box align=center]{.25\linewidth}
%   \ekvcHash\foo{a=a,b=b,c=c}
%     {\ekvcValueSplit{a}{#1}\foobar}
%   \newcommand*\foobar[2]{a is #2.\par
%                          b is \ekvcValue{b}{#1}.\par
%                          c is \ekvcValue{c}{#1}.\par}
%   \foo{}
% \end{exresult}
%
% \begin{function}{\ekvcValueSplitFast}
%   \begin{syntax}
%     \cs{ekvcValueSplitFast}\{\key\}\marg{key list}\marg{next}
%   \end{syntax}
%   This behaves just like |\ekvcValueSplit|, but it won't run the same tests,
%   hence it is faster but more error prone, just like the relation between
%   |\ekvcValue| and |\ekvcValueFast|.
% \end{function}
%
%
% \subsection{Secondary Keys}\label{sec:secondaries}
%
% To remove some of the limitations with the approach that each primary key
% matches an argument or hash entry, you can define secondary keys. Those have
% to be defined for each macro but it doesn't matter whether that macro was a
% split or a hash variant. If a secondary key references another key it doesn't
% matter whether that other key is primary or secondary.
%
% Secondary keys can have a prefix (like |long|) which are called |p|-type
% prefix and must have a type (like |meta|) which are called |t|-type prefix.
% Some types might require some |p|-prefixes, while others might forbid those.
%
% Please keep in mind that key names shouldn't start with \texttt{EKVC\string|}.
%
% \begin{function}{\ekvcSecondaryKeys}
%   \begin{syntax}
%     \cs{ekvcSecondaryKeys}\meta{cs}\{\kv, \ldots\}
%   \end{syntax}
%   This is the front facing macro to define secondary keys. For the macro
%   \meta{cs} define \key\ to have definition \val. The general syntax for \key\
%   should be
%   \begin{quote}\ttfamily\small
%     \meta{prefix} \meta{name}
%   \end{quote}
%   Where \meta{prefix} is a space separated list of optional |p|-type prefixes
%   followed by one |t|-type prefix. The syntax of \val\ is dependent on the
%   used |t|-prefix.
% \end{function}
%
%
% \subsubsection{\texttt{p}-type Prefixes}
%
% There is only one |p|-prefix available, which is |long|.
%
% \begin{function}{long}
%   The following key will be defined |\long|.
% \end{function}
%
%
% \subsubsection{\texttt{t}-type Prefixes}
%
% If you're familiar with \expkvd\ you'll notice that the |t|-type prefixes
% provided here are much fewer. The expansion only concept doesn't allow for
% great variety in the auto-defined keys.
%
% The syntax examples of the |t|-prefixes will show which |p|-prefix will be
% automatically used by printing those black (\texttt{\enflong}), which
% will be available in grey (\texttt{\alllong}), and which will be disallowed in
% red (\texttt{\notlong}). This will be put flush right next to the syntax line.
%
% \begin{function}{meta}
%   \begin{syntax}
%     meta \key\ = \{\kv, \ldots\} \prefixes2
%   \end{syntax}
%   With a |meta| key you can set other keys. Whenever \key\ is used the keys in
%   the \kv\ list will be set to the values given there. You can use the \val\
%   given to \key\ by using |#1| in the \kv\ list. The keys in the \kv\ list can
%   be primary and secondary ones.
% \end{function}
%
% \begin{function}{nmeta}
%   \begin{syntax}
%     nmeta \key\ = \{\kv, \ldots\} \prefixes3
%   \end{syntax}
%   An |nmeta| key is like a |meta| key, but it doesn't take a value, so the
%   \kv\ list is static.
% \end{function}
%
% \begin{function}{alias}
%   \begin{syntax}
%     alias \key\ = \meta{key_2} \prefixes3
%   \end{syntax}
%   This assigns the definition of \meta{key_2} to \key. As a result \key\ is an
%   alias for \meta{key_2} behaving just the same. Both the value taking and the
%   |NoVal| version (that's \expkv\ slang for a key not accepting a value) will
%   be copied if they are defined when |alias| is used. Of course, \meta{key_2}
%   has to be defined, be it as a primary or secondary one.
% \end{function}
%
% \begin{function}{default}
%   \begin{syntax}
%     default \key\ = \marg{default} \prefixes3
%   \end{syntax}
%   If \key\ is a defined value taking key, you can define a |NoVal| version
%   with this that will behave as if \key\ was given \meta{default} as its
%   \val. Note that this doesn't change the initial values of primary keys set
%   at definition time in |\ekvcSplit| and friends. \key\ can be a primary or
%   secondary key.
% \end{function}
%
% \begin{function}{flag-bool}
%   \begin{syntax}
%     flag-bool \key\ = \meta{cs} \prefixes3
%   \end{syntax}
%   This is a secondary key that doesn't directly involve any of the primary or
%   secondary keys.
%   This defines \key\ to take a value, which should be either |true| or
%   |false|, and set the flag called \meta{cs} accordingly as a boolean.
%   If \meta{cs} isn't defined yet it will be initialised as a flag.
%   Please also read \autoref{sec:flags}.
% \end{function}
%
% \begin{function}{flag-true,flag-false}
%   \begin{syntax}
%     flag-true \key\ = \meta{cs} \prefixes3
%   \end{syntax}
%   This is a secondary key that doesn't directly involve any of the primary or
%   secondary keys.
%   This defines \key\ to take no value and set the flag called \meta{cs} to
%   |true| or |false|, respectively.
%   If \meta{cs} isn't defined yet it will be initialised as a flag.
%   Please also read \autoref{sec:flags}.
% \end{function}
%
% \begin{function}{flag-raise}
%   \begin{syntax}
%     flag-raise \key\ = \meta{cs} \prefixes3
%   \end{syntax}
%   This is a secondary key that doesn't directly involve any of the primary or
%   secondary keys.
%   This defines \key\ to take no value and raise the flag called \meta{cs}.
%   If \meta{cs} isn't defined yet it will be initialised as a flag.
%   Please also read \autoref{sec:flags}.
% \end{function}
%
%
% \subsection{Changing the Initial Values}\label{sec:change}
%
% \begin{function}{\ekvcChange}
%   \begin{syntax}
%     \cs{ekvcChange}\meta{cs}\{\kv,\ldots\}
%   \end{syntax}
%   This processes the \kv\ list (just like it would be processed by \meta{cs})
%   to set new defaults for the macro \meta{cs} defined with \expkvc\ (meaning
%   the values used if you don't provide anything at use time). Inside the \kv\
%   list both primary and secondary keys can be used. If \meta{cs} was defined
%   |\long| earlier it will still be |\long|, every other \TeX\ prefix will be
%   stripped (but \expkvc\ doesn't support them anywhere else so that should be
%   fine). The resulting new defaults will be stored inside the \meta{cs}
%   locally (just as the original defaults were).
% \end{function}
%
% Consider the following example:\\
% \begin{minipage}[c]{.75\linewidth}
% \begin{lstlisting}
% \ekvcSplit\foo{a=a,b=b}{a is #1.\par b is #2.\par}
% \foo{} % initial defaults
% \ekvcChange\foo{b=B}
% \foo{} % new defaults
% \ekvcSecondaryKeys\foo{meta c={a={#1},b={#1}}}
% \ekvcChange\foo{c=c}
% \foo{} % newer defaults
% \end{lstlisting}
% \end{minipage}^^A
% \begin{exresult}[nobeforeafter,box align=center]{.25\linewidth}
%   \ekvcSplit\foo{a=a,b=b}{a is #1.\par b is #2.\par}
%   \foo{}
%   \ekvcChange\foo{b=B}
%   \foo{}
%   \ekvcSecondaryKeys\foo{meta c={a={#1},b={#1}}}
%   \ekvcChange\foo{c=c}
%   \foo{}
% \end{exresult}
%
% As a result with this the typical setup macro could be implemented:
% \begin{lstlisting}
% \ekvcHashAndUse\foo{key=a,key=b}
% \newcommand\foosetup{\ekvcChange\foo}
% \end{lstlisting}
% Of course the usage is limited to a single macro |\foo|, hence this might be
% not as powerful as similar macros used with other \kv\ interfaces. But at
% least a few similar macros could be grouped using the same key parsing macro
% internally.
%
%
% \subsection{Flags}\label{sec:flags}
%
% The idea of flags is taken from \pkg{expl3}. They provide a way to store
% numerical information expandably, however only incrementing and accessing
% works expandably, decrementing is unexpandable. A flag has a height, which
% is a numerical value, and which can be raised by $1$. Flags come at a high
% computational cost (accessing them is slow and they require more memory than
% normal \TeX\ data types like registers, both getting linearly worse with the
% height), so don't use them if not necessary.
%
% The state of flags is always changed locally to the current group, but not to
% the current macro, so if you're using one of the |t|-types involving flags
% bear in mind that they can affect other macros using the same flags at the
% current group level!
%
% \expkvc\ provides some macros to access, alter, and use flags. Flags of
% \expkvc\ don't share a name space with the flags of \pkg{expl3}.
%
% \begin{function}{\ekvcFlagNew}
%   \begin{syntax}
%     \cs{ekvcFlagNew}\meta{flag}
%   \end{syntax}
%   This initialises the macro \meta{flag} as a new flag. It isn't checked
%   whether the macro \meta{flag} is currently undefined.  A \meta{flag} will
%   expand to the flag's current height with a trailing space (so you can use it
%   directly with |\ifnum| for example and it will terminate the number scanning
%   itself).
% \end{function}
%
%  All other macros dealing with flags take as a parameter a macro defined as a
%  \meta{flag} with |\ekvcFlagNew|.
%
% \begin{function}{\ekvcFlagHeight}
%   \begin{syntax}
%     \cs{ekvcFlagHeight}\meta{flag}
%   \end{syntax}
%   This expands to the current height of \meta{flag} in a single step of
%   expansion (without a trailing space).
% \end{function}
%
% \begin{function}{\ekvcFlagRaise}
%   \begin{syntax}
%     \cs{ekvcFlagRaise}\meta{flag}
%   \end{syntax}
%   This expandably raises the height of \meta{flag} by $1$.
% \end{function}
%
% \begin{function}{\ekvcFlagSetTrue,\ekvcFlagSetFalse}
%   \begin{syntax}
%     \cs{ekvcFlagSetTrue}\meta{flag}
%   \end{syntax}
%   By interpreting an even value as false and an odd value as true we can use a
%   flag as a boolean. This expandably sets \meta{flag} to true or false,
%   respectively, by raising it if necessary.
% \end{function}
%
% \begin{function}{\ekvcFlagIf}
%   \begin{syntax}
%     \cs{ekvcFlagIf}\meta{flag}\marg{true}\marg{false}
%   \end{syntax}
%   This interprets a \meta{flag} as a boolean and expands to either \meta{true}
%   or \meta{false}.
% \end{function}
%
% \begin{function}{\ekvcFlagIfRaised}
%   \begin{syntax}
%     \cs{ekvcFlagIfRaised}\meta{flag}\marg{true}\marg{false}
%   \end{syntax}
%   This tests whether the \meta{flag} is raised, meaning it has a height
%   greater than zero, and if so expands to \meta{true} else to \meta{false}.
% \end{function}
%
% \begin{function}{\ekvcFlagReset}
%   \begin{syntax}
%     \cs{ekvcFlagReset}\meta{flag}
%   \end{syntax}
%   This resets a flag (so restores its height to $0$). This operation is
%   \emph{not} expandable and done locally. If you really intend to use flags
%   you can reset them every now and then to keep the performance hit low.
% \end{function}
%
% \begin{function}{\ekvcFlagGetHeight}
%   \begin{syntax}
%     \cs{ekvcFlagGetHeight}\meta{flag}\marg{next}
%   \end{syntax}
%   This retrieves the current height of the \meta{flag} and provides it as a
%   braced argument to \meta{next}, leaving \meta{next}\marg{height} in the
%   input stream.
% \end{function}
%
% \begin{function}{\ekvcFlagGetHeights}
%   \begin{syntax}
%     \cs{ekvcFlagGetHeights}\marg{flag-list}\marg{next}
%   \end{syntax}
%   This retrieves the current height of each \meta{flag} in the
%   \meta{flag-list} and provides them as a single argument to \meta{next}.
%   Inside that argument each height is enclosed in a set of braces
%   individually. The \meta{flag-list} is just a single argument containing the
%   \meta{flag}s.
%   So a usage like
%   \texttt
%     {^^A
%       \cs[no-index]{ekvcFlagGetHeights}\hskip0pt^^A
%       \{\cs[no-index]{myflagA}\cs[no-index]{myflagB}\}\hskip0pt^^A
%       \{\cs[no-index]{stuff}\}^^A
%     }
%   will expand to
%   \texttt{\cs[no-index]{stuff}\{\marg{height-A}\marg{height-B}\}}.
% \end{function}
%
%
% \subsection{Further Examples}
%
% How could a documentation be a good documentation without some basic examples?
% Say we want to define a small macro expanding to some character description
% (who knows why this has to be expandable?). A character description will not
% have too many items to it, so we use |\ekvcSplit| (the comments with the
% parameter numbers are of course not necessary and just ease reading the
% example).
% \begin{lstlisting}
% \ekvcSplit\character
%   {
%     name=John Doe,            % #1
%     age=any,                  % #2
%     nationality=the Universe, % #3
%     hobby=to exist,           % #4
%     type=Mister,              % #5
%     pronoun=He,               % #6
%     possessive=his,           % #7
%   }
%   {#1 is a #5 from #3. #6 is of #2 age and #7 hobby is #4.\par}
% \end{lstlisting}
% Also we want to give some short cuts so that it's easier to describe several
% persons.
% \begin{lstlisting}
% \ekvcSecondaryKeys\character
%   {
%     alias pro = pronoun,
%     alias pos = possessive,
%     nmeta me =
%       {
%         name=Jonathan,
%         age=a young,
%         nationality=Germany,
%         hobby=\TeX\ coding,
%       },
%     meta lady =
%       {type=Lady, pronoun=She, possessive=her, name=Jane Doe, #1},
%     nmeta paulo =
%       {
%         name=Paulo,
%         type=duck,
%         age=a young,
%         nationality=Brazil,
%         hobby=to quack,
%       }
%   }
% \end{lstlisting}
% Now we can describe people using
% \begin{lstlisting}
% \character{}
% \character{me}
% \character{paulo}
% \character
%   {lady={name=Evelyn,nationality=Ireland,age=the best,hobby=reading}}
% \character
%   {
%     name=Our sun, type=star, nationality=our solar system, pro=It,
%     age=an old, pos=its, hobby=shining
%   }
% \end{lstlisting}
% As one might see, the |lady| key could actually have been an |nmeta| key as
% well, as all that is done with the argument is using it as a \kv\ list.
%
% The result of only the first two usages would be:
% \begin{exresult}{\linewidth}
%   \ekvcSplit\character
%     {
%       name=John Doe,
%       age=any,
%       nationality=the Universe,
%       hobby=to exist,
%       type=Mister,
%       pronoun=He,
%       possessive=his,
%     }
%     {#1 is a #5 from #3. #6 is of #2 age and #7 hobby is #4.\par}
%   \ekvcSecondaryKeys\character
%     {
%       alias pro = pronoun,
%       alias pos = possessive,
%       nmeta me =
%         {
%           name=Jonathan,
%           age=a young,
%           nationality=Germany,
%           hobby=\TeX\ coding,
%         },
%       meta lady =
%         {type=Lady, pronoun=She, possessive=her, name=Jane Doe, #1},
%       nmeta paulo =
%         {
%           name=Paulo,
%           type=duck,
%           age=a young,
%           nationality=Brazil,
%           hobby=to quack,
%         }
%     }
%   \character{}
%   \character{me}
% \end{exresult}
%
% Using \pkg{xparse} or \expkv's \cs{ekvoptarg} or \cs{ekvoptargTF} and
% forwarding arguments one can easily define \kv\ macros with actual optional
% and mandatory arguments as well. A small nonsense example (which should
% perhaps use |\ekvcSplitAndForward| instead of |\ekvcHashAndForward| since it
% only uses four keys and one other argument -- and isn't expandable since it
% uses a \env{tabular} environment, so it would've been better to use a more
% feature rich \kv\ interface most likely, \emph{e.g.}, the one provided by
% \expkvd):
% \begin{lstlisting}
% \makeatletter
% \newcommand*\nonsense{\ekvoptarg\nonsense@a{}}
% \ekvcHashAndForward\nonsense@a\nonsense@b
%   {
%     keyA = A,
%     keyB = B,
%     keyC = c,
%     keyD = d,
%   }
% \newcommand*\nonsense@b[2]
%   {%
%     \begin{tabular}{lll}
%       key & A & \ekvcValue{keyA}{#1} \\
%           & B & \ekvcValue{keyB}{#1} \\
%           & C & \ekvcValue{keyC}{#1} \\
%           & D & \ekvcValue{keyD}{#1} \\
%       \multicolumn{2}{l}{mandatory} & #2 \\
%     \end{tabular}%
%   }
% \makeatother
% \end{lstlisting}
% And then we would be able to do some nonsense
% \begin{lstlisting}
% \nonsense{}
% \nonsense[keyA=hihi]{haha}
% \nonsense[keyA=hihi, keyB=A]{hehe}
% \nonsense[keyC=huhu, keyA=hihi, keyB=A]{haha}
% \end{lstlisting}
% resulting in
% \begin{exresult}{\linewidth}
%   \makeatletter
%   \newcommand*\nonsense{\ekvoptarg\nonsense@a{}}
%   \ekvcHashAndForward\nonsense@a\nonsense@b
%     {
%       keyA = A,
%       keyB = B,
%       keyC = c,
%       keyD = d,
%     }
%   \newcommand*\nonsense@b[2]
%     {%
%       \begin{tabular}{lll}
%         key & A & \ekvcValue{keyA}{#1} \\
%             & B & \ekvcValue{keyB}{#1} \\
%             & C & \ekvcValue{keyC}{#1} \\
%             & D & \ekvcValue{keyD}{#1} \\
%         \multicolumn{2}{l}{mandatory} & #2 \\
%       \end{tabular}%
%     }
%   \makeatother
%   \nonsense{}
%   \nonsense[keyA=hihi]{haha}
%   \nonsense[keyA=hihi, keyB=A]{hehe}
%   \nonsense[keyC=huhu, keyA=hihi, keyB=A]{haha}
% \end{exresult}
%
%
% \subsection{Freedom for Keys!}
%
% If this was the \TeX book this subsection would have a double bend sign. Not
% because it is overly complicated, but because it shows things which could
% break \expkvc's expandability and its alignment safety. This is for
% experienced users wanting to get the most flexibility and knowing what they
% are doing.
%
% In case you're wondering, it is possible to define other keys than the
% primaries and the secondary types listed in \autoref{sec:secondaries} for a
% macro defined with \expkvc\ by using the low-level interface of \expkv\ or
% even the interface provided by \expkvd. The set name used for \expkvc's keys
% is the macro name, including the leading backslash, or more precisely
% |\string|\meta{cs} is used. This can be exploited to define additional keys
% with arbitrary code. Consider the following \emph{bad} example:
% \begin{lstlisting}
% \ekvcSplit\foo{a=A,b=B}{a is #1.\par b is #2}
% \protected\ekvdef{\string\foo}{c}{\def\fooC{#1}}
% \end{lstlisting}
% This would define a key named |c| that will store its value inside a macro.
% The issue with this is that this can't be done expandably. As a result, the
% macro |\foo| isn't always expandable any more (not that bad if this was never
% required; killjoy if it was) and as soon as the key |c| is used, it is also no
% longer alignment safe\footnote{This means that the \kv-list can't contain
%   alignment markers that are not inside an additional set of braces if used
%   inside a \TeX\ alignment.}
% (might be bad depending on the usage).
%
% So why do I show you this? Because we could as well do something useful like
% create a key that pre-parses the input and after that passes the parsed value
% on. This parsing would have to be completely expandable though. For the
% pass-on part we can use the following function:
%
% \begin{function}{\ekvcPass}
%   \begin{syntax}
%     \cs{ekvcPass}\meta{cs}\marg{key}\marg{value}
%   \end{syntax}
%   This passes \meta{value} on to \meta{key} for the \expkvc-macro \meta{cs}.
%   It should be used inside the key parsing of a macro defined with \expkvc,
%   else this most likely results in a low level \TeX\ error.
% \end{function}
%
% With this we could for example split the value of a key at a hyphen and pass
% the parts to different keys:\\
% \begin{minipage}[c]{.75\linewidth}
% \begin{lstlisting}
% \ekvcSplit\foo{a=A,b=B}{a is #1.\par b is #2.\par}
% \ekvdef{\string\foo}{c}{\fooSplit#1\par}
% \def\fooSplit#1-#2\par
%   {\ekvcPass\foo{a}{#1}\ekvcPass\foo{b}{#2}}
% \foo{}
% \foo{c=1-2}
% \end{lstlisting}
% \end{minipage}
% \begin{exresult}[nobeforeafter,box align=center]{.25\linewidth}
%   \ekvcSplit\foo{a=A,b=B}{a is #1.\par b is #2.\par}
%   \ekvdef{\string\foo}{c}{\fooSplit#1\par}
%   \def\fooSplit#1-#2\par{\ekvcPass\foo{a}{#1}\ekvcPass\foo{b}{#2}}
%   \foo{}
%   \foo{c=1-2}
% \end{exresult}
%
%
% \subsection{Speed Considerations}\label{sec:speed}
%
% As already mentioned in the introduction there are some speed considerations
% implied if you choose to define macros via \expkvc. However the overhead isn't
% the factor which should hinder you to use \expkvc\ if you found a reasonable
% use case. The key-parsing is still faster than with most other \kv\ packages
% (see the ``Comparisons'' subsection in the \expkv\ documentation).
%
% The speed considerations in this subsection use the first example in this
% documentation as the benchmark. So we have seven keys and a short sentence
% which should be typeset. For comparisons I use the following equivalent
% \expkvd\ definitions. Each result is the average between changing no keys from
% their initial values and altering four. Furthermore I'll compare three
% variants of \expkvc\ with the \expkvd\ definitions, namely the split example
% from above, a hash variant using |\ekvcValue| and a hash variant using
% |\ekvcValueFast|.
% \begin{lstlisting}
% \usepackage{expkv-def}
% \ekvdefinekeys{keys}
%   {%
%     ,store   name         = \KEYSname
%     ,initial name         = John Doe
%     ,store   age          = \KEYSage
%     ,initial age          = any
%     ,store   nationality  = \KEYSnationality
%     ,initial nationality  = the Universe
%     ,store   hobby        = \KEYShobby
%     ,initial hobby        = to exist
%     ,store   type         = \KEYStype
%     ,initial type         = Mister
%     ,store   pronoun      = \KEYSpronoun
%     ,initial pronoun      = He
%     ,store   possessive   = \KEYSpossessive
%     ,initial possessive   = his
%   }
% \newcommand*\KEYS[1]
%   {%
%     \begingroup
%       \ekvset{keys}{#1}%
%       \KEYSname\ is a \KEYStype\ from \KEYSnationality. \KEYSpronoun\ is
%       of \KEYSage\ age and \KEYSpossessive\ hobby is \KEYShobby.%
%     \endgroup
%   }
% \end{lstlisting}
%
% The first comparison removes the typesetting part from all the definitions, so
% that only the key parsing is compared. In this comparison the |\ekvcValue|
% and |\ekvcValueFast| variants will not differ, as they are exactly the same
% until the key usage. We find that the split approach is $1.4$ times slower
% than the \expkvd\ setup and the hash variants end up in the middle at $1.17$
% times slower.
%
% Next we put the typesetting part back in. Every call of the macros will
% typeset the sentences into a box register in horizontal mode. With the
% typesetting part (which includes the accessing of values) the fastest remains
% the \expkvd\ definitions, but split is close at $1.16$ times slower, followed
% by the hash variant with fast accesses at $1.36$ times slower, and the safe
% hash access variant ranks in the slowest $1.8$ times slower than \expkvd.
%
% Just in case you're wondering now, a simple macro taking seven arguments is
% $30$ to $40$ times faster than any of those in the argument grabbing and \kv\
% parsing part and only $1.5$ to $2.8$ times faster if the typesetting part is
% factored in. So the real choke isn't the parsing.
%
% So to summarize this, if you have a reasonable use case for expandable \kv\
% parsing macros you should go on and define them using \expkvc. If you have a
% reasonable use case for \kv\ parsing macros but defining them expandable isn't
% necessary for your use you should take advantage of the greater flexibility of
% non-expandable \kv\ setups (but if you're after maximum speed there aren't
% that many \kv\ parsers beating \expkvc). And if you are after maximum
% performance maybe ditching the \kv\ interface altogether is a good idea, but
% depending on the number of arguments your interface might get convoluted.
%
%
% \subsection{Useless Macros}
%
% Perhaps these macros aren't completely useless, but I figured from a user's
% point of view I wouldn't know what I should do with these.
%
% \begin{function}{\ekvcDate,\ekvcVersion}
%   These two macros store the version and the date of the package/generic code.
% \end{function}
%
%
% \subsection{Bugs}
%
% Of course I don't think there are any bugs (who would knowingly distribute
% buggy software as long as he isn't a multi-million dollar corporation?). But
% if you find some please let me know. For this one might find my email address
% on the first page or file an issue on Github:
% \url{https://github.com/Skillmon/tex_expkv-cs}
%
%
% \subsection{License}
%
% Copyright \textcopyright\ 2020\unless\ifnum\year=2020--\the\year\fi\
% Jonathan P. Spratte
%
% \medskip\noindent
% This work may be distributed and/or modified under the conditions of the
% \LaTeX\ Project Public License (LPPL), either version 1.3c of this license or
% (at your option) any later version. The latest version of this license is in
% the file:
%
% \url{http://www.latex-project.org/lppl.txt}
%
% \noindent
% This work is ``maintained'' (as per LPPL maintenance status) by
%
% \mbox{Jonathan P. Spratte}.
%
% \end{documentation}^^A=<<
%
% \begin{implementation}^^A>>=
%
% \clearpage
%
% \section{Implementation}^^A>>=
%^^A the LaTeX package >>=
% \subsection{The \LaTeX\ Package}
% Just like for \expkv\ we provide a small \LaTeX\ package that sets up things
% such that we behave nicely on \LaTeX\ packages and files system. It'll
% |\input| the generic code which implements the functionality.
% \gobbledocstriptag
%<*pkg>
%    \begin{macrocode}
\RequirePackage{expkv}
\def\ekvc@tmp
  {%
    \ProvidesFile{expkv-cs.tex}%
      [%
        \ekvcDate\space v\ekvcVersion\space
        define expandable key=val macros using expkv%
      ]%
  }
\input{expkv-cs.tex}
\ProvidesPackage{expkv-cs}%
  [%
    \ekvcDate\space v\ekvcVersion\space
    define expandable key=val macros using expkv%
  ]
%    \end{macrocode}
% \gobbledocstriptag
%</pkg>
%^^A=<<
%^^A main file >>=
% \subsection{The Generic Code}
% The rest of this implementation will be the generic code.
% \gobbledocstriptag
%<*tex>
%
% Load \expkv\ if the package didn't already do so -- since \expkv\ has
% safeguards against being loaded twice this does no harm and the overhead
% isn't that big. Also we reuse some of the internals of \expkv\ to save us from
% retyping them.
%    \begin{macrocode}
\input expkv
%    \end{macrocode}
%
% We make sure that \file{expkv-cs.tex} is only input once:
%    \begin{macrocode}
\expandafter\ifx\csname ekvcVersion\endcsname\relax
\else
  \expandafter\endinput
\fi
%    \end{macrocode}
%
% \begin{macro}{\ekvcVersion,\ekvcDate}
% We're on our first input, so lets store the version and date in a macro.
%    \begin{macrocode}
\def\ekvcVersion{0.10}
\def\ekvcDate{2021-06-03}
%    \end{macrocode}
% \end{macro}
%
% If the \LaTeX\ format is loaded we want to be a good file and report back who
% we are, for this the package will have defined |\ekvc@tmp| to use
% |\ProvidesFile|, else this will expand to a |\relax| and do no harm.
%    \begin{macrocode}
\csname ekvc@tmp\endcsname
%    \end{macrocode}
%
% Store the category code of |@| to later be able to reset it and change it to
% 11 for now.
%    \begin{macrocode}
\expandafter\chardef\csname ekvc@tmp\endcsname=\catcode`\@
\catcode`\@=11
%    \end{macrocode}
% |\ekvc@tmp| will be reused later, but we don't need it to ever store
% information long-term after \expkvc\ was initialized.
%
% \begin{macro}[internal]{\ekvc@keycount}
%   We'll need to keep count how many keys must be defined for each macro in the
%   |split| variants.
%    \begin{macrocode}
\newcount\ekvc@keycount
%    \end{macrocode}
% \end{macro}
%
% \begin{macro}[internal]{\ekvc@long,\ekvc@any@long}
%   Some macros will have to be defined long. These two will be let to |\long|
%   when this should be the case.
%    \begin{macrocode}
\let\ekvc@long\ekv@empty
\let\ekvc@any@long\ekv@empty
%    \end{macrocode}
% \end{macro}
%
% \begin{macro}[internal]{\ekvc@ifdefined}
%   We want to test whether a macro is already defined. This test checks for a
%   defined macro that isn't |\relax|.
%    \begin{macrocode}
\long\def\ekvc@ifdefined#1%
  {%
    \ifdefined#1%
      \ifx\relax#1%
        \ekv@fi@gobble
      \fi
      \@firstofone
      \ekv@fi@firstoftwo
    \fi
    \@secondoftwo
  }
%    \end{macrocode}
% \end{macro}
%
% \begin{macro}[internal]
%   {
%     \ekvc@ekvset@pre@expander,
%     \ekvc@ekvset@pre@expander@a,
%     \ekvc@ekvset@pre@expander@b
%   }
%   This macro expands |\ekvset| twice so that the first two steps of expansion
%   don't have to be made every time the \expkvc\ macros are used. We have to do
%   a little magic trick to get the macro parameter |#1| for the macro
%   definition this is used in, even though we're calling |\unexpanded|. We do
%   that by splitting the expanded |\ekvset| at some marks and place |##1| in
%   between. At this spot we also add |\ekv@alignsafe| and |\ekv@endalignsafe|
%   to ensure that macros created with \expkvc\ are alignment safe.
%    \begin{macrocode}
\def\ekvc@ekvset@pre@expander#1%
  {%
    \expandafter\ekvc@ekvset@pre@expander@a\ekvset{#1}\ekvc@stop\ekvc@stop
  }
\def\ekvc@ekvset@pre@expander@a
  {%
    \expandafter\ekvc@ekvset@pre@expander@b
  }
\def\ekvc@ekvset@pre@expander@b#1\ekvc@stop#2\ekvc@stop
  {%
    \unexpanded\expandafter{\ekv@alignsafe}%
    \unexpanded{#1}##1\unexpanded{#2}%
    \unexpanded\expandafter{\ekv@endalignsafe}%
  }
%    \end{macrocode}
% \end{macro}
%
% \begin{macro}{\ekvcSplitAndUse}
%   The first user macro we want to set up can be reused for
%   \cs[no-index]{ekvcSplitAndForward} and \cs[no-index]{ekvcSplit}. We'll split
%   this one up so that the test whether the macro is already defined doesn't
%   run twice.
%    \begin{macrocode}
\protected\long\def\ekvcSplitAndUse#1#2%
  {%
    \let\ekvc@helpers@needed\@firstoftwo
    \ekvc@ifdefined#1%
      {\ekvc@err@already@defined#1}%
      {\ekvcSplitAndUse@#1{}{#2}}%
  }
%    \end{macrocode}
% \end{macro}
%
% \begin{macro}[internal]{\ekvcSplitAndUse@}
%   The actual macro setting up things. We need to set some variables, forward
%   the key list to |\ekvc@SetupSplitKeys|, and afterwards define the front
%   facing macro to call |\ekvset| and put the initials and the argument sorting
%   macro behind it. The internals |\ekvc@any@long|, |\ekvc@initials| and
%   |\ekvc@keycount| will be set correctly by |\ekvc@SetupSplitKeys|.
%    \begin{macrocode}
\protected\long\def\ekvcSplitAndUse@#1#2#3%
  {%
    \edef\ekvc@set{\string#1}%
    \ekvc@SetupSplitKeys{#3}%
    \ekvc@helpers@needed
      {%
        \ekvc@any@long\edef#1##1%
          {%
            \expandafter\ekvc@ekvset@pre@expander\expandafter{\ekvc@set}%
            \unexpanded\expandafter
              {\csname ekvc@split@\the\ekvc@keycount\endcsname}%
            \unexpanded\expandafter{\ekvc@initials{}#2}%
          }%
      }%
      {%
        \ekvc@any@long\edef#1##1%
          {%
            \expandafter\ekvc@ekvset@pre@expander\expandafter{\ekvc@set}%
            \unexpanded{#2}%
            \unexpanded\expandafter{\ekvc@initials}%
          }%
      }%
  }
%    \end{macrocode}
% \end{macro}
%
%
% \begin{macro}{\ekvcSplitAndForward}
%   This just reuses |\ekvcSplitAndUse@| with a non-empty second argument,
%   resulting in that argument to be called after the splitting.
%    \begin{macrocode}
\protected\long\def\ekvcSplitAndForward#1#2#3%
  {%
    \let\ekvc@helpers@needed\@firstoftwo
    \ekvc@ifdefined#1%
      {\ekvc@err@already@defined#1}%
      {\ekvcSplitAndUse@#1{{#2}}{#3}}%
  }
%    \end{macrocode}
% \end{macro}
%
% \begin{macro}{\ekvcSplit}
%   The first half is just |\ekvcSplitAndForward| then we define the macro to
%   which the parsed key list is forwarded. There we need to allow for up to
%   nine arguments.
%    \begin{macrocode}
\protected\long\def\ekvcSplit#1#2#3%
  {%
    \let\ekvc@helpers@needed\@secondoftwo
    \ekvc@ifdefined#1%
      {\ekvc@err@already@defined#1}%
      {%
        \expandafter
        \ekvcSplitAndUse@\expandafter#1\csname ekvc@\string#1\endcsname{#2}%
        \ifnum\ekvc@keycount<1
          \ekvc@any@long\expandafter\def\csname ekvc@\string#1\endcsname{#3}%
        \else
          \ifnum\ekvc@keycount>9
            \ekvc@err@toomany{#1}%
            \let#1\ekvc@undefined
          \else
            \ekvcSplit@build@argspec
            \ekvc@any@long\expandafter
            \def\csname ekvc@\string#1\expandafter\endcsname\ekvc@tmp{#3}%
          \fi
        \fi
      }%
  }
%    \end{macrocode}
% \end{macro}
%
% \begin{macro}[internal]{\ekvcSplit@build@argspec,\ekvcSplit@build@argspec@}
%    \begin{macrocode}
\protected\def\ekvcSplit@build@argspec
  {%
    \begingroup
      \edef\ekvc@tmp
        {\endgroup\def\unexpanded{\ekvc@tmp}{\ekvcSplit@build@argspec@{1}}}%
    \ekvc@tmp
  }
\def\ekvcSplit@build@argspec@#1%
  {%
    \ifnum#1>\ekvc@keycount
      \ekv@fi@gobble
    \fi
    \@firstofone
      {%
        \unexpanded\expandafter{\csname ekvc@splitmark@#1\endcsname####}#1%
        \expandafter\ekvcSplit@build@argspec@\expandafter{\the\numexpr#1+1}%
      }%
  }
%    \end{macrocode}
% \end{macro}
%
% \begin{macro}[internal]
%   {
%     \ekvc@SetupSplitKeys, \ekvc@SetupSplitKeys@a, \ekvc@SetupSplitKeys@b,
%     \ekvc@SetupSplitKeys@c
%   }
%   These macros parse the list of keys and set up the key macros. First we need
%   to initialise some macros and start |\ekvparse|.
%    \begin{macrocode}
\protected\long\def\ekvc@SetupSplitKeys
  {%
    \ekvc@keycount=0
    \let\ekvc@any@long\ekv@empty
    \let\ekvc@initials\ekv@empty
    \ekvparse\ekvc@err@value@required\ekvc@SetupSplitKeys@a
  }
%    \end{macrocode}
%   Then we need to step the key counter for each key. Also we have to check
%   whether this key has a |long| prefix so we initialise |\ekvc@long|.
%    \begin{macrocode}
\protected\def\ekvc@SetupSplitKeys@a#1%
  {%
    \advance\ekvc@keycount1
    \let\ekvc@long\ekv@empty
    \ekvc@ifspace{#1}%
      {\ekvc@SetupSplitKeys@b#1\ekvc@stop}%
      {\ekvc@SetupSplitKeys@c{#1}}%
  }
%    \end{macrocode}
%   If there was a space, there might be a prefix. If so call the prefix macro,
%   else call the next step |\ekvc@SetupSplitKeys@c| which will define the key
%   macro and add the key's value to the initials list.
%    \begin{macrocode}
\protected\def\ekvc@SetupSplitKeys@b#1 #2\ekvc@stop
  {%
    \ekv@ifdefined{ekvc@split@p@#1}%
      {\csname ekvc@split@p@#1\endcsname{#2}}%
      {\ekvc@SetupSplitKeys@c{#1 #2}}%
  }
%    \end{macrocode}
%   The inner definition is grouped, because we don't want to actually define
%   the marks we build with |\csname|. We have to append the value to the
%   |\ekvc@initials| list here with the correct split mark. The key macro will
%   read everything up to those split marks and change the value following it to
%   the value given to the key. Additionally we'll need a sorting macro for each
%   key count in use so we set it up with |\ekvc@setup@splitmacro|.
%    \begin{macrocode}
\protected\long\def\ekvc@SetupSplitKeys@c#1#2%
  {%
    \begingroup
    \edef\ekvc@tmp
      {%
        \endgroup
        \long\def\unexpanded{\ekvc@tmp}####1####2%
            \unexpanded\expandafter
            {\csname ekvc@splitmark@\the\ekvc@keycount\endcsname}####3%
          {%
            ####2%
            \unexpanded\expandafter
              {\csname ekvc@splitmark@\the\ekvc@keycount\endcsname}{####1}%
          }%
%    \end{macrocode}
%   The short variant needs a bit of special treatment. The key macro will be
%   short to throw the correct error, but since there might be long macros
%   somewhere the reordering of arguments needs to be long, so for short keys we
%   use a two step approach, first grabbing only the short argument, then
%   reordering.
%    \begin{macrocode}
        \unless\ifx\ekvc@long\long
          \let\unexpanded\expandafter
            {\csname ekvc@\ekvc@set(#1)\endcsname\ekvc@tmp}%
          \def\unexpanded{\ekvc@tmp}####1%
            {%
              \unexpanded\expandafter{\csname ekvc@\ekvc@set(#1)\endcsname}%
                {####1}%
            }%
        \fi
        \def\unexpanded{\ekvc@initials}%
          {%
            \unexpanded\expandafter{\ekvc@initials}%
            \unexpanded\expandafter
              {\csname ekvc@splitmark@\the\ekvc@keycount\endcsname{#2}}%
          }%
      }%
    \ekvc@tmp
    \ekvlet\ekvc@set{#1}\ekvc@tmp
    \ekvc@helpers@needed
      {\expandafter\ekvc@setup@splitmacro\expandafter{\the\ekvc@keycount}}%
      {}%
  }
%    \end{macrocode}
% \end{macro}
%
% \begin{macro}[internal]{\ekvc@split@p@long}
%   The |long| prefix lets the internals |\ekvc@long| and |\ekvc@any@long| to
%   |\long| so that the key macro will be long.
%    \begin{macrocode}
\protected\def\ekvc@split@p@long
  {%
    \let\ekvc@long\long
    \let\ekvc@any@long\long
    \ekvc@SetupSplitKeys@c
  }
%    \end{macrocode}
% \end{macro}
%
% \begin{macro}[internal]{\ekvc@defarggobbler}
%   This is needed to define a macro with 1-9 parameters programmatically.
%   \LaTeX's \cs[no-index]{newcommand} does something similar for example.
%    \begin{macrocode}
\protected\def\ekvc@defarggobbler#1{\def\ekvc@tmp##1#1##2##{##1#1}}
%    \end{macrocode}
% \end{macro}
%
% \begin{macro}[internal]
%   {
%     \ekvc@setup@splitmacro,
%     \ekvc@split@1, \ekvc@split@2, \ekvc@split@3,
%     \ekvc@split@4, \ekvc@split@5, \ekvc@split@6,
%     \ekvc@split@7
%   }
%   Since the first few split macros are different from the others we manually
%   set those up now. All the others will be defined as needed (always
%   globally).  The split macros just read up until the correct split mark, move
%   that argument into a list and reinsert the rest, calling the next split
%   macro afterwards.
%    \begin{macrocode}
\begingroup
\edef\ekvc@tmp
  {%
    \long\gdef\unexpanded\expandafter{\csname ekvc@split@1\endcsname}%
        \unexpanded\expandafter{\csname ekvc@splitmark@1\endcsname}%
        ##1##2##3%
      {##3{##1}##2}%
    \long\gdef\unexpanded\expandafter{\csname ekvc@split@2\endcsname}%
        \unexpanded\expandafter{\csname ekvc@splitmark@1\endcsname}##1%
        \unexpanded\expandafter{\csname ekvc@splitmark@2\endcsname}##2%
        ##3##4%
      {##4{##1}{##2}##3}%
    \long\gdef\unexpanded\expandafter{\csname ekvc@split@3\endcsname}%
        \unexpanded\expandafter{\csname ekvc@splitmark@1\endcsname}##1%
        \unexpanded\expandafter{\csname ekvc@splitmark@2\endcsname}##2%
        \unexpanded\expandafter{\csname ekvc@splitmark@3\endcsname}##3%
        ##4##5%
      {##5{##1}{##2}{##3}##4}%
    \long\gdef\unexpanded\expandafter{\csname ekvc@split@4\endcsname}%
        \unexpanded\expandafter{\csname ekvc@splitmark@1\endcsname}##1%
        \unexpanded\expandafter{\csname ekvc@splitmark@2\endcsname}##2%
        \unexpanded\expandafter{\csname ekvc@splitmark@3\endcsname}##3%
        \unexpanded\expandafter{\csname ekvc@splitmark@4\endcsname}##4%
        ##5##6%
      {##6{##1}{##2}{##3}{##4}##5}%
    \long\gdef\unexpanded\expandafter{\csname ekvc@split@5\endcsname}%
        \unexpanded\expandafter{\csname ekvc@splitmark@1\endcsname}##1%
        \unexpanded\expandafter{\csname ekvc@splitmark@2\endcsname}##2%
        \unexpanded\expandafter{\csname ekvc@splitmark@3\endcsname}##3%
        \unexpanded\expandafter{\csname ekvc@splitmark@4\endcsname}##4%
        \unexpanded\expandafter{\csname ekvc@splitmark@5\endcsname}##5%
        ##6##7%
      {##7{##1}{##2}{##3}{##4}{##5}##6}%
    \long\gdef\unexpanded\expandafter{\csname ekvc@split@6\endcsname}%
        \unexpanded\expandafter{\csname ekvc@splitmark@1\endcsname}##1%
        \unexpanded\expandafter{\csname ekvc@splitmark@2\endcsname}##2%
        \unexpanded\expandafter{\csname ekvc@splitmark@3\endcsname}##3%
        \unexpanded\expandafter{\csname ekvc@splitmark@4\endcsname}##4%
        \unexpanded\expandafter{\csname ekvc@splitmark@5\endcsname}##5%
        \unexpanded\expandafter{\csname ekvc@splitmark@6\endcsname}##6%
        ##7##8%
      {##8{##1}{##2}{##3}{##4}{##5}{##6}##7}%
    \long\gdef\unexpanded\expandafter{\csname ekvc@split@7\endcsname}%
        \unexpanded\expandafter{\csname ekvc@splitmark@1\endcsname}##1%
        \unexpanded\expandafter{\csname ekvc@splitmark@2\endcsname}##2%
        \unexpanded\expandafter{\csname ekvc@splitmark@3\endcsname}##3%
        \unexpanded\expandafter{\csname ekvc@splitmark@4\endcsname}##4%
        \unexpanded\expandafter{\csname ekvc@splitmark@5\endcsname}##5%
        \unexpanded\expandafter{\csname ekvc@splitmark@6\endcsname}##6%
        \unexpanded\expandafter{\csname ekvc@splitmark@7\endcsname}##7%
        ##8##9%
      {##9{##1}{##2}{##3}{##4}{##5}{##6}{##7}##8}%
  }
\ekvc@tmp
\endgroup
\protected\def\ekvc@setup@splitmacro#1%
  {%
    \ekv@ifdefined{ekvc@split@#1}{}%
      {%
        \begingroup
          \edef\ekvc@tmp
            {%
              \long\gdef
                  \unexpanded\expandafter{\csname ekvc@split@#1\endcsname}%
                  ####1%
                  \unexpanded\expandafter{\csname ekvc@splitmark@#1\endcsname}%
                  ####2####3%
                {%
                  \unexpanded\expandafter
                    {\csname ekvc@split@\the\numexpr#1-1\relax\endcsname}%
                  ####1{{####2}####3}%
                }%
            }%
          \ekvc@tmp
        \endgroup
      }%
  }
%    \end{macrocode}
% \end{macro}
%
% \begin{macro}{\ekvcHashAndUse}
%   |\ekvcHashAndUse| works just like |\ekvcSplitAndUse|.
%    \begin{macrocode}
\protected\long\def\ekvcHashAndUse#1#2%
  {%
    \let\ekvc@helpers@needed\@firstoftwo
    \ekvc@ifdefined#1%
      {\ekvc@err@already@defined#1}%
      {\ekvcHashAndUse@#1{}{#2}}%
  }
%    \end{macrocode}
% \end{macro}
%
% \begin{macro}[internal]{\ekvcHashAndUse@}
%   This is more or less the same as |\ekvcSplitAndUse@|. Instead of an empty
%   group we place a marker after the initials, we don't use the sorting macros
%   of |split|, but instead pack all the values in one argument.
%    \begin{macrocode}
\protected\long\def\ekvcHashAndUse@#1#2#3%
  {%
    \edef\ekvc@set{\string#1}%
    \ekvc@SetupHashKeys{#3}%
    \ekvc@helpers@needed
      {%
        \ekvc@any@long\edef#1##1%
          {%
            \expandafter\ekvc@ekvset@pre@expander\expandafter{\ekvc@set}%
            \unexpanded{\ekvc@hash@pack@argument}%
            \unexpanded\expandafter{\ekvc@initials\ekvc@stop#2}%
          }%
      }%
      {%
        \ekvc@any@long\edef#1##1%
          {%
            \expandafter\ekvc@ekvset@pre@expander\expandafter{\ekvc@set}%
            \unexpanded{#2}%
            \unexpanded\expandafter{\ekvc@initials\ekvc@stop}%
          }%
      }%
  }
%    \end{macrocode}
% \end{macro}
%
% \begin{macro}{\ekvcHashAndForward}
%   |\ekvcHashAndForward| works just like |\ekvcSplitAndForward|.
%    \begin{macrocode}
\protected\long\def\ekvcHashAndForward#1#2#3%
  {%
    \let\ekvc@helpers@needed\@firstoftwo
    \ekvc@ifdefined#1%
      {\ekvc@err@already@defined#1}%
      {\ekvcHashAndUse@#1{{#2}}{#3}}%
  }
%    \end{macrocode}
% \end{macro}
%
% \begin{macro}{\ekvcHash}
%   |\ekvcHash| does the same as |\ekvcSplit|, but has the advantage of not
%   needing to count arguments, so the definition of the internal macro is a bit
%   more straight forward.
%    \begin{macrocode}
\protected\long\def\ekvcHash#1#2#3%
  {%
    \let\ekvc@helpers@needed\@secondoftwo
    \ekvc@ifdefined#1%
      {\ekvc@err@already@defined#1}%
      {%
        \expandafter
        \ekvcHashAndUse@\expandafter#1\csname ekvc@\string#1\endcsname{#2}%
        \ekvc@any@long\expandafter\def\csname ekvc@\string#1\endcsname
            ##1\ekvc@stop
          {#3}%
      }%
  }
%    \end{macrocode}
% \end{macro}
%
% \begin{macro}[internal]{\ekvc@hash@pack@argument}
%   All this macro does is pack the values into one argument and forward that to
%   the next macro.
%    \begin{macrocode}
\long\def\ekvc@hash@pack@argument#1\ekvc@stop#2{#2{#1}}
%    \end{macrocode}
% \end{macro}
%
% \begin{macro}[internal]
%   {\ekvc@SetupHashKeys,\ekvc@SetupHashKeys@a,\ekvc@SetupHashKeys@b}
%   This should look awfully familiar as well, since it's just the same as for
%   the split keys with a few other names here and there.
%    \begin{macrocode}
\protected\long\def\ekvc@SetupHashKeys#1%
  {%
    \let\ekvc@any@long\ekv@empty
    \let\ekvc@initials\ekv@empty
    \ekvparse\ekvc@err@value@required\ekvc@SetupHashKeys@a{#1}%
  }
\protected\def\ekvc@SetupHashKeys@a#1%
  {%
    \let\ekvc@long\ekv@empty
    \ekvc@ifspace{#1}%
      {\ekvc@SetupHashKeys@b#1\ekvc@stop}%
      {\ekvc@SetupHashKeys@c{#1}}%
  }
\protected\def\ekvc@SetupHashKeys@b#1 #2\ekvc@stop
  {%
    \ekv@ifdefined{ekvc@hash@p@#1}%
      {\csname ekvc@hash@p@#1\endcsname{#2}}%
      {\ekvc@SetupHashKeys@c{#1 #2}}%
  }
%    \end{macrocode}
%   Yes, even the defining macro looks awfully familiar. Instead of numbered we
%   have named marks. Still the key macros grab everything up to their
%   respective mark and reorder the arguments. The same quirk is applied for
%   short keys. And instead of the |\ekvc@setup@splitmacro| we use
%   |\ekvc@setup@hashmacro|.
%    \begin{macrocode}
\protected\long\def\ekvc@SetupHashKeys@c#1#2%
  {%
    \begingroup
    \edef\ekvc@tmp
      {%
        \endgroup
        \long\def\unexpanded{\ekvc@tmp}####1####2%
            \unexpanded\expandafter{\csname ekvc@hashmark@#1\endcsname}####3%
          {%
            ####2%
            \unexpanded\expandafter{\csname ekvc@hashmark@#1\endcsname}{####1}%
          }%
        \unless\ifx\ekvc@long\long
          \let\unexpanded\expandafter
            {\csname ekvc@\ekvc@set(#1)\endcsname\ekvc@tmp}%
          \def\unexpanded{\ekvc@tmp}####1%
            {%
              \unexpanded\expandafter{\csname ekvc@\ekvc@set(#1)\endcsname}%
                {####1}%
            }%
        \fi
        \def\unexpanded{\ekvc@initials}%
          {%
            \unexpanded\expandafter{\ekvc@initials}%
            \unexpanded\expandafter{\csname ekvc@hashmark@#1\endcsname{#2}}%
          }%
      }%
    \ekvc@tmp
    \ekvlet\ekvc@set{#1}\ekvc@tmp
    \ekvc@setup@hashmacro{#1}%
  }
%    \end{macrocode}
% \end{macro}
%
% \begin{macro}[internal]{\ekvc@hash@p@long}
%   Nothing astonishing here either.
%    \begin{macrocode}
\protected\def\ekvc@hash@p@long
  {%
    \let\ekvc@long\long
    \let\ekvc@any@long\long
    \ekvc@SetupHashKeys@c
  }
%    \end{macrocode}
% \end{macro}
%
% \begin{macro}[internal]{\ekvc@setup@hashmacro}
%   The safe hash macros will be executed inside of an |\unexpanded| expansion
%   context, so they have to insert braces for that once they are done.
%   Most of the tests which have to be executed will already be done, but we
%   have to play safe if the hash doesn't show up in the hash list. Therefore we
%   use some |\ekvc@mark|s and |\ekvc@stop| to throw errors if the hash isn't
%   found in the right place. The fast variants have an easier life and just
%   return the correct value.
%    \begin{macrocode}
\protected\def\ekvc@setup@hashmacro#1%
  {%
    \ekv@ifdefined{ekvc@fasthash@#1}{}%
      {%
        \begingroup
          \edef\ekvc@tmp
            {%
              \long\gdef
                \unexpanded\expandafter{\csname ekvc@fasthash@#1\endcsname}%
                  ####1%
                  \unexpanded\expandafter{\csname ekvc@hashmark@#1\endcsname}%
                  ####2####3\unexpanded{\ekvc@stop}%
                {####2}%
              \long\gdef
                \unexpanded\expandafter{\csname ekvc@safehash@#1\endcsname}%
                  ####1%
                {%
                  \unexpanded\expandafter{\csname ekvc@@safehash@#1\endcsname}%
                  ####1\unexpanded{\ekvc@mark}{}%
                  \unexpanded\expandafter
                    {%
                      \csname ekvc@hashmark@#1\endcsname{}%
                      \ekvc@mark{\ekvc@err@missing@hash{#1}}\ekvc@stop
                    }%
                }%
              \long\gdef
                \unexpanded\expandafter{\csname ekvc@@safehash@#1\endcsname}%
                  ####1%
                  \unexpanded\expandafter{\csname ekvc@hashmark@#1\endcsname}%
                  ####2####3\unexpanded{\ekvc@mark}####4####5%
                  \unexpanded{\ekvc@stop}%
                {%
                  ####4{####2}%
                }%
              \long\gdef\unexpanded\expandafter
                {\csname ekvc@fastsplithash@#1\endcsname}%
                  ####1%
                  \unexpanded\expandafter{\csname ekvc@hashmark@#1\endcsname}%
                  ####2####3\unexpanded{\ekvc@stop}####4%
                {%
                  ####4{####1####3}{####2}%
                }%
              \long\gdef\unexpanded\expandafter
                {\csname ekvc@safesplithash@#1\endcsname}####1%
                {%
                  \unexpanded\expandafter
                    {\csname ekvc@@safesplithash@#1\endcsname}%
                  ####1\unexpanded{\ekvc@mark\ekvc@safe@found@hash}%
                  \unexpanded\expandafter
                    {%
                      \csname ekvc@hashmark@#1\endcsname{}%
                      \ekvc@mark{\ekvc@err@missing@hash{#1}\ekvc@safe@no@hash}%
                      \ekvc@stop
                    }%
                }%
              \long\gdef\unexpanded\expandafter
                {\csname ekvc@@safesplithash@#1\endcsname}%
                  ####1%
                  \unexpanded\expandafter{\csname ekvc@hashmark@#1\endcsname}%
                  ####2####3\unexpanded{\ekvc@mark}####4####5%
                  \unexpanded{\ekvc@stop}%
                {%
                  ####4{####2}####1####3\unexpanded{\ekvc@stop}%
                }%
            }%
          \ekvc@tmp
        \endgroup
      }%
  }
%    \end{macrocode}
% \end{macro}
%
% \begin{macro}{\ekvcValue}
%   All this does is a few consistency checks on the first argument (not empty,
%   hash macro exists) and then call that hash-grabbing macro that will also
%   test whether the hash is inside of |#2| or not.
%    \begin{macrocode}
\long\def\ekvcValue#1%
  {%
    \unexpanded
    \ekv@ifdefined{ekvc@safehash@#1}%
      {\csname ekvc@safehash@#1\endcsname}%
      {\ekvc@err@unknown@hash{#1}\@firstoftwo{{}}}%
  }
%    \end{macrocode}
% \end{macro}
%
% \begin{macro}{\ekvcValueFast}
%   To be as fast as possible, this doesn't test for anything, assuming the user
%   knows best.
%    \begin{macrocode}
\long\def\ekvcValueFast#1#2{\csname ekvc@fasthash@#1\endcsname#2\ekvc@stop}
%    \end{macrocode}
% \end{macro}
%
% \begin{macro}{\ekvcValueSplit,\ekvcValueSplit@recover}
%   This splits off a single value.
%    \begin{macrocode}
\long\def\ekvcValueSplit#1%
  {%
    \ekv@ifdefined{ekvc@safesplithash@#1}%
      {\csname ekvc@safesplithash@#1\endcsname}%
      {\ekvc@err@unknown@hash{#1}\ekvcValueSplit@recover}%
  }
\long\def\ekvcValueSplit@recover#1#2%
  {%
    #2{#1}{}%
  }
%    \end{macrocode}
% \end{macro}
%
% \begin{macro}[internal]{\ekvc@safe@found@hash, \ekvc@safe@no@hash}
%    \begin{macrocode}
\long\def\ekvc@safe@found@hash#1#2\ekvc@stop#3%
  {%
    #3{#2}{#1}%
  }
\long\def\ekvc@safe@no@hash#1#2\ekvc@mark\ekvc@safe@found@hash\ekvc@stop#3%
  {%
    #3{#2}{}%
  }
%    \end{macrocode}
% \end{macro}
%
% \begin{macro}{\ekvcValueSplitFast}
%   Again a fast approach which doesn't provide too many safety measurements.
%   This needs to build the hash function and expand it before passing the
%   results to the next control sequence. The first step only builds the control
%   sequence.
%    \begin{macrocode}
\long\def\ekvcValueSplitFast#1#2%
  {%
    \csname ekvc@fastsplithash@#1\endcsname#2\ekvc@stop
  }
%    \end{macrocode}
% \end{macro}
%
% \begin{macro}[internal]
%   {
%     \ekvc@safehash@,\ekvc@fasthash@,
%     \ekvc@safesplithash@,\ekvc@fastsplithash@
%   }
%   At least in the empty hash case we can provide a meaningful error message
%   without affecting performance by just defining the macro that would be build
%   in that case. There is of course a downside to this, the error will not be
%   thrown by |\ekvcValueFast| in three expansion steps. The safe hash variant
%   has to also stop the |\unexpanded| expansion.
%    \begin{macrocode}
\long\def\ekvc@safehash@#1{\ekvc@err@empty@hash{}}
\long\def\ekvc@fasthash@#1\ekvc@stop{\ekvc@err@empty@hash}
\long\def\ekvc@safesplithash@#1#2{\ekvc@err@empty@hash#2{#1}{}}
\long\def\ekvc@fastsplithash@#1\ekvc@stop#2{\ekvc@err@empty@hash#2{#1}{}}
%    \end{macrocode}
% \end{macro}
%
% \begin{macro}{\ekvcSecondaryKeys}
%   The secondary keys are defined pretty similar to the way the originals are,
%   but here we also introduce some key types (those have a |@t@| in their name)
%   additionally to the prefixes.
%    \begin{macrocode}
\protected\long\def\ekvcSecondaryKeys#1#2%
  {%
    \edef\ekvc@set{\string#1}%
    \ekvparse\ekvc@err@value@required\ekvcSecondaryKeys@a{#2}%
  }
\protected\def\ekvcSecondaryKeys@a#1%
  {%
    \let\ekvc@long\ekv@empty
    \ekvc@ifspace{#1}%
      {\ekvcSecondaryKeys@b#1\ekvc@stop}%
      {\ekvc@err@missing@type{#1}\@gobble}%
  }
\protected\def\ekvcSecondaryKeys@b#1 #2\ekvc@stop
  {%
    \ekv@ifdefined{ekvc@p@#1}%
      {\csname ekvc@p@#1\endcsname}%
      {%
        \ekv@ifdefined{ekvc@t@#1}%
          {\csname ekvc@t@#1\endcsname}%
          {\ekvc@err@unknown@keytype{#1}\@firstoftwo\@gobble}%
      }%
      {#2}%
  }
%    \end{macrocode}
% \end{macro}
%
% \begin{macro}{\ekvcChange}
%   This can be used to change the defaults of an \expkvc\ defined macro. It
%   checks whether there is a set with the correct name and that the macro is
%   defined. If both is true the real work is done by |\ekvc@change|.
%    \begin{macrocode}
\protected\long\def\ekvcChange#1%
  {%
    \ekvifdefinedset{\string#1}%
      {%
        \ekvc@ifdefined#1%
          {\ekvc@change#1}%
          {\ekvc@err@no@key@macro#1\@gobble}%
      }%
      {\ekvc@err@no@key@macro#1\@gobble}%
  }
%    \end{macrocode}
% \end{macro}
%
% \begin{macro}[internal]
%   {\ekvc@change,\ekvc@change@a,\ekvc@change@b,\ekvc@change@c}
%   First we need to see whether the macro is currently |\long|. For this we get
%   the meaning and will parse it. |#1| is the macro name in which we want to
%   change the defaults.
%    \begin{macrocode}
\protected\def\ekvc@change#1%
  {\expandafter\ekvc@change@a\meaning#1\ekv@stop#1}
%    \end{macrocode}
%   A temporary definition to get the stringified |macro:|. |##1| will be the
%   list of prefixes, we don't care for the exact contents of |##2| and |##3|.
%    \begin{macrocode}
\def\ekvc@change@a#1%
  {%
    \protected\def\ekvc@change@a##1#1##2->##3\ekv@stop
      {%
        \ekvc@change@iflong{##1}%
          {\ekvc@change@b{}}%
          {\ekvc@change@b{\long}}%
      }%
  }
\expandafter\ekvc@change@a\expandafter{\detokenize{macro:}}
%    \end{macrocode}
%   Next we expand the macro once to get its contents (including the current
%   default values with their markers). |#1| is either |\long| or empty, |#2| is
%   the macro.
%    \begin{macrocode}
\protected\def\ekvc@change@b#1#2%
  {\expandafter\ekvc@change@c\expandafter{#2{##1}}{#1}#2}
%    \end{macrocode}
%   Here we place an unbalanced closing brace after the expansion of the macro.
%   Then we just parse the \kv-list with |\ekvset|, that will exchange the
%   values behind the markers. Once those are changed we give control to
%   |\ekvc@change@d|. The |\ekvset| step might horribly fail if the user defined
%   some keys that don't behave nice. |#1| is the expansion of the macro, |#2|
%   is either |\long| or empty, |#3| is the macro, and |#4| is the \kv-list
%   containing the new defaults.
%    \begin{macrocode}
\ekv@exparg{\protected\long\def\ekvc@change@c#1#2#3#4}%
  {%
    \expandafter\iffalse\expandafter{\expandafter\fi
      \ekvset{\string#3}{#4}%
      \ekvc@change@d{#2}{#3}%
      #1%
    }%
  }
%    \end{macrocode}
%   The final step needs to put an unbalanced opening brace after |\def|. We do
%   that with the help of a temporary macro which stores everything necessary
%   for |\def| and expand an |\iffalse}\fi| construct to leave a single
%   opening brace. |#1| will be either empty or |\long| and |#2| is the macro.
%   Each of the macros defined with \expkvc\ takes exactly one parameter, so we
%   put that here as |####1|.
%    \begin{macrocode}
\protected\def\ekvc@change@d#1#2%
  {%
    \def\ekvc@tmp{#1\def#2####1}%
    \expandafter\ekvc@tmp\expandafter{\iffalse}\fi
  }
%    \end{macrocode}
% \end{macro}
%
% \begin{macro}[internal]{\ekvc@change@iflong,\ekvc@change@iflong@}
%   Checking whether a string contains the string representation of |\long| can
%   be done by gobbling everything up to the first |\long| and checking whether
%   the result is completely empty. We need a temporary macro to get the result
%   of |\string\long| inside the definitions.
%    \begin{macrocode}
\def\ekvc@change@iflong#1%
  {%
    \protected\def\ekvc@change@iflong##1%
      {\expandafter\ekv@ifempty\expandafter{\ekvc@change@iflong@##1#1}}%
    \def\ekvc@change@iflong@##1#1{}
  }
\expandafter\ekvc@change@iflong\expandafter{\string\long}
%    \end{macrocode}
% \end{macro}
%
% \begin{macro}{\ekvcPass}
%   This macro can be used to pass a value to a key of some macro (this way
%   more complicated key codes are possible that in the end pass processed
%   values on to some macro). The implemantation is pretty straight forward.
%    \begin{macrocode}
\long\def\ekvcPass#1#2%
  {%
    \ekvifdefined{\string#1}{#2}%
      {\csname\ekv@name{\string#1}{#2}\endcsname}%
      {\ekvc@err@unknown@key@or@macro{#1}{#2}\@gobble}%
  }
%    \end{macrocode}
% \end{macro}
%
%
% \subsubsection{Secondary Key Types}
%
% \begin{macro}[internal]{\ekvc@p@long,\ekvc@after@ptype}
%   The prefixes are pretty straight forward again. Just set |\ekvc@long| and
%   forward to the |@t@| type.
%    \begin{macrocode}
\protected\def\ekvc@p@long#1%
  {%
    \ekvc@ifspace{#1}%
      {%
        \let\ekvc@long\long
        \ekvc@after@ptype#1\ekvc@stop
      }%
      {\ekvc@err@missing@type{long #1}\@gobble}%
  }
\protected\def\ekvc@after@ptype#1 #2\ekvc@stop
  {%
    \ekv@ifdefined{ekvc@t@#1}%
      {\csname ekvc@t@#1\endcsname{#2}}%
      {\ekvc@err@unknown@keytype{#1}\@gobble}%
  }
%    \end{macrocode}
% \end{macro}
%
% \begin{macro}[internal]
%   {
%     \ekvc@t@meta, \ekvc@t@nmeta,
%     \ekvc@type@meta, \ekvc@type@meta@a, \ekvc@type@meta@b
%   }
%   The |meta| and |nmeta| key types use a nested |\ekvset| to set other keys in
%   the same macro's \set.
%    \begin{macrocode}
\protected\def\ekvc@t@meta
  {%
    \edef\ekvc@tmp{\ekvc@set}%
    \expandafter\ekvc@type@meta\expandafter{\ekvc@tmp}\ekvc@long{##1}\ekvlet
  }
\protected\def\ekvc@t@nmeta#1%
  {%
    \ekvc@assert@not@long{nmeta #1}%
    \edef\ekvc@tmp{\ekvc@set}%
    \expandafter\ekvc@type@meta\expandafter{\ekvc@tmp}{}{}\ekvletNoVal{#1}%
  }
\protected\long\def\ekvc@type@meta#1#2#3#4#5#6%
  {%
    \expandafter\ekvc@type@meta@a\expandafter{\ekvset{#1}{#6}}{#2}{#3}%
    #4\ekvc@set{#5}\ekvc@tmp
  }
\protected\def\ekvc@type@meta@a
  {%
    \expandafter\ekvc@type@meta@b\expandafter
  }
\protected\long\def\ekvc@type@meta@b#1#2#3%
  {%
    #2\def\ekvc@tmp#3{#1}%
  }
%    \end{macrocode}
% \end{macro}
%
% \begin{macro}[internal]{\ekvc@t@alias}
%   |alias| just checks whether there is a key and/or |NoVal| key defined with
%   the target name and |\let| the key to those.
%    \begin{macrocode}
\protected\def\ekvc@t@alias#1#2%
  {%
    \ekvc@assert@not@long{alias #1}%
    \let\ekvc@tmp\@firstofone
    \ekvifdefined\ekvc@set{#2}%
      {%
        \ekvletkv\ekvc@set{#1}\ekvc@set{#2}%
        \let\ekvc@tmp\@gobble
      }%
      {}%
    \ekvifdefinedNoVal\ekvc@set{#2}%
      {%
        \ekvletkvNoVal\ekvc@set{#1}\ekvc@set{#2}%
        \let\ekvc@tmp\@gobble
      }%
      {}%
    \ekvc@tmp{\ekvc@err@unknown@key{#2}}%
  }
%    \end{macrocode}
% \end{macro}
%
% \begin{macro}[internal]{\ekvc@t@default}
%   The |default| key can be used to set a |NoVal| key for an existing key. It
%   will just pass the \val\ to the key macro of that other key.
%    \begin{macrocode}
\protected\long\def\ekvc@t@default#1#2%
  {%
    \ekvifdefined\ekvc@set{#1}%
      {%
        \ekvc@assert@not@long{default #1}%
        \edef\ekvc@tmp
          {%
            \unexpanded\expandafter
              {\csname\ekv@name\ekvc@set{#1}\endcsname{#2}}%
          }%
        \ekvletNoVal\ekvc@set{#1}\ekvc@tmp
      }%
      {\ekvc@err@unknown@key{#1}}%
  }
%    \end{macrocode}
% \end{macro}
%
% \begin{macro}[internal]{\ekvc@t@flag-bool}
%    \begin{macrocode}
\protected\expandafter\def\csname ekvc@t@flag-bool\endcsname#1#2%
  {%
    \ekvc@assert@not@long{flag-bool #1}%
    \unless\ifdefined#2\ekvcFlagNew#2\fi
    \ekvdef\ekvc@set{#1}%
      {%
        \ekv@ifdefined{ekvc@flag@set@##1}%
          {%
            \csname ekvc@flag@set@##1\expandafter\endcsname
              \ekvcFlagHeight#2\ekv@stop#2%
          }%
          {\ekvc@err@invalid@bool{##1}}%
      }%
  }
%    \end{macrocode}
% \end{macro}
%
% \begin{macro}[internal]
%   {\ekvc@t@flag-true,\ekvc@t@flag-false,\ekvc@t@flag-raise,\ekvc@type@flag}
%    \begin{macrocode}
\protected\def\ekvc@type@flag#1#2#3#4%
  {%
    \ekvc@assert@not@long{flag-#1 #3}%
    \unless\ifdefined#4\ekvcFlagNew#4\fi
    \ekv@exparg{\ekvdefNoVal\ekvc@set{#3}}{#2#4}%
  }
\protected\expandafter\def\csname ekvc@t@flag-true\endcsname
  {\ekvc@type@flag{true}\ekvcFlagSetTrue}
\protected\expandafter\def\csname ekvc@t@flag-false\endcsname
  {\ekvc@type@flag{false}\ekvcFlagSetFalse}
\protected\expandafter\def\csname ekvc@t@flag-raise\endcsname
  {\ekvc@type@flag{raise}\ekvcFlagRaise}
%    \end{macrocode}
% \end{macro}
%
%
% \subsubsection{Flags}
%
% The basic idea of flags is to store information by the fact that \TeX\
% expandably assigns the meaning |\relax| to undefined control sequences which
% were built with |\csname|. This mechanism is borrowed from \pkg{expl3}.
%
% \begin{macro}[internal]{\ekvc@flag@name,\ekvc@flag@namescheme}
%   Flags follow a simple naming scheme which we define here. |\ekvc@flag@name|
%   will store the name of an internal function that is used to build names of
%   the second naming scheme defined by |\ekvc@flag@namescheme|.
%    \begin{macrocode}
\def\ekvc@flag@name{ekvcf\string}
\def\ekvc@flag@namescheme#1#2{ekvch#2#1}
%    \end{macrocode}
% \end{macro}
%
% \begin{macro}{\ekvcFlagHeight}
%   For semantic reasons we use |\number| with another name.
%    \begin{macrocode}
\let\ekvcFlagHeight\number
%    \end{macrocode}
% \end{macro}
%
% \begin{macro}{\ekvcFlagNew}
%   This macro defines a new flag. It stores the function build with the
%   |\ekvc@flag@name| naming scheme after the internal function
%   |\ekvc@flag@height| that'll determine the current flag height. It'll also
%   define the macro named via |\ekvc@flag@name| to build names according to
%   |\ekvc@flag@namescheme|.
%    \begin{macrocode}
\protected\def\ekvcFlagNew#1%
  {%
    \edef#1%
      {%
        \unexpanded{\ekvc@flag@height}%
          \unexpanded\expandafter{\csname\ekvc@flag@name#1\endcsname}%
      }%
    \ekv@expargtwice
      {\expandafter\def\csname\ekvc@flag@name#1\endcsname##1}%
      {\expandafter\ekvc@flag@namescheme\expandafter{\string#1}{##1}}%
  }
%    \end{macrocode}
% \end{macro}
%
% \begin{macro}[internal]{\ekvc@flag@height,\ekvc@flag@height@}
%   This macro gets the height of a flag by a simple loop. The first loop
%   iteration differs a bit from the following in that it doesn't have to get
%   the current iteration count. The space at the end of |\ekvc@flag@height|
%   ends the |\number| evaluation.
%    \begin{macrocode}
\def\ekvc@flag@height#1%
  {%
    \ifcsname#10\endcsname
      \ekvc@flag@height@1\ekv@stop#1%
    \fi
    \@firstofone{0} % leave this space
  }
\def\ekvc@flag@height@#1\ekv@stop#2\fi\@firstofone#3%
  {%
    \fi
    \ifcsname#2{#1}\endcsname
      \expandafter\ekvc@flag@height@\the\numexpr#1+1\relax\ekv@stop#2%
    \fi
    \@firstofone{#1}%
  }
%    \end{macrocode}
% \end{macro}
%
% \begin{macro}{\ekvcFlagRaise}
%   Raising a flag simply means letting the |\ekvc@flag@namescheme| macro for
%   the current height to relax. The result of raising a flag is that its height
%   is bigger by $1$.
%    \begin{macrocode}
\ekv@exparg{\def\ekvcFlagRaise#1}%
  {%
    \expandafter\expandafter\expandafter\@gobble\expandafter
      \csname\ekvc@flag@namescheme{\string#1}{\ekvcFlagHeight#1}\endcsname
  }
%    \end{macrocode}
% \end{macro}
%
% \begin{macro}{\ekvcFlagSetTrue,\ekvcFlagSetFalse}
% \begin{macro}[internal]{\ekvc@flag@set@true,\ekvc@flag@set@false}
%   A flag is considered true if its current height is odd, and as false if it
%   is even. Therefore |\ekvcFlagSetTrue| and |\ekvcFlagSetFalse| only need to
%   raise the flag if the opposing boolean value is the current one.
%    \begin{macrocode}
\def\ekvcFlagSetTrue#1%
  {\expandafter\ekvc@flag@set@true\ekvcFlagHeight#1\ekv@stop#1}
\def\ekvcFlagSetFalse#1%
  {\expandafter\ekvc@flag@set@false\ekvcFlagHeight#1\ekv@stop#1}
%    \end{macrocode}
%   We can expand |\ekvc@flag@namescheme| at definition time here, which is why
%   we're using a temporary definition to set up |\ekvc@flag@set@true| and
%   |\ekvc@flag@set@false|.
%    \begin{macrocode}
\def\ekvc@flag@set@true#1%
  {%
    \def\ekvc@flag@set@true##1\ekv@stop##2%
      {%
        \ifodd##1
          \ekv@fi@gobble
        \fi
        \@firstofone{\expandafter\@gobble\csname#1\endcsname}%
      }%
    \def\ekvc@flag@set@false##1\ekv@stop##2%
      {%
        \ifodd##1
          \ekv@fi@firstofone
        \fi
        \@gobble{\expandafter\@gobble\csname#1\endcsname}%
      }%
  }
\expandafter\ekvc@flag@set@true\expandafter
  {\ekvc@flag@namescheme{\string#2}{#1}}
%    \end{macrocode}
% \end{macro}
% \end{macro}
%
% \begin{macro}{\ekvcFlagIf}
%   As already explained, truthiness is defined as a flag's height is odd, so we
%   just branch accordingly here.
%    \begin{macrocode}
\def\ekvcFlagIf#1%
  {%
    \ifodd#1%
      \ekv@fi@firstoftwo
    \fi
    \@secondoftwo
  }
%    \end{macrocode}
% \end{macro}
%
% \begin{macro}{\ekvcFlagIfRaised}
%   This macro uses flags as a switch, if a flag's current height is bigger than
%   $0$ this test yields true.
%    \begin{macrocode}
\ekv@exparg{\def\ekvcFlagIfRaised#1}%
  {%
    \expandafter\ifcsname\ekvc@flag@namescheme{\string#1}0\endcsname
      \ekv@fi@firstoftwo
    \fi
    \@secondoftwo
  }
%    \end{macrocode}
% \end{macro}
%
% \begin{macro}{\ekvcFlagReset}
% \begin{macro}[internal]{\ekvc@flag@reset,\ekvc@flag@reset@}
%   Resetting works by locally letting all the defined internal macros named
%   after |\ekvc@flag@namescheme| to undefined.
%    \begin{macrocode}
\protected\def\ekvcFlagReset#1%
  {\expandafter\ekvc@flag@reset\csname\ekvc@flag@name#1\endcsname}
\protected\def\ekvc@flag@reset#1%
  {%
    \ifcsname#10\endcsname
      \expandafter\let\csname#10\endcsname\ekvc@undefined
      \ekvc@flag@reset@1\ekv@stop#1%
    \fi
  }
\protected\def\ekvc@flag@reset@#1\ekv@stop#2\fi
  {%
    \fi
    \ifcsname#2{#1}\endcsname
      \expandafter\let\csname#2{#1}\endcsname\ekvc@undefined
      \expandafter\ekvc@flag@reset@\the\numexpr#1+1\relax\ekv@stop#2%
    \fi
  }
%    \end{macrocode}
% \end{macro}
% \end{macro}
%
% \begin{macro}{\ekvcFlagGetHeight}
% \begin{macro}[internal]{\ekvc@flag@get@height@single}
%   These are just small helpers, first getting the height of the flag and then
%   passing it on to the user supplied code.
%    \begin{macrocode}
\def\ekvcFlagGetHeight#1%
  {\expandafter\ekvc@flag@get@height@single\ekvcFlagHeight#1\ekv@stop}
\long\def\ekvc@flag@get@height@single#1\ekv@stop#2{#2{#1}}
%    \end{macrocode}
% \end{macro}
% \end{macro}
%
% \begin{macro}{\ekvcFlagGetHeights}
% \begin{macro}[internal]
%   {\ekvc@flag@get@heights,\ekvc@flag@get@heights@,\ekvc@flag@get@heights@done}
%   This works by a simple loop that stops at |\ekv@stop|. As long as that
%   marker isn't hit, get the next flags height and put it into a list after
%   |\ekv@stop|. |\ekvc@flag@get@heights@| uses the same marker name for the
%   end of the height, which shouldn't clash in any case. Once we're done we
%   remove the remainder of the current iteration and leave the user supplied
%   code in the input stream with all the flags' heights as a single argument.
%    \begin{macrocode}
\def\ekvcFlagGetHeights#1%
  {%
    \ekvc@flag@get@heights#1\ekv@stop{}%
  }
\def\ekvc@flag@get@heights#1%
  {%
    \ekv@gobbleto@stop#1\ekvc@flag@get@heights@done\ekv@stop
    \expandafter\ekvc@flag@get@heights@\ekvcFlagHeight#1\ekv@stop
  }
\def\ekvc@flag@get@heights@#1\ekv@stop#2\ekv@stop#3%
  {\ekvc@flag@get@heights#2\ekv@stop{#3{#1}}}
\long\def\ekvc@flag@get@heights@done
    \ekv@stop
    \expandafter\ekvc@flag@get@heights@\ekvcFlagHeight\ekv@stop\ekv@stop#1#2%
  {#2{#1}}
%    \end{macrocode}
% \end{macro}
% \end{macro}
%
%
% \subsubsection{Helper Macros}
%
% \begin{macro}[internal]{\ekvc@ifspace,\ekvc@ifspace@}
% A test which can be reduced to an if-empty by gobbling everything up to the
% first space.
%    \begin{macrocode}
\long\def\ekvc@ifspace#1%
  {%
    \ekvc@ifspace@#1 \ekv@ifempty@B
      \ekv@ifempty@false\ekv@ifempty@A\ekv@ifempty@B\@firstoftwo
  }
\long\def\ekvc@ifspace@#1 % keep this space
  {%
    \ekv@ifempty@\ekv@ifempty@A
  }
%    \end{macrocode}
% \end{macro}
%
%
% \subsubsection{Assertions}
% \begin{macro}[internal]{\ekvc@assert@not@long}
%   Some keys don't want to be |long| and we have to educate the user, so let's
%   throw an error if someone wanted these to be long.
%    \begin{macrocode}
\long\def\ekvc@assert@not@long#1{\ifx\ekvc@long\long\ekvc@err@no@long{#1}\fi}
%    \end{macrocode}
% \end{macro}
%
%
% \subsubsection{Messages}
%
% \begin{macro}[internal]
%   {
%     \ekvc@err@toomany,\ekvc@err@value@required,\ekvc@err@missing@type,
%     \ekvc@err@already@defined,\ekvc@err@no@key@macro
%   }
%   Boring unexpandable error messages.
%    \begin{macrocode}
\protected\def\ekvc@err@toomany#1%
  {%
    \errmessage{expkv-cs Error: Too many keys for macro `\string#1'}%
  }
\protected\def\ekvc@err@value@required#1%
  {%
    \errmessage{expkv-cs Error: Missing value for key `\unexpanded{#1}'}%
  }
\protected\def\ekvc@err@missing@type#1%
  {%
    \errmessage
      {expkv-cs Error: Missing type for secondary key `\unexpanded{#1}'}%
  }
\protected\def\ekvc@err@no@long#1%
  {%
    \errmessage
      {expkv-cs Error: prefix `long' not accepted for `\unexpanded{#1}'}%
  }
\protected\def\ekvc@err@already@defined#1%
  {%
    \errmessage{expkv-cs Error: Macro `\string#1' already defined}%
  }
\protected\def\ekvc@err@unknown@keytype#1%
  {%
    \errmessage{expkv-cs Error: Unknown key type `\unexpanded{#1}'}%
  }
\protected\def\ekvc@err@unknown@key#1%
  {%
    \errmessage
      {expkv-cs Error: Unknown key `\unexpanded{#1}' for macro `\ekvc@set'}%
  }
\long\def\ekvc@err@no@key@macro#1%
  {\errmessage{expkv-cs Error: \string#1 is no key=val macro}}
%    \end{macrocode}
% \end{macro}
%
% \begin{macro}[internal]{\ekvc@err}
%   We need a way to throw error messages expandably in some contexts.
%    \begin{macrocode}
\ekv@exparg{\long\def\ekvc@err#1}{\ekverr{expkv-cs}{#1}}
%    \end{macrocode}
% \end{macro}
%
% \begin{macro}[internal]
%   {
%     \ekvc@err@unknown@hash,\ekvc@err@empty@hash,\ekvc@err@missing@hash,
%     \ekvc@err@invalid@bool
%   }
%   And here are the expandable error messages.
%    \begin{macrocode}
\long\def\ekvc@err@unknown@hash#1{\ekvc@err{unknown hash `#1'}}
\long\def\ekvc@err@missing@hash#1{\ekvc@err{hash `#1' not found}}
\long\def\ekvc@err@empty@hash{\ekvc@err{empty hash}}
\def\ekvc@err@invalid@bool#1{\ekvc@err{invalid boolean value `#1'}}
\long\def\ekvc@err@unknown@key@or@macro#1#2%
  {\ekvc@err{unknown key `#2' for macro #1}}
%    \end{macrocode}
% \end{macro}
%
%
% Now everything that's left is to reset the category code of |@|.
%    \begin{macrocode}
\catcode`\@=\ekvc@tmp
%    \end{macrocode}
%
% \gobbledocstriptag
%</tex>
%^^A=<<
%
%^^A=<<
%
% \end{implementation}^^A=<<
%
% \clearpage
% \PrintIndex
%
\endinput
%
^^A vim: ft=tex fdm=marker fmr=>>=,=<<
}{}
\ProvidesFile{expkv-cs.dtx}
  [\csname ekvcDate\endcsname\ define expandable key=val macros using expkv]
\PassOptionsToPackage{full}{textcomp}
\documentclass{l3doc}
\RequirePackage[oldstylenums,nott]{kpfonts}
\input{glyphtounicode}
\pdfgentounicode=1
\RequirePackage{tcolorbox}
\newtcolorbox{exresult}[2][]
  {colback=expkvgrey!10!white,colframe=expkvgrey,fontupper=\small,width={#2},#1}
\newtcbox\exres[1][]
  {
     colback=expkvgrey!10!white
    ,colframe=expkvgrey
    ,size=small
    ,nobeforeafter
    ,tcbox raise base
    ,fontupper=\small
    ,#1
  }
\RequirePackage{listings}
\RequirePackage{booktabs}
\RequirePackage{array}
\RequirePackage{collcell}
\RequirePackage{siunitx}
\RequirePackage{xcolor}
\RequirePackage{caption}
\RequirePackage{microtype}
\RequirePackage{accsupp}
\lstdefinelanguage{expkvc}[primitive]{TeX}
  {
    ,moretexcs=[4]^^A e-TeX
      {
        expanded,
        numexpr,
        protected,
      }
    ,moretexcs=[5]^^A plain/LaTeX
      {
        approx,
        begin,
        empty,
        item,
        LaTeX,
        makeatletter,makeatother,multicolumn,
        newcommand,newdimen,
        RequirePackage,
        rule,
        TeX,
        textit,texttt,
        usepackage,
      }
    ,moretexcs=[6]^^A used packages
      {
        ^^A xfp
        fpeval,
        ^^A keyval
        setkeys,
        ^^A yax
        defactiveparameter,storevalue,setparameterlist
      }
    ,moretexcs=[2]^^A expkv-cs macros
      {
        ekvcSplit,ekvcSplitAndForward,ekvcSplitAndUse,
        ekvcHash,ekvcHashAndForward,ekvcHashAndUse,
        ekvcValue,ekvcValueFast,ekvcValueSplit,ekvcValueSplitFast,
        ekvcSecondaryKeys,
        ekvcChange,
        ekvcFlagNew,
        ekvcFlagHeight,
        ekvcFlagRaise,
        ekvcFlagSetTrue,ekvcFlagSetFalse,
        ekvcFlagIf,ekvcFlagIfRaised,
        ekvcFlagReset,
        ekvcFlagGetHeight,ekvcFlagGetHeights,
        ekvcPass
      }
    ,moretexcs=[3]^^A expkv-pkg macros
      {
        ^^A expkv
        ekvdef,ekvdefNoVal,^^A
        ekvlet,ekvletNoVal,ekvletkv,ekvletkvNoVal,^^A
        ekvdefunknown,ekvdefunknownNoVal,^^A
        ekvredirectunknown,ekvredirectunknownNoVal,^^A
        ekvset,ekvsetSneaked,^^A
        ekvsetdef,ekvsetSneakeddef,ekvsetdefSneaked,^^A
        ekvparse,^^A
        ekvVersion,ekvDate,^^A
        ekvifdefined,ekvifdefinedNoVal,ekvifdefinedset,^^A
        ekvbreak,ekvbreakPreSneak,ekvbreakPostSneak,^^A
        ekvsneak,ekvsneakPre,^^A
        ekvchangeset,^^A
        ekvoptarg,ekvoptargTF,^^A
        ekverr,
        ^^A expkv-def
        ekvdefinekeys
      }
  }
\colorlet{codeparam}{cyan!65!black}
\def\mylstwd{.55em}
\lstset
  {
    ,language=expkvc
    ,flexiblecolumns=false
    ,basewidth=\mylstwd
    ,gobble=2
    ,basicstyle=\fontfamily{jkp}\itshape
    ,texcsstyle=*[1]{\bfseries\color{expkvgrey}} ^^A primitives
    ,texcsstyle=*[2]{\bfseries\color{expkvred}}  ^^A expkv
    ,texcsstyle=*[3]{\color{expkvred}}           ^^A expkv-pkg
    ,texcsstyle=*[4]{\bfseries\color{expkvgrey}} ^^A e-TeX
    ,texcsstyle=*[5]{\bfseries\color{expkvgrey}} ^^A plain/LaTeX macros
    ,texcsstyle=*[6]{}                           ^^A macros of other packages
    ,commentstyle=\color[gray]{0.4}
    ,literate=
              {\{} {{\CodeUpBf\{}}{1}
              {\}} {{\CodeUpBf\}}}{1}
              {$}  {{\CodeUpBf\$}}{1}
              {[}  {{\CodeUp[}}{1}
              {]}  {{\CodeUp]}}{1}
              {(}  {{\CodeUp(}}{1}
              {)}  {{\CodeUp)}}{1}
              {*}  {{$*$}}{1}
              {1}  {{\CodeUp{1}}}{1}
              {2}  {{\CodeUp{2}}}{1}
              {3}  {{\CodeUp{3}}}{1}
              {4}  {{\CodeUp{4}}}{1}
              {5}  {{\CodeUp{5}}}{1}
              {6}  {{\CodeUp{6}}}{1}
              {7}  {{\CodeUp{7}}}{1}
              {8}  {{\CodeUp{8}}}{1}
              {9}  {{\CodeUp{9}}}{1}
              {0}  {{\CodeUp{0}}}{1}
              {##} {{\CodeColored{codeparam}{1}{\#}}}{1}
              {##1}{{\CodeColored{codeparam}{2}{\#1}}}{2}
              {##2}{{\CodeColored{codeparam}{2}{\#2}}}{2}
              {##3}{{\CodeColored{codeparam}{2}{\#3}}}{2}
              {##4}{{\CodeColored{codeparam}{2}{\#4}}}{2}
              {##5}{{\CodeColored{codeparam}{2}{\#5}}}{2}
              {##6}{{\CodeColored{codeparam}{2}{\#6}}}{2}
              {##7}{{\CodeColored{codeparam}{2}{\#7}}}{2}
              {##8}{{\CodeColored{codeparam}{2}{\#8}}}{2}
              {##9}{{\CodeColored{codeparam}{2}{\#9}}}{2}
              {<key>}{{$\langle$}key{$\rangle$}}{5}
              {<set>}{{$\langle$}set{$\rangle$}}{5}
              {<set1>}{{$\langle$}set1{$\rangle$}}{6}
              {<set2>}{{$\langle$}set2{$\rangle$}}{6}
    ^^A,literate=*{<key>}{\key}{4}{<set>}{\set}{4}
  }
\newcommand*\CodeColored[3]{\textcolor{#1}{\makebox[\dimexpr\mylstwd*#2]{#3}}}
\newcommand*\CodeUpBf[1]{\makebox[\mylstwd]{\textup{\textbf{#1}}}}
\newcommand*\CodeUp[1]{\makebox[\mylstwd]{\textup{#1}}}
\RequirePackage{randtext}
\let\metaORIG\meta
\protected\def\meta #1{\texttt{\metaORIG{#1}}}
\renewcommand*\thefootnote{\fnsymbol{footnote}}
\definecolor{expkvred}{HTML}{9F393D}
\colorlet{expkvgrey}{black!75}
\makeatletter
\newcommand*\example{\par\smallskip\noindent\textit{Example:} \ignorespaces}
\newcommand*\expkv
  {^^A
    \texorpdfstring
      {^^A
        \mbox
          {^^A
            \BeginAccSupp{ActualText=expkv}^^A
            \href{https://github.com/Skillmon/tex_expkv}
              {^^A
                \rmfamily
                \bfseries
                {\color{expkvgrey}e\kern-.05em x\kern-.05em}^^A
                \lower.493ex
                  \hbox{{\color{expkvgrey}P}\kern-.1em{\color{expkvred}k}}^^A
                \kern-.18em{\color{expkvred}v}^^A
              }^^A
            \EndAccSupp{}^^A
          }^^A
      }
      {expkv}^^A
  }
\newcommand*\expkvpkg[1]
  {^^A
    \texorpdfstring
      {^^A
        \mbox
          {^^A
            \BeginAccSupp{ActualText=expkv-#1}^^A
            \href{https://github.com/Skillmon/tex_expkv-#1}
              {^^A
                \rmfamily
                \bfseries
                {\color{expkvgrey}e\kern-.05em x\kern-.05em}^^A
                \lower.493ex
                  \hbox{{\color{expkvgrey}P}\kern-.1em{\color{expkvred}k}}^^A
                \kern-.18em{\color{expkvred}v}^^A
                {\color{expkvgrey}^^A
                  \kern.05em\rule[-.1ex]{.08em}{1.2ex}\kern.05em\textsc{#1}^^A
                }^^A
              }^^A
            \EndAccSupp{}^^A
          }^^A
      }
      {expkv-#1}^^A
  }
\newcommand*\expkvd{\expkvpkg{def}}
\newcommand*\expkvc{\expkvpkg{cs}}
\newcommand\kv{\meta{key}=\meta{value}}
\newcommand\key{\meta{key}}
\newcommand\val{\meta{value}}
\newcommand\set{\meta{set}}
\newcommand\enflong{\textcolor{black}{long}}
\newcommand\alllong{\textcolor{gray}{long}}
\newcommand\notlong{\textcolor{red!80!black}{long}}
\newcommand\prefixes[1]
  {^^A
    \hfill
    \ifcase\numexpr#1\relax\or
      \enflong\or
      \alllong\or
      \notlong\fi
  }
\hypersetup{linkcolor=red!80!black,urlcolor=purple!80!black}
\DoNotIndex{\def,\edef,\,,\=,\begingroup,\catcode,\chardef,\csname,\endcsname}
\DoNotIndex{\endgroup,\endinput,\errmessage,\expandafter,\input,\let,\long,\z@}
\DoNotIndex{\protected,\ProvidesFile,\ProvidesPackage,\relax,\space,\advance}
\DoNotIndex{\@,\unexpanded,\string,\expanded,\dimexpr,\global,\glueexpr,\hbox}
\DoNotIndex{\numexpr,\RequirePackage,\setbox,\the,\unless,\xdef,\gdef,\newcount}
\DoNotIndex{\number}
\DoNotIndex{\^,\@firstofone,\@firstoftwo,\@gobble,\@secondoftwo}
\DoNotIndex{\ifcsname}
\DoNotIndex{\ifx}
\DoNotIndex{\ifodd}
\DoNotIndex{\ifnum}
\DoNotIndex{\ifdefined}
\DoNotIndex{\iffalse}
\DoNotIndex{\iftrue}
\DoNotIndex{\else}
\DoNotIndex{\fi}
\@gobble\fi ^^A ignoring \ifx, \ifcsname, etc., but only one \fi
\@gobble\fi ^^A ignoring \ifx, \ifcsname, etc., but only one \fi
\@gobble\fi ^^A ignoring \ifx, \ifcsname, etc., but only one \fi
\@gobble\fi ^^A ignoring \ifx, \ifcsname, etc., but only one \fi
\@gobble\fi ^^A ignoring \ifx, \ifcsname, etc., but only one \fi
\@gobble\fi ^^A ignoring \ifx, \ifcsname, etc., but only one \fi
\@ifdefinable\gobbledocstriptag{\def\gobbledocstriptag#1>{}}
\makeatother
\begin{document}
  \title
    {^^A
      \texorpdfstring
        {^^A
          \huge\expkvc
          \\[\medskipamount]
          \Large define expandable \kv\ macros using \expkv
        }{expkv-cs - define expandable key=val macros using expkv}^^A
    }
  \date{\ekvcDate\space v\ekvcVersion}
  \author{Jonathan P. Spratte\thanks{\protect\randomize{jspratte@yahoo.de}}}
  \DocInput{expkv-cs.dtx}
\end{document}
%</driver>^^A=<<
% \fi
%
% \maketitle
% \renewcommand*\thefootnote{\arabic{footnote}}
%
% \begin{abstract}
% \noindent\parfillskip=0pt
% \expkvc\ provides two small interfaces to define expandable \kv\ macros using
% \expkv. It therefore lowers the entrance boundary to expandable \kv\ macros.
% The stylised name is \expkvc\ but the files use \file{expkv-cs}, this is due
% to CTAN-rules which don't allow \string| in package names since that is the
% pipe symbol in *nix shells.
% \end{abstract}
%
% \tableofcontents
%
% \begin{documentation}^^A>>=
%
% \section{Documentation}
%
% The \expkv\ package enables the new possibility of creating \kv\ macros which
% are fully expandable. The creation of such macros is however cumbersome for
% the average user. \expkvc\ tries to step in here. It provides interfaces to
% define \kv\ macros without worrying too much about the implementation. In case
% you're wondering now, the |cs| in \expkvc\ stands for control sequence,
% because |def| was already taken by \expkvd\ and ``control sequence'' is the
% term D.\,E.\,Knuth used in his \TeX book for named commands hence macros
% (though he also used the term ``macro''). So \expkvc\ defines control
% sequences for and with \expkv.
%
% There are two different approaches supported by this package. The first is
% splitting the keys up into individual arguments, the second is providing all
% the keys as a single argument to the underlying macro and getting an
% individual \val\ by using a hash. Well, actually there is no real hash, just
% some markers which are parsed, but this shouldn't be apparent to the user, the
% behaviour matches that of a hash-table.
%
% In addition to these two methods of defining a macro with primary keys a way
% to define secondary keys, which can reference the primary ones, is provided.
% These secondary keys don't correspond to an argument or an entry in the hash
% table directly but might come in handy for the average use case. Each macro
% has its own set of primary and secondary keys.
%
% A word of advice you should consider: If your macro doesn't have to be
% expandable (and often it doesn't) consider not using \expkvc. The interface
% has some overhead (though it still is fast -- check \autoref{sec:speed}) and
% the approach has its limits in versatility. If you don't need to be
% expandable, you should consider either defining your keys manually using
% \expkv\ or using \expkvd\ for convenience. Or you resort to another \kv\
% interface. Nevertheless setting up macros with \expkvc, especially with
% |\ekvcSplit|, is very convenient in my opinion, so if you just want to define
% a single macro with just a few keys this might be the way to go.
%
% \expkvc\ is usable as generic code and as a \LaTeX\ package. It'll
% automatically load \expkv\ in the same mode as well. To use it, just use one
% of
% \begin{lstlisting}
% \usepackage{expkv-cs} % LaTeX
% \input expkv-cs       % plainTeX
% \end{lstlisting}
%
%
% \subsection{Define Macros and Primary Keys}
%
% All macros defined with \expkvc\ have to be previously undefined or have the
% |\meaning| of |\relax|. This is necessary as there is no way to automatically
% undefine keys once they are set up (neither \expkv\ nor \expkvc\ keep track of
% defined keys) -- so to make sure there are no conflicts only new definitions
% are allowed (that's not the case for individual keys, only for frontend
% macros).
%
%
% \subsubsection{Primary Keys}\label{sec:primaries}
%
% In the following descriptions there will be one argument named \meta{primary
% keys}. This argument should be a \kv\ list where each \key\ will be one
% primary key and \val\ the associated initial value. By default all keys are
% defined short, but you can define long keys by prefixing \key\ with |long|
% (\emph{e.g.}, \texttt{long name=Jonathan P. Spratte}). You only need |long| if
% the key should be able to take a |\par| token. Note however that |long| keys
% are a microscopic grain faster (due to some internals of \expkvc). Only if at
% least one of the keys was |long| the \meta{cs} in the following defining
% macros will be |\long|. For obvious reasons there is no possibility to define
% a macro or key as |\protected|.
%
% At the moment \expkvc\ doesn't require any internal keys, but I can't foresee
% whether this will be the case in the future as well, as it might turn out that
% some features I deem useful can't be implemented without such internal keys.
% Because of this, please don't use key names starting with
% \texttt{EKVC\string|} as that should be the private name space.
%
%
% \subsubsection{Split}
%
% The split variants will provide the key values as separate arguments. This
% limits the number of keys for which this is truly useful.
%
% \begin{function}{\ekvcSplit}
%   \begin{syntax}
%     \cs{ekvcSplit}\meta{cs}\marg{primary keys}\marg{definition}
%   \end{syntax}
%   This defines \meta{cs} to be a macro taking one mandatory argument which
%   should contain a \kv\ list. The \meta{primary keys} will be defined for this
%   macro (see \autoref{sec:primaries}). The \meta{definition} is the code that
%   will be executed. You can access the \val\ of a \key\ by using a macro
%   parameter from |#1| to |#9|. The order of the macro parameters will be the
%   order provided in the \meta{primary keys} list (so |#1| is the \val\ of the
%   key defined first). With |\ekvcSplit| you can define macros using at most
%   nine primary keys.
% \end{function}
% \example The following defines a macro |\foo| that takes the keys |a| and |b|
% and outputs their values in a textual form:\\
% \begin{minipage}{.75\linewidth}
% \begin{lstlisting}
% \ekvcSplit\foo{a=a,b=b}{a is #1.\par b is #2.\par}
% \foo{}
% \foo{b=e}
% \end{lstlisting}
% \end{minipage}^^A
% \begin{exresult}[nobeforeafter,box align=center]{.25\linewidth}
%   \ekvcSplit\foo{a=a,b=b}{a is #1.\par b is #2.\par}
%   \foo{}
%   \foo{b=e}
% \end{exresult}
%
% \begin{function}{\ekvcSplitAndForward}
%   \begin{syntax}
%     \cs{ekvcSplitAndForward}\meta{cs}\marg{after}\marg{primary keys}
%   \end{syntax}
%   This defines \meta{cs} to be a macro taking one mandatory argument which
%   should contain a \kv\ list. You can use as many primary keys as you want
%   with this. The primary keys will be forwarded to \meta{after} as braced
%   arguments (as many as necessary for your primary keys). The order of the
%   braced arguments will be the order of your primary key definitions. In
%   \meta{after} you can use just a single control sequence, or some arbitrary
%   stuff which will be left in the input stream before your braced values (with
%   one set of braces stripped from \meta{after}), so both of the following
%   would be fine:
% \begin{lstlisting}
% \ekvcSplitAndForward\foo\foo@aux{keyA = A, keyB = B}
% \ekvcSplitAndForward\foo{\foo@aux{more args}}{keyA = A, keyB = B}
% \end{lstlisting}
% \end{function}
% In the first case |\foo@aux| should take at least two arguments (|keyA| and
% |keyB|), in the second case at least three (|more args|, |keyA|, and |keyB|).
%
% \begin{function}{\ekvcSplitAndUse}
%   \begin{syntax}
%     \cs{ekvcSplitAndUse}\meta{cs}\marg{primary keys}
%   \end{syntax}
%   This will roughly do the same as |\ekvcSplitAndForward|, but instead of
%   specifying what will be used after splitting the keys, \meta{cs} will use
%   what follows the \kv\ list. So its syntax will be
% \end{function}
%   \begin{syntax}
%     \meta{cs}\{\key=\val, \ldots\}\marg{after}
%   \end{syntax}
% and the code in |after| should expect at least as many arguments as the number
% of keys defined for \meta{cs}.
%
% \subsubsection{Hash}
%
% The hash variants will provide the key values as a single argument in which
% you can access specific values using a special macro. The implementation might
% be more convenient and scale better, \emph{but} it is much slower (for a
% primitive macro with a single key benchmarking was almost $1.7$ times slower,
% the root of which being the key access with |\ekvcValue|, not the parsing, and
% for a key access using |\ekvcValueFast| it was still about $1.2$ times
% slower). So if your macro uses less than ten primary keys, you should consider
% using the split approach.
%
% \begin{function}{\ekvcHash}
%   \begin{syntax}
%     \cs{ekvcHash}\meta{cs}\marg{primary keys}\marg{definition}
%   \end{syntax}
%   This defines \meta{cs} to be a macro taking one mandatory argument which
%   should contain a \kv\ list. You can use as many primary keys as you want.
%   The primary keys will be forwarded as a single argument containing every key
%   to the underlying macro. The underlying macro is defined as
%   \meta{definition}, in which you can access the \val\ of a \key\ by using
%   \texttt{\cs[no-index]{ekvcValue}\{\key\}\{\#1\}}.
% \end{function}
% \example This defines an equivalent macro like the |\foo| defined with
% |\ekvcSplit| earlier:\\
% \begin{minipage}{.75\linewidth}
% \begin{lstlisting}
% \ekvcHash\foo{a=a,b=b}{a is \ekvcValue{a}{#1}.\par
%                        b is \ekvcValue{b}{#1}.\par}
% \foo{}
% \foo{b=e}
% \end{lstlisting}
% \end{minipage}^^A
% \begin{exresult}[nobeforeafter,box align=center]{.25\linewidth}
%   \ekvcHash\foo{a=a,b=b}{a is \ekvcValue{a}{#1}.\par
%                          b is \ekvcValue{b}{#1}.\par}
%   \foo{}
%   \foo{b=e}
% \end{exresult}
%
% \begin{function}{\ekvcHashAndForward}
%   \begin{syntax}
%     \cs{ekvcHashAndForward}\meta{cs}\marg{after}\marg{primary keys}
%   \end{syntax}
%   This defines \meta{cs} to be a macro taking one mandatory argument which
%   should contain a \kv\ list. You can use as many primary keys as you want.
%   The primary keys will be forwarded as a single argument containing every key
%   to \meta{after}. You can use a single macro for \meta{after} or use some
%   arbitrary stuff, which will be left in the input stream before the hashed
%   \kv\ list with one set of braces stripped. In the
%   macro called in \meta{after} you can access the \val\ of a \key\ by using
%   \texttt{\cs[no-index]{ekvcValue}\{\key\}\{\#1\}} (or whichever argument the
%   hashed \kv\ list will be).
% \end{function}
%
% \begin{function}{\ekvcHashAndUse}
%   \begin{syntax}
%     \cs{ekvcHashAndUse}\meta{cs}\marg{primary keys}
%   \end{syntax}
%   This will roughly do the same as |\ekvcHashAndForward|, but instead of
%   specifying what will be used after hashing the keys, \meta{cs} will use what
%   follows the \kv\ list. So its syntax will be
% \end{function}
%   \begin{syntax}
%     \meta{cs}\{\key=\val, \ldots\}\marg{after}
%   \end{syntax}
%
% \begin{function}{\ekvcValue}
%   \begin{syntax}
%     \cs{ekvcValue}\{\key\}\marg{key list}
%   \end{syntax}
%   This is a safe way to access your keys in a hash variant. \key\
%   is the key which's \val\ you want to use out of the \meta{key list}.
%   \meta{key list} should be the key list argument forwarded to your underlying
%   macro by |\ekvcHash|, |\ekvcHashAndForward|, or |\ekvcHashAndUse|. It will
%   be tested whether the hash function to access that \key\ exists, the \key\
%   argument is not empty, and that the \meta{key list} really contains a \val\
%   of that \key. This macro needs exactly two steps of expansion and if used
%   inside of an |\edef| or |\expanded| context will protect the \val\ from
%   further expanding.
% \end{function}
%
% \begin{function}{\ekvcValueFast}
%   \begin{syntax}
%     \cs{ekvcValueFast}\{\key\}\marg{key list}
%   \end{syntax}
%   This behaves just like |\ekvcValue|, but \emph{without any} safety tests. As
%   a result this is about $1.4$ times faster \emph{but} will throw low level
%   \TeX\ errors eventually if the hash function isn't defined or the \key\
%   isn't part of the \meta{key list} (\emph{e.g.}, because it was defined as a
%   key for another macro -- all macros share the same hash function per \key).
%   Use it if you know what you're doing. This macro needs exactly three steps
%   of expansion in the no-errors case.
% \end{function}
%
% \begin{function}{\ekvcValueSplit}
%   \begin{syntax}
%     \cs{ekvcValueSplit}\{\key\}\marg{key list}\marg{next}
%   \end{syntax}
%   If you need a specific \key\ from a \meta{key list} more than once, it'll
%   be a good idea to only extract it once and from then on keep it as a
%   separate argument. Hence the macro \cs{ekvcValueSplit} will extract one
%   specific \key's value from the list and forward the remainder of the list
%   as the first and the \key's value as the second argument to \meta{next}, so
%   the result of this will be \meta{next}\marg{key list'}\marg{value} with
%   \meta{key list'} the remaining list. This is almost as fast as |\ekvcValue|
%   and runs the same tests. Keep in mind that you can't fetch for the same
%   \key\ again from \meta{key list'} as it got removed.
% \end{function}
% \example The following defines a macro |\foo| which will take three keys.
% Since the next parsing step will need the value of one of the keys multiple
% times we split that key off the list:\\
% \begin{minipage}{.75\linewidth}
% \begin{lstlisting}
% \ekvcHash\foo{a=a,b=b,c=c}
%   {\ekvcValueSplit{a}{#1}\foobar}
% \newcommand*\foobar[2]{a is #2.\par
%                        b is \ekvcValue{b}{#1}.\par
%                        c is \ekvcValue{c}{#1}.\par}
% \foo{}
% \end{lstlisting}
% \end{minipage}^^A
% \begin{exresult}[nobeforeafter,box align=center]{.25\linewidth}
%   \ekvcHash\foo{a=a,b=b,c=c}
%     {\ekvcValueSplit{a}{#1}\foobar}
%   \newcommand*\foobar[2]{a is #2.\par
%                          b is \ekvcValue{b}{#1}.\par
%                          c is \ekvcValue{c}{#1}.\par}
%   \foo{}
% \end{exresult}
%
% \begin{function}{\ekvcValueSplitFast}
%   \begin{syntax}
%     \cs{ekvcValueSplitFast}\{\key\}\marg{key list}\marg{next}
%   \end{syntax}
%   This behaves just like |\ekvcValueSplit|, but it won't run the same tests,
%   hence it is faster but more error prone, just like the relation between
%   |\ekvcValue| and |\ekvcValueFast|.
% \end{function}
%
%
% \subsection{Secondary Keys}\label{sec:secondaries}
%
% To remove some of the limitations with the approach that each primary key
% matches an argument or hash entry, you can define secondary keys. Those have
% to be defined for each macro but it doesn't matter whether that macro was a
% split or a hash variant. If a secondary key references another key it doesn't
% matter whether that other key is primary or secondary.
%
% Secondary keys can have a prefix (like |long|) which are called |p|-type
% prefix and must have a type (like |meta|) which are called |t|-type prefix.
% Some types might require some |p|-prefixes, while others might forbid those.
%
% Please keep in mind that key names shouldn't start with \texttt{EKVC\string|}.
%
% \begin{function}{\ekvcSecondaryKeys}
%   \begin{syntax}
%     \cs{ekvcSecondaryKeys}\meta{cs}\{\kv, \ldots\}
%   \end{syntax}
%   This is the front facing macro to define secondary keys. For the macro
%   \meta{cs} define \key\ to have definition \val. The general syntax for \key\
%   should be
%   \begin{quote}\ttfamily\small
%     \meta{prefix} \meta{name}
%   \end{quote}
%   Where \meta{prefix} is a space separated list of optional |p|-type prefixes
%   followed by one |t|-type prefix. The syntax of \val\ is dependent on the
%   used |t|-prefix.
% \end{function}
%
%
% \subsubsection{\texttt{p}-type Prefixes}
%
% There is only one |p|-prefix available, which is |long|.
%
% \begin{function}{long}
%   The following key will be defined |\long|.
% \end{function}
%
%
% \subsubsection{\texttt{t}-type Prefixes}
%
% If you're familiar with \expkvd\ you'll notice that the |t|-type prefixes
% provided here are much fewer. The expansion only concept doesn't allow for
% great variety in the auto-defined keys.
%
% The syntax examples of the |t|-prefixes will show which |p|-prefix will be
% automatically used by printing those black (\texttt{\enflong}), which
% will be available in grey (\texttt{\alllong}), and which will be disallowed in
% red (\texttt{\notlong}). This will be put flush right next to the syntax line.
%
% \begin{function}{meta}
%   \begin{syntax}
%     meta \key\ = \{\kv, \ldots\} \prefixes2
%   \end{syntax}
%   With a |meta| key you can set other keys. Whenever \key\ is used the keys in
%   the \kv\ list will be set to the values given there. You can use the \val\
%   given to \key\ by using |#1| in the \kv\ list. The keys in the \kv\ list can
%   be primary and secondary ones.
% \end{function}
%
% \begin{function}{nmeta}
%   \begin{syntax}
%     nmeta \key\ = \{\kv, \ldots\} \prefixes3
%   \end{syntax}
%   An |nmeta| key is like a |meta| key, but it doesn't take a value, so the
%   \kv\ list is static.
% \end{function}
%
% \begin{function}{alias}
%   \begin{syntax}
%     alias \key\ = \meta{key_2} \prefixes3
%   \end{syntax}
%   This assigns the definition of \meta{key_2} to \key. As a result \key\ is an
%   alias for \meta{key_2} behaving just the same. Both the value taking and the
%   |NoVal| version (that's \expkv\ slang for a key not accepting a value) will
%   be copied if they are defined when |alias| is used. Of course, \meta{key_2}
%   has to be defined, be it as a primary or secondary one.
% \end{function}
%
% \begin{function}{default}
%   \begin{syntax}
%     default \key\ = \marg{default} \prefixes3
%   \end{syntax}
%   If \key\ is a defined value taking key, you can define a |NoVal| version
%   with this that will behave as if \key\ was given \meta{default} as its
%   \val. Note that this doesn't change the initial values of primary keys set
%   at definition time in |\ekvcSplit| and friends. \key\ can be a primary or
%   secondary key.
% \end{function}
%
% \begin{function}{flag-bool}
%   \begin{syntax}
%     flag-bool \key\ = \meta{cs} \prefixes3
%   \end{syntax}
%   This is a secondary key that doesn't directly involve any of the primary or
%   secondary keys.
%   This defines \key\ to take a value, which should be either |true| or
%   |false|, and set the flag called \meta{cs} accordingly as a boolean.
%   If \meta{cs} isn't defined yet it will be initialised as a flag.
%   Please also read \autoref{sec:flags}.
% \end{function}
%
% \begin{function}{flag-true,flag-false}
%   \begin{syntax}
%     flag-true \key\ = \meta{cs} \prefixes3
%   \end{syntax}
%   This is a secondary key that doesn't directly involve any of the primary or
%   secondary keys.
%   This defines \key\ to take no value and set the flag called \meta{cs} to
%   |true| or |false|, respectively.
%   If \meta{cs} isn't defined yet it will be initialised as a flag.
%   Please also read \autoref{sec:flags}.
% \end{function}
%
% \begin{function}{flag-raise}
%   \begin{syntax}
%     flag-raise \key\ = \meta{cs} \prefixes3
%   \end{syntax}
%   This is a secondary key that doesn't directly involve any of the primary or
%   secondary keys.
%   This defines \key\ to take no value and raise the flag called \meta{cs}.
%   If \meta{cs} isn't defined yet it will be initialised as a flag.
%   Please also read \autoref{sec:flags}.
% \end{function}
%
%
% \subsection{Changing the Initial Values}\label{sec:change}
%
% \begin{function}{\ekvcChange}
%   \begin{syntax}
%     \cs{ekvcChange}\meta{cs}\{\kv,\ldots\}
%   \end{syntax}
%   This processes the \kv\ list (just like it would be processed by \meta{cs})
%   to set new defaults for the macro \meta{cs} defined with \expkvc\ (meaning
%   the values used if you don't provide anything at use time). Inside the \kv\
%   list both primary and secondary keys can be used. If \meta{cs} was defined
%   |\long| earlier it will still be |\long|, every other \TeX\ prefix will be
%   stripped (but \expkvc\ doesn't support them anywhere else so that should be
%   fine). The resulting new defaults will be stored inside the \meta{cs}
%   locally (just as the original defaults were).
% \end{function}
%
% Consider the following example:\\
% \begin{minipage}[c]{.75\linewidth}
% \begin{lstlisting}
% \ekvcSplit\foo{a=a,b=b}{a is #1.\par b is #2.\par}
% \foo{} % initial defaults
% \ekvcChange\foo{b=B}
% \foo{} % new defaults
% \ekvcSecondaryKeys\foo{meta c={a={#1},b={#1}}}
% \ekvcChange\foo{c=c}
% \foo{} % newer defaults
% \end{lstlisting}
% \end{minipage}^^A
% \begin{exresult}[nobeforeafter,box align=center]{.25\linewidth}
%   \ekvcSplit\foo{a=a,b=b}{a is #1.\par b is #2.\par}
%   \foo{}
%   \ekvcChange\foo{b=B}
%   \foo{}
%   \ekvcSecondaryKeys\foo{meta c={a={#1},b={#1}}}
%   \ekvcChange\foo{c=c}
%   \foo{}
% \end{exresult}
%
% As a result with this the typical setup macro could be implemented:
% \begin{lstlisting}
% \ekvcHashAndUse\foo{key=a,key=b}
% \newcommand\foosetup{\ekvcChange\foo}
% \end{lstlisting}
% Of course the usage is limited to a single macro |\foo|, hence this might be
% not as powerful as similar macros used with other \kv\ interfaces. But at
% least a few similar macros could be grouped using the same key parsing macro
% internally.
%
%
% \subsection{Flags}\label{sec:flags}
%
% The idea of flags is taken from \pkg{expl3}. They provide a way to store
% numerical information expandably, however only incrementing and accessing
% works expandably, decrementing is unexpandable. A flag has a height, which
% is a numerical value, and which can be raised by $1$. Flags come at a high
% computational cost (accessing them is slow and they require more memory than
% normal \TeX\ data types like registers, both getting linearly worse with the
% height), so don't use them if not necessary.
%
% The state of flags is always changed locally to the current group, but not to
% the current macro, so if you're using one of the |t|-types involving flags
% bear in mind that they can affect other macros using the same flags at the
% current group level!
%
% \expkvc\ provides some macros to access, alter, and use flags. Flags of
% \expkvc\ don't share a name space with the flags of \pkg{expl3}.
%
% \begin{function}{\ekvcFlagNew}
%   \begin{syntax}
%     \cs{ekvcFlagNew}\meta{flag}
%   \end{syntax}
%   This initialises the macro \meta{flag} as a new flag. It isn't checked
%   whether the macro \meta{flag} is currently undefined.  A \meta{flag} will
%   expand to the flag's current height with a trailing space (so you can use it
%   directly with |\ifnum| for example and it will terminate the number scanning
%   itself).
% \end{function}
%
%  All other macros dealing with flags take as a parameter a macro defined as a
%  \meta{flag} with |\ekvcFlagNew|.
%
% \begin{function}{\ekvcFlagHeight}
%   \begin{syntax}
%     \cs{ekvcFlagHeight}\meta{flag}
%   \end{syntax}
%   This expands to the current height of \meta{flag} in a single step of
%   expansion (without a trailing space).
% \end{function}
%
% \begin{function}{\ekvcFlagRaise}
%   \begin{syntax}
%     \cs{ekvcFlagRaise}\meta{flag}
%   \end{syntax}
%   This expandably raises the height of \meta{flag} by $1$.
% \end{function}
%
% \begin{function}{\ekvcFlagSetTrue,\ekvcFlagSetFalse}
%   \begin{syntax}
%     \cs{ekvcFlagSetTrue}\meta{flag}
%   \end{syntax}
%   By interpreting an even value as false and an odd value as true we can use a
%   flag as a boolean. This expandably sets \meta{flag} to true or false,
%   respectively, by raising it if necessary.
% \end{function}
%
% \begin{function}{\ekvcFlagIf}
%   \begin{syntax}
%     \cs{ekvcFlagIf}\meta{flag}\marg{true}\marg{false}
%   \end{syntax}
%   This interprets a \meta{flag} as a boolean and expands to either \meta{true}
%   or \meta{false}.
% \end{function}
%
% \begin{function}{\ekvcFlagIfRaised}
%   \begin{syntax}
%     \cs{ekvcFlagIfRaised}\meta{flag}\marg{true}\marg{false}
%   \end{syntax}
%   This tests whether the \meta{flag} is raised, meaning it has a height
%   greater than zero, and if so expands to \meta{true} else to \meta{false}.
% \end{function}
%
% \begin{function}{\ekvcFlagReset}
%   \begin{syntax}
%     \cs{ekvcFlagReset}\meta{flag}
%   \end{syntax}
%   This resets a flag (so restores its height to $0$). This operation is
%   \emph{not} expandable and done locally. If you really intend to use flags
%   you can reset them every now and then to keep the performance hit low.
% \end{function}
%
% \begin{function}{\ekvcFlagGetHeight}
%   \begin{syntax}
%     \cs{ekvcFlagGetHeight}\meta{flag}\marg{next}
%   \end{syntax}
%   This retrieves the current height of the \meta{flag} and provides it as a
%   braced argument to \meta{next}, leaving \meta{next}\marg{height} in the
%   input stream.
% \end{function}
%
% \begin{function}{\ekvcFlagGetHeights}
%   \begin{syntax}
%     \cs{ekvcFlagGetHeights}\marg{flag-list}\marg{next}
%   \end{syntax}
%   This retrieves the current height of each \meta{flag} in the
%   \meta{flag-list} and provides them as a single argument to \meta{next}.
%   Inside that argument each height is enclosed in a set of braces
%   individually. The \meta{flag-list} is just a single argument containing the
%   \meta{flag}s.
%   So a usage like
%   \texttt
%     {^^A
%       \cs[no-index]{ekvcFlagGetHeights}\hskip0pt^^A
%       \{\cs[no-index]{myflagA}\cs[no-index]{myflagB}\}\hskip0pt^^A
%       \{\cs[no-index]{stuff}\}^^A
%     }
%   will expand to
%   \texttt{\cs[no-index]{stuff}\{\marg{height-A}\marg{height-B}\}}.
% \end{function}
%
%
% \subsection{Further Examples}
%
% How could a documentation be a good documentation without some basic examples?
% Say we want to define a small macro expanding to some character description
% (who knows why this has to be expandable?). A character description will not
% have too many items to it, so we use |\ekvcSplit| (the comments with the
% parameter numbers are of course not necessary and just ease reading the
% example).
% \begin{lstlisting}
% \ekvcSplit\character
%   {
%     name=John Doe,            % #1
%     age=any,                  % #2
%     nationality=the Universe, % #3
%     hobby=to exist,           % #4
%     type=Mister,              % #5
%     pronoun=He,               % #6
%     possessive=his,           % #7
%   }
%   {#1 is a #5 from #3. #6 is of #2 age and #7 hobby is #4.\par}
% \end{lstlisting}
% Also we want to give some short cuts so that it's easier to describe several
% persons.
% \begin{lstlisting}
% \ekvcSecondaryKeys\character
%   {
%     alias pro = pronoun,
%     alias pos = possessive,
%     nmeta me =
%       {
%         name=Jonathan,
%         age=a young,
%         nationality=Germany,
%         hobby=\TeX\ coding,
%       },
%     meta lady =
%       {type=Lady, pronoun=She, possessive=her, name=Jane Doe, #1},
%     nmeta paulo =
%       {
%         name=Paulo,
%         type=duck,
%         age=a young,
%         nationality=Brazil,
%         hobby=to quack,
%       }
%   }
% \end{lstlisting}
% Now we can describe people using
% \begin{lstlisting}
% \character{}
% \character{me}
% \character{paulo}
% \character
%   {lady={name=Evelyn,nationality=Ireland,age=the best,hobby=reading}}
% \character
%   {
%     name=Our sun, type=star, nationality=our solar system, pro=It,
%     age=an old, pos=its, hobby=shining
%   }
% \end{lstlisting}
% As one might see, the |lady| key could actually have been an |nmeta| key as
% well, as all that is done with the argument is using it as a \kv\ list.
%
% The result of only the first two usages would be:
% \begin{exresult}{\linewidth}
%   \ekvcSplit\character
%     {
%       name=John Doe,
%       age=any,
%       nationality=the Universe,
%       hobby=to exist,
%       type=Mister,
%       pronoun=He,
%       possessive=his,
%     }
%     {#1 is a #5 from #3. #6 is of #2 age and #7 hobby is #4.\par}
%   \ekvcSecondaryKeys\character
%     {
%       alias pro = pronoun,
%       alias pos = possessive,
%       nmeta me =
%         {
%           name=Jonathan,
%           age=a young,
%           nationality=Germany,
%           hobby=\TeX\ coding,
%         },
%       meta lady =
%         {type=Lady, pronoun=She, possessive=her, name=Jane Doe, #1},
%       nmeta paulo =
%         {
%           name=Paulo,
%           type=duck,
%           age=a young,
%           nationality=Brazil,
%           hobby=to quack,
%         }
%     }
%   \character{}
%   \character{me}
% \end{exresult}
%
% Using \pkg{xparse} or \expkv's \cs{ekvoptarg} or \cs{ekvoptargTF} and
% forwarding arguments one can easily define \kv\ macros with actual optional
% and mandatory arguments as well. A small nonsense example (which should
% perhaps use |\ekvcSplitAndForward| instead of |\ekvcHashAndForward| since it
% only uses four keys and one other argument -- and isn't expandable since it
% uses a \env{tabular} environment, so it would've been better to use a more
% feature rich \kv\ interface most likely, \emph{e.g.}, the one provided by
% \expkvd):
% \begin{lstlisting}
% \makeatletter
% \newcommand*\nonsense{\ekvoptarg\nonsense@a{}}
% \ekvcHashAndForward\nonsense@a\nonsense@b
%   {
%     keyA = A,
%     keyB = B,
%     keyC = c,
%     keyD = d,
%   }
% \newcommand*\nonsense@b[2]
%   {%
%     \begin{tabular}{lll}
%       key & A & \ekvcValue{keyA}{#1} \\
%           & B & \ekvcValue{keyB}{#1} \\
%           & C & \ekvcValue{keyC}{#1} \\
%           & D & \ekvcValue{keyD}{#1} \\
%       \multicolumn{2}{l}{mandatory} & #2 \\
%     \end{tabular}%
%   }
% \makeatother
% \end{lstlisting}
% And then we would be able to do some nonsense
% \begin{lstlisting}
% \nonsense{}
% \nonsense[keyA=hihi]{haha}
% \nonsense[keyA=hihi, keyB=A]{hehe}
% \nonsense[keyC=huhu, keyA=hihi, keyB=A]{haha}
% \end{lstlisting}
% resulting in
% \begin{exresult}{\linewidth}
%   \makeatletter
%   \newcommand*\nonsense{\ekvoptarg\nonsense@a{}}
%   \ekvcHashAndForward\nonsense@a\nonsense@b
%     {
%       keyA = A,
%       keyB = B,
%       keyC = c,
%       keyD = d,
%     }
%   \newcommand*\nonsense@b[2]
%     {%
%       \begin{tabular}{lll}
%         key & A & \ekvcValue{keyA}{#1} \\
%             & B & \ekvcValue{keyB}{#1} \\
%             & C & \ekvcValue{keyC}{#1} \\
%             & D & \ekvcValue{keyD}{#1} \\
%         \multicolumn{2}{l}{mandatory} & #2 \\
%       \end{tabular}%
%     }
%   \makeatother
%   \nonsense{}
%   \nonsense[keyA=hihi]{haha}
%   \nonsense[keyA=hihi, keyB=A]{hehe}
%   \nonsense[keyC=huhu, keyA=hihi, keyB=A]{haha}
% \end{exresult}
%
%
% \subsection{Freedom for Keys!}
%
% If this was the \TeX book this subsection would have a double bend sign. Not
% because it is overly complicated, but because it shows things which could
% break \expkvc's expandability and its alignment safety. This is for
% experienced users wanting to get the most flexibility and knowing what they
% are doing.
%
% In case you're wondering, it is possible to define other keys than the
% primaries and the secondary types listed in \autoref{sec:secondaries} for a
% macro defined with \expkvc\ by using the low-level interface of \expkv\ or
% even the interface provided by \expkvd. The set name used for \expkvc's keys
% is the macro name, including the leading backslash, or more precisely
% |\string|\meta{cs} is used. This can be exploited to define additional keys
% with arbitrary code. Consider the following \emph{bad} example:
% \begin{lstlisting}
% \ekvcSplit\foo{a=A,b=B}{a is #1.\par b is #2}
% \protected\ekvdef{\string\foo}{c}{\def\fooC{#1}}
% \end{lstlisting}
% This would define a key named |c| that will store its value inside a macro.
% The issue with this is that this can't be done expandably. As a result, the
% macro |\foo| isn't always expandable any more (not that bad if this was never
% required; killjoy if it was) and as soon as the key |c| is used, it is also no
% longer alignment safe\footnote{This means that the \kv-list can't contain
%   alignment markers that are not inside an additional set of braces if used
%   inside a \TeX\ alignment.}
% (might be bad depending on the usage).
%
% So why do I show you this? Because we could as well do something useful like
% create a key that pre-parses the input and after that passes the parsed value
% on. This parsing would have to be completely expandable though. For the
% pass-on part we can use the following function:
%
% \begin{function}{\ekvcPass}
%   \begin{syntax}
%     \cs{ekvcPass}\meta{cs}\marg{key}\marg{value}
%   \end{syntax}
%   This passes \meta{value} on to \meta{key} for the \expkvc-macro \meta{cs}.
%   It should be used inside the key parsing of a macro defined with \expkvc,
%   else this most likely results in a low level \TeX\ error.
% \end{function}
%
% With this we could for example split the value of a key at a hyphen and pass
% the parts to different keys:\\
% \begin{minipage}[c]{.75\linewidth}
% \begin{lstlisting}
% \ekvcSplit\foo{a=A,b=B}{a is #1.\par b is #2.\par}
% \ekvdef{\string\foo}{c}{\fooSplit#1\par}
% \def\fooSplit#1-#2\par
%   {\ekvcPass\foo{a}{#1}\ekvcPass\foo{b}{#2}}
% \foo{}
% \foo{c=1-2}
% \end{lstlisting}
% \end{minipage}
% \begin{exresult}[nobeforeafter,box align=center]{.25\linewidth}
%   \ekvcSplit\foo{a=A,b=B}{a is #1.\par b is #2.\par}
%   \ekvdef{\string\foo}{c}{\fooSplit#1\par}
%   \def\fooSplit#1-#2\par{\ekvcPass\foo{a}{#1}\ekvcPass\foo{b}{#2}}
%   \foo{}
%   \foo{c=1-2}
% \end{exresult}
%
%
% \subsection{Speed Considerations}\label{sec:speed}
%
% As already mentioned in the introduction there are some speed considerations
% implied if you choose to define macros via \expkvc. However the overhead isn't
% the factor which should hinder you to use \expkvc\ if you found a reasonable
% use case. The key-parsing is still faster than with most other \kv\ packages
% (see the ``Comparisons'' subsection in the \expkv\ documentation).
%
% The speed considerations in this subsection use the first example in this
% documentation as the benchmark. So we have seven keys and a short sentence
% which should be typeset. For comparisons I use the following equivalent
% \expkvd\ definitions. Each result is the average between changing no keys from
% their initial values and altering four. Furthermore I'll compare three
% variants of \expkvc\ with the \expkvd\ definitions, namely the split example
% from above, a hash variant using |\ekvcValue| and a hash variant using
% |\ekvcValueFast|.
% \begin{lstlisting}
% \usepackage{expkv-def}
% \ekvdefinekeys{keys}
%   {%
%     ,store   name         = \KEYSname
%     ,initial name         = John Doe
%     ,store   age          = \KEYSage
%     ,initial age          = any
%     ,store   nationality  = \KEYSnationality
%     ,initial nationality  = the Universe
%     ,store   hobby        = \KEYShobby
%     ,initial hobby        = to exist
%     ,store   type         = \KEYStype
%     ,initial type         = Mister
%     ,store   pronoun      = \KEYSpronoun
%     ,initial pronoun      = He
%     ,store   possessive   = \KEYSpossessive
%     ,initial possessive   = his
%   }
% \newcommand*\KEYS[1]
%   {%
%     \begingroup
%       \ekvset{keys}{#1}%
%       \KEYSname\ is a \KEYStype\ from \KEYSnationality. \KEYSpronoun\ is
%       of \KEYSage\ age and \KEYSpossessive\ hobby is \KEYShobby.%
%     \endgroup
%   }
% \end{lstlisting}
%
% The first comparison removes the typesetting part from all the definitions, so
% that only the key parsing is compared. In this comparison the |\ekvcValue|
% and |\ekvcValueFast| variants will not differ, as they are exactly the same
% until the key usage. We find that the split approach is $1.4$ times slower
% than the \expkvd\ setup and the hash variants end up in the middle at $1.17$
% times slower.
%
% Next we put the typesetting part back in. Every call of the macros will
% typeset the sentences into a box register in horizontal mode. With the
% typesetting part (which includes the accessing of values) the fastest remains
% the \expkvd\ definitions, but split is close at $1.16$ times slower, followed
% by the hash variant with fast accesses at $1.36$ times slower, and the safe
% hash access variant ranks in the slowest $1.8$ times slower than \expkvd.
%
% Just in case you're wondering now, a simple macro taking seven arguments is
% $30$ to $40$ times faster than any of those in the argument grabbing and \kv\
% parsing part and only $1.5$ to $2.8$ times faster if the typesetting part is
% factored in. So the real choke isn't the parsing.
%
% So to summarize this, if you have a reasonable use case for expandable \kv\
% parsing macros you should go on and define them using \expkvc. If you have a
% reasonable use case for \kv\ parsing macros but defining them expandable isn't
% necessary for your use you should take advantage of the greater flexibility of
% non-expandable \kv\ setups (but if you're after maximum speed there aren't
% that many \kv\ parsers beating \expkvc). And if you are after maximum
% performance maybe ditching the \kv\ interface altogether is a good idea, but
% depending on the number of arguments your interface might get convoluted.
%
%
% \subsection{Useless Macros}
%
% Perhaps these macros aren't completely useless, but I figured from a user's
% point of view I wouldn't know what I should do with these.
%
% \begin{function}{\ekvcDate,\ekvcVersion}
%   These two macros store the version and the date of the package/generic code.
% \end{function}
%
%
% \subsection{Bugs}
%
% Of course I don't think there are any bugs (who would knowingly distribute
% buggy software as long as he isn't a multi-million dollar corporation?). But
% if you find some please let me know. For this one might find my email address
% on the first page or file an issue on Github:
% \url{https://github.com/Skillmon/tex_expkv-cs}
%
%
% \subsection{License}
%
% Copyright \textcopyright\ 2020\unless\ifnum\year=2020--\the\year\fi\
% Jonathan P. Spratte
%
% \medskip\noindent
% This work may be distributed and/or modified under the conditions of the
% \LaTeX\ Project Public License (LPPL), either version 1.3c of this license or
% (at your option) any later version. The latest version of this license is in
% the file:
%
% \url{http://www.latex-project.org/lppl.txt}
%
% \noindent
% This work is ``maintained'' (as per LPPL maintenance status) by
%
% \mbox{Jonathan P. Spratte}.
%
% \end{documentation}^^A=<<
%
% \begin{implementation}^^A>>=
%
% \clearpage
%
% \section{Implementation}^^A>>=
%^^A the LaTeX package >>=
% \subsection{The \LaTeX\ Package}
% Just like for \expkv\ we provide a small \LaTeX\ package that sets up things
% such that we behave nicely on \LaTeX\ packages and files system. It'll
% |\input| the generic code which implements the functionality.
% \gobbledocstriptag
%<*pkg>
%    \begin{macrocode}
\RequirePackage{expkv}
\def\ekvc@tmp
  {%
    \ProvidesFile{expkv-cs.tex}%
      [%
        \ekvcDate\space v\ekvcVersion\space
        define expandable key=val macros using expkv%
      ]%
  }
% \iffalse meta-comment
%
% File: expkv-cs.dtx Copyright (C) 2020-2021 Jonathan P. Spratte
%
% This work  may be  distributed and/or  modified under  the conditions  of the
% LaTeX Project Public License (LPPL),  either version 1.3c  of this license or
% (at your option) any later version.  The latest version of this license is in
% the file:
%
%   http://www.latex-project.org/lppl.txt
%
% ------------------------------------------------------------------------------
%
%<*driver>^^A>>=
\def\nameofplainTeX{plain}
\ifx\fmtname\nameofplainTeX\else
  \expandafter\begingroup
\fi
\input l3docstrip.tex
\askforoverwritefalse
\preamble

--------------------------------------------------------------
expkv-cs -- define expandable key=val macros using expkv
E-mail: jspratte@yahoo.de
Released under the LaTeX Project Public License v1.3c or later
See http://www.latex-project.org/lppl.txt
--------------------------------------------------------------

Copyright (C) 2020-2021 Jonathan P. Spratte

This  work may be  distributed and/or  modified under  the conditions  of the
LaTeX Project Public License (LPPL),  either version 1.3c  of this license or
(at your option) any later version.  The latest version of this license is in
the file:

  http://www.latex-project.org/lppl.txt

This work is "maintained" (as per LPPL maintenance status) by
  Jonathan P. Spratte.

This work consists of the file  expkv-cs.dtx
and the derived files           expkv-cs.pdf
                                expkv-cs.sty
                                expkv-cs.tex

\endpreamble
% stop docstrip adding \endinput
\postamble
\endpostamble
\generate{\file{expkv-cs.sty}{\from{expkv-cs.dtx}{pkg}}}
\generate{\file{expkv-cs.tex}{\from{expkv-cs.dtx}{tex}}}
\ifx\fmtname\nameofplainTeX
  \expandafter\endbatchfile
\else
  \expandafter\endgroup
\fi
%
\IfFileExists{expkv-cs.tex}{\input{expkv-cs.tex}}{}
\ProvidesFile{expkv-cs.dtx}
  [\csname ekvcDate\endcsname\ define expandable key=val macros using expkv]
\PassOptionsToPackage{full}{textcomp}
\documentclass{l3doc}
\RequirePackage[oldstylenums,nott]{kpfonts}
\input{glyphtounicode}
\pdfgentounicode=1
\RequirePackage{tcolorbox}
\newtcolorbox{exresult}[2][]
  {colback=expkvgrey!10!white,colframe=expkvgrey,fontupper=\small,width={#2},#1}
\newtcbox\exres[1][]
  {
     colback=expkvgrey!10!white
    ,colframe=expkvgrey
    ,size=small
    ,nobeforeafter
    ,tcbox raise base
    ,fontupper=\small
    ,#1
  }
\RequirePackage{listings}
\RequirePackage{booktabs}
\RequirePackage{array}
\RequirePackage{collcell}
\RequirePackage{siunitx}
\RequirePackage{xcolor}
\RequirePackage{caption}
\RequirePackage{microtype}
\RequirePackage{accsupp}
\lstdefinelanguage{expkvc}[primitive]{TeX}
  {
    ,moretexcs=[4]^^A e-TeX
      {
        expanded,
        numexpr,
        protected,
      }
    ,moretexcs=[5]^^A plain/LaTeX
      {
        approx,
        begin,
        empty,
        item,
        LaTeX,
        makeatletter,makeatother,multicolumn,
        newcommand,newdimen,
        RequirePackage,
        rule,
        TeX,
        textit,texttt,
        usepackage,
      }
    ,moretexcs=[6]^^A used packages
      {
        ^^A xfp
        fpeval,
        ^^A keyval
        setkeys,
        ^^A yax
        defactiveparameter,storevalue,setparameterlist
      }
    ,moretexcs=[2]^^A expkv-cs macros
      {
        ekvcSplit,ekvcSplitAndForward,ekvcSplitAndUse,
        ekvcHash,ekvcHashAndForward,ekvcHashAndUse,
        ekvcValue,ekvcValueFast,ekvcValueSplit,ekvcValueSplitFast,
        ekvcSecondaryKeys,
        ekvcChange,
        ekvcFlagNew,
        ekvcFlagHeight,
        ekvcFlagRaise,
        ekvcFlagSetTrue,ekvcFlagSetFalse,
        ekvcFlagIf,ekvcFlagIfRaised,
        ekvcFlagReset,
        ekvcFlagGetHeight,ekvcFlagGetHeights,
        ekvcPass
      }
    ,moretexcs=[3]^^A expkv-pkg macros
      {
        ^^A expkv
        ekvdef,ekvdefNoVal,^^A
        ekvlet,ekvletNoVal,ekvletkv,ekvletkvNoVal,^^A
        ekvdefunknown,ekvdefunknownNoVal,^^A
        ekvredirectunknown,ekvredirectunknownNoVal,^^A
        ekvset,ekvsetSneaked,^^A
        ekvsetdef,ekvsetSneakeddef,ekvsetdefSneaked,^^A
        ekvparse,^^A
        ekvVersion,ekvDate,^^A
        ekvifdefined,ekvifdefinedNoVal,ekvifdefinedset,^^A
        ekvbreak,ekvbreakPreSneak,ekvbreakPostSneak,^^A
        ekvsneak,ekvsneakPre,^^A
        ekvchangeset,^^A
        ekvoptarg,ekvoptargTF,^^A
        ekverr,
        ^^A expkv-def
        ekvdefinekeys
      }
  }
\colorlet{codeparam}{cyan!65!black}
\def\mylstwd{.55em}
\lstset
  {
    ,language=expkvc
    ,flexiblecolumns=false
    ,basewidth=\mylstwd
    ,gobble=2
    ,basicstyle=\fontfamily{jkp}\itshape
    ,texcsstyle=*[1]{\bfseries\color{expkvgrey}} ^^A primitives
    ,texcsstyle=*[2]{\bfseries\color{expkvred}}  ^^A expkv
    ,texcsstyle=*[3]{\color{expkvred}}           ^^A expkv-pkg
    ,texcsstyle=*[4]{\bfseries\color{expkvgrey}} ^^A e-TeX
    ,texcsstyle=*[5]{\bfseries\color{expkvgrey}} ^^A plain/LaTeX macros
    ,texcsstyle=*[6]{}                           ^^A macros of other packages
    ,commentstyle=\color[gray]{0.4}
    ,literate=
              {\{} {{\CodeUpBf\{}}{1}
              {\}} {{\CodeUpBf\}}}{1}
              {$}  {{\CodeUpBf\$}}{1}
              {[}  {{\CodeUp[}}{1}
              {]}  {{\CodeUp]}}{1}
              {(}  {{\CodeUp(}}{1}
              {)}  {{\CodeUp)}}{1}
              {*}  {{$*$}}{1}
              {1}  {{\CodeUp{1}}}{1}
              {2}  {{\CodeUp{2}}}{1}
              {3}  {{\CodeUp{3}}}{1}
              {4}  {{\CodeUp{4}}}{1}
              {5}  {{\CodeUp{5}}}{1}
              {6}  {{\CodeUp{6}}}{1}
              {7}  {{\CodeUp{7}}}{1}
              {8}  {{\CodeUp{8}}}{1}
              {9}  {{\CodeUp{9}}}{1}
              {0}  {{\CodeUp{0}}}{1}
              {##} {{\CodeColored{codeparam}{1}{\#}}}{1}
              {##1}{{\CodeColored{codeparam}{2}{\#1}}}{2}
              {##2}{{\CodeColored{codeparam}{2}{\#2}}}{2}
              {##3}{{\CodeColored{codeparam}{2}{\#3}}}{2}
              {##4}{{\CodeColored{codeparam}{2}{\#4}}}{2}
              {##5}{{\CodeColored{codeparam}{2}{\#5}}}{2}
              {##6}{{\CodeColored{codeparam}{2}{\#6}}}{2}
              {##7}{{\CodeColored{codeparam}{2}{\#7}}}{2}
              {##8}{{\CodeColored{codeparam}{2}{\#8}}}{2}
              {##9}{{\CodeColored{codeparam}{2}{\#9}}}{2}
              {<key>}{{$\langle$}key{$\rangle$}}{5}
              {<set>}{{$\langle$}set{$\rangle$}}{5}
              {<set1>}{{$\langle$}set1{$\rangle$}}{6}
              {<set2>}{{$\langle$}set2{$\rangle$}}{6}
    ^^A,literate=*{<key>}{\key}{4}{<set>}{\set}{4}
  }
\newcommand*\CodeColored[3]{\textcolor{#1}{\makebox[\dimexpr\mylstwd*#2]{#3}}}
\newcommand*\CodeUpBf[1]{\makebox[\mylstwd]{\textup{\textbf{#1}}}}
\newcommand*\CodeUp[1]{\makebox[\mylstwd]{\textup{#1}}}
\RequirePackage{randtext}
\let\metaORIG\meta
\protected\def\meta #1{\texttt{\metaORIG{#1}}}
\renewcommand*\thefootnote{\fnsymbol{footnote}}
\definecolor{expkvred}{HTML}{9F393D}
\colorlet{expkvgrey}{black!75}
\makeatletter
\newcommand*\example{\par\smallskip\noindent\textit{Example:} \ignorespaces}
\newcommand*\expkv
  {^^A
    \texorpdfstring
      {^^A
        \mbox
          {^^A
            \BeginAccSupp{ActualText=expkv}^^A
            \href{https://github.com/Skillmon/tex_expkv}
              {^^A
                \rmfamily
                \bfseries
                {\color{expkvgrey}e\kern-.05em x\kern-.05em}^^A
                \lower.493ex
                  \hbox{{\color{expkvgrey}P}\kern-.1em{\color{expkvred}k}}^^A
                \kern-.18em{\color{expkvred}v}^^A
              }^^A
            \EndAccSupp{}^^A
          }^^A
      }
      {expkv}^^A
  }
\newcommand*\expkvpkg[1]
  {^^A
    \texorpdfstring
      {^^A
        \mbox
          {^^A
            \BeginAccSupp{ActualText=expkv-#1}^^A
            \href{https://github.com/Skillmon/tex_expkv-#1}
              {^^A
                \rmfamily
                \bfseries
                {\color{expkvgrey}e\kern-.05em x\kern-.05em}^^A
                \lower.493ex
                  \hbox{{\color{expkvgrey}P}\kern-.1em{\color{expkvred}k}}^^A
                \kern-.18em{\color{expkvred}v}^^A
                {\color{expkvgrey}^^A
                  \kern.05em\rule[-.1ex]{.08em}{1.2ex}\kern.05em\textsc{#1}^^A
                }^^A
              }^^A
            \EndAccSupp{}^^A
          }^^A
      }
      {expkv-#1}^^A
  }
\newcommand*\expkvd{\expkvpkg{def}}
\newcommand*\expkvc{\expkvpkg{cs}}
\newcommand\kv{\meta{key}=\meta{value}}
\newcommand\key{\meta{key}}
\newcommand\val{\meta{value}}
\newcommand\set{\meta{set}}
\newcommand\enflong{\textcolor{black}{long}}
\newcommand\alllong{\textcolor{gray}{long}}
\newcommand\notlong{\textcolor{red!80!black}{long}}
\newcommand\prefixes[1]
  {^^A
    \hfill
    \ifcase\numexpr#1\relax\or
      \enflong\or
      \alllong\or
      \notlong\fi
  }
\hypersetup{linkcolor=red!80!black,urlcolor=purple!80!black}
\DoNotIndex{\def,\edef,\,,\=,\begingroup,\catcode,\chardef,\csname,\endcsname}
\DoNotIndex{\endgroup,\endinput,\errmessage,\expandafter,\input,\let,\long,\z@}
\DoNotIndex{\protected,\ProvidesFile,\ProvidesPackage,\relax,\space,\advance}
\DoNotIndex{\@,\unexpanded,\string,\expanded,\dimexpr,\global,\glueexpr,\hbox}
\DoNotIndex{\numexpr,\RequirePackage,\setbox,\the,\unless,\xdef,\gdef,\newcount}
\DoNotIndex{\number}
\DoNotIndex{\^,\@firstofone,\@firstoftwo,\@gobble,\@secondoftwo}
\DoNotIndex{\ifcsname}
\DoNotIndex{\ifx}
\DoNotIndex{\ifodd}
\DoNotIndex{\ifnum}
\DoNotIndex{\ifdefined}
\DoNotIndex{\iffalse}
\DoNotIndex{\iftrue}
\DoNotIndex{\else}
\DoNotIndex{\fi}
\@gobble\fi ^^A ignoring \ifx, \ifcsname, etc., but only one \fi
\@gobble\fi ^^A ignoring \ifx, \ifcsname, etc., but only one \fi
\@gobble\fi ^^A ignoring \ifx, \ifcsname, etc., but only one \fi
\@gobble\fi ^^A ignoring \ifx, \ifcsname, etc., but only one \fi
\@gobble\fi ^^A ignoring \ifx, \ifcsname, etc., but only one \fi
\@gobble\fi ^^A ignoring \ifx, \ifcsname, etc., but only one \fi
\@ifdefinable\gobbledocstriptag{\def\gobbledocstriptag#1>{}}
\makeatother
\begin{document}
  \title
    {^^A
      \texorpdfstring
        {^^A
          \huge\expkvc
          \\[\medskipamount]
          \Large define expandable \kv\ macros using \expkv
        }{expkv-cs - define expandable key=val macros using expkv}^^A
    }
  \date{\ekvcDate\space v\ekvcVersion}
  \author{Jonathan P. Spratte\thanks{\protect\randomize{jspratte@yahoo.de}}}
  \DocInput{expkv-cs.dtx}
\end{document}
%</driver>^^A=<<
% \fi
%
% \maketitle
% \renewcommand*\thefootnote{\arabic{footnote}}
%
% \begin{abstract}
% \noindent\parfillskip=0pt
% \expkvc\ provides two small interfaces to define expandable \kv\ macros using
% \expkv. It therefore lowers the entrance boundary to expandable \kv\ macros.
% The stylised name is \expkvc\ but the files use \file{expkv-cs}, this is due
% to CTAN-rules which don't allow \string| in package names since that is the
% pipe symbol in *nix shells.
% \end{abstract}
%
% \tableofcontents
%
% \begin{documentation}^^A>>=
%
% \section{Documentation}
%
% The \expkv\ package enables the new possibility of creating \kv\ macros which
% are fully expandable. The creation of such macros is however cumbersome for
% the average user. \expkvc\ tries to step in here. It provides interfaces to
% define \kv\ macros without worrying too much about the implementation. In case
% you're wondering now, the |cs| in \expkvc\ stands for control sequence,
% because |def| was already taken by \expkvd\ and ``control sequence'' is the
% term D.\,E.\,Knuth used in his \TeX book for named commands hence macros
% (though he also used the term ``macro''). So \expkvc\ defines control
% sequences for and with \expkv.
%
% There are two different approaches supported by this package. The first is
% splitting the keys up into individual arguments, the second is providing all
% the keys as a single argument to the underlying macro and getting an
% individual \val\ by using a hash. Well, actually there is no real hash, just
% some markers which are parsed, but this shouldn't be apparent to the user, the
% behaviour matches that of a hash-table.
%
% In addition to these two methods of defining a macro with primary keys a way
% to define secondary keys, which can reference the primary ones, is provided.
% These secondary keys don't correspond to an argument or an entry in the hash
% table directly but might come in handy for the average use case. Each macro
% has its own set of primary and secondary keys.
%
% A word of advice you should consider: If your macro doesn't have to be
% expandable (and often it doesn't) consider not using \expkvc. The interface
% has some overhead (though it still is fast -- check \autoref{sec:speed}) and
% the approach has its limits in versatility. If you don't need to be
% expandable, you should consider either defining your keys manually using
% \expkv\ or using \expkvd\ for convenience. Or you resort to another \kv\
% interface. Nevertheless setting up macros with \expkvc, especially with
% |\ekvcSplit|, is very convenient in my opinion, so if you just want to define
% a single macro with just a few keys this might be the way to go.
%
% \expkvc\ is usable as generic code and as a \LaTeX\ package. It'll
% automatically load \expkv\ in the same mode as well. To use it, just use one
% of
% \begin{lstlisting}
% \usepackage{expkv-cs} % LaTeX
% \input expkv-cs       % plainTeX
% \end{lstlisting}
%
%
% \subsection{Define Macros and Primary Keys}
%
% All macros defined with \expkvc\ have to be previously undefined or have the
% |\meaning| of |\relax|. This is necessary as there is no way to automatically
% undefine keys once they are set up (neither \expkv\ nor \expkvc\ keep track of
% defined keys) -- so to make sure there are no conflicts only new definitions
% are allowed (that's not the case for individual keys, only for frontend
% macros).
%
%
% \subsubsection{Primary Keys}\label{sec:primaries}
%
% In the following descriptions there will be one argument named \meta{primary
% keys}. This argument should be a \kv\ list where each \key\ will be one
% primary key and \val\ the associated initial value. By default all keys are
% defined short, but you can define long keys by prefixing \key\ with |long|
% (\emph{e.g.}, \texttt{long name=Jonathan P. Spratte}). You only need |long| if
% the key should be able to take a |\par| token. Note however that |long| keys
% are a microscopic grain faster (due to some internals of \expkvc). Only if at
% least one of the keys was |long| the \meta{cs} in the following defining
% macros will be |\long|. For obvious reasons there is no possibility to define
% a macro or key as |\protected|.
%
% At the moment \expkvc\ doesn't require any internal keys, but I can't foresee
% whether this will be the case in the future as well, as it might turn out that
% some features I deem useful can't be implemented without such internal keys.
% Because of this, please don't use key names starting with
% \texttt{EKVC\string|} as that should be the private name space.
%
%
% \subsubsection{Split}
%
% The split variants will provide the key values as separate arguments. This
% limits the number of keys for which this is truly useful.
%
% \begin{function}{\ekvcSplit}
%   \begin{syntax}
%     \cs{ekvcSplit}\meta{cs}\marg{primary keys}\marg{definition}
%   \end{syntax}
%   This defines \meta{cs} to be a macro taking one mandatory argument which
%   should contain a \kv\ list. The \meta{primary keys} will be defined for this
%   macro (see \autoref{sec:primaries}). The \meta{definition} is the code that
%   will be executed. You can access the \val\ of a \key\ by using a macro
%   parameter from |#1| to |#9|. The order of the macro parameters will be the
%   order provided in the \meta{primary keys} list (so |#1| is the \val\ of the
%   key defined first). With |\ekvcSplit| you can define macros using at most
%   nine primary keys.
% \end{function}
% \example The following defines a macro |\foo| that takes the keys |a| and |b|
% and outputs their values in a textual form:\\
% \begin{minipage}{.75\linewidth}
% \begin{lstlisting}
% \ekvcSplit\foo{a=a,b=b}{a is #1.\par b is #2.\par}
% \foo{}
% \foo{b=e}
% \end{lstlisting}
% \end{minipage}^^A
% \begin{exresult}[nobeforeafter,box align=center]{.25\linewidth}
%   \ekvcSplit\foo{a=a,b=b}{a is #1.\par b is #2.\par}
%   \foo{}
%   \foo{b=e}
% \end{exresult}
%
% \begin{function}{\ekvcSplitAndForward}
%   \begin{syntax}
%     \cs{ekvcSplitAndForward}\meta{cs}\marg{after}\marg{primary keys}
%   \end{syntax}
%   This defines \meta{cs} to be a macro taking one mandatory argument which
%   should contain a \kv\ list. You can use as many primary keys as you want
%   with this. The primary keys will be forwarded to \meta{after} as braced
%   arguments (as many as necessary for your primary keys). The order of the
%   braced arguments will be the order of your primary key definitions. In
%   \meta{after} you can use just a single control sequence, or some arbitrary
%   stuff which will be left in the input stream before your braced values (with
%   one set of braces stripped from \meta{after}), so both of the following
%   would be fine:
% \begin{lstlisting}
% \ekvcSplitAndForward\foo\foo@aux{keyA = A, keyB = B}
% \ekvcSplitAndForward\foo{\foo@aux{more args}}{keyA = A, keyB = B}
% \end{lstlisting}
% \end{function}
% In the first case |\foo@aux| should take at least two arguments (|keyA| and
% |keyB|), in the second case at least three (|more args|, |keyA|, and |keyB|).
%
% \begin{function}{\ekvcSplitAndUse}
%   \begin{syntax}
%     \cs{ekvcSplitAndUse}\meta{cs}\marg{primary keys}
%   \end{syntax}
%   This will roughly do the same as |\ekvcSplitAndForward|, but instead of
%   specifying what will be used after splitting the keys, \meta{cs} will use
%   what follows the \kv\ list. So its syntax will be
% \end{function}
%   \begin{syntax}
%     \meta{cs}\{\key=\val, \ldots\}\marg{after}
%   \end{syntax}
% and the code in |after| should expect at least as many arguments as the number
% of keys defined for \meta{cs}.
%
% \subsubsection{Hash}
%
% The hash variants will provide the key values as a single argument in which
% you can access specific values using a special macro. The implementation might
% be more convenient and scale better, \emph{but} it is much slower (for a
% primitive macro with a single key benchmarking was almost $1.7$ times slower,
% the root of which being the key access with |\ekvcValue|, not the parsing, and
% for a key access using |\ekvcValueFast| it was still about $1.2$ times
% slower). So if your macro uses less than ten primary keys, you should consider
% using the split approach.
%
% \begin{function}{\ekvcHash}
%   \begin{syntax}
%     \cs{ekvcHash}\meta{cs}\marg{primary keys}\marg{definition}
%   \end{syntax}
%   This defines \meta{cs} to be a macro taking one mandatory argument which
%   should contain a \kv\ list. You can use as many primary keys as you want.
%   The primary keys will be forwarded as a single argument containing every key
%   to the underlying macro. The underlying macro is defined as
%   \meta{definition}, in which you can access the \val\ of a \key\ by using
%   \texttt{\cs[no-index]{ekvcValue}\{\key\}\{\#1\}}.
% \end{function}
% \example This defines an equivalent macro like the |\foo| defined with
% |\ekvcSplit| earlier:\\
% \begin{minipage}{.75\linewidth}
% \begin{lstlisting}
% \ekvcHash\foo{a=a,b=b}{a is \ekvcValue{a}{#1}.\par
%                        b is \ekvcValue{b}{#1}.\par}
% \foo{}
% \foo{b=e}
% \end{lstlisting}
% \end{minipage}^^A
% \begin{exresult}[nobeforeafter,box align=center]{.25\linewidth}
%   \ekvcHash\foo{a=a,b=b}{a is \ekvcValue{a}{#1}.\par
%                          b is \ekvcValue{b}{#1}.\par}
%   \foo{}
%   \foo{b=e}
% \end{exresult}
%
% \begin{function}{\ekvcHashAndForward}
%   \begin{syntax}
%     \cs{ekvcHashAndForward}\meta{cs}\marg{after}\marg{primary keys}
%   \end{syntax}
%   This defines \meta{cs} to be a macro taking one mandatory argument which
%   should contain a \kv\ list. You can use as many primary keys as you want.
%   The primary keys will be forwarded as a single argument containing every key
%   to \meta{after}. You can use a single macro for \meta{after} or use some
%   arbitrary stuff, which will be left in the input stream before the hashed
%   \kv\ list with one set of braces stripped. In the
%   macro called in \meta{after} you can access the \val\ of a \key\ by using
%   \texttt{\cs[no-index]{ekvcValue}\{\key\}\{\#1\}} (or whichever argument the
%   hashed \kv\ list will be).
% \end{function}
%
% \begin{function}{\ekvcHashAndUse}
%   \begin{syntax}
%     \cs{ekvcHashAndUse}\meta{cs}\marg{primary keys}
%   \end{syntax}
%   This will roughly do the same as |\ekvcHashAndForward|, but instead of
%   specifying what will be used after hashing the keys, \meta{cs} will use what
%   follows the \kv\ list. So its syntax will be
% \end{function}
%   \begin{syntax}
%     \meta{cs}\{\key=\val, \ldots\}\marg{after}
%   \end{syntax}
%
% \begin{function}{\ekvcValue}
%   \begin{syntax}
%     \cs{ekvcValue}\{\key\}\marg{key list}
%   \end{syntax}
%   This is a safe way to access your keys in a hash variant. \key\
%   is the key which's \val\ you want to use out of the \meta{key list}.
%   \meta{key list} should be the key list argument forwarded to your underlying
%   macro by |\ekvcHash|, |\ekvcHashAndForward|, or |\ekvcHashAndUse|. It will
%   be tested whether the hash function to access that \key\ exists, the \key\
%   argument is not empty, and that the \meta{key list} really contains a \val\
%   of that \key. This macro needs exactly two steps of expansion and if used
%   inside of an |\edef| or |\expanded| context will protect the \val\ from
%   further expanding.
% \end{function}
%
% \begin{function}{\ekvcValueFast}
%   \begin{syntax}
%     \cs{ekvcValueFast}\{\key\}\marg{key list}
%   \end{syntax}
%   This behaves just like |\ekvcValue|, but \emph{without any} safety tests. As
%   a result this is about $1.4$ times faster \emph{but} will throw low level
%   \TeX\ errors eventually if the hash function isn't defined or the \key\
%   isn't part of the \meta{key list} (\emph{e.g.}, because it was defined as a
%   key for another macro -- all macros share the same hash function per \key).
%   Use it if you know what you're doing. This macro needs exactly three steps
%   of expansion in the no-errors case.
% \end{function}
%
% \begin{function}{\ekvcValueSplit}
%   \begin{syntax}
%     \cs{ekvcValueSplit}\{\key\}\marg{key list}\marg{next}
%   \end{syntax}
%   If you need a specific \key\ from a \meta{key list} more than once, it'll
%   be a good idea to only extract it once and from then on keep it as a
%   separate argument. Hence the macro \cs{ekvcValueSplit} will extract one
%   specific \key's value from the list and forward the remainder of the list
%   as the first and the \key's value as the second argument to \meta{next}, so
%   the result of this will be \meta{next}\marg{key list'}\marg{value} with
%   \meta{key list'} the remaining list. This is almost as fast as |\ekvcValue|
%   and runs the same tests. Keep in mind that you can't fetch for the same
%   \key\ again from \meta{key list'} as it got removed.
% \end{function}
% \example The following defines a macro |\foo| which will take three keys.
% Since the next parsing step will need the value of one of the keys multiple
% times we split that key off the list:\\
% \begin{minipage}{.75\linewidth}
% \begin{lstlisting}
% \ekvcHash\foo{a=a,b=b,c=c}
%   {\ekvcValueSplit{a}{#1}\foobar}
% \newcommand*\foobar[2]{a is #2.\par
%                        b is \ekvcValue{b}{#1}.\par
%                        c is \ekvcValue{c}{#1}.\par}
% \foo{}
% \end{lstlisting}
% \end{minipage}^^A
% \begin{exresult}[nobeforeafter,box align=center]{.25\linewidth}
%   \ekvcHash\foo{a=a,b=b,c=c}
%     {\ekvcValueSplit{a}{#1}\foobar}
%   \newcommand*\foobar[2]{a is #2.\par
%                          b is \ekvcValue{b}{#1}.\par
%                          c is \ekvcValue{c}{#1}.\par}
%   \foo{}
% \end{exresult}
%
% \begin{function}{\ekvcValueSplitFast}
%   \begin{syntax}
%     \cs{ekvcValueSplitFast}\{\key\}\marg{key list}\marg{next}
%   \end{syntax}
%   This behaves just like |\ekvcValueSplit|, but it won't run the same tests,
%   hence it is faster but more error prone, just like the relation between
%   |\ekvcValue| and |\ekvcValueFast|.
% \end{function}
%
%
% \subsection{Secondary Keys}\label{sec:secondaries}
%
% To remove some of the limitations with the approach that each primary key
% matches an argument or hash entry, you can define secondary keys. Those have
% to be defined for each macro but it doesn't matter whether that macro was a
% split or a hash variant. If a secondary key references another key it doesn't
% matter whether that other key is primary or secondary.
%
% Secondary keys can have a prefix (like |long|) which are called |p|-type
% prefix and must have a type (like |meta|) which are called |t|-type prefix.
% Some types might require some |p|-prefixes, while others might forbid those.
%
% Please keep in mind that key names shouldn't start with \texttt{EKVC\string|}.
%
% \begin{function}{\ekvcSecondaryKeys}
%   \begin{syntax}
%     \cs{ekvcSecondaryKeys}\meta{cs}\{\kv, \ldots\}
%   \end{syntax}
%   This is the front facing macro to define secondary keys. For the macro
%   \meta{cs} define \key\ to have definition \val. The general syntax for \key\
%   should be
%   \begin{quote}\ttfamily\small
%     \meta{prefix} \meta{name}
%   \end{quote}
%   Where \meta{prefix} is a space separated list of optional |p|-type prefixes
%   followed by one |t|-type prefix. The syntax of \val\ is dependent on the
%   used |t|-prefix.
% \end{function}
%
%
% \subsubsection{\texttt{p}-type Prefixes}
%
% There is only one |p|-prefix available, which is |long|.
%
% \begin{function}{long}
%   The following key will be defined |\long|.
% \end{function}
%
%
% \subsubsection{\texttt{t}-type Prefixes}
%
% If you're familiar with \expkvd\ you'll notice that the |t|-type prefixes
% provided here are much fewer. The expansion only concept doesn't allow for
% great variety in the auto-defined keys.
%
% The syntax examples of the |t|-prefixes will show which |p|-prefix will be
% automatically used by printing those black (\texttt{\enflong}), which
% will be available in grey (\texttt{\alllong}), and which will be disallowed in
% red (\texttt{\notlong}). This will be put flush right next to the syntax line.
%
% \begin{function}{meta}
%   \begin{syntax}
%     meta \key\ = \{\kv, \ldots\} \prefixes2
%   \end{syntax}
%   With a |meta| key you can set other keys. Whenever \key\ is used the keys in
%   the \kv\ list will be set to the values given there. You can use the \val\
%   given to \key\ by using |#1| in the \kv\ list. The keys in the \kv\ list can
%   be primary and secondary ones.
% \end{function}
%
% \begin{function}{nmeta}
%   \begin{syntax}
%     nmeta \key\ = \{\kv, \ldots\} \prefixes3
%   \end{syntax}
%   An |nmeta| key is like a |meta| key, but it doesn't take a value, so the
%   \kv\ list is static.
% \end{function}
%
% \begin{function}{alias}
%   \begin{syntax}
%     alias \key\ = \meta{key_2} \prefixes3
%   \end{syntax}
%   This assigns the definition of \meta{key_2} to \key. As a result \key\ is an
%   alias for \meta{key_2} behaving just the same. Both the value taking and the
%   |NoVal| version (that's \expkv\ slang for a key not accepting a value) will
%   be copied if they are defined when |alias| is used. Of course, \meta{key_2}
%   has to be defined, be it as a primary or secondary one.
% \end{function}
%
% \begin{function}{default}
%   \begin{syntax}
%     default \key\ = \marg{default} \prefixes3
%   \end{syntax}
%   If \key\ is a defined value taking key, you can define a |NoVal| version
%   with this that will behave as if \key\ was given \meta{default} as its
%   \val. Note that this doesn't change the initial values of primary keys set
%   at definition time in |\ekvcSplit| and friends. \key\ can be a primary or
%   secondary key.
% \end{function}
%
% \begin{function}{flag-bool}
%   \begin{syntax}
%     flag-bool \key\ = \meta{cs} \prefixes3
%   \end{syntax}
%   This is a secondary key that doesn't directly involve any of the primary or
%   secondary keys.
%   This defines \key\ to take a value, which should be either |true| or
%   |false|, and set the flag called \meta{cs} accordingly as a boolean.
%   If \meta{cs} isn't defined yet it will be initialised as a flag.
%   Please also read \autoref{sec:flags}.
% \end{function}
%
% \begin{function}{flag-true,flag-false}
%   \begin{syntax}
%     flag-true \key\ = \meta{cs} \prefixes3
%   \end{syntax}
%   This is a secondary key that doesn't directly involve any of the primary or
%   secondary keys.
%   This defines \key\ to take no value and set the flag called \meta{cs} to
%   |true| or |false|, respectively.
%   If \meta{cs} isn't defined yet it will be initialised as a flag.
%   Please also read \autoref{sec:flags}.
% \end{function}
%
% \begin{function}{flag-raise}
%   \begin{syntax}
%     flag-raise \key\ = \meta{cs} \prefixes3
%   \end{syntax}
%   This is a secondary key that doesn't directly involve any of the primary or
%   secondary keys.
%   This defines \key\ to take no value and raise the flag called \meta{cs}.
%   If \meta{cs} isn't defined yet it will be initialised as a flag.
%   Please also read \autoref{sec:flags}.
% \end{function}
%
%
% \subsection{Changing the Initial Values}\label{sec:change}
%
% \begin{function}{\ekvcChange}
%   \begin{syntax}
%     \cs{ekvcChange}\meta{cs}\{\kv,\ldots\}
%   \end{syntax}
%   This processes the \kv\ list (just like it would be processed by \meta{cs})
%   to set new defaults for the macro \meta{cs} defined with \expkvc\ (meaning
%   the values used if you don't provide anything at use time). Inside the \kv\
%   list both primary and secondary keys can be used. If \meta{cs} was defined
%   |\long| earlier it will still be |\long|, every other \TeX\ prefix will be
%   stripped (but \expkvc\ doesn't support them anywhere else so that should be
%   fine). The resulting new defaults will be stored inside the \meta{cs}
%   locally (just as the original defaults were).
% \end{function}
%
% Consider the following example:\\
% \begin{minipage}[c]{.75\linewidth}
% \begin{lstlisting}
% \ekvcSplit\foo{a=a,b=b}{a is #1.\par b is #2.\par}
% \foo{} % initial defaults
% \ekvcChange\foo{b=B}
% \foo{} % new defaults
% \ekvcSecondaryKeys\foo{meta c={a={#1},b={#1}}}
% \ekvcChange\foo{c=c}
% \foo{} % newer defaults
% \end{lstlisting}
% \end{minipage}^^A
% \begin{exresult}[nobeforeafter,box align=center]{.25\linewidth}
%   \ekvcSplit\foo{a=a,b=b}{a is #1.\par b is #2.\par}
%   \foo{}
%   \ekvcChange\foo{b=B}
%   \foo{}
%   \ekvcSecondaryKeys\foo{meta c={a={#1},b={#1}}}
%   \ekvcChange\foo{c=c}
%   \foo{}
% \end{exresult}
%
% As a result with this the typical setup macro could be implemented:
% \begin{lstlisting}
% \ekvcHashAndUse\foo{key=a,key=b}
% \newcommand\foosetup{\ekvcChange\foo}
% \end{lstlisting}
% Of course the usage is limited to a single macro |\foo|, hence this might be
% not as powerful as similar macros used with other \kv\ interfaces. But at
% least a few similar macros could be grouped using the same key parsing macro
% internally.
%
%
% \subsection{Flags}\label{sec:flags}
%
% The idea of flags is taken from \pkg{expl3}. They provide a way to store
% numerical information expandably, however only incrementing and accessing
% works expandably, decrementing is unexpandable. A flag has a height, which
% is a numerical value, and which can be raised by $1$. Flags come at a high
% computational cost (accessing them is slow and they require more memory than
% normal \TeX\ data types like registers, both getting linearly worse with the
% height), so don't use them if not necessary.
%
% The state of flags is always changed locally to the current group, but not to
% the current macro, so if you're using one of the |t|-types involving flags
% bear in mind that they can affect other macros using the same flags at the
% current group level!
%
% \expkvc\ provides some macros to access, alter, and use flags. Flags of
% \expkvc\ don't share a name space with the flags of \pkg{expl3}.
%
% \begin{function}{\ekvcFlagNew}
%   \begin{syntax}
%     \cs{ekvcFlagNew}\meta{flag}
%   \end{syntax}
%   This initialises the macro \meta{flag} as a new flag. It isn't checked
%   whether the macro \meta{flag} is currently undefined.  A \meta{flag} will
%   expand to the flag's current height with a trailing space (so you can use it
%   directly with |\ifnum| for example and it will terminate the number scanning
%   itself).
% \end{function}
%
%  All other macros dealing with flags take as a parameter a macro defined as a
%  \meta{flag} with |\ekvcFlagNew|.
%
% \begin{function}{\ekvcFlagHeight}
%   \begin{syntax}
%     \cs{ekvcFlagHeight}\meta{flag}
%   \end{syntax}
%   This expands to the current height of \meta{flag} in a single step of
%   expansion (without a trailing space).
% \end{function}
%
% \begin{function}{\ekvcFlagRaise}
%   \begin{syntax}
%     \cs{ekvcFlagRaise}\meta{flag}
%   \end{syntax}
%   This expandably raises the height of \meta{flag} by $1$.
% \end{function}
%
% \begin{function}{\ekvcFlagSetTrue,\ekvcFlagSetFalse}
%   \begin{syntax}
%     \cs{ekvcFlagSetTrue}\meta{flag}
%   \end{syntax}
%   By interpreting an even value as false and an odd value as true we can use a
%   flag as a boolean. This expandably sets \meta{flag} to true or false,
%   respectively, by raising it if necessary.
% \end{function}
%
% \begin{function}{\ekvcFlagIf}
%   \begin{syntax}
%     \cs{ekvcFlagIf}\meta{flag}\marg{true}\marg{false}
%   \end{syntax}
%   This interprets a \meta{flag} as a boolean and expands to either \meta{true}
%   or \meta{false}.
% \end{function}
%
% \begin{function}{\ekvcFlagIfRaised}
%   \begin{syntax}
%     \cs{ekvcFlagIfRaised}\meta{flag}\marg{true}\marg{false}
%   \end{syntax}
%   This tests whether the \meta{flag} is raised, meaning it has a height
%   greater than zero, and if so expands to \meta{true} else to \meta{false}.
% \end{function}
%
% \begin{function}{\ekvcFlagReset}
%   \begin{syntax}
%     \cs{ekvcFlagReset}\meta{flag}
%   \end{syntax}
%   This resets a flag (so restores its height to $0$). This operation is
%   \emph{not} expandable and done locally. If you really intend to use flags
%   you can reset them every now and then to keep the performance hit low.
% \end{function}
%
% \begin{function}{\ekvcFlagGetHeight}
%   \begin{syntax}
%     \cs{ekvcFlagGetHeight}\meta{flag}\marg{next}
%   \end{syntax}
%   This retrieves the current height of the \meta{flag} and provides it as a
%   braced argument to \meta{next}, leaving \meta{next}\marg{height} in the
%   input stream.
% \end{function}
%
% \begin{function}{\ekvcFlagGetHeights}
%   \begin{syntax}
%     \cs{ekvcFlagGetHeights}\marg{flag-list}\marg{next}
%   \end{syntax}
%   This retrieves the current height of each \meta{flag} in the
%   \meta{flag-list} and provides them as a single argument to \meta{next}.
%   Inside that argument each height is enclosed in a set of braces
%   individually. The \meta{flag-list} is just a single argument containing the
%   \meta{flag}s.
%   So a usage like
%   \texttt
%     {^^A
%       \cs[no-index]{ekvcFlagGetHeights}\hskip0pt^^A
%       \{\cs[no-index]{myflagA}\cs[no-index]{myflagB}\}\hskip0pt^^A
%       \{\cs[no-index]{stuff}\}^^A
%     }
%   will expand to
%   \texttt{\cs[no-index]{stuff}\{\marg{height-A}\marg{height-B}\}}.
% \end{function}
%
%
% \subsection{Further Examples}
%
% How could a documentation be a good documentation without some basic examples?
% Say we want to define a small macro expanding to some character description
% (who knows why this has to be expandable?). A character description will not
% have too many items to it, so we use |\ekvcSplit| (the comments with the
% parameter numbers are of course not necessary and just ease reading the
% example).
% \begin{lstlisting}
% \ekvcSplit\character
%   {
%     name=John Doe,            % #1
%     age=any,                  % #2
%     nationality=the Universe, % #3
%     hobby=to exist,           % #4
%     type=Mister,              % #5
%     pronoun=He,               % #6
%     possessive=his,           % #7
%   }
%   {#1 is a #5 from #3. #6 is of #2 age and #7 hobby is #4.\par}
% \end{lstlisting}
% Also we want to give some short cuts so that it's easier to describe several
% persons.
% \begin{lstlisting}
% \ekvcSecondaryKeys\character
%   {
%     alias pro = pronoun,
%     alias pos = possessive,
%     nmeta me =
%       {
%         name=Jonathan,
%         age=a young,
%         nationality=Germany,
%         hobby=\TeX\ coding,
%       },
%     meta lady =
%       {type=Lady, pronoun=She, possessive=her, name=Jane Doe, #1},
%     nmeta paulo =
%       {
%         name=Paulo,
%         type=duck,
%         age=a young,
%         nationality=Brazil,
%         hobby=to quack,
%       }
%   }
% \end{lstlisting}
% Now we can describe people using
% \begin{lstlisting}
% \character{}
% \character{me}
% \character{paulo}
% \character
%   {lady={name=Evelyn,nationality=Ireland,age=the best,hobby=reading}}
% \character
%   {
%     name=Our sun, type=star, nationality=our solar system, pro=It,
%     age=an old, pos=its, hobby=shining
%   }
% \end{lstlisting}
% As one might see, the |lady| key could actually have been an |nmeta| key as
% well, as all that is done with the argument is using it as a \kv\ list.
%
% The result of only the first two usages would be:
% \begin{exresult}{\linewidth}
%   \ekvcSplit\character
%     {
%       name=John Doe,
%       age=any,
%       nationality=the Universe,
%       hobby=to exist,
%       type=Mister,
%       pronoun=He,
%       possessive=his,
%     }
%     {#1 is a #5 from #3. #6 is of #2 age and #7 hobby is #4.\par}
%   \ekvcSecondaryKeys\character
%     {
%       alias pro = pronoun,
%       alias pos = possessive,
%       nmeta me =
%         {
%           name=Jonathan,
%           age=a young,
%           nationality=Germany,
%           hobby=\TeX\ coding,
%         },
%       meta lady =
%         {type=Lady, pronoun=She, possessive=her, name=Jane Doe, #1},
%       nmeta paulo =
%         {
%           name=Paulo,
%           type=duck,
%           age=a young,
%           nationality=Brazil,
%           hobby=to quack,
%         }
%     }
%   \character{}
%   \character{me}
% \end{exresult}
%
% Using \pkg{xparse} or \expkv's \cs{ekvoptarg} or \cs{ekvoptargTF} and
% forwarding arguments one can easily define \kv\ macros with actual optional
% and mandatory arguments as well. A small nonsense example (which should
% perhaps use |\ekvcSplitAndForward| instead of |\ekvcHashAndForward| since it
% only uses four keys and one other argument -- and isn't expandable since it
% uses a \env{tabular} environment, so it would've been better to use a more
% feature rich \kv\ interface most likely, \emph{e.g.}, the one provided by
% \expkvd):
% \begin{lstlisting}
% \makeatletter
% \newcommand*\nonsense{\ekvoptarg\nonsense@a{}}
% \ekvcHashAndForward\nonsense@a\nonsense@b
%   {
%     keyA = A,
%     keyB = B,
%     keyC = c,
%     keyD = d,
%   }
% \newcommand*\nonsense@b[2]
%   {%
%     \begin{tabular}{lll}
%       key & A & \ekvcValue{keyA}{#1} \\
%           & B & \ekvcValue{keyB}{#1} \\
%           & C & \ekvcValue{keyC}{#1} \\
%           & D & \ekvcValue{keyD}{#1} \\
%       \multicolumn{2}{l}{mandatory} & #2 \\
%     \end{tabular}%
%   }
% \makeatother
% \end{lstlisting}
% And then we would be able to do some nonsense
% \begin{lstlisting}
% \nonsense{}
% \nonsense[keyA=hihi]{haha}
% \nonsense[keyA=hihi, keyB=A]{hehe}
% \nonsense[keyC=huhu, keyA=hihi, keyB=A]{haha}
% \end{lstlisting}
% resulting in
% \begin{exresult}{\linewidth}
%   \makeatletter
%   \newcommand*\nonsense{\ekvoptarg\nonsense@a{}}
%   \ekvcHashAndForward\nonsense@a\nonsense@b
%     {
%       keyA = A,
%       keyB = B,
%       keyC = c,
%       keyD = d,
%     }
%   \newcommand*\nonsense@b[2]
%     {%
%       \begin{tabular}{lll}
%         key & A & \ekvcValue{keyA}{#1} \\
%             & B & \ekvcValue{keyB}{#1} \\
%             & C & \ekvcValue{keyC}{#1} \\
%             & D & \ekvcValue{keyD}{#1} \\
%         \multicolumn{2}{l}{mandatory} & #2 \\
%       \end{tabular}%
%     }
%   \makeatother
%   \nonsense{}
%   \nonsense[keyA=hihi]{haha}
%   \nonsense[keyA=hihi, keyB=A]{hehe}
%   \nonsense[keyC=huhu, keyA=hihi, keyB=A]{haha}
% \end{exresult}
%
%
% \subsection{Freedom for Keys!}
%
% If this was the \TeX book this subsection would have a double bend sign. Not
% because it is overly complicated, but because it shows things which could
% break \expkvc's expandability and its alignment safety. This is for
% experienced users wanting to get the most flexibility and knowing what they
% are doing.
%
% In case you're wondering, it is possible to define other keys than the
% primaries and the secondary types listed in \autoref{sec:secondaries} for a
% macro defined with \expkvc\ by using the low-level interface of \expkv\ or
% even the interface provided by \expkvd. The set name used for \expkvc's keys
% is the macro name, including the leading backslash, or more precisely
% |\string|\meta{cs} is used. This can be exploited to define additional keys
% with arbitrary code. Consider the following \emph{bad} example:
% \begin{lstlisting}
% \ekvcSplit\foo{a=A,b=B}{a is #1.\par b is #2}
% \protected\ekvdef{\string\foo}{c}{\def\fooC{#1}}
% \end{lstlisting}
% This would define a key named |c| that will store its value inside a macro.
% The issue with this is that this can't be done expandably. As a result, the
% macro |\foo| isn't always expandable any more (not that bad if this was never
% required; killjoy if it was) and as soon as the key |c| is used, it is also no
% longer alignment safe\footnote{This means that the \kv-list can't contain
%   alignment markers that are not inside an additional set of braces if used
%   inside a \TeX\ alignment.}
% (might be bad depending on the usage).
%
% So why do I show you this? Because we could as well do something useful like
% create a key that pre-parses the input and after that passes the parsed value
% on. This parsing would have to be completely expandable though. For the
% pass-on part we can use the following function:
%
% \begin{function}{\ekvcPass}
%   \begin{syntax}
%     \cs{ekvcPass}\meta{cs}\marg{key}\marg{value}
%   \end{syntax}
%   This passes \meta{value} on to \meta{key} for the \expkvc-macro \meta{cs}.
%   It should be used inside the key parsing of a macro defined with \expkvc,
%   else this most likely results in a low level \TeX\ error.
% \end{function}
%
% With this we could for example split the value of a key at a hyphen and pass
% the parts to different keys:\\
% \begin{minipage}[c]{.75\linewidth}
% \begin{lstlisting}
% \ekvcSplit\foo{a=A,b=B}{a is #1.\par b is #2.\par}
% \ekvdef{\string\foo}{c}{\fooSplit#1\par}
% \def\fooSplit#1-#2\par
%   {\ekvcPass\foo{a}{#1}\ekvcPass\foo{b}{#2}}
% \foo{}
% \foo{c=1-2}
% \end{lstlisting}
% \end{minipage}
% \begin{exresult}[nobeforeafter,box align=center]{.25\linewidth}
%   \ekvcSplit\foo{a=A,b=B}{a is #1.\par b is #2.\par}
%   \ekvdef{\string\foo}{c}{\fooSplit#1\par}
%   \def\fooSplit#1-#2\par{\ekvcPass\foo{a}{#1}\ekvcPass\foo{b}{#2}}
%   \foo{}
%   \foo{c=1-2}
% \end{exresult}
%
%
% \subsection{Speed Considerations}\label{sec:speed}
%
% As already mentioned in the introduction there are some speed considerations
% implied if you choose to define macros via \expkvc. However the overhead isn't
% the factor which should hinder you to use \expkvc\ if you found a reasonable
% use case. The key-parsing is still faster than with most other \kv\ packages
% (see the ``Comparisons'' subsection in the \expkv\ documentation).
%
% The speed considerations in this subsection use the first example in this
% documentation as the benchmark. So we have seven keys and a short sentence
% which should be typeset. For comparisons I use the following equivalent
% \expkvd\ definitions. Each result is the average between changing no keys from
% their initial values and altering four. Furthermore I'll compare three
% variants of \expkvc\ with the \expkvd\ definitions, namely the split example
% from above, a hash variant using |\ekvcValue| and a hash variant using
% |\ekvcValueFast|.
% \begin{lstlisting}
% \usepackage{expkv-def}
% \ekvdefinekeys{keys}
%   {%
%     ,store   name         = \KEYSname
%     ,initial name         = John Doe
%     ,store   age          = \KEYSage
%     ,initial age          = any
%     ,store   nationality  = \KEYSnationality
%     ,initial nationality  = the Universe
%     ,store   hobby        = \KEYShobby
%     ,initial hobby        = to exist
%     ,store   type         = \KEYStype
%     ,initial type         = Mister
%     ,store   pronoun      = \KEYSpronoun
%     ,initial pronoun      = He
%     ,store   possessive   = \KEYSpossessive
%     ,initial possessive   = his
%   }
% \newcommand*\KEYS[1]
%   {%
%     \begingroup
%       \ekvset{keys}{#1}%
%       \KEYSname\ is a \KEYStype\ from \KEYSnationality. \KEYSpronoun\ is
%       of \KEYSage\ age and \KEYSpossessive\ hobby is \KEYShobby.%
%     \endgroup
%   }
% \end{lstlisting}
%
% The first comparison removes the typesetting part from all the definitions, so
% that only the key parsing is compared. In this comparison the |\ekvcValue|
% and |\ekvcValueFast| variants will not differ, as they are exactly the same
% until the key usage. We find that the split approach is $1.4$ times slower
% than the \expkvd\ setup and the hash variants end up in the middle at $1.17$
% times slower.
%
% Next we put the typesetting part back in. Every call of the macros will
% typeset the sentences into a box register in horizontal mode. With the
% typesetting part (which includes the accessing of values) the fastest remains
% the \expkvd\ definitions, but split is close at $1.16$ times slower, followed
% by the hash variant with fast accesses at $1.36$ times slower, and the safe
% hash access variant ranks in the slowest $1.8$ times slower than \expkvd.
%
% Just in case you're wondering now, a simple macro taking seven arguments is
% $30$ to $40$ times faster than any of those in the argument grabbing and \kv\
% parsing part and only $1.5$ to $2.8$ times faster if the typesetting part is
% factored in. So the real choke isn't the parsing.
%
% So to summarize this, if you have a reasonable use case for expandable \kv\
% parsing macros you should go on and define them using \expkvc. If you have a
% reasonable use case for \kv\ parsing macros but defining them expandable isn't
% necessary for your use you should take advantage of the greater flexibility of
% non-expandable \kv\ setups (but if you're after maximum speed there aren't
% that many \kv\ parsers beating \expkvc). And if you are after maximum
% performance maybe ditching the \kv\ interface altogether is a good idea, but
% depending on the number of arguments your interface might get convoluted.
%
%
% \subsection{Useless Macros}
%
% Perhaps these macros aren't completely useless, but I figured from a user's
% point of view I wouldn't know what I should do with these.
%
% \begin{function}{\ekvcDate,\ekvcVersion}
%   These two macros store the version and the date of the package/generic code.
% \end{function}
%
%
% \subsection{Bugs}
%
% Of course I don't think there are any bugs (who would knowingly distribute
% buggy software as long as he isn't a multi-million dollar corporation?). But
% if you find some please let me know. For this one might find my email address
% on the first page or file an issue on Github:
% \url{https://github.com/Skillmon/tex_expkv-cs}
%
%
% \subsection{License}
%
% Copyright \textcopyright\ 2020\unless\ifnum\year=2020--\the\year\fi\
% Jonathan P. Spratte
%
% \medskip\noindent
% This work may be distributed and/or modified under the conditions of the
% \LaTeX\ Project Public License (LPPL), either version 1.3c of this license or
% (at your option) any later version. The latest version of this license is in
% the file:
%
% \url{http://www.latex-project.org/lppl.txt}
%
% \noindent
% This work is ``maintained'' (as per LPPL maintenance status) by
%
% \mbox{Jonathan P. Spratte}.
%
% \end{documentation}^^A=<<
%
% \begin{implementation}^^A>>=
%
% \clearpage
%
% \section{Implementation}^^A>>=
%^^A the LaTeX package >>=
% \subsection{The \LaTeX\ Package}
% Just like for \expkv\ we provide a small \LaTeX\ package that sets up things
% such that we behave nicely on \LaTeX\ packages and files system. It'll
% |\input| the generic code which implements the functionality.
% \gobbledocstriptag
%<*pkg>
%    \begin{macrocode}
\RequirePackage{expkv}
\def\ekvc@tmp
  {%
    \ProvidesFile{expkv-cs.tex}%
      [%
        \ekvcDate\space v\ekvcVersion\space
        define expandable key=val macros using expkv%
      ]%
  }
\input{expkv-cs.tex}
\ProvidesPackage{expkv-cs}%
  [%
    \ekvcDate\space v\ekvcVersion\space
    define expandable key=val macros using expkv%
  ]
%    \end{macrocode}
% \gobbledocstriptag
%</pkg>
%^^A=<<
%^^A main file >>=
% \subsection{The Generic Code}
% The rest of this implementation will be the generic code.
% \gobbledocstriptag
%<*tex>
%
% Load \expkv\ if the package didn't already do so -- since \expkv\ has
% safeguards against being loaded twice this does no harm and the overhead
% isn't that big. Also we reuse some of the internals of \expkv\ to save us from
% retyping them.
%    \begin{macrocode}
\input expkv
%    \end{macrocode}
%
% We make sure that \file{expkv-cs.tex} is only input once:
%    \begin{macrocode}
\expandafter\ifx\csname ekvcVersion\endcsname\relax
\else
  \expandafter\endinput
\fi
%    \end{macrocode}
%
% \begin{macro}{\ekvcVersion,\ekvcDate}
% We're on our first input, so lets store the version and date in a macro.
%    \begin{macrocode}
\def\ekvcVersion{0.10}
\def\ekvcDate{2021-06-03}
%    \end{macrocode}
% \end{macro}
%
% If the \LaTeX\ format is loaded we want to be a good file and report back who
% we are, for this the package will have defined |\ekvc@tmp| to use
% |\ProvidesFile|, else this will expand to a |\relax| and do no harm.
%    \begin{macrocode}
\csname ekvc@tmp\endcsname
%    \end{macrocode}
%
% Store the category code of |@| to later be able to reset it and change it to
% 11 for now.
%    \begin{macrocode}
\expandafter\chardef\csname ekvc@tmp\endcsname=\catcode`\@
\catcode`\@=11
%    \end{macrocode}
% |\ekvc@tmp| will be reused later, but we don't need it to ever store
% information long-term after \expkvc\ was initialized.
%
% \begin{macro}[internal]{\ekvc@keycount}
%   We'll need to keep count how many keys must be defined for each macro in the
%   |split| variants.
%    \begin{macrocode}
\newcount\ekvc@keycount
%    \end{macrocode}
% \end{macro}
%
% \begin{macro}[internal]{\ekvc@long,\ekvc@any@long}
%   Some macros will have to be defined long. These two will be let to |\long|
%   when this should be the case.
%    \begin{macrocode}
\let\ekvc@long\ekv@empty
\let\ekvc@any@long\ekv@empty
%    \end{macrocode}
% \end{macro}
%
% \begin{macro}[internal]{\ekvc@ifdefined}
%   We want to test whether a macro is already defined. This test checks for a
%   defined macro that isn't |\relax|.
%    \begin{macrocode}
\long\def\ekvc@ifdefined#1%
  {%
    \ifdefined#1%
      \ifx\relax#1%
        \ekv@fi@gobble
      \fi
      \@firstofone
      \ekv@fi@firstoftwo
    \fi
    \@secondoftwo
  }
%    \end{macrocode}
% \end{macro}
%
% \begin{macro}[internal]
%   {
%     \ekvc@ekvset@pre@expander,
%     \ekvc@ekvset@pre@expander@a,
%     \ekvc@ekvset@pre@expander@b
%   }
%   This macro expands |\ekvset| twice so that the first two steps of expansion
%   don't have to be made every time the \expkvc\ macros are used. We have to do
%   a little magic trick to get the macro parameter |#1| for the macro
%   definition this is used in, even though we're calling |\unexpanded|. We do
%   that by splitting the expanded |\ekvset| at some marks and place |##1| in
%   between. At this spot we also add |\ekv@alignsafe| and |\ekv@endalignsafe|
%   to ensure that macros created with \expkvc\ are alignment safe.
%    \begin{macrocode}
\def\ekvc@ekvset@pre@expander#1%
  {%
    \expandafter\ekvc@ekvset@pre@expander@a\ekvset{#1}\ekvc@stop\ekvc@stop
  }
\def\ekvc@ekvset@pre@expander@a
  {%
    \expandafter\ekvc@ekvset@pre@expander@b
  }
\def\ekvc@ekvset@pre@expander@b#1\ekvc@stop#2\ekvc@stop
  {%
    \unexpanded\expandafter{\ekv@alignsafe}%
    \unexpanded{#1}##1\unexpanded{#2}%
    \unexpanded\expandafter{\ekv@endalignsafe}%
  }
%    \end{macrocode}
% \end{macro}
%
% \begin{macro}{\ekvcSplitAndUse}
%   The first user macro we want to set up can be reused for
%   \cs[no-index]{ekvcSplitAndForward} and \cs[no-index]{ekvcSplit}. We'll split
%   this one up so that the test whether the macro is already defined doesn't
%   run twice.
%    \begin{macrocode}
\protected\long\def\ekvcSplitAndUse#1#2%
  {%
    \let\ekvc@helpers@needed\@firstoftwo
    \ekvc@ifdefined#1%
      {\ekvc@err@already@defined#1}%
      {\ekvcSplitAndUse@#1{}{#2}}%
  }
%    \end{macrocode}
% \end{macro}
%
% \begin{macro}[internal]{\ekvcSplitAndUse@}
%   The actual macro setting up things. We need to set some variables, forward
%   the key list to |\ekvc@SetupSplitKeys|, and afterwards define the front
%   facing macro to call |\ekvset| and put the initials and the argument sorting
%   macro behind it. The internals |\ekvc@any@long|, |\ekvc@initials| and
%   |\ekvc@keycount| will be set correctly by |\ekvc@SetupSplitKeys|.
%    \begin{macrocode}
\protected\long\def\ekvcSplitAndUse@#1#2#3%
  {%
    \edef\ekvc@set{\string#1}%
    \ekvc@SetupSplitKeys{#3}%
    \ekvc@helpers@needed
      {%
        \ekvc@any@long\edef#1##1%
          {%
            \expandafter\ekvc@ekvset@pre@expander\expandafter{\ekvc@set}%
            \unexpanded\expandafter
              {\csname ekvc@split@\the\ekvc@keycount\endcsname}%
            \unexpanded\expandafter{\ekvc@initials{}#2}%
          }%
      }%
      {%
        \ekvc@any@long\edef#1##1%
          {%
            \expandafter\ekvc@ekvset@pre@expander\expandafter{\ekvc@set}%
            \unexpanded{#2}%
            \unexpanded\expandafter{\ekvc@initials}%
          }%
      }%
  }
%    \end{macrocode}
% \end{macro}
%
%
% \begin{macro}{\ekvcSplitAndForward}
%   This just reuses |\ekvcSplitAndUse@| with a non-empty second argument,
%   resulting in that argument to be called after the splitting.
%    \begin{macrocode}
\protected\long\def\ekvcSplitAndForward#1#2#3%
  {%
    \let\ekvc@helpers@needed\@firstoftwo
    \ekvc@ifdefined#1%
      {\ekvc@err@already@defined#1}%
      {\ekvcSplitAndUse@#1{{#2}}{#3}}%
  }
%    \end{macrocode}
% \end{macro}
%
% \begin{macro}{\ekvcSplit}
%   The first half is just |\ekvcSplitAndForward| then we define the macro to
%   which the parsed key list is forwarded. There we need to allow for up to
%   nine arguments.
%    \begin{macrocode}
\protected\long\def\ekvcSplit#1#2#3%
  {%
    \let\ekvc@helpers@needed\@secondoftwo
    \ekvc@ifdefined#1%
      {\ekvc@err@already@defined#1}%
      {%
        \expandafter
        \ekvcSplitAndUse@\expandafter#1\csname ekvc@\string#1\endcsname{#2}%
        \ifnum\ekvc@keycount<1
          \ekvc@any@long\expandafter\def\csname ekvc@\string#1\endcsname{#3}%
        \else
          \ifnum\ekvc@keycount>9
            \ekvc@err@toomany{#1}%
            \let#1\ekvc@undefined
          \else
            \ekvcSplit@build@argspec
            \ekvc@any@long\expandafter
            \def\csname ekvc@\string#1\expandafter\endcsname\ekvc@tmp{#3}%
          \fi
        \fi
      }%
  }
%    \end{macrocode}
% \end{macro}
%
% \begin{macro}[internal]{\ekvcSplit@build@argspec,\ekvcSplit@build@argspec@}
%    \begin{macrocode}
\protected\def\ekvcSplit@build@argspec
  {%
    \begingroup
      \edef\ekvc@tmp
        {\endgroup\def\unexpanded{\ekvc@tmp}{\ekvcSplit@build@argspec@{1}}}%
    \ekvc@tmp
  }
\def\ekvcSplit@build@argspec@#1%
  {%
    \ifnum#1>\ekvc@keycount
      \ekv@fi@gobble
    \fi
    \@firstofone
      {%
        \unexpanded\expandafter{\csname ekvc@splitmark@#1\endcsname####}#1%
        \expandafter\ekvcSplit@build@argspec@\expandafter{\the\numexpr#1+1}%
      }%
  }
%    \end{macrocode}
% \end{macro}
%
% \begin{macro}[internal]
%   {
%     \ekvc@SetupSplitKeys, \ekvc@SetupSplitKeys@a, \ekvc@SetupSplitKeys@b,
%     \ekvc@SetupSplitKeys@c
%   }
%   These macros parse the list of keys and set up the key macros. First we need
%   to initialise some macros and start |\ekvparse|.
%    \begin{macrocode}
\protected\long\def\ekvc@SetupSplitKeys
  {%
    \ekvc@keycount=0
    \let\ekvc@any@long\ekv@empty
    \let\ekvc@initials\ekv@empty
    \ekvparse\ekvc@err@value@required\ekvc@SetupSplitKeys@a
  }
%    \end{macrocode}
%   Then we need to step the key counter for each key. Also we have to check
%   whether this key has a |long| prefix so we initialise |\ekvc@long|.
%    \begin{macrocode}
\protected\def\ekvc@SetupSplitKeys@a#1%
  {%
    \advance\ekvc@keycount1
    \let\ekvc@long\ekv@empty
    \ekvc@ifspace{#1}%
      {\ekvc@SetupSplitKeys@b#1\ekvc@stop}%
      {\ekvc@SetupSplitKeys@c{#1}}%
  }
%    \end{macrocode}
%   If there was a space, there might be a prefix. If so call the prefix macro,
%   else call the next step |\ekvc@SetupSplitKeys@c| which will define the key
%   macro and add the key's value to the initials list.
%    \begin{macrocode}
\protected\def\ekvc@SetupSplitKeys@b#1 #2\ekvc@stop
  {%
    \ekv@ifdefined{ekvc@split@p@#1}%
      {\csname ekvc@split@p@#1\endcsname{#2}}%
      {\ekvc@SetupSplitKeys@c{#1 #2}}%
  }
%    \end{macrocode}
%   The inner definition is grouped, because we don't want to actually define
%   the marks we build with |\csname|. We have to append the value to the
%   |\ekvc@initials| list here with the correct split mark. The key macro will
%   read everything up to those split marks and change the value following it to
%   the value given to the key. Additionally we'll need a sorting macro for each
%   key count in use so we set it up with |\ekvc@setup@splitmacro|.
%    \begin{macrocode}
\protected\long\def\ekvc@SetupSplitKeys@c#1#2%
  {%
    \begingroup
    \edef\ekvc@tmp
      {%
        \endgroup
        \long\def\unexpanded{\ekvc@tmp}####1####2%
            \unexpanded\expandafter
            {\csname ekvc@splitmark@\the\ekvc@keycount\endcsname}####3%
          {%
            ####2%
            \unexpanded\expandafter
              {\csname ekvc@splitmark@\the\ekvc@keycount\endcsname}{####1}%
          }%
%    \end{macrocode}
%   The short variant needs a bit of special treatment. The key macro will be
%   short to throw the correct error, but since there might be long macros
%   somewhere the reordering of arguments needs to be long, so for short keys we
%   use a two step approach, first grabbing only the short argument, then
%   reordering.
%    \begin{macrocode}
        \unless\ifx\ekvc@long\long
          \let\unexpanded\expandafter
            {\csname ekvc@\ekvc@set(#1)\endcsname\ekvc@tmp}%
          \def\unexpanded{\ekvc@tmp}####1%
            {%
              \unexpanded\expandafter{\csname ekvc@\ekvc@set(#1)\endcsname}%
                {####1}%
            }%
        \fi
        \def\unexpanded{\ekvc@initials}%
          {%
            \unexpanded\expandafter{\ekvc@initials}%
            \unexpanded\expandafter
              {\csname ekvc@splitmark@\the\ekvc@keycount\endcsname{#2}}%
          }%
      }%
    \ekvc@tmp
    \ekvlet\ekvc@set{#1}\ekvc@tmp
    \ekvc@helpers@needed
      {\expandafter\ekvc@setup@splitmacro\expandafter{\the\ekvc@keycount}}%
      {}%
  }
%    \end{macrocode}
% \end{macro}
%
% \begin{macro}[internal]{\ekvc@split@p@long}
%   The |long| prefix lets the internals |\ekvc@long| and |\ekvc@any@long| to
%   |\long| so that the key macro will be long.
%    \begin{macrocode}
\protected\def\ekvc@split@p@long
  {%
    \let\ekvc@long\long
    \let\ekvc@any@long\long
    \ekvc@SetupSplitKeys@c
  }
%    \end{macrocode}
% \end{macro}
%
% \begin{macro}[internal]{\ekvc@defarggobbler}
%   This is needed to define a macro with 1-9 parameters programmatically.
%   \LaTeX's \cs[no-index]{newcommand} does something similar for example.
%    \begin{macrocode}
\protected\def\ekvc@defarggobbler#1{\def\ekvc@tmp##1#1##2##{##1#1}}
%    \end{macrocode}
% \end{macro}
%
% \begin{macro}[internal]
%   {
%     \ekvc@setup@splitmacro,
%     \ekvc@split@1, \ekvc@split@2, \ekvc@split@3,
%     \ekvc@split@4, \ekvc@split@5, \ekvc@split@6,
%     \ekvc@split@7
%   }
%   Since the first few split macros are different from the others we manually
%   set those up now. All the others will be defined as needed (always
%   globally).  The split macros just read up until the correct split mark, move
%   that argument into a list and reinsert the rest, calling the next split
%   macro afterwards.
%    \begin{macrocode}
\begingroup
\edef\ekvc@tmp
  {%
    \long\gdef\unexpanded\expandafter{\csname ekvc@split@1\endcsname}%
        \unexpanded\expandafter{\csname ekvc@splitmark@1\endcsname}%
        ##1##2##3%
      {##3{##1}##2}%
    \long\gdef\unexpanded\expandafter{\csname ekvc@split@2\endcsname}%
        \unexpanded\expandafter{\csname ekvc@splitmark@1\endcsname}##1%
        \unexpanded\expandafter{\csname ekvc@splitmark@2\endcsname}##2%
        ##3##4%
      {##4{##1}{##2}##3}%
    \long\gdef\unexpanded\expandafter{\csname ekvc@split@3\endcsname}%
        \unexpanded\expandafter{\csname ekvc@splitmark@1\endcsname}##1%
        \unexpanded\expandafter{\csname ekvc@splitmark@2\endcsname}##2%
        \unexpanded\expandafter{\csname ekvc@splitmark@3\endcsname}##3%
        ##4##5%
      {##5{##1}{##2}{##3}##4}%
    \long\gdef\unexpanded\expandafter{\csname ekvc@split@4\endcsname}%
        \unexpanded\expandafter{\csname ekvc@splitmark@1\endcsname}##1%
        \unexpanded\expandafter{\csname ekvc@splitmark@2\endcsname}##2%
        \unexpanded\expandafter{\csname ekvc@splitmark@3\endcsname}##3%
        \unexpanded\expandafter{\csname ekvc@splitmark@4\endcsname}##4%
        ##5##6%
      {##6{##1}{##2}{##3}{##4}##5}%
    \long\gdef\unexpanded\expandafter{\csname ekvc@split@5\endcsname}%
        \unexpanded\expandafter{\csname ekvc@splitmark@1\endcsname}##1%
        \unexpanded\expandafter{\csname ekvc@splitmark@2\endcsname}##2%
        \unexpanded\expandafter{\csname ekvc@splitmark@3\endcsname}##3%
        \unexpanded\expandafter{\csname ekvc@splitmark@4\endcsname}##4%
        \unexpanded\expandafter{\csname ekvc@splitmark@5\endcsname}##5%
        ##6##7%
      {##7{##1}{##2}{##3}{##4}{##5}##6}%
    \long\gdef\unexpanded\expandafter{\csname ekvc@split@6\endcsname}%
        \unexpanded\expandafter{\csname ekvc@splitmark@1\endcsname}##1%
        \unexpanded\expandafter{\csname ekvc@splitmark@2\endcsname}##2%
        \unexpanded\expandafter{\csname ekvc@splitmark@3\endcsname}##3%
        \unexpanded\expandafter{\csname ekvc@splitmark@4\endcsname}##4%
        \unexpanded\expandafter{\csname ekvc@splitmark@5\endcsname}##5%
        \unexpanded\expandafter{\csname ekvc@splitmark@6\endcsname}##6%
        ##7##8%
      {##8{##1}{##2}{##3}{##4}{##5}{##6}##7}%
    \long\gdef\unexpanded\expandafter{\csname ekvc@split@7\endcsname}%
        \unexpanded\expandafter{\csname ekvc@splitmark@1\endcsname}##1%
        \unexpanded\expandafter{\csname ekvc@splitmark@2\endcsname}##2%
        \unexpanded\expandafter{\csname ekvc@splitmark@3\endcsname}##3%
        \unexpanded\expandafter{\csname ekvc@splitmark@4\endcsname}##4%
        \unexpanded\expandafter{\csname ekvc@splitmark@5\endcsname}##5%
        \unexpanded\expandafter{\csname ekvc@splitmark@6\endcsname}##6%
        \unexpanded\expandafter{\csname ekvc@splitmark@7\endcsname}##7%
        ##8##9%
      {##9{##1}{##2}{##3}{##4}{##5}{##6}{##7}##8}%
  }
\ekvc@tmp
\endgroup
\protected\def\ekvc@setup@splitmacro#1%
  {%
    \ekv@ifdefined{ekvc@split@#1}{}%
      {%
        \begingroup
          \edef\ekvc@tmp
            {%
              \long\gdef
                  \unexpanded\expandafter{\csname ekvc@split@#1\endcsname}%
                  ####1%
                  \unexpanded\expandafter{\csname ekvc@splitmark@#1\endcsname}%
                  ####2####3%
                {%
                  \unexpanded\expandafter
                    {\csname ekvc@split@\the\numexpr#1-1\relax\endcsname}%
                  ####1{{####2}####3}%
                }%
            }%
          \ekvc@tmp
        \endgroup
      }%
  }
%    \end{macrocode}
% \end{macro}
%
% \begin{macro}{\ekvcHashAndUse}
%   |\ekvcHashAndUse| works just like |\ekvcSplitAndUse|.
%    \begin{macrocode}
\protected\long\def\ekvcHashAndUse#1#2%
  {%
    \let\ekvc@helpers@needed\@firstoftwo
    \ekvc@ifdefined#1%
      {\ekvc@err@already@defined#1}%
      {\ekvcHashAndUse@#1{}{#2}}%
  }
%    \end{macrocode}
% \end{macro}
%
% \begin{macro}[internal]{\ekvcHashAndUse@}
%   This is more or less the same as |\ekvcSplitAndUse@|. Instead of an empty
%   group we place a marker after the initials, we don't use the sorting macros
%   of |split|, but instead pack all the values in one argument.
%    \begin{macrocode}
\protected\long\def\ekvcHashAndUse@#1#2#3%
  {%
    \edef\ekvc@set{\string#1}%
    \ekvc@SetupHashKeys{#3}%
    \ekvc@helpers@needed
      {%
        \ekvc@any@long\edef#1##1%
          {%
            \expandafter\ekvc@ekvset@pre@expander\expandafter{\ekvc@set}%
            \unexpanded{\ekvc@hash@pack@argument}%
            \unexpanded\expandafter{\ekvc@initials\ekvc@stop#2}%
          }%
      }%
      {%
        \ekvc@any@long\edef#1##1%
          {%
            \expandafter\ekvc@ekvset@pre@expander\expandafter{\ekvc@set}%
            \unexpanded{#2}%
            \unexpanded\expandafter{\ekvc@initials\ekvc@stop}%
          }%
      }%
  }
%    \end{macrocode}
% \end{macro}
%
% \begin{macro}{\ekvcHashAndForward}
%   |\ekvcHashAndForward| works just like |\ekvcSplitAndForward|.
%    \begin{macrocode}
\protected\long\def\ekvcHashAndForward#1#2#3%
  {%
    \let\ekvc@helpers@needed\@firstoftwo
    \ekvc@ifdefined#1%
      {\ekvc@err@already@defined#1}%
      {\ekvcHashAndUse@#1{{#2}}{#3}}%
  }
%    \end{macrocode}
% \end{macro}
%
% \begin{macro}{\ekvcHash}
%   |\ekvcHash| does the same as |\ekvcSplit|, but has the advantage of not
%   needing to count arguments, so the definition of the internal macro is a bit
%   more straight forward.
%    \begin{macrocode}
\protected\long\def\ekvcHash#1#2#3%
  {%
    \let\ekvc@helpers@needed\@secondoftwo
    \ekvc@ifdefined#1%
      {\ekvc@err@already@defined#1}%
      {%
        \expandafter
        \ekvcHashAndUse@\expandafter#1\csname ekvc@\string#1\endcsname{#2}%
        \ekvc@any@long\expandafter\def\csname ekvc@\string#1\endcsname
            ##1\ekvc@stop
          {#3}%
      }%
  }
%    \end{macrocode}
% \end{macro}
%
% \begin{macro}[internal]{\ekvc@hash@pack@argument}
%   All this macro does is pack the values into one argument and forward that to
%   the next macro.
%    \begin{macrocode}
\long\def\ekvc@hash@pack@argument#1\ekvc@stop#2{#2{#1}}
%    \end{macrocode}
% \end{macro}
%
% \begin{macro}[internal]
%   {\ekvc@SetupHashKeys,\ekvc@SetupHashKeys@a,\ekvc@SetupHashKeys@b}
%   This should look awfully familiar as well, since it's just the same as for
%   the split keys with a few other names here and there.
%    \begin{macrocode}
\protected\long\def\ekvc@SetupHashKeys#1%
  {%
    \let\ekvc@any@long\ekv@empty
    \let\ekvc@initials\ekv@empty
    \ekvparse\ekvc@err@value@required\ekvc@SetupHashKeys@a{#1}%
  }
\protected\def\ekvc@SetupHashKeys@a#1%
  {%
    \let\ekvc@long\ekv@empty
    \ekvc@ifspace{#1}%
      {\ekvc@SetupHashKeys@b#1\ekvc@stop}%
      {\ekvc@SetupHashKeys@c{#1}}%
  }
\protected\def\ekvc@SetupHashKeys@b#1 #2\ekvc@stop
  {%
    \ekv@ifdefined{ekvc@hash@p@#1}%
      {\csname ekvc@hash@p@#1\endcsname{#2}}%
      {\ekvc@SetupHashKeys@c{#1 #2}}%
  }
%    \end{macrocode}
%   Yes, even the defining macro looks awfully familiar. Instead of numbered we
%   have named marks. Still the key macros grab everything up to their
%   respective mark and reorder the arguments. The same quirk is applied for
%   short keys. And instead of the |\ekvc@setup@splitmacro| we use
%   |\ekvc@setup@hashmacro|.
%    \begin{macrocode}
\protected\long\def\ekvc@SetupHashKeys@c#1#2%
  {%
    \begingroup
    \edef\ekvc@tmp
      {%
        \endgroup
        \long\def\unexpanded{\ekvc@tmp}####1####2%
            \unexpanded\expandafter{\csname ekvc@hashmark@#1\endcsname}####3%
          {%
            ####2%
            \unexpanded\expandafter{\csname ekvc@hashmark@#1\endcsname}{####1}%
          }%
        \unless\ifx\ekvc@long\long
          \let\unexpanded\expandafter
            {\csname ekvc@\ekvc@set(#1)\endcsname\ekvc@tmp}%
          \def\unexpanded{\ekvc@tmp}####1%
            {%
              \unexpanded\expandafter{\csname ekvc@\ekvc@set(#1)\endcsname}%
                {####1}%
            }%
        \fi
        \def\unexpanded{\ekvc@initials}%
          {%
            \unexpanded\expandafter{\ekvc@initials}%
            \unexpanded\expandafter{\csname ekvc@hashmark@#1\endcsname{#2}}%
          }%
      }%
    \ekvc@tmp
    \ekvlet\ekvc@set{#1}\ekvc@tmp
    \ekvc@setup@hashmacro{#1}%
  }
%    \end{macrocode}
% \end{macro}
%
% \begin{macro}[internal]{\ekvc@hash@p@long}
%   Nothing astonishing here either.
%    \begin{macrocode}
\protected\def\ekvc@hash@p@long
  {%
    \let\ekvc@long\long
    \let\ekvc@any@long\long
    \ekvc@SetupHashKeys@c
  }
%    \end{macrocode}
% \end{macro}
%
% \begin{macro}[internal]{\ekvc@setup@hashmacro}
%   The safe hash macros will be executed inside of an |\unexpanded| expansion
%   context, so they have to insert braces for that once they are done.
%   Most of the tests which have to be executed will already be done, but we
%   have to play safe if the hash doesn't show up in the hash list. Therefore we
%   use some |\ekvc@mark|s and |\ekvc@stop| to throw errors if the hash isn't
%   found in the right place. The fast variants have an easier life and just
%   return the correct value.
%    \begin{macrocode}
\protected\def\ekvc@setup@hashmacro#1%
  {%
    \ekv@ifdefined{ekvc@fasthash@#1}{}%
      {%
        \begingroup
          \edef\ekvc@tmp
            {%
              \long\gdef
                \unexpanded\expandafter{\csname ekvc@fasthash@#1\endcsname}%
                  ####1%
                  \unexpanded\expandafter{\csname ekvc@hashmark@#1\endcsname}%
                  ####2####3\unexpanded{\ekvc@stop}%
                {####2}%
              \long\gdef
                \unexpanded\expandafter{\csname ekvc@safehash@#1\endcsname}%
                  ####1%
                {%
                  \unexpanded\expandafter{\csname ekvc@@safehash@#1\endcsname}%
                  ####1\unexpanded{\ekvc@mark}{}%
                  \unexpanded\expandafter
                    {%
                      \csname ekvc@hashmark@#1\endcsname{}%
                      \ekvc@mark{\ekvc@err@missing@hash{#1}}\ekvc@stop
                    }%
                }%
              \long\gdef
                \unexpanded\expandafter{\csname ekvc@@safehash@#1\endcsname}%
                  ####1%
                  \unexpanded\expandafter{\csname ekvc@hashmark@#1\endcsname}%
                  ####2####3\unexpanded{\ekvc@mark}####4####5%
                  \unexpanded{\ekvc@stop}%
                {%
                  ####4{####2}%
                }%
              \long\gdef\unexpanded\expandafter
                {\csname ekvc@fastsplithash@#1\endcsname}%
                  ####1%
                  \unexpanded\expandafter{\csname ekvc@hashmark@#1\endcsname}%
                  ####2####3\unexpanded{\ekvc@stop}####4%
                {%
                  ####4{####1####3}{####2}%
                }%
              \long\gdef\unexpanded\expandafter
                {\csname ekvc@safesplithash@#1\endcsname}####1%
                {%
                  \unexpanded\expandafter
                    {\csname ekvc@@safesplithash@#1\endcsname}%
                  ####1\unexpanded{\ekvc@mark\ekvc@safe@found@hash}%
                  \unexpanded\expandafter
                    {%
                      \csname ekvc@hashmark@#1\endcsname{}%
                      \ekvc@mark{\ekvc@err@missing@hash{#1}\ekvc@safe@no@hash}%
                      \ekvc@stop
                    }%
                }%
              \long\gdef\unexpanded\expandafter
                {\csname ekvc@@safesplithash@#1\endcsname}%
                  ####1%
                  \unexpanded\expandafter{\csname ekvc@hashmark@#1\endcsname}%
                  ####2####3\unexpanded{\ekvc@mark}####4####5%
                  \unexpanded{\ekvc@stop}%
                {%
                  ####4{####2}####1####3\unexpanded{\ekvc@stop}%
                }%
            }%
          \ekvc@tmp
        \endgroup
      }%
  }
%    \end{macrocode}
% \end{macro}
%
% \begin{macro}{\ekvcValue}
%   All this does is a few consistency checks on the first argument (not empty,
%   hash macro exists) and then call that hash-grabbing macro that will also
%   test whether the hash is inside of |#2| or not.
%    \begin{macrocode}
\long\def\ekvcValue#1%
  {%
    \unexpanded
    \ekv@ifdefined{ekvc@safehash@#1}%
      {\csname ekvc@safehash@#1\endcsname}%
      {\ekvc@err@unknown@hash{#1}\@firstoftwo{{}}}%
  }
%    \end{macrocode}
% \end{macro}
%
% \begin{macro}{\ekvcValueFast}
%   To be as fast as possible, this doesn't test for anything, assuming the user
%   knows best.
%    \begin{macrocode}
\long\def\ekvcValueFast#1#2{\csname ekvc@fasthash@#1\endcsname#2\ekvc@stop}
%    \end{macrocode}
% \end{macro}
%
% \begin{macro}{\ekvcValueSplit,\ekvcValueSplit@recover}
%   This splits off a single value.
%    \begin{macrocode}
\long\def\ekvcValueSplit#1%
  {%
    \ekv@ifdefined{ekvc@safesplithash@#1}%
      {\csname ekvc@safesplithash@#1\endcsname}%
      {\ekvc@err@unknown@hash{#1}\ekvcValueSplit@recover}%
  }
\long\def\ekvcValueSplit@recover#1#2%
  {%
    #2{#1}{}%
  }
%    \end{macrocode}
% \end{macro}
%
% \begin{macro}[internal]{\ekvc@safe@found@hash, \ekvc@safe@no@hash}
%    \begin{macrocode}
\long\def\ekvc@safe@found@hash#1#2\ekvc@stop#3%
  {%
    #3{#2}{#1}%
  }
\long\def\ekvc@safe@no@hash#1#2\ekvc@mark\ekvc@safe@found@hash\ekvc@stop#3%
  {%
    #3{#2}{}%
  }
%    \end{macrocode}
% \end{macro}
%
% \begin{macro}{\ekvcValueSplitFast}
%   Again a fast approach which doesn't provide too many safety measurements.
%   This needs to build the hash function and expand it before passing the
%   results to the next control sequence. The first step only builds the control
%   sequence.
%    \begin{macrocode}
\long\def\ekvcValueSplitFast#1#2%
  {%
    \csname ekvc@fastsplithash@#1\endcsname#2\ekvc@stop
  }
%    \end{macrocode}
% \end{macro}
%
% \begin{macro}[internal]
%   {
%     \ekvc@safehash@,\ekvc@fasthash@,
%     \ekvc@safesplithash@,\ekvc@fastsplithash@
%   }
%   At least in the empty hash case we can provide a meaningful error message
%   without affecting performance by just defining the macro that would be build
%   in that case. There is of course a downside to this, the error will not be
%   thrown by |\ekvcValueFast| in three expansion steps. The safe hash variant
%   has to also stop the |\unexpanded| expansion.
%    \begin{macrocode}
\long\def\ekvc@safehash@#1{\ekvc@err@empty@hash{}}
\long\def\ekvc@fasthash@#1\ekvc@stop{\ekvc@err@empty@hash}
\long\def\ekvc@safesplithash@#1#2{\ekvc@err@empty@hash#2{#1}{}}
\long\def\ekvc@fastsplithash@#1\ekvc@stop#2{\ekvc@err@empty@hash#2{#1}{}}
%    \end{macrocode}
% \end{macro}
%
% \begin{macro}{\ekvcSecondaryKeys}
%   The secondary keys are defined pretty similar to the way the originals are,
%   but here we also introduce some key types (those have a |@t@| in their name)
%   additionally to the prefixes.
%    \begin{macrocode}
\protected\long\def\ekvcSecondaryKeys#1#2%
  {%
    \edef\ekvc@set{\string#1}%
    \ekvparse\ekvc@err@value@required\ekvcSecondaryKeys@a{#2}%
  }
\protected\def\ekvcSecondaryKeys@a#1%
  {%
    \let\ekvc@long\ekv@empty
    \ekvc@ifspace{#1}%
      {\ekvcSecondaryKeys@b#1\ekvc@stop}%
      {\ekvc@err@missing@type{#1}\@gobble}%
  }
\protected\def\ekvcSecondaryKeys@b#1 #2\ekvc@stop
  {%
    \ekv@ifdefined{ekvc@p@#1}%
      {\csname ekvc@p@#1\endcsname}%
      {%
        \ekv@ifdefined{ekvc@t@#1}%
          {\csname ekvc@t@#1\endcsname}%
          {\ekvc@err@unknown@keytype{#1}\@firstoftwo\@gobble}%
      }%
      {#2}%
  }
%    \end{macrocode}
% \end{macro}
%
% \begin{macro}{\ekvcChange}
%   This can be used to change the defaults of an \expkvc\ defined macro. It
%   checks whether there is a set with the correct name and that the macro is
%   defined. If both is true the real work is done by |\ekvc@change|.
%    \begin{macrocode}
\protected\long\def\ekvcChange#1%
  {%
    \ekvifdefinedset{\string#1}%
      {%
        \ekvc@ifdefined#1%
          {\ekvc@change#1}%
          {\ekvc@err@no@key@macro#1\@gobble}%
      }%
      {\ekvc@err@no@key@macro#1\@gobble}%
  }
%    \end{macrocode}
% \end{macro}
%
% \begin{macro}[internal]
%   {\ekvc@change,\ekvc@change@a,\ekvc@change@b,\ekvc@change@c}
%   First we need to see whether the macro is currently |\long|. For this we get
%   the meaning and will parse it. |#1| is the macro name in which we want to
%   change the defaults.
%    \begin{macrocode}
\protected\def\ekvc@change#1%
  {\expandafter\ekvc@change@a\meaning#1\ekv@stop#1}
%    \end{macrocode}
%   A temporary definition to get the stringified |macro:|. |##1| will be the
%   list of prefixes, we don't care for the exact contents of |##2| and |##3|.
%    \begin{macrocode}
\def\ekvc@change@a#1%
  {%
    \protected\def\ekvc@change@a##1#1##2->##3\ekv@stop
      {%
        \ekvc@change@iflong{##1}%
          {\ekvc@change@b{}}%
          {\ekvc@change@b{\long}}%
      }%
  }
\expandafter\ekvc@change@a\expandafter{\detokenize{macro:}}
%    \end{macrocode}
%   Next we expand the macro once to get its contents (including the current
%   default values with their markers). |#1| is either |\long| or empty, |#2| is
%   the macro.
%    \begin{macrocode}
\protected\def\ekvc@change@b#1#2%
  {\expandafter\ekvc@change@c\expandafter{#2{##1}}{#1}#2}
%    \end{macrocode}
%   Here we place an unbalanced closing brace after the expansion of the macro.
%   Then we just parse the \kv-list with |\ekvset|, that will exchange the
%   values behind the markers. Once those are changed we give control to
%   |\ekvc@change@d|. The |\ekvset| step might horribly fail if the user defined
%   some keys that don't behave nice. |#1| is the expansion of the macro, |#2|
%   is either |\long| or empty, |#3| is the macro, and |#4| is the \kv-list
%   containing the new defaults.
%    \begin{macrocode}
\ekv@exparg{\protected\long\def\ekvc@change@c#1#2#3#4}%
  {%
    \expandafter\iffalse\expandafter{\expandafter\fi
      \ekvset{\string#3}{#4}%
      \ekvc@change@d{#2}{#3}%
      #1%
    }%
  }
%    \end{macrocode}
%   The final step needs to put an unbalanced opening brace after |\def|. We do
%   that with the help of a temporary macro which stores everything necessary
%   for |\def| and expand an |\iffalse}\fi| construct to leave a single
%   opening brace. |#1| will be either empty or |\long| and |#2| is the macro.
%   Each of the macros defined with \expkvc\ takes exactly one parameter, so we
%   put that here as |####1|.
%    \begin{macrocode}
\protected\def\ekvc@change@d#1#2%
  {%
    \def\ekvc@tmp{#1\def#2####1}%
    \expandafter\ekvc@tmp\expandafter{\iffalse}\fi
  }
%    \end{macrocode}
% \end{macro}
%
% \begin{macro}[internal]{\ekvc@change@iflong,\ekvc@change@iflong@}
%   Checking whether a string contains the string representation of |\long| can
%   be done by gobbling everything up to the first |\long| and checking whether
%   the result is completely empty. We need a temporary macro to get the result
%   of |\string\long| inside the definitions.
%    \begin{macrocode}
\def\ekvc@change@iflong#1%
  {%
    \protected\def\ekvc@change@iflong##1%
      {\expandafter\ekv@ifempty\expandafter{\ekvc@change@iflong@##1#1}}%
    \def\ekvc@change@iflong@##1#1{}
  }
\expandafter\ekvc@change@iflong\expandafter{\string\long}
%    \end{macrocode}
% \end{macro}
%
% \begin{macro}{\ekvcPass}
%   This macro can be used to pass a value to a key of some macro (this way
%   more complicated key codes are possible that in the end pass processed
%   values on to some macro). The implemantation is pretty straight forward.
%    \begin{macrocode}
\long\def\ekvcPass#1#2%
  {%
    \ekvifdefined{\string#1}{#2}%
      {\csname\ekv@name{\string#1}{#2}\endcsname}%
      {\ekvc@err@unknown@key@or@macro{#1}{#2}\@gobble}%
  }
%    \end{macrocode}
% \end{macro}
%
%
% \subsubsection{Secondary Key Types}
%
% \begin{macro}[internal]{\ekvc@p@long,\ekvc@after@ptype}
%   The prefixes are pretty straight forward again. Just set |\ekvc@long| and
%   forward to the |@t@| type.
%    \begin{macrocode}
\protected\def\ekvc@p@long#1%
  {%
    \ekvc@ifspace{#1}%
      {%
        \let\ekvc@long\long
        \ekvc@after@ptype#1\ekvc@stop
      }%
      {\ekvc@err@missing@type{long #1}\@gobble}%
  }
\protected\def\ekvc@after@ptype#1 #2\ekvc@stop
  {%
    \ekv@ifdefined{ekvc@t@#1}%
      {\csname ekvc@t@#1\endcsname{#2}}%
      {\ekvc@err@unknown@keytype{#1}\@gobble}%
  }
%    \end{macrocode}
% \end{macro}
%
% \begin{macro}[internal]
%   {
%     \ekvc@t@meta, \ekvc@t@nmeta,
%     \ekvc@type@meta, \ekvc@type@meta@a, \ekvc@type@meta@b
%   }
%   The |meta| and |nmeta| key types use a nested |\ekvset| to set other keys in
%   the same macro's \set.
%    \begin{macrocode}
\protected\def\ekvc@t@meta
  {%
    \edef\ekvc@tmp{\ekvc@set}%
    \expandafter\ekvc@type@meta\expandafter{\ekvc@tmp}\ekvc@long{##1}\ekvlet
  }
\protected\def\ekvc@t@nmeta#1%
  {%
    \ekvc@assert@not@long{nmeta #1}%
    \edef\ekvc@tmp{\ekvc@set}%
    \expandafter\ekvc@type@meta\expandafter{\ekvc@tmp}{}{}\ekvletNoVal{#1}%
  }
\protected\long\def\ekvc@type@meta#1#2#3#4#5#6%
  {%
    \expandafter\ekvc@type@meta@a\expandafter{\ekvset{#1}{#6}}{#2}{#3}%
    #4\ekvc@set{#5}\ekvc@tmp
  }
\protected\def\ekvc@type@meta@a
  {%
    \expandafter\ekvc@type@meta@b\expandafter
  }
\protected\long\def\ekvc@type@meta@b#1#2#3%
  {%
    #2\def\ekvc@tmp#3{#1}%
  }
%    \end{macrocode}
% \end{macro}
%
% \begin{macro}[internal]{\ekvc@t@alias}
%   |alias| just checks whether there is a key and/or |NoVal| key defined with
%   the target name and |\let| the key to those.
%    \begin{macrocode}
\protected\def\ekvc@t@alias#1#2%
  {%
    \ekvc@assert@not@long{alias #1}%
    \let\ekvc@tmp\@firstofone
    \ekvifdefined\ekvc@set{#2}%
      {%
        \ekvletkv\ekvc@set{#1}\ekvc@set{#2}%
        \let\ekvc@tmp\@gobble
      }%
      {}%
    \ekvifdefinedNoVal\ekvc@set{#2}%
      {%
        \ekvletkvNoVal\ekvc@set{#1}\ekvc@set{#2}%
        \let\ekvc@tmp\@gobble
      }%
      {}%
    \ekvc@tmp{\ekvc@err@unknown@key{#2}}%
  }
%    \end{macrocode}
% \end{macro}
%
% \begin{macro}[internal]{\ekvc@t@default}
%   The |default| key can be used to set a |NoVal| key for an existing key. It
%   will just pass the \val\ to the key macro of that other key.
%    \begin{macrocode}
\protected\long\def\ekvc@t@default#1#2%
  {%
    \ekvifdefined\ekvc@set{#1}%
      {%
        \ekvc@assert@not@long{default #1}%
        \edef\ekvc@tmp
          {%
            \unexpanded\expandafter
              {\csname\ekv@name\ekvc@set{#1}\endcsname{#2}}%
          }%
        \ekvletNoVal\ekvc@set{#1}\ekvc@tmp
      }%
      {\ekvc@err@unknown@key{#1}}%
  }
%    \end{macrocode}
% \end{macro}
%
% \begin{macro}[internal]{\ekvc@t@flag-bool}
%    \begin{macrocode}
\protected\expandafter\def\csname ekvc@t@flag-bool\endcsname#1#2%
  {%
    \ekvc@assert@not@long{flag-bool #1}%
    \unless\ifdefined#2\ekvcFlagNew#2\fi
    \ekvdef\ekvc@set{#1}%
      {%
        \ekv@ifdefined{ekvc@flag@set@##1}%
          {%
            \csname ekvc@flag@set@##1\expandafter\endcsname
              \ekvcFlagHeight#2\ekv@stop#2%
          }%
          {\ekvc@err@invalid@bool{##1}}%
      }%
  }
%    \end{macrocode}
% \end{macro}
%
% \begin{macro}[internal]
%   {\ekvc@t@flag-true,\ekvc@t@flag-false,\ekvc@t@flag-raise,\ekvc@type@flag}
%    \begin{macrocode}
\protected\def\ekvc@type@flag#1#2#3#4%
  {%
    \ekvc@assert@not@long{flag-#1 #3}%
    \unless\ifdefined#4\ekvcFlagNew#4\fi
    \ekv@exparg{\ekvdefNoVal\ekvc@set{#3}}{#2#4}%
  }
\protected\expandafter\def\csname ekvc@t@flag-true\endcsname
  {\ekvc@type@flag{true}\ekvcFlagSetTrue}
\protected\expandafter\def\csname ekvc@t@flag-false\endcsname
  {\ekvc@type@flag{false}\ekvcFlagSetFalse}
\protected\expandafter\def\csname ekvc@t@flag-raise\endcsname
  {\ekvc@type@flag{raise}\ekvcFlagRaise}
%    \end{macrocode}
% \end{macro}
%
%
% \subsubsection{Flags}
%
% The basic idea of flags is to store information by the fact that \TeX\
% expandably assigns the meaning |\relax| to undefined control sequences which
% were built with |\csname|. This mechanism is borrowed from \pkg{expl3}.
%
% \begin{macro}[internal]{\ekvc@flag@name,\ekvc@flag@namescheme}
%   Flags follow a simple naming scheme which we define here. |\ekvc@flag@name|
%   will store the name of an internal function that is used to build names of
%   the second naming scheme defined by |\ekvc@flag@namescheme|.
%    \begin{macrocode}
\def\ekvc@flag@name{ekvcf\string}
\def\ekvc@flag@namescheme#1#2{ekvch#2#1}
%    \end{macrocode}
% \end{macro}
%
% \begin{macro}{\ekvcFlagHeight}
%   For semantic reasons we use |\number| with another name.
%    \begin{macrocode}
\let\ekvcFlagHeight\number
%    \end{macrocode}
% \end{macro}
%
% \begin{macro}{\ekvcFlagNew}
%   This macro defines a new flag. It stores the function build with the
%   |\ekvc@flag@name| naming scheme after the internal function
%   |\ekvc@flag@height| that'll determine the current flag height. It'll also
%   define the macro named via |\ekvc@flag@name| to build names according to
%   |\ekvc@flag@namescheme|.
%    \begin{macrocode}
\protected\def\ekvcFlagNew#1%
  {%
    \edef#1%
      {%
        \unexpanded{\ekvc@flag@height}%
          \unexpanded\expandafter{\csname\ekvc@flag@name#1\endcsname}%
      }%
    \ekv@expargtwice
      {\expandafter\def\csname\ekvc@flag@name#1\endcsname##1}%
      {\expandafter\ekvc@flag@namescheme\expandafter{\string#1}{##1}}%
  }
%    \end{macrocode}
% \end{macro}
%
% \begin{macro}[internal]{\ekvc@flag@height,\ekvc@flag@height@}
%   This macro gets the height of a flag by a simple loop. The first loop
%   iteration differs a bit from the following in that it doesn't have to get
%   the current iteration count. The space at the end of |\ekvc@flag@height|
%   ends the |\number| evaluation.
%    \begin{macrocode}
\def\ekvc@flag@height#1%
  {%
    \ifcsname#10\endcsname
      \ekvc@flag@height@1\ekv@stop#1%
    \fi
    \@firstofone{0} % leave this space
  }
\def\ekvc@flag@height@#1\ekv@stop#2\fi\@firstofone#3%
  {%
    \fi
    \ifcsname#2{#1}\endcsname
      \expandafter\ekvc@flag@height@\the\numexpr#1+1\relax\ekv@stop#2%
    \fi
    \@firstofone{#1}%
  }
%    \end{macrocode}
% \end{macro}
%
% \begin{macro}{\ekvcFlagRaise}
%   Raising a flag simply means letting the |\ekvc@flag@namescheme| macro for
%   the current height to relax. The result of raising a flag is that its height
%   is bigger by $1$.
%    \begin{macrocode}
\ekv@exparg{\def\ekvcFlagRaise#1}%
  {%
    \expandafter\expandafter\expandafter\@gobble\expandafter
      \csname\ekvc@flag@namescheme{\string#1}{\ekvcFlagHeight#1}\endcsname
  }
%    \end{macrocode}
% \end{macro}
%
% \begin{macro}{\ekvcFlagSetTrue,\ekvcFlagSetFalse}
% \begin{macro}[internal]{\ekvc@flag@set@true,\ekvc@flag@set@false}
%   A flag is considered true if its current height is odd, and as false if it
%   is even. Therefore |\ekvcFlagSetTrue| and |\ekvcFlagSetFalse| only need to
%   raise the flag if the opposing boolean value is the current one.
%    \begin{macrocode}
\def\ekvcFlagSetTrue#1%
  {\expandafter\ekvc@flag@set@true\ekvcFlagHeight#1\ekv@stop#1}
\def\ekvcFlagSetFalse#1%
  {\expandafter\ekvc@flag@set@false\ekvcFlagHeight#1\ekv@stop#1}
%    \end{macrocode}
%   We can expand |\ekvc@flag@namescheme| at definition time here, which is why
%   we're using a temporary definition to set up |\ekvc@flag@set@true| and
%   |\ekvc@flag@set@false|.
%    \begin{macrocode}
\def\ekvc@flag@set@true#1%
  {%
    \def\ekvc@flag@set@true##1\ekv@stop##2%
      {%
        \ifodd##1
          \ekv@fi@gobble
        \fi
        \@firstofone{\expandafter\@gobble\csname#1\endcsname}%
      }%
    \def\ekvc@flag@set@false##1\ekv@stop##2%
      {%
        \ifodd##1
          \ekv@fi@firstofone
        \fi
        \@gobble{\expandafter\@gobble\csname#1\endcsname}%
      }%
  }
\expandafter\ekvc@flag@set@true\expandafter
  {\ekvc@flag@namescheme{\string#2}{#1}}
%    \end{macrocode}
% \end{macro}
% \end{macro}
%
% \begin{macro}{\ekvcFlagIf}
%   As already explained, truthiness is defined as a flag's height is odd, so we
%   just branch accordingly here.
%    \begin{macrocode}
\def\ekvcFlagIf#1%
  {%
    \ifodd#1%
      \ekv@fi@firstoftwo
    \fi
    \@secondoftwo
  }
%    \end{macrocode}
% \end{macro}
%
% \begin{macro}{\ekvcFlagIfRaised}
%   This macro uses flags as a switch, if a flag's current height is bigger than
%   $0$ this test yields true.
%    \begin{macrocode}
\ekv@exparg{\def\ekvcFlagIfRaised#1}%
  {%
    \expandafter\ifcsname\ekvc@flag@namescheme{\string#1}0\endcsname
      \ekv@fi@firstoftwo
    \fi
    \@secondoftwo
  }
%    \end{macrocode}
% \end{macro}
%
% \begin{macro}{\ekvcFlagReset}
% \begin{macro}[internal]{\ekvc@flag@reset,\ekvc@flag@reset@}
%   Resetting works by locally letting all the defined internal macros named
%   after |\ekvc@flag@namescheme| to undefined.
%    \begin{macrocode}
\protected\def\ekvcFlagReset#1%
  {\expandafter\ekvc@flag@reset\csname\ekvc@flag@name#1\endcsname}
\protected\def\ekvc@flag@reset#1%
  {%
    \ifcsname#10\endcsname
      \expandafter\let\csname#10\endcsname\ekvc@undefined
      \ekvc@flag@reset@1\ekv@stop#1%
    \fi
  }
\protected\def\ekvc@flag@reset@#1\ekv@stop#2\fi
  {%
    \fi
    \ifcsname#2{#1}\endcsname
      \expandafter\let\csname#2{#1}\endcsname\ekvc@undefined
      \expandafter\ekvc@flag@reset@\the\numexpr#1+1\relax\ekv@stop#2%
    \fi
  }
%    \end{macrocode}
% \end{macro}
% \end{macro}
%
% \begin{macro}{\ekvcFlagGetHeight}
% \begin{macro}[internal]{\ekvc@flag@get@height@single}
%   These are just small helpers, first getting the height of the flag and then
%   passing it on to the user supplied code.
%    \begin{macrocode}
\def\ekvcFlagGetHeight#1%
  {\expandafter\ekvc@flag@get@height@single\ekvcFlagHeight#1\ekv@stop}
\long\def\ekvc@flag@get@height@single#1\ekv@stop#2{#2{#1}}
%    \end{macrocode}
% \end{macro}
% \end{macro}
%
% \begin{macro}{\ekvcFlagGetHeights}
% \begin{macro}[internal]
%   {\ekvc@flag@get@heights,\ekvc@flag@get@heights@,\ekvc@flag@get@heights@done}
%   This works by a simple loop that stops at |\ekv@stop|. As long as that
%   marker isn't hit, get the next flags height and put it into a list after
%   |\ekv@stop|. |\ekvc@flag@get@heights@| uses the same marker name for the
%   end of the height, which shouldn't clash in any case. Once we're done we
%   remove the remainder of the current iteration and leave the user supplied
%   code in the input stream with all the flags' heights as a single argument.
%    \begin{macrocode}
\def\ekvcFlagGetHeights#1%
  {%
    \ekvc@flag@get@heights#1\ekv@stop{}%
  }
\def\ekvc@flag@get@heights#1%
  {%
    \ekv@gobbleto@stop#1\ekvc@flag@get@heights@done\ekv@stop
    \expandafter\ekvc@flag@get@heights@\ekvcFlagHeight#1\ekv@stop
  }
\def\ekvc@flag@get@heights@#1\ekv@stop#2\ekv@stop#3%
  {\ekvc@flag@get@heights#2\ekv@stop{#3{#1}}}
\long\def\ekvc@flag@get@heights@done
    \ekv@stop
    \expandafter\ekvc@flag@get@heights@\ekvcFlagHeight\ekv@stop\ekv@stop#1#2%
  {#2{#1}}
%    \end{macrocode}
% \end{macro}
% \end{macro}
%
%
% \subsubsection{Helper Macros}
%
% \begin{macro}[internal]{\ekvc@ifspace,\ekvc@ifspace@}
% A test which can be reduced to an if-empty by gobbling everything up to the
% first space.
%    \begin{macrocode}
\long\def\ekvc@ifspace#1%
  {%
    \ekvc@ifspace@#1 \ekv@ifempty@B
      \ekv@ifempty@false\ekv@ifempty@A\ekv@ifempty@B\@firstoftwo
  }
\long\def\ekvc@ifspace@#1 % keep this space
  {%
    \ekv@ifempty@\ekv@ifempty@A
  }
%    \end{macrocode}
% \end{macro}
%
%
% \subsubsection{Assertions}
% \begin{macro}[internal]{\ekvc@assert@not@long}
%   Some keys don't want to be |long| and we have to educate the user, so let's
%   throw an error if someone wanted these to be long.
%    \begin{macrocode}
\long\def\ekvc@assert@not@long#1{\ifx\ekvc@long\long\ekvc@err@no@long{#1}\fi}
%    \end{macrocode}
% \end{macro}
%
%
% \subsubsection{Messages}
%
% \begin{macro}[internal]
%   {
%     \ekvc@err@toomany,\ekvc@err@value@required,\ekvc@err@missing@type,
%     \ekvc@err@already@defined,\ekvc@err@no@key@macro
%   }
%   Boring unexpandable error messages.
%    \begin{macrocode}
\protected\def\ekvc@err@toomany#1%
  {%
    \errmessage{expkv-cs Error: Too many keys for macro `\string#1'}%
  }
\protected\def\ekvc@err@value@required#1%
  {%
    \errmessage{expkv-cs Error: Missing value for key `\unexpanded{#1}'}%
  }
\protected\def\ekvc@err@missing@type#1%
  {%
    \errmessage
      {expkv-cs Error: Missing type for secondary key `\unexpanded{#1}'}%
  }
\protected\def\ekvc@err@no@long#1%
  {%
    \errmessage
      {expkv-cs Error: prefix `long' not accepted for `\unexpanded{#1}'}%
  }
\protected\def\ekvc@err@already@defined#1%
  {%
    \errmessage{expkv-cs Error: Macro `\string#1' already defined}%
  }
\protected\def\ekvc@err@unknown@keytype#1%
  {%
    \errmessage{expkv-cs Error: Unknown key type `\unexpanded{#1}'}%
  }
\protected\def\ekvc@err@unknown@key#1%
  {%
    \errmessage
      {expkv-cs Error: Unknown key `\unexpanded{#1}' for macro `\ekvc@set'}%
  }
\long\def\ekvc@err@no@key@macro#1%
  {\errmessage{expkv-cs Error: \string#1 is no key=val macro}}
%    \end{macrocode}
% \end{macro}
%
% \begin{macro}[internal]{\ekvc@err}
%   We need a way to throw error messages expandably in some contexts.
%    \begin{macrocode}
\ekv@exparg{\long\def\ekvc@err#1}{\ekverr{expkv-cs}{#1}}
%    \end{macrocode}
% \end{macro}
%
% \begin{macro}[internal]
%   {
%     \ekvc@err@unknown@hash,\ekvc@err@empty@hash,\ekvc@err@missing@hash,
%     \ekvc@err@invalid@bool
%   }
%   And here are the expandable error messages.
%    \begin{macrocode}
\long\def\ekvc@err@unknown@hash#1{\ekvc@err{unknown hash `#1'}}
\long\def\ekvc@err@missing@hash#1{\ekvc@err{hash `#1' not found}}
\long\def\ekvc@err@empty@hash{\ekvc@err{empty hash}}
\def\ekvc@err@invalid@bool#1{\ekvc@err{invalid boolean value `#1'}}
\long\def\ekvc@err@unknown@key@or@macro#1#2%
  {\ekvc@err{unknown key `#2' for macro #1}}
%    \end{macrocode}
% \end{macro}
%
%
% Now everything that's left is to reset the category code of |@|.
%    \begin{macrocode}
\catcode`\@=\ekvc@tmp
%    \end{macrocode}
%
% \gobbledocstriptag
%</tex>
%^^A=<<
%
%^^A=<<
%
% \end{implementation}^^A=<<
%
% \clearpage
% \PrintIndex
%
\endinput
%
^^A vim: ft=tex fdm=marker fmr=>>=,=<<

\ProvidesPackage{expkv-cs}%
  [%
    \ekvcDate\space v\ekvcVersion\space
    define expandable key=val macros using expkv%
  ]
%    \end{macrocode}
% \gobbledocstriptag
%</pkg>
%^^A=<<
%^^A main file >>=
% \subsection{The Generic Code}
% The rest of this implementation will be the generic code.
% \gobbledocstriptag
%<*tex>
%
% Load \expkv\ if the package didn't already do so -- since \expkv\ has
% safeguards against being loaded twice this does no harm and the overhead
% isn't that big. Also we reuse some of the internals of \expkv\ to save us from
% retyping them.
%    \begin{macrocode}
\input expkv
%    \end{macrocode}
%
% We make sure that \file{expkv-cs.tex} is only input once:
%    \begin{macrocode}
\expandafter\ifx\csname ekvcVersion\endcsname\relax
\else
  \expandafter\endinput
\fi
%    \end{macrocode}
%
% \begin{macro}{\ekvcVersion,\ekvcDate}
% We're on our first input, so lets store the version and date in a macro.
%    \begin{macrocode}
\def\ekvcVersion{0.10}
\def\ekvcDate{2021-06-03}
%    \end{macrocode}
% \end{macro}
%
% If the \LaTeX\ format is loaded we want to be a good file and report back who
% we are, for this the package will have defined |\ekvc@tmp| to use
% |\ProvidesFile|, else this will expand to a |\relax| and do no harm.
%    \begin{macrocode}
\csname ekvc@tmp\endcsname
%    \end{macrocode}
%
% Store the category code of |@| to later be able to reset it and change it to
% 11 for now.
%    \begin{macrocode}
\expandafter\chardef\csname ekvc@tmp\endcsname=\catcode`\@
\catcode`\@=11
%    \end{macrocode}
% |\ekvc@tmp| will be reused later, but we don't need it to ever store
% information long-term after \expkvc\ was initialized.
%
% \begin{macro}[internal]{\ekvc@keycount}
%   We'll need to keep count how many keys must be defined for each macro in the
%   |split| variants.
%    \begin{macrocode}
\newcount\ekvc@keycount
%    \end{macrocode}
% \end{macro}
%
% \begin{macro}[internal]{\ekvc@long,\ekvc@any@long}
%   Some macros will have to be defined long. These two will be let to |\long|
%   when this should be the case.
%    \begin{macrocode}
\let\ekvc@long\ekv@empty
\let\ekvc@any@long\ekv@empty
%    \end{macrocode}
% \end{macro}
%
% \begin{macro}[internal]{\ekvc@ifdefined}
%   We want to test whether a macro is already defined. This test checks for a
%   defined macro that isn't |\relax|.
%    \begin{macrocode}
\long\def\ekvc@ifdefined#1%
  {%
    \ifdefined#1%
      \ifx\relax#1%
        \ekv@fi@gobble
      \fi
      \@firstofone
      \ekv@fi@firstoftwo
    \fi
    \@secondoftwo
  }
%    \end{macrocode}
% \end{macro}
%
% \begin{macro}[internal]
%   {
%     \ekvc@ekvset@pre@expander,
%     \ekvc@ekvset@pre@expander@a,
%     \ekvc@ekvset@pre@expander@b
%   }
%   This macro expands |\ekvset| twice so that the first two steps of expansion
%   don't have to be made every time the \expkvc\ macros are used. We have to do
%   a little magic trick to get the macro parameter |#1| for the macro
%   definition this is used in, even though we're calling |\unexpanded|. We do
%   that by splitting the expanded |\ekvset| at some marks and place |##1| in
%   between. At this spot we also add |\ekv@alignsafe| and |\ekv@endalignsafe|
%   to ensure that macros created with \expkvc\ are alignment safe.
%    \begin{macrocode}
\def\ekvc@ekvset@pre@expander#1%
  {%
    \expandafter\ekvc@ekvset@pre@expander@a\ekvset{#1}\ekvc@stop\ekvc@stop
  }
\def\ekvc@ekvset@pre@expander@a
  {%
    \expandafter\ekvc@ekvset@pre@expander@b
  }
\def\ekvc@ekvset@pre@expander@b#1\ekvc@stop#2\ekvc@stop
  {%
    \unexpanded\expandafter{\ekv@alignsafe}%
    \unexpanded{#1}##1\unexpanded{#2}%
    \unexpanded\expandafter{\ekv@endalignsafe}%
  }
%    \end{macrocode}
% \end{macro}
%
% \begin{macro}{\ekvcSplitAndUse}
%   The first user macro we want to set up can be reused for
%   \cs[no-index]{ekvcSplitAndForward} and \cs[no-index]{ekvcSplit}. We'll split
%   this one up so that the test whether the macro is already defined doesn't
%   run twice.
%    \begin{macrocode}
\protected\long\def\ekvcSplitAndUse#1#2%
  {%
    \let\ekvc@helpers@needed\@firstoftwo
    \ekvc@ifdefined#1%
      {\ekvc@err@already@defined#1}%
      {\ekvcSplitAndUse@#1{}{#2}}%
  }
%    \end{macrocode}
% \end{macro}
%
% \begin{macro}[internal]{\ekvcSplitAndUse@}
%   The actual macro setting up things. We need to set some variables, forward
%   the key list to |\ekvc@SetupSplitKeys|, and afterwards define the front
%   facing macro to call |\ekvset| and put the initials and the argument sorting
%   macro behind it. The internals |\ekvc@any@long|, |\ekvc@initials| and
%   |\ekvc@keycount| will be set correctly by |\ekvc@SetupSplitKeys|.
%    \begin{macrocode}
\protected\long\def\ekvcSplitAndUse@#1#2#3%
  {%
    \edef\ekvc@set{\string#1}%
    \ekvc@SetupSplitKeys{#3}%
    \ekvc@helpers@needed
      {%
        \ekvc@any@long\edef#1##1%
          {%
            \expandafter\ekvc@ekvset@pre@expander\expandafter{\ekvc@set}%
            \unexpanded\expandafter
              {\csname ekvc@split@\the\ekvc@keycount\endcsname}%
            \unexpanded\expandafter{\ekvc@initials{}#2}%
          }%
      }%
      {%
        \ekvc@any@long\edef#1##1%
          {%
            \expandafter\ekvc@ekvset@pre@expander\expandafter{\ekvc@set}%
            \unexpanded{#2}%
            \unexpanded\expandafter{\ekvc@initials}%
          }%
      }%
  }
%    \end{macrocode}
% \end{macro}
%
%
% \begin{macro}{\ekvcSplitAndForward}
%   This just reuses |\ekvcSplitAndUse@| with a non-empty second argument,
%   resulting in that argument to be called after the splitting.
%    \begin{macrocode}
\protected\long\def\ekvcSplitAndForward#1#2#3%
  {%
    \let\ekvc@helpers@needed\@firstoftwo
    \ekvc@ifdefined#1%
      {\ekvc@err@already@defined#1}%
      {\ekvcSplitAndUse@#1{{#2}}{#3}}%
  }
%    \end{macrocode}
% \end{macro}
%
% \begin{macro}{\ekvcSplit}
%   The first half is just |\ekvcSplitAndForward| then we define the macro to
%   which the parsed key list is forwarded. There we need to allow for up to
%   nine arguments.
%    \begin{macrocode}
\protected\long\def\ekvcSplit#1#2#3%
  {%
    \let\ekvc@helpers@needed\@secondoftwo
    \ekvc@ifdefined#1%
      {\ekvc@err@already@defined#1}%
      {%
        \expandafter
        \ekvcSplitAndUse@\expandafter#1\csname ekvc@\string#1\endcsname{#2}%
        \ifnum\ekvc@keycount<1
          \ekvc@any@long\expandafter\def\csname ekvc@\string#1\endcsname{#3}%
        \else
          \ifnum\ekvc@keycount>9
            \ekvc@err@toomany{#1}%
            \let#1\ekvc@undefined
          \else
            \ekvcSplit@build@argspec
            \ekvc@any@long\expandafter
            \def\csname ekvc@\string#1\expandafter\endcsname\ekvc@tmp{#3}%
          \fi
        \fi
      }%
  }
%    \end{macrocode}
% \end{macro}
%
% \begin{macro}[internal]{\ekvcSplit@build@argspec,\ekvcSplit@build@argspec@}
%    \begin{macrocode}
\protected\def\ekvcSplit@build@argspec
  {%
    \begingroup
      \edef\ekvc@tmp
        {\endgroup\def\unexpanded{\ekvc@tmp}{\ekvcSplit@build@argspec@{1}}}%
    \ekvc@tmp
  }
\def\ekvcSplit@build@argspec@#1%
  {%
    \ifnum#1>\ekvc@keycount
      \ekv@fi@gobble
    \fi
    \@firstofone
      {%
        \unexpanded\expandafter{\csname ekvc@splitmark@#1\endcsname####}#1%
        \expandafter\ekvcSplit@build@argspec@\expandafter{\the\numexpr#1+1}%
      }%
  }
%    \end{macrocode}
% \end{macro}
%
% \begin{macro}[internal]
%   {
%     \ekvc@SetupSplitKeys, \ekvc@SetupSplitKeys@a, \ekvc@SetupSplitKeys@b,
%     \ekvc@SetupSplitKeys@c
%   }
%   These macros parse the list of keys and set up the key macros. First we need
%   to initialise some macros and start |\ekvparse|.
%    \begin{macrocode}
\protected\long\def\ekvc@SetupSplitKeys
  {%
    \ekvc@keycount=0
    \let\ekvc@any@long\ekv@empty
    \let\ekvc@initials\ekv@empty
    \ekvparse\ekvc@err@value@required\ekvc@SetupSplitKeys@a
  }
%    \end{macrocode}
%   Then we need to step the key counter for each key. Also we have to check
%   whether this key has a |long| prefix so we initialise |\ekvc@long|.
%    \begin{macrocode}
\protected\def\ekvc@SetupSplitKeys@a#1%
  {%
    \advance\ekvc@keycount1
    \let\ekvc@long\ekv@empty
    \ekvc@ifspace{#1}%
      {\ekvc@SetupSplitKeys@b#1\ekvc@stop}%
      {\ekvc@SetupSplitKeys@c{#1}}%
  }
%    \end{macrocode}
%   If there was a space, there might be a prefix. If so call the prefix macro,
%   else call the next step |\ekvc@SetupSplitKeys@c| which will define the key
%   macro and add the key's value to the initials list.
%    \begin{macrocode}
\protected\def\ekvc@SetupSplitKeys@b#1 #2\ekvc@stop
  {%
    \ekv@ifdefined{ekvc@split@p@#1}%
      {\csname ekvc@split@p@#1\endcsname{#2}}%
      {\ekvc@SetupSplitKeys@c{#1 #2}}%
  }
%    \end{macrocode}
%   The inner definition is grouped, because we don't want to actually define
%   the marks we build with |\csname|. We have to append the value to the
%   |\ekvc@initials| list here with the correct split mark. The key macro will
%   read everything up to those split marks and change the value following it to
%   the value given to the key. Additionally we'll need a sorting macro for each
%   key count in use so we set it up with |\ekvc@setup@splitmacro|.
%    \begin{macrocode}
\protected\long\def\ekvc@SetupSplitKeys@c#1#2%
  {%
    \begingroup
    \edef\ekvc@tmp
      {%
        \endgroup
        \long\def\unexpanded{\ekvc@tmp}####1####2%
            \unexpanded\expandafter
            {\csname ekvc@splitmark@\the\ekvc@keycount\endcsname}####3%
          {%
            ####2%
            \unexpanded\expandafter
              {\csname ekvc@splitmark@\the\ekvc@keycount\endcsname}{####1}%
          }%
%    \end{macrocode}
%   The short variant needs a bit of special treatment. The key macro will be
%   short to throw the correct error, but since there might be long macros
%   somewhere the reordering of arguments needs to be long, so for short keys we
%   use a two step approach, first grabbing only the short argument, then
%   reordering.
%    \begin{macrocode}
        \unless\ifx\ekvc@long\long
          \let\unexpanded\expandafter
            {\csname ekvc@\ekvc@set(#1)\endcsname\ekvc@tmp}%
          \def\unexpanded{\ekvc@tmp}####1%
            {%
              \unexpanded\expandafter{\csname ekvc@\ekvc@set(#1)\endcsname}%
                {####1}%
            }%
        \fi
        \def\unexpanded{\ekvc@initials}%
          {%
            \unexpanded\expandafter{\ekvc@initials}%
            \unexpanded\expandafter
              {\csname ekvc@splitmark@\the\ekvc@keycount\endcsname{#2}}%
          }%
      }%
    \ekvc@tmp
    \ekvlet\ekvc@set{#1}\ekvc@tmp
    \ekvc@helpers@needed
      {\expandafter\ekvc@setup@splitmacro\expandafter{\the\ekvc@keycount}}%
      {}%
  }
%    \end{macrocode}
% \end{macro}
%
% \begin{macro}[internal]{\ekvc@split@p@long}
%   The |long| prefix lets the internals |\ekvc@long| and |\ekvc@any@long| to
%   |\long| so that the key macro will be long.
%    \begin{macrocode}
\protected\def\ekvc@split@p@long
  {%
    \let\ekvc@long\long
    \let\ekvc@any@long\long
    \ekvc@SetupSplitKeys@c
  }
%    \end{macrocode}
% \end{macro}
%
% \begin{macro}[internal]{\ekvc@defarggobbler}
%   This is needed to define a macro with 1-9 parameters programmatically.
%   \LaTeX's \cs[no-index]{newcommand} does something similar for example.
%    \begin{macrocode}
\protected\def\ekvc@defarggobbler#1{\def\ekvc@tmp##1#1##2##{##1#1}}
%    \end{macrocode}
% \end{macro}
%
% \begin{macro}[internal]
%   {
%     \ekvc@setup@splitmacro,
%     \ekvc@split@1, \ekvc@split@2, \ekvc@split@3,
%     \ekvc@split@4, \ekvc@split@5, \ekvc@split@6,
%     \ekvc@split@7
%   }
%   Since the first few split macros are different from the others we manually
%   set those up now. All the others will be defined as needed (always
%   globally).  The split macros just read up until the correct split mark, move
%   that argument into a list and reinsert the rest, calling the next split
%   macro afterwards.
%    \begin{macrocode}
\begingroup
\edef\ekvc@tmp
  {%
    \long\gdef\unexpanded\expandafter{\csname ekvc@split@1\endcsname}%
        \unexpanded\expandafter{\csname ekvc@splitmark@1\endcsname}%
        ##1##2##3%
      {##3{##1}##2}%
    \long\gdef\unexpanded\expandafter{\csname ekvc@split@2\endcsname}%
        \unexpanded\expandafter{\csname ekvc@splitmark@1\endcsname}##1%
        \unexpanded\expandafter{\csname ekvc@splitmark@2\endcsname}##2%
        ##3##4%
      {##4{##1}{##2}##3}%
    \long\gdef\unexpanded\expandafter{\csname ekvc@split@3\endcsname}%
        \unexpanded\expandafter{\csname ekvc@splitmark@1\endcsname}##1%
        \unexpanded\expandafter{\csname ekvc@splitmark@2\endcsname}##2%
        \unexpanded\expandafter{\csname ekvc@splitmark@3\endcsname}##3%
        ##4##5%
      {##5{##1}{##2}{##3}##4}%
    \long\gdef\unexpanded\expandafter{\csname ekvc@split@4\endcsname}%
        \unexpanded\expandafter{\csname ekvc@splitmark@1\endcsname}##1%
        \unexpanded\expandafter{\csname ekvc@splitmark@2\endcsname}##2%
        \unexpanded\expandafter{\csname ekvc@splitmark@3\endcsname}##3%
        \unexpanded\expandafter{\csname ekvc@splitmark@4\endcsname}##4%
        ##5##6%
      {##6{##1}{##2}{##3}{##4}##5}%
    \long\gdef\unexpanded\expandafter{\csname ekvc@split@5\endcsname}%
        \unexpanded\expandafter{\csname ekvc@splitmark@1\endcsname}##1%
        \unexpanded\expandafter{\csname ekvc@splitmark@2\endcsname}##2%
        \unexpanded\expandafter{\csname ekvc@splitmark@3\endcsname}##3%
        \unexpanded\expandafter{\csname ekvc@splitmark@4\endcsname}##4%
        \unexpanded\expandafter{\csname ekvc@splitmark@5\endcsname}##5%
        ##6##7%
      {##7{##1}{##2}{##3}{##4}{##5}##6}%
    \long\gdef\unexpanded\expandafter{\csname ekvc@split@6\endcsname}%
        \unexpanded\expandafter{\csname ekvc@splitmark@1\endcsname}##1%
        \unexpanded\expandafter{\csname ekvc@splitmark@2\endcsname}##2%
        \unexpanded\expandafter{\csname ekvc@splitmark@3\endcsname}##3%
        \unexpanded\expandafter{\csname ekvc@splitmark@4\endcsname}##4%
        \unexpanded\expandafter{\csname ekvc@splitmark@5\endcsname}##5%
        \unexpanded\expandafter{\csname ekvc@splitmark@6\endcsname}##6%
        ##7##8%
      {##8{##1}{##2}{##3}{##4}{##5}{##6}##7}%
    \long\gdef\unexpanded\expandafter{\csname ekvc@split@7\endcsname}%
        \unexpanded\expandafter{\csname ekvc@splitmark@1\endcsname}##1%
        \unexpanded\expandafter{\csname ekvc@splitmark@2\endcsname}##2%
        \unexpanded\expandafter{\csname ekvc@splitmark@3\endcsname}##3%
        \unexpanded\expandafter{\csname ekvc@splitmark@4\endcsname}##4%
        \unexpanded\expandafter{\csname ekvc@splitmark@5\endcsname}##5%
        \unexpanded\expandafter{\csname ekvc@splitmark@6\endcsname}##6%
        \unexpanded\expandafter{\csname ekvc@splitmark@7\endcsname}##7%
        ##8##9%
      {##9{##1}{##2}{##3}{##4}{##5}{##6}{##7}##8}%
  }
\ekvc@tmp
\endgroup
\protected\def\ekvc@setup@splitmacro#1%
  {%
    \ekv@ifdefined{ekvc@split@#1}{}%
      {%
        \begingroup
          \edef\ekvc@tmp
            {%
              \long\gdef
                  \unexpanded\expandafter{\csname ekvc@split@#1\endcsname}%
                  ####1%
                  \unexpanded\expandafter{\csname ekvc@splitmark@#1\endcsname}%
                  ####2####3%
                {%
                  \unexpanded\expandafter
                    {\csname ekvc@split@\the\numexpr#1-1\relax\endcsname}%
                  ####1{{####2}####3}%
                }%
            }%
          \ekvc@tmp
        \endgroup
      }%
  }
%    \end{macrocode}
% \end{macro}
%
% \begin{macro}{\ekvcHashAndUse}
%   |\ekvcHashAndUse| works just like |\ekvcSplitAndUse|.
%    \begin{macrocode}
\protected\long\def\ekvcHashAndUse#1#2%
  {%
    \let\ekvc@helpers@needed\@firstoftwo
    \ekvc@ifdefined#1%
      {\ekvc@err@already@defined#1}%
      {\ekvcHashAndUse@#1{}{#2}}%
  }
%    \end{macrocode}
% \end{macro}
%
% \begin{macro}[internal]{\ekvcHashAndUse@}
%   This is more or less the same as |\ekvcSplitAndUse@|. Instead of an empty
%   group we place a marker after the initials, we don't use the sorting macros
%   of |split|, but instead pack all the values in one argument.
%    \begin{macrocode}
\protected\long\def\ekvcHashAndUse@#1#2#3%
  {%
    \edef\ekvc@set{\string#1}%
    \ekvc@SetupHashKeys{#3}%
    \ekvc@helpers@needed
      {%
        \ekvc@any@long\edef#1##1%
          {%
            \expandafter\ekvc@ekvset@pre@expander\expandafter{\ekvc@set}%
            \unexpanded{\ekvc@hash@pack@argument}%
            \unexpanded\expandafter{\ekvc@initials\ekvc@stop#2}%
          }%
      }%
      {%
        \ekvc@any@long\edef#1##1%
          {%
            \expandafter\ekvc@ekvset@pre@expander\expandafter{\ekvc@set}%
            \unexpanded{#2}%
            \unexpanded\expandafter{\ekvc@initials\ekvc@stop}%
          }%
      }%
  }
%    \end{macrocode}
% \end{macro}
%
% \begin{macro}{\ekvcHashAndForward}
%   |\ekvcHashAndForward| works just like |\ekvcSplitAndForward|.
%    \begin{macrocode}
\protected\long\def\ekvcHashAndForward#1#2#3%
  {%
    \let\ekvc@helpers@needed\@firstoftwo
    \ekvc@ifdefined#1%
      {\ekvc@err@already@defined#1}%
      {\ekvcHashAndUse@#1{{#2}}{#3}}%
  }
%    \end{macrocode}
% \end{macro}
%
% \begin{macro}{\ekvcHash}
%   |\ekvcHash| does the same as |\ekvcSplit|, but has the advantage of not
%   needing to count arguments, so the definition of the internal macro is a bit
%   more straight forward.
%    \begin{macrocode}
\protected\long\def\ekvcHash#1#2#3%
  {%
    \let\ekvc@helpers@needed\@secondoftwo
    \ekvc@ifdefined#1%
      {\ekvc@err@already@defined#1}%
      {%
        \expandafter
        \ekvcHashAndUse@\expandafter#1\csname ekvc@\string#1\endcsname{#2}%
        \ekvc@any@long\expandafter\def\csname ekvc@\string#1\endcsname
            ##1\ekvc@stop
          {#3}%
      }%
  }
%    \end{macrocode}
% \end{macro}
%
% \begin{macro}[internal]{\ekvc@hash@pack@argument}
%   All this macro does is pack the values into one argument and forward that to
%   the next macro.
%    \begin{macrocode}
\long\def\ekvc@hash@pack@argument#1\ekvc@stop#2{#2{#1}}
%    \end{macrocode}
% \end{macro}
%
% \begin{macro}[internal]
%   {\ekvc@SetupHashKeys,\ekvc@SetupHashKeys@a,\ekvc@SetupHashKeys@b}
%   This should look awfully familiar as well, since it's just the same as for
%   the split keys with a few other names here and there.
%    \begin{macrocode}
\protected\long\def\ekvc@SetupHashKeys#1%
  {%
    \let\ekvc@any@long\ekv@empty
    \let\ekvc@initials\ekv@empty
    \ekvparse\ekvc@err@value@required\ekvc@SetupHashKeys@a{#1}%
  }
\protected\def\ekvc@SetupHashKeys@a#1%
  {%
    \let\ekvc@long\ekv@empty
    \ekvc@ifspace{#1}%
      {\ekvc@SetupHashKeys@b#1\ekvc@stop}%
      {\ekvc@SetupHashKeys@c{#1}}%
  }
\protected\def\ekvc@SetupHashKeys@b#1 #2\ekvc@stop
  {%
    \ekv@ifdefined{ekvc@hash@p@#1}%
      {\csname ekvc@hash@p@#1\endcsname{#2}}%
      {\ekvc@SetupHashKeys@c{#1 #2}}%
  }
%    \end{macrocode}
%   Yes, even the defining macro looks awfully familiar. Instead of numbered we
%   have named marks. Still the key macros grab everything up to their
%   respective mark and reorder the arguments. The same quirk is applied for
%   short keys. And instead of the |\ekvc@setup@splitmacro| we use
%   |\ekvc@setup@hashmacro|.
%    \begin{macrocode}
\protected\long\def\ekvc@SetupHashKeys@c#1#2%
  {%
    \begingroup
    \edef\ekvc@tmp
      {%
        \endgroup
        \long\def\unexpanded{\ekvc@tmp}####1####2%
            \unexpanded\expandafter{\csname ekvc@hashmark@#1\endcsname}####3%
          {%
            ####2%
            \unexpanded\expandafter{\csname ekvc@hashmark@#1\endcsname}{####1}%
          }%
        \unless\ifx\ekvc@long\long
          \let\unexpanded\expandafter
            {\csname ekvc@\ekvc@set(#1)\endcsname\ekvc@tmp}%
          \def\unexpanded{\ekvc@tmp}####1%
            {%
              \unexpanded\expandafter{\csname ekvc@\ekvc@set(#1)\endcsname}%
                {####1}%
            }%
        \fi
        \def\unexpanded{\ekvc@initials}%
          {%
            \unexpanded\expandafter{\ekvc@initials}%
            \unexpanded\expandafter{\csname ekvc@hashmark@#1\endcsname{#2}}%
          }%
      }%
    \ekvc@tmp
    \ekvlet\ekvc@set{#1}\ekvc@tmp
    \ekvc@setup@hashmacro{#1}%
  }
%    \end{macrocode}
% \end{macro}
%
% \begin{macro}[internal]{\ekvc@hash@p@long}
%   Nothing astonishing here either.
%    \begin{macrocode}
\protected\def\ekvc@hash@p@long
  {%
    \let\ekvc@long\long
    \let\ekvc@any@long\long
    \ekvc@SetupHashKeys@c
  }
%    \end{macrocode}
% \end{macro}
%
% \begin{macro}[internal]{\ekvc@setup@hashmacro}
%   The safe hash macros will be executed inside of an |\unexpanded| expansion
%   context, so they have to insert braces for that once they are done.
%   Most of the tests which have to be executed will already be done, but we
%   have to play safe if the hash doesn't show up in the hash list. Therefore we
%   use some |\ekvc@mark|s and |\ekvc@stop| to throw errors if the hash isn't
%   found in the right place. The fast variants have an easier life and just
%   return the correct value.
%    \begin{macrocode}
\protected\def\ekvc@setup@hashmacro#1%
  {%
    \ekv@ifdefined{ekvc@fasthash@#1}{}%
      {%
        \begingroup
          \edef\ekvc@tmp
            {%
              \long\gdef
                \unexpanded\expandafter{\csname ekvc@fasthash@#1\endcsname}%
                  ####1%
                  \unexpanded\expandafter{\csname ekvc@hashmark@#1\endcsname}%
                  ####2####3\unexpanded{\ekvc@stop}%
                {####2}%
              \long\gdef
                \unexpanded\expandafter{\csname ekvc@safehash@#1\endcsname}%
                  ####1%
                {%
                  \unexpanded\expandafter{\csname ekvc@@safehash@#1\endcsname}%
                  ####1\unexpanded{\ekvc@mark}{}%
                  \unexpanded\expandafter
                    {%
                      \csname ekvc@hashmark@#1\endcsname{}%
                      \ekvc@mark{\ekvc@err@missing@hash{#1}}\ekvc@stop
                    }%
                }%
              \long\gdef
                \unexpanded\expandafter{\csname ekvc@@safehash@#1\endcsname}%
                  ####1%
                  \unexpanded\expandafter{\csname ekvc@hashmark@#1\endcsname}%
                  ####2####3\unexpanded{\ekvc@mark}####4####5%
                  \unexpanded{\ekvc@stop}%
                {%
                  ####4{####2}%
                }%
              \long\gdef\unexpanded\expandafter
                {\csname ekvc@fastsplithash@#1\endcsname}%
                  ####1%
                  \unexpanded\expandafter{\csname ekvc@hashmark@#1\endcsname}%
                  ####2####3\unexpanded{\ekvc@stop}####4%
                {%
                  ####4{####1####3}{####2}%
                }%
              \long\gdef\unexpanded\expandafter
                {\csname ekvc@safesplithash@#1\endcsname}####1%
                {%
                  \unexpanded\expandafter
                    {\csname ekvc@@safesplithash@#1\endcsname}%
                  ####1\unexpanded{\ekvc@mark\ekvc@safe@found@hash}%
                  \unexpanded\expandafter
                    {%
                      \csname ekvc@hashmark@#1\endcsname{}%
                      \ekvc@mark{\ekvc@err@missing@hash{#1}\ekvc@safe@no@hash}%
                      \ekvc@stop
                    }%
                }%
              \long\gdef\unexpanded\expandafter
                {\csname ekvc@@safesplithash@#1\endcsname}%
                  ####1%
                  \unexpanded\expandafter{\csname ekvc@hashmark@#1\endcsname}%
                  ####2####3\unexpanded{\ekvc@mark}####4####5%
                  \unexpanded{\ekvc@stop}%
                {%
                  ####4{####2}####1####3\unexpanded{\ekvc@stop}%
                }%
            }%
          \ekvc@tmp
        \endgroup
      }%
  }
%    \end{macrocode}
% \end{macro}
%
% \begin{macro}{\ekvcValue}
%   All this does is a few consistency checks on the first argument (not empty,
%   hash macro exists) and then call that hash-grabbing macro that will also
%   test whether the hash is inside of |#2| or not.
%    \begin{macrocode}
\long\def\ekvcValue#1%
  {%
    \unexpanded
    \ekv@ifdefined{ekvc@safehash@#1}%
      {\csname ekvc@safehash@#1\endcsname}%
      {\ekvc@err@unknown@hash{#1}\@firstoftwo{{}}}%
  }
%    \end{macrocode}
% \end{macro}
%
% \begin{macro}{\ekvcValueFast}
%   To be as fast as possible, this doesn't test for anything, assuming the user
%   knows best.
%    \begin{macrocode}
\long\def\ekvcValueFast#1#2{\csname ekvc@fasthash@#1\endcsname#2\ekvc@stop}
%    \end{macrocode}
% \end{macro}
%
% \begin{macro}{\ekvcValueSplit,\ekvcValueSplit@recover}
%   This splits off a single value.
%    \begin{macrocode}
\long\def\ekvcValueSplit#1%
  {%
    \ekv@ifdefined{ekvc@safesplithash@#1}%
      {\csname ekvc@safesplithash@#1\endcsname}%
      {\ekvc@err@unknown@hash{#1}\ekvcValueSplit@recover}%
  }
\long\def\ekvcValueSplit@recover#1#2%
  {%
    #2{#1}{}%
  }
%    \end{macrocode}
% \end{macro}
%
% \begin{macro}[internal]{\ekvc@safe@found@hash, \ekvc@safe@no@hash}
%    \begin{macrocode}
\long\def\ekvc@safe@found@hash#1#2\ekvc@stop#3%
  {%
    #3{#2}{#1}%
  }
\long\def\ekvc@safe@no@hash#1#2\ekvc@mark\ekvc@safe@found@hash\ekvc@stop#3%
  {%
    #3{#2}{}%
  }
%    \end{macrocode}
% \end{macro}
%
% \begin{macro}{\ekvcValueSplitFast}
%   Again a fast approach which doesn't provide too many safety measurements.
%   This needs to build the hash function and expand it before passing the
%   results to the next control sequence. The first step only builds the control
%   sequence.
%    \begin{macrocode}
\long\def\ekvcValueSplitFast#1#2%
  {%
    \csname ekvc@fastsplithash@#1\endcsname#2\ekvc@stop
  }
%    \end{macrocode}
% \end{macro}
%
% \begin{macro}[internal]
%   {
%     \ekvc@safehash@,\ekvc@fasthash@,
%     \ekvc@safesplithash@,\ekvc@fastsplithash@
%   }
%   At least in the empty hash case we can provide a meaningful error message
%   without affecting performance by just defining the macro that would be build
%   in that case. There is of course a downside to this, the error will not be
%   thrown by |\ekvcValueFast| in three expansion steps. The safe hash variant
%   has to also stop the |\unexpanded| expansion.
%    \begin{macrocode}
\long\def\ekvc@safehash@#1{\ekvc@err@empty@hash{}}
\long\def\ekvc@fasthash@#1\ekvc@stop{\ekvc@err@empty@hash}
\long\def\ekvc@safesplithash@#1#2{\ekvc@err@empty@hash#2{#1}{}}
\long\def\ekvc@fastsplithash@#1\ekvc@stop#2{\ekvc@err@empty@hash#2{#1}{}}
%    \end{macrocode}
% \end{macro}
%
% \begin{macro}{\ekvcSecondaryKeys}
%   The secondary keys are defined pretty similar to the way the originals are,
%   but here we also introduce some key types (those have a |@t@| in their name)
%   additionally to the prefixes.
%    \begin{macrocode}
\protected\long\def\ekvcSecondaryKeys#1#2%
  {%
    \edef\ekvc@set{\string#1}%
    \ekvparse\ekvc@err@value@required\ekvcSecondaryKeys@a{#2}%
  }
\protected\def\ekvcSecondaryKeys@a#1%
  {%
    \let\ekvc@long\ekv@empty
    \ekvc@ifspace{#1}%
      {\ekvcSecondaryKeys@b#1\ekvc@stop}%
      {\ekvc@err@missing@type{#1}\@gobble}%
  }
\protected\def\ekvcSecondaryKeys@b#1 #2\ekvc@stop
  {%
    \ekv@ifdefined{ekvc@p@#1}%
      {\csname ekvc@p@#1\endcsname}%
      {%
        \ekv@ifdefined{ekvc@t@#1}%
          {\csname ekvc@t@#1\endcsname}%
          {\ekvc@err@unknown@keytype{#1}\@firstoftwo\@gobble}%
      }%
      {#2}%
  }
%    \end{macrocode}
% \end{macro}
%
% \begin{macro}{\ekvcChange}
%   This can be used to change the defaults of an \expkvc\ defined macro. It
%   checks whether there is a set with the correct name and that the macro is
%   defined. If both is true the real work is done by |\ekvc@change|.
%    \begin{macrocode}
\protected\long\def\ekvcChange#1%
  {%
    \ekvifdefinedset{\string#1}%
      {%
        \ekvc@ifdefined#1%
          {\ekvc@change#1}%
          {\ekvc@err@no@key@macro#1\@gobble}%
      }%
      {\ekvc@err@no@key@macro#1\@gobble}%
  }
%    \end{macrocode}
% \end{macro}
%
% \begin{macro}[internal]
%   {\ekvc@change,\ekvc@change@a,\ekvc@change@b,\ekvc@change@c}
%   First we need to see whether the macro is currently |\long|. For this we get
%   the meaning and will parse it. |#1| is the macro name in which we want to
%   change the defaults.
%    \begin{macrocode}
\protected\def\ekvc@change#1%
  {\expandafter\ekvc@change@a\meaning#1\ekv@stop#1}
%    \end{macrocode}
%   A temporary definition to get the stringified |macro:|. |##1| will be the
%   list of prefixes, we don't care for the exact contents of |##2| and |##3|.
%    \begin{macrocode}
\def\ekvc@change@a#1%
  {%
    \protected\def\ekvc@change@a##1#1##2->##3\ekv@stop
      {%
        \ekvc@change@iflong{##1}%
          {\ekvc@change@b{}}%
          {\ekvc@change@b{\long}}%
      }%
  }
\expandafter\ekvc@change@a\expandafter{\detokenize{macro:}}
%    \end{macrocode}
%   Next we expand the macro once to get its contents (including the current
%   default values with their markers). |#1| is either |\long| or empty, |#2| is
%   the macro.
%    \begin{macrocode}
\protected\def\ekvc@change@b#1#2%
  {\expandafter\ekvc@change@c\expandafter{#2{##1}}{#1}#2}
%    \end{macrocode}
%   Here we place an unbalanced closing brace after the expansion of the macro.
%   Then we just parse the \kv-list with |\ekvset|, that will exchange the
%   values behind the markers. Once those are changed we give control to
%   |\ekvc@change@d|. The |\ekvset| step might horribly fail if the user defined
%   some keys that don't behave nice. |#1| is the expansion of the macro, |#2|
%   is either |\long| or empty, |#3| is the macro, and |#4| is the \kv-list
%   containing the new defaults.
%    \begin{macrocode}
\ekv@exparg{\protected\long\def\ekvc@change@c#1#2#3#4}%
  {%
    \expandafter\iffalse\expandafter{\expandafter\fi
      \ekvset{\string#3}{#4}%
      \ekvc@change@d{#2}{#3}%
      #1%
    }%
  }
%    \end{macrocode}
%   The final step needs to put an unbalanced opening brace after |\def|. We do
%   that with the help of a temporary macro which stores everything necessary
%   for |\def| and expand an |\iffalse}\fi| construct to leave a single
%   opening brace. |#1| will be either empty or |\long| and |#2| is the macro.
%   Each of the macros defined with \expkvc\ takes exactly one parameter, so we
%   put that here as |####1|.
%    \begin{macrocode}
\protected\def\ekvc@change@d#1#2%
  {%
    \def\ekvc@tmp{#1\def#2####1}%
    \expandafter\ekvc@tmp\expandafter{\iffalse}\fi
  }
%    \end{macrocode}
% \end{macro}
%
% \begin{macro}[internal]{\ekvc@change@iflong,\ekvc@change@iflong@}
%   Checking whether a string contains the string representation of |\long| can
%   be done by gobbling everything up to the first |\long| and checking whether
%   the result is completely empty. We need a temporary macro to get the result
%   of |\string\long| inside the definitions.
%    \begin{macrocode}
\def\ekvc@change@iflong#1%
  {%
    \protected\def\ekvc@change@iflong##1%
      {\expandafter\ekv@ifempty\expandafter{\ekvc@change@iflong@##1#1}}%
    \def\ekvc@change@iflong@##1#1{}
  }
\expandafter\ekvc@change@iflong\expandafter{\string\long}
%    \end{macrocode}
% \end{macro}
%
% \begin{macro}{\ekvcPass}
%   This macro can be used to pass a value to a key of some macro (this way
%   more complicated key codes are possible that in the end pass processed
%   values on to some macro). The implemantation is pretty straight forward.
%    \begin{macrocode}
\long\def\ekvcPass#1#2%
  {%
    \ekvifdefined{\string#1}{#2}%
      {\csname\ekv@name{\string#1}{#2}\endcsname}%
      {\ekvc@err@unknown@key@or@macro{#1}{#2}\@gobble}%
  }
%    \end{macrocode}
% \end{macro}
%
%
% \subsubsection{Secondary Key Types}
%
% \begin{macro}[internal]{\ekvc@p@long,\ekvc@after@ptype}
%   The prefixes are pretty straight forward again. Just set |\ekvc@long| and
%   forward to the |@t@| type.
%    \begin{macrocode}
\protected\def\ekvc@p@long#1%
  {%
    \ekvc@ifspace{#1}%
      {%
        \let\ekvc@long\long
        \ekvc@after@ptype#1\ekvc@stop
      }%
      {\ekvc@err@missing@type{long #1}\@gobble}%
  }
\protected\def\ekvc@after@ptype#1 #2\ekvc@stop
  {%
    \ekv@ifdefined{ekvc@t@#1}%
      {\csname ekvc@t@#1\endcsname{#2}}%
      {\ekvc@err@unknown@keytype{#1}\@gobble}%
  }
%    \end{macrocode}
% \end{macro}
%
% \begin{macro}[internal]
%   {
%     \ekvc@t@meta, \ekvc@t@nmeta,
%     \ekvc@type@meta, \ekvc@type@meta@a, \ekvc@type@meta@b
%   }
%   The |meta| and |nmeta| key types use a nested |\ekvset| to set other keys in
%   the same macro's \set.
%    \begin{macrocode}
\protected\def\ekvc@t@meta
  {%
    \edef\ekvc@tmp{\ekvc@set}%
    \expandafter\ekvc@type@meta\expandafter{\ekvc@tmp}\ekvc@long{##1}\ekvlet
  }
\protected\def\ekvc@t@nmeta#1%
  {%
    \ekvc@assert@not@long{nmeta #1}%
    \edef\ekvc@tmp{\ekvc@set}%
    \expandafter\ekvc@type@meta\expandafter{\ekvc@tmp}{}{}\ekvletNoVal{#1}%
  }
\protected\long\def\ekvc@type@meta#1#2#3#4#5#6%
  {%
    \expandafter\ekvc@type@meta@a\expandafter{\ekvset{#1}{#6}}{#2}{#3}%
    #4\ekvc@set{#5}\ekvc@tmp
  }
\protected\def\ekvc@type@meta@a
  {%
    \expandafter\ekvc@type@meta@b\expandafter
  }
\protected\long\def\ekvc@type@meta@b#1#2#3%
  {%
    #2\def\ekvc@tmp#3{#1}%
  }
%    \end{macrocode}
% \end{macro}
%
% \begin{macro}[internal]{\ekvc@t@alias}
%   |alias| just checks whether there is a key and/or |NoVal| key defined with
%   the target name and |\let| the key to those.
%    \begin{macrocode}
\protected\def\ekvc@t@alias#1#2%
  {%
    \ekvc@assert@not@long{alias #1}%
    \let\ekvc@tmp\@firstofone
    \ekvifdefined\ekvc@set{#2}%
      {%
        \ekvletkv\ekvc@set{#1}\ekvc@set{#2}%
        \let\ekvc@tmp\@gobble
      }%
      {}%
    \ekvifdefinedNoVal\ekvc@set{#2}%
      {%
        \ekvletkvNoVal\ekvc@set{#1}\ekvc@set{#2}%
        \let\ekvc@tmp\@gobble
      }%
      {}%
    \ekvc@tmp{\ekvc@err@unknown@key{#2}}%
  }
%    \end{macrocode}
% \end{macro}
%
% \begin{macro}[internal]{\ekvc@t@default}
%   The |default| key can be used to set a |NoVal| key for an existing key. It
%   will just pass the \val\ to the key macro of that other key.
%    \begin{macrocode}
\protected\long\def\ekvc@t@default#1#2%
  {%
    \ekvifdefined\ekvc@set{#1}%
      {%
        \ekvc@assert@not@long{default #1}%
        \edef\ekvc@tmp
          {%
            \unexpanded\expandafter
              {\csname\ekv@name\ekvc@set{#1}\endcsname{#2}}%
          }%
        \ekvletNoVal\ekvc@set{#1}\ekvc@tmp
      }%
      {\ekvc@err@unknown@key{#1}}%
  }
%    \end{macrocode}
% \end{macro}
%
% \begin{macro}[internal]{\ekvc@t@flag-bool}
%    \begin{macrocode}
\protected\expandafter\def\csname ekvc@t@flag-bool\endcsname#1#2%
  {%
    \ekvc@assert@not@long{flag-bool #1}%
    \unless\ifdefined#2\ekvcFlagNew#2\fi
    \ekvdef\ekvc@set{#1}%
      {%
        \ekv@ifdefined{ekvc@flag@set@##1}%
          {%
            \csname ekvc@flag@set@##1\expandafter\endcsname
              \ekvcFlagHeight#2\ekv@stop#2%
          }%
          {\ekvc@err@invalid@bool{##1}}%
      }%
  }
%    \end{macrocode}
% \end{macro}
%
% \begin{macro}[internal]
%   {\ekvc@t@flag-true,\ekvc@t@flag-false,\ekvc@t@flag-raise,\ekvc@type@flag}
%    \begin{macrocode}
\protected\def\ekvc@type@flag#1#2#3#4%
  {%
    \ekvc@assert@not@long{flag-#1 #3}%
    \unless\ifdefined#4\ekvcFlagNew#4\fi
    \ekv@exparg{\ekvdefNoVal\ekvc@set{#3}}{#2#4}%
  }
\protected\expandafter\def\csname ekvc@t@flag-true\endcsname
  {\ekvc@type@flag{true}\ekvcFlagSetTrue}
\protected\expandafter\def\csname ekvc@t@flag-false\endcsname
  {\ekvc@type@flag{false}\ekvcFlagSetFalse}
\protected\expandafter\def\csname ekvc@t@flag-raise\endcsname
  {\ekvc@type@flag{raise}\ekvcFlagRaise}
%    \end{macrocode}
% \end{macro}
%
%
% \subsubsection{Flags}
%
% The basic idea of flags is to store information by the fact that \TeX\
% expandably assigns the meaning |\relax| to undefined control sequences which
% were built with |\csname|. This mechanism is borrowed from \pkg{expl3}.
%
% \begin{macro}[internal]{\ekvc@flag@name,\ekvc@flag@namescheme}
%   Flags follow a simple naming scheme which we define here. |\ekvc@flag@name|
%   will store the name of an internal function that is used to build names of
%   the second naming scheme defined by |\ekvc@flag@namescheme|.
%    \begin{macrocode}
\def\ekvc@flag@name{ekvcf\string}
\def\ekvc@flag@namescheme#1#2{ekvch#2#1}
%    \end{macrocode}
% \end{macro}
%
% \begin{macro}{\ekvcFlagHeight}
%   For semantic reasons we use |\number| with another name.
%    \begin{macrocode}
\let\ekvcFlagHeight\number
%    \end{macrocode}
% \end{macro}
%
% \begin{macro}{\ekvcFlagNew}
%   This macro defines a new flag. It stores the function build with the
%   |\ekvc@flag@name| naming scheme after the internal function
%   |\ekvc@flag@height| that'll determine the current flag height. It'll also
%   define the macro named via |\ekvc@flag@name| to build names according to
%   |\ekvc@flag@namescheme|.
%    \begin{macrocode}
\protected\def\ekvcFlagNew#1%
  {%
    \edef#1%
      {%
        \unexpanded{\ekvc@flag@height}%
          \unexpanded\expandafter{\csname\ekvc@flag@name#1\endcsname}%
      }%
    \ekv@expargtwice
      {\expandafter\def\csname\ekvc@flag@name#1\endcsname##1}%
      {\expandafter\ekvc@flag@namescheme\expandafter{\string#1}{##1}}%
  }
%    \end{macrocode}
% \end{macro}
%
% \begin{macro}[internal]{\ekvc@flag@height,\ekvc@flag@height@}
%   This macro gets the height of a flag by a simple loop. The first loop
%   iteration differs a bit from the following in that it doesn't have to get
%   the current iteration count. The space at the end of |\ekvc@flag@height|
%   ends the |\number| evaluation.
%    \begin{macrocode}
\def\ekvc@flag@height#1%
  {%
    \ifcsname#10\endcsname
      \ekvc@flag@height@1\ekv@stop#1%
    \fi
    \@firstofone{0} % leave this space
  }
\def\ekvc@flag@height@#1\ekv@stop#2\fi\@firstofone#3%
  {%
    \fi
    \ifcsname#2{#1}\endcsname
      \expandafter\ekvc@flag@height@\the\numexpr#1+1\relax\ekv@stop#2%
    \fi
    \@firstofone{#1}%
  }
%    \end{macrocode}
% \end{macro}
%
% \begin{macro}{\ekvcFlagRaise}
%   Raising a flag simply means letting the |\ekvc@flag@namescheme| macro for
%   the current height to relax. The result of raising a flag is that its height
%   is bigger by $1$.
%    \begin{macrocode}
\ekv@exparg{\def\ekvcFlagRaise#1}%
  {%
    \expandafter\expandafter\expandafter\@gobble\expandafter
      \csname\ekvc@flag@namescheme{\string#1}{\ekvcFlagHeight#1}\endcsname
  }
%    \end{macrocode}
% \end{macro}
%
% \begin{macro}{\ekvcFlagSetTrue,\ekvcFlagSetFalse}
% \begin{macro}[internal]{\ekvc@flag@set@true,\ekvc@flag@set@false}
%   A flag is considered true if its current height is odd, and as false if it
%   is even. Therefore |\ekvcFlagSetTrue| and |\ekvcFlagSetFalse| only need to
%   raise the flag if the opposing boolean value is the current one.
%    \begin{macrocode}
\def\ekvcFlagSetTrue#1%
  {\expandafter\ekvc@flag@set@true\ekvcFlagHeight#1\ekv@stop#1}
\def\ekvcFlagSetFalse#1%
  {\expandafter\ekvc@flag@set@false\ekvcFlagHeight#1\ekv@stop#1}
%    \end{macrocode}
%   We can expand |\ekvc@flag@namescheme| at definition time here, which is why
%   we're using a temporary definition to set up |\ekvc@flag@set@true| and
%   |\ekvc@flag@set@false|.
%    \begin{macrocode}
\def\ekvc@flag@set@true#1%
  {%
    \def\ekvc@flag@set@true##1\ekv@stop##2%
      {%
        \ifodd##1
          \ekv@fi@gobble
        \fi
        \@firstofone{\expandafter\@gobble\csname#1\endcsname}%
      }%
    \def\ekvc@flag@set@false##1\ekv@stop##2%
      {%
        \ifodd##1
          \ekv@fi@firstofone
        \fi
        \@gobble{\expandafter\@gobble\csname#1\endcsname}%
      }%
  }
\expandafter\ekvc@flag@set@true\expandafter
  {\ekvc@flag@namescheme{\string#2}{#1}}
%    \end{macrocode}
% \end{macro}
% \end{macro}
%
% \begin{macro}{\ekvcFlagIf}
%   As already explained, truthiness is defined as a flag's height is odd, so we
%   just branch accordingly here.
%    \begin{macrocode}
\def\ekvcFlagIf#1%
  {%
    \ifodd#1%
      \ekv@fi@firstoftwo
    \fi
    \@secondoftwo
  }
%    \end{macrocode}
% \end{macro}
%
% \begin{macro}{\ekvcFlagIfRaised}
%   This macro uses flags as a switch, if a flag's current height is bigger than
%   $0$ this test yields true.
%    \begin{macrocode}
\ekv@exparg{\def\ekvcFlagIfRaised#1}%
  {%
    \expandafter\ifcsname\ekvc@flag@namescheme{\string#1}0\endcsname
      \ekv@fi@firstoftwo
    \fi
    \@secondoftwo
  }
%    \end{macrocode}
% \end{macro}
%
% \begin{macro}{\ekvcFlagReset}
% \begin{macro}[internal]{\ekvc@flag@reset,\ekvc@flag@reset@}
%   Resetting works by locally letting all the defined internal macros named
%   after |\ekvc@flag@namescheme| to undefined.
%    \begin{macrocode}
\protected\def\ekvcFlagReset#1%
  {\expandafter\ekvc@flag@reset\csname\ekvc@flag@name#1\endcsname}
\protected\def\ekvc@flag@reset#1%
  {%
    \ifcsname#10\endcsname
      \expandafter\let\csname#10\endcsname\ekvc@undefined
      \ekvc@flag@reset@1\ekv@stop#1%
    \fi
  }
\protected\def\ekvc@flag@reset@#1\ekv@stop#2\fi
  {%
    \fi
    \ifcsname#2{#1}\endcsname
      \expandafter\let\csname#2{#1}\endcsname\ekvc@undefined
      \expandafter\ekvc@flag@reset@\the\numexpr#1+1\relax\ekv@stop#2%
    \fi
  }
%    \end{macrocode}
% \end{macro}
% \end{macro}
%
% \begin{macro}{\ekvcFlagGetHeight}
% \begin{macro}[internal]{\ekvc@flag@get@height@single}
%   These are just small helpers, first getting the height of the flag and then
%   passing it on to the user supplied code.
%    \begin{macrocode}
\def\ekvcFlagGetHeight#1%
  {\expandafter\ekvc@flag@get@height@single\ekvcFlagHeight#1\ekv@stop}
\long\def\ekvc@flag@get@height@single#1\ekv@stop#2{#2{#1}}
%    \end{macrocode}
% \end{macro}
% \end{macro}
%
% \begin{macro}{\ekvcFlagGetHeights}
% \begin{macro}[internal]
%   {\ekvc@flag@get@heights,\ekvc@flag@get@heights@,\ekvc@flag@get@heights@done}
%   This works by a simple loop that stops at |\ekv@stop|. As long as that
%   marker isn't hit, get the next flags height and put it into a list after
%   |\ekv@stop|. |\ekvc@flag@get@heights@| uses the same marker name for the
%   end of the height, which shouldn't clash in any case. Once we're done we
%   remove the remainder of the current iteration and leave the user supplied
%   code in the input stream with all the flags' heights as a single argument.
%    \begin{macrocode}
\def\ekvcFlagGetHeights#1%
  {%
    \ekvc@flag@get@heights#1\ekv@stop{}%
  }
\def\ekvc@flag@get@heights#1%
  {%
    \ekv@gobbleto@stop#1\ekvc@flag@get@heights@done\ekv@stop
    \expandafter\ekvc@flag@get@heights@\ekvcFlagHeight#1\ekv@stop
  }
\def\ekvc@flag@get@heights@#1\ekv@stop#2\ekv@stop#3%
  {\ekvc@flag@get@heights#2\ekv@stop{#3{#1}}}
\long\def\ekvc@flag@get@heights@done
    \ekv@stop
    \expandafter\ekvc@flag@get@heights@\ekvcFlagHeight\ekv@stop\ekv@stop#1#2%
  {#2{#1}}
%    \end{macrocode}
% \end{macro}
% \end{macro}
%
%
% \subsubsection{Helper Macros}
%
% \begin{macro}[internal]{\ekvc@ifspace,\ekvc@ifspace@}
% A test which can be reduced to an if-empty by gobbling everything up to the
% first space.
%    \begin{macrocode}
\long\def\ekvc@ifspace#1%
  {%
    \ekvc@ifspace@#1 \ekv@ifempty@B
      \ekv@ifempty@false\ekv@ifempty@A\ekv@ifempty@B\@firstoftwo
  }
\long\def\ekvc@ifspace@#1 % keep this space
  {%
    \ekv@ifempty@\ekv@ifempty@A
  }
%    \end{macrocode}
% \end{macro}
%
%
% \subsubsection{Assertions}
% \begin{macro}[internal]{\ekvc@assert@not@long}
%   Some keys don't want to be |long| and we have to educate the user, so let's
%   throw an error if someone wanted these to be long.
%    \begin{macrocode}
\long\def\ekvc@assert@not@long#1{\ifx\ekvc@long\long\ekvc@err@no@long{#1}\fi}
%    \end{macrocode}
% \end{macro}
%
%
% \subsubsection{Messages}
%
% \begin{macro}[internal]
%   {
%     \ekvc@err@toomany,\ekvc@err@value@required,\ekvc@err@missing@type,
%     \ekvc@err@already@defined,\ekvc@err@no@key@macro
%   }
%   Boring unexpandable error messages.
%    \begin{macrocode}
\protected\def\ekvc@err@toomany#1%
  {%
    \errmessage{expkv-cs Error: Too many keys for macro `\string#1'}%
  }
\protected\def\ekvc@err@value@required#1%
  {%
    \errmessage{expkv-cs Error: Missing value for key `\unexpanded{#1}'}%
  }
\protected\def\ekvc@err@missing@type#1%
  {%
    \errmessage
      {expkv-cs Error: Missing type for secondary key `\unexpanded{#1}'}%
  }
\protected\def\ekvc@err@no@long#1%
  {%
    \errmessage
      {expkv-cs Error: prefix `long' not accepted for `\unexpanded{#1}'}%
  }
\protected\def\ekvc@err@already@defined#1%
  {%
    \errmessage{expkv-cs Error: Macro `\string#1' already defined}%
  }
\protected\def\ekvc@err@unknown@keytype#1%
  {%
    \errmessage{expkv-cs Error: Unknown key type `\unexpanded{#1}'}%
  }
\protected\def\ekvc@err@unknown@key#1%
  {%
    \errmessage
      {expkv-cs Error: Unknown key `\unexpanded{#1}' for macro `\ekvc@set'}%
  }
\long\def\ekvc@err@no@key@macro#1%
  {\errmessage{expkv-cs Error: \string#1 is no key=val macro}}
%    \end{macrocode}
% \end{macro}
%
% \begin{macro}[internal]{\ekvc@err}
%   We need a way to throw error messages expandably in some contexts.
%    \begin{macrocode}
\ekv@exparg{\long\def\ekvc@err#1}{\ekverr{expkv-cs}{#1}}
%    \end{macrocode}
% \end{macro}
%
% \begin{macro}[internal]
%   {
%     \ekvc@err@unknown@hash,\ekvc@err@empty@hash,\ekvc@err@missing@hash,
%     \ekvc@err@invalid@bool
%   }
%   And here are the expandable error messages.
%    \begin{macrocode}
\long\def\ekvc@err@unknown@hash#1{\ekvc@err{unknown hash `#1'}}
\long\def\ekvc@err@missing@hash#1{\ekvc@err{hash `#1' not found}}
\long\def\ekvc@err@empty@hash{\ekvc@err{empty hash}}
\def\ekvc@err@invalid@bool#1{\ekvc@err{invalid boolean value `#1'}}
\long\def\ekvc@err@unknown@key@or@macro#1#2%
  {\ekvc@err{unknown key `#2' for macro #1}}
%    \end{macrocode}
% \end{macro}
%
%
% Now everything that's left is to reset the category code of |@|.
%    \begin{macrocode}
\catcode`\@=\ekvc@tmp
%    \end{macrocode}
%
% \gobbledocstriptag
%</tex>
%^^A=<<
%
%^^A=<<
%
% \end{implementation}^^A=<<
%
% \clearpage
% \PrintIndex
%
\endinput
%
^^A vim: ft=tex fdm=marker fmr=>>=,=<<
}{}
\ProvidesFile{expkv-cs.dtx}
  [\csname ekvcDate\endcsname\ define expandable key=val macros using expkv]
\PassOptionsToPackage{full}{textcomp}
\documentclass{l3doc}
\RequirePackage[oldstylenums,nott]{kpfonts}
\input{glyphtounicode}
\pdfgentounicode=1
\RequirePackage{tcolorbox}
\newtcolorbox{exresult}[2][]
  {colback=expkvgrey!10!white,colframe=expkvgrey,fontupper=\small,width={#2},#1}
\newtcbox\exres[1][]
  {
     colback=expkvgrey!10!white
    ,colframe=expkvgrey
    ,size=small
    ,nobeforeafter
    ,tcbox raise base
    ,fontupper=\small
    ,#1
  }
\RequirePackage{listings}
\RequirePackage{booktabs}
\RequirePackage{array}
\RequirePackage{collcell}
\RequirePackage{siunitx}
\RequirePackage{xcolor}
\RequirePackage{caption}
\RequirePackage{microtype}
\RequirePackage{accsupp}
\lstdefinelanguage{expkvc}[primitive]{TeX}
  {
    ,moretexcs=[4]^^A e-TeX
      {
        expanded,
        numexpr,
        protected,
      }
    ,moretexcs=[5]^^A plain/LaTeX
      {
        approx,
        begin,
        empty,
        item,
        LaTeX,
        makeatletter,makeatother,multicolumn,
        newcommand,newdimen,
        RequirePackage,
        rule,
        TeX,
        textit,texttt,
        usepackage,
      }
    ,moretexcs=[6]^^A used packages
      {
        ^^A xfp
        fpeval,
        ^^A keyval
        setkeys,
        ^^A yax
        defactiveparameter,storevalue,setparameterlist
      }
    ,moretexcs=[2]^^A expkv-cs macros
      {
        ekvcSplit,ekvcSplitAndForward,ekvcSplitAndUse,
        ekvcHash,ekvcHashAndForward,ekvcHashAndUse,
        ekvcValue,ekvcValueFast,ekvcValueSplit,ekvcValueSplitFast,
        ekvcSecondaryKeys,
        ekvcChange,
        ekvcFlagNew,
        ekvcFlagHeight,
        ekvcFlagRaise,
        ekvcFlagSetTrue,ekvcFlagSetFalse,
        ekvcFlagIf,ekvcFlagIfRaised,
        ekvcFlagReset,
        ekvcFlagGetHeight,ekvcFlagGetHeights,
        ekvcPass
      }
    ,moretexcs=[3]^^A expkv-pkg macros
      {
        ^^A expkv
        ekvdef,ekvdefNoVal,^^A
        ekvlet,ekvletNoVal,ekvletkv,ekvletkvNoVal,^^A
        ekvdefunknown,ekvdefunknownNoVal,^^A
        ekvredirectunknown,ekvredirectunknownNoVal,^^A
        ekvset,ekvsetSneaked,^^A
        ekvsetdef,ekvsetSneakeddef,ekvsetdefSneaked,^^A
        ekvparse,^^A
        ekvVersion,ekvDate,^^A
        ekvifdefined,ekvifdefinedNoVal,ekvifdefinedset,^^A
        ekvbreak,ekvbreakPreSneak,ekvbreakPostSneak,^^A
        ekvsneak,ekvsneakPre,^^A
        ekvchangeset,^^A
        ekvoptarg,ekvoptargTF,^^A
        ekverr,
        ^^A expkv-def
        ekvdefinekeys
      }
  }
\colorlet{codeparam}{cyan!65!black}
\def\mylstwd{.55em}
\lstset
  {
    ,language=expkvc
    ,flexiblecolumns=false
    ,basewidth=\mylstwd
    ,gobble=2
    ,basicstyle=\fontfamily{jkp}\itshape
    ,texcsstyle=*[1]{\bfseries\color{expkvgrey}} ^^A primitives
    ,texcsstyle=*[2]{\bfseries\color{expkvred}}  ^^A expkv
    ,texcsstyle=*[3]{\color{expkvred}}           ^^A expkv-pkg
    ,texcsstyle=*[4]{\bfseries\color{expkvgrey}} ^^A e-TeX
    ,texcsstyle=*[5]{\bfseries\color{expkvgrey}} ^^A plain/LaTeX macros
    ,texcsstyle=*[6]{}                           ^^A macros of other packages
    ,commentstyle=\color[gray]{0.4}
    ,literate=
              {\{} {{\CodeUpBf\{}}{1}
              {\}} {{\CodeUpBf\}}}{1}
              {$}  {{\CodeUpBf\$}}{1}
              {[}  {{\CodeUp[}}{1}
              {]}  {{\CodeUp]}}{1}
              {(}  {{\CodeUp(}}{1}
              {)}  {{\CodeUp)}}{1}
              {*}  {{$*$}}{1}
              {1}  {{\CodeUp{1}}}{1}
              {2}  {{\CodeUp{2}}}{1}
              {3}  {{\CodeUp{3}}}{1}
              {4}  {{\CodeUp{4}}}{1}
              {5}  {{\CodeUp{5}}}{1}
              {6}  {{\CodeUp{6}}}{1}
              {7}  {{\CodeUp{7}}}{1}
              {8}  {{\CodeUp{8}}}{1}
              {9}  {{\CodeUp{9}}}{1}
              {0}  {{\CodeUp{0}}}{1}
              {##} {{\CodeColored{codeparam}{1}{\#}}}{1}
              {##1}{{\CodeColored{codeparam}{2}{\#1}}}{2}
              {##2}{{\CodeColored{codeparam}{2}{\#2}}}{2}
              {##3}{{\CodeColored{codeparam}{2}{\#3}}}{2}
              {##4}{{\CodeColored{codeparam}{2}{\#4}}}{2}
              {##5}{{\CodeColored{codeparam}{2}{\#5}}}{2}
              {##6}{{\CodeColored{codeparam}{2}{\#6}}}{2}
              {##7}{{\CodeColored{codeparam}{2}{\#7}}}{2}
              {##8}{{\CodeColored{codeparam}{2}{\#8}}}{2}
              {##9}{{\CodeColored{codeparam}{2}{\#9}}}{2}
              {<key>}{{$\langle$}key{$\rangle$}}{5}
              {<set>}{{$\langle$}set{$\rangle$}}{5}
              {<set1>}{{$\langle$}set1{$\rangle$}}{6}
              {<set2>}{{$\langle$}set2{$\rangle$}}{6}
    ^^A,literate=*{<key>}{\key}{4}{<set>}{\set}{4}
  }
\newcommand*\CodeColored[3]{\textcolor{#1}{\makebox[\dimexpr\mylstwd*#2]{#3}}}
\newcommand*\CodeUpBf[1]{\makebox[\mylstwd]{\textup{\textbf{#1}}}}
\newcommand*\CodeUp[1]{\makebox[\mylstwd]{\textup{#1}}}
\RequirePackage{randtext}
\let\metaORIG\meta
\protected\def\meta #1{\texttt{\metaORIG{#1}}}
\renewcommand*\thefootnote{\fnsymbol{footnote}}
\definecolor{expkvred}{HTML}{9F393D}
\colorlet{expkvgrey}{black!75}
\makeatletter
\newcommand*\example{\par\smallskip\noindent\textit{Example:} \ignorespaces}
\newcommand*\expkv
  {^^A
    \texorpdfstring
      {^^A
        \mbox
          {^^A
            \BeginAccSupp{ActualText=expkv}^^A
            \href{https://github.com/Skillmon/tex_expkv}
              {^^A
                \rmfamily
                \bfseries
                {\color{expkvgrey}e\kern-.05em x\kern-.05em}^^A
                \lower.493ex
                  \hbox{{\color{expkvgrey}P}\kern-.1em{\color{expkvred}k}}^^A
                \kern-.18em{\color{expkvred}v}^^A
              }^^A
            \EndAccSupp{}^^A
          }^^A
      }
      {expkv}^^A
  }
\newcommand*\expkvpkg[1]
  {^^A
    \texorpdfstring
      {^^A
        \mbox
          {^^A
            \BeginAccSupp{ActualText=expkv-#1}^^A
            \href{https://github.com/Skillmon/tex_expkv-#1}
              {^^A
                \rmfamily
                \bfseries
                {\color{expkvgrey}e\kern-.05em x\kern-.05em}^^A
                \lower.493ex
                  \hbox{{\color{expkvgrey}P}\kern-.1em{\color{expkvred}k}}^^A
                \kern-.18em{\color{expkvred}v}^^A
                {\color{expkvgrey}^^A
                  \kern.05em\rule[-.1ex]{.08em}{1.2ex}\kern.05em\textsc{#1}^^A
                }^^A
              }^^A
            \EndAccSupp{}^^A
          }^^A
      }
      {expkv-#1}^^A
  }
\newcommand*\expkvd{\expkvpkg{def}}
\newcommand*\expkvc{\expkvpkg{cs}}
\newcommand\kv{\meta{key}=\meta{value}}
\newcommand\key{\meta{key}}
\newcommand\val{\meta{value}}
\newcommand\set{\meta{set}}
\newcommand\enflong{\textcolor{black}{long}}
\newcommand\alllong{\textcolor{gray}{long}}
\newcommand\notlong{\textcolor{red!80!black}{long}}
\newcommand\prefixes[1]
  {^^A
    \hfill
    \ifcase\numexpr#1\relax\or
      \enflong\or
      \alllong\or
      \notlong\fi
  }
\hypersetup{linkcolor=red!80!black,urlcolor=purple!80!black}
\DoNotIndex{\def,\edef,\,,\=,\begingroup,\catcode,\chardef,\csname,\endcsname}
\DoNotIndex{\endgroup,\endinput,\errmessage,\expandafter,\input,\let,\long,\z@}
\DoNotIndex{\protected,\ProvidesFile,\ProvidesPackage,\relax,\space,\advance}
\DoNotIndex{\@,\unexpanded,\string,\expanded,\dimexpr,\global,\glueexpr,\hbox}
\DoNotIndex{\numexpr,\RequirePackage,\setbox,\the,\unless,\xdef,\gdef,\newcount}
\DoNotIndex{\number}
\DoNotIndex{\^,\@firstofone,\@firstoftwo,\@gobble,\@secondoftwo}
\DoNotIndex{\ifcsname}
\DoNotIndex{\ifx}
\DoNotIndex{\ifodd}
\DoNotIndex{\ifnum}
\DoNotIndex{\ifdefined}
\DoNotIndex{\iffalse}
\DoNotIndex{\iftrue}
\DoNotIndex{\else}
\DoNotIndex{\fi}
\@gobble\fi ^^A ignoring \ifx, \ifcsname, etc., but only one \fi
\@gobble\fi ^^A ignoring \ifx, \ifcsname, etc., but only one \fi
\@gobble\fi ^^A ignoring \ifx, \ifcsname, etc., but only one \fi
\@gobble\fi ^^A ignoring \ifx, \ifcsname, etc., but only one \fi
\@gobble\fi ^^A ignoring \ifx, \ifcsname, etc., but only one \fi
\@gobble\fi ^^A ignoring \ifx, \ifcsname, etc., but only one \fi
\@ifdefinable\gobbledocstriptag{\def\gobbledocstriptag#1>{}}
\makeatother
\begin{document}
  \title
    {^^A
      \texorpdfstring
        {^^A
          \huge\expkvc
          \\[\medskipamount]
          \Large define expandable \kv\ macros using \expkv
        }{expkv-cs - define expandable key=val macros using expkv}^^A
    }
  \date{\ekvcDate\space v\ekvcVersion}
  \author{Jonathan P. Spratte\thanks{\protect\randomize{jspratte@yahoo.de}}}
  \DocInput{expkv-cs.dtx}
\end{document}
%</driver>^^A=<<
% \fi
%
% \maketitle
% \renewcommand*\thefootnote{\arabic{footnote}}
%
% \begin{abstract}
% \noindent\parfillskip=0pt
% \expkvc\ provides two small interfaces to define expandable \kv\ macros using
% \expkv. It therefore lowers the entrance boundary to expandable \kv\ macros.
% The stylised name is \expkvc\ but the files use \file{expkv-cs}, this is due
% to CTAN-rules which don't allow \string| in package names since that is the
% pipe symbol in *nix shells.
% \end{abstract}
%
% \tableofcontents
%
% \begin{documentation}^^A>>=
%
% \section{Documentation}
%
% The \expkv\ package enables the new possibility of creating \kv\ macros which
% are fully expandable. The creation of such macros is however cumbersome for
% the average user. \expkvc\ tries to step in here. It provides interfaces to
% define \kv\ macros without worrying too much about the implementation. In case
% you're wondering now, the |cs| in \expkvc\ stands for control sequence,
% because |def| was already taken by \expkvd\ and ``control sequence'' is the
% term D.\,E.\,Knuth used in his \TeX book for named commands hence macros
% (though he also used the term ``macro''). So \expkvc\ defines control
% sequences for and with \expkv.
%
% There are two different approaches supported by this package. The first is
% splitting the keys up into individual arguments, the second is providing all
% the keys as a single argument to the underlying macro and getting an
% individual \val\ by using a hash. Well, actually there is no real hash, just
% some markers which are parsed, but this shouldn't be apparent to the user, the
% behaviour matches that of a hash-table.
%
% In addition to these two methods of defining a macro with primary keys a way
% to define secondary keys, which can reference the primary ones, is provided.
% These secondary keys don't correspond to an argument or an entry in the hash
% table directly but might come in handy for the average use case. Each macro
% has its own set of primary and secondary keys.
%
% A word of advice you should consider: If your macro doesn't have to be
% expandable (and often it doesn't) consider not using \expkvc. The interface
% has some overhead (though it still is fast -- check \autoref{sec:speed}) and
% the approach has its limits in versatility. If you don't need to be
% expandable, you should consider either defining your keys manually using
% \expkv\ or using \expkvd\ for convenience. Or you resort to another \kv\
% interface. Nevertheless setting up macros with \expkvc, especially with
% |\ekvcSplit|, is very convenient in my opinion, so if you just want to define
% a single macro with just a few keys this might be the way to go.
%
% \expkvc\ is usable as generic code and as a \LaTeX\ package. It'll
% automatically load \expkv\ in the same mode as well. To use it, just use one
% of
% \begin{lstlisting}
% \usepackage{expkv-cs} % LaTeX
% \input expkv-cs       % plainTeX
% \end{lstlisting}
%
%
% \subsection{Define Macros and Primary Keys}
%
% All macros defined with \expkvc\ have to be previously undefined or have the
% |\meaning| of |\relax|. This is necessary as there is no way to automatically
% undefine keys once they are set up (neither \expkv\ nor \expkvc\ keep track of
% defined keys) -- so to make sure there are no conflicts only new definitions
% are allowed (that's not the case for individual keys, only for frontend
% macros).
%
%
% \subsubsection{Primary Keys}\label{sec:primaries}
%
% In the following descriptions there will be one argument named \meta{primary
% keys}. This argument should be a \kv\ list where each \key\ will be one
% primary key and \val\ the associated initial value. By default all keys are
% defined short, but you can define long keys by prefixing \key\ with |long|
% (\emph{e.g.}, \texttt{long name=Jonathan P. Spratte}). You only need |long| if
% the key should be able to take a |\par| token. Note however that |long| keys
% are a microscopic grain faster (due to some internals of \expkvc). Only if at
% least one of the keys was |long| the \meta{cs} in the following defining
% macros will be |\long|. For obvious reasons there is no possibility to define
% a macro or key as |\protected|.
%
% At the moment \expkvc\ doesn't require any internal keys, but I can't foresee
% whether this will be the case in the future as well, as it might turn out that
% some features I deem useful can't be implemented without such internal keys.
% Because of this, please don't use key names starting with
% \texttt{EKVC\string|} as that should be the private name space.
%
%
% \subsubsection{Split}
%
% The split variants will provide the key values as separate arguments. This
% limits the number of keys for which this is truly useful.
%
% \begin{function}{\ekvcSplit}
%   \begin{syntax}
%     \cs{ekvcSplit}\meta{cs}\marg{primary keys}\marg{definition}
%   \end{syntax}
%   This defines \meta{cs} to be a macro taking one mandatory argument which
%   should contain a \kv\ list. The \meta{primary keys} will be defined for this
%   macro (see \autoref{sec:primaries}). The \meta{definition} is the code that
%   will be executed. You can access the \val\ of a \key\ by using a macro
%   parameter from |#1| to |#9|. The order of the macro parameters will be the
%   order provided in the \meta{primary keys} list (so |#1| is the \val\ of the
%   key defined first). With |\ekvcSplit| you can define macros using at most
%   nine primary keys.
% \end{function}
% \example The following defines a macro |\foo| that takes the keys |a| and |b|
% and outputs their values in a textual form:\\
% \begin{minipage}{.75\linewidth}
% \begin{lstlisting}
% \ekvcSplit\foo{a=a,b=b}{a is #1.\par b is #2.\par}
% \foo{}
% \foo{b=e}
% \end{lstlisting}
% \end{minipage}^^A
% \begin{exresult}[nobeforeafter,box align=center]{.25\linewidth}
%   \ekvcSplit\foo{a=a,b=b}{a is #1.\par b is #2.\par}
%   \foo{}
%   \foo{b=e}
% \end{exresult}
%
% \begin{function}{\ekvcSplitAndForward}
%   \begin{syntax}
%     \cs{ekvcSplitAndForward}\meta{cs}\marg{after}\marg{primary keys}
%   \end{syntax}
%   This defines \meta{cs} to be a macro taking one mandatory argument which
%   should contain a \kv\ list. You can use as many primary keys as you want
%   with this. The primary keys will be forwarded to \meta{after} as braced
%   arguments (as many as necessary for your primary keys). The order of the
%   braced arguments will be the order of your primary key definitions. In
%   \meta{after} you can use just a single control sequence, or some arbitrary
%   stuff which will be left in the input stream before your braced values (with
%   one set of braces stripped from \meta{after}), so both of the following
%   would be fine:
% \begin{lstlisting}
% \ekvcSplitAndForward\foo\foo@aux{keyA = A, keyB = B}
% \ekvcSplitAndForward\foo{\foo@aux{more args}}{keyA = A, keyB = B}
% \end{lstlisting}
% \end{function}
% In the first case |\foo@aux| should take at least two arguments (|keyA| and
% |keyB|), in the second case at least three (|more args|, |keyA|, and |keyB|).
%
% \begin{function}{\ekvcSplitAndUse}
%   \begin{syntax}
%     \cs{ekvcSplitAndUse}\meta{cs}\marg{primary keys}
%   \end{syntax}
%   This will roughly do the same as |\ekvcSplitAndForward|, but instead of
%   specifying what will be used after splitting the keys, \meta{cs} will use
%   what follows the \kv\ list. So its syntax will be
% \end{function}
%   \begin{syntax}
%     \meta{cs}\{\key=\val, \ldots\}\marg{after}
%   \end{syntax}
% and the code in |after| should expect at least as many arguments as the number
% of keys defined for \meta{cs}.
%
% \subsubsection{Hash}
%
% The hash variants will provide the key values as a single argument in which
% you can access specific values using a special macro. The implementation might
% be more convenient and scale better, \emph{but} it is much slower (for a
% primitive macro with a single key benchmarking was almost $1.7$ times slower,
% the root of which being the key access with |\ekvcValue|, not the parsing, and
% for a key access using |\ekvcValueFast| it was still about $1.2$ times
% slower). So if your macro uses less than ten primary keys, you should consider
% using the split approach.
%
% \begin{function}{\ekvcHash}
%   \begin{syntax}
%     \cs{ekvcHash}\meta{cs}\marg{primary keys}\marg{definition}
%   \end{syntax}
%   This defines \meta{cs} to be a macro taking one mandatory argument which
%   should contain a \kv\ list. You can use as many primary keys as you want.
%   The primary keys will be forwarded as a single argument containing every key
%   to the underlying macro. The underlying macro is defined as
%   \meta{definition}, in which you can access the \val\ of a \key\ by using
%   \texttt{\cs[no-index]{ekvcValue}\{\key\}\{\#1\}}.
% \end{function}
% \example This defines an equivalent macro like the |\foo| defined with
% |\ekvcSplit| earlier:\\
% \begin{minipage}{.75\linewidth}
% \begin{lstlisting}
% \ekvcHash\foo{a=a,b=b}{a is \ekvcValue{a}{#1}.\par
%                        b is \ekvcValue{b}{#1}.\par}
% \foo{}
% \foo{b=e}
% \end{lstlisting}
% \end{minipage}^^A
% \begin{exresult}[nobeforeafter,box align=center]{.25\linewidth}
%   \ekvcHash\foo{a=a,b=b}{a is \ekvcValue{a}{#1}.\par
%                          b is \ekvcValue{b}{#1}.\par}
%   \foo{}
%   \foo{b=e}
% \end{exresult}
%
% \begin{function}{\ekvcHashAndForward}
%   \begin{syntax}
%     \cs{ekvcHashAndForward}\meta{cs}\marg{after}\marg{primary keys}
%   \end{syntax}
%   This defines \meta{cs} to be a macro taking one mandatory argument which
%   should contain a \kv\ list. You can use as many primary keys as you want.
%   The primary keys will be forwarded as a single argument containing every key
%   to \meta{after}. You can use a single macro for \meta{after} or use some
%   arbitrary stuff, which will be left in the input stream before the hashed
%   \kv\ list with one set of braces stripped. In the
%   macro called in \meta{after} you can access the \val\ of a \key\ by using
%   \texttt{\cs[no-index]{ekvcValue}\{\key\}\{\#1\}} (or whichever argument the
%   hashed \kv\ list will be).
% \end{function}
%
% \begin{function}{\ekvcHashAndUse}
%   \begin{syntax}
%     \cs{ekvcHashAndUse}\meta{cs}\marg{primary keys}
%   \end{syntax}
%   This will roughly do the same as |\ekvcHashAndForward|, but instead of
%   specifying what will be used after hashing the keys, \meta{cs} will use what
%   follows the \kv\ list. So its syntax will be
% \end{function}
%   \begin{syntax}
%     \meta{cs}\{\key=\val, \ldots\}\marg{after}
%   \end{syntax}
%
% \begin{function}{\ekvcValue}
%   \begin{syntax}
%     \cs{ekvcValue}\{\key\}\marg{key list}
%   \end{syntax}
%   This is a safe way to access your keys in a hash variant. \key\
%   is the key which's \val\ you want to use out of the \meta{key list}.
%   \meta{key list} should be the key list argument forwarded to your underlying
%   macro by |\ekvcHash|, |\ekvcHashAndForward|, or |\ekvcHashAndUse|. It will
%   be tested whether the hash function to access that \key\ exists, the \key\
%   argument is not empty, and that the \meta{key list} really contains a \val\
%   of that \key. This macro needs exactly two steps of expansion and if used
%   inside of an |\edef| or |\expanded| context will protect the \val\ from
%   further expanding.
% \end{function}
%
% \begin{function}{\ekvcValueFast}
%   \begin{syntax}
%     \cs{ekvcValueFast}\{\key\}\marg{key list}
%   \end{syntax}
%   This behaves just like |\ekvcValue|, but \emph{without any} safety tests. As
%   a result this is about $1.4$ times faster \emph{but} will throw low level
%   \TeX\ errors eventually if the hash function isn't defined or the \key\
%   isn't part of the \meta{key list} (\emph{e.g.}, because it was defined as a
%   key for another macro -- all macros share the same hash function per \key).
%   Use it if you know what you're doing. This macro needs exactly three steps
%   of expansion in the no-errors case.
% \end{function}
%
% \begin{function}{\ekvcValueSplit}
%   \begin{syntax}
%     \cs{ekvcValueSplit}\{\key\}\marg{key list}\marg{next}
%   \end{syntax}
%   If you need a specific \key\ from a \meta{key list} more than once, it'll
%   be a good idea to only extract it once and from then on keep it as a
%   separate argument. Hence the macro \cs{ekvcValueSplit} will extract one
%   specific \key's value from the list and forward the remainder of the list
%   as the first and the \key's value as the second argument to \meta{next}, so
%   the result of this will be \meta{next}\marg{key list'}\marg{value} with
%   \meta{key list'} the remaining list. This is almost as fast as |\ekvcValue|
%   and runs the same tests. Keep in mind that you can't fetch for the same
%   \key\ again from \meta{key list'} as it got removed.
% \end{function}
% \example The following defines a macro |\foo| which will take three keys.
% Since the next parsing step will need the value of one of the keys multiple
% times we split that key off the list:\\
% \begin{minipage}{.75\linewidth}
% \begin{lstlisting}
% \ekvcHash\foo{a=a,b=b,c=c}
%   {\ekvcValueSplit{a}{#1}\foobar}
% \newcommand*\foobar[2]{a is #2.\par
%                        b is \ekvcValue{b}{#1}.\par
%                        c is \ekvcValue{c}{#1}.\par}
% \foo{}
% \end{lstlisting}
% \end{minipage}^^A
% \begin{exresult}[nobeforeafter,box align=center]{.25\linewidth}
%   \ekvcHash\foo{a=a,b=b,c=c}
%     {\ekvcValueSplit{a}{#1}\foobar}
%   \newcommand*\foobar[2]{a is #2.\par
%                          b is \ekvcValue{b}{#1}.\par
%                          c is \ekvcValue{c}{#1}.\par}
%   \foo{}
% \end{exresult}
%
% \begin{function}{\ekvcValueSplitFast}
%   \begin{syntax}
%     \cs{ekvcValueSplitFast}\{\key\}\marg{key list}\marg{next}
%   \end{syntax}
%   This behaves just like |\ekvcValueSplit|, but it won't run the same tests,
%   hence it is faster but more error prone, just like the relation between
%   |\ekvcValue| and |\ekvcValueFast|.
% \end{function}
%
%
% \subsection{Secondary Keys}\label{sec:secondaries}
%
% To remove some of the limitations with the approach that each primary key
% matches an argument or hash entry, you can define secondary keys. Those have
% to be defined for each macro but it doesn't matter whether that macro was a
% split or a hash variant. If a secondary key references another key it doesn't
% matter whether that other key is primary or secondary.
%
% Secondary keys can have a prefix (like |long|) which are called |p|-type
% prefix and must have a type (like |meta|) which are called |t|-type prefix.
% Some types might require some |p|-prefixes, while others might forbid those.
%
% Please keep in mind that key names shouldn't start with \texttt{EKVC\string|}.
%
% \begin{function}{\ekvcSecondaryKeys}
%   \begin{syntax}
%     \cs{ekvcSecondaryKeys}\meta{cs}\{\kv, \ldots\}
%   \end{syntax}
%   This is the front facing macro to define secondary keys. For the macro
%   \meta{cs} define \key\ to have definition \val. The general syntax for \key\
%   should be
%   \begin{quote}\ttfamily\small
%     \meta{prefix} \meta{name}
%   \end{quote}
%   Where \meta{prefix} is a space separated list of optional |p|-type prefixes
%   followed by one |t|-type prefix. The syntax of \val\ is dependent on the
%   used |t|-prefix.
% \end{function}
%
%
% \subsubsection{\texttt{p}-type Prefixes}
%
% There is only one |p|-prefix available, which is |long|.
%
% \begin{function}{long}
%   The following key will be defined |\long|.
% \end{function}
%
%
% \subsubsection{\texttt{t}-type Prefixes}
%
% If you're familiar with \expkvd\ you'll notice that the |t|-type prefixes
% provided here are much fewer. The expansion only concept doesn't allow for
% great variety in the auto-defined keys.
%
% The syntax examples of the |t|-prefixes will show which |p|-prefix will be
% automatically used by printing those black (\texttt{\enflong}), which
% will be available in grey (\texttt{\alllong}), and which will be disallowed in
% red (\texttt{\notlong}). This will be put flush right next to the syntax line.
%
% \begin{function}{meta}
%   \begin{syntax}
%     meta \key\ = \{\kv, \ldots\} \prefixes2
%   \end{syntax}
%   With a |meta| key you can set other keys. Whenever \key\ is used the keys in
%   the \kv\ list will be set to the values given there. You can use the \val\
%   given to \key\ by using |#1| in the \kv\ list. The keys in the \kv\ list can
%   be primary and secondary ones.
% \end{function}
%
% \begin{function}{nmeta}
%   \begin{syntax}
%     nmeta \key\ = \{\kv, \ldots\} \prefixes3
%   \end{syntax}
%   An |nmeta| key is like a |meta| key, but it doesn't take a value, so the
%   \kv\ list is static.
% \end{function}
%
% \begin{function}{alias}
%   \begin{syntax}
%     alias \key\ = \meta{key_2} \prefixes3
%   \end{syntax}
%   This assigns the definition of \meta{key_2} to \key. As a result \key\ is an
%   alias for \meta{key_2} behaving just the same. Both the value taking and the
%   |NoVal| version (that's \expkv\ slang for a key not accepting a value) will
%   be copied if they are defined when |alias| is used. Of course, \meta{key_2}
%   has to be defined, be it as a primary or secondary one.
% \end{function}
%
% \begin{function}{default}
%   \begin{syntax}
%     default \key\ = \marg{default} \prefixes3
%   \end{syntax}
%   If \key\ is a defined value taking key, you can define a |NoVal| version
%   with this that will behave as if \key\ was given \meta{default} as its
%   \val. Note that this doesn't change the initial values of primary keys set
%   at definition time in |\ekvcSplit| and friends. \key\ can be a primary or
%   secondary key.
% \end{function}
%
% \begin{function}{flag-bool}
%   \begin{syntax}
%     flag-bool \key\ = \meta{cs} \prefixes3
%   \end{syntax}
%   This is a secondary key that doesn't directly involve any of the primary or
%   secondary keys.
%   This defines \key\ to take a value, which should be either |true| or
%   |false|, and set the flag called \meta{cs} accordingly as a boolean.
%   If \meta{cs} isn't defined yet it will be initialised as a flag.
%   Please also read \autoref{sec:flags}.
% \end{function}
%
% \begin{function}{flag-true,flag-false}
%   \begin{syntax}
%     flag-true \key\ = \meta{cs} \prefixes3
%   \end{syntax}
%   This is a secondary key that doesn't directly involve any of the primary or
%   secondary keys.
%   This defines \key\ to take no value and set the flag called \meta{cs} to
%   |true| or |false|, respectively.
%   If \meta{cs} isn't defined yet it will be initialised as a flag.
%   Please also read \autoref{sec:flags}.
% \end{function}
%
% \begin{function}{flag-raise}
%   \begin{syntax}
%     flag-raise \key\ = \meta{cs} \prefixes3
%   \end{syntax}
%   This is a secondary key that doesn't directly involve any of the primary or
%   secondary keys.
%   This defines \key\ to take no value and raise the flag called \meta{cs}.
%   If \meta{cs} isn't defined yet it will be initialised as a flag.
%   Please also read \autoref{sec:flags}.
% \end{function}
%
%
% \subsection{Changing the Initial Values}\label{sec:change}
%
% \begin{function}{\ekvcChange}
%   \begin{syntax}
%     \cs{ekvcChange}\meta{cs}\{\kv,\ldots\}
%   \end{syntax}
%   This processes the \kv\ list (just like it would be processed by \meta{cs})
%   to set new defaults for the macro \meta{cs} defined with \expkvc\ (meaning
%   the values used if you don't provide anything at use time). Inside the \kv\
%   list both primary and secondary keys can be used. If \meta{cs} was defined
%   |\long| earlier it will still be |\long|, every other \TeX\ prefix will be
%   stripped (but \expkvc\ doesn't support them anywhere else so that should be
%   fine). The resulting new defaults will be stored inside the \meta{cs}
%   locally (just as the original defaults were).
% \end{function}
%
% Consider the following example:\\
% \begin{minipage}[c]{.75\linewidth}
% \begin{lstlisting}
% \ekvcSplit\foo{a=a,b=b}{a is #1.\par b is #2.\par}
% \foo{} % initial defaults
% \ekvcChange\foo{b=B}
% \foo{} % new defaults
% \ekvcSecondaryKeys\foo{meta c={a={#1},b={#1}}}
% \ekvcChange\foo{c=c}
% \foo{} % newer defaults
% \end{lstlisting}
% \end{minipage}^^A
% \begin{exresult}[nobeforeafter,box align=center]{.25\linewidth}
%   \ekvcSplit\foo{a=a,b=b}{a is #1.\par b is #2.\par}
%   \foo{}
%   \ekvcChange\foo{b=B}
%   \foo{}
%   \ekvcSecondaryKeys\foo{meta c={a={#1},b={#1}}}
%   \ekvcChange\foo{c=c}
%   \foo{}
% \end{exresult}
%
% As a result with this the typical setup macro could be implemented:
% \begin{lstlisting}
% \ekvcHashAndUse\foo{key=a,key=b}
% \newcommand\foosetup{\ekvcChange\foo}
% \end{lstlisting}
% Of course the usage is limited to a single macro |\foo|, hence this might be
% not as powerful as similar macros used with other \kv\ interfaces. But at
% least a few similar macros could be grouped using the same key parsing macro
% internally.
%
%
% \subsection{Flags}\label{sec:flags}
%
% The idea of flags is taken from \pkg{expl3}. They provide a way to store
% numerical information expandably, however only incrementing and accessing
% works expandably, decrementing is unexpandable. A flag has a height, which
% is a numerical value, and which can be raised by $1$. Flags come at a high
% computational cost (accessing them is slow and they require more memory than
% normal \TeX\ data types like registers, both getting linearly worse with the
% height), so don't use them if not necessary.
%
% The state of flags is always changed locally to the current group, but not to
% the current macro, so if you're using one of the |t|-types involving flags
% bear in mind that they can affect other macros using the same flags at the
% current group level!
%
% \expkvc\ provides some macros to access, alter, and use flags. Flags of
% \expkvc\ don't share a name space with the flags of \pkg{expl3}.
%
% \begin{function}{\ekvcFlagNew}
%   \begin{syntax}
%     \cs{ekvcFlagNew}\meta{flag}
%   \end{syntax}
%   This initialises the macro \meta{flag} as a new flag. It isn't checked
%   whether the macro \meta{flag} is currently undefined.  A \meta{flag} will
%   expand to the flag's current height with a trailing space (so you can use it
%   directly with |\ifnum| for example and it will terminate the number scanning
%   itself).
% \end{function}
%
%  All other macros dealing with flags take as a parameter a macro defined as a
%  \meta{flag} with |\ekvcFlagNew|.
%
% \begin{function}{\ekvcFlagHeight}
%   \begin{syntax}
%     \cs{ekvcFlagHeight}\meta{flag}
%   \end{syntax}
%   This expands to the current height of \meta{flag} in a single step of
%   expansion (without a trailing space).
% \end{function}
%
% \begin{function}{\ekvcFlagRaise}
%   \begin{syntax}
%     \cs{ekvcFlagRaise}\meta{flag}
%   \end{syntax}
%   This expandably raises the height of \meta{flag} by $1$.
% \end{function}
%
% \begin{function}{\ekvcFlagSetTrue,\ekvcFlagSetFalse}
%   \begin{syntax}
%     \cs{ekvcFlagSetTrue}\meta{flag}
%   \end{syntax}
%   By interpreting an even value as false and an odd value as true we can use a
%   flag as a boolean. This expandably sets \meta{flag} to true or false,
%   respectively, by raising it if necessary.
% \end{function}
%
% \begin{function}{\ekvcFlagIf}
%   \begin{syntax}
%     \cs{ekvcFlagIf}\meta{flag}\marg{true}\marg{false}
%   \end{syntax}
%   This interprets a \meta{flag} as a boolean and expands to either \meta{true}
%   or \meta{false}.
% \end{function}
%
% \begin{function}{\ekvcFlagIfRaised}
%   \begin{syntax}
%     \cs{ekvcFlagIfRaised}\meta{flag}\marg{true}\marg{false}
%   \end{syntax}
%   This tests whether the \meta{flag} is raised, meaning it has a height
%   greater than zero, and if so expands to \meta{true} else to \meta{false}.
% \end{function}
%
% \begin{function}{\ekvcFlagReset}
%   \begin{syntax}
%     \cs{ekvcFlagReset}\meta{flag}
%   \end{syntax}
%   This resets a flag (so restores its height to $0$). This operation is
%   \emph{not} expandable and done locally. If you really intend to use flags
%   you can reset them every now and then to keep the performance hit low.
% \end{function}
%
% \begin{function}{\ekvcFlagGetHeight}
%   \begin{syntax}
%     \cs{ekvcFlagGetHeight}\meta{flag}\marg{next}
%   \end{syntax}
%   This retrieves the current height of the \meta{flag} and provides it as a
%   braced argument to \meta{next}, leaving \meta{next}\marg{height} in the
%   input stream.
% \end{function}
%
% \begin{function}{\ekvcFlagGetHeights}
%   \begin{syntax}
%     \cs{ekvcFlagGetHeights}\marg{flag-list}\marg{next}
%   \end{syntax}
%   This retrieves the current height of each \meta{flag} in the
%   \meta{flag-list} and provides them as a single argument to \meta{next}.
%   Inside that argument each height is enclosed in a set of braces
%   individually. The \meta{flag-list} is just a single argument containing the
%   \meta{flag}s.
%   So a usage like
%   \texttt
%     {^^A
%       \cs[no-index]{ekvcFlagGetHeights}\hskip0pt^^A
%       \{\cs[no-index]{myflagA}\cs[no-index]{myflagB}\}\hskip0pt^^A
%       \{\cs[no-index]{stuff}\}^^A
%     }
%   will expand to
%   \texttt{\cs[no-index]{stuff}\{\marg{height-A}\marg{height-B}\}}.
% \end{function}
%
%
% \subsection{Further Examples}
%
% How could a documentation be a good documentation without some basic examples?
% Say we want to define a small macro expanding to some character description
% (who knows why this has to be expandable?). A character description will not
% have too many items to it, so we use |\ekvcSplit| (the comments with the
% parameter numbers are of course not necessary and just ease reading the
% example).
% \begin{lstlisting}
% \ekvcSplit\character
%   {
%     name=John Doe,            % #1
%     age=any,                  % #2
%     nationality=the Universe, % #3
%     hobby=to exist,           % #4
%     type=Mister,              % #5
%     pronoun=He,               % #6
%     possessive=his,           % #7
%   }
%   {#1 is a #5 from #3. #6 is of #2 age and #7 hobby is #4.\par}
% \end{lstlisting}
% Also we want to give some short cuts so that it's easier to describe several
% persons.
% \begin{lstlisting}
% \ekvcSecondaryKeys\character
%   {
%     alias pro = pronoun,
%     alias pos = possessive,
%     nmeta me =
%       {
%         name=Jonathan,
%         age=a young,
%         nationality=Germany,
%         hobby=\TeX\ coding,
%       },
%     meta lady =
%       {type=Lady, pronoun=She, possessive=her, name=Jane Doe, #1},
%     nmeta paulo =
%       {
%         name=Paulo,
%         type=duck,
%         age=a young,
%         nationality=Brazil,
%         hobby=to quack,
%       }
%   }
% \end{lstlisting}
% Now we can describe people using
% \begin{lstlisting}
% \character{}
% \character{me}
% \character{paulo}
% \character
%   {lady={name=Evelyn,nationality=Ireland,age=the best,hobby=reading}}
% \character
%   {
%     name=Our sun, type=star, nationality=our solar system, pro=It,
%     age=an old, pos=its, hobby=shining
%   }
% \end{lstlisting}
% As one might see, the |lady| key could actually have been an |nmeta| key as
% well, as all that is done with the argument is using it as a \kv\ list.
%
% The result of only the first two usages would be:
% \begin{exresult}{\linewidth}
%   \ekvcSplit\character
%     {
%       name=John Doe,
%       age=any,
%       nationality=the Universe,
%       hobby=to exist,
%       type=Mister,
%       pronoun=He,
%       possessive=his,
%     }
%     {#1 is a #5 from #3. #6 is of #2 age and #7 hobby is #4.\par}
%   \ekvcSecondaryKeys\character
%     {
%       alias pro = pronoun,
%       alias pos = possessive,
%       nmeta me =
%         {
%           name=Jonathan,
%           age=a young,
%           nationality=Germany,
%           hobby=\TeX\ coding,
%         },
%       meta lady =
%         {type=Lady, pronoun=She, possessive=her, name=Jane Doe, #1},
%       nmeta paulo =
%         {
%           name=Paulo,
%           type=duck,
%           age=a young,
%           nationality=Brazil,
%           hobby=to quack,
%         }
%     }
%   \character{}
%   \character{me}
% \end{exresult}
%
% Using \pkg{xparse} or \expkv's \cs{ekvoptarg} or \cs{ekvoptargTF} and
% forwarding arguments one can easily define \kv\ macros with actual optional
% and mandatory arguments as well. A small nonsense example (which should
% perhaps use |\ekvcSplitAndForward| instead of |\ekvcHashAndForward| since it
% only uses four keys and one other argument -- and isn't expandable since it
% uses a \env{tabular} environment, so it would've been better to use a more
% feature rich \kv\ interface most likely, \emph{e.g.}, the one provided by
% \expkvd):
% \begin{lstlisting}
% \makeatletter
% \newcommand*\nonsense{\ekvoptarg\nonsense@a{}}
% \ekvcHashAndForward\nonsense@a\nonsense@b
%   {
%     keyA = A,
%     keyB = B,
%     keyC = c,
%     keyD = d,
%   }
% \newcommand*\nonsense@b[2]
%   {%
%     \begin{tabular}{lll}
%       key & A & \ekvcValue{keyA}{#1} \\
%           & B & \ekvcValue{keyB}{#1} \\
%           & C & \ekvcValue{keyC}{#1} \\
%           & D & \ekvcValue{keyD}{#1} \\
%       \multicolumn{2}{l}{mandatory} & #2 \\
%     \end{tabular}%
%   }
% \makeatother
% \end{lstlisting}
% And then we would be able to do some nonsense
% \begin{lstlisting}
% \nonsense{}
% \nonsense[keyA=hihi]{haha}
% \nonsense[keyA=hihi, keyB=A]{hehe}
% \nonsense[keyC=huhu, keyA=hihi, keyB=A]{haha}
% \end{lstlisting}
% resulting in
% \begin{exresult}{\linewidth}
%   \makeatletter
%   \newcommand*\nonsense{\ekvoptarg\nonsense@a{}}
%   \ekvcHashAndForward\nonsense@a\nonsense@b
%     {
%       keyA = A,
%       keyB = B,
%       keyC = c,
%       keyD = d,
%     }
%   \newcommand*\nonsense@b[2]
%     {%
%       \begin{tabular}{lll}
%         key & A & \ekvcValue{keyA}{#1} \\
%             & B & \ekvcValue{keyB}{#1} \\
%             & C & \ekvcValue{keyC}{#1} \\
%             & D & \ekvcValue{keyD}{#1} \\
%         \multicolumn{2}{l}{mandatory} & #2 \\
%       \end{tabular}%
%     }
%   \makeatother
%   \nonsense{}
%   \nonsense[keyA=hihi]{haha}
%   \nonsense[keyA=hihi, keyB=A]{hehe}
%   \nonsense[keyC=huhu, keyA=hihi, keyB=A]{haha}
% \end{exresult}
%
%
% \subsection{Freedom for Keys!}
%
% If this was the \TeX book this subsection would have a double bend sign. Not
% because it is overly complicated, but because it shows things which could
% break \expkvc's expandability and its alignment safety. This is for
% experienced users wanting to get the most flexibility and knowing what they
% are doing.
%
% In case you're wondering, it is possible to define other keys than the
% primaries and the secondary types listed in \autoref{sec:secondaries} for a
% macro defined with \expkvc\ by using the low-level interface of \expkv\ or
% even the interface provided by \expkvd. The set name used for \expkvc's keys
% is the macro name, including the leading backslash, or more precisely
% |\string|\meta{cs} is used. This can be exploited to define additional keys
% with arbitrary code. Consider the following \emph{bad} example:
% \begin{lstlisting}
% \ekvcSplit\foo{a=A,b=B}{a is #1.\par b is #2}
% \protected\ekvdef{\string\foo}{c}{\def\fooC{#1}}
% \end{lstlisting}
% This would define a key named |c| that will store its value inside a macro.
% The issue with this is that this can't be done expandably. As a result, the
% macro |\foo| isn't always expandable any more (not that bad if this was never
% required; killjoy if it was) and as soon as the key |c| is used, it is also no
% longer alignment safe\footnote{This means that the \kv-list can't contain
%   alignment markers that are not inside an additional set of braces if used
%   inside a \TeX\ alignment.}
% (might be bad depending on the usage).
%
% So why do I show you this? Because we could as well do something useful like
% create a key that pre-parses the input and after that passes the parsed value
% on. This parsing would have to be completely expandable though. For the
% pass-on part we can use the following function:
%
% \begin{function}{\ekvcPass}
%   \begin{syntax}
%     \cs{ekvcPass}\meta{cs}\marg{key}\marg{value}
%   \end{syntax}
%   This passes \meta{value} on to \meta{key} for the \expkvc-macro \meta{cs}.
%   It should be used inside the key parsing of a macro defined with \expkvc,
%   else this most likely results in a low level \TeX\ error.
% \end{function}
%
% With this we could for example split the value of a key at a hyphen and pass
% the parts to different keys:\\
% \begin{minipage}[c]{.75\linewidth}
% \begin{lstlisting}
% \ekvcSplit\foo{a=A,b=B}{a is #1.\par b is #2.\par}
% \ekvdef{\string\foo}{c}{\fooSplit#1\par}
% \def\fooSplit#1-#2\par
%   {\ekvcPass\foo{a}{#1}\ekvcPass\foo{b}{#2}}
% \foo{}
% \foo{c=1-2}
% \end{lstlisting}
% \end{minipage}
% \begin{exresult}[nobeforeafter,box align=center]{.25\linewidth}
%   \ekvcSplit\foo{a=A,b=B}{a is #1.\par b is #2.\par}
%   \ekvdef{\string\foo}{c}{\fooSplit#1\par}
%   \def\fooSplit#1-#2\par{\ekvcPass\foo{a}{#1}\ekvcPass\foo{b}{#2}}
%   \foo{}
%   \foo{c=1-2}
% \end{exresult}
%
%
% \subsection{Speed Considerations}\label{sec:speed}
%
% As already mentioned in the introduction there are some speed considerations
% implied if you choose to define macros via \expkvc. However the overhead isn't
% the factor which should hinder you to use \expkvc\ if you found a reasonable
% use case. The key-parsing is still faster than with most other \kv\ packages
% (see the ``Comparisons'' subsection in the \expkv\ documentation).
%
% The speed considerations in this subsection use the first example in this
% documentation as the benchmark. So we have seven keys and a short sentence
% which should be typeset. For comparisons I use the following equivalent
% \expkvd\ definitions. Each result is the average between changing no keys from
% their initial values and altering four. Furthermore I'll compare three
% variants of \expkvc\ with the \expkvd\ definitions, namely the split example
% from above, a hash variant using |\ekvcValue| and a hash variant using
% |\ekvcValueFast|.
% \begin{lstlisting}
% \usepackage{expkv-def}
% \ekvdefinekeys{keys}
%   {%
%     ,store   name         = \KEYSname
%     ,initial name         = John Doe
%     ,store   age          = \KEYSage
%     ,initial age          = any
%     ,store   nationality  = \KEYSnationality
%     ,initial nationality  = the Universe
%     ,store   hobby        = \KEYShobby
%     ,initial hobby        = to exist
%     ,store   type         = \KEYStype
%     ,initial type         = Mister
%     ,store   pronoun      = \KEYSpronoun
%     ,initial pronoun      = He
%     ,store   possessive   = \KEYSpossessive
%     ,initial possessive   = his
%   }
% \newcommand*\KEYS[1]
%   {%
%     \begingroup
%       \ekvset{keys}{#1}%
%       \KEYSname\ is a \KEYStype\ from \KEYSnationality. \KEYSpronoun\ is
%       of \KEYSage\ age and \KEYSpossessive\ hobby is \KEYShobby.%
%     \endgroup
%   }
% \end{lstlisting}
%
% The first comparison removes the typesetting part from all the definitions, so
% that only the key parsing is compared. In this comparison the |\ekvcValue|
% and |\ekvcValueFast| variants will not differ, as they are exactly the same
% until the key usage. We find that the split approach is $1.4$ times slower
% than the \expkvd\ setup and the hash variants end up in the middle at $1.17$
% times slower.
%
% Next we put the typesetting part back in. Every call of the macros will
% typeset the sentences into a box register in horizontal mode. With the
% typesetting part (which includes the accessing of values) the fastest remains
% the \expkvd\ definitions, but split is close at $1.16$ times slower, followed
% by the hash variant with fast accesses at $1.36$ times slower, and the safe
% hash access variant ranks in the slowest $1.8$ times slower than \expkvd.
%
% Just in case you're wondering now, a simple macro taking seven arguments is
% $30$ to $40$ times faster than any of those in the argument grabbing and \kv\
% parsing part and only $1.5$ to $2.8$ times faster if the typesetting part is
% factored in. So the real choke isn't the parsing.
%
% So to summarize this, if you have a reasonable use case for expandable \kv\
% parsing macros you should go on and define them using \expkvc. If you have a
% reasonable use case for \kv\ parsing macros but defining them expandable isn't
% necessary for your use you should take advantage of the greater flexibility of
% non-expandable \kv\ setups (but if you're after maximum speed there aren't
% that many \kv\ parsers beating \expkvc). And if you are after maximum
% performance maybe ditching the \kv\ interface altogether is a good idea, but
% depending on the number of arguments your interface might get convoluted.
%
%
% \subsection{Useless Macros}
%
% Perhaps these macros aren't completely useless, but I figured from a user's
% point of view I wouldn't know what I should do with these.
%
% \begin{function}{\ekvcDate,\ekvcVersion}
%   These two macros store the version and the date of the package/generic code.
% \end{function}
%
%
% \subsection{Bugs}
%
% Of course I don't think there are any bugs (who would knowingly distribute
% buggy software as long as he isn't a multi-million dollar corporation?). But
% if you find some please let me know. For this one might find my email address
% on the first page or file an issue on Github:
% \url{https://github.com/Skillmon/tex_expkv-cs}
%
%
% \subsection{License}
%
% Copyright \textcopyright\ 2020\unless\ifnum\year=2020--\the\year\fi\
% Jonathan P. Spratte
%
% \medskip\noindent
% This work may be distributed and/or modified under the conditions of the
% \LaTeX\ Project Public License (LPPL), either version 1.3c of this license or
% (at your option) any later version. The latest version of this license is in
% the file:
%
% \url{http://www.latex-project.org/lppl.txt}
%
% \noindent
% This work is ``maintained'' (as per LPPL maintenance status) by
%
% \mbox{Jonathan P. Spratte}.
%
% \end{documentation}^^A=<<
%
% \begin{implementation}^^A>>=
%
% \clearpage
%
% \section{Implementation}^^A>>=
%^^A the LaTeX package >>=
% \subsection{The \LaTeX\ Package}
% Just like for \expkv\ we provide a small \LaTeX\ package that sets up things
% such that we behave nicely on \LaTeX\ packages and files system. It'll
% |\input| the generic code which implements the functionality.
% \gobbledocstriptag
%<*pkg>
%    \begin{macrocode}
\RequirePackage{expkv}
\def\ekvc@tmp
  {%
    \ProvidesFile{expkv-cs.tex}%
      [%
        \ekvcDate\space v\ekvcVersion\space
        define expandable key=val macros using expkv%
      ]%
  }
% \iffalse meta-comment
%
% File: expkv-cs.dtx Copyright (C) 2020-2021 Jonathan P. Spratte
%
% This work  may be  distributed and/or  modified under  the conditions  of the
% LaTeX Project Public License (LPPL),  either version 1.3c  of this license or
% (at your option) any later version.  The latest version of this license is in
% the file:
%
%   http://www.latex-project.org/lppl.txt
%
% ------------------------------------------------------------------------------
%
%<*driver>^^A>>=
\def\nameofplainTeX{plain}
\ifx\fmtname\nameofplainTeX\else
  \expandafter\begingroup
\fi
\input l3docstrip.tex
\askforoverwritefalse
\preamble

--------------------------------------------------------------
expkv-cs -- define expandable key=val macros using expkv
E-mail: jspratte@yahoo.de
Released under the LaTeX Project Public License v1.3c or later
See http://www.latex-project.org/lppl.txt
--------------------------------------------------------------

Copyright (C) 2020-2021 Jonathan P. Spratte

This  work may be  distributed and/or  modified under  the conditions  of the
LaTeX Project Public License (LPPL),  either version 1.3c  of this license or
(at your option) any later version.  The latest version of this license is in
the file:

  http://www.latex-project.org/lppl.txt

This work is "maintained" (as per LPPL maintenance status) by
  Jonathan P. Spratte.

This work consists of the file  expkv-cs.dtx
and the derived files           expkv-cs.pdf
                                expkv-cs.sty
                                expkv-cs.tex

\endpreamble
% stop docstrip adding \endinput
\postamble
\endpostamble
\generate{\file{expkv-cs.sty}{\from{expkv-cs.dtx}{pkg}}}
\generate{\file{expkv-cs.tex}{\from{expkv-cs.dtx}{tex}}}
\ifx\fmtname\nameofplainTeX
  \expandafter\endbatchfile
\else
  \expandafter\endgroup
\fi
%
\IfFileExists{expkv-cs.tex}{% \iffalse meta-comment
%
% File: expkv-cs.dtx Copyright (C) 2020-2021 Jonathan P. Spratte
%
% This work  may be  distributed and/or  modified under  the conditions  of the
% LaTeX Project Public License (LPPL),  either version 1.3c  of this license or
% (at your option) any later version.  The latest version of this license is in
% the file:
%
%   http://www.latex-project.org/lppl.txt
%
% ------------------------------------------------------------------------------
%
%<*driver>^^A>>=
\def\nameofplainTeX{plain}
\ifx\fmtname\nameofplainTeX\else
  \expandafter\begingroup
\fi
\input l3docstrip.tex
\askforoverwritefalse
\preamble

--------------------------------------------------------------
expkv-cs -- define expandable key=val macros using expkv
E-mail: jspratte@yahoo.de
Released under the LaTeX Project Public License v1.3c or later
See http://www.latex-project.org/lppl.txt
--------------------------------------------------------------

Copyright (C) 2020-2021 Jonathan P. Spratte

This  work may be  distributed and/or  modified under  the conditions  of the
LaTeX Project Public License (LPPL),  either version 1.3c  of this license or
(at your option) any later version.  The latest version of this license is in
the file:

  http://www.latex-project.org/lppl.txt

This work is "maintained" (as per LPPL maintenance status) by
  Jonathan P. Spratte.

This work consists of the file  expkv-cs.dtx
and the derived files           expkv-cs.pdf
                                expkv-cs.sty
                                expkv-cs.tex

\endpreamble
% stop docstrip adding \endinput
\postamble
\endpostamble
\generate{\file{expkv-cs.sty}{\from{expkv-cs.dtx}{pkg}}}
\generate{\file{expkv-cs.tex}{\from{expkv-cs.dtx}{tex}}}
\ifx\fmtname\nameofplainTeX
  \expandafter\endbatchfile
\else
  \expandafter\endgroup
\fi
%
\IfFileExists{expkv-cs.tex}{\input{expkv-cs.tex}}{}
\ProvidesFile{expkv-cs.dtx}
  [\csname ekvcDate\endcsname\ define expandable key=val macros using expkv]
\PassOptionsToPackage{full}{textcomp}
\documentclass{l3doc}
\RequirePackage[oldstylenums,nott]{kpfonts}
\input{glyphtounicode}
\pdfgentounicode=1
\RequirePackage{tcolorbox}
\newtcolorbox{exresult}[2][]
  {colback=expkvgrey!10!white,colframe=expkvgrey,fontupper=\small,width={#2},#1}
\newtcbox\exres[1][]
  {
     colback=expkvgrey!10!white
    ,colframe=expkvgrey
    ,size=small
    ,nobeforeafter
    ,tcbox raise base
    ,fontupper=\small
    ,#1
  }
\RequirePackage{listings}
\RequirePackage{booktabs}
\RequirePackage{array}
\RequirePackage{collcell}
\RequirePackage{siunitx}
\RequirePackage{xcolor}
\RequirePackage{caption}
\RequirePackage{microtype}
\RequirePackage{accsupp}
\lstdefinelanguage{expkvc}[primitive]{TeX}
  {
    ,moretexcs=[4]^^A e-TeX
      {
        expanded,
        numexpr,
        protected,
      }
    ,moretexcs=[5]^^A plain/LaTeX
      {
        approx,
        begin,
        empty,
        item,
        LaTeX,
        makeatletter,makeatother,multicolumn,
        newcommand,newdimen,
        RequirePackage,
        rule,
        TeX,
        textit,texttt,
        usepackage,
      }
    ,moretexcs=[6]^^A used packages
      {
        ^^A xfp
        fpeval,
        ^^A keyval
        setkeys,
        ^^A yax
        defactiveparameter,storevalue,setparameterlist
      }
    ,moretexcs=[2]^^A expkv-cs macros
      {
        ekvcSplit,ekvcSplitAndForward,ekvcSplitAndUse,
        ekvcHash,ekvcHashAndForward,ekvcHashAndUse,
        ekvcValue,ekvcValueFast,ekvcValueSplit,ekvcValueSplitFast,
        ekvcSecondaryKeys,
        ekvcChange,
        ekvcFlagNew,
        ekvcFlagHeight,
        ekvcFlagRaise,
        ekvcFlagSetTrue,ekvcFlagSetFalse,
        ekvcFlagIf,ekvcFlagIfRaised,
        ekvcFlagReset,
        ekvcFlagGetHeight,ekvcFlagGetHeights,
        ekvcPass
      }
    ,moretexcs=[3]^^A expkv-pkg macros
      {
        ^^A expkv
        ekvdef,ekvdefNoVal,^^A
        ekvlet,ekvletNoVal,ekvletkv,ekvletkvNoVal,^^A
        ekvdefunknown,ekvdefunknownNoVal,^^A
        ekvredirectunknown,ekvredirectunknownNoVal,^^A
        ekvset,ekvsetSneaked,^^A
        ekvsetdef,ekvsetSneakeddef,ekvsetdefSneaked,^^A
        ekvparse,^^A
        ekvVersion,ekvDate,^^A
        ekvifdefined,ekvifdefinedNoVal,ekvifdefinedset,^^A
        ekvbreak,ekvbreakPreSneak,ekvbreakPostSneak,^^A
        ekvsneak,ekvsneakPre,^^A
        ekvchangeset,^^A
        ekvoptarg,ekvoptargTF,^^A
        ekverr,
        ^^A expkv-def
        ekvdefinekeys
      }
  }
\colorlet{codeparam}{cyan!65!black}
\def\mylstwd{.55em}
\lstset
  {
    ,language=expkvc
    ,flexiblecolumns=false
    ,basewidth=\mylstwd
    ,gobble=2
    ,basicstyle=\fontfamily{jkp}\itshape
    ,texcsstyle=*[1]{\bfseries\color{expkvgrey}} ^^A primitives
    ,texcsstyle=*[2]{\bfseries\color{expkvred}}  ^^A expkv
    ,texcsstyle=*[3]{\color{expkvred}}           ^^A expkv-pkg
    ,texcsstyle=*[4]{\bfseries\color{expkvgrey}} ^^A e-TeX
    ,texcsstyle=*[5]{\bfseries\color{expkvgrey}} ^^A plain/LaTeX macros
    ,texcsstyle=*[6]{}                           ^^A macros of other packages
    ,commentstyle=\color[gray]{0.4}
    ,literate=
              {\{} {{\CodeUpBf\{}}{1}
              {\}} {{\CodeUpBf\}}}{1}
              {$}  {{\CodeUpBf\$}}{1}
              {[}  {{\CodeUp[}}{1}
              {]}  {{\CodeUp]}}{1}
              {(}  {{\CodeUp(}}{1}
              {)}  {{\CodeUp)}}{1}
              {*}  {{$*$}}{1}
              {1}  {{\CodeUp{1}}}{1}
              {2}  {{\CodeUp{2}}}{1}
              {3}  {{\CodeUp{3}}}{1}
              {4}  {{\CodeUp{4}}}{1}
              {5}  {{\CodeUp{5}}}{1}
              {6}  {{\CodeUp{6}}}{1}
              {7}  {{\CodeUp{7}}}{1}
              {8}  {{\CodeUp{8}}}{1}
              {9}  {{\CodeUp{9}}}{1}
              {0}  {{\CodeUp{0}}}{1}
              {##} {{\CodeColored{codeparam}{1}{\#}}}{1}
              {##1}{{\CodeColored{codeparam}{2}{\#1}}}{2}
              {##2}{{\CodeColored{codeparam}{2}{\#2}}}{2}
              {##3}{{\CodeColored{codeparam}{2}{\#3}}}{2}
              {##4}{{\CodeColored{codeparam}{2}{\#4}}}{2}
              {##5}{{\CodeColored{codeparam}{2}{\#5}}}{2}
              {##6}{{\CodeColored{codeparam}{2}{\#6}}}{2}
              {##7}{{\CodeColored{codeparam}{2}{\#7}}}{2}
              {##8}{{\CodeColored{codeparam}{2}{\#8}}}{2}
              {##9}{{\CodeColored{codeparam}{2}{\#9}}}{2}
              {<key>}{{$\langle$}key{$\rangle$}}{5}
              {<set>}{{$\langle$}set{$\rangle$}}{5}
              {<set1>}{{$\langle$}set1{$\rangle$}}{6}
              {<set2>}{{$\langle$}set2{$\rangle$}}{6}
    ^^A,literate=*{<key>}{\key}{4}{<set>}{\set}{4}
  }
\newcommand*\CodeColored[3]{\textcolor{#1}{\makebox[\dimexpr\mylstwd*#2]{#3}}}
\newcommand*\CodeUpBf[1]{\makebox[\mylstwd]{\textup{\textbf{#1}}}}
\newcommand*\CodeUp[1]{\makebox[\mylstwd]{\textup{#1}}}
\RequirePackage{randtext}
\let\metaORIG\meta
\protected\def\meta #1{\texttt{\metaORIG{#1}}}
\renewcommand*\thefootnote{\fnsymbol{footnote}}
\definecolor{expkvred}{HTML}{9F393D}
\colorlet{expkvgrey}{black!75}
\makeatletter
\newcommand*\example{\par\smallskip\noindent\textit{Example:} \ignorespaces}
\newcommand*\expkv
  {^^A
    \texorpdfstring
      {^^A
        \mbox
          {^^A
            \BeginAccSupp{ActualText=expkv}^^A
            \href{https://github.com/Skillmon/tex_expkv}
              {^^A
                \rmfamily
                \bfseries
                {\color{expkvgrey}e\kern-.05em x\kern-.05em}^^A
                \lower.493ex
                  \hbox{{\color{expkvgrey}P}\kern-.1em{\color{expkvred}k}}^^A
                \kern-.18em{\color{expkvred}v}^^A
              }^^A
            \EndAccSupp{}^^A
          }^^A
      }
      {expkv}^^A
  }
\newcommand*\expkvpkg[1]
  {^^A
    \texorpdfstring
      {^^A
        \mbox
          {^^A
            \BeginAccSupp{ActualText=expkv-#1}^^A
            \href{https://github.com/Skillmon/tex_expkv-#1}
              {^^A
                \rmfamily
                \bfseries
                {\color{expkvgrey}e\kern-.05em x\kern-.05em}^^A
                \lower.493ex
                  \hbox{{\color{expkvgrey}P}\kern-.1em{\color{expkvred}k}}^^A
                \kern-.18em{\color{expkvred}v}^^A
                {\color{expkvgrey}^^A
                  \kern.05em\rule[-.1ex]{.08em}{1.2ex}\kern.05em\textsc{#1}^^A
                }^^A
              }^^A
            \EndAccSupp{}^^A
          }^^A
      }
      {expkv-#1}^^A
  }
\newcommand*\expkvd{\expkvpkg{def}}
\newcommand*\expkvc{\expkvpkg{cs}}
\newcommand\kv{\meta{key}=\meta{value}}
\newcommand\key{\meta{key}}
\newcommand\val{\meta{value}}
\newcommand\set{\meta{set}}
\newcommand\enflong{\textcolor{black}{long}}
\newcommand\alllong{\textcolor{gray}{long}}
\newcommand\notlong{\textcolor{red!80!black}{long}}
\newcommand\prefixes[1]
  {^^A
    \hfill
    \ifcase\numexpr#1\relax\or
      \enflong\or
      \alllong\or
      \notlong\fi
  }
\hypersetup{linkcolor=red!80!black,urlcolor=purple!80!black}
\DoNotIndex{\def,\edef,\,,\=,\begingroup,\catcode,\chardef,\csname,\endcsname}
\DoNotIndex{\endgroup,\endinput,\errmessage,\expandafter,\input,\let,\long,\z@}
\DoNotIndex{\protected,\ProvidesFile,\ProvidesPackage,\relax,\space,\advance}
\DoNotIndex{\@,\unexpanded,\string,\expanded,\dimexpr,\global,\glueexpr,\hbox}
\DoNotIndex{\numexpr,\RequirePackage,\setbox,\the,\unless,\xdef,\gdef,\newcount}
\DoNotIndex{\number}
\DoNotIndex{\^,\@firstofone,\@firstoftwo,\@gobble,\@secondoftwo}
\DoNotIndex{\ifcsname}
\DoNotIndex{\ifx}
\DoNotIndex{\ifodd}
\DoNotIndex{\ifnum}
\DoNotIndex{\ifdefined}
\DoNotIndex{\iffalse}
\DoNotIndex{\iftrue}
\DoNotIndex{\else}
\DoNotIndex{\fi}
\@gobble\fi ^^A ignoring \ifx, \ifcsname, etc., but only one \fi
\@gobble\fi ^^A ignoring \ifx, \ifcsname, etc., but only one \fi
\@gobble\fi ^^A ignoring \ifx, \ifcsname, etc., but only one \fi
\@gobble\fi ^^A ignoring \ifx, \ifcsname, etc., but only one \fi
\@gobble\fi ^^A ignoring \ifx, \ifcsname, etc., but only one \fi
\@gobble\fi ^^A ignoring \ifx, \ifcsname, etc., but only one \fi
\@ifdefinable\gobbledocstriptag{\def\gobbledocstriptag#1>{}}
\makeatother
\begin{document}
  \title
    {^^A
      \texorpdfstring
        {^^A
          \huge\expkvc
          \\[\medskipamount]
          \Large define expandable \kv\ macros using \expkv
        }{expkv-cs - define expandable key=val macros using expkv}^^A
    }
  \date{\ekvcDate\space v\ekvcVersion}
  \author{Jonathan P. Spratte\thanks{\protect\randomize{jspratte@yahoo.de}}}
  \DocInput{expkv-cs.dtx}
\end{document}
%</driver>^^A=<<
% \fi
%
% \maketitle
% \renewcommand*\thefootnote{\arabic{footnote}}
%
% \begin{abstract}
% \noindent\parfillskip=0pt
% \expkvc\ provides two small interfaces to define expandable \kv\ macros using
% \expkv. It therefore lowers the entrance boundary to expandable \kv\ macros.
% The stylised name is \expkvc\ but the files use \file{expkv-cs}, this is due
% to CTAN-rules which don't allow \string| in package names since that is the
% pipe symbol in *nix shells.
% \end{abstract}
%
% \tableofcontents
%
% \begin{documentation}^^A>>=
%
% \section{Documentation}
%
% The \expkv\ package enables the new possibility of creating \kv\ macros which
% are fully expandable. The creation of such macros is however cumbersome for
% the average user. \expkvc\ tries to step in here. It provides interfaces to
% define \kv\ macros without worrying too much about the implementation. In case
% you're wondering now, the |cs| in \expkvc\ stands for control sequence,
% because |def| was already taken by \expkvd\ and ``control sequence'' is the
% term D.\,E.\,Knuth used in his \TeX book for named commands hence macros
% (though he also used the term ``macro''). So \expkvc\ defines control
% sequences for and with \expkv.
%
% There are two different approaches supported by this package. The first is
% splitting the keys up into individual arguments, the second is providing all
% the keys as a single argument to the underlying macro and getting an
% individual \val\ by using a hash. Well, actually there is no real hash, just
% some markers which are parsed, but this shouldn't be apparent to the user, the
% behaviour matches that of a hash-table.
%
% In addition to these two methods of defining a macro with primary keys a way
% to define secondary keys, which can reference the primary ones, is provided.
% These secondary keys don't correspond to an argument or an entry in the hash
% table directly but might come in handy for the average use case. Each macro
% has its own set of primary and secondary keys.
%
% A word of advice you should consider: If your macro doesn't have to be
% expandable (and often it doesn't) consider not using \expkvc. The interface
% has some overhead (though it still is fast -- check \autoref{sec:speed}) and
% the approach has its limits in versatility. If you don't need to be
% expandable, you should consider either defining your keys manually using
% \expkv\ or using \expkvd\ for convenience. Or you resort to another \kv\
% interface. Nevertheless setting up macros with \expkvc, especially with
% |\ekvcSplit|, is very convenient in my opinion, so if you just want to define
% a single macro with just a few keys this might be the way to go.
%
% \expkvc\ is usable as generic code and as a \LaTeX\ package. It'll
% automatically load \expkv\ in the same mode as well. To use it, just use one
% of
% \begin{lstlisting}
% \usepackage{expkv-cs} % LaTeX
% \input expkv-cs       % plainTeX
% \end{lstlisting}
%
%
% \subsection{Define Macros and Primary Keys}
%
% All macros defined with \expkvc\ have to be previously undefined or have the
% |\meaning| of |\relax|. This is necessary as there is no way to automatically
% undefine keys once they are set up (neither \expkv\ nor \expkvc\ keep track of
% defined keys) -- so to make sure there are no conflicts only new definitions
% are allowed (that's not the case for individual keys, only for frontend
% macros).
%
%
% \subsubsection{Primary Keys}\label{sec:primaries}
%
% In the following descriptions there will be one argument named \meta{primary
% keys}. This argument should be a \kv\ list where each \key\ will be one
% primary key and \val\ the associated initial value. By default all keys are
% defined short, but you can define long keys by prefixing \key\ with |long|
% (\emph{e.g.}, \texttt{long name=Jonathan P. Spratte}). You only need |long| if
% the key should be able to take a |\par| token. Note however that |long| keys
% are a microscopic grain faster (due to some internals of \expkvc). Only if at
% least one of the keys was |long| the \meta{cs} in the following defining
% macros will be |\long|. For obvious reasons there is no possibility to define
% a macro or key as |\protected|.
%
% At the moment \expkvc\ doesn't require any internal keys, but I can't foresee
% whether this will be the case in the future as well, as it might turn out that
% some features I deem useful can't be implemented without such internal keys.
% Because of this, please don't use key names starting with
% \texttt{EKVC\string|} as that should be the private name space.
%
%
% \subsubsection{Split}
%
% The split variants will provide the key values as separate arguments. This
% limits the number of keys for which this is truly useful.
%
% \begin{function}{\ekvcSplit}
%   \begin{syntax}
%     \cs{ekvcSplit}\meta{cs}\marg{primary keys}\marg{definition}
%   \end{syntax}
%   This defines \meta{cs} to be a macro taking one mandatory argument which
%   should contain a \kv\ list. The \meta{primary keys} will be defined for this
%   macro (see \autoref{sec:primaries}). The \meta{definition} is the code that
%   will be executed. You can access the \val\ of a \key\ by using a macro
%   parameter from |#1| to |#9|. The order of the macro parameters will be the
%   order provided in the \meta{primary keys} list (so |#1| is the \val\ of the
%   key defined first). With |\ekvcSplit| you can define macros using at most
%   nine primary keys.
% \end{function}
% \example The following defines a macro |\foo| that takes the keys |a| and |b|
% and outputs their values in a textual form:\\
% \begin{minipage}{.75\linewidth}
% \begin{lstlisting}
% \ekvcSplit\foo{a=a,b=b}{a is #1.\par b is #2.\par}
% \foo{}
% \foo{b=e}
% \end{lstlisting}
% \end{minipage}^^A
% \begin{exresult}[nobeforeafter,box align=center]{.25\linewidth}
%   \ekvcSplit\foo{a=a,b=b}{a is #1.\par b is #2.\par}
%   \foo{}
%   \foo{b=e}
% \end{exresult}
%
% \begin{function}{\ekvcSplitAndForward}
%   \begin{syntax}
%     \cs{ekvcSplitAndForward}\meta{cs}\marg{after}\marg{primary keys}
%   \end{syntax}
%   This defines \meta{cs} to be a macro taking one mandatory argument which
%   should contain a \kv\ list. You can use as many primary keys as you want
%   with this. The primary keys will be forwarded to \meta{after} as braced
%   arguments (as many as necessary for your primary keys). The order of the
%   braced arguments will be the order of your primary key definitions. In
%   \meta{after} you can use just a single control sequence, or some arbitrary
%   stuff which will be left in the input stream before your braced values (with
%   one set of braces stripped from \meta{after}), so both of the following
%   would be fine:
% \begin{lstlisting}
% \ekvcSplitAndForward\foo\foo@aux{keyA = A, keyB = B}
% \ekvcSplitAndForward\foo{\foo@aux{more args}}{keyA = A, keyB = B}
% \end{lstlisting}
% \end{function}
% In the first case |\foo@aux| should take at least two arguments (|keyA| and
% |keyB|), in the second case at least three (|more args|, |keyA|, and |keyB|).
%
% \begin{function}{\ekvcSplitAndUse}
%   \begin{syntax}
%     \cs{ekvcSplitAndUse}\meta{cs}\marg{primary keys}
%   \end{syntax}
%   This will roughly do the same as |\ekvcSplitAndForward|, but instead of
%   specifying what will be used after splitting the keys, \meta{cs} will use
%   what follows the \kv\ list. So its syntax will be
% \end{function}
%   \begin{syntax}
%     \meta{cs}\{\key=\val, \ldots\}\marg{after}
%   \end{syntax}
% and the code in |after| should expect at least as many arguments as the number
% of keys defined for \meta{cs}.
%
% \subsubsection{Hash}
%
% The hash variants will provide the key values as a single argument in which
% you can access specific values using a special macro. The implementation might
% be more convenient and scale better, \emph{but} it is much slower (for a
% primitive macro with a single key benchmarking was almost $1.7$ times slower,
% the root of which being the key access with |\ekvcValue|, not the parsing, and
% for a key access using |\ekvcValueFast| it was still about $1.2$ times
% slower). So if your macro uses less than ten primary keys, you should consider
% using the split approach.
%
% \begin{function}{\ekvcHash}
%   \begin{syntax}
%     \cs{ekvcHash}\meta{cs}\marg{primary keys}\marg{definition}
%   \end{syntax}
%   This defines \meta{cs} to be a macro taking one mandatory argument which
%   should contain a \kv\ list. You can use as many primary keys as you want.
%   The primary keys will be forwarded as a single argument containing every key
%   to the underlying macro. The underlying macro is defined as
%   \meta{definition}, in which you can access the \val\ of a \key\ by using
%   \texttt{\cs[no-index]{ekvcValue}\{\key\}\{\#1\}}.
% \end{function}
% \example This defines an equivalent macro like the |\foo| defined with
% |\ekvcSplit| earlier:\\
% \begin{minipage}{.75\linewidth}
% \begin{lstlisting}
% \ekvcHash\foo{a=a,b=b}{a is \ekvcValue{a}{#1}.\par
%                        b is \ekvcValue{b}{#1}.\par}
% \foo{}
% \foo{b=e}
% \end{lstlisting}
% \end{minipage}^^A
% \begin{exresult}[nobeforeafter,box align=center]{.25\linewidth}
%   \ekvcHash\foo{a=a,b=b}{a is \ekvcValue{a}{#1}.\par
%                          b is \ekvcValue{b}{#1}.\par}
%   \foo{}
%   \foo{b=e}
% \end{exresult}
%
% \begin{function}{\ekvcHashAndForward}
%   \begin{syntax}
%     \cs{ekvcHashAndForward}\meta{cs}\marg{after}\marg{primary keys}
%   \end{syntax}
%   This defines \meta{cs} to be a macro taking one mandatory argument which
%   should contain a \kv\ list. You can use as many primary keys as you want.
%   The primary keys will be forwarded as a single argument containing every key
%   to \meta{after}. You can use a single macro for \meta{after} or use some
%   arbitrary stuff, which will be left in the input stream before the hashed
%   \kv\ list with one set of braces stripped. In the
%   macro called in \meta{after} you can access the \val\ of a \key\ by using
%   \texttt{\cs[no-index]{ekvcValue}\{\key\}\{\#1\}} (or whichever argument the
%   hashed \kv\ list will be).
% \end{function}
%
% \begin{function}{\ekvcHashAndUse}
%   \begin{syntax}
%     \cs{ekvcHashAndUse}\meta{cs}\marg{primary keys}
%   \end{syntax}
%   This will roughly do the same as |\ekvcHashAndForward|, but instead of
%   specifying what will be used after hashing the keys, \meta{cs} will use what
%   follows the \kv\ list. So its syntax will be
% \end{function}
%   \begin{syntax}
%     \meta{cs}\{\key=\val, \ldots\}\marg{after}
%   \end{syntax}
%
% \begin{function}{\ekvcValue}
%   \begin{syntax}
%     \cs{ekvcValue}\{\key\}\marg{key list}
%   \end{syntax}
%   This is a safe way to access your keys in a hash variant. \key\
%   is the key which's \val\ you want to use out of the \meta{key list}.
%   \meta{key list} should be the key list argument forwarded to your underlying
%   macro by |\ekvcHash|, |\ekvcHashAndForward|, or |\ekvcHashAndUse|. It will
%   be tested whether the hash function to access that \key\ exists, the \key\
%   argument is not empty, and that the \meta{key list} really contains a \val\
%   of that \key. This macro needs exactly two steps of expansion and if used
%   inside of an |\edef| or |\expanded| context will protect the \val\ from
%   further expanding.
% \end{function}
%
% \begin{function}{\ekvcValueFast}
%   \begin{syntax}
%     \cs{ekvcValueFast}\{\key\}\marg{key list}
%   \end{syntax}
%   This behaves just like |\ekvcValue|, but \emph{without any} safety tests. As
%   a result this is about $1.4$ times faster \emph{but} will throw low level
%   \TeX\ errors eventually if the hash function isn't defined or the \key\
%   isn't part of the \meta{key list} (\emph{e.g.}, because it was defined as a
%   key for another macro -- all macros share the same hash function per \key).
%   Use it if you know what you're doing. This macro needs exactly three steps
%   of expansion in the no-errors case.
% \end{function}
%
% \begin{function}{\ekvcValueSplit}
%   \begin{syntax}
%     \cs{ekvcValueSplit}\{\key\}\marg{key list}\marg{next}
%   \end{syntax}
%   If you need a specific \key\ from a \meta{key list} more than once, it'll
%   be a good idea to only extract it once and from then on keep it as a
%   separate argument. Hence the macro \cs{ekvcValueSplit} will extract one
%   specific \key's value from the list and forward the remainder of the list
%   as the first and the \key's value as the second argument to \meta{next}, so
%   the result of this will be \meta{next}\marg{key list'}\marg{value} with
%   \meta{key list'} the remaining list. This is almost as fast as |\ekvcValue|
%   and runs the same tests. Keep in mind that you can't fetch for the same
%   \key\ again from \meta{key list'} as it got removed.
% \end{function}
% \example The following defines a macro |\foo| which will take three keys.
% Since the next parsing step will need the value of one of the keys multiple
% times we split that key off the list:\\
% \begin{minipage}{.75\linewidth}
% \begin{lstlisting}
% \ekvcHash\foo{a=a,b=b,c=c}
%   {\ekvcValueSplit{a}{#1}\foobar}
% \newcommand*\foobar[2]{a is #2.\par
%                        b is \ekvcValue{b}{#1}.\par
%                        c is \ekvcValue{c}{#1}.\par}
% \foo{}
% \end{lstlisting}
% \end{minipage}^^A
% \begin{exresult}[nobeforeafter,box align=center]{.25\linewidth}
%   \ekvcHash\foo{a=a,b=b,c=c}
%     {\ekvcValueSplit{a}{#1}\foobar}
%   \newcommand*\foobar[2]{a is #2.\par
%                          b is \ekvcValue{b}{#1}.\par
%                          c is \ekvcValue{c}{#1}.\par}
%   \foo{}
% \end{exresult}
%
% \begin{function}{\ekvcValueSplitFast}
%   \begin{syntax}
%     \cs{ekvcValueSplitFast}\{\key\}\marg{key list}\marg{next}
%   \end{syntax}
%   This behaves just like |\ekvcValueSplit|, but it won't run the same tests,
%   hence it is faster but more error prone, just like the relation between
%   |\ekvcValue| and |\ekvcValueFast|.
% \end{function}
%
%
% \subsection{Secondary Keys}\label{sec:secondaries}
%
% To remove some of the limitations with the approach that each primary key
% matches an argument or hash entry, you can define secondary keys. Those have
% to be defined for each macro but it doesn't matter whether that macro was a
% split or a hash variant. If a secondary key references another key it doesn't
% matter whether that other key is primary or secondary.
%
% Secondary keys can have a prefix (like |long|) which are called |p|-type
% prefix and must have a type (like |meta|) which are called |t|-type prefix.
% Some types might require some |p|-prefixes, while others might forbid those.
%
% Please keep in mind that key names shouldn't start with \texttt{EKVC\string|}.
%
% \begin{function}{\ekvcSecondaryKeys}
%   \begin{syntax}
%     \cs{ekvcSecondaryKeys}\meta{cs}\{\kv, \ldots\}
%   \end{syntax}
%   This is the front facing macro to define secondary keys. For the macro
%   \meta{cs} define \key\ to have definition \val. The general syntax for \key\
%   should be
%   \begin{quote}\ttfamily\small
%     \meta{prefix} \meta{name}
%   \end{quote}
%   Where \meta{prefix} is a space separated list of optional |p|-type prefixes
%   followed by one |t|-type prefix. The syntax of \val\ is dependent on the
%   used |t|-prefix.
% \end{function}
%
%
% \subsubsection{\texttt{p}-type Prefixes}
%
% There is only one |p|-prefix available, which is |long|.
%
% \begin{function}{long}
%   The following key will be defined |\long|.
% \end{function}
%
%
% \subsubsection{\texttt{t}-type Prefixes}
%
% If you're familiar with \expkvd\ you'll notice that the |t|-type prefixes
% provided here are much fewer. The expansion only concept doesn't allow for
% great variety in the auto-defined keys.
%
% The syntax examples of the |t|-prefixes will show which |p|-prefix will be
% automatically used by printing those black (\texttt{\enflong}), which
% will be available in grey (\texttt{\alllong}), and which will be disallowed in
% red (\texttt{\notlong}). This will be put flush right next to the syntax line.
%
% \begin{function}{meta}
%   \begin{syntax}
%     meta \key\ = \{\kv, \ldots\} \prefixes2
%   \end{syntax}
%   With a |meta| key you can set other keys. Whenever \key\ is used the keys in
%   the \kv\ list will be set to the values given there. You can use the \val\
%   given to \key\ by using |#1| in the \kv\ list. The keys in the \kv\ list can
%   be primary and secondary ones.
% \end{function}
%
% \begin{function}{nmeta}
%   \begin{syntax}
%     nmeta \key\ = \{\kv, \ldots\} \prefixes3
%   \end{syntax}
%   An |nmeta| key is like a |meta| key, but it doesn't take a value, so the
%   \kv\ list is static.
% \end{function}
%
% \begin{function}{alias}
%   \begin{syntax}
%     alias \key\ = \meta{key_2} \prefixes3
%   \end{syntax}
%   This assigns the definition of \meta{key_2} to \key. As a result \key\ is an
%   alias for \meta{key_2} behaving just the same. Both the value taking and the
%   |NoVal| version (that's \expkv\ slang for a key not accepting a value) will
%   be copied if they are defined when |alias| is used. Of course, \meta{key_2}
%   has to be defined, be it as a primary or secondary one.
% \end{function}
%
% \begin{function}{default}
%   \begin{syntax}
%     default \key\ = \marg{default} \prefixes3
%   \end{syntax}
%   If \key\ is a defined value taking key, you can define a |NoVal| version
%   with this that will behave as if \key\ was given \meta{default} as its
%   \val. Note that this doesn't change the initial values of primary keys set
%   at definition time in |\ekvcSplit| and friends. \key\ can be a primary or
%   secondary key.
% \end{function}
%
% \begin{function}{flag-bool}
%   \begin{syntax}
%     flag-bool \key\ = \meta{cs} \prefixes3
%   \end{syntax}
%   This is a secondary key that doesn't directly involve any of the primary or
%   secondary keys.
%   This defines \key\ to take a value, which should be either |true| or
%   |false|, and set the flag called \meta{cs} accordingly as a boolean.
%   If \meta{cs} isn't defined yet it will be initialised as a flag.
%   Please also read \autoref{sec:flags}.
% \end{function}
%
% \begin{function}{flag-true,flag-false}
%   \begin{syntax}
%     flag-true \key\ = \meta{cs} \prefixes3
%   \end{syntax}
%   This is a secondary key that doesn't directly involve any of the primary or
%   secondary keys.
%   This defines \key\ to take no value and set the flag called \meta{cs} to
%   |true| or |false|, respectively.
%   If \meta{cs} isn't defined yet it will be initialised as a flag.
%   Please also read \autoref{sec:flags}.
% \end{function}
%
% \begin{function}{flag-raise}
%   \begin{syntax}
%     flag-raise \key\ = \meta{cs} \prefixes3
%   \end{syntax}
%   This is a secondary key that doesn't directly involve any of the primary or
%   secondary keys.
%   This defines \key\ to take no value and raise the flag called \meta{cs}.
%   If \meta{cs} isn't defined yet it will be initialised as a flag.
%   Please also read \autoref{sec:flags}.
% \end{function}
%
%
% \subsection{Changing the Initial Values}\label{sec:change}
%
% \begin{function}{\ekvcChange}
%   \begin{syntax}
%     \cs{ekvcChange}\meta{cs}\{\kv,\ldots\}
%   \end{syntax}
%   This processes the \kv\ list (just like it would be processed by \meta{cs})
%   to set new defaults for the macro \meta{cs} defined with \expkvc\ (meaning
%   the values used if you don't provide anything at use time). Inside the \kv\
%   list both primary and secondary keys can be used. If \meta{cs} was defined
%   |\long| earlier it will still be |\long|, every other \TeX\ prefix will be
%   stripped (but \expkvc\ doesn't support them anywhere else so that should be
%   fine). The resulting new defaults will be stored inside the \meta{cs}
%   locally (just as the original defaults were).
% \end{function}
%
% Consider the following example:\\
% \begin{minipage}[c]{.75\linewidth}
% \begin{lstlisting}
% \ekvcSplit\foo{a=a,b=b}{a is #1.\par b is #2.\par}
% \foo{} % initial defaults
% \ekvcChange\foo{b=B}
% \foo{} % new defaults
% \ekvcSecondaryKeys\foo{meta c={a={#1},b={#1}}}
% \ekvcChange\foo{c=c}
% \foo{} % newer defaults
% \end{lstlisting}
% \end{minipage}^^A
% \begin{exresult}[nobeforeafter,box align=center]{.25\linewidth}
%   \ekvcSplit\foo{a=a,b=b}{a is #1.\par b is #2.\par}
%   \foo{}
%   \ekvcChange\foo{b=B}
%   \foo{}
%   \ekvcSecondaryKeys\foo{meta c={a={#1},b={#1}}}
%   \ekvcChange\foo{c=c}
%   \foo{}
% \end{exresult}
%
% As a result with this the typical setup macro could be implemented:
% \begin{lstlisting}
% \ekvcHashAndUse\foo{key=a,key=b}
% \newcommand\foosetup{\ekvcChange\foo}
% \end{lstlisting}
% Of course the usage is limited to a single macro |\foo|, hence this might be
% not as powerful as similar macros used with other \kv\ interfaces. But at
% least a few similar macros could be grouped using the same key parsing macro
% internally.
%
%
% \subsection{Flags}\label{sec:flags}
%
% The idea of flags is taken from \pkg{expl3}. They provide a way to store
% numerical information expandably, however only incrementing and accessing
% works expandably, decrementing is unexpandable. A flag has a height, which
% is a numerical value, and which can be raised by $1$. Flags come at a high
% computational cost (accessing them is slow and they require more memory than
% normal \TeX\ data types like registers, both getting linearly worse with the
% height), so don't use them if not necessary.
%
% The state of flags is always changed locally to the current group, but not to
% the current macro, so if you're using one of the |t|-types involving flags
% bear in mind that they can affect other macros using the same flags at the
% current group level!
%
% \expkvc\ provides some macros to access, alter, and use flags. Flags of
% \expkvc\ don't share a name space with the flags of \pkg{expl3}.
%
% \begin{function}{\ekvcFlagNew}
%   \begin{syntax}
%     \cs{ekvcFlagNew}\meta{flag}
%   \end{syntax}
%   This initialises the macro \meta{flag} as a new flag. It isn't checked
%   whether the macro \meta{flag} is currently undefined.  A \meta{flag} will
%   expand to the flag's current height with a trailing space (so you can use it
%   directly with |\ifnum| for example and it will terminate the number scanning
%   itself).
% \end{function}
%
%  All other macros dealing with flags take as a parameter a macro defined as a
%  \meta{flag} with |\ekvcFlagNew|.
%
% \begin{function}{\ekvcFlagHeight}
%   \begin{syntax}
%     \cs{ekvcFlagHeight}\meta{flag}
%   \end{syntax}
%   This expands to the current height of \meta{flag} in a single step of
%   expansion (without a trailing space).
% \end{function}
%
% \begin{function}{\ekvcFlagRaise}
%   \begin{syntax}
%     \cs{ekvcFlagRaise}\meta{flag}
%   \end{syntax}
%   This expandably raises the height of \meta{flag} by $1$.
% \end{function}
%
% \begin{function}{\ekvcFlagSetTrue,\ekvcFlagSetFalse}
%   \begin{syntax}
%     \cs{ekvcFlagSetTrue}\meta{flag}
%   \end{syntax}
%   By interpreting an even value as false and an odd value as true we can use a
%   flag as a boolean. This expandably sets \meta{flag} to true or false,
%   respectively, by raising it if necessary.
% \end{function}
%
% \begin{function}{\ekvcFlagIf}
%   \begin{syntax}
%     \cs{ekvcFlagIf}\meta{flag}\marg{true}\marg{false}
%   \end{syntax}
%   This interprets a \meta{flag} as a boolean and expands to either \meta{true}
%   or \meta{false}.
% \end{function}
%
% \begin{function}{\ekvcFlagIfRaised}
%   \begin{syntax}
%     \cs{ekvcFlagIfRaised}\meta{flag}\marg{true}\marg{false}
%   \end{syntax}
%   This tests whether the \meta{flag} is raised, meaning it has a height
%   greater than zero, and if so expands to \meta{true} else to \meta{false}.
% \end{function}
%
% \begin{function}{\ekvcFlagReset}
%   \begin{syntax}
%     \cs{ekvcFlagReset}\meta{flag}
%   \end{syntax}
%   This resets a flag (so restores its height to $0$). This operation is
%   \emph{not} expandable and done locally. If you really intend to use flags
%   you can reset them every now and then to keep the performance hit low.
% \end{function}
%
% \begin{function}{\ekvcFlagGetHeight}
%   \begin{syntax}
%     \cs{ekvcFlagGetHeight}\meta{flag}\marg{next}
%   \end{syntax}
%   This retrieves the current height of the \meta{flag} and provides it as a
%   braced argument to \meta{next}, leaving \meta{next}\marg{height} in the
%   input stream.
% \end{function}
%
% \begin{function}{\ekvcFlagGetHeights}
%   \begin{syntax}
%     \cs{ekvcFlagGetHeights}\marg{flag-list}\marg{next}
%   \end{syntax}
%   This retrieves the current height of each \meta{flag} in the
%   \meta{flag-list} and provides them as a single argument to \meta{next}.
%   Inside that argument each height is enclosed in a set of braces
%   individually. The \meta{flag-list} is just a single argument containing the
%   \meta{flag}s.
%   So a usage like
%   \texttt
%     {^^A
%       \cs[no-index]{ekvcFlagGetHeights}\hskip0pt^^A
%       \{\cs[no-index]{myflagA}\cs[no-index]{myflagB}\}\hskip0pt^^A
%       \{\cs[no-index]{stuff}\}^^A
%     }
%   will expand to
%   \texttt{\cs[no-index]{stuff}\{\marg{height-A}\marg{height-B}\}}.
% \end{function}
%
%
% \subsection{Further Examples}
%
% How could a documentation be a good documentation without some basic examples?
% Say we want to define a small macro expanding to some character description
% (who knows why this has to be expandable?). A character description will not
% have too many items to it, so we use |\ekvcSplit| (the comments with the
% parameter numbers are of course not necessary and just ease reading the
% example).
% \begin{lstlisting}
% \ekvcSplit\character
%   {
%     name=John Doe,            % #1
%     age=any,                  % #2
%     nationality=the Universe, % #3
%     hobby=to exist,           % #4
%     type=Mister,              % #5
%     pronoun=He,               % #6
%     possessive=his,           % #7
%   }
%   {#1 is a #5 from #3. #6 is of #2 age and #7 hobby is #4.\par}
% \end{lstlisting}
% Also we want to give some short cuts so that it's easier to describe several
% persons.
% \begin{lstlisting}
% \ekvcSecondaryKeys\character
%   {
%     alias pro = pronoun,
%     alias pos = possessive,
%     nmeta me =
%       {
%         name=Jonathan,
%         age=a young,
%         nationality=Germany,
%         hobby=\TeX\ coding,
%       },
%     meta lady =
%       {type=Lady, pronoun=She, possessive=her, name=Jane Doe, #1},
%     nmeta paulo =
%       {
%         name=Paulo,
%         type=duck,
%         age=a young,
%         nationality=Brazil,
%         hobby=to quack,
%       }
%   }
% \end{lstlisting}
% Now we can describe people using
% \begin{lstlisting}
% \character{}
% \character{me}
% \character{paulo}
% \character
%   {lady={name=Evelyn,nationality=Ireland,age=the best,hobby=reading}}
% \character
%   {
%     name=Our sun, type=star, nationality=our solar system, pro=It,
%     age=an old, pos=its, hobby=shining
%   }
% \end{lstlisting}
% As one might see, the |lady| key could actually have been an |nmeta| key as
% well, as all that is done with the argument is using it as a \kv\ list.
%
% The result of only the first two usages would be:
% \begin{exresult}{\linewidth}
%   \ekvcSplit\character
%     {
%       name=John Doe,
%       age=any,
%       nationality=the Universe,
%       hobby=to exist,
%       type=Mister,
%       pronoun=He,
%       possessive=his,
%     }
%     {#1 is a #5 from #3. #6 is of #2 age and #7 hobby is #4.\par}
%   \ekvcSecondaryKeys\character
%     {
%       alias pro = pronoun,
%       alias pos = possessive,
%       nmeta me =
%         {
%           name=Jonathan,
%           age=a young,
%           nationality=Germany,
%           hobby=\TeX\ coding,
%         },
%       meta lady =
%         {type=Lady, pronoun=She, possessive=her, name=Jane Doe, #1},
%       nmeta paulo =
%         {
%           name=Paulo,
%           type=duck,
%           age=a young,
%           nationality=Brazil,
%           hobby=to quack,
%         }
%     }
%   \character{}
%   \character{me}
% \end{exresult}
%
% Using \pkg{xparse} or \expkv's \cs{ekvoptarg} or \cs{ekvoptargTF} and
% forwarding arguments one can easily define \kv\ macros with actual optional
% and mandatory arguments as well. A small nonsense example (which should
% perhaps use |\ekvcSplitAndForward| instead of |\ekvcHashAndForward| since it
% only uses four keys and one other argument -- and isn't expandable since it
% uses a \env{tabular} environment, so it would've been better to use a more
% feature rich \kv\ interface most likely, \emph{e.g.}, the one provided by
% \expkvd):
% \begin{lstlisting}
% \makeatletter
% \newcommand*\nonsense{\ekvoptarg\nonsense@a{}}
% \ekvcHashAndForward\nonsense@a\nonsense@b
%   {
%     keyA = A,
%     keyB = B,
%     keyC = c,
%     keyD = d,
%   }
% \newcommand*\nonsense@b[2]
%   {%
%     \begin{tabular}{lll}
%       key & A & \ekvcValue{keyA}{#1} \\
%           & B & \ekvcValue{keyB}{#1} \\
%           & C & \ekvcValue{keyC}{#1} \\
%           & D & \ekvcValue{keyD}{#1} \\
%       \multicolumn{2}{l}{mandatory} & #2 \\
%     \end{tabular}%
%   }
% \makeatother
% \end{lstlisting}
% And then we would be able to do some nonsense
% \begin{lstlisting}
% \nonsense{}
% \nonsense[keyA=hihi]{haha}
% \nonsense[keyA=hihi, keyB=A]{hehe}
% \nonsense[keyC=huhu, keyA=hihi, keyB=A]{haha}
% \end{lstlisting}
% resulting in
% \begin{exresult}{\linewidth}
%   \makeatletter
%   \newcommand*\nonsense{\ekvoptarg\nonsense@a{}}
%   \ekvcHashAndForward\nonsense@a\nonsense@b
%     {
%       keyA = A,
%       keyB = B,
%       keyC = c,
%       keyD = d,
%     }
%   \newcommand*\nonsense@b[2]
%     {%
%       \begin{tabular}{lll}
%         key & A & \ekvcValue{keyA}{#1} \\
%             & B & \ekvcValue{keyB}{#1} \\
%             & C & \ekvcValue{keyC}{#1} \\
%             & D & \ekvcValue{keyD}{#1} \\
%         \multicolumn{2}{l}{mandatory} & #2 \\
%       \end{tabular}%
%     }
%   \makeatother
%   \nonsense{}
%   \nonsense[keyA=hihi]{haha}
%   \nonsense[keyA=hihi, keyB=A]{hehe}
%   \nonsense[keyC=huhu, keyA=hihi, keyB=A]{haha}
% \end{exresult}
%
%
% \subsection{Freedom for Keys!}
%
% If this was the \TeX book this subsection would have a double bend sign. Not
% because it is overly complicated, but because it shows things which could
% break \expkvc's expandability and its alignment safety. This is for
% experienced users wanting to get the most flexibility and knowing what they
% are doing.
%
% In case you're wondering, it is possible to define other keys than the
% primaries and the secondary types listed in \autoref{sec:secondaries} for a
% macro defined with \expkvc\ by using the low-level interface of \expkv\ or
% even the interface provided by \expkvd. The set name used for \expkvc's keys
% is the macro name, including the leading backslash, or more precisely
% |\string|\meta{cs} is used. This can be exploited to define additional keys
% with arbitrary code. Consider the following \emph{bad} example:
% \begin{lstlisting}
% \ekvcSplit\foo{a=A,b=B}{a is #1.\par b is #2}
% \protected\ekvdef{\string\foo}{c}{\def\fooC{#1}}
% \end{lstlisting}
% This would define a key named |c| that will store its value inside a macro.
% The issue with this is that this can't be done expandably. As a result, the
% macro |\foo| isn't always expandable any more (not that bad if this was never
% required; killjoy if it was) and as soon as the key |c| is used, it is also no
% longer alignment safe\footnote{This means that the \kv-list can't contain
%   alignment markers that are not inside an additional set of braces if used
%   inside a \TeX\ alignment.}
% (might be bad depending on the usage).
%
% So why do I show you this? Because we could as well do something useful like
% create a key that pre-parses the input and after that passes the parsed value
% on. This parsing would have to be completely expandable though. For the
% pass-on part we can use the following function:
%
% \begin{function}{\ekvcPass}
%   \begin{syntax}
%     \cs{ekvcPass}\meta{cs}\marg{key}\marg{value}
%   \end{syntax}
%   This passes \meta{value} on to \meta{key} for the \expkvc-macro \meta{cs}.
%   It should be used inside the key parsing of a macro defined with \expkvc,
%   else this most likely results in a low level \TeX\ error.
% \end{function}
%
% With this we could for example split the value of a key at a hyphen and pass
% the parts to different keys:\\
% \begin{minipage}[c]{.75\linewidth}
% \begin{lstlisting}
% \ekvcSplit\foo{a=A,b=B}{a is #1.\par b is #2.\par}
% \ekvdef{\string\foo}{c}{\fooSplit#1\par}
% \def\fooSplit#1-#2\par
%   {\ekvcPass\foo{a}{#1}\ekvcPass\foo{b}{#2}}
% \foo{}
% \foo{c=1-2}
% \end{lstlisting}
% \end{minipage}
% \begin{exresult}[nobeforeafter,box align=center]{.25\linewidth}
%   \ekvcSplit\foo{a=A,b=B}{a is #1.\par b is #2.\par}
%   \ekvdef{\string\foo}{c}{\fooSplit#1\par}
%   \def\fooSplit#1-#2\par{\ekvcPass\foo{a}{#1}\ekvcPass\foo{b}{#2}}
%   \foo{}
%   \foo{c=1-2}
% \end{exresult}
%
%
% \subsection{Speed Considerations}\label{sec:speed}
%
% As already mentioned in the introduction there are some speed considerations
% implied if you choose to define macros via \expkvc. However the overhead isn't
% the factor which should hinder you to use \expkvc\ if you found a reasonable
% use case. The key-parsing is still faster than with most other \kv\ packages
% (see the ``Comparisons'' subsection in the \expkv\ documentation).
%
% The speed considerations in this subsection use the first example in this
% documentation as the benchmark. So we have seven keys and a short sentence
% which should be typeset. For comparisons I use the following equivalent
% \expkvd\ definitions. Each result is the average between changing no keys from
% their initial values and altering four. Furthermore I'll compare three
% variants of \expkvc\ with the \expkvd\ definitions, namely the split example
% from above, a hash variant using |\ekvcValue| and a hash variant using
% |\ekvcValueFast|.
% \begin{lstlisting}
% \usepackage{expkv-def}
% \ekvdefinekeys{keys}
%   {%
%     ,store   name         = \KEYSname
%     ,initial name         = John Doe
%     ,store   age          = \KEYSage
%     ,initial age          = any
%     ,store   nationality  = \KEYSnationality
%     ,initial nationality  = the Universe
%     ,store   hobby        = \KEYShobby
%     ,initial hobby        = to exist
%     ,store   type         = \KEYStype
%     ,initial type         = Mister
%     ,store   pronoun      = \KEYSpronoun
%     ,initial pronoun      = He
%     ,store   possessive   = \KEYSpossessive
%     ,initial possessive   = his
%   }
% \newcommand*\KEYS[1]
%   {%
%     \begingroup
%       \ekvset{keys}{#1}%
%       \KEYSname\ is a \KEYStype\ from \KEYSnationality. \KEYSpronoun\ is
%       of \KEYSage\ age and \KEYSpossessive\ hobby is \KEYShobby.%
%     \endgroup
%   }
% \end{lstlisting}
%
% The first comparison removes the typesetting part from all the definitions, so
% that only the key parsing is compared. In this comparison the |\ekvcValue|
% and |\ekvcValueFast| variants will not differ, as they are exactly the same
% until the key usage. We find that the split approach is $1.4$ times slower
% than the \expkvd\ setup and the hash variants end up in the middle at $1.17$
% times slower.
%
% Next we put the typesetting part back in. Every call of the macros will
% typeset the sentences into a box register in horizontal mode. With the
% typesetting part (which includes the accessing of values) the fastest remains
% the \expkvd\ definitions, but split is close at $1.16$ times slower, followed
% by the hash variant with fast accesses at $1.36$ times slower, and the safe
% hash access variant ranks in the slowest $1.8$ times slower than \expkvd.
%
% Just in case you're wondering now, a simple macro taking seven arguments is
% $30$ to $40$ times faster than any of those in the argument grabbing and \kv\
% parsing part and only $1.5$ to $2.8$ times faster if the typesetting part is
% factored in. So the real choke isn't the parsing.
%
% So to summarize this, if you have a reasonable use case for expandable \kv\
% parsing macros you should go on and define them using \expkvc. If you have a
% reasonable use case for \kv\ parsing macros but defining them expandable isn't
% necessary for your use you should take advantage of the greater flexibility of
% non-expandable \kv\ setups (but if you're after maximum speed there aren't
% that many \kv\ parsers beating \expkvc). And if you are after maximum
% performance maybe ditching the \kv\ interface altogether is a good idea, but
% depending on the number of arguments your interface might get convoluted.
%
%
% \subsection{Useless Macros}
%
% Perhaps these macros aren't completely useless, but I figured from a user's
% point of view I wouldn't know what I should do with these.
%
% \begin{function}{\ekvcDate,\ekvcVersion}
%   These two macros store the version and the date of the package/generic code.
% \end{function}
%
%
% \subsection{Bugs}
%
% Of course I don't think there are any bugs (who would knowingly distribute
% buggy software as long as he isn't a multi-million dollar corporation?). But
% if you find some please let me know. For this one might find my email address
% on the first page or file an issue on Github:
% \url{https://github.com/Skillmon/tex_expkv-cs}
%
%
% \subsection{License}
%
% Copyright \textcopyright\ 2020\unless\ifnum\year=2020--\the\year\fi\
% Jonathan P. Spratte
%
% \medskip\noindent
% This work may be distributed and/or modified under the conditions of the
% \LaTeX\ Project Public License (LPPL), either version 1.3c of this license or
% (at your option) any later version. The latest version of this license is in
% the file:
%
% \url{http://www.latex-project.org/lppl.txt}
%
% \noindent
% This work is ``maintained'' (as per LPPL maintenance status) by
%
% \mbox{Jonathan P. Spratte}.
%
% \end{documentation}^^A=<<
%
% \begin{implementation}^^A>>=
%
% \clearpage
%
% \section{Implementation}^^A>>=
%^^A the LaTeX package >>=
% \subsection{The \LaTeX\ Package}
% Just like for \expkv\ we provide a small \LaTeX\ package that sets up things
% such that we behave nicely on \LaTeX\ packages and files system. It'll
% |\input| the generic code which implements the functionality.
% \gobbledocstriptag
%<*pkg>
%    \begin{macrocode}
\RequirePackage{expkv}
\def\ekvc@tmp
  {%
    \ProvidesFile{expkv-cs.tex}%
      [%
        \ekvcDate\space v\ekvcVersion\space
        define expandable key=val macros using expkv%
      ]%
  }
\input{expkv-cs.tex}
\ProvidesPackage{expkv-cs}%
  [%
    \ekvcDate\space v\ekvcVersion\space
    define expandable key=val macros using expkv%
  ]
%    \end{macrocode}
% \gobbledocstriptag
%</pkg>
%^^A=<<
%^^A main file >>=
% \subsection{The Generic Code}
% The rest of this implementation will be the generic code.
% \gobbledocstriptag
%<*tex>
%
% Load \expkv\ if the package didn't already do so -- since \expkv\ has
% safeguards against being loaded twice this does no harm and the overhead
% isn't that big. Also we reuse some of the internals of \expkv\ to save us from
% retyping them.
%    \begin{macrocode}
\input expkv
%    \end{macrocode}
%
% We make sure that \file{expkv-cs.tex} is only input once:
%    \begin{macrocode}
\expandafter\ifx\csname ekvcVersion\endcsname\relax
\else
  \expandafter\endinput
\fi
%    \end{macrocode}
%
% \begin{macro}{\ekvcVersion,\ekvcDate}
% We're on our first input, so lets store the version and date in a macro.
%    \begin{macrocode}
\def\ekvcVersion{0.10}
\def\ekvcDate{2021-06-03}
%    \end{macrocode}
% \end{macro}
%
% If the \LaTeX\ format is loaded we want to be a good file and report back who
% we are, for this the package will have defined |\ekvc@tmp| to use
% |\ProvidesFile|, else this will expand to a |\relax| and do no harm.
%    \begin{macrocode}
\csname ekvc@tmp\endcsname
%    \end{macrocode}
%
% Store the category code of |@| to later be able to reset it and change it to
% 11 for now.
%    \begin{macrocode}
\expandafter\chardef\csname ekvc@tmp\endcsname=\catcode`\@
\catcode`\@=11
%    \end{macrocode}
% |\ekvc@tmp| will be reused later, but we don't need it to ever store
% information long-term after \expkvc\ was initialized.
%
% \begin{macro}[internal]{\ekvc@keycount}
%   We'll need to keep count how many keys must be defined for each macro in the
%   |split| variants.
%    \begin{macrocode}
\newcount\ekvc@keycount
%    \end{macrocode}
% \end{macro}
%
% \begin{macro}[internal]{\ekvc@long,\ekvc@any@long}
%   Some macros will have to be defined long. These two will be let to |\long|
%   when this should be the case.
%    \begin{macrocode}
\let\ekvc@long\ekv@empty
\let\ekvc@any@long\ekv@empty
%    \end{macrocode}
% \end{macro}
%
% \begin{macro}[internal]{\ekvc@ifdefined}
%   We want to test whether a macro is already defined. This test checks for a
%   defined macro that isn't |\relax|.
%    \begin{macrocode}
\long\def\ekvc@ifdefined#1%
  {%
    \ifdefined#1%
      \ifx\relax#1%
        \ekv@fi@gobble
      \fi
      \@firstofone
      \ekv@fi@firstoftwo
    \fi
    \@secondoftwo
  }
%    \end{macrocode}
% \end{macro}
%
% \begin{macro}[internal]
%   {
%     \ekvc@ekvset@pre@expander,
%     \ekvc@ekvset@pre@expander@a,
%     \ekvc@ekvset@pre@expander@b
%   }
%   This macro expands |\ekvset| twice so that the first two steps of expansion
%   don't have to be made every time the \expkvc\ macros are used. We have to do
%   a little magic trick to get the macro parameter |#1| for the macro
%   definition this is used in, even though we're calling |\unexpanded|. We do
%   that by splitting the expanded |\ekvset| at some marks and place |##1| in
%   between. At this spot we also add |\ekv@alignsafe| and |\ekv@endalignsafe|
%   to ensure that macros created with \expkvc\ are alignment safe.
%    \begin{macrocode}
\def\ekvc@ekvset@pre@expander#1%
  {%
    \expandafter\ekvc@ekvset@pre@expander@a\ekvset{#1}\ekvc@stop\ekvc@stop
  }
\def\ekvc@ekvset@pre@expander@a
  {%
    \expandafter\ekvc@ekvset@pre@expander@b
  }
\def\ekvc@ekvset@pre@expander@b#1\ekvc@stop#2\ekvc@stop
  {%
    \unexpanded\expandafter{\ekv@alignsafe}%
    \unexpanded{#1}##1\unexpanded{#2}%
    \unexpanded\expandafter{\ekv@endalignsafe}%
  }
%    \end{macrocode}
% \end{macro}
%
% \begin{macro}{\ekvcSplitAndUse}
%   The first user macro we want to set up can be reused for
%   \cs[no-index]{ekvcSplitAndForward} and \cs[no-index]{ekvcSplit}. We'll split
%   this one up so that the test whether the macro is already defined doesn't
%   run twice.
%    \begin{macrocode}
\protected\long\def\ekvcSplitAndUse#1#2%
  {%
    \let\ekvc@helpers@needed\@firstoftwo
    \ekvc@ifdefined#1%
      {\ekvc@err@already@defined#1}%
      {\ekvcSplitAndUse@#1{}{#2}}%
  }
%    \end{macrocode}
% \end{macro}
%
% \begin{macro}[internal]{\ekvcSplitAndUse@}
%   The actual macro setting up things. We need to set some variables, forward
%   the key list to |\ekvc@SetupSplitKeys|, and afterwards define the front
%   facing macro to call |\ekvset| and put the initials and the argument sorting
%   macro behind it. The internals |\ekvc@any@long|, |\ekvc@initials| and
%   |\ekvc@keycount| will be set correctly by |\ekvc@SetupSplitKeys|.
%    \begin{macrocode}
\protected\long\def\ekvcSplitAndUse@#1#2#3%
  {%
    \edef\ekvc@set{\string#1}%
    \ekvc@SetupSplitKeys{#3}%
    \ekvc@helpers@needed
      {%
        \ekvc@any@long\edef#1##1%
          {%
            \expandafter\ekvc@ekvset@pre@expander\expandafter{\ekvc@set}%
            \unexpanded\expandafter
              {\csname ekvc@split@\the\ekvc@keycount\endcsname}%
            \unexpanded\expandafter{\ekvc@initials{}#2}%
          }%
      }%
      {%
        \ekvc@any@long\edef#1##1%
          {%
            \expandafter\ekvc@ekvset@pre@expander\expandafter{\ekvc@set}%
            \unexpanded{#2}%
            \unexpanded\expandafter{\ekvc@initials}%
          }%
      }%
  }
%    \end{macrocode}
% \end{macro}
%
%
% \begin{macro}{\ekvcSplitAndForward}
%   This just reuses |\ekvcSplitAndUse@| with a non-empty second argument,
%   resulting in that argument to be called after the splitting.
%    \begin{macrocode}
\protected\long\def\ekvcSplitAndForward#1#2#3%
  {%
    \let\ekvc@helpers@needed\@firstoftwo
    \ekvc@ifdefined#1%
      {\ekvc@err@already@defined#1}%
      {\ekvcSplitAndUse@#1{{#2}}{#3}}%
  }
%    \end{macrocode}
% \end{macro}
%
% \begin{macro}{\ekvcSplit}
%   The first half is just |\ekvcSplitAndForward| then we define the macro to
%   which the parsed key list is forwarded. There we need to allow for up to
%   nine arguments.
%    \begin{macrocode}
\protected\long\def\ekvcSplit#1#2#3%
  {%
    \let\ekvc@helpers@needed\@secondoftwo
    \ekvc@ifdefined#1%
      {\ekvc@err@already@defined#1}%
      {%
        \expandafter
        \ekvcSplitAndUse@\expandafter#1\csname ekvc@\string#1\endcsname{#2}%
        \ifnum\ekvc@keycount<1
          \ekvc@any@long\expandafter\def\csname ekvc@\string#1\endcsname{#3}%
        \else
          \ifnum\ekvc@keycount>9
            \ekvc@err@toomany{#1}%
            \let#1\ekvc@undefined
          \else
            \ekvcSplit@build@argspec
            \ekvc@any@long\expandafter
            \def\csname ekvc@\string#1\expandafter\endcsname\ekvc@tmp{#3}%
          \fi
        \fi
      }%
  }
%    \end{macrocode}
% \end{macro}
%
% \begin{macro}[internal]{\ekvcSplit@build@argspec,\ekvcSplit@build@argspec@}
%    \begin{macrocode}
\protected\def\ekvcSplit@build@argspec
  {%
    \begingroup
      \edef\ekvc@tmp
        {\endgroup\def\unexpanded{\ekvc@tmp}{\ekvcSplit@build@argspec@{1}}}%
    \ekvc@tmp
  }
\def\ekvcSplit@build@argspec@#1%
  {%
    \ifnum#1>\ekvc@keycount
      \ekv@fi@gobble
    \fi
    \@firstofone
      {%
        \unexpanded\expandafter{\csname ekvc@splitmark@#1\endcsname####}#1%
        \expandafter\ekvcSplit@build@argspec@\expandafter{\the\numexpr#1+1}%
      }%
  }
%    \end{macrocode}
% \end{macro}
%
% \begin{macro}[internal]
%   {
%     \ekvc@SetupSplitKeys, \ekvc@SetupSplitKeys@a, \ekvc@SetupSplitKeys@b,
%     \ekvc@SetupSplitKeys@c
%   }
%   These macros parse the list of keys and set up the key macros. First we need
%   to initialise some macros and start |\ekvparse|.
%    \begin{macrocode}
\protected\long\def\ekvc@SetupSplitKeys
  {%
    \ekvc@keycount=0
    \let\ekvc@any@long\ekv@empty
    \let\ekvc@initials\ekv@empty
    \ekvparse\ekvc@err@value@required\ekvc@SetupSplitKeys@a
  }
%    \end{macrocode}
%   Then we need to step the key counter for each key. Also we have to check
%   whether this key has a |long| prefix so we initialise |\ekvc@long|.
%    \begin{macrocode}
\protected\def\ekvc@SetupSplitKeys@a#1%
  {%
    \advance\ekvc@keycount1
    \let\ekvc@long\ekv@empty
    \ekvc@ifspace{#1}%
      {\ekvc@SetupSplitKeys@b#1\ekvc@stop}%
      {\ekvc@SetupSplitKeys@c{#1}}%
  }
%    \end{macrocode}
%   If there was a space, there might be a prefix. If so call the prefix macro,
%   else call the next step |\ekvc@SetupSplitKeys@c| which will define the key
%   macro and add the key's value to the initials list.
%    \begin{macrocode}
\protected\def\ekvc@SetupSplitKeys@b#1 #2\ekvc@stop
  {%
    \ekv@ifdefined{ekvc@split@p@#1}%
      {\csname ekvc@split@p@#1\endcsname{#2}}%
      {\ekvc@SetupSplitKeys@c{#1 #2}}%
  }
%    \end{macrocode}
%   The inner definition is grouped, because we don't want to actually define
%   the marks we build with |\csname|. We have to append the value to the
%   |\ekvc@initials| list here with the correct split mark. The key macro will
%   read everything up to those split marks and change the value following it to
%   the value given to the key. Additionally we'll need a sorting macro for each
%   key count in use so we set it up with |\ekvc@setup@splitmacro|.
%    \begin{macrocode}
\protected\long\def\ekvc@SetupSplitKeys@c#1#2%
  {%
    \begingroup
    \edef\ekvc@tmp
      {%
        \endgroup
        \long\def\unexpanded{\ekvc@tmp}####1####2%
            \unexpanded\expandafter
            {\csname ekvc@splitmark@\the\ekvc@keycount\endcsname}####3%
          {%
            ####2%
            \unexpanded\expandafter
              {\csname ekvc@splitmark@\the\ekvc@keycount\endcsname}{####1}%
          }%
%    \end{macrocode}
%   The short variant needs a bit of special treatment. The key macro will be
%   short to throw the correct error, but since there might be long macros
%   somewhere the reordering of arguments needs to be long, so for short keys we
%   use a two step approach, first grabbing only the short argument, then
%   reordering.
%    \begin{macrocode}
        \unless\ifx\ekvc@long\long
          \let\unexpanded\expandafter
            {\csname ekvc@\ekvc@set(#1)\endcsname\ekvc@tmp}%
          \def\unexpanded{\ekvc@tmp}####1%
            {%
              \unexpanded\expandafter{\csname ekvc@\ekvc@set(#1)\endcsname}%
                {####1}%
            }%
        \fi
        \def\unexpanded{\ekvc@initials}%
          {%
            \unexpanded\expandafter{\ekvc@initials}%
            \unexpanded\expandafter
              {\csname ekvc@splitmark@\the\ekvc@keycount\endcsname{#2}}%
          }%
      }%
    \ekvc@tmp
    \ekvlet\ekvc@set{#1}\ekvc@tmp
    \ekvc@helpers@needed
      {\expandafter\ekvc@setup@splitmacro\expandafter{\the\ekvc@keycount}}%
      {}%
  }
%    \end{macrocode}
% \end{macro}
%
% \begin{macro}[internal]{\ekvc@split@p@long}
%   The |long| prefix lets the internals |\ekvc@long| and |\ekvc@any@long| to
%   |\long| so that the key macro will be long.
%    \begin{macrocode}
\protected\def\ekvc@split@p@long
  {%
    \let\ekvc@long\long
    \let\ekvc@any@long\long
    \ekvc@SetupSplitKeys@c
  }
%    \end{macrocode}
% \end{macro}
%
% \begin{macro}[internal]{\ekvc@defarggobbler}
%   This is needed to define a macro with 1-9 parameters programmatically.
%   \LaTeX's \cs[no-index]{newcommand} does something similar for example.
%    \begin{macrocode}
\protected\def\ekvc@defarggobbler#1{\def\ekvc@tmp##1#1##2##{##1#1}}
%    \end{macrocode}
% \end{macro}
%
% \begin{macro}[internal]
%   {
%     \ekvc@setup@splitmacro,
%     \ekvc@split@1, \ekvc@split@2, \ekvc@split@3,
%     \ekvc@split@4, \ekvc@split@5, \ekvc@split@6,
%     \ekvc@split@7
%   }
%   Since the first few split macros are different from the others we manually
%   set those up now. All the others will be defined as needed (always
%   globally).  The split macros just read up until the correct split mark, move
%   that argument into a list and reinsert the rest, calling the next split
%   macro afterwards.
%    \begin{macrocode}
\begingroup
\edef\ekvc@tmp
  {%
    \long\gdef\unexpanded\expandafter{\csname ekvc@split@1\endcsname}%
        \unexpanded\expandafter{\csname ekvc@splitmark@1\endcsname}%
        ##1##2##3%
      {##3{##1}##2}%
    \long\gdef\unexpanded\expandafter{\csname ekvc@split@2\endcsname}%
        \unexpanded\expandafter{\csname ekvc@splitmark@1\endcsname}##1%
        \unexpanded\expandafter{\csname ekvc@splitmark@2\endcsname}##2%
        ##3##4%
      {##4{##1}{##2}##3}%
    \long\gdef\unexpanded\expandafter{\csname ekvc@split@3\endcsname}%
        \unexpanded\expandafter{\csname ekvc@splitmark@1\endcsname}##1%
        \unexpanded\expandafter{\csname ekvc@splitmark@2\endcsname}##2%
        \unexpanded\expandafter{\csname ekvc@splitmark@3\endcsname}##3%
        ##4##5%
      {##5{##1}{##2}{##3}##4}%
    \long\gdef\unexpanded\expandafter{\csname ekvc@split@4\endcsname}%
        \unexpanded\expandafter{\csname ekvc@splitmark@1\endcsname}##1%
        \unexpanded\expandafter{\csname ekvc@splitmark@2\endcsname}##2%
        \unexpanded\expandafter{\csname ekvc@splitmark@3\endcsname}##3%
        \unexpanded\expandafter{\csname ekvc@splitmark@4\endcsname}##4%
        ##5##6%
      {##6{##1}{##2}{##3}{##4}##5}%
    \long\gdef\unexpanded\expandafter{\csname ekvc@split@5\endcsname}%
        \unexpanded\expandafter{\csname ekvc@splitmark@1\endcsname}##1%
        \unexpanded\expandafter{\csname ekvc@splitmark@2\endcsname}##2%
        \unexpanded\expandafter{\csname ekvc@splitmark@3\endcsname}##3%
        \unexpanded\expandafter{\csname ekvc@splitmark@4\endcsname}##4%
        \unexpanded\expandafter{\csname ekvc@splitmark@5\endcsname}##5%
        ##6##7%
      {##7{##1}{##2}{##3}{##4}{##5}##6}%
    \long\gdef\unexpanded\expandafter{\csname ekvc@split@6\endcsname}%
        \unexpanded\expandafter{\csname ekvc@splitmark@1\endcsname}##1%
        \unexpanded\expandafter{\csname ekvc@splitmark@2\endcsname}##2%
        \unexpanded\expandafter{\csname ekvc@splitmark@3\endcsname}##3%
        \unexpanded\expandafter{\csname ekvc@splitmark@4\endcsname}##4%
        \unexpanded\expandafter{\csname ekvc@splitmark@5\endcsname}##5%
        \unexpanded\expandafter{\csname ekvc@splitmark@6\endcsname}##6%
        ##7##8%
      {##8{##1}{##2}{##3}{##4}{##5}{##6}##7}%
    \long\gdef\unexpanded\expandafter{\csname ekvc@split@7\endcsname}%
        \unexpanded\expandafter{\csname ekvc@splitmark@1\endcsname}##1%
        \unexpanded\expandafter{\csname ekvc@splitmark@2\endcsname}##2%
        \unexpanded\expandafter{\csname ekvc@splitmark@3\endcsname}##3%
        \unexpanded\expandafter{\csname ekvc@splitmark@4\endcsname}##4%
        \unexpanded\expandafter{\csname ekvc@splitmark@5\endcsname}##5%
        \unexpanded\expandafter{\csname ekvc@splitmark@6\endcsname}##6%
        \unexpanded\expandafter{\csname ekvc@splitmark@7\endcsname}##7%
        ##8##9%
      {##9{##1}{##2}{##3}{##4}{##5}{##6}{##7}##8}%
  }
\ekvc@tmp
\endgroup
\protected\def\ekvc@setup@splitmacro#1%
  {%
    \ekv@ifdefined{ekvc@split@#1}{}%
      {%
        \begingroup
          \edef\ekvc@tmp
            {%
              \long\gdef
                  \unexpanded\expandafter{\csname ekvc@split@#1\endcsname}%
                  ####1%
                  \unexpanded\expandafter{\csname ekvc@splitmark@#1\endcsname}%
                  ####2####3%
                {%
                  \unexpanded\expandafter
                    {\csname ekvc@split@\the\numexpr#1-1\relax\endcsname}%
                  ####1{{####2}####3}%
                }%
            }%
          \ekvc@tmp
        \endgroup
      }%
  }
%    \end{macrocode}
% \end{macro}
%
% \begin{macro}{\ekvcHashAndUse}
%   |\ekvcHashAndUse| works just like |\ekvcSplitAndUse|.
%    \begin{macrocode}
\protected\long\def\ekvcHashAndUse#1#2%
  {%
    \let\ekvc@helpers@needed\@firstoftwo
    \ekvc@ifdefined#1%
      {\ekvc@err@already@defined#1}%
      {\ekvcHashAndUse@#1{}{#2}}%
  }
%    \end{macrocode}
% \end{macro}
%
% \begin{macro}[internal]{\ekvcHashAndUse@}
%   This is more or less the same as |\ekvcSplitAndUse@|. Instead of an empty
%   group we place a marker after the initials, we don't use the sorting macros
%   of |split|, but instead pack all the values in one argument.
%    \begin{macrocode}
\protected\long\def\ekvcHashAndUse@#1#2#3%
  {%
    \edef\ekvc@set{\string#1}%
    \ekvc@SetupHashKeys{#3}%
    \ekvc@helpers@needed
      {%
        \ekvc@any@long\edef#1##1%
          {%
            \expandafter\ekvc@ekvset@pre@expander\expandafter{\ekvc@set}%
            \unexpanded{\ekvc@hash@pack@argument}%
            \unexpanded\expandafter{\ekvc@initials\ekvc@stop#2}%
          }%
      }%
      {%
        \ekvc@any@long\edef#1##1%
          {%
            \expandafter\ekvc@ekvset@pre@expander\expandafter{\ekvc@set}%
            \unexpanded{#2}%
            \unexpanded\expandafter{\ekvc@initials\ekvc@stop}%
          }%
      }%
  }
%    \end{macrocode}
% \end{macro}
%
% \begin{macro}{\ekvcHashAndForward}
%   |\ekvcHashAndForward| works just like |\ekvcSplitAndForward|.
%    \begin{macrocode}
\protected\long\def\ekvcHashAndForward#1#2#3%
  {%
    \let\ekvc@helpers@needed\@firstoftwo
    \ekvc@ifdefined#1%
      {\ekvc@err@already@defined#1}%
      {\ekvcHashAndUse@#1{{#2}}{#3}}%
  }
%    \end{macrocode}
% \end{macro}
%
% \begin{macro}{\ekvcHash}
%   |\ekvcHash| does the same as |\ekvcSplit|, but has the advantage of not
%   needing to count arguments, so the definition of the internal macro is a bit
%   more straight forward.
%    \begin{macrocode}
\protected\long\def\ekvcHash#1#2#3%
  {%
    \let\ekvc@helpers@needed\@secondoftwo
    \ekvc@ifdefined#1%
      {\ekvc@err@already@defined#1}%
      {%
        \expandafter
        \ekvcHashAndUse@\expandafter#1\csname ekvc@\string#1\endcsname{#2}%
        \ekvc@any@long\expandafter\def\csname ekvc@\string#1\endcsname
            ##1\ekvc@stop
          {#3}%
      }%
  }
%    \end{macrocode}
% \end{macro}
%
% \begin{macro}[internal]{\ekvc@hash@pack@argument}
%   All this macro does is pack the values into one argument and forward that to
%   the next macro.
%    \begin{macrocode}
\long\def\ekvc@hash@pack@argument#1\ekvc@stop#2{#2{#1}}
%    \end{macrocode}
% \end{macro}
%
% \begin{macro}[internal]
%   {\ekvc@SetupHashKeys,\ekvc@SetupHashKeys@a,\ekvc@SetupHashKeys@b}
%   This should look awfully familiar as well, since it's just the same as for
%   the split keys with a few other names here and there.
%    \begin{macrocode}
\protected\long\def\ekvc@SetupHashKeys#1%
  {%
    \let\ekvc@any@long\ekv@empty
    \let\ekvc@initials\ekv@empty
    \ekvparse\ekvc@err@value@required\ekvc@SetupHashKeys@a{#1}%
  }
\protected\def\ekvc@SetupHashKeys@a#1%
  {%
    \let\ekvc@long\ekv@empty
    \ekvc@ifspace{#1}%
      {\ekvc@SetupHashKeys@b#1\ekvc@stop}%
      {\ekvc@SetupHashKeys@c{#1}}%
  }
\protected\def\ekvc@SetupHashKeys@b#1 #2\ekvc@stop
  {%
    \ekv@ifdefined{ekvc@hash@p@#1}%
      {\csname ekvc@hash@p@#1\endcsname{#2}}%
      {\ekvc@SetupHashKeys@c{#1 #2}}%
  }
%    \end{macrocode}
%   Yes, even the defining macro looks awfully familiar. Instead of numbered we
%   have named marks. Still the key macros grab everything up to their
%   respective mark and reorder the arguments. The same quirk is applied for
%   short keys. And instead of the |\ekvc@setup@splitmacro| we use
%   |\ekvc@setup@hashmacro|.
%    \begin{macrocode}
\protected\long\def\ekvc@SetupHashKeys@c#1#2%
  {%
    \begingroup
    \edef\ekvc@tmp
      {%
        \endgroup
        \long\def\unexpanded{\ekvc@tmp}####1####2%
            \unexpanded\expandafter{\csname ekvc@hashmark@#1\endcsname}####3%
          {%
            ####2%
            \unexpanded\expandafter{\csname ekvc@hashmark@#1\endcsname}{####1}%
          }%
        \unless\ifx\ekvc@long\long
          \let\unexpanded\expandafter
            {\csname ekvc@\ekvc@set(#1)\endcsname\ekvc@tmp}%
          \def\unexpanded{\ekvc@tmp}####1%
            {%
              \unexpanded\expandafter{\csname ekvc@\ekvc@set(#1)\endcsname}%
                {####1}%
            }%
        \fi
        \def\unexpanded{\ekvc@initials}%
          {%
            \unexpanded\expandafter{\ekvc@initials}%
            \unexpanded\expandafter{\csname ekvc@hashmark@#1\endcsname{#2}}%
          }%
      }%
    \ekvc@tmp
    \ekvlet\ekvc@set{#1}\ekvc@tmp
    \ekvc@setup@hashmacro{#1}%
  }
%    \end{macrocode}
% \end{macro}
%
% \begin{macro}[internal]{\ekvc@hash@p@long}
%   Nothing astonishing here either.
%    \begin{macrocode}
\protected\def\ekvc@hash@p@long
  {%
    \let\ekvc@long\long
    \let\ekvc@any@long\long
    \ekvc@SetupHashKeys@c
  }
%    \end{macrocode}
% \end{macro}
%
% \begin{macro}[internal]{\ekvc@setup@hashmacro}
%   The safe hash macros will be executed inside of an |\unexpanded| expansion
%   context, so they have to insert braces for that once they are done.
%   Most of the tests which have to be executed will already be done, but we
%   have to play safe if the hash doesn't show up in the hash list. Therefore we
%   use some |\ekvc@mark|s and |\ekvc@stop| to throw errors if the hash isn't
%   found in the right place. The fast variants have an easier life and just
%   return the correct value.
%    \begin{macrocode}
\protected\def\ekvc@setup@hashmacro#1%
  {%
    \ekv@ifdefined{ekvc@fasthash@#1}{}%
      {%
        \begingroup
          \edef\ekvc@tmp
            {%
              \long\gdef
                \unexpanded\expandafter{\csname ekvc@fasthash@#1\endcsname}%
                  ####1%
                  \unexpanded\expandafter{\csname ekvc@hashmark@#1\endcsname}%
                  ####2####3\unexpanded{\ekvc@stop}%
                {####2}%
              \long\gdef
                \unexpanded\expandafter{\csname ekvc@safehash@#1\endcsname}%
                  ####1%
                {%
                  \unexpanded\expandafter{\csname ekvc@@safehash@#1\endcsname}%
                  ####1\unexpanded{\ekvc@mark}{}%
                  \unexpanded\expandafter
                    {%
                      \csname ekvc@hashmark@#1\endcsname{}%
                      \ekvc@mark{\ekvc@err@missing@hash{#1}}\ekvc@stop
                    }%
                }%
              \long\gdef
                \unexpanded\expandafter{\csname ekvc@@safehash@#1\endcsname}%
                  ####1%
                  \unexpanded\expandafter{\csname ekvc@hashmark@#1\endcsname}%
                  ####2####3\unexpanded{\ekvc@mark}####4####5%
                  \unexpanded{\ekvc@stop}%
                {%
                  ####4{####2}%
                }%
              \long\gdef\unexpanded\expandafter
                {\csname ekvc@fastsplithash@#1\endcsname}%
                  ####1%
                  \unexpanded\expandafter{\csname ekvc@hashmark@#1\endcsname}%
                  ####2####3\unexpanded{\ekvc@stop}####4%
                {%
                  ####4{####1####3}{####2}%
                }%
              \long\gdef\unexpanded\expandafter
                {\csname ekvc@safesplithash@#1\endcsname}####1%
                {%
                  \unexpanded\expandafter
                    {\csname ekvc@@safesplithash@#1\endcsname}%
                  ####1\unexpanded{\ekvc@mark\ekvc@safe@found@hash}%
                  \unexpanded\expandafter
                    {%
                      \csname ekvc@hashmark@#1\endcsname{}%
                      \ekvc@mark{\ekvc@err@missing@hash{#1}\ekvc@safe@no@hash}%
                      \ekvc@stop
                    }%
                }%
              \long\gdef\unexpanded\expandafter
                {\csname ekvc@@safesplithash@#1\endcsname}%
                  ####1%
                  \unexpanded\expandafter{\csname ekvc@hashmark@#1\endcsname}%
                  ####2####3\unexpanded{\ekvc@mark}####4####5%
                  \unexpanded{\ekvc@stop}%
                {%
                  ####4{####2}####1####3\unexpanded{\ekvc@stop}%
                }%
            }%
          \ekvc@tmp
        \endgroup
      }%
  }
%    \end{macrocode}
% \end{macro}
%
% \begin{macro}{\ekvcValue}
%   All this does is a few consistency checks on the first argument (not empty,
%   hash macro exists) and then call that hash-grabbing macro that will also
%   test whether the hash is inside of |#2| or not.
%    \begin{macrocode}
\long\def\ekvcValue#1%
  {%
    \unexpanded
    \ekv@ifdefined{ekvc@safehash@#1}%
      {\csname ekvc@safehash@#1\endcsname}%
      {\ekvc@err@unknown@hash{#1}\@firstoftwo{{}}}%
  }
%    \end{macrocode}
% \end{macro}
%
% \begin{macro}{\ekvcValueFast}
%   To be as fast as possible, this doesn't test for anything, assuming the user
%   knows best.
%    \begin{macrocode}
\long\def\ekvcValueFast#1#2{\csname ekvc@fasthash@#1\endcsname#2\ekvc@stop}
%    \end{macrocode}
% \end{macro}
%
% \begin{macro}{\ekvcValueSplit,\ekvcValueSplit@recover}
%   This splits off a single value.
%    \begin{macrocode}
\long\def\ekvcValueSplit#1%
  {%
    \ekv@ifdefined{ekvc@safesplithash@#1}%
      {\csname ekvc@safesplithash@#1\endcsname}%
      {\ekvc@err@unknown@hash{#1}\ekvcValueSplit@recover}%
  }
\long\def\ekvcValueSplit@recover#1#2%
  {%
    #2{#1}{}%
  }
%    \end{macrocode}
% \end{macro}
%
% \begin{macro}[internal]{\ekvc@safe@found@hash, \ekvc@safe@no@hash}
%    \begin{macrocode}
\long\def\ekvc@safe@found@hash#1#2\ekvc@stop#3%
  {%
    #3{#2}{#1}%
  }
\long\def\ekvc@safe@no@hash#1#2\ekvc@mark\ekvc@safe@found@hash\ekvc@stop#3%
  {%
    #3{#2}{}%
  }
%    \end{macrocode}
% \end{macro}
%
% \begin{macro}{\ekvcValueSplitFast}
%   Again a fast approach which doesn't provide too many safety measurements.
%   This needs to build the hash function and expand it before passing the
%   results to the next control sequence. The first step only builds the control
%   sequence.
%    \begin{macrocode}
\long\def\ekvcValueSplitFast#1#2%
  {%
    \csname ekvc@fastsplithash@#1\endcsname#2\ekvc@stop
  }
%    \end{macrocode}
% \end{macro}
%
% \begin{macro}[internal]
%   {
%     \ekvc@safehash@,\ekvc@fasthash@,
%     \ekvc@safesplithash@,\ekvc@fastsplithash@
%   }
%   At least in the empty hash case we can provide a meaningful error message
%   without affecting performance by just defining the macro that would be build
%   in that case. There is of course a downside to this, the error will not be
%   thrown by |\ekvcValueFast| in three expansion steps. The safe hash variant
%   has to also stop the |\unexpanded| expansion.
%    \begin{macrocode}
\long\def\ekvc@safehash@#1{\ekvc@err@empty@hash{}}
\long\def\ekvc@fasthash@#1\ekvc@stop{\ekvc@err@empty@hash}
\long\def\ekvc@safesplithash@#1#2{\ekvc@err@empty@hash#2{#1}{}}
\long\def\ekvc@fastsplithash@#1\ekvc@stop#2{\ekvc@err@empty@hash#2{#1}{}}
%    \end{macrocode}
% \end{macro}
%
% \begin{macro}{\ekvcSecondaryKeys}
%   The secondary keys are defined pretty similar to the way the originals are,
%   but here we also introduce some key types (those have a |@t@| in their name)
%   additionally to the prefixes.
%    \begin{macrocode}
\protected\long\def\ekvcSecondaryKeys#1#2%
  {%
    \edef\ekvc@set{\string#1}%
    \ekvparse\ekvc@err@value@required\ekvcSecondaryKeys@a{#2}%
  }
\protected\def\ekvcSecondaryKeys@a#1%
  {%
    \let\ekvc@long\ekv@empty
    \ekvc@ifspace{#1}%
      {\ekvcSecondaryKeys@b#1\ekvc@stop}%
      {\ekvc@err@missing@type{#1}\@gobble}%
  }
\protected\def\ekvcSecondaryKeys@b#1 #2\ekvc@stop
  {%
    \ekv@ifdefined{ekvc@p@#1}%
      {\csname ekvc@p@#1\endcsname}%
      {%
        \ekv@ifdefined{ekvc@t@#1}%
          {\csname ekvc@t@#1\endcsname}%
          {\ekvc@err@unknown@keytype{#1}\@firstoftwo\@gobble}%
      }%
      {#2}%
  }
%    \end{macrocode}
% \end{macro}
%
% \begin{macro}{\ekvcChange}
%   This can be used to change the defaults of an \expkvc\ defined macro. It
%   checks whether there is a set with the correct name and that the macro is
%   defined. If both is true the real work is done by |\ekvc@change|.
%    \begin{macrocode}
\protected\long\def\ekvcChange#1%
  {%
    \ekvifdefinedset{\string#1}%
      {%
        \ekvc@ifdefined#1%
          {\ekvc@change#1}%
          {\ekvc@err@no@key@macro#1\@gobble}%
      }%
      {\ekvc@err@no@key@macro#1\@gobble}%
  }
%    \end{macrocode}
% \end{macro}
%
% \begin{macro}[internal]
%   {\ekvc@change,\ekvc@change@a,\ekvc@change@b,\ekvc@change@c}
%   First we need to see whether the macro is currently |\long|. For this we get
%   the meaning and will parse it. |#1| is the macro name in which we want to
%   change the defaults.
%    \begin{macrocode}
\protected\def\ekvc@change#1%
  {\expandafter\ekvc@change@a\meaning#1\ekv@stop#1}
%    \end{macrocode}
%   A temporary definition to get the stringified |macro:|. |##1| will be the
%   list of prefixes, we don't care for the exact contents of |##2| and |##3|.
%    \begin{macrocode}
\def\ekvc@change@a#1%
  {%
    \protected\def\ekvc@change@a##1#1##2->##3\ekv@stop
      {%
        \ekvc@change@iflong{##1}%
          {\ekvc@change@b{}}%
          {\ekvc@change@b{\long}}%
      }%
  }
\expandafter\ekvc@change@a\expandafter{\detokenize{macro:}}
%    \end{macrocode}
%   Next we expand the macro once to get its contents (including the current
%   default values with their markers). |#1| is either |\long| or empty, |#2| is
%   the macro.
%    \begin{macrocode}
\protected\def\ekvc@change@b#1#2%
  {\expandafter\ekvc@change@c\expandafter{#2{##1}}{#1}#2}
%    \end{macrocode}
%   Here we place an unbalanced closing brace after the expansion of the macro.
%   Then we just parse the \kv-list with |\ekvset|, that will exchange the
%   values behind the markers. Once those are changed we give control to
%   |\ekvc@change@d|. The |\ekvset| step might horribly fail if the user defined
%   some keys that don't behave nice. |#1| is the expansion of the macro, |#2|
%   is either |\long| or empty, |#3| is the macro, and |#4| is the \kv-list
%   containing the new defaults.
%    \begin{macrocode}
\ekv@exparg{\protected\long\def\ekvc@change@c#1#2#3#4}%
  {%
    \expandafter\iffalse\expandafter{\expandafter\fi
      \ekvset{\string#3}{#4}%
      \ekvc@change@d{#2}{#3}%
      #1%
    }%
  }
%    \end{macrocode}
%   The final step needs to put an unbalanced opening brace after |\def|. We do
%   that with the help of a temporary macro which stores everything necessary
%   for |\def| and expand an |\iffalse}\fi| construct to leave a single
%   opening brace. |#1| will be either empty or |\long| and |#2| is the macro.
%   Each of the macros defined with \expkvc\ takes exactly one parameter, so we
%   put that here as |####1|.
%    \begin{macrocode}
\protected\def\ekvc@change@d#1#2%
  {%
    \def\ekvc@tmp{#1\def#2####1}%
    \expandafter\ekvc@tmp\expandafter{\iffalse}\fi
  }
%    \end{macrocode}
% \end{macro}
%
% \begin{macro}[internal]{\ekvc@change@iflong,\ekvc@change@iflong@}
%   Checking whether a string contains the string representation of |\long| can
%   be done by gobbling everything up to the first |\long| and checking whether
%   the result is completely empty. We need a temporary macro to get the result
%   of |\string\long| inside the definitions.
%    \begin{macrocode}
\def\ekvc@change@iflong#1%
  {%
    \protected\def\ekvc@change@iflong##1%
      {\expandafter\ekv@ifempty\expandafter{\ekvc@change@iflong@##1#1}}%
    \def\ekvc@change@iflong@##1#1{}
  }
\expandafter\ekvc@change@iflong\expandafter{\string\long}
%    \end{macrocode}
% \end{macro}
%
% \begin{macro}{\ekvcPass}
%   This macro can be used to pass a value to a key of some macro (this way
%   more complicated key codes are possible that in the end pass processed
%   values on to some macro). The implemantation is pretty straight forward.
%    \begin{macrocode}
\long\def\ekvcPass#1#2%
  {%
    \ekvifdefined{\string#1}{#2}%
      {\csname\ekv@name{\string#1}{#2}\endcsname}%
      {\ekvc@err@unknown@key@or@macro{#1}{#2}\@gobble}%
  }
%    \end{macrocode}
% \end{macro}
%
%
% \subsubsection{Secondary Key Types}
%
% \begin{macro}[internal]{\ekvc@p@long,\ekvc@after@ptype}
%   The prefixes are pretty straight forward again. Just set |\ekvc@long| and
%   forward to the |@t@| type.
%    \begin{macrocode}
\protected\def\ekvc@p@long#1%
  {%
    \ekvc@ifspace{#1}%
      {%
        \let\ekvc@long\long
        \ekvc@after@ptype#1\ekvc@stop
      }%
      {\ekvc@err@missing@type{long #1}\@gobble}%
  }
\protected\def\ekvc@after@ptype#1 #2\ekvc@stop
  {%
    \ekv@ifdefined{ekvc@t@#1}%
      {\csname ekvc@t@#1\endcsname{#2}}%
      {\ekvc@err@unknown@keytype{#1}\@gobble}%
  }
%    \end{macrocode}
% \end{macro}
%
% \begin{macro}[internal]
%   {
%     \ekvc@t@meta, \ekvc@t@nmeta,
%     \ekvc@type@meta, \ekvc@type@meta@a, \ekvc@type@meta@b
%   }
%   The |meta| and |nmeta| key types use a nested |\ekvset| to set other keys in
%   the same macro's \set.
%    \begin{macrocode}
\protected\def\ekvc@t@meta
  {%
    \edef\ekvc@tmp{\ekvc@set}%
    \expandafter\ekvc@type@meta\expandafter{\ekvc@tmp}\ekvc@long{##1}\ekvlet
  }
\protected\def\ekvc@t@nmeta#1%
  {%
    \ekvc@assert@not@long{nmeta #1}%
    \edef\ekvc@tmp{\ekvc@set}%
    \expandafter\ekvc@type@meta\expandafter{\ekvc@tmp}{}{}\ekvletNoVal{#1}%
  }
\protected\long\def\ekvc@type@meta#1#2#3#4#5#6%
  {%
    \expandafter\ekvc@type@meta@a\expandafter{\ekvset{#1}{#6}}{#2}{#3}%
    #4\ekvc@set{#5}\ekvc@tmp
  }
\protected\def\ekvc@type@meta@a
  {%
    \expandafter\ekvc@type@meta@b\expandafter
  }
\protected\long\def\ekvc@type@meta@b#1#2#3%
  {%
    #2\def\ekvc@tmp#3{#1}%
  }
%    \end{macrocode}
% \end{macro}
%
% \begin{macro}[internal]{\ekvc@t@alias}
%   |alias| just checks whether there is a key and/or |NoVal| key defined with
%   the target name and |\let| the key to those.
%    \begin{macrocode}
\protected\def\ekvc@t@alias#1#2%
  {%
    \ekvc@assert@not@long{alias #1}%
    \let\ekvc@tmp\@firstofone
    \ekvifdefined\ekvc@set{#2}%
      {%
        \ekvletkv\ekvc@set{#1}\ekvc@set{#2}%
        \let\ekvc@tmp\@gobble
      }%
      {}%
    \ekvifdefinedNoVal\ekvc@set{#2}%
      {%
        \ekvletkvNoVal\ekvc@set{#1}\ekvc@set{#2}%
        \let\ekvc@tmp\@gobble
      }%
      {}%
    \ekvc@tmp{\ekvc@err@unknown@key{#2}}%
  }
%    \end{macrocode}
% \end{macro}
%
% \begin{macro}[internal]{\ekvc@t@default}
%   The |default| key can be used to set a |NoVal| key for an existing key. It
%   will just pass the \val\ to the key macro of that other key.
%    \begin{macrocode}
\protected\long\def\ekvc@t@default#1#2%
  {%
    \ekvifdefined\ekvc@set{#1}%
      {%
        \ekvc@assert@not@long{default #1}%
        \edef\ekvc@tmp
          {%
            \unexpanded\expandafter
              {\csname\ekv@name\ekvc@set{#1}\endcsname{#2}}%
          }%
        \ekvletNoVal\ekvc@set{#1}\ekvc@tmp
      }%
      {\ekvc@err@unknown@key{#1}}%
  }
%    \end{macrocode}
% \end{macro}
%
% \begin{macro}[internal]{\ekvc@t@flag-bool}
%    \begin{macrocode}
\protected\expandafter\def\csname ekvc@t@flag-bool\endcsname#1#2%
  {%
    \ekvc@assert@not@long{flag-bool #1}%
    \unless\ifdefined#2\ekvcFlagNew#2\fi
    \ekvdef\ekvc@set{#1}%
      {%
        \ekv@ifdefined{ekvc@flag@set@##1}%
          {%
            \csname ekvc@flag@set@##1\expandafter\endcsname
              \ekvcFlagHeight#2\ekv@stop#2%
          }%
          {\ekvc@err@invalid@bool{##1}}%
      }%
  }
%    \end{macrocode}
% \end{macro}
%
% \begin{macro}[internal]
%   {\ekvc@t@flag-true,\ekvc@t@flag-false,\ekvc@t@flag-raise,\ekvc@type@flag}
%    \begin{macrocode}
\protected\def\ekvc@type@flag#1#2#3#4%
  {%
    \ekvc@assert@not@long{flag-#1 #3}%
    \unless\ifdefined#4\ekvcFlagNew#4\fi
    \ekv@exparg{\ekvdefNoVal\ekvc@set{#3}}{#2#4}%
  }
\protected\expandafter\def\csname ekvc@t@flag-true\endcsname
  {\ekvc@type@flag{true}\ekvcFlagSetTrue}
\protected\expandafter\def\csname ekvc@t@flag-false\endcsname
  {\ekvc@type@flag{false}\ekvcFlagSetFalse}
\protected\expandafter\def\csname ekvc@t@flag-raise\endcsname
  {\ekvc@type@flag{raise}\ekvcFlagRaise}
%    \end{macrocode}
% \end{macro}
%
%
% \subsubsection{Flags}
%
% The basic idea of flags is to store information by the fact that \TeX\
% expandably assigns the meaning |\relax| to undefined control sequences which
% were built with |\csname|. This mechanism is borrowed from \pkg{expl3}.
%
% \begin{macro}[internal]{\ekvc@flag@name,\ekvc@flag@namescheme}
%   Flags follow a simple naming scheme which we define here. |\ekvc@flag@name|
%   will store the name of an internal function that is used to build names of
%   the second naming scheme defined by |\ekvc@flag@namescheme|.
%    \begin{macrocode}
\def\ekvc@flag@name{ekvcf\string}
\def\ekvc@flag@namescheme#1#2{ekvch#2#1}
%    \end{macrocode}
% \end{macro}
%
% \begin{macro}{\ekvcFlagHeight}
%   For semantic reasons we use |\number| with another name.
%    \begin{macrocode}
\let\ekvcFlagHeight\number
%    \end{macrocode}
% \end{macro}
%
% \begin{macro}{\ekvcFlagNew}
%   This macro defines a new flag. It stores the function build with the
%   |\ekvc@flag@name| naming scheme after the internal function
%   |\ekvc@flag@height| that'll determine the current flag height. It'll also
%   define the macro named via |\ekvc@flag@name| to build names according to
%   |\ekvc@flag@namescheme|.
%    \begin{macrocode}
\protected\def\ekvcFlagNew#1%
  {%
    \edef#1%
      {%
        \unexpanded{\ekvc@flag@height}%
          \unexpanded\expandafter{\csname\ekvc@flag@name#1\endcsname}%
      }%
    \ekv@expargtwice
      {\expandafter\def\csname\ekvc@flag@name#1\endcsname##1}%
      {\expandafter\ekvc@flag@namescheme\expandafter{\string#1}{##1}}%
  }
%    \end{macrocode}
% \end{macro}
%
% \begin{macro}[internal]{\ekvc@flag@height,\ekvc@flag@height@}
%   This macro gets the height of a flag by a simple loop. The first loop
%   iteration differs a bit from the following in that it doesn't have to get
%   the current iteration count. The space at the end of |\ekvc@flag@height|
%   ends the |\number| evaluation.
%    \begin{macrocode}
\def\ekvc@flag@height#1%
  {%
    \ifcsname#10\endcsname
      \ekvc@flag@height@1\ekv@stop#1%
    \fi
    \@firstofone{0} % leave this space
  }
\def\ekvc@flag@height@#1\ekv@stop#2\fi\@firstofone#3%
  {%
    \fi
    \ifcsname#2{#1}\endcsname
      \expandafter\ekvc@flag@height@\the\numexpr#1+1\relax\ekv@stop#2%
    \fi
    \@firstofone{#1}%
  }
%    \end{macrocode}
% \end{macro}
%
% \begin{macro}{\ekvcFlagRaise}
%   Raising a flag simply means letting the |\ekvc@flag@namescheme| macro for
%   the current height to relax. The result of raising a flag is that its height
%   is bigger by $1$.
%    \begin{macrocode}
\ekv@exparg{\def\ekvcFlagRaise#1}%
  {%
    \expandafter\expandafter\expandafter\@gobble\expandafter
      \csname\ekvc@flag@namescheme{\string#1}{\ekvcFlagHeight#1}\endcsname
  }
%    \end{macrocode}
% \end{macro}
%
% \begin{macro}{\ekvcFlagSetTrue,\ekvcFlagSetFalse}
% \begin{macro}[internal]{\ekvc@flag@set@true,\ekvc@flag@set@false}
%   A flag is considered true if its current height is odd, and as false if it
%   is even. Therefore |\ekvcFlagSetTrue| and |\ekvcFlagSetFalse| only need to
%   raise the flag if the opposing boolean value is the current one.
%    \begin{macrocode}
\def\ekvcFlagSetTrue#1%
  {\expandafter\ekvc@flag@set@true\ekvcFlagHeight#1\ekv@stop#1}
\def\ekvcFlagSetFalse#1%
  {\expandafter\ekvc@flag@set@false\ekvcFlagHeight#1\ekv@stop#1}
%    \end{macrocode}
%   We can expand |\ekvc@flag@namescheme| at definition time here, which is why
%   we're using a temporary definition to set up |\ekvc@flag@set@true| and
%   |\ekvc@flag@set@false|.
%    \begin{macrocode}
\def\ekvc@flag@set@true#1%
  {%
    \def\ekvc@flag@set@true##1\ekv@stop##2%
      {%
        \ifodd##1
          \ekv@fi@gobble
        \fi
        \@firstofone{\expandafter\@gobble\csname#1\endcsname}%
      }%
    \def\ekvc@flag@set@false##1\ekv@stop##2%
      {%
        \ifodd##1
          \ekv@fi@firstofone
        \fi
        \@gobble{\expandafter\@gobble\csname#1\endcsname}%
      }%
  }
\expandafter\ekvc@flag@set@true\expandafter
  {\ekvc@flag@namescheme{\string#2}{#1}}
%    \end{macrocode}
% \end{macro}
% \end{macro}
%
% \begin{macro}{\ekvcFlagIf}
%   As already explained, truthiness is defined as a flag's height is odd, so we
%   just branch accordingly here.
%    \begin{macrocode}
\def\ekvcFlagIf#1%
  {%
    \ifodd#1%
      \ekv@fi@firstoftwo
    \fi
    \@secondoftwo
  }
%    \end{macrocode}
% \end{macro}
%
% \begin{macro}{\ekvcFlagIfRaised}
%   This macro uses flags as a switch, if a flag's current height is bigger than
%   $0$ this test yields true.
%    \begin{macrocode}
\ekv@exparg{\def\ekvcFlagIfRaised#1}%
  {%
    \expandafter\ifcsname\ekvc@flag@namescheme{\string#1}0\endcsname
      \ekv@fi@firstoftwo
    \fi
    \@secondoftwo
  }
%    \end{macrocode}
% \end{macro}
%
% \begin{macro}{\ekvcFlagReset}
% \begin{macro}[internal]{\ekvc@flag@reset,\ekvc@flag@reset@}
%   Resetting works by locally letting all the defined internal macros named
%   after |\ekvc@flag@namescheme| to undefined.
%    \begin{macrocode}
\protected\def\ekvcFlagReset#1%
  {\expandafter\ekvc@flag@reset\csname\ekvc@flag@name#1\endcsname}
\protected\def\ekvc@flag@reset#1%
  {%
    \ifcsname#10\endcsname
      \expandafter\let\csname#10\endcsname\ekvc@undefined
      \ekvc@flag@reset@1\ekv@stop#1%
    \fi
  }
\protected\def\ekvc@flag@reset@#1\ekv@stop#2\fi
  {%
    \fi
    \ifcsname#2{#1}\endcsname
      \expandafter\let\csname#2{#1}\endcsname\ekvc@undefined
      \expandafter\ekvc@flag@reset@\the\numexpr#1+1\relax\ekv@stop#2%
    \fi
  }
%    \end{macrocode}
% \end{macro}
% \end{macro}
%
% \begin{macro}{\ekvcFlagGetHeight}
% \begin{macro}[internal]{\ekvc@flag@get@height@single}
%   These are just small helpers, first getting the height of the flag and then
%   passing it on to the user supplied code.
%    \begin{macrocode}
\def\ekvcFlagGetHeight#1%
  {\expandafter\ekvc@flag@get@height@single\ekvcFlagHeight#1\ekv@stop}
\long\def\ekvc@flag@get@height@single#1\ekv@stop#2{#2{#1}}
%    \end{macrocode}
% \end{macro}
% \end{macro}
%
% \begin{macro}{\ekvcFlagGetHeights}
% \begin{macro}[internal]
%   {\ekvc@flag@get@heights,\ekvc@flag@get@heights@,\ekvc@flag@get@heights@done}
%   This works by a simple loop that stops at |\ekv@stop|. As long as that
%   marker isn't hit, get the next flags height and put it into a list after
%   |\ekv@stop|. |\ekvc@flag@get@heights@| uses the same marker name for the
%   end of the height, which shouldn't clash in any case. Once we're done we
%   remove the remainder of the current iteration and leave the user supplied
%   code in the input stream with all the flags' heights as a single argument.
%    \begin{macrocode}
\def\ekvcFlagGetHeights#1%
  {%
    \ekvc@flag@get@heights#1\ekv@stop{}%
  }
\def\ekvc@flag@get@heights#1%
  {%
    \ekv@gobbleto@stop#1\ekvc@flag@get@heights@done\ekv@stop
    \expandafter\ekvc@flag@get@heights@\ekvcFlagHeight#1\ekv@stop
  }
\def\ekvc@flag@get@heights@#1\ekv@stop#2\ekv@stop#3%
  {\ekvc@flag@get@heights#2\ekv@stop{#3{#1}}}
\long\def\ekvc@flag@get@heights@done
    \ekv@stop
    \expandafter\ekvc@flag@get@heights@\ekvcFlagHeight\ekv@stop\ekv@stop#1#2%
  {#2{#1}}
%    \end{macrocode}
% \end{macro}
% \end{macro}
%
%
% \subsubsection{Helper Macros}
%
% \begin{macro}[internal]{\ekvc@ifspace,\ekvc@ifspace@}
% A test which can be reduced to an if-empty by gobbling everything up to the
% first space.
%    \begin{macrocode}
\long\def\ekvc@ifspace#1%
  {%
    \ekvc@ifspace@#1 \ekv@ifempty@B
      \ekv@ifempty@false\ekv@ifempty@A\ekv@ifempty@B\@firstoftwo
  }
\long\def\ekvc@ifspace@#1 % keep this space
  {%
    \ekv@ifempty@\ekv@ifempty@A
  }
%    \end{macrocode}
% \end{macro}
%
%
% \subsubsection{Assertions}
% \begin{macro}[internal]{\ekvc@assert@not@long}
%   Some keys don't want to be |long| and we have to educate the user, so let's
%   throw an error if someone wanted these to be long.
%    \begin{macrocode}
\long\def\ekvc@assert@not@long#1{\ifx\ekvc@long\long\ekvc@err@no@long{#1}\fi}
%    \end{macrocode}
% \end{macro}
%
%
% \subsubsection{Messages}
%
% \begin{macro}[internal]
%   {
%     \ekvc@err@toomany,\ekvc@err@value@required,\ekvc@err@missing@type,
%     \ekvc@err@already@defined,\ekvc@err@no@key@macro
%   }
%   Boring unexpandable error messages.
%    \begin{macrocode}
\protected\def\ekvc@err@toomany#1%
  {%
    \errmessage{expkv-cs Error: Too many keys for macro `\string#1'}%
  }
\protected\def\ekvc@err@value@required#1%
  {%
    \errmessage{expkv-cs Error: Missing value for key `\unexpanded{#1}'}%
  }
\protected\def\ekvc@err@missing@type#1%
  {%
    \errmessage
      {expkv-cs Error: Missing type for secondary key `\unexpanded{#1}'}%
  }
\protected\def\ekvc@err@no@long#1%
  {%
    \errmessage
      {expkv-cs Error: prefix `long' not accepted for `\unexpanded{#1}'}%
  }
\protected\def\ekvc@err@already@defined#1%
  {%
    \errmessage{expkv-cs Error: Macro `\string#1' already defined}%
  }
\protected\def\ekvc@err@unknown@keytype#1%
  {%
    \errmessage{expkv-cs Error: Unknown key type `\unexpanded{#1}'}%
  }
\protected\def\ekvc@err@unknown@key#1%
  {%
    \errmessage
      {expkv-cs Error: Unknown key `\unexpanded{#1}' for macro `\ekvc@set'}%
  }
\long\def\ekvc@err@no@key@macro#1%
  {\errmessage{expkv-cs Error: \string#1 is no key=val macro}}
%    \end{macrocode}
% \end{macro}
%
% \begin{macro}[internal]{\ekvc@err}
%   We need a way to throw error messages expandably in some contexts.
%    \begin{macrocode}
\ekv@exparg{\long\def\ekvc@err#1}{\ekverr{expkv-cs}{#1}}
%    \end{macrocode}
% \end{macro}
%
% \begin{macro}[internal]
%   {
%     \ekvc@err@unknown@hash,\ekvc@err@empty@hash,\ekvc@err@missing@hash,
%     \ekvc@err@invalid@bool
%   }
%   And here are the expandable error messages.
%    \begin{macrocode}
\long\def\ekvc@err@unknown@hash#1{\ekvc@err{unknown hash `#1'}}
\long\def\ekvc@err@missing@hash#1{\ekvc@err{hash `#1' not found}}
\long\def\ekvc@err@empty@hash{\ekvc@err{empty hash}}
\def\ekvc@err@invalid@bool#1{\ekvc@err{invalid boolean value `#1'}}
\long\def\ekvc@err@unknown@key@or@macro#1#2%
  {\ekvc@err{unknown key `#2' for macro #1}}
%    \end{macrocode}
% \end{macro}
%
%
% Now everything that's left is to reset the category code of |@|.
%    \begin{macrocode}
\catcode`\@=\ekvc@tmp
%    \end{macrocode}
%
% \gobbledocstriptag
%</tex>
%^^A=<<
%
%^^A=<<
%
% \end{implementation}^^A=<<
%
% \clearpage
% \PrintIndex
%
\endinput
%
^^A vim: ft=tex fdm=marker fmr=>>=,=<<
}{}
\ProvidesFile{expkv-cs.dtx}
  [\csname ekvcDate\endcsname\ define expandable key=val macros using expkv]
\PassOptionsToPackage{full}{textcomp}
\documentclass{l3doc}
\RequirePackage[oldstylenums,nott]{kpfonts}
\input{glyphtounicode}
\pdfgentounicode=1
\RequirePackage{tcolorbox}
\newtcolorbox{exresult}[2][]
  {colback=expkvgrey!10!white,colframe=expkvgrey,fontupper=\small,width={#2},#1}
\newtcbox\exres[1][]
  {
     colback=expkvgrey!10!white
    ,colframe=expkvgrey
    ,size=small
    ,nobeforeafter
    ,tcbox raise base
    ,fontupper=\small
    ,#1
  }
\RequirePackage{listings}
\RequirePackage{booktabs}
\RequirePackage{array}
\RequirePackage{collcell}
\RequirePackage{siunitx}
\RequirePackage{xcolor}
\RequirePackage{caption}
\RequirePackage{microtype}
\RequirePackage{accsupp}
\lstdefinelanguage{expkvc}[primitive]{TeX}
  {
    ,moretexcs=[4]^^A e-TeX
      {
        expanded,
        numexpr,
        protected,
      }
    ,moretexcs=[5]^^A plain/LaTeX
      {
        approx,
        begin,
        empty,
        item,
        LaTeX,
        makeatletter,makeatother,multicolumn,
        newcommand,newdimen,
        RequirePackage,
        rule,
        TeX,
        textit,texttt,
        usepackage,
      }
    ,moretexcs=[6]^^A used packages
      {
        ^^A xfp
        fpeval,
        ^^A keyval
        setkeys,
        ^^A yax
        defactiveparameter,storevalue,setparameterlist
      }
    ,moretexcs=[2]^^A expkv-cs macros
      {
        ekvcSplit,ekvcSplitAndForward,ekvcSplitAndUse,
        ekvcHash,ekvcHashAndForward,ekvcHashAndUse,
        ekvcValue,ekvcValueFast,ekvcValueSplit,ekvcValueSplitFast,
        ekvcSecondaryKeys,
        ekvcChange,
        ekvcFlagNew,
        ekvcFlagHeight,
        ekvcFlagRaise,
        ekvcFlagSetTrue,ekvcFlagSetFalse,
        ekvcFlagIf,ekvcFlagIfRaised,
        ekvcFlagReset,
        ekvcFlagGetHeight,ekvcFlagGetHeights,
        ekvcPass
      }
    ,moretexcs=[3]^^A expkv-pkg macros
      {
        ^^A expkv
        ekvdef,ekvdefNoVal,^^A
        ekvlet,ekvletNoVal,ekvletkv,ekvletkvNoVal,^^A
        ekvdefunknown,ekvdefunknownNoVal,^^A
        ekvredirectunknown,ekvredirectunknownNoVal,^^A
        ekvset,ekvsetSneaked,^^A
        ekvsetdef,ekvsetSneakeddef,ekvsetdefSneaked,^^A
        ekvparse,^^A
        ekvVersion,ekvDate,^^A
        ekvifdefined,ekvifdefinedNoVal,ekvifdefinedset,^^A
        ekvbreak,ekvbreakPreSneak,ekvbreakPostSneak,^^A
        ekvsneak,ekvsneakPre,^^A
        ekvchangeset,^^A
        ekvoptarg,ekvoptargTF,^^A
        ekverr,
        ^^A expkv-def
        ekvdefinekeys
      }
  }
\colorlet{codeparam}{cyan!65!black}
\def\mylstwd{.55em}
\lstset
  {
    ,language=expkvc
    ,flexiblecolumns=false
    ,basewidth=\mylstwd
    ,gobble=2
    ,basicstyle=\fontfamily{jkp}\itshape
    ,texcsstyle=*[1]{\bfseries\color{expkvgrey}} ^^A primitives
    ,texcsstyle=*[2]{\bfseries\color{expkvred}}  ^^A expkv
    ,texcsstyle=*[3]{\color{expkvred}}           ^^A expkv-pkg
    ,texcsstyle=*[4]{\bfseries\color{expkvgrey}} ^^A e-TeX
    ,texcsstyle=*[5]{\bfseries\color{expkvgrey}} ^^A plain/LaTeX macros
    ,texcsstyle=*[6]{}                           ^^A macros of other packages
    ,commentstyle=\color[gray]{0.4}
    ,literate=
              {\{} {{\CodeUpBf\{}}{1}
              {\}} {{\CodeUpBf\}}}{1}
              {$}  {{\CodeUpBf\$}}{1}
              {[}  {{\CodeUp[}}{1}
              {]}  {{\CodeUp]}}{1}
              {(}  {{\CodeUp(}}{1}
              {)}  {{\CodeUp)}}{1}
              {*}  {{$*$}}{1}
              {1}  {{\CodeUp{1}}}{1}
              {2}  {{\CodeUp{2}}}{1}
              {3}  {{\CodeUp{3}}}{1}
              {4}  {{\CodeUp{4}}}{1}
              {5}  {{\CodeUp{5}}}{1}
              {6}  {{\CodeUp{6}}}{1}
              {7}  {{\CodeUp{7}}}{1}
              {8}  {{\CodeUp{8}}}{1}
              {9}  {{\CodeUp{9}}}{1}
              {0}  {{\CodeUp{0}}}{1}
              {##} {{\CodeColored{codeparam}{1}{\#}}}{1}
              {##1}{{\CodeColored{codeparam}{2}{\#1}}}{2}
              {##2}{{\CodeColored{codeparam}{2}{\#2}}}{2}
              {##3}{{\CodeColored{codeparam}{2}{\#3}}}{2}
              {##4}{{\CodeColored{codeparam}{2}{\#4}}}{2}
              {##5}{{\CodeColored{codeparam}{2}{\#5}}}{2}
              {##6}{{\CodeColored{codeparam}{2}{\#6}}}{2}
              {##7}{{\CodeColored{codeparam}{2}{\#7}}}{2}
              {##8}{{\CodeColored{codeparam}{2}{\#8}}}{2}
              {##9}{{\CodeColored{codeparam}{2}{\#9}}}{2}
              {<key>}{{$\langle$}key{$\rangle$}}{5}
              {<set>}{{$\langle$}set{$\rangle$}}{5}
              {<set1>}{{$\langle$}set1{$\rangle$}}{6}
              {<set2>}{{$\langle$}set2{$\rangle$}}{6}
    ^^A,literate=*{<key>}{\key}{4}{<set>}{\set}{4}
  }
\newcommand*\CodeColored[3]{\textcolor{#1}{\makebox[\dimexpr\mylstwd*#2]{#3}}}
\newcommand*\CodeUpBf[1]{\makebox[\mylstwd]{\textup{\textbf{#1}}}}
\newcommand*\CodeUp[1]{\makebox[\mylstwd]{\textup{#1}}}
\RequirePackage{randtext}
\let\metaORIG\meta
\protected\def\meta #1{\texttt{\metaORIG{#1}}}
\renewcommand*\thefootnote{\fnsymbol{footnote}}
\definecolor{expkvred}{HTML}{9F393D}
\colorlet{expkvgrey}{black!75}
\makeatletter
\newcommand*\example{\par\smallskip\noindent\textit{Example:} \ignorespaces}
\newcommand*\expkv
  {^^A
    \texorpdfstring
      {^^A
        \mbox
          {^^A
            \BeginAccSupp{ActualText=expkv}^^A
            \href{https://github.com/Skillmon/tex_expkv}
              {^^A
                \rmfamily
                \bfseries
                {\color{expkvgrey}e\kern-.05em x\kern-.05em}^^A
                \lower.493ex
                  \hbox{{\color{expkvgrey}P}\kern-.1em{\color{expkvred}k}}^^A
                \kern-.18em{\color{expkvred}v}^^A
              }^^A
            \EndAccSupp{}^^A
          }^^A
      }
      {expkv}^^A
  }
\newcommand*\expkvpkg[1]
  {^^A
    \texorpdfstring
      {^^A
        \mbox
          {^^A
            \BeginAccSupp{ActualText=expkv-#1}^^A
            \href{https://github.com/Skillmon/tex_expkv-#1}
              {^^A
                \rmfamily
                \bfseries
                {\color{expkvgrey}e\kern-.05em x\kern-.05em}^^A
                \lower.493ex
                  \hbox{{\color{expkvgrey}P}\kern-.1em{\color{expkvred}k}}^^A
                \kern-.18em{\color{expkvred}v}^^A
                {\color{expkvgrey}^^A
                  \kern.05em\rule[-.1ex]{.08em}{1.2ex}\kern.05em\textsc{#1}^^A
                }^^A
              }^^A
            \EndAccSupp{}^^A
          }^^A
      }
      {expkv-#1}^^A
  }
\newcommand*\expkvd{\expkvpkg{def}}
\newcommand*\expkvc{\expkvpkg{cs}}
\newcommand\kv{\meta{key}=\meta{value}}
\newcommand\key{\meta{key}}
\newcommand\val{\meta{value}}
\newcommand\set{\meta{set}}
\newcommand\enflong{\textcolor{black}{long}}
\newcommand\alllong{\textcolor{gray}{long}}
\newcommand\notlong{\textcolor{red!80!black}{long}}
\newcommand\prefixes[1]
  {^^A
    \hfill
    \ifcase\numexpr#1\relax\or
      \enflong\or
      \alllong\or
      \notlong\fi
  }
\hypersetup{linkcolor=red!80!black,urlcolor=purple!80!black}
\DoNotIndex{\def,\edef,\,,\=,\begingroup,\catcode,\chardef,\csname,\endcsname}
\DoNotIndex{\endgroup,\endinput,\errmessage,\expandafter,\input,\let,\long,\z@}
\DoNotIndex{\protected,\ProvidesFile,\ProvidesPackage,\relax,\space,\advance}
\DoNotIndex{\@,\unexpanded,\string,\expanded,\dimexpr,\global,\glueexpr,\hbox}
\DoNotIndex{\numexpr,\RequirePackage,\setbox,\the,\unless,\xdef,\gdef,\newcount}
\DoNotIndex{\number}
\DoNotIndex{\^,\@firstofone,\@firstoftwo,\@gobble,\@secondoftwo}
\DoNotIndex{\ifcsname}
\DoNotIndex{\ifx}
\DoNotIndex{\ifodd}
\DoNotIndex{\ifnum}
\DoNotIndex{\ifdefined}
\DoNotIndex{\iffalse}
\DoNotIndex{\iftrue}
\DoNotIndex{\else}
\DoNotIndex{\fi}
\@gobble\fi ^^A ignoring \ifx, \ifcsname, etc., but only one \fi
\@gobble\fi ^^A ignoring \ifx, \ifcsname, etc., but only one \fi
\@gobble\fi ^^A ignoring \ifx, \ifcsname, etc., but only one \fi
\@gobble\fi ^^A ignoring \ifx, \ifcsname, etc., but only one \fi
\@gobble\fi ^^A ignoring \ifx, \ifcsname, etc., but only one \fi
\@gobble\fi ^^A ignoring \ifx, \ifcsname, etc., but only one \fi
\@ifdefinable\gobbledocstriptag{\def\gobbledocstriptag#1>{}}
\makeatother
\begin{document}
  \title
    {^^A
      \texorpdfstring
        {^^A
          \huge\expkvc
          \\[\medskipamount]
          \Large define expandable \kv\ macros using \expkv
        }{expkv-cs - define expandable key=val macros using expkv}^^A
    }
  \date{\ekvcDate\space v\ekvcVersion}
  \author{Jonathan P. Spratte\thanks{\protect\randomize{jspratte@yahoo.de}}}
  \DocInput{expkv-cs.dtx}
\end{document}
%</driver>^^A=<<
% \fi
%
% \maketitle
% \renewcommand*\thefootnote{\arabic{footnote}}
%
% \begin{abstract}
% \noindent\parfillskip=0pt
% \expkvc\ provides two small interfaces to define expandable \kv\ macros using
% \expkv. It therefore lowers the entrance boundary to expandable \kv\ macros.
% The stylised name is \expkvc\ but the files use \file{expkv-cs}, this is due
% to CTAN-rules which don't allow \string| in package names since that is the
% pipe symbol in *nix shells.
% \end{abstract}
%
% \tableofcontents
%
% \begin{documentation}^^A>>=
%
% \section{Documentation}
%
% The \expkv\ package enables the new possibility of creating \kv\ macros which
% are fully expandable. The creation of such macros is however cumbersome for
% the average user. \expkvc\ tries to step in here. It provides interfaces to
% define \kv\ macros without worrying too much about the implementation. In case
% you're wondering now, the |cs| in \expkvc\ stands for control sequence,
% because |def| was already taken by \expkvd\ and ``control sequence'' is the
% term D.\,E.\,Knuth used in his \TeX book for named commands hence macros
% (though he also used the term ``macro''). So \expkvc\ defines control
% sequences for and with \expkv.
%
% There are two different approaches supported by this package. The first is
% splitting the keys up into individual arguments, the second is providing all
% the keys as a single argument to the underlying macro and getting an
% individual \val\ by using a hash. Well, actually there is no real hash, just
% some markers which are parsed, but this shouldn't be apparent to the user, the
% behaviour matches that of a hash-table.
%
% In addition to these two methods of defining a macro with primary keys a way
% to define secondary keys, which can reference the primary ones, is provided.
% These secondary keys don't correspond to an argument or an entry in the hash
% table directly but might come in handy for the average use case. Each macro
% has its own set of primary and secondary keys.
%
% A word of advice you should consider: If your macro doesn't have to be
% expandable (and often it doesn't) consider not using \expkvc. The interface
% has some overhead (though it still is fast -- check \autoref{sec:speed}) and
% the approach has its limits in versatility. If you don't need to be
% expandable, you should consider either defining your keys manually using
% \expkv\ or using \expkvd\ for convenience. Or you resort to another \kv\
% interface. Nevertheless setting up macros with \expkvc, especially with
% |\ekvcSplit|, is very convenient in my opinion, so if you just want to define
% a single macro with just a few keys this might be the way to go.
%
% \expkvc\ is usable as generic code and as a \LaTeX\ package. It'll
% automatically load \expkv\ in the same mode as well. To use it, just use one
% of
% \begin{lstlisting}
% \usepackage{expkv-cs} % LaTeX
% \input expkv-cs       % plainTeX
% \end{lstlisting}
%
%
% \subsection{Define Macros and Primary Keys}
%
% All macros defined with \expkvc\ have to be previously undefined or have the
% |\meaning| of |\relax|. This is necessary as there is no way to automatically
% undefine keys once they are set up (neither \expkv\ nor \expkvc\ keep track of
% defined keys) -- so to make sure there are no conflicts only new definitions
% are allowed (that's not the case for individual keys, only for frontend
% macros).
%
%
% \subsubsection{Primary Keys}\label{sec:primaries}
%
% In the following descriptions there will be one argument named \meta{primary
% keys}. This argument should be a \kv\ list where each \key\ will be one
% primary key and \val\ the associated initial value. By default all keys are
% defined short, but you can define long keys by prefixing \key\ with |long|
% (\emph{e.g.}, \texttt{long name=Jonathan P. Spratte}). You only need |long| if
% the key should be able to take a |\par| token. Note however that |long| keys
% are a microscopic grain faster (due to some internals of \expkvc). Only if at
% least one of the keys was |long| the \meta{cs} in the following defining
% macros will be |\long|. For obvious reasons there is no possibility to define
% a macro or key as |\protected|.
%
% At the moment \expkvc\ doesn't require any internal keys, but I can't foresee
% whether this will be the case in the future as well, as it might turn out that
% some features I deem useful can't be implemented without such internal keys.
% Because of this, please don't use key names starting with
% \texttt{EKVC\string|} as that should be the private name space.
%
%
% \subsubsection{Split}
%
% The split variants will provide the key values as separate arguments. This
% limits the number of keys for which this is truly useful.
%
% \begin{function}{\ekvcSplit}
%   \begin{syntax}
%     \cs{ekvcSplit}\meta{cs}\marg{primary keys}\marg{definition}
%   \end{syntax}
%   This defines \meta{cs} to be a macro taking one mandatory argument which
%   should contain a \kv\ list. The \meta{primary keys} will be defined for this
%   macro (see \autoref{sec:primaries}). The \meta{definition} is the code that
%   will be executed. You can access the \val\ of a \key\ by using a macro
%   parameter from |#1| to |#9|. The order of the macro parameters will be the
%   order provided in the \meta{primary keys} list (so |#1| is the \val\ of the
%   key defined first). With |\ekvcSplit| you can define macros using at most
%   nine primary keys.
% \end{function}
% \example The following defines a macro |\foo| that takes the keys |a| and |b|
% and outputs their values in a textual form:\\
% \begin{minipage}{.75\linewidth}
% \begin{lstlisting}
% \ekvcSplit\foo{a=a,b=b}{a is #1.\par b is #2.\par}
% \foo{}
% \foo{b=e}
% \end{lstlisting}
% \end{minipage}^^A
% \begin{exresult}[nobeforeafter,box align=center]{.25\linewidth}
%   \ekvcSplit\foo{a=a,b=b}{a is #1.\par b is #2.\par}
%   \foo{}
%   \foo{b=e}
% \end{exresult}
%
% \begin{function}{\ekvcSplitAndForward}
%   \begin{syntax}
%     \cs{ekvcSplitAndForward}\meta{cs}\marg{after}\marg{primary keys}
%   \end{syntax}
%   This defines \meta{cs} to be a macro taking one mandatory argument which
%   should contain a \kv\ list. You can use as many primary keys as you want
%   with this. The primary keys will be forwarded to \meta{after} as braced
%   arguments (as many as necessary for your primary keys). The order of the
%   braced arguments will be the order of your primary key definitions. In
%   \meta{after} you can use just a single control sequence, or some arbitrary
%   stuff which will be left in the input stream before your braced values (with
%   one set of braces stripped from \meta{after}), so both of the following
%   would be fine:
% \begin{lstlisting}
% \ekvcSplitAndForward\foo\foo@aux{keyA = A, keyB = B}
% \ekvcSplitAndForward\foo{\foo@aux{more args}}{keyA = A, keyB = B}
% \end{lstlisting}
% \end{function}
% In the first case |\foo@aux| should take at least two arguments (|keyA| and
% |keyB|), in the second case at least three (|more args|, |keyA|, and |keyB|).
%
% \begin{function}{\ekvcSplitAndUse}
%   \begin{syntax}
%     \cs{ekvcSplitAndUse}\meta{cs}\marg{primary keys}
%   \end{syntax}
%   This will roughly do the same as |\ekvcSplitAndForward|, but instead of
%   specifying what will be used after splitting the keys, \meta{cs} will use
%   what follows the \kv\ list. So its syntax will be
% \end{function}
%   \begin{syntax}
%     \meta{cs}\{\key=\val, \ldots\}\marg{after}
%   \end{syntax}
% and the code in |after| should expect at least as many arguments as the number
% of keys defined for \meta{cs}.
%
% \subsubsection{Hash}
%
% The hash variants will provide the key values as a single argument in which
% you can access specific values using a special macro. The implementation might
% be more convenient and scale better, \emph{but} it is much slower (for a
% primitive macro with a single key benchmarking was almost $1.7$ times slower,
% the root of which being the key access with |\ekvcValue|, not the parsing, and
% for a key access using |\ekvcValueFast| it was still about $1.2$ times
% slower). So if your macro uses less than ten primary keys, you should consider
% using the split approach.
%
% \begin{function}{\ekvcHash}
%   \begin{syntax}
%     \cs{ekvcHash}\meta{cs}\marg{primary keys}\marg{definition}
%   \end{syntax}
%   This defines \meta{cs} to be a macro taking one mandatory argument which
%   should contain a \kv\ list. You can use as many primary keys as you want.
%   The primary keys will be forwarded as a single argument containing every key
%   to the underlying macro. The underlying macro is defined as
%   \meta{definition}, in which you can access the \val\ of a \key\ by using
%   \texttt{\cs[no-index]{ekvcValue}\{\key\}\{\#1\}}.
% \end{function}
% \example This defines an equivalent macro like the |\foo| defined with
% |\ekvcSplit| earlier:\\
% \begin{minipage}{.75\linewidth}
% \begin{lstlisting}
% \ekvcHash\foo{a=a,b=b}{a is \ekvcValue{a}{#1}.\par
%                        b is \ekvcValue{b}{#1}.\par}
% \foo{}
% \foo{b=e}
% \end{lstlisting}
% \end{minipage}^^A
% \begin{exresult}[nobeforeafter,box align=center]{.25\linewidth}
%   \ekvcHash\foo{a=a,b=b}{a is \ekvcValue{a}{#1}.\par
%                          b is \ekvcValue{b}{#1}.\par}
%   \foo{}
%   \foo{b=e}
% \end{exresult}
%
% \begin{function}{\ekvcHashAndForward}
%   \begin{syntax}
%     \cs{ekvcHashAndForward}\meta{cs}\marg{after}\marg{primary keys}
%   \end{syntax}
%   This defines \meta{cs} to be a macro taking one mandatory argument which
%   should contain a \kv\ list. You can use as many primary keys as you want.
%   The primary keys will be forwarded as a single argument containing every key
%   to \meta{after}. You can use a single macro for \meta{after} or use some
%   arbitrary stuff, which will be left in the input stream before the hashed
%   \kv\ list with one set of braces stripped. In the
%   macro called in \meta{after} you can access the \val\ of a \key\ by using
%   \texttt{\cs[no-index]{ekvcValue}\{\key\}\{\#1\}} (or whichever argument the
%   hashed \kv\ list will be).
% \end{function}
%
% \begin{function}{\ekvcHashAndUse}
%   \begin{syntax}
%     \cs{ekvcHashAndUse}\meta{cs}\marg{primary keys}
%   \end{syntax}
%   This will roughly do the same as |\ekvcHashAndForward|, but instead of
%   specifying what will be used after hashing the keys, \meta{cs} will use what
%   follows the \kv\ list. So its syntax will be
% \end{function}
%   \begin{syntax}
%     \meta{cs}\{\key=\val, \ldots\}\marg{after}
%   \end{syntax}
%
% \begin{function}{\ekvcValue}
%   \begin{syntax}
%     \cs{ekvcValue}\{\key\}\marg{key list}
%   \end{syntax}
%   This is a safe way to access your keys in a hash variant. \key\
%   is the key which's \val\ you want to use out of the \meta{key list}.
%   \meta{key list} should be the key list argument forwarded to your underlying
%   macro by |\ekvcHash|, |\ekvcHashAndForward|, or |\ekvcHashAndUse|. It will
%   be tested whether the hash function to access that \key\ exists, the \key\
%   argument is not empty, and that the \meta{key list} really contains a \val\
%   of that \key. This macro needs exactly two steps of expansion and if used
%   inside of an |\edef| or |\expanded| context will protect the \val\ from
%   further expanding.
% \end{function}
%
% \begin{function}{\ekvcValueFast}
%   \begin{syntax}
%     \cs{ekvcValueFast}\{\key\}\marg{key list}
%   \end{syntax}
%   This behaves just like |\ekvcValue|, but \emph{without any} safety tests. As
%   a result this is about $1.4$ times faster \emph{but} will throw low level
%   \TeX\ errors eventually if the hash function isn't defined or the \key\
%   isn't part of the \meta{key list} (\emph{e.g.}, because it was defined as a
%   key for another macro -- all macros share the same hash function per \key).
%   Use it if you know what you're doing. This macro needs exactly three steps
%   of expansion in the no-errors case.
% \end{function}
%
% \begin{function}{\ekvcValueSplit}
%   \begin{syntax}
%     \cs{ekvcValueSplit}\{\key\}\marg{key list}\marg{next}
%   \end{syntax}
%   If you need a specific \key\ from a \meta{key list} more than once, it'll
%   be a good idea to only extract it once and from then on keep it as a
%   separate argument. Hence the macro \cs{ekvcValueSplit} will extract one
%   specific \key's value from the list and forward the remainder of the list
%   as the first and the \key's value as the second argument to \meta{next}, so
%   the result of this will be \meta{next}\marg{key list'}\marg{value} with
%   \meta{key list'} the remaining list. This is almost as fast as |\ekvcValue|
%   and runs the same tests. Keep in mind that you can't fetch for the same
%   \key\ again from \meta{key list'} as it got removed.
% \end{function}
% \example The following defines a macro |\foo| which will take three keys.
% Since the next parsing step will need the value of one of the keys multiple
% times we split that key off the list:\\
% \begin{minipage}{.75\linewidth}
% \begin{lstlisting}
% \ekvcHash\foo{a=a,b=b,c=c}
%   {\ekvcValueSplit{a}{#1}\foobar}
% \newcommand*\foobar[2]{a is #2.\par
%                        b is \ekvcValue{b}{#1}.\par
%                        c is \ekvcValue{c}{#1}.\par}
% \foo{}
% \end{lstlisting}
% \end{minipage}^^A
% \begin{exresult}[nobeforeafter,box align=center]{.25\linewidth}
%   \ekvcHash\foo{a=a,b=b,c=c}
%     {\ekvcValueSplit{a}{#1}\foobar}
%   \newcommand*\foobar[2]{a is #2.\par
%                          b is \ekvcValue{b}{#1}.\par
%                          c is \ekvcValue{c}{#1}.\par}
%   \foo{}
% \end{exresult}
%
% \begin{function}{\ekvcValueSplitFast}
%   \begin{syntax}
%     \cs{ekvcValueSplitFast}\{\key\}\marg{key list}\marg{next}
%   \end{syntax}
%   This behaves just like |\ekvcValueSplit|, but it won't run the same tests,
%   hence it is faster but more error prone, just like the relation between
%   |\ekvcValue| and |\ekvcValueFast|.
% \end{function}
%
%
% \subsection{Secondary Keys}\label{sec:secondaries}
%
% To remove some of the limitations with the approach that each primary key
% matches an argument or hash entry, you can define secondary keys. Those have
% to be defined for each macro but it doesn't matter whether that macro was a
% split or a hash variant. If a secondary key references another key it doesn't
% matter whether that other key is primary or secondary.
%
% Secondary keys can have a prefix (like |long|) which are called |p|-type
% prefix and must have a type (like |meta|) which are called |t|-type prefix.
% Some types might require some |p|-prefixes, while others might forbid those.
%
% Please keep in mind that key names shouldn't start with \texttt{EKVC\string|}.
%
% \begin{function}{\ekvcSecondaryKeys}
%   \begin{syntax}
%     \cs{ekvcSecondaryKeys}\meta{cs}\{\kv, \ldots\}
%   \end{syntax}
%   This is the front facing macro to define secondary keys. For the macro
%   \meta{cs} define \key\ to have definition \val. The general syntax for \key\
%   should be
%   \begin{quote}\ttfamily\small
%     \meta{prefix} \meta{name}
%   \end{quote}
%   Where \meta{prefix} is a space separated list of optional |p|-type prefixes
%   followed by one |t|-type prefix. The syntax of \val\ is dependent on the
%   used |t|-prefix.
% \end{function}
%
%
% \subsubsection{\texttt{p}-type Prefixes}
%
% There is only one |p|-prefix available, which is |long|.
%
% \begin{function}{long}
%   The following key will be defined |\long|.
% \end{function}
%
%
% \subsubsection{\texttt{t}-type Prefixes}
%
% If you're familiar with \expkvd\ you'll notice that the |t|-type prefixes
% provided here are much fewer. The expansion only concept doesn't allow for
% great variety in the auto-defined keys.
%
% The syntax examples of the |t|-prefixes will show which |p|-prefix will be
% automatically used by printing those black (\texttt{\enflong}), which
% will be available in grey (\texttt{\alllong}), and which will be disallowed in
% red (\texttt{\notlong}). This will be put flush right next to the syntax line.
%
% \begin{function}{meta}
%   \begin{syntax}
%     meta \key\ = \{\kv, \ldots\} \prefixes2
%   \end{syntax}
%   With a |meta| key you can set other keys. Whenever \key\ is used the keys in
%   the \kv\ list will be set to the values given there. You can use the \val\
%   given to \key\ by using |#1| in the \kv\ list. The keys in the \kv\ list can
%   be primary and secondary ones.
% \end{function}
%
% \begin{function}{nmeta}
%   \begin{syntax}
%     nmeta \key\ = \{\kv, \ldots\} \prefixes3
%   \end{syntax}
%   An |nmeta| key is like a |meta| key, but it doesn't take a value, so the
%   \kv\ list is static.
% \end{function}
%
% \begin{function}{alias}
%   \begin{syntax}
%     alias \key\ = \meta{key_2} \prefixes3
%   \end{syntax}
%   This assigns the definition of \meta{key_2} to \key. As a result \key\ is an
%   alias for \meta{key_2} behaving just the same. Both the value taking and the
%   |NoVal| version (that's \expkv\ slang for a key not accepting a value) will
%   be copied if they are defined when |alias| is used. Of course, \meta{key_2}
%   has to be defined, be it as a primary or secondary one.
% \end{function}
%
% \begin{function}{default}
%   \begin{syntax}
%     default \key\ = \marg{default} \prefixes3
%   \end{syntax}
%   If \key\ is a defined value taking key, you can define a |NoVal| version
%   with this that will behave as if \key\ was given \meta{default} as its
%   \val. Note that this doesn't change the initial values of primary keys set
%   at definition time in |\ekvcSplit| and friends. \key\ can be a primary or
%   secondary key.
% \end{function}
%
% \begin{function}{flag-bool}
%   \begin{syntax}
%     flag-bool \key\ = \meta{cs} \prefixes3
%   \end{syntax}
%   This is a secondary key that doesn't directly involve any of the primary or
%   secondary keys.
%   This defines \key\ to take a value, which should be either |true| or
%   |false|, and set the flag called \meta{cs} accordingly as a boolean.
%   If \meta{cs} isn't defined yet it will be initialised as a flag.
%   Please also read \autoref{sec:flags}.
% \end{function}
%
% \begin{function}{flag-true,flag-false}
%   \begin{syntax}
%     flag-true \key\ = \meta{cs} \prefixes3
%   \end{syntax}
%   This is a secondary key that doesn't directly involve any of the primary or
%   secondary keys.
%   This defines \key\ to take no value and set the flag called \meta{cs} to
%   |true| or |false|, respectively.
%   If \meta{cs} isn't defined yet it will be initialised as a flag.
%   Please also read \autoref{sec:flags}.
% \end{function}
%
% \begin{function}{flag-raise}
%   \begin{syntax}
%     flag-raise \key\ = \meta{cs} \prefixes3
%   \end{syntax}
%   This is a secondary key that doesn't directly involve any of the primary or
%   secondary keys.
%   This defines \key\ to take no value and raise the flag called \meta{cs}.
%   If \meta{cs} isn't defined yet it will be initialised as a flag.
%   Please also read \autoref{sec:flags}.
% \end{function}
%
%
% \subsection{Changing the Initial Values}\label{sec:change}
%
% \begin{function}{\ekvcChange}
%   \begin{syntax}
%     \cs{ekvcChange}\meta{cs}\{\kv,\ldots\}
%   \end{syntax}
%   This processes the \kv\ list (just like it would be processed by \meta{cs})
%   to set new defaults for the macro \meta{cs} defined with \expkvc\ (meaning
%   the values used if you don't provide anything at use time). Inside the \kv\
%   list both primary and secondary keys can be used. If \meta{cs} was defined
%   |\long| earlier it will still be |\long|, every other \TeX\ prefix will be
%   stripped (but \expkvc\ doesn't support them anywhere else so that should be
%   fine). The resulting new defaults will be stored inside the \meta{cs}
%   locally (just as the original defaults were).
% \end{function}
%
% Consider the following example:\\
% \begin{minipage}[c]{.75\linewidth}
% \begin{lstlisting}
% \ekvcSplit\foo{a=a,b=b}{a is #1.\par b is #2.\par}
% \foo{} % initial defaults
% \ekvcChange\foo{b=B}
% \foo{} % new defaults
% \ekvcSecondaryKeys\foo{meta c={a={#1},b={#1}}}
% \ekvcChange\foo{c=c}
% \foo{} % newer defaults
% \end{lstlisting}
% \end{minipage}^^A
% \begin{exresult}[nobeforeafter,box align=center]{.25\linewidth}
%   \ekvcSplit\foo{a=a,b=b}{a is #1.\par b is #2.\par}
%   \foo{}
%   \ekvcChange\foo{b=B}
%   \foo{}
%   \ekvcSecondaryKeys\foo{meta c={a={#1},b={#1}}}
%   \ekvcChange\foo{c=c}
%   \foo{}
% \end{exresult}
%
% As a result with this the typical setup macro could be implemented:
% \begin{lstlisting}
% \ekvcHashAndUse\foo{key=a,key=b}
% \newcommand\foosetup{\ekvcChange\foo}
% \end{lstlisting}
% Of course the usage is limited to a single macro |\foo|, hence this might be
% not as powerful as similar macros used with other \kv\ interfaces. But at
% least a few similar macros could be grouped using the same key parsing macro
% internally.
%
%
% \subsection{Flags}\label{sec:flags}
%
% The idea of flags is taken from \pkg{expl3}. They provide a way to store
% numerical information expandably, however only incrementing and accessing
% works expandably, decrementing is unexpandable. A flag has a height, which
% is a numerical value, and which can be raised by $1$. Flags come at a high
% computational cost (accessing them is slow and they require more memory than
% normal \TeX\ data types like registers, both getting linearly worse with the
% height), so don't use them if not necessary.
%
% The state of flags is always changed locally to the current group, but not to
% the current macro, so if you're using one of the |t|-types involving flags
% bear in mind that they can affect other macros using the same flags at the
% current group level!
%
% \expkvc\ provides some macros to access, alter, and use flags. Flags of
% \expkvc\ don't share a name space with the flags of \pkg{expl3}.
%
% \begin{function}{\ekvcFlagNew}
%   \begin{syntax}
%     \cs{ekvcFlagNew}\meta{flag}
%   \end{syntax}
%   This initialises the macro \meta{flag} as a new flag. It isn't checked
%   whether the macro \meta{flag} is currently undefined.  A \meta{flag} will
%   expand to the flag's current height with a trailing space (so you can use it
%   directly with |\ifnum| for example and it will terminate the number scanning
%   itself).
% \end{function}
%
%  All other macros dealing with flags take as a parameter a macro defined as a
%  \meta{flag} with |\ekvcFlagNew|.
%
% \begin{function}{\ekvcFlagHeight}
%   \begin{syntax}
%     \cs{ekvcFlagHeight}\meta{flag}
%   \end{syntax}
%   This expands to the current height of \meta{flag} in a single step of
%   expansion (without a trailing space).
% \end{function}
%
% \begin{function}{\ekvcFlagRaise}
%   \begin{syntax}
%     \cs{ekvcFlagRaise}\meta{flag}
%   \end{syntax}
%   This expandably raises the height of \meta{flag} by $1$.
% \end{function}
%
% \begin{function}{\ekvcFlagSetTrue,\ekvcFlagSetFalse}
%   \begin{syntax}
%     \cs{ekvcFlagSetTrue}\meta{flag}
%   \end{syntax}
%   By interpreting an even value as false and an odd value as true we can use a
%   flag as a boolean. This expandably sets \meta{flag} to true or false,
%   respectively, by raising it if necessary.
% \end{function}
%
% \begin{function}{\ekvcFlagIf}
%   \begin{syntax}
%     \cs{ekvcFlagIf}\meta{flag}\marg{true}\marg{false}
%   \end{syntax}
%   This interprets a \meta{flag} as a boolean and expands to either \meta{true}
%   or \meta{false}.
% \end{function}
%
% \begin{function}{\ekvcFlagIfRaised}
%   \begin{syntax}
%     \cs{ekvcFlagIfRaised}\meta{flag}\marg{true}\marg{false}
%   \end{syntax}
%   This tests whether the \meta{flag} is raised, meaning it has a height
%   greater than zero, and if so expands to \meta{true} else to \meta{false}.
% \end{function}
%
% \begin{function}{\ekvcFlagReset}
%   \begin{syntax}
%     \cs{ekvcFlagReset}\meta{flag}
%   \end{syntax}
%   This resets a flag (so restores its height to $0$). This operation is
%   \emph{not} expandable and done locally. If you really intend to use flags
%   you can reset them every now and then to keep the performance hit low.
% \end{function}
%
% \begin{function}{\ekvcFlagGetHeight}
%   \begin{syntax}
%     \cs{ekvcFlagGetHeight}\meta{flag}\marg{next}
%   \end{syntax}
%   This retrieves the current height of the \meta{flag} and provides it as a
%   braced argument to \meta{next}, leaving \meta{next}\marg{height} in the
%   input stream.
% \end{function}
%
% \begin{function}{\ekvcFlagGetHeights}
%   \begin{syntax}
%     \cs{ekvcFlagGetHeights}\marg{flag-list}\marg{next}
%   \end{syntax}
%   This retrieves the current height of each \meta{flag} in the
%   \meta{flag-list} and provides them as a single argument to \meta{next}.
%   Inside that argument each height is enclosed in a set of braces
%   individually. The \meta{flag-list} is just a single argument containing the
%   \meta{flag}s.
%   So a usage like
%   \texttt
%     {^^A
%       \cs[no-index]{ekvcFlagGetHeights}\hskip0pt^^A
%       \{\cs[no-index]{myflagA}\cs[no-index]{myflagB}\}\hskip0pt^^A
%       \{\cs[no-index]{stuff}\}^^A
%     }
%   will expand to
%   \texttt{\cs[no-index]{stuff}\{\marg{height-A}\marg{height-B}\}}.
% \end{function}
%
%
% \subsection{Further Examples}
%
% How could a documentation be a good documentation without some basic examples?
% Say we want to define a small macro expanding to some character description
% (who knows why this has to be expandable?). A character description will not
% have too many items to it, so we use |\ekvcSplit| (the comments with the
% parameter numbers are of course not necessary and just ease reading the
% example).
% \begin{lstlisting}
% \ekvcSplit\character
%   {
%     name=John Doe,            % #1
%     age=any,                  % #2
%     nationality=the Universe, % #3
%     hobby=to exist,           % #4
%     type=Mister,              % #5
%     pronoun=He,               % #6
%     possessive=his,           % #7
%   }
%   {#1 is a #5 from #3. #6 is of #2 age and #7 hobby is #4.\par}
% \end{lstlisting}
% Also we want to give some short cuts so that it's easier to describe several
% persons.
% \begin{lstlisting}
% \ekvcSecondaryKeys\character
%   {
%     alias pro = pronoun,
%     alias pos = possessive,
%     nmeta me =
%       {
%         name=Jonathan,
%         age=a young,
%         nationality=Germany,
%         hobby=\TeX\ coding,
%       },
%     meta lady =
%       {type=Lady, pronoun=She, possessive=her, name=Jane Doe, #1},
%     nmeta paulo =
%       {
%         name=Paulo,
%         type=duck,
%         age=a young,
%         nationality=Brazil,
%         hobby=to quack,
%       }
%   }
% \end{lstlisting}
% Now we can describe people using
% \begin{lstlisting}
% \character{}
% \character{me}
% \character{paulo}
% \character
%   {lady={name=Evelyn,nationality=Ireland,age=the best,hobby=reading}}
% \character
%   {
%     name=Our sun, type=star, nationality=our solar system, pro=It,
%     age=an old, pos=its, hobby=shining
%   }
% \end{lstlisting}
% As one might see, the |lady| key could actually have been an |nmeta| key as
% well, as all that is done with the argument is using it as a \kv\ list.
%
% The result of only the first two usages would be:
% \begin{exresult}{\linewidth}
%   \ekvcSplit\character
%     {
%       name=John Doe,
%       age=any,
%       nationality=the Universe,
%       hobby=to exist,
%       type=Mister,
%       pronoun=He,
%       possessive=his,
%     }
%     {#1 is a #5 from #3. #6 is of #2 age and #7 hobby is #4.\par}
%   \ekvcSecondaryKeys\character
%     {
%       alias pro = pronoun,
%       alias pos = possessive,
%       nmeta me =
%         {
%           name=Jonathan,
%           age=a young,
%           nationality=Germany,
%           hobby=\TeX\ coding,
%         },
%       meta lady =
%         {type=Lady, pronoun=She, possessive=her, name=Jane Doe, #1},
%       nmeta paulo =
%         {
%           name=Paulo,
%           type=duck,
%           age=a young,
%           nationality=Brazil,
%           hobby=to quack,
%         }
%     }
%   \character{}
%   \character{me}
% \end{exresult}
%
% Using \pkg{xparse} or \expkv's \cs{ekvoptarg} or \cs{ekvoptargTF} and
% forwarding arguments one can easily define \kv\ macros with actual optional
% and mandatory arguments as well. A small nonsense example (which should
% perhaps use |\ekvcSplitAndForward| instead of |\ekvcHashAndForward| since it
% only uses four keys and one other argument -- and isn't expandable since it
% uses a \env{tabular} environment, so it would've been better to use a more
% feature rich \kv\ interface most likely, \emph{e.g.}, the one provided by
% \expkvd):
% \begin{lstlisting}
% \makeatletter
% \newcommand*\nonsense{\ekvoptarg\nonsense@a{}}
% \ekvcHashAndForward\nonsense@a\nonsense@b
%   {
%     keyA = A,
%     keyB = B,
%     keyC = c,
%     keyD = d,
%   }
% \newcommand*\nonsense@b[2]
%   {%
%     \begin{tabular}{lll}
%       key & A & \ekvcValue{keyA}{#1} \\
%           & B & \ekvcValue{keyB}{#1} \\
%           & C & \ekvcValue{keyC}{#1} \\
%           & D & \ekvcValue{keyD}{#1} \\
%       \multicolumn{2}{l}{mandatory} & #2 \\
%     \end{tabular}%
%   }
% \makeatother
% \end{lstlisting}
% And then we would be able to do some nonsense
% \begin{lstlisting}
% \nonsense{}
% \nonsense[keyA=hihi]{haha}
% \nonsense[keyA=hihi, keyB=A]{hehe}
% \nonsense[keyC=huhu, keyA=hihi, keyB=A]{haha}
% \end{lstlisting}
% resulting in
% \begin{exresult}{\linewidth}
%   \makeatletter
%   \newcommand*\nonsense{\ekvoptarg\nonsense@a{}}
%   \ekvcHashAndForward\nonsense@a\nonsense@b
%     {
%       keyA = A,
%       keyB = B,
%       keyC = c,
%       keyD = d,
%     }
%   \newcommand*\nonsense@b[2]
%     {%
%       \begin{tabular}{lll}
%         key & A & \ekvcValue{keyA}{#1} \\
%             & B & \ekvcValue{keyB}{#1} \\
%             & C & \ekvcValue{keyC}{#1} \\
%             & D & \ekvcValue{keyD}{#1} \\
%         \multicolumn{2}{l}{mandatory} & #2 \\
%       \end{tabular}%
%     }
%   \makeatother
%   \nonsense{}
%   \nonsense[keyA=hihi]{haha}
%   \nonsense[keyA=hihi, keyB=A]{hehe}
%   \nonsense[keyC=huhu, keyA=hihi, keyB=A]{haha}
% \end{exresult}
%
%
% \subsection{Freedom for Keys!}
%
% If this was the \TeX book this subsection would have a double bend sign. Not
% because it is overly complicated, but because it shows things which could
% break \expkvc's expandability and its alignment safety. This is for
% experienced users wanting to get the most flexibility and knowing what they
% are doing.
%
% In case you're wondering, it is possible to define other keys than the
% primaries and the secondary types listed in \autoref{sec:secondaries} for a
% macro defined with \expkvc\ by using the low-level interface of \expkv\ or
% even the interface provided by \expkvd. The set name used for \expkvc's keys
% is the macro name, including the leading backslash, or more precisely
% |\string|\meta{cs} is used. This can be exploited to define additional keys
% with arbitrary code. Consider the following \emph{bad} example:
% \begin{lstlisting}
% \ekvcSplit\foo{a=A,b=B}{a is #1.\par b is #2}
% \protected\ekvdef{\string\foo}{c}{\def\fooC{#1}}
% \end{lstlisting}
% This would define a key named |c| that will store its value inside a macro.
% The issue with this is that this can't be done expandably. As a result, the
% macro |\foo| isn't always expandable any more (not that bad if this was never
% required; killjoy if it was) and as soon as the key |c| is used, it is also no
% longer alignment safe\footnote{This means that the \kv-list can't contain
%   alignment markers that are not inside an additional set of braces if used
%   inside a \TeX\ alignment.}
% (might be bad depending on the usage).
%
% So why do I show you this? Because we could as well do something useful like
% create a key that pre-parses the input and after that passes the parsed value
% on. This parsing would have to be completely expandable though. For the
% pass-on part we can use the following function:
%
% \begin{function}{\ekvcPass}
%   \begin{syntax}
%     \cs{ekvcPass}\meta{cs}\marg{key}\marg{value}
%   \end{syntax}
%   This passes \meta{value} on to \meta{key} for the \expkvc-macro \meta{cs}.
%   It should be used inside the key parsing of a macro defined with \expkvc,
%   else this most likely results in a low level \TeX\ error.
% \end{function}
%
% With this we could for example split the value of a key at a hyphen and pass
% the parts to different keys:\\
% \begin{minipage}[c]{.75\linewidth}
% \begin{lstlisting}
% \ekvcSplit\foo{a=A,b=B}{a is #1.\par b is #2.\par}
% \ekvdef{\string\foo}{c}{\fooSplit#1\par}
% \def\fooSplit#1-#2\par
%   {\ekvcPass\foo{a}{#1}\ekvcPass\foo{b}{#2}}
% \foo{}
% \foo{c=1-2}
% \end{lstlisting}
% \end{minipage}
% \begin{exresult}[nobeforeafter,box align=center]{.25\linewidth}
%   \ekvcSplit\foo{a=A,b=B}{a is #1.\par b is #2.\par}
%   \ekvdef{\string\foo}{c}{\fooSplit#1\par}
%   \def\fooSplit#1-#2\par{\ekvcPass\foo{a}{#1}\ekvcPass\foo{b}{#2}}
%   \foo{}
%   \foo{c=1-2}
% \end{exresult}
%
%
% \subsection{Speed Considerations}\label{sec:speed}
%
% As already mentioned in the introduction there are some speed considerations
% implied if you choose to define macros via \expkvc. However the overhead isn't
% the factor which should hinder you to use \expkvc\ if you found a reasonable
% use case. The key-parsing is still faster than with most other \kv\ packages
% (see the ``Comparisons'' subsection in the \expkv\ documentation).
%
% The speed considerations in this subsection use the first example in this
% documentation as the benchmark. So we have seven keys and a short sentence
% which should be typeset. For comparisons I use the following equivalent
% \expkvd\ definitions. Each result is the average between changing no keys from
% their initial values and altering four. Furthermore I'll compare three
% variants of \expkvc\ with the \expkvd\ definitions, namely the split example
% from above, a hash variant using |\ekvcValue| and a hash variant using
% |\ekvcValueFast|.
% \begin{lstlisting}
% \usepackage{expkv-def}
% \ekvdefinekeys{keys}
%   {%
%     ,store   name         = \KEYSname
%     ,initial name         = John Doe
%     ,store   age          = \KEYSage
%     ,initial age          = any
%     ,store   nationality  = \KEYSnationality
%     ,initial nationality  = the Universe
%     ,store   hobby        = \KEYShobby
%     ,initial hobby        = to exist
%     ,store   type         = \KEYStype
%     ,initial type         = Mister
%     ,store   pronoun      = \KEYSpronoun
%     ,initial pronoun      = He
%     ,store   possessive   = \KEYSpossessive
%     ,initial possessive   = his
%   }
% \newcommand*\KEYS[1]
%   {%
%     \begingroup
%       \ekvset{keys}{#1}%
%       \KEYSname\ is a \KEYStype\ from \KEYSnationality. \KEYSpronoun\ is
%       of \KEYSage\ age and \KEYSpossessive\ hobby is \KEYShobby.%
%     \endgroup
%   }
% \end{lstlisting}
%
% The first comparison removes the typesetting part from all the definitions, so
% that only the key parsing is compared. In this comparison the |\ekvcValue|
% and |\ekvcValueFast| variants will not differ, as they are exactly the same
% until the key usage. We find that the split approach is $1.4$ times slower
% than the \expkvd\ setup and the hash variants end up in the middle at $1.17$
% times slower.
%
% Next we put the typesetting part back in. Every call of the macros will
% typeset the sentences into a box register in horizontal mode. With the
% typesetting part (which includes the accessing of values) the fastest remains
% the \expkvd\ definitions, but split is close at $1.16$ times slower, followed
% by the hash variant with fast accesses at $1.36$ times slower, and the safe
% hash access variant ranks in the slowest $1.8$ times slower than \expkvd.
%
% Just in case you're wondering now, a simple macro taking seven arguments is
% $30$ to $40$ times faster than any of those in the argument grabbing and \kv\
% parsing part and only $1.5$ to $2.8$ times faster if the typesetting part is
% factored in. So the real choke isn't the parsing.
%
% So to summarize this, if you have a reasonable use case for expandable \kv\
% parsing macros you should go on and define them using \expkvc. If you have a
% reasonable use case for \kv\ parsing macros but defining them expandable isn't
% necessary for your use you should take advantage of the greater flexibility of
% non-expandable \kv\ setups (but if you're after maximum speed there aren't
% that many \kv\ parsers beating \expkvc). And if you are after maximum
% performance maybe ditching the \kv\ interface altogether is a good idea, but
% depending on the number of arguments your interface might get convoluted.
%
%
% \subsection{Useless Macros}
%
% Perhaps these macros aren't completely useless, but I figured from a user's
% point of view I wouldn't know what I should do with these.
%
% \begin{function}{\ekvcDate,\ekvcVersion}
%   These two macros store the version and the date of the package/generic code.
% \end{function}
%
%
% \subsection{Bugs}
%
% Of course I don't think there are any bugs (who would knowingly distribute
% buggy software as long as he isn't a multi-million dollar corporation?). But
% if you find some please let me know. For this one might find my email address
% on the first page or file an issue on Github:
% \url{https://github.com/Skillmon/tex_expkv-cs}
%
%
% \subsection{License}
%
% Copyright \textcopyright\ 2020\unless\ifnum\year=2020--\the\year\fi\
% Jonathan P. Spratte
%
% \medskip\noindent
% This work may be distributed and/or modified under the conditions of the
% \LaTeX\ Project Public License (LPPL), either version 1.3c of this license or
% (at your option) any later version. The latest version of this license is in
% the file:
%
% \url{http://www.latex-project.org/lppl.txt}
%
% \noindent
% This work is ``maintained'' (as per LPPL maintenance status) by
%
% \mbox{Jonathan P. Spratte}.
%
% \end{documentation}^^A=<<
%
% \begin{implementation}^^A>>=
%
% \clearpage
%
% \section{Implementation}^^A>>=
%^^A the LaTeX package >>=
% \subsection{The \LaTeX\ Package}
% Just like for \expkv\ we provide a small \LaTeX\ package that sets up things
% such that we behave nicely on \LaTeX\ packages and files system. It'll
% |\input| the generic code which implements the functionality.
% \gobbledocstriptag
%<*pkg>
%    \begin{macrocode}
\RequirePackage{expkv}
\def\ekvc@tmp
  {%
    \ProvidesFile{expkv-cs.tex}%
      [%
        \ekvcDate\space v\ekvcVersion\space
        define expandable key=val macros using expkv%
      ]%
  }
% \iffalse meta-comment
%
% File: expkv-cs.dtx Copyright (C) 2020-2021 Jonathan P. Spratte
%
% This work  may be  distributed and/or  modified under  the conditions  of the
% LaTeX Project Public License (LPPL),  either version 1.3c  of this license or
% (at your option) any later version.  The latest version of this license is in
% the file:
%
%   http://www.latex-project.org/lppl.txt
%
% ------------------------------------------------------------------------------
%
%<*driver>^^A>>=
\def\nameofplainTeX{plain}
\ifx\fmtname\nameofplainTeX\else
  \expandafter\begingroup
\fi
\input l3docstrip.tex
\askforoverwritefalse
\preamble

--------------------------------------------------------------
expkv-cs -- define expandable key=val macros using expkv
E-mail: jspratte@yahoo.de
Released under the LaTeX Project Public License v1.3c or later
See http://www.latex-project.org/lppl.txt
--------------------------------------------------------------

Copyright (C) 2020-2021 Jonathan P. Spratte

This  work may be  distributed and/or  modified under  the conditions  of the
LaTeX Project Public License (LPPL),  either version 1.3c  of this license or
(at your option) any later version.  The latest version of this license is in
the file:

  http://www.latex-project.org/lppl.txt

This work is "maintained" (as per LPPL maintenance status) by
  Jonathan P. Spratte.

This work consists of the file  expkv-cs.dtx
and the derived files           expkv-cs.pdf
                                expkv-cs.sty
                                expkv-cs.tex

\endpreamble
% stop docstrip adding \endinput
\postamble
\endpostamble
\generate{\file{expkv-cs.sty}{\from{expkv-cs.dtx}{pkg}}}
\generate{\file{expkv-cs.tex}{\from{expkv-cs.dtx}{tex}}}
\ifx\fmtname\nameofplainTeX
  \expandafter\endbatchfile
\else
  \expandafter\endgroup
\fi
%
\IfFileExists{expkv-cs.tex}{\input{expkv-cs.tex}}{}
\ProvidesFile{expkv-cs.dtx}
  [\csname ekvcDate\endcsname\ define expandable key=val macros using expkv]
\PassOptionsToPackage{full}{textcomp}
\documentclass{l3doc}
\RequirePackage[oldstylenums,nott]{kpfonts}
\input{glyphtounicode}
\pdfgentounicode=1
\RequirePackage{tcolorbox}
\newtcolorbox{exresult}[2][]
  {colback=expkvgrey!10!white,colframe=expkvgrey,fontupper=\small,width={#2},#1}
\newtcbox\exres[1][]
  {
     colback=expkvgrey!10!white
    ,colframe=expkvgrey
    ,size=small
    ,nobeforeafter
    ,tcbox raise base
    ,fontupper=\small
    ,#1
  }
\RequirePackage{listings}
\RequirePackage{booktabs}
\RequirePackage{array}
\RequirePackage{collcell}
\RequirePackage{siunitx}
\RequirePackage{xcolor}
\RequirePackage{caption}
\RequirePackage{microtype}
\RequirePackage{accsupp}
\lstdefinelanguage{expkvc}[primitive]{TeX}
  {
    ,moretexcs=[4]^^A e-TeX
      {
        expanded,
        numexpr,
        protected,
      }
    ,moretexcs=[5]^^A plain/LaTeX
      {
        approx,
        begin,
        empty,
        item,
        LaTeX,
        makeatletter,makeatother,multicolumn,
        newcommand,newdimen,
        RequirePackage,
        rule,
        TeX,
        textit,texttt,
        usepackage,
      }
    ,moretexcs=[6]^^A used packages
      {
        ^^A xfp
        fpeval,
        ^^A keyval
        setkeys,
        ^^A yax
        defactiveparameter,storevalue,setparameterlist
      }
    ,moretexcs=[2]^^A expkv-cs macros
      {
        ekvcSplit,ekvcSplitAndForward,ekvcSplitAndUse,
        ekvcHash,ekvcHashAndForward,ekvcHashAndUse,
        ekvcValue,ekvcValueFast,ekvcValueSplit,ekvcValueSplitFast,
        ekvcSecondaryKeys,
        ekvcChange,
        ekvcFlagNew,
        ekvcFlagHeight,
        ekvcFlagRaise,
        ekvcFlagSetTrue,ekvcFlagSetFalse,
        ekvcFlagIf,ekvcFlagIfRaised,
        ekvcFlagReset,
        ekvcFlagGetHeight,ekvcFlagGetHeights,
        ekvcPass
      }
    ,moretexcs=[3]^^A expkv-pkg macros
      {
        ^^A expkv
        ekvdef,ekvdefNoVal,^^A
        ekvlet,ekvletNoVal,ekvletkv,ekvletkvNoVal,^^A
        ekvdefunknown,ekvdefunknownNoVal,^^A
        ekvredirectunknown,ekvredirectunknownNoVal,^^A
        ekvset,ekvsetSneaked,^^A
        ekvsetdef,ekvsetSneakeddef,ekvsetdefSneaked,^^A
        ekvparse,^^A
        ekvVersion,ekvDate,^^A
        ekvifdefined,ekvifdefinedNoVal,ekvifdefinedset,^^A
        ekvbreak,ekvbreakPreSneak,ekvbreakPostSneak,^^A
        ekvsneak,ekvsneakPre,^^A
        ekvchangeset,^^A
        ekvoptarg,ekvoptargTF,^^A
        ekverr,
        ^^A expkv-def
        ekvdefinekeys
      }
  }
\colorlet{codeparam}{cyan!65!black}
\def\mylstwd{.55em}
\lstset
  {
    ,language=expkvc
    ,flexiblecolumns=false
    ,basewidth=\mylstwd
    ,gobble=2
    ,basicstyle=\fontfamily{jkp}\itshape
    ,texcsstyle=*[1]{\bfseries\color{expkvgrey}} ^^A primitives
    ,texcsstyle=*[2]{\bfseries\color{expkvred}}  ^^A expkv
    ,texcsstyle=*[3]{\color{expkvred}}           ^^A expkv-pkg
    ,texcsstyle=*[4]{\bfseries\color{expkvgrey}} ^^A e-TeX
    ,texcsstyle=*[5]{\bfseries\color{expkvgrey}} ^^A plain/LaTeX macros
    ,texcsstyle=*[6]{}                           ^^A macros of other packages
    ,commentstyle=\color[gray]{0.4}
    ,literate=
              {\{} {{\CodeUpBf\{}}{1}
              {\}} {{\CodeUpBf\}}}{1}
              {$}  {{\CodeUpBf\$}}{1}
              {[}  {{\CodeUp[}}{1}
              {]}  {{\CodeUp]}}{1}
              {(}  {{\CodeUp(}}{1}
              {)}  {{\CodeUp)}}{1}
              {*}  {{$*$}}{1}
              {1}  {{\CodeUp{1}}}{1}
              {2}  {{\CodeUp{2}}}{1}
              {3}  {{\CodeUp{3}}}{1}
              {4}  {{\CodeUp{4}}}{1}
              {5}  {{\CodeUp{5}}}{1}
              {6}  {{\CodeUp{6}}}{1}
              {7}  {{\CodeUp{7}}}{1}
              {8}  {{\CodeUp{8}}}{1}
              {9}  {{\CodeUp{9}}}{1}
              {0}  {{\CodeUp{0}}}{1}
              {##} {{\CodeColored{codeparam}{1}{\#}}}{1}
              {##1}{{\CodeColored{codeparam}{2}{\#1}}}{2}
              {##2}{{\CodeColored{codeparam}{2}{\#2}}}{2}
              {##3}{{\CodeColored{codeparam}{2}{\#3}}}{2}
              {##4}{{\CodeColored{codeparam}{2}{\#4}}}{2}
              {##5}{{\CodeColored{codeparam}{2}{\#5}}}{2}
              {##6}{{\CodeColored{codeparam}{2}{\#6}}}{2}
              {##7}{{\CodeColored{codeparam}{2}{\#7}}}{2}
              {##8}{{\CodeColored{codeparam}{2}{\#8}}}{2}
              {##9}{{\CodeColored{codeparam}{2}{\#9}}}{2}
              {<key>}{{$\langle$}key{$\rangle$}}{5}
              {<set>}{{$\langle$}set{$\rangle$}}{5}
              {<set1>}{{$\langle$}set1{$\rangle$}}{6}
              {<set2>}{{$\langle$}set2{$\rangle$}}{6}
    ^^A,literate=*{<key>}{\key}{4}{<set>}{\set}{4}
  }
\newcommand*\CodeColored[3]{\textcolor{#1}{\makebox[\dimexpr\mylstwd*#2]{#3}}}
\newcommand*\CodeUpBf[1]{\makebox[\mylstwd]{\textup{\textbf{#1}}}}
\newcommand*\CodeUp[1]{\makebox[\mylstwd]{\textup{#1}}}
\RequirePackage{randtext}
\let\metaORIG\meta
\protected\def\meta #1{\texttt{\metaORIG{#1}}}
\renewcommand*\thefootnote{\fnsymbol{footnote}}
\definecolor{expkvred}{HTML}{9F393D}
\colorlet{expkvgrey}{black!75}
\makeatletter
\newcommand*\example{\par\smallskip\noindent\textit{Example:} \ignorespaces}
\newcommand*\expkv
  {^^A
    \texorpdfstring
      {^^A
        \mbox
          {^^A
            \BeginAccSupp{ActualText=expkv}^^A
            \href{https://github.com/Skillmon/tex_expkv}
              {^^A
                \rmfamily
                \bfseries
                {\color{expkvgrey}e\kern-.05em x\kern-.05em}^^A
                \lower.493ex
                  \hbox{{\color{expkvgrey}P}\kern-.1em{\color{expkvred}k}}^^A
                \kern-.18em{\color{expkvred}v}^^A
              }^^A
            \EndAccSupp{}^^A
          }^^A
      }
      {expkv}^^A
  }
\newcommand*\expkvpkg[1]
  {^^A
    \texorpdfstring
      {^^A
        \mbox
          {^^A
            \BeginAccSupp{ActualText=expkv-#1}^^A
            \href{https://github.com/Skillmon/tex_expkv-#1}
              {^^A
                \rmfamily
                \bfseries
                {\color{expkvgrey}e\kern-.05em x\kern-.05em}^^A
                \lower.493ex
                  \hbox{{\color{expkvgrey}P}\kern-.1em{\color{expkvred}k}}^^A
                \kern-.18em{\color{expkvred}v}^^A
                {\color{expkvgrey}^^A
                  \kern.05em\rule[-.1ex]{.08em}{1.2ex}\kern.05em\textsc{#1}^^A
                }^^A
              }^^A
            \EndAccSupp{}^^A
          }^^A
      }
      {expkv-#1}^^A
  }
\newcommand*\expkvd{\expkvpkg{def}}
\newcommand*\expkvc{\expkvpkg{cs}}
\newcommand\kv{\meta{key}=\meta{value}}
\newcommand\key{\meta{key}}
\newcommand\val{\meta{value}}
\newcommand\set{\meta{set}}
\newcommand\enflong{\textcolor{black}{long}}
\newcommand\alllong{\textcolor{gray}{long}}
\newcommand\notlong{\textcolor{red!80!black}{long}}
\newcommand\prefixes[1]
  {^^A
    \hfill
    \ifcase\numexpr#1\relax\or
      \enflong\or
      \alllong\or
      \notlong\fi
  }
\hypersetup{linkcolor=red!80!black,urlcolor=purple!80!black}
\DoNotIndex{\def,\edef,\,,\=,\begingroup,\catcode,\chardef,\csname,\endcsname}
\DoNotIndex{\endgroup,\endinput,\errmessage,\expandafter,\input,\let,\long,\z@}
\DoNotIndex{\protected,\ProvidesFile,\ProvidesPackage,\relax,\space,\advance}
\DoNotIndex{\@,\unexpanded,\string,\expanded,\dimexpr,\global,\glueexpr,\hbox}
\DoNotIndex{\numexpr,\RequirePackage,\setbox,\the,\unless,\xdef,\gdef,\newcount}
\DoNotIndex{\number}
\DoNotIndex{\^,\@firstofone,\@firstoftwo,\@gobble,\@secondoftwo}
\DoNotIndex{\ifcsname}
\DoNotIndex{\ifx}
\DoNotIndex{\ifodd}
\DoNotIndex{\ifnum}
\DoNotIndex{\ifdefined}
\DoNotIndex{\iffalse}
\DoNotIndex{\iftrue}
\DoNotIndex{\else}
\DoNotIndex{\fi}
\@gobble\fi ^^A ignoring \ifx, \ifcsname, etc., but only one \fi
\@gobble\fi ^^A ignoring \ifx, \ifcsname, etc., but only one \fi
\@gobble\fi ^^A ignoring \ifx, \ifcsname, etc., but only one \fi
\@gobble\fi ^^A ignoring \ifx, \ifcsname, etc., but only one \fi
\@gobble\fi ^^A ignoring \ifx, \ifcsname, etc., but only one \fi
\@gobble\fi ^^A ignoring \ifx, \ifcsname, etc., but only one \fi
\@ifdefinable\gobbledocstriptag{\def\gobbledocstriptag#1>{}}
\makeatother
\begin{document}
  \title
    {^^A
      \texorpdfstring
        {^^A
          \huge\expkvc
          \\[\medskipamount]
          \Large define expandable \kv\ macros using \expkv
        }{expkv-cs - define expandable key=val macros using expkv}^^A
    }
  \date{\ekvcDate\space v\ekvcVersion}
  \author{Jonathan P. Spratte\thanks{\protect\randomize{jspratte@yahoo.de}}}
  \DocInput{expkv-cs.dtx}
\end{document}
%</driver>^^A=<<
% \fi
%
% \maketitle
% \renewcommand*\thefootnote{\arabic{footnote}}
%
% \begin{abstract}
% \noindent\parfillskip=0pt
% \expkvc\ provides two small interfaces to define expandable \kv\ macros using
% \expkv. It therefore lowers the entrance boundary to expandable \kv\ macros.
% The stylised name is \expkvc\ but the files use \file{expkv-cs}, this is due
% to CTAN-rules which don't allow \string| in package names since that is the
% pipe symbol in *nix shells.
% \end{abstract}
%
% \tableofcontents
%
% \begin{documentation}^^A>>=
%
% \section{Documentation}
%
% The \expkv\ package enables the new possibility of creating \kv\ macros which
% are fully expandable. The creation of such macros is however cumbersome for
% the average user. \expkvc\ tries to step in here. It provides interfaces to
% define \kv\ macros without worrying too much about the implementation. In case
% you're wondering now, the |cs| in \expkvc\ stands for control sequence,
% because |def| was already taken by \expkvd\ and ``control sequence'' is the
% term D.\,E.\,Knuth used in his \TeX book for named commands hence macros
% (though he also used the term ``macro''). So \expkvc\ defines control
% sequences for and with \expkv.
%
% There are two different approaches supported by this package. The first is
% splitting the keys up into individual arguments, the second is providing all
% the keys as a single argument to the underlying macro and getting an
% individual \val\ by using a hash. Well, actually there is no real hash, just
% some markers which are parsed, but this shouldn't be apparent to the user, the
% behaviour matches that of a hash-table.
%
% In addition to these two methods of defining a macro with primary keys a way
% to define secondary keys, which can reference the primary ones, is provided.
% These secondary keys don't correspond to an argument or an entry in the hash
% table directly but might come in handy for the average use case. Each macro
% has its own set of primary and secondary keys.
%
% A word of advice you should consider: If your macro doesn't have to be
% expandable (and often it doesn't) consider not using \expkvc. The interface
% has some overhead (though it still is fast -- check \autoref{sec:speed}) and
% the approach has its limits in versatility. If you don't need to be
% expandable, you should consider either defining your keys manually using
% \expkv\ or using \expkvd\ for convenience. Or you resort to another \kv\
% interface. Nevertheless setting up macros with \expkvc, especially with
% |\ekvcSplit|, is very convenient in my opinion, so if you just want to define
% a single macro with just a few keys this might be the way to go.
%
% \expkvc\ is usable as generic code and as a \LaTeX\ package. It'll
% automatically load \expkv\ in the same mode as well. To use it, just use one
% of
% \begin{lstlisting}
% \usepackage{expkv-cs} % LaTeX
% \input expkv-cs       % plainTeX
% \end{lstlisting}
%
%
% \subsection{Define Macros and Primary Keys}
%
% All macros defined with \expkvc\ have to be previously undefined or have the
% |\meaning| of |\relax|. This is necessary as there is no way to automatically
% undefine keys once they are set up (neither \expkv\ nor \expkvc\ keep track of
% defined keys) -- so to make sure there are no conflicts only new definitions
% are allowed (that's not the case for individual keys, only for frontend
% macros).
%
%
% \subsubsection{Primary Keys}\label{sec:primaries}
%
% In the following descriptions there will be one argument named \meta{primary
% keys}. This argument should be a \kv\ list where each \key\ will be one
% primary key and \val\ the associated initial value. By default all keys are
% defined short, but you can define long keys by prefixing \key\ with |long|
% (\emph{e.g.}, \texttt{long name=Jonathan P. Spratte}). You only need |long| if
% the key should be able to take a |\par| token. Note however that |long| keys
% are a microscopic grain faster (due to some internals of \expkvc). Only if at
% least one of the keys was |long| the \meta{cs} in the following defining
% macros will be |\long|. For obvious reasons there is no possibility to define
% a macro or key as |\protected|.
%
% At the moment \expkvc\ doesn't require any internal keys, but I can't foresee
% whether this will be the case in the future as well, as it might turn out that
% some features I deem useful can't be implemented without such internal keys.
% Because of this, please don't use key names starting with
% \texttt{EKVC\string|} as that should be the private name space.
%
%
% \subsubsection{Split}
%
% The split variants will provide the key values as separate arguments. This
% limits the number of keys for which this is truly useful.
%
% \begin{function}{\ekvcSplit}
%   \begin{syntax}
%     \cs{ekvcSplit}\meta{cs}\marg{primary keys}\marg{definition}
%   \end{syntax}
%   This defines \meta{cs} to be a macro taking one mandatory argument which
%   should contain a \kv\ list. The \meta{primary keys} will be defined for this
%   macro (see \autoref{sec:primaries}). The \meta{definition} is the code that
%   will be executed. You can access the \val\ of a \key\ by using a macro
%   parameter from |#1| to |#9|. The order of the macro parameters will be the
%   order provided in the \meta{primary keys} list (so |#1| is the \val\ of the
%   key defined first). With |\ekvcSplit| you can define macros using at most
%   nine primary keys.
% \end{function}
% \example The following defines a macro |\foo| that takes the keys |a| and |b|
% and outputs their values in a textual form:\\
% \begin{minipage}{.75\linewidth}
% \begin{lstlisting}
% \ekvcSplit\foo{a=a,b=b}{a is #1.\par b is #2.\par}
% \foo{}
% \foo{b=e}
% \end{lstlisting}
% \end{minipage}^^A
% \begin{exresult}[nobeforeafter,box align=center]{.25\linewidth}
%   \ekvcSplit\foo{a=a,b=b}{a is #1.\par b is #2.\par}
%   \foo{}
%   \foo{b=e}
% \end{exresult}
%
% \begin{function}{\ekvcSplitAndForward}
%   \begin{syntax}
%     \cs{ekvcSplitAndForward}\meta{cs}\marg{after}\marg{primary keys}
%   \end{syntax}
%   This defines \meta{cs} to be a macro taking one mandatory argument which
%   should contain a \kv\ list. You can use as many primary keys as you want
%   with this. The primary keys will be forwarded to \meta{after} as braced
%   arguments (as many as necessary for your primary keys). The order of the
%   braced arguments will be the order of your primary key definitions. In
%   \meta{after} you can use just a single control sequence, or some arbitrary
%   stuff which will be left in the input stream before your braced values (with
%   one set of braces stripped from \meta{after}), so both of the following
%   would be fine:
% \begin{lstlisting}
% \ekvcSplitAndForward\foo\foo@aux{keyA = A, keyB = B}
% \ekvcSplitAndForward\foo{\foo@aux{more args}}{keyA = A, keyB = B}
% \end{lstlisting}
% \end{function}
% In the first case |\foo@aux| should take at least two arguments (|keyA| and
% |keyB|), in the second case at least three (|more args|, |keyA|, and |keyB|).
%
% \begin{function}{\ekvcSplitAndUse}
%   \begin{syntax}
%     \cs{ekvcSplitAndUse}\meta{cs}\marg{primary keys}
%   \end{syntax}
%   This will roughly do the same as |\ekvcSplitAndForward|, but instead of
%   specifying what will be used after splitting the keys, \meta{cs} will use
%   what follows the \kv\ list. So its syntax will be
% \end{function}
%   \begin{syntax}
%     \meta{cs}\{\key=\val, \ldots\}\marg{after}
%   \end{syntax}
% and the code in |after| should expect at least as many arguments as the number
% of keys defined for \meta{cs}.
%
% \subsubsection{Hash}
%
% The hash variants will provide the key values as a single argument in which
% you can access specific values using a special macro. The implementation might
% be more convenient and scale better, \emph{but} it is much slower (for a
% primitive macro with a single key benchmarking was almost $1.7$ times slower,
% the root of which being the key access with |\ekvcValue|, not the parsing, and
% for a key access using |\ekvcValueFast| it was still about $1.2$ times
% slower). So if your macro uses less than ten primary keys, you should consider
% using the split approach.
%
% \begin{function}{\ekvcHash}
%   \begin{syntax}
%     \cs{ekvcHash}\meta{cs}\marg{primary keys}\marg{definition}
%   \end{syntax}
%   This defines \meta{cs} to be a macro taking one mandatory argument which
%   should contain a \kv\ list. You can use as many primary keys as you want.
%   The primary keys will be forwarded as a single argument containing every key
%   to the underlying macro. The underlying macro is defined as
%   \meta{definition}, in which you can access the \val\ of a \key\ by using
%   \texttt{\cs[no-index]{ekvcValue}\{\key\}\{\#1\}}.
% \end{function}
% \example This defines an equivalent macro like the |\foo| defined with
% |\ekvcSplit| earlier:\\
% \begin{minipage}{.75\linewidth}
% \begin{lstlisting}
% \ekvcHash\foo{a=a,b=b}{a is \ekvcValue{a}{#1}.\par
%                        b is \ekvcValue{b}{#1}.\par}
% \foo{}
% \foo{b=e}
% \end{lstlisting}
% \end{minipage}^^A
% \begin{exresult}[nobeforeafter,box align=center]{.25\linewidth}
%   \ekvcHash\foo{a=a,b=b}{a is \ekvcValue{a}{#1}.\par
%                          b is \ekvcValue{b}{#1}.\par}
%   \foo{}
%   \foo{b=e}
% \end{exresult}
%
% \begin{function}{\ekvcHashAndForward}
%   \begin{syntax}
%     \cs{ekvcHashAndForward}\meta{cs}\marg{after}\marg{primary keys}
%   \end{syntax}
%   This defines \meta{cs} to be a macro taking one mandatory argument which
%   should contain a \kv\ list. You can use as many primary keys as you want.
%   The primary keys will be forwarded as a single argument containing every key
%   to \meta{after}. You can use a single macro for \meta{after} or use some
%   arbitrary stuff, which will be left in the input stream before the hashed
%   \kv\ list with one set of braces stripped. In the
%   macro called in \meta{after} you can access the \val\ of a \key\ by using
%   \texttt{\cs[no-index]{ekvcValue}\{\key\}\{\#1\}} (or whichever argument the
%   hashed \kv\ list will be).
% \end{function}
%
% \begin{function}{\ekvcHashAndUse}
%   \begin{syntax}
%     \cs{ekvcHashAndUse}\meta{cs}\marg{primary keys}
%   \end{syntax}
%   This will roughly do the same as |\ekvcHashAndForward|, but instead of
%   specifying what will be used after hashing the keys, \meta{cs} will use what
%   follows the \kv\ list. So its syntax will be
% \end{function}
%   \begin{syntax}
%     \meta{cs}\{\key=\val, \ldots\}\marg{after}
%   \end{syntax}
%
% \begin{function}{\ekvcValue}
%   \begin{syntax}
%     \cs{ekvcValue}\{\key\}\marg{key list}
%   \end{syntax}
%   This is a safe way to access your keys in a hash variant. \key\
%   is the key which's \val\ you want to use out of the \meta{key list}.
%   \meta{key list} should be the key list argument forwarded to your underlying
%   macro by |\ekvcHash|, |\ekvcHashAndForward|, or |\ekvcHashAndUse|. It will
%   be tested whether the hash function to access that \key\ exists, the \key\
%   argument is not empty, and that the \meta{key list} really contains a \val\
%   of that \key. This macro needs exactly two steps of expansion and if used
%   inside of an |\edef| or |\expanded| context will protect the \val\ from
%   further expanding.
% \end{function}
%
% \begin{function}{\ekvcValueFast}
%   \begin{syntax}
%     \cs{ekvcValueFast}\{\key\}\marg{key list}
%   \end{syntax}
%   This behaves just like |\ekvcValue|, but \emph{without any} safety tests. As
%   a result this is about $1.4$ times faster \emph{but} will throw low level
%   \TeX\ errors eventually if the hash function isn't defined or the \key\
%   isn't part of the \meta{key list} (\emph{e.g.}, because it was defined as a
%   key for another macro -- all macros share the same hash function per \key).
%   Use it if you know what you're doing. This macro needs exactly three steps
%   of expansion in the no-errors case.
% \end{function}
%
% \begin{function}{\ekvcValueSplit}
%   \begin{syntax}
%     \cs{ekvcValueSplit}\{\key\}\marg{key list}\marg{next}
%   \end{syntax}
%   If you need a specific \key\ from a \meta{key list} more than once, it'll
%   be a good idea to only extract it once and from then on keep it as a
%   separate argument. Hence the macro \cs{ekvcValueSplit} will extract one
%   specific \key's value from the list and forward the remainder of the list
%   as the first and the \key's value as the second argument to \meta{next}, so
%   the result of this will be \meta{next}\marg{key list'}\marg{value} with
%   \meta{key list'} the remaining list. This is almost as fast as |\ekvcValue|
%   and runs the same tests. Keep in mind that you can't fetch for the same
%   \key\ again from \meta{key list'} as it got removed.
% \end{function}
% \example The following defines a macro |\foo| which will take three keys.
% Since the next parsing step will need the value of one of the keys multiple
% times we split that key off the list:\\
% \begin{minipage}{.75\linewidth}
% \begin{lstlisting}
% \ekvcHash\foo{a=a,b=b,c=c}
%   {\ekvcValueSplit{a}{#1}\foobar}
% \newcommand*\foobar[2]{a is #2.\par
%                        b is \ekvcValue{b}{#1}.\par
%                        c is \ekvcValue{c}{#1}.\par}
% \foo{}
% \end{lstlisting}
% \end{minipage}^^A
% \begin{exresult}[nobeforeafter,box align=center]{.25\linewidth}
%   \ekvcHash\foo{a=a,b=b,c=c}
%     {\ekvcValueSplit{a}{#1}\foobar}
%   \newcommand*\foobar[2]{a is #2.\par
%                          b is \ekvcValue{b}{#1}.\par
%                          c is \ekvcValue{c}{#1}.\par}
%   \foo{}
% \end{exresult}
%
% \begin{function}{\ekvcValueSplitFast}
%   \begin{syntax}
%     \cs{ekvcValueSplitFast}\{\key\}\marg{key list}\marg{next}
%   \end{syntax}
%   This behaves just like |\ekvcValueSplit|, but it won't run the same tests,
%   hence it is faster but more error prone, just like the relation between
%   |\ekvcValue| and |\ekvcValueFast|.
% \end{function}
%
%
% \subsection{Secondary Keys}\label{sec:secondaries}
%
% To remove some of the limitations with the approach that each primary key
% matches an argument or hash entry, you can define secondary keys. Those have
% to be defined for each macro but it doesn't matter whether that macro was a
% split or a hash variant. If a secondary key references another key it doesn't
% matter whether that other key is primary or secondary.
%
% Secondary keys can have a prefix (like |long|) which are called |p|-type
% prefix and must have a type (like |meta|) which are called |t|-type prefix.
% Some types might require some |p|-prefixes, while others might forbid those.
%
% Please keep in mind that key names shouldn't start with \texttt{EKVC\string|}.
%
% \begin{function}{\ekvcSecondaryKeys}
%   \begin{syntax}
%     \cs{ekvcSecondaryKeys}\meta{cs}\{\kv, \ldots\}
%   \end{syntax}
%   This is the front facing macro to define secondary keys. For the macro
%   \meta{cs} define \key\ to have definition \val. The general syntax for \key\
%   should be
%   \begin{quote}\ttfamily\small
%     \meta{prefix} \meta{name}
%   \end{quote}
%   Where \meta{prefix} is a space separated list of optional |p|-type prefixes
%   followed by one |t|-type prefix. The syntax of \val\ is dependent on the
%   used |t|-prefix.
% \end{function}
%
%
% \subsubsection{\texttt{p}-type Prefixes}
%
% There is only one |p|-prefix available, which is |long|.
%
% \begin{function}{long}
%   The following key will be defined |\long|.
% \end{function}
%
%
% \subsubsection{\texttt{t}-type Prefixes}
%
% If you're familiar with \expkvd\ you'll notice that the |t|-type prefixes
% provided here are much fewer. The expansion only concept doesn't allow for
% great variety in the auto-defined keys.
%
% The syntax examples of the |t|-prefixes will show which |p|-prefix will be
% automatically used by printing those black (\texttt{\enflong}), which
% will be available in grey (\texttt{\alllong}), and which will be disallowed in
% red (\texttt{\notlong}). This will be put flush right next to the syntax line.
%
% \begin{function}{meta}
%   \begin{syntax}
%     meta \key\ = \{\kv, \ldots\} \prefixes2
%   \end{syntax}
%   With a |meta| key you can set other keys. Whenever \key\ is used the keys in
%   the \kv\ list will be set to the values given there. You can use the \val\
%   given to \key\ by using |#1| in the \kv\ list. The keys in the \kv\ list can
%   be primary and secondary ones.
% \end{function}
%
% \begin{function}{nmeta}
%   \begin{syntax}
%     nmeta \key\ = \{\kv, \ldots\} \prefixes3
%   \end{syntax}
%   An |nmeta| key is like a |meta| key, but it doesn't take a value, so the
%   \kv\ list is static.
% \end{function}
%
% \begin{function}{alias}
%   \begin{syntax}
%     alias \key\ = \meta{key_2} \prefixes3
%   \end{syntax}
%   This assigns the definition of \meta{key_2} to \key. As a result \key\ is an
%   alias for \meta{key_2} behaving just the same. Both the value taking and the
%   |NoVal| version (that's \expkv\ slang for a key not accepting a value) will
%   be copied if they are defined when |alias| is used. Of course, \meta{key_2}
%   has to be defined, be it as a primary or secondary one.
% \end{function}
%
% \begin{function}{default}
%   \begin{syntax}
%     default \key\ = \marg{default} \prefixes3
%   \end{syntax}
%   If \key\ is a defined value taking key, you can define a |NoVal| version
%   with this that will behave as if \key\ was given \meta{default} as its
%   \val. Note that this doesn't change the initial values of primary keys set
%   at definition time in |\ekvcSplit| and friends. \key\ can be a primary or
%   secondary key.
% \end{function}
%
% \begin{function}{flag-bool}
%   \begin{syntax}
%     flag-bool \key\ = \meta{cs} \prefixes3
%   \end{syntax}
%   This is a secondary key that doesn't directly involve any of the primary or
%   secondary keys.
%   This defines \key\ to take a value, which should be either |true| or
%   |false|, and set the flag called \meta{cs} accordingly as a boolean.
%   If \meta{cs} isn't defined yet it will be initialised as a flag.
%   Please also read \autoref{sec:flags}.
% \end{function}
%
% \begin{function}{flag-true,flag-false}
%   \begin{syntax}
%     flag-true \key\ = \meta{cs} \prefixes3
%   \end{syntax}
%   This is a secondary key that doesn't directly involve any of the primary or
%   secondary keys.
%   This defines \key\ to take no value and set the flag called \meta{cs} to
%   |true| or |false|, respectively.
%   If \meta{cs} isn't defined yet it will be initialised as a flag.
%   Please also read \autoref{sec:flags}.
% \end{function}
%
% \begin{function}{flag-raise}
%   \begin{syntax}
%     flag-raise \key\ = \meta{cs} \prefixes3
%   \end{syntax}
%   This is a secondary key that doesn't directly involve any of the primary or
%   secondary keys.
%   This defines \key\ to take no value and raise the flag called \meta{cs}.
%   If \meta{cs} isn't defined yet it will be initialised as a flag.
%   Please also read \autoref{sec:flags}.
% \end{function}
%
%
% \subsection{Changing the Initial Values}\label{sec:change}
%
% \begin{function}{\ekvcChange}
%   \begin{syntax}
%     \cs{ekvcChange}\meta{cs}\{\kv,\ldots\}
%   \end{syntax}
%   This processes the \kv\ list (just like it would be processed by \meta{cs})
%   to set new defaults for the macro \meta{cs} defined with \expkvc\ (meaning
%   the values used if you don't provide anything at use time). Inside the \kv\
%   list both primary and secondary keys can be used. If \meta{cs} was defined
%   |\long| earlier it will still be |\long|, every other \TeX\ prefix will be
%   stripped (but \expkvc\ doesn't support them anywhere else so that should be
%   fine). The resulting new defaults will be stored inside the \meta{cs}
%   locally (just as the original defaults were).
% \end{function}
%
% Consider the following example:\\
% \begin{minipage}[c]{.75\linewidth}
% \begin{lstlisting}
% \ekvcSplit\foo{a=a,b=b}{a is #1.\par b is #2.\par}
% \foo{} % initial defaults
% \ekvcChange\foo{b=B}
% \foo{} % new defaults
% \ekvcSecondaryKeys\foo{meta c={a={#1},b={#1}}}
% \ekvcChange\foo{c=c}
% \foo{} % newer defaults
% \end{lstlisting}
% \end{minipage}^^A
% \begin{exresult}[nobeforeafter,box align=center]{.25\linewidth}
%   \ekvcSplit\foo{a=a,b=b}{a is #1.\par b is #2.\par}
%   \foo{}
%   \ekvcChange\foo{b=B}
%   \foo{}
%   \ekvcSecondaryKeys\foo{meta c={a={#1},b={#1}}}
%   \ekvcChange\foo{c=c}
%   \foo{}
% \end{exresult}
%
% As a result with this the typical setup macro could be implemented:
% \begin{lstlisting}
% \ekvcHashAndUse\foo{key=a,key=b}
% \newcommand\foosetup{\ekvcChange\foo}
% \end{lstlisting}
% Of course the usage is limited to a single macro |\foo|, hence this might be
% not as powerful as similar macros used with other \kv\ interfaces. But at
% least a few similar macros could be grouped using the same key parsing macro
% internally.
%
%
% \subsection{Flags}\label{sec:flags}
%
% The idea of flags is taken from \pkg{expl3}. They provide a way to store
% numerical information expandably, however only incrementing and accessing
% works expandably, decrementing is unexpandable. A flag has a height, which
% is a numerical value, and which can be raised by $1$. Flags come at a high
% computational cost (accessing them is slow and they require more memory than
% normal \TeX\ data types like registers, both getting linearly worse with the
% height), so don't use them if not necessary.
%
% The state of flags is always changed locally to the current group, but not to
% the current macro, so if you're using one of the |t|-types involving flags
% bear in mind that they can affect other macros using the same flags at the
% current group level!
%
% \expkvc\ provides some macros to access, alter, and use flags. Flags of
% \expkvc\ don't share a name space with the flags of \pkg{expl3}.
%
% \begin{function}{\ekvcFlagNew}
%   \begin{syntax}
%     \cs{ekvcFlagNew}\meta{flag}
%   \end{syntax}
%   This initialises the macro \meta{flag} as a new flag. It isn't checked
%   whether the macro \meta{flag} is currently undefined.  A \meta{flag} will
%   expand to the flag's current height with a trailing space (so you can use it
%   directly with |\ifnum| for example and it will terminate the number scanning
%   itself).
% \end{function}
%
%  All other macros dealing with flags take as a parameter a macro defined as a
%  \meta{flag} with |\ekvcFlagNew|.
%
% \begin{function}{\ekvcFlagHeight}
%   \begin{syntax}
%     \cs{ekvcFlagHeight}\meta{flag}
%   \end{syntax}
%   This expands to the current height of \meta{flag} in a single step of
%   expansion (without a trailing space).
% \end{function}
%
% \begin{function}{\ekvcFlagRaise}
%   \begin{syntax}
%     \cs{ekvcFlagRaise}\meta{flag}
%   \end{syntax}
%   This expandably raises the height of \meta{flag} by $1$.
% \end{function}
%
% \begin{function}{\ekvcFlagSetTrue,\ekvcFlagSetFalse}
%   \begin{syntax}
%     \cs{ekvcFlagSetTrue}\meta{flag}
%   \end{syntax}
%   By interpreting an even value as false and an odd value as true we can use a
%   flag as a boolean. This expandably sets \meta{flag} to true or false,
%   respectively, by raising it if necessary.
% \end{function}
%
% \begin{function}{\ekvcFlagIf}
%   \begin{syntax}
%     \cs{ekvcFlagIf}\meta{flag}\marg{true}\marg{false}
%   \end{syntax}
%   This interprets a \meta{flag} as a boolean and expands to either \meta{true}
%   or \meta{false}.
% \end{function}
%
% \begin{function}{\ekvcFlagIfRaised}
%   \begin{syntax}
%     \cs{ekvcFlagIfRaised}\meta{flag}\marg{true}\marg{false}
%   \end{syntax}
%   This tests whether the \meta{flag} is raised, meaning it has a height
%   greater than zero, and if so expands to \meta{true} else to \meta{false}.
% \end{function}
%
% \begin{function}{\ekvcFlagReset}
%   \begin{syntax}
%     \cs{ekvcFlagReset}\meta{flag}
%   \end{syntax}
%   This resets a flag (so restores its height to $0$). This operation is
%   \emph{not} expandable and done locally. If you really intend to use flags
%   you can reset them every now and then to keep the performance hit low.
% \end{function}
%
% \begin{function}{\ekvcFlagGetHeight}
%   \begin{syntax}
%     \cs{ekvcFlagGetHeight}\meta{flag}\marg{next}
%   \end{syntax}
%   This retrieves the current height of the \meta{flag} and provides it as a
%   braced argument to \meta{next}, leaving \meta{next}\marg{height} in the
%   input stream.
% \end{function}
%
% \begin{function}{\ekvcFlagGetHeights}
%   \begin{syntax}
%     \cs{ekvcFlagGetHeights}\marg{flag-list}\marg{next}
%   \end{syntax}
%   This retrieves the current height of each \meta{flag} in the
%   \meta{flag-list} and provides them as a single argument to \meta{next}.
%   Inside that argument each height is enclosed in a set of braces
%   individually. The \meta{flag-list} is just a single argument containing the
%   \meta{flag}s.
%   So a usage like
%   \texttt
%     {^^A
%       \cs[no-index]{ekvcFlagGetHeights}\hskip0pt^^A
%       \{\cs[no-index]{myflagA}\cs[no-index]{myflagB}\}\hskip0pt^^A
%       \{\cs[no-index]{stuff}\}^^A
%     }
%   will expand to
%   \texttt{\cs[no-index]{stuff}\{\marg{height-A}\marg{height-B}\}}.
% \end{function}
%
%
% \subsection{Further Examples}
%
% How could a documentation be a good documentation without some basic examples?
% Say we want to define a small macro expanding to some character description
% (who knows why this has to be expandable?). A character description will not
% have too many items to it, so we use |\ekvcSplit| (the comments with the
% parameter numbers are of course not necessary and just ease reading the
% example).
% \begin{lstlisting}
% \ekvcSplit\character
%   {
%     name=John Doe,            % #1
%     age=any,                  % #2
%     nationality=the Universe, % #3
%     hobby=to exist,           % #4
%     type=Mister,              % #5
%     pronoun=He,               % #6
%     possessive=his,           % #7
%   }
%   {#1 is a #5 from #3. #6 is of #2 age and #7 hobby is #4.\par}
% \end{lstlisting}
% Also we want to give some short cuts so that it's easier to describe several
% persons.
% \begin{lstlisting}
% \ekvcSecondaryKeys\character
%   {
%     alias pro = pronoun,
%     alias pos = possessive,
%     nmeta me =
%       {
%         name=Jonathan,
%         age=a young,
%         nationality=Germany,
%         hobby=\TeX\ coding,
%       },
%     meta lady =
%       {type=Lady, pronoun=She, possessive=her, name=Jane Doe, #1},
%     nmeta paulo =
%       {
%         name=Paulo,
%         type=duck,
%         age=a young,
%         nationality=Brazil,
%         hobby=to quack,
%       }
%   }
% \end{lstlisting}
% Now we can describe people using
% \begin{lstlisting}
% \character{}
% \character{me}
% \character{paulo}
% \character
%   {lady={name=Evelyn,nationality=Ireland,age=the best,hobby=reading}}
% \character
%   {
%     name=Our sun, type=star, nationality=our solar system, pro=It,
%     age=an old, pos=its, hobby=shining
%   }
% \end{lstlisting}
% As one might see, the |lady| key could actually have been an |nmeta| key as
% well, as all that is done with the argument is using it as a \kv\ list.
%
% The result of only the first two usages would be:
% \begin{exresult}{\linewidth}
%   \ekvcSplit\character
%     {
%       name=John Doe,
%       age=any,
%       nationality=the Universe,
%       hobby=to exist,
%       type=Mister,
%       pronoun=He,
%       possessive=his,
%     }
%     {#1 is a #5 from #3. #6 is of #2 age and #7 hobby is #4.\par}
%   \ekvcSecondaryKeys\character
%     {
%       alias pro = pronoun,
%       alias pos = possessive,
%       nmeta me =
%         {
%           name=Jonathan,
%           age=a young,
%           nationality=Germany,
%           hobby=\TeX\ coding,
%         },
%       meta lady =
%         {type=Lady, pronoun=She, possessive=her, name=Jane Doe, #1},
%       nmeta paulo =
%         {
%           name=Paulo,
%           type=duck,
%           age=a young,
%           nationality=Brazil,
%           hobby=to quack,
%         }
%     }
%   \character{}
%   \character{me}
% \end{exresult}
%
% Using \pkg{xparse} or \expkv's \cs{ekvoptarg} or \cs{ekvoptargTF} and
% forwarding arguments one can easily define \kv\ macros with actual optional
% and mandatory arguments as well. A small nonsense example (which should
% perhaps use |\ekvcSplitAndForward| instead of |\ekvcHashAndForward| since it
% only uses four keys and one other argument -- and isn't expandable since it
% uses a \env{tabular} environment, so it would've been better to use a more
% feature rich \kv\ interface most likely, \emph{e.g.}, the one provided by
% \expkvd):
% \begin{lstlisting}
% \makeatletter
% \newcommand*\nonsense{\ekvoptarg\nonsense@a{}}
% \ekvcHashAndForward\nonsense@a\nonsense@b
%   {
%     keyA = A,
%     keyB = B,
%     keyC = c,
%     keyD = d,
%   }
% \newcommand*\nonsense@b[2]
%   {%
%     \begin{tabular}{lll}
%       key & A & \ekvcValue{keyA}{#1} \\
%           & B & \ekvcValue{keyB}{#1} \\
%           & C & \ekvcValue{keyC}{#1} \\
%           & D & \ekvcValue{keyD}{#1} \\
%       \multicolumn{2}{l}{mandatory} & #2 \\
%     \end{tabular}%
%   }
% \makeatother
% \end{lstlisting}
% And then we would be able to do some nonsense
% \begin{lstlisting}
% \nonsense{}
% \nonsense[keyA=hihi]{haha}
% \nonsense[keyA=hihi, keyB=A]{hehe}
% \nonsense[keyC=huhu, keyA=hihi, keyB=A]{haha}
% \end{lstlisting}
% resulting in
% \begin{exresult}{\linewidth}
%   \makeatletter
%   \newcommand*\nonsense{\ekvoptarg\nonsense@a{}}
%   \ekvcHashAndForward\nonsense@a\nonsense@b
%     {
%       keyA = A,
%       keyB = B,
%       keyC = c,
%       keyD = d,
%     }
%   \newcommand*\nonsense@b[2]
%     {%
%       \begin{tabular}{lll}
%         key & A & \ekvcValue{keyA}{#1} \\
%             & B & \ekvcValue{keyB}{#1} \\
%             & C & \ekvcValue{keyC}{#1} \\
%             & D & \ekvcValue{keyD}{#1} \\
%         \multicolumn{2}{l}{mandatory} & #2 \\
%       \end{tabular}%
%     }
%   \makeatother
%   \nonsense{}
%   \nonsense[keyA=hihi]{haha}
%   \nonsense[keyA=hihi, keyB=A]{hehe}
%   \nonsense[keyC=huhu, keyA=hihi, keyB=A]{haha}
% \end{exresult}
%
%
% \subsection{Freedom for Keys!}
%
% If this was the \TeX book this subsection would have a double bend sign. Not
% because it is overly complicated, but because it shows things which could
% break \expkvc's expandability and its alignment safety. This is for
% experienced users wanting to get the most flexibility and knowing what they
% are doing.
%
% In case you're wondering, it is possible to define other keys than the
% primaries and the secondary types listed in \autoref{sec:secondaries} for a
% macro defined with \expkvc\ by using the low-level interface of \expkv\ or
% even the interface provided by \expkvd. The set name used for \expkvc's keys
% is the macro name, including the leading backslash, or more precisely
% |\string|\meta{cs} is used. This can be exploited to define additional keys
% with arbitrary code. Consider the following \emph{bad} example:
% \begin{lstlisting}
% \ekvcSplit\foo{a=A,b=B}{a is #1.\par b is #2}
% \protected\ekvdef{\string\foo}{c}{\def\fooC{#1}}
% \end{lstlisting}
% This would define a key named |c| that will store its value inside a macro.
% The issue with this is that this can't be done expandably. As a result, the
% macro |\foo| isn't always expandable any more (not that bad if this was never
% required; killjoy if it was) and as soon as the key |c| is used, it is also no
% longer alignment safe\footnote{This means that the \kv-list can't contain
%   alignment markers that are not inside an additional set of braces if used
%   inside a \TeX\ alignment.}
% (might be bad depending on the usage).
%
% So why do I show you this? Because we could as well do something useful like
% create a key that pre-parses the input and after that passes the parsed value
% on. This parsing would have to be completely expandable though. For the
% pass-on part we can use the following function:
%
% \begin{function}{\ekvcPass}
%   \begin{syntax}
%     \cs{ekvcPass}\meta{cs}\marg{key}\marg{value}
%   \end{syntax}
%   This passes \meta{value} on to \meta{key} for the \expkvc-macro \meta{cs}.
%   It should be used inside the key parsing of a macro defined with \expkvc,
%   else this most likely results in a low level \TeX\ error.
% \end{function}
%
% With this we could for example split the value of a key at a hyphen and pass
% the parts to different keys:\\
% \begin{minipage}[c]{.75\linewidth}
% \begin{lstlisting}
% \ekvcSplit\foo{a=A,b=B}{a is #1.\par b is #2.\par}
% \ekvdef{\string\foo}{c}{\fooSplit#1\par}
% \def\fooSplit#1-#2\par
%   {\ekvcPass\foo{a}{#1}\ekvcPass\foo{b}{#2}}
% \foo{}
% \foo{c=1-2}
% \end{lstlisting}
% \end{minipage}
% \begin{exresult}[nobeforeafter,box align=center]{.25\linewidth}
%   \ekvcSplit\foo{a=A,b=B}{a is #1.\par b is #2.\par}
%   \ekvdef{\string\foo}{c}{\fooSplit#1\par}
%   \def\fooSplit#1-#2\par{\ekvcPass\foo{a}{#1}\ekvcPass\foo{b}{#2}}
%   \foo{}
%   \foo{c=1-2}
% \end{exresult}
%
%
% \subsection{Speed Considerations}\label{sec:speed}
%
% As already mentioned in the introduction there are some speed considerations
% implied if you choose to define macros via \expkvc. However the overhead isn't
% the factor which should hinder you to use \expkvc\ if you found a reasonable
% use case. The key-parsing is still faster than with most other \kv\ packages
% (see the ``Comparisons'' subsection in the \expkv\ documentation).
%
% The speed considerations in this subsection use the first example in this
% documentation as the benchmark. So we have seven keys and a short sentence
% which should be typeset. For comparisons I use the following equivalent
% \expkvd\ definitions. Each result is the average between changing no keys from
% their initial values and altering four. Furthermore I'll compare three
% variants of \expkvc\ with the \expkvd\ definitions, namely the split example
% from above, a hash variant using |\ekvcValue| and a hash variant using
% |\ekvcValueFast|.
% \begin{lstlisting}
% \usepackage{expkv-def}
% \ekvdefinekeys{keys}
%   {%
%     ,store   name         = \KEYSname
%     ,initial name         = John Doe
%     ,store   age          = \KEYSage
%     ,initial age          = any
%     ,store   nationality  = \KEYSnationality
%     ,initial nationality  = the Universe
%     ,store   hobby        = \KEYShobby
%     ,initial hobby        = to exist
%     ,store   type         = \KEYStype
%     ,initial type         = Mister
%     ,store   pronoun      = \KEYSpronoun
%     ,initial pronoun      = He
%     ,store   possessive   = \KEYSpossessive
%     ,initial possessive   = his
%   }
% \newcommand*\KEYS[1]
%   {%
%     \begingroup
%       \ekvset{keys}{#1}%
%       \KEYSname\ is a \KEYStype\ from \KEYSnationality. \KEYSpronoun\ is
%       of \KEYSage\ age and \KEYSpossessive\ hobby is \KEYShobby.%
%     \endgroup
%   }
% \end{lstlisting}
%
% The first comparison removes the typesetting part from all the definitions, so
% that only the key parsing is compared. In this comparison the |\ekvcValue|
% and |\ekvcValueFast| variants will not differ, as they are exactly the same
% until the key usage. We find that the split approach is $1.4$ times slower
% than the \expkvd\ setup and the hash variants end up in the middle at $1.17$
% times slower.
%
% Next we put the typesetting part back in. Every call of the macros will
% typeset the sentences into a box register in horizontal mode. With the
% typesetting part (which includes the accessing of values) the fastest remains
% the \expkvd\ definitions, but split is close at $1.16$ times slower, followed
% by the hash variant with fast accesses at $1.36$ times slower, and the safe
% hash access variant ranks in the slowest $1.8$ times slower than \expkvd.
%
% Just in case you're wondering now, a simple macro taking seven arguments is
% $30$ to $40$ times faster than any of those in the argument grabbing and \kv\
% parsing part and only $1.5$ to $2.8$ times faster if the typesetting part is
% factored in. So the real choke isn't the parsing.
%
% So to summarize this, if you have a reasonable use case for expandable \kv\
% parsing macros you should go on and define them using \expkvc. If you have a
% reasonable use case for \kv\ parsing macros but defining them expandable isn't
% necessary for your use you should take advantage of the greater flexibility of
% non-expandable \kv\ setups (but if you're after maximum speed there aren't
% that many \kv\ parsers beating \expkvc). And if you are after maximum
% performance maybe ditching the \kv\ interface altogether is a good idea, but
% depending on the number of arguments your interface might get convoluted.
%
%
% \subsection{Useless Macros}
%
% Perhaps these macros aren't completely useless, but I figured from a user's
% point of view I wouldn't know what I should do with these.
%
% \begin{function}{\ekvcDate,\ekvcVersion}
%   These two macros store the version and the date of the package/generic code.
% \end{function}
%
%
% \subsection{Bugs}
%
% Of course I don't think there are any bugs (who would knowingly distribute
% buggy software as long as he isn't a multi-million dollar corporation?). But
% if you find some please let me know. For this one might find my email address
% on the first page or file an issue on Github:
% \url{https://github.com/Skillmon/tex_expkv-cs}
%
%
% \subsection{License}
%
% Copyright \textcopyright\ 2020\unless\ifnum\year=2020--\the\year\fi\
% Jonathan P. Spratte
%
% \medskip\noindent
% This work may be distributed and/or modified under the conditions of the
% \LaTeX\ Project Public License (LPPL), either version 1.3c of this license or
% (at your option) any later version. The latest version of this license is in
% the file:
%
% \url{http://www.latex-project.org/lppl.txt}
%
% \noindent
% This work is ``maintained'' (as per LPPL maintenance status) by
%
% \mbox{Jonathan P. Spratte}.
%
% \end{documentation}^^A=<<
%
% \begin{implementation}^^A>>=
%
% \clearpage
%
% \section{Implementation}^^A>>=
%^^A the LaTeX package >>=
% \subsection{The \LaTeX\ Package}
% Just like for \expkv\ we provide a small \LaTeX\ package that sets up things
% such that we behave nicely on \LaTeX\ packages and files system. It'll
% |\input| the generic code which implements the functionality.
% \gobbledocstriptag
%<*pkg>
%    \begin{macrocode}
\RequirePackage{expkv}
\def\ekvc@tmp
  {%
    \ProvidesFile{expkv-cs.tex}%
      [%
        \ekvcDate\space v\ekvcVersion\space
        define expandable key=val macros using expkv%
      ]%
  }
\input{expkv-cs.tex}
\ProvidesPackage{expkv-cs}%
  [%
    \ekvcDate\space v\ekvcVersion\space
    define expandable key=val macros using expkv%
  ]
%    \end{macrocode}
% \gobbledocstriptag
%</pkg>
%^^A=<<
%^^A main file >>=
% \subsection{The Generic Code}
% The rest of this implementation will be the generic code.
% \gobbledocstriptag
%<*tex>
%
% Load \expkv\ if the package didn't already do so -- since \expkv\ has
% safeguards against being loaded twice this does no harm and the overhead
% isn't that big. Also we reuse some of the internals of \expkv\ to save us from
% retyping them.
%    \begin{macrocode}
\input expkv
%    \end{macrocode}
%
% We make sure that \file{expkv-cs.tex} is only input once:
%    \begin{macrocode}
\expandafter\ifx\csname ekvcVersion\endcsname\relax
\else
  \expandafter\endinput
\fi
%    \end{macrocode}
%
% \begin{macro}{\ekvcVersion,\ekvcDate}
% We're on our first input, so lets store the version and date in a macro.
%    \begin{macrocode}
\def\ekvcVersion{0.10}
\def\ekvcDate{2021-06-03}
%    \end{macrocode}
% \end{macro}
%
% If the \LaTeX\ format is loaded we want to be a good file and report back who
% we are, for this the package will have defined |\ekvc@tmp| to use
% |\ProvidesFile|, else this will expand to a |\relax| and do no harm.
%    \begin{macrocode}
\csname ekvc@tmp\endcsname
%    \end{macrocode}
%
% Store the category code of |@| to later be able to reset it and change it to
% 11 for now.
%    \begin{macrocode}
\expandafter\chardef\csname ekvc@tmp\endcsname=\catcode`\@
\catcode`\@=11
%    \end{macrocode}
% |\ekvc@tmp| will be reused later, but we don't need it to ever store
% information long-term after \expkvc\ was initialized.
%
% \begin{macro}[internal]{\ekvc@keycount}
%   We'll need to keep count how many keys must be defined for each macro in the
%   |split| variants.
%    \begin{macrocode}
\newcount\ekvc@keycount
%    \end{macrocode}
% \end{macro}
%
% \begin{macro}[internal]{\ekvc@long,\ekvc@any@long}
%   Some macros will have to be defined long. These two will be let to |\long|
%   when this should be the case.
%    \begin{macrocode}
\let\ekvc@long\ekv@empty
\let\ekvc@any@long\ekv@empty
%    \end{macrocode}
% \end{macro}
%
% \begin{macro}[internal]{\ekvc@ifdefined}
%   We want to test whether a macro is already defined. This test checks for a
%   defined macro that isn't |\relax|.
%    \begin{macrocode}
\long\def\ekvc@ifdefined#1%
  {%
    \ifdefined#1%
      \ifx\relax#1%
        \ekv@fi@gobble
      \fi
      \@firstofone
      \ekv@fi@firstoftwo
    \fi
    \@secondoftwo
  }
%    \end{macrocode}
% \end{macro}
%
% \begin{macro}[internal]
%   {
%     \ekvc@ekvset@pre@expander,
%     \ekvc@ekvset@pre@expander@a,
%     \ekvc@ekvset@pre@expander@b
%   }
%   This macro expands |\ekvset| twice so that the first two steps of expansion
%   don't have to be made every time the \expkvc\ macros are used. We have to do
%   a little magic trick to get the macro parameter |#1| for the macro
%   definition this is used in, even though we're calling |\unexpanded|. We do
%   that by splitting the expanded |\ekvset| at some marks and place |##1| in
%   between. At this spot we also add |\ekv@alignsafe| and |\ekv@endalignsafe|
%   to ensure that macros created with \expkvc\ are alignment safe.
%    \begin{macrocode}
\def\ekvc@ekvset@pre@expander#1%
  {%
    \expandafter\ekvc@ekvset@pre@expander@a\ekvset{#1}\ekvc@stop\ekvc@stop
  }
\def\ekvc@ekvset@pre@expander@a
  {%
    \expandafter\ekvc@ekvset@pre@expander@b
  }
\def\ekvc@ekvset@pre@expander@b#1\ekvc@stop#2\ekvc@stop
  {%
    \unexpanded\expandafter{\ekv@alignsafe}%
    \unexpanded{#1}##1\unexpanded{#2}%
    \unexpanded\expandafter{\ekv@endalignsafe}%
  }
%    \end{macrocode}
% \end{macro}
%
% \begin{macro}{\ekvcSplitAndUse}
%   The first user macro we want to set up can be reused for
%   \cs[no-index]{ekvcSplitAndForward} and \cs[no-index]{ekvcSplit}. We'll split
%   this one up so that the test whether the macro is already defined doesn't
%   run twice.
%    \begin{macrocode}
\protected\long\def\ekvcSplitAndUse#1#2%
  {%
    \let\ekvc@helpers@needed\@firstoftwo
    \ekvc@ifdefined#1%
      {\ekvc@err@already@defined#1}%
      {\ekvcSplitAndUse@#1{}{#2}}%
  }
%    \end{macrocode}
% \end{macro}
%
% \begin{macro}[internal]{\ekvcSplitAndUse@}
%   The actual macro setting up things. We need to set some variables, forward
%   the key list to |\ekvc@SetupSplitKeys|, and afterwards define the front
%   facing macro to call |\ekvset| and put the initials and the argument sorting
%   macro behind it. The internals |\ekvc@any@long|, |\ekvc@initials| and
%   |\ekvc@keycount| will be set correctly by |\ekvc@SetupSplitKeys|.
%    \begin{macrocode}
\protected\long\def\ekvcSplitAndUse@#1#2#3%
  {%
    \edef\ekvc@set{\string#1}%
    \ekvc@SetupSplitKeys{#3}%
    \ekvc@helpers@needed
      {%
        \ekvc@any@long\edef#1##1%
          {%
            \expandafter\ekvc@ekvset@pre@expander\expandafter{\ekvc@set}%
            \unexpanded\expandafter
              {\csname ekvc@split@\the\ekvc@keycount\endcsname}%
            \unexpanded\expandafter{\ekvc@initials{}#2}%
          }%
      }%
      {%
        \ekvc@any@long\edef#1##1%
          {%
            \expandafter\ekvc@ekvset@pre@expander\expandafter{\ekvc@set}%
            \unexpanded{#2}%
            \unexpanded\expandafter{\ekvc@initials}%
          }%
      }%
  }
%    \end{macrocode}
% \end{macro}
%
%
% \begin{macro}{\ekvcSplitAndForward}
%   This just reuses |\ekvcSplitAndUse@| with a non-empty second argument,
%   resulting in that argument to be called after the splitting.
%    \begin{macrocode}
\protected\long\def\ekvcSplitAndForward#1#2#3%
  {%
    \let\ekvc@helpers@needed\@firstoftwo
    \ekvc@ifdefined#1%
      {\ekvc@err@already@defined#1}%
      {\ekvcSplitAndUse@#1{{#2}}{#3}}%
  }
%    \end{macrocode}
% \end{macro}
%
% \begin{macro}{\ekvcSplit}
%   The first half is just |\ekvcSplitAndForward| then we define the macro to
%   which the parsed key list is forwarded. There we need to allow for up to
%   nine arguments.
%    \begin{macrocode}
\protected\long\def\ekvcSplit#1#2#3%
  {%
    \let\ekvc@helpers@needed\@secondoftwo
    \ekvc@ifdefined#1%
      {\ekvc@err@already@defined#1}%
      {%
        \expandafter
        \ekvcSplitAndUse@\expandafter#1\csname ekvc@\string#1\endcsname{#2}%
        \ifnum\ekvc@keycount<1
          \ekvc@any@long\expandafter\def\csname ekvc@\string#1\endcsname{#3}%
        \else
          \ifnum\ekvc@keycount>9
            \ekvc@err@toomany{#1}%
            \let#1\ekvc@undefined
          \else
            \ekvcSplit@build@argspec
            \ekvc@any@long\expandafter
            \def\csname ekvc@\string#1\expandafter\endcsname\ekvc@tmp{#3}%
          \fi
        \fi
      }%
  }
%    \end{macrocode}
% \end{macro}
%
% \begin{macro}[internal]{\ekvcSplit@build@argspec,\ekvcSplit@build@argspec@}
%    \begin{macrocode}
\protected\def\ekvcSplit@build@argspec
  {%
    \begingroup
      \edef\ekvc@tmp
        {\endgroup\def\unexpanded{\ekvc@tmp}{\ekvcSplit@build@argspec@{1}}}%
    \ekvc@tmp
  }
\def\ekvcSplit@build@argspec@#1%
  {%
    \ifnum#1>\ekvc@keycount
      \ekv@fi@gobble
    \fi
    \@firstofone
      {%
        \unexpanded\expandafter{\csname ekvc@splitmark@#1\endcsname####}#1%
        \expandafter\ekvcSplit@build@argspec@\expandafter{\the\numexpr#1+1}%
      }%
  }
%    \end{macrocode}
% \end{macro}
%
% \begin{macro}[internal]
%   {
%     \ekvc@SetupSplitKeys, \ekvc@SetupSplitKeys@a, \ekvc@SetupSplitKeys@b,
%     \ekvc@SetupSplitKeys@c
%   }
%   These macros parse the list of keys and set up the key macros. First we need
%   to initialise some macros and start |\ekvparse|.
%    \begin{macrocode}
\protected\long\def\ekvc@SetupSplitKeys
  {%
    \ekvc@keycount=0
    \let\ekvc@any@long\ekv@empty
    \let\ekvc@initials\ekv@empty
    \ekvparse\ekvc@err@value@required\ekvc@SetupSplitKeys@a
  }
%    \end{macrocode}
%   Then we need to step the key counter for each key. Also we have to check
%   whether this key has a |long| prefix so we initialise |\ekvc@long|.
%    \begin{macrocode}
\protected\def\ekvc@SetupSplitKeys@a#1%
  {%
    \advance\ekvc@keycount1
    \let\ekvc@long\ekv@empty
    \ekvc@ifspace{#1}%
      {\ekvc@SetupSplitKeys@b#1\ekvc@stop}%
      {\ekvc@SetupSplitKeys@c{#1}}%
  }
%    \end{macrocode}
%   If there was a space, there might be a prefix. If so call the prefix macro,
%   else call the next step |\ekvc@SetupSplitKeys@c| which will define the key
%   macro and add the key's value to the initials list.
%    \begin{macrocode}
\protected\def\ekvc@SetupSplitKeys@b#1 #2\ekvc@stop
  {%
    \ekv@ifdefined{ekvc@split@p@#1}%
      {\csname ekvc@split@p@#1\endcsname{#2}}%
      {\ekvc@SetupSplitKeys@c{#1 #2}}%
  }
%    \end{macrocode}
%   The inner definition is grouped, because we don't want to actually define
%   the marks we build with |\csname|. We have to append the value to the
%   |\ekvc@initials| list here with the correct split mark. The key macro will
%   read everything up to those split marks and change the value following it to
%   the value given to the key. Additionally we'll need a sorting macro for each
%   key count in use so we set it up with |\ekvc@setup@splitmacro|.
%    \begin{macrocode}
\protected\long\def\ekvc@SetupSplitKeys@c#1#2%
  {%
    \begingroup
    \edef\ekvc@tmp
      {%
        \endgroup
        \long\def\unexpanded{\ekvc@tmp}####1####2%
            \unexpanded\expandafter
            {\csname ekvc@splitmark@\the\ekvc@keycount\endcsname}####3%
          {%
            ####2%
            \unexpanded\expandafter
              {\csname ekvc@splitmark@\the\ekvc@keycount\endcsname}{####1}%
          }%
%    \end{macrocode}
%   The short variant needs a bit of special treatment. The key macro will be
%   short to throw the correct error, but since there might be long macros
%   somewhere the reordering of arguments needs to be long, so for short keys we
%   use a two step approach, first grabbing only the short argument, then
%   reordering.
%    \begin{macrocode}
        \unless\ifx\ekvc@long\long
          \let\unexpanded\expandafter
            {\csname ekvc@\ekvc@set(#1)\endcsname\ekvc@tmp}%
          \def\unexpanded{\ekvc@tmp}####1%
            {%
              \unexpanded\expandafter{\csname ekvc@\ekvc@set(#1)\endcsname}%
                {####1}%
            }%
        \fi
        \def\unexpanded{\ekvc@initials}%
          {%
            \unexpanded\expandafter{\ekvc@initials}%
            \unexpanded\expandafter
              {\csname ekvc@splitmark@\the\ekvc@keycount\endcsname{#2}}%
          }%
      }%
    \ekvc@tmp
    \ekvlet\ekvc@set{#1}\ekvc@tmp
    \ekvc@helpers@needed
      {\expandafter\ekvc@setup@splitmacro\expandafter{\the\ekvc@keycount}}%
      {}%
  }
%    \end{macrocode}
% \end{macro}
%
% \begin{macro}[internal]{\ekvc@split@p@long}
%   The |long| prefix lets the internals |\ekvc@long| and |\ekvc@any@long| to
%   |\long| so that the key macro will be long.
%    \begin{macrocode}
\protected\def\ekvc@split@p@long
  {%
    \let\ekvc@long\long
    \let\ekvc@any@long\long
    \ekvc@SetupSplitKeys@c
  }
%    \end{macrocode}
% \end{macro}
%
% \begin{macro}[internal]{\ekvc@defarggobbler}
%   This is needed to define a macro with 1-9 parameters programmatically.
%   \LaTeX's \cs[no-index]{newcommand} does something similar for example.
%    \begin{macrocode}
\protected\def\ekvc@defarggobbler#1{\def\ekvc@tmp##1#1##2##{##1#1}}
%    \end{macrocode}
% \end{macro}
%
% \begin{macro}[internal]
%   {
%     \ekvc@setup@splitmacro,
%     \ekvc@split@1, \ekvc@split@2, \ekvc@split@3,
%     \ekvc@split@4, \ekvc@split@5, \ekvc@split@6,
%     \ekvc@split@7
%   }
%   Since the first few split macros are different from the others we manually
%   set those up now. All the others will be defined as needed (always
%   globally).  The split macros just read up until the correct split mark, move
%   that argument into a list and reinsert the rest, calling the next split
%   macro afterwards.
%    \begin{macrocode}
\begingroup
\edef\ekvc@tmp
  {%
    \long\gdef\unexpanded\expandafter{\csname ekvc@split@1\endcsname}%
        \unexpanded\expandafter{\csname ekvc@splitmark@1\endcsname}%
        ##1##2##3%
      {##3{##1}##2}%
    \long\gdef\unexpanded\expandafter{\csname ekvc@split@2\endcsname}%
        \unexpanded\expandafter{\csname ekvc@splitmark@1\endcsname}##1%
        \unexpanded\expandafter{\csname ekvc@splitmark@2\endcsname}##2%
        ##3##4%
      {##4{##1}{##2}##3}%
    \long\gdef\unexpanded\expandafter{\csname ekvc@split@3\endcsname}%
        \unexpanded\expandafter{\csname ekvc@splitmark@1\endcsname}##1%
        \unexpanded\expandafter{\csname ekvc@splitmark@2\endcsname}##2%
        \unexpanded\expandafter{\csname ekvc@splitmark@3\endcsname}##3%
        ##4##5%
      {##5{##1}{##2}{##3}##4}%
    \long\gdef\unexpanded\expandafter{\csname ekvc@split@4\endcsname}%
        \unexpanded\expandafter{\csname ekvc@splitmark@1\endcsname}##1%
        \unexpanded\expandafter{\csname ekvc@splitmark@2\endcsname}##2%
        \unexpanded\expandafter{\csname ekvc@splitmark@3\endcsname}##3%
        \unexpanded\expandafter{\csname ekvc@splitmark@4\endcsname}##4%
        ##5##6%
      {##6{##1}{##2}{##3}{##4}##5}%
    \long\gdef\unexpanded\expandafter{\csname ekvc@split@5\endcsname}%
        \unexpanded\expandafter{\csname ekvc@splitmark@1\endcsname}##1%
        \unexpanded\expandafter{\csname ekvc@splitmark@2\endcsname}##2%
        \unexpanded\expandafter{\csname ekvc@splitmark@3\endcsname}##3%
        \unexpanded\expandafter{\csname ekvc@splitmark@4\endcsname}##4%
        \unexpanded\expandafter{\csname ekvc@splitmark@5\endcsname}##5%
        ##6##7%
      {##7{##1}{##2}{##3}{##4}{##5}##6}%
    \long\gdef\unexpanded\expandafter{\csname ekvc@split@6\endcsname}%
        \unexpanded\expandafter{\csname ekvc@splitmark@1\endcsname}##1%
        \unexpanded\expandafter{\csname ekvc@splitmark@2\endcsname}##2%
        \unexpanded\expandafter{\csname ekvc@splitmark@3\endcsname}##3%
        \unexpanded\expandafter{\csname ekvc@splitmark@4\endcsname}##4%
        \unexpanded\expandafter{\csname ekvc@splitmark@5\endcsname}##5%
        \unexpanded\expandafter{\csname ekvc@splitmark@6\endcsname}##6%
        ##7##8%
      {##8{##1}{##2}{##3}{##4}{##5}{##6}##7}%
    \long\gdef\unexpanded\expandafter{\csname ekvc@split@7\endcsname}%
        \unexpanded\expandafter{\csname ekvc@splitmark@1\endcsname}##1%
        \unexpanded\expandafter{\csname ekvc@splitmark@2\endcsname}##2%
        \unexpanded\expandafter{\csname ekvc@splitmark@3\endcsname}##3%
        \unexpanded\expandafter{\csname ekvc@splitmark@4\endcsname}##4%
        \unexpanded\expandafter{\csname ekvc@splitmark@5\endcsname}##5%
        \unexpanded\expandafter{\csname ekvc@splitmark@6\endcsname}##6%
        \unexpanded\expandafter{\csname ekvc@splitmark@7\endcsname}##7%
        ##8##9%
      {##9{##1}{##2}{##3}{##4}{##5}{##6}{##7}##8}%
  }
\ekvc@tmp
\endgroup
\protected\def\ekvc@setup@splitmacro#1%
  {%
    \ekv@ifdefined{ekvc@split@#1}{}%
      {%
        \begingroup
          \edef\ekvc@tmp
            {%
              \long\gdef
                  \unexpanded\expandafter{\csname ekvc@split@#1\endcsname}%
                  ####1%
                  \unexpanded\expandafter{\csname ekvc@splitmark@#1\endcsname}%
                  ####2####3%
                {%
                  \unexpanded\expandafter
                    {\csname ekvc@split@\the\numexpr#1-1\relax\endcsname}%
                  ####1{{####2}####3}%
                }%
            }%
          \ekvc@tmp
        \endgroup
      }%
  }
%    \end{macrocode}
% \end{macro}
%
% \begin{macro}{\ekvcHashAndUse}
%   |\ekvcHashAndUse| works just like |\ekvcSplitAndUse|.
%    \begin{macrocode}
\protected\long\def\ekvcHashAndUse#1#2%
  {%
    \let\ekvc@helpers@needed\@firstoftwo
    \ekvc@ifdefined#1%
      {\ekvc@err@already@defined#1}%
      {\ekvcHashAndUse@#1{}{#2}}%
  }
%    \end{macrocode}
% \end{macro}
%
% \begin{macro}[internal]{\ekvcHashAndUse@}
%   This is more or less the same as |\ekvcSplitAndUse@|. Instead of an empty
%   group we place a marker after the initials, we don't use the sorting macros
%   of |split|, but instead pack all the values in one argument.
%    \begin{macrocode}
\protected\long\def\ekvcHashAndUse@#1#2#3%
  {%
    \edef\ekvc@set{\string#1}%
    \ekvc@SetupHashKeys{#3}%
    \ekvc@helpers@needed
      {%
        \ekvc@any@long\edef#1##1%
          {%
            \expandafter\ekvc@ekvset@pre@expander\expandafter{\ekvc@set}%
            \unexpanded{\ekvc@hash@pack@argument}%
            \unexpanded\expandafter{\ekvc@initials\ekvc@stop#2}%
          }%
      }%
      {%
        \ekvc@any@long\edef#1##1%
          {%
            \expandafter\ekvc@ekvset@pre@expander\expandafter{\ekvc@set}%
            \unexpanded{#2}%
            \unexpanded\expandafter{\ekvc@initials\ekvc@stop}%
          }%
      }%
  }
%    \end{macrocode}
% \end{macro}
%
% \begin{macro}{\ekvcHashAndForward}
%   |\ekvcHashAndForward| works just like |\ekvcSplitAndForward|.
%    \begin{macrocode}
\protected\long\def\ekvcHashAndForward#1#2#3%
  {%
    \let\ekvc@helpers@needed\@firstoftwo
    \ekvc@ifdefined#1%
      {\ekvc@err@already@defined#1}%
      {\ekvcHashAndUse@#1{{#2}}{#3}}%
  }
%    \end{macrocode}
% \end{macro}
%
% \begin{macro}{\ekvcHash}
%   |\ekvcHash| does the same as |\ekvcSplit|, but has the advantage of not
%   needing to count arguments, so the definition of the internal macro is a bit
%   more straight forward.
%    \begin{macrocode}
\protected\long\def\ekvcHash#1#2#3%
  {%
    \let\ekvc@helpers@needed\@secondoftwo
    \ekvc@ifdefined#1%
      {\ekvc@err@already@defined#1}%
      {%
        \expandafter
        \ekvcHashAndUse@\expandafter#1\csname ekvc@\string#1\endcsname{#2}%
        \ekvc@any@long\expandafter\def\csname ekvc@\string#1\endcsname
            ##1\ekvc@stop
          {#3}%
      }%
  }
%    \end{macrocode}
% \end{macro}
%
% \begin{macro}[internal]{\ekvc@hash@pack@argument}
%   All this macro does is pack the values into one argument and forward that to
%   the next macro.
%    \begin{macrocode}
\long\def\ekvc@hash@pack@argument#1\ekvc@stop#2{#2{#1}}
%    \end{macrocode}
% \end{macro}
%
% \begin{macro}[internal]
%   {\ekvc@SetupHashKeys,\ekvc@SetupHashKeys@a,\ekvc@SetupHashKeys@b}
%   This should look awfully familiar as well, since it's just the same as for
%   the split keys with a few other names here and there.
%    \begin{macrocode}
\protected\long\def\ekvc@SetupHashKeys#1%
  {%
    \let\ekvc@any@long\ekv@empty
    \let\ekvc@initials\ekv@empty
    \ekvparse\ekvc@err@value@required\ekvc@SetupHashKeys@a{#1}%
  }
\protected\def\ekvc@SetupHashKeys@a#1%
  {%
    \let\ekvc@long\ekv@empty
    \ekvc@ifspace{#1}%
      {\ekvc@SetupHashKeys@b#1\ekvc@stop}%
      {\ekvc@SetupHashKeys@c{#1}}%
  }
\protected\def\ekvc@SetupHashKeys@b#1 #2\ekvc@stop
  {%
    \ekv@ifdefined{ekvc@hash@p@#1}%
      {\csname ekvc@hash@p@#1\endcsname{#2}}%
      {\ekvc@SetupHashKeys@c{#1 #2}}%
  }
%    \end{macrocode}
%   Yes, even the defining macro looks awfully familiar. Instead of numbered we
%   have named marks. Still the key macros grab everything up to their
%   respective mark and reorder the arguments. The same quirk is applied for
%   short keys. And instead of the |\ekvc@setup@splitmacro| we use
%   |\ekvc@setup@hashmacro|.
%    \begin{macrocode}
\protected\long\def\ekvc@SetupHashKeys@c#1#2%
  {%
    \begingroup
    \edef\ekvc@tmp
      {%
        \endgroup
        \long\def\unexpanded{\ekvc@tmp}####1####2%
            \unexpanded\expandafter{\csname ekvc@hashmark@#1\endcsname}####3%
          {%
            ####2%
            \unexpanded\expandafter{\csname ekvc@hashmark@#1\endcsname}{####1}%
          }%
        \unless\ifx\ekvc@long\long
          \let\unexpanded\expandafter
            {\csname ekvc@\ekvc@set(#1)\endcsname\ekvc@tmp}%
          \def\unexpanded{\ekvc@tmp}####1%
            {%
              \unexpanded\expandafter{\csname ekvc@\ekvc@set(#1)\endcsname}%
                {####1}%
            }%
        \fi
        \def\unexpanded{\ekvc@initials}%
          {%
            \unexpanded\expandafter{\ekvc@initials}%
            \unexpanded\expandafter{\csname ekvc@hashmark@#1\endcsname{#2}}%
          }%
      }%
    \ekvc@tmp
    \ekvlet\ekvc@set{#1}\ekvc@tmp
    \ekvc@setup@hashmacro{#1}%
  }
%    \end{macrocode}
% \end{macro}
%
% \begin{macro}[internal]{\ekvc@hash@p@long}
%   Nothing astonishing here either.
%    \begin{macrocode}
\protected\def\ekvc@hash@p@long
  {%
    \let\ekvc@long\long
    \let\ekvc@any@long\long
    \ekvc@SetupHashKeys@c
  }
%    \end{macrocode}
% \end{macro}
%
% \begin{macro}[internal]{\ekvc@setup@hashmacro}
%   The safe hash macros will be executed inside of an |\unexpanded| expansion
%   context, so they have to insert braces for that once they are done.
%   Most of the tests which have to be executed will already be done, but we
%   have to play safe if the hash doesn't show up in the hash list. Therefore we
%   use some |\ekvc@mark|s and |\ekvc@stop| to throw errors if the hash isn't
%   found in the right place. The fast variants have an easier life and just
%   return the correct value.
%    \begin{macrocode}
\protected\def\ekvc@setup@hashmacro#1%
  {%
    \ekv@ifdefined{ekvc@fasthash@#1}{}%
      {%
        \begingroup
          \edef\ekvc@tmp
            {%
              \long\gdef
                \unexpanded\expandafter{\csname ekvc@fasthash@#1\endcsname}%
                  ####1%
                  \unexpanded\expandafter{\csname ekvc@hashmark@#1\endcsname}%
                  ####2####3\unexpanded{\ekvc@stop}%
                {####2}%
              \long\gdef
                \unexpanded\expandafter{\csname ekvc@safehash@#1\endcsname}%
                  ####1%
                {%
                  \unexpanded\expandafter{\csname ekvc@@safehash@#1\endcsname}%
                  ####1\unexpanded{\ekvc@mark}{}%
                  \unexpanded\expandafter
                    {%
                      \csname ekvc@hashmark@#1\endcsname{}%
                      \ekvc@mark{\ekvc@err@missing@hash{#1}}\ekvc@stop
                    }%
                }%
              \long\gdef
                \unexpanded\expandafter{\csname ekvc@@safehash@#1\endcsname}%
                  ####1%
                  \unexpanded\expandafter{\csname ekvc@hashmark@#1\endcsname}%
                  ####2####3\unexpanded{\ekvc@mark}####4####5%
                  \unexpanded{\ekvc@stop}%
                {%
                  ####4{####2}%
                }%
              \long\gdef\unexpanded\expandafter
                {\csname ekvc@fastsplithash@#1\endcsname}%
                  ####1%
                  \unexpanded\expandafter{\csname ekvc@hashmark@#1\endcsname}%
                  ####2####3\unexpanded{\ekvc@stop}####4%
                {%
                  ####4{####1####3}{####2}%
                }%
              \long\gdef\unexpanded\expandafter
                {\csname ekvc@safesplithash@#1\endcsname}####1%
                {%
                  \unexpanded\expandafter
                    {\csname ekvc@@safesplithash@#1\endcsname}%
                  ####1\unexpanded{\ekvc@mark\ekvc@safe@found@hash}%
                  \unexpanded\expandafter
                    {%
                      \csname ekvc@hashmark@#1\endcsname{}%
                      \ekvc@mark{\ekvc@err@missing@hash{#1}\ekvc@safe@no@hash}%
                      \ekvc@stop
                    }%
                }%
              \long\gdef\unexpanded\expandafter
                {\csname ekvc@@safesplithash@#1\endcsname}%
                  ####1%
                  \unexpanded\expandafter{\csname ekvc@hashmark@#1\endcsname}%
                  ####2####3\unexpanded{\ekvc@mark}####4####5%
                  \unexpanded{\ekvc@stop}%
                {%
                  ####4{####2}####1####3\unexpanded{\ekvc@stop}%
                }%
            }%
          \ekvc@tmp
        \endgroup
      }%
  }
%    \end{macrocode}
% \end{macro}
%
% \begin{macro}{\ekvcValue}
%   All this does is a few consistency checks on the first argument (not empty,
%   hash macro exists) and then call that hash-grabbing macro that will also
%   test whether the hash is inside of |#2| or not.
%    \begin{macrocode}
\long\def\ekvcValue#1%
  {%
    \unexpanded
    \ekv@ifdefined{ekvc@safehash@#1}%
      {\csname ekvc@safehash@#1\endcsname}%
      {\ekvc@err@unknown@hash{#1}\@firstoftwo{{}}}%
  }
%    \end{macrocode}
% \end{macro}
%
% \begin{macro}{\ekvcValueFast}
%   To be as fast as possible, this doesn't test for anything, assuming the user
%   knows best.
%    \begin{macrocode}
\long\def\ekvcValueFast#1#2{\csname ekvc@fasthash@#1\endcsname#2\ekvc@stop}
%    \end{macrocode}
% \end{macro}
%
% \begin{macro}{\ekvcValueSplit,\ekvcValueSplit@recover}
%   This splits off a single value.
%    \begin{macrocode}
\long\def\ekvcValueSplit#1%
  {%
    \ekv@ifdefined{ekvc@safesplithash@#1}%
      {\csname ekvc@safesplithash@#1\endcsname}%
      {\ekvc@err@unknown@hash{#1}\ekvcValueSplit@recover}%
  }
\long\def\ekvcValueSplit@recover#1#2%
  {%
    #2{#1}{}%
  }
%    \end{macrocode}
% \end{macro}
%
% \begin{macro}[internal]{\ekvc@safe@found@hash, \ekvc@safe@no@hash}
%    \begin{macrocode}
\long\def\ekvc@safe@found@hash#1#2\ekvc@stop#3%
  {%
    #3{#2}{#1}%
  }
\long\def\ekvc@safe@no@hash#1#2\ekvc@mark\ekvc@safe@found@hash\ekvc@stop#3%
  {%
    #3{#2}{}%
  }
%    \end{macrocode}
% \end{macro}
%
% \begin{macro}{\ekvcValueSplitFast}
%   Again a fast approach which doesn't provide too many safety measurements.
%   This needs to build the hash function and expand it before passing the
%   results to the next control sequence. The first step only builds the control
%   sequence.
%    \begin{macrocode}
\long\def\ekvcValueSplitFast#1#2%
  {%
    \csname ekvc@fastsplithash@#1\endcsname#2\ekvc@stop
  }
%    \end{macrocode}
% \end{macro}
%
% \begin{macro}[internal]
%   {
%     \ekvc@safehash@,\ekvc@fasthash@,
%     \ekvc@safesplithash@,\ekvc@fastsplithash@
%   }
%   At least in the empty hash case we can provide a meaningful error message
%   without affecting performance by just defining the macro that would be build
%   in that case. There is of course a downside to this, the error will not be
%   thrown by |\ekvcValueFast| in three expansion steps. The safe hash variant
%   has to also stop the |\unexpanded| expansion.
%    \begin{macrocode}
\long\def\ekvc@safehash@#1{\ekvc@err@empty@hash{}}
\long\def\ekvc@fasthash@#1\ekvc@stop{\ekvc@err@empty@hash}
\long\def\ekvc@safesplithash@#1#2{\ekvc@err@empty@hash#2{#1}{}}
\long\def\ekvc@fastsplithash@#1\ekvc@stop#2{\ekvc@err@empty@hash#2{#1}{}}
%    \end{macrocode}
% \end{macro}
%
% \begin{macro}{\ekvcSecondaryKeys}
%   The secondary keys are defined pretty similar to the way the originals are,
%   but here we also introduce some key types (those have a |@t@| in their name)
%   additionally to the prefixes.
%    \begin{macrocode}
\protected\long\def\ekvcSecondaryKeys#1#2%
  {%
    \edef\ekvc@set{\string#1}%
    \ekvparse\ekvc@err@value@required\ekvcSecondaryKeys@a{#2}%
  }
\protected\def\ekvcSecondaryKeys@a#1%
  {%
    \let\ekvc@long\ekv@empty
    \ekvc@ifspace{#1}%
      {\ekvcSecondaryKeys@b#1\ekvc@stop}%
      {\ekvc@err@missing@type{#1}\@gobble}%
  }
\protected\def\ekvcSecondaryKeys@b#1 #2\ekvc@stop
  {%
    \ekv@ifdefined{ekvc@p@#1}%
      {\csname ekvc@p@#1\endcsname}%
      {%
        \ekv@ifdefined{ekvc@t@#1}%
          {\csname ekvc@t@#1\endcsname}%
          {\ekvc@err@unknown@keytype{#1}\@firstoftwo\@gobble}%
      }%
      {#2}%
  }
%    \end{macrocode}
% \end{macro}
%
% \begin{macro}{\ekvcChange}
%   This can be used to change the defaults of an \expkvc\ defined macro. It
%   checks whether there is a set with the correct name and that the macro is
%   defined. If both is true the real work is done by |\ekvc@change|.
%    \begin{macrocode}
\protected\long\def\ekvcChange#1%
  {%
    \ekvifdefinedset{\string#1}%
      {%
        \ekvc@ifdefined#1%
          {\ekvc@change#1}%
          {\ekvc@err@no@key@macro#1\@gobble}%
      }%
      {\ekvc@err@no@key@macro#1\@gobble}%
  }
%    \end{macrocode}
% \end{macro}
%
% \begin{macro}[internal]
%   {\ekvc@change,\ekvc@change@a,\ekvc@change@b,\ekvc@change@c}
%   First we need to see whether the macro is currently |\long|. For this we get
%   the meaning and will parse it. |#1| is the macro name in which we want to
%   change the defaults.
%    \begin{macrocode}
\protected\def\ekvc@change#1%
  {\expandafter\ekvc@change@a\meaning#1\ekv@stop#1}
%    \end{macrocode}
%   A temporary definition to get the stringified |macro:|. |##1| will be the
%   list of prefixes, we don't care for the exact contents of |##2| and |##3|.
%    \begin{macrocode}
\def\ekvc@change@a#1%
  {%
    \protected\def\ekvc@change@a##1#1##2->##3\ekv@stop
      {%
        \ekvc@change@iflong{##1}%
          {\ekvc@change@b{}}%
          {\ekvc@change@b{\long}}%
      }%
  }
\expandafter\ekvc@change@a\expandafter{\detokenize{macro:}}
%    \end{macrocode}
%   Next we expand the macro once to get its contents (including the current
%   default values with their markers). |#1| is either |\long| or empty, |#2| is
%   the macro.
%    \begin{macrocode}
\protected\def\ekvc@change@b#1#2%
  {\expandafter\ekvc@change@c\expandafter{#2{##1}}{#1}#2}
%    \end{macrocode}
%   Here we place an unbalanced closing brace after the expansion of the macro.
%   Then we just parse the \kv-list with |\ekvset|, that will exchange the
%   values behind the markers. Once those are changed we give control to
%   |\ekvc@change@d|. The |\ekvset| step might horribly fail if the user defined
%   some keys that don't behave nice. |#1| is the expansion of the macro, |#2|
%   is either |\long| or empty, |#3| is the macro, and |#4| is the \kv-list
%   containing the new defaults.
%    \begin{macrocode}
\ekv@exparg{\protected\long\def\ekvc@change@c#1#2#3#4}%
  {%
    \expandafter\iffalse\expandafter{\expandafter\fi
      \ekvset{\string#3}{#4}%
      \ekvc@change@d{#2}{#3}%
      #1%
    }%
  }
%    \end{macrocode}
%   The final step needs to put an unbalanced opening brace after |\def|. We do
%   that with the help of a temporary macro which stores everything necessary
%   for |\def| and expand an |\iffalse}\fi| construct to leave a single
%   opening brace. |#1| will be either empty or |\long| and |#2| is the macro.
%   Each of the macros defined with \expkvc\ takes exactly one parameter, so we
%   put that here as |####1|.
%    \begin{macrocode}
\protected\def\ekvc@change@d#1#2%
  {%
    \def\ekvc@tmp{#1\def#2####1}%
    \expandafter\ekvc@tmp\expandafter{\iffalse}\fi
  }
%    \end{macrocode}
% \end{macro}
%
% \begin{macro}[internal]{\ekvc@change@iflong,\ekvc@change@iflong@}
%   Checking whether a string contains the string representation of |\long| can
%   be done by gobbling everything up to the first |\long| and checking whether
%   the result is completely empty. We need a temporary macro to get the result
%   of |\string\long| inside the definitions.
%    \begin{macrocode}
\def\ekvc@change@iflong#1%
  {%
    \protected\def\ekvc@change@iflong##1%
      {\expandafter\ekv@ifempty\expandafter{\ekvc@change@iflong@##1#1}}%
    \def\ekvc@change@iflong@##1#1{}
  }
\expandafter\ekvc@change@iflong\expandafter{\string\long}
%    \end{macrocode}
% \end{macro}
%
% \begin{macro}{\ekvcPass}
%   This macro can be used to pass a value to a key of some macro (this way
%   more complicated key codes are possible that in the end pass processed
%   values on to some macro). The implemantation is pretty straight forward.
%    \begin{macrocode}
\long\def\ekvcPass#1#2%
  {%
    \ekvifdefined{\string#1}{#2}%
      {\csname\ekv@name{\string#1}{#2}\endcsname}%
      {\ekvc@err@unknown@key@or@macro{#1}{#2}\@gobble}%
  }
%    \end{macrocode}
% \end{macro}
%
%
% \subsubsection{Secondary Key Types}
%
% \begin{macro}[internal]{\ekvc@p@long,\ekvc@after@ptype}
%   The prefixes are pretty straight forward again. Just set |\ekvc@long| and
%   forward to the |@t@| type.
%    \begin{macrocode}
\protected\def\ekvc@p@long#1%
  {%
    \ekvc@ifspace{#1}%
      {%
        \let\ekvc@long\long
        \ekvc@after@ptype#1\ekvc@stop
      }%
      {\ekvc@err@missing@type{long #1}\@gobble}%
  }
\protected\def\ekvc@after@ptype#1 #2\ekvc@stop
  {%
    \ekv@ifdefined{ekvc@t@#1}%
      {\csname ekvc@t@#1\endcsname{#2}}%
      {\ekvc@err@unknown@keytype{#1}\@gobble}%
  }
%    \end{macrocode}
% \end{macro}
%
% \begin{macro}[internal]
%   {
%     \ekvc@t@meta, \ekvc@t@nmeta,
%     \ekvc@type@meta, \ekvc@type@meta@a, \ekvc@type@meta@b
%   }
%   The |meta| and |nmeta| key types use a nested |\ekvset| to set other keys in
%   the same macro's \set.
%    \begin{macrocode}
\protected\def\ekvc@t@meta
  {%
    \edef\ekvc@tmp{\ekvc@set}%
    \expandafter\ekvc@type@meta\expandafter{\ekvc@tmp}\ekvc@long{##1}\ekvlet
  }
\protected\def\ekvc@t@nmeta#1%
  {%
    \ekvc@assert@not@long{nmeta #1}%
    \edef\ekvc@tmp{\ekvc@set}%
    \expandafter\ekvc@type@meta\expandafter{\ekvc@tmp}{}{}\ekvletNoVal{#1}%
  }
\protected\long\def\ekvc@type@meta#1#2#3#4#5#6%
  {%
    \expandafter\ekvc@type@meta@a\expandafter{\ekvset{#1}{#6}}{#2}{#3}%
    #4\ekvc@set{#5}\ekvc@tmp
  }
\protected\def\ekvc@type@meta@a
  {%
    \expandafter\ekvc@type@meta@b\expandafter
  }
\protected\long\def\ekvc@type@meta@b#1#2#3%
  {%
    #2\def\ekvc@tmp#3{#1}%
  }
%    \end{macrocode}
% \end{macro}
%
% \begin{macro}[internal]{\ekvc@t@alias}
%   |alias| just checks whether there is a key and/or |NoVal| key defined with
%   the target name and |\let| the key to those.
%    \begin{macrocode}
\protected\def\ekvc@t@alias#1#2%
  {%
    \ekvc@assert@not@long{alias #1}%
    \let\ekvc@tmp\@firstofone
    \ekvifdefined\ekvc@set{#2}%
      {%
        \ekvletkv\ekvc@set{#1}\ekvc@set{#2}%
        \let\ekvc@tmp\@gobble
      }%
      {}%
    \ekvifdefinedNoVal\ekvc@set{#2}%
      {%
        \ekvletkvNoVal\ekvc@set{#1}\ekvc@set{#2}%
        \let\ekvc@tmp\@gobble
      }%
      {}%
    \ekvc@tmp{\ekvc@err@unknown@key{#2}}%
  }
%    \end{macrocode}
% \end{macro}
%
% \begin{macro}[internal]{\ekvc@t@default}
%   The |default| key can be used to set a |NoVal| key for an existing key. It
%   will just pass the \val\ to the key macro of that other key.
%    \begin{macrocode}
\protected\long\def\ekvc@t@default#1#2%
  {%
    \ekvifdefined\ekvc@set{#1}%
      {%
        \ekvc@assert@not@long{default #1}%
        \edef\ekvc@tmp
          {%
            \unexpanded\expandafter
              {\csname\ekv@name\ekvc@set{#1}\endcsname{#2}}%
          }%
        \ekvletNoVal\ekvc@set{#1}\ekvc@tmp
      }%
      {\ekvc@err@unknown@key{#1}}%
  }
%    \end{macrocode}
% \end{macro}
%
% \begin{macro}[internal]{\ekvc@t@flag-bool}
%    \begin{macrocode}
\protected\expandafter\def\csname ekvc@t@flag-bool\endcsname#1#2%
  {%
    \ekvc@assert@not@long{flag-bool #1}%
    \unless\ifdefined#2\ekvcFlagNew#2\fi
    \ekvdef\ekvc@set{#1}%
      {%
        \ekv@ifdefined{ekvc@flag@set@##1}%
          {%
            \csname ekvc@flag@set@##1\expandafter\endcsname
              \ekvcFlagHeight#2\ekv@stop#2%
          }%
          {\ekvc@err@invalid@bool{##1}}%
      }%
  }
%    \end{macrocode}
% \end{macro}
%
% \begin{macro}[internal]
%   {\ekvc@t@flag-true,\ekvc@t@flag-false,\ekvc@t@flag-raise,\ekvc@type@flag}
%    \begin{macrocode}
\protected\def\ekvc@type@flag#1#2#3#4%
  {%
    \ekvc@assert@not@long{flag-#1 #3}%
    \unless\ifdefined#4\ekvcFlagNew#4\fi
    \ekv@exparg{\ekvdefNoVal\ekvc@set{#3}}{#2#4}%
  }
\protected\expandafter\def\csname ekvc@t@flag-true\endcsname
  {\ekvc@type@flag{true}\ekvcFlagSetTrue}
\protected\expandafter\def\csname ekvc@t@flag-false\endcsname
  {\ekvc@type@flag{false}\ekvcFlagSetFalse}
\protected\expandafter\def\csname ekvc@t@flag-raise\endcsname
  {\ekvc@type@flag{raise}\ekvcFlagRaise}
%    \end{macrocode}
% \end{macro}
%
%
% \subsubsection{Flags}
%
% The basic idea of flags is to store information by the fact that \TeX\
% expandably assigns the meaning |\relax| to undefined control sequences which
% were built with |\csname|. This mechanism is borrowed from \pkg{expl3}.
%
% \begin{macro}[internal]{\ekvc@flag@name,\ekvc@flag@namescheme}
%   Flags follow a simple naming scheme which we define here. |\ekvc@flag@name|
%   will store the name of an internal function that is used to build names of
%   the second naming scheme defined by |\ekvc@flag@namescheme|.
%    \begin{macrocode}
\def\ekvc@flag@name{ekvcf\string}
\def\ekvc@flag@namescheme#1#2{ekvch#2#1}
%    \end{macrocode}
% \end{macro}
%
% \begin{macro}{\ekvcFlagHeight}
%   For semantic reasons we use |\number| with another name.
%    \begin{macrocode}
\let\ekvcFlagHeight\number
%    \end{macrocode}
% \end{macro}
%
% \begin{macro}{\ekvcFlagNew}
%   This macro defines a new flag. It stores the function build with the
%   |\ekvc@flag@name| naming scheme after the internal function
%   |\ekvc@flag@height| that'll determine the current flag height. It'll also
%   define the macro named via |\ekvc@flag@name| to build names according to
%   |\ekvc@flag@namescheme|.
%    \begin{macrocode}
\protected\def\ekvcFlagNew#1%
  {%
    \edef#1%
      {%
        \unexpanded{\ekvc@flag@height}%
          \unexpanded\expandafter{\csname\ekvc@flag@name#1\endcsname}%
      }%
    \ekv@expargtwice
      {\expandafter\def\csname\ekvc@flag@name#1\endcsname##1}%
      {\expandafter\ekvc@flag@namescheme\expandafter{\string#1}{##1}}%
  }
%    \end{macrocode}
% \end{macro}
%
% \begin{macro}[internal]{\ekvc@flag@height,\ekvc@flag@height@}
%   This macro gets the height of a flag by a simple loop. The first loop
%   iteration differs a bit from the following in that it doesn't have to get
%   the current iteration count. The space at the end of |\ekvc@flag@height|
%   ends the |\number| evaluation.
%    \begin{macrocode}
\def\ekvc@flag@height#1%
  {%
    \ifcsname#10\endcsname
      \ekvc@flag@height@1\ekv@stop#1%
    \fi
    \@firstofone{0} % leave this space
  }
\def\ekvc@flag@height@#1\ekv@stop#2\fi\@firstofone#3%
  {%
    \fi
    \ifcsname#2{#1}\endcsname
      \expandafter\ekvc@flag@height@\the\numexpr#1+1\relax\ekv@stop#2%
    \fi
    \@firstofone{#1}%
  }
%    \end{macrocode}
% \end{macro}
%
% \begin{macro}{\ekvcFlagRaise}
%   Raising a flag simply means letting the |\ekvc@flag@namescheme| macro for
%   the current height to relax. The result of raising a flag is that its height
%   is bigger by $1$.
%    \begin{macrocode}
\ekv@exparg{\def\ekvcFlagRaise#1}%
  {%
    \expandafter\expandafter\expandafter\@gobble\expandafter
      \csname\ekvc@flag@namescheme{\string#1}{\ekvcFlagHeight#1}\endcsname
  }
%    \end{macrocode}
% \end{macro}
%
% \begin{macro}{\ekvcFlagSetTrue,\ekvcFlagSetFalse}
% \begin{macro}[internal]{\ekvc@flag@set@true,\ekvc@flag@set@false}
%   A flag is considered true if its current height is odd, and as false if it
%   is even. Therefore |\ekvcFlagSetTrue| and |\ekvcFlagSetFalse| only need to
%   raise the flag if the opposing boolean value is the current one.
%    \begin{macrocode}
\def\ekvcFlagSetTrue#1%
  {\expandafter\ekvc@flag@set@true\ekvcFlagHeight#1\ekv@stop#1}
\def\ekvcFlagSetFalse#1%
  {\expandafter\ekvc@flag@set@false\ekvcFlagHeight#1\ekv@stop#1}
%    \end{macrocode}
%   We can expand |\ekvc@flag@namescheme| at definition time here, which is why
%   we're using a temporary definition to set up |\ekvc@flag@set@true| and
%   |\ekvc@flag@set@false|.
%    \begin{macrocode}
\def\ekvc@flag@set@true#1%
  {%
    \def\ekvc@flag@set@true##1\ekv@stop##2%
      {%
        \ifodd##1
          \ekv@fi@gobble
        \fi
        \@firstofone{\expandafter\@gobble\csname#1\endcsname}%
      }%
    \def\ekvc@flag@set@false##1\ekv@stop##2%
      {%
        \ifodd##1
          \ekv@fi@firstofone
        \fi
        \@gobble{\expandafter\@gobble\csname#1\endcsname}%
      }%
  }
\expandafter\ekvc@flag@set@true\expandafter
  {\ekvc@flag@namescheme{\string#2}{#1}}
%    \end{macrocode}
% \end{macro}
% \end{macro}
%
% \begin{macro}{\ekvcFlagIf}
%   As already explained, truthiness is defined as a flag's height is odd, so we
%   just branch accordingly here.
%    \begin{macrocode}
\def\ekvcFlagIf#1%
  {%
    \ifodd#1%
      \ekv@fi@firstoftwo
    \fi
    \@secondoftwo
  }
%    \end{macrocode}
% \end{macro}
%
% \begin{macro}{\ekvcFlagIfRaised}
%   This macro uses flags as a switch, if a flag's current height is bigger than
%   $0$ this test yields true.
%    \begin{macrocode}
\ekv@exparg{\def\ekvcFlagIfRaised#1}%
  {%
    \expandafter\ifcsname\ekvc@flag@namescheme{\string#1}0\endcsname
      \ekv@fi@firstoftwo
    \fi
    \@secondoftwo
  }
%    \end{macrocode}
% \end{macro}
%
% \begin{macro}{\ekvcFlagReset}
% \begin{macro}[internal]{\ekvc@flag@reset,\ekvc@flag@reset@}
%   Resetting works by locally letting all the defined internal macros named
%   after |\ekvc@flag@namescheme| to undefined.
%    \begin{macrocode}
\protected\def\ekvcFlagReset#1%
  {\expandafter\ekvc@flag@reset\csname\ekvc@flag@name#1\endcsname}
\protected\def\ekvc@flag@reset#1%
  {%
    \ifcsname#10\endcsname
      \expandafter\let\csname#10\endcsname\ekvc@undefined
      \ekvc@flag@reset@1\ekv@stop#1%
    \fi
  }
\protected\def\ekvc@flag@reset@#1\ekv@stop#2\fi
  {%
    \fi
    \ifcsname#2{#1}\endcsname
      \expandafter\let\csname#2{#1}\endcsname\ekvc@undefined
      \expandafter\ekvc@flag@reset@\the\numexpr#1+1\relax\ekv@stop#2%
    \fi
  }
%    \end{macrocode}
% \end{macro}
% \end{macro}
%
% \begin{macro}{\ekvcFlagGetHeight}
% \begin{macro}[internal]{\ekvc@flag@get@height@single}
%   These are just small helpers, first getting the height of the flag and then
%   passing it on to the user supplied code.
%    \begin{macrocode}
\def\ekvcFlagGetHeight#1%
  {\expandafter\ekvc@flag@get@height@single\ekvcFlagHeight#1\ekv@stop}
\long\def\ekvc@flag@get@height@single#1\ekv@stop#2{#2{#1}}
%    \end{macrocode}
% \end{macro}
% \end{macro}
%
% \begin{macro}{\ekvcFlagGetHeights}
% \begin{macro}[internal]
%   {\ekvc@flag@get@heights,\ekvc@flag@get@heights@,\ekvc@flag@get@heights@done}
%   This works by a simple loop that stops at |\ekv@stop|. As long as that
%   marker isn't hit, get the next flags height and put it into a list after
%   |\ekv@stop|. |\ekvc@flag@get@heights@| uses the same marker name for the
%   end of the height, which shouldn't clash in any case. Once we're done we
%   remove the remainder of the current iteration and leave the user supplied
%   code in the input stream with all the flags' heights as a single argument.
%    \begin{macrocode}
\def\ekvcFlagGetHeights#1%
  {%
    \ekvc@flag@get@heights#1\ekv@stop{}%
  }
\def\ekvc@flag@get@heights#1%
  {%
    \ekv@gobbleto@stop#1\ekvc@flag@get@heights@done\ekv@stop
    \expandafter\ekvc@flag@get@heights@\ekvcFlagHeight#1\ekv@stop
  }
\def\ekvc@flag@get@heights@#1\ekv@stop#2\ekv@stop#3%
  {\ekvc@flag@get@heights#2\ekv@stop{#3{#1}}}
\long\def\ekvc@flag@get@heights@done
    \ekv@stop
    \expandafter\ekvc@flag@get@heights@\ekvcFlagHeight\ekv@stop\ekv@stop#1#2%
  {#2{#1}}
%    \end{macrocode}
% \end{macro}
% \end{macro}
%
%
% \subsubsection{Helper Macros}
%
% \begin{macro}[internal]{\ekvc@ifspace,\ekvc@ifspace@}
% A test which can be reduced to an if-empty by gobbling everything up to the
% first space.
%    \begin{macrocode}
\long\def\ekvc@ifspace#1%
  {%
    \ekvc@ifspace@#1 \ekv@ifempty@B
      \ekv@ifempty@false\ekv@ifempty@A\ekv@ifempty@B\@firstoftwo
  }
\long\def\ekvc@ifspace@#1 % keep this space
  {%
    \ekv@ifempty@\ekv@ifempty@A
  }
%    \end{macrocode}
% \end{macro}
%
%
% \subsubsection{Assertions}
% \begin{macro}[internal]{\ekvc@assert@not@long}
%   Some keys don't want to be |long| and we have to educate the user, so let's
%   throw an error if someone wanted these to be long.
%    \begin{macrocode}
\long\def\ekvc@assert@not@long#1{\ifx\ekvc@long\long\ekvc@err@no@long{#1}\fi}
%    \end{macrocode}
% \end{macro}
%
%
% \subsubsection{Messages}
%
% \begin{macro}[internal]
%   {
%     \ekvc@err@toomany,\ekvc@err@value@required,\ekvc@err@missing@type,
%     \ekvc@err@already@defined,\ekvc@err@no@key@macro
%   }
%   Boring unexpandable error messages.
%    \begin{macrocode}
\protected\def\ekvc@err@toomany#1%
  {%
    \errmessage{expkv-cs Error: Too many keys for macro `\string#1'}%
  }
\protected\def\ekvc@err@value@required#1%
  {%
    \errmessage{expkv-cs Error: Missing value for key `\unexpanded{#1}'}%
  }
\protected\def\ekvc@err@missing@type#1%
  {%
    \errmessage
      {expkv-cs Error: Missing type for secondary key `\unexpanded{#1}'}%
  }
\protected\def\ekvc@err@no@long#1%
  {%
    \errmessage
      {expkv-cs Error: prefix `long' not accepted for `\unexpanded{#1}'}%
  }
\protected\def\ekvc@err@already@defined#1%
  {%
    \errmessage{expkv-cs Error: Macro `\string#1' already defined}%
  }
\protected\def\ekvc@err@unknown@keytype#1%
  {%
    \errmessage{expkv-cs Error: Unknown key type `\unexpanded{#1}'}%
  }
\protected\def\ekvc@err@unknown@key#1%
  {%
    \errmessage
      {expkv-cs Error: Unknown key `\unexpanded{#1}' for macro `\ekvc@set'}%
  }
\long\def\ekvc@err@no@key@macro#1%
  {\errmessage{expkv-cs Error: \string#1 is no key=val macro}}
%    \end{macrocode}
% \end{macro}
%
% \begin{macro}[internal]{\ekvc@err}
%   We need a way to throw error messages expandably in some contexts.
%    \begin{macrocode}
\ekv@exparg{\long\def\ekvc@err#1}{\ekverr{expkv-cs}{#1}}
%    \end{macrocode}
% \end{macro}
%
% \begin{macro}[internal]
%   {
%     \ekvc@err@unknown@hash,\ekvc@err@empty@hash,\ekvc@err@missing@hash,
%     \ekvc@err@invalid@bool
%   }
%   And here are the expandable error messages.
%    \begin{macrocode}
\long\def\ekvc@err@unknown@hash#1{\ekvc@err{unknown hash `#1'}}
\long\def\ekvc@err@missing@hash#1{\ekvc@err{hash `#1' not found}}
\long\def\ekvc@err@empty@hash{\ekvc@err{empty hash}}
\def\ekvc@err@invalid@bool#1{\ekvc@err{invalid boolean value `#1'}}
\long\def\ekvc@err@unknown@key@or@macro#1#2%
  {\ekvc@err{unknown key `#2' for macro #1}}
%    \end{macrocode}
% \end{macro}
%
%
% Now everything that's left is to reset the category code of |@|.
%    \begin{macrocode}
\catcode`\@=\ekvc@tmp
%    \end{macrocode}
%
% \gobbledocstriptag
%</tex>
%^^A=<<
%
%^^A=<<
%
% \end{implementation}^^A=<<
%
% \clearpage
% \PrintIndex
%
\endinput
%
^^A vim: ft=tex fdm=marker fmr=>>=,=<<

\ProvidesPackage{expkv-cs}%
  [%
    \ekvcDate\space v\ekvcVersion\space
    define expandable key=val macros using expkv%
  ]
%    \end{macrocode}
% \gobbledocstriptag
%</pkg>
%^^A=<<
%^^A main file >>=
% \subsection{The Generic Code}
% The rest of this implementation will be the generic code.
% \gobbledocstriptag
%<*tex>
%
% Load \expkv\ if the package didn't already do so -- since \expkv\ has
% safeguards against being loaded twice this does no harm and the overhead
% isn't that big. Also we reuse some of the internals of \expkv\ to save us from
% retyping them.
%    \begin{macrocode}
\input expkv
%    \end{macrocode}
%
% We make sure that \file{expkv-cs.tex} is only input once:
%    \begin{macrocode}
\expandafter\ifx\csname ekvcVersion\endcsname\relax
\else
  \expandafter\endinput
\fi
%    \end{macrocode}
%
% \begin{macro}{\ekvcVersion,\ekvcDate}
% We're on our first input, so lets store the version and date in a macro.
%    \begin{macrocode}
\def\ekvcVersion{0.10}
\def\ekvcDate{2021-06-03}
%    \end{macrocode}
% \end{macro}
%
% If the \LaTeX\ format is loaded we want to be a good file and report back who
% we are, for this the package will have defined |\ekvc@tmp| to use
% |\ProvidesFile|, else this will expand to a |\relax| and do no harm.
%    \begin{macrocode}
\csname ekvc@tmp\endcsname
%    \end{macrocode}
%
% Store the category code of |@| to later be able to reset it and change it to
% 11 for now.
%    \begin{macrocode}
\expandafter\chardef\csname ekvc@tmp\endcsname=\catcode`\@
\catcode`\@=11
%    \end{macrocode}
% |\ekvc@tmp| will be reused later, but we don't need it to ever store
% information long-term after \expkvc\ was initialized.
%
% \begin{macro}[internal]{\ekvc@keycount}
%   We'll need to keep count how many keys must be defined for each macro in the
%   |split| variants.
%    \begin{macrocode}
\newcount\ekvc@keycount
%    \end{macrocode}
% \end{macro}
%
% \begin{macro}[internal]{\ekvc@long,\ekvc@any@long}
%   Some macros will have to be defined long. These two will be let to |\long|
%   when this should be the case.
%    \begin{macrocode}
\let\ekvc@long\ekv@empty
\let\ekvc@any@long\ekv@empty
%    \end{macrocode}
% \end{macro}
%
% \begin{macro}[internal]{\ekvc@ifdefined}
%   We want to test whether a macro is already defined. This test checks for a
%   defined macro that isn't |\relax|.
%    \begin{macrocode}
\long\def\ekvc@ifdefined#1%
  {%
    \ifdefined#1%
      \ifx\relax#1%
        \ekv@fi@gobble
      \fi
      \@firstofone
      \ekv@fi@firstoftwo
    \fi
    \@secondoftwo
  }
%    \end{macrocode}
% \end{macro}
%
% \begin{macro}[internal]
%   {
%     \ekvc@ekvset@pre@expander,
%     \ekvc@ekvset@pre@expander@a,
%     \ekvc@ekvset@pre@expander@b
%   }
%   This macro expands |\ekvset| twice so that the first two steps of expansion
%   don't have to be made every time the \expkvc\ macros are used. We have to do
%   a little magic trick to get the macro parameter |#1| for the macro
%   definition this is used in, even though we're calling |\unexpanded|. We do
%   that by splitting the expanded |\ekvset| at some marks and place |##1| in
%   between. At this spot we also add |\ekv@alignsafe| and |\ekv@endalignsafe|
%   to ensure that macros created with \expkvc\ are alignment safe.
%    \begin{macrocode}
\def\ekvc@ekvset@pre@expander#1%
  {%
    \expandafter\ekvc@ekvset@pre@expander@a\ekvset{#1}\ekvc@stop\ekvc@stop
  }
\def\ekvc@ekvset@pre@expander@a
  {%
    \expandafter\ekvc@ekvset@pre@expander@b
  }
\def\ekvc@ekvset@pre@expander@b#1\ekvc@stop#2\ekvc@stop
  {%
    \unexpanded\expandafter{\ekv@alignsafe}%
    \unexpanded{#1}##1\unexpanded{#2}%
    \unexpanded\expandafter{\ekv@endalignsafe}%
  }
%    \end{macrocode}
% \end{macro}
%
% \begin{macro}{\ekvcSplitAndUse}
%   The first user macro we want to set up can be reused for
%   \cs[no-index]{ekvcSplitAndForward} and \cs[no-index]{ekvcSplit}. We'll split
%   this one up so that the test whether the macro is already defined doesn't
%   run twice.
%    \begin{macrocode}
\protected\long\def\ekvcSplitAndUse#1#2%
  {%
    \let\ekvc@helpers@needed\@firstoftwo
    \ekvc@ifdefined#1%
      {\ekvc@err@already@defined#1}%
      {\ekvcSplitAndUse@#1{}{#2}}%
  }
%    \end{macrocode}
% \end{macro}
%
% \begin{macro}[internal]{\ekvcSplitAndUse@}
%   The actual macro setting up things. We need to set some variables, forward
%   the key list to |\ekvc@SetupSplitKeys|, and afterwards define the front
%   facing macro to call |\ekvset| and put the initials and the argument sorting
%   macro behind it. The internals |\ekvc@any@long|, |\ekvc@initials| and
%   |\ekvc@keycount| will be set correctly by |\ekvc@SetupSplitKeys|.
%    \begin{macrocode}
\protected\long\def\ekvcSplitAndUse@#1#2#3%
  {%
    \edef\ekvc@set{\string#1}%
    \ekvc@SetupSplitKeys{#3}%
    \ekvc@helpers@needed
      {%
        \ekvc@any@long\edef#1##1%
          {%
            \expandafter\ekvc@ekvset@pre@expander\expandafter{\ekvc@set}%
            \unexpanded\expandafter
              {\csname ekvc@split@\the\ekvc@keycount\endcsname}%
            \unexpanded\expandafter{\ekvc@initials{}#2}%
          }%
      }%
      {%
        \ekvc@any@long\edef#1##1%
          {%
            \expandafter\ekvc@ekvset@pre@expander\expandafter{\ekvc@set}%
            \unexpanded{#2}%
            \unexpanded\expandafter{\ekvc@initials}%
          }%
      }%
  }
%    \end{macrocode}
% \end{macro}
%
%
% \begin{macro}{\ekvcSplitAndForward}
%   This just reuses |\ekvcSplitAndUse@| with a non-empty second argument,
%   resulting in that argument to be called after the splitting.
%    \begin{macrocode}
\protected\long\def\ekvcSplitAndForward#1#2#3%
  {%
    \let\ekvc@helpers@needed\@firstoftwo
    \ekvc@ifdefined#1%
      {\ekvc@err@already@defined#1}%
      {\ekvcSplitAndUse@#1{{#2}}{#3}}%
  }
%    \end{macrocode}
% \end{macro}
%
% \begin{macro}{\ekvcSplit}
%   The first half is just |\ekvcSplitAndForward| then we define the macro to
%   which the parsed key list is forwarded. There we need to allow for up to
%   nine arguments.
%    \begin{macrocode}
\protected\long\def\ekvcSplit#1#2#3%
  {%
    \let\ekvc@helpers@needed\@secondoftwo
    \ekvc@ifdefined#1%
      {\ekvc@err@already@defined#1}%
      {%
        \expandafter
        \ekvcSplitAndUse@\expandafter#1\csname ekvc@\string#1\endcsname{#2}%
        \ifnum\ekvc@keycount<1
          \ekvc@any@long\expandafter\def\csname ekvc@\string#1\endcsname{#3}%
        \else
          \ifnum\ekvc@keycount>9
            \ekvc@err@toomany{#1}%
            \let#1\ekvc@undefined
          \else
            \ekvcSplit@build@argspec
            \ekvc@any@long\expandafter
            \def\csname ekvc@\string#1\expandafter\endcsname\ekvc@tmp{#3}%
          \fi
        \fi
      }%
  }
%    \end{macrocode}
% \end{macro}
%
% \begin{macro}[internal]{\ekvcSplit@build@argspec,\ekvcSplit@build@argspec@}
%    \begin{macrocode}
\protected\def\ekvcSplit@build@argspec
  {%
    \begingroup
      \edef\ekvc@tmp
        {\endgroup\def\unexpanded{\ekvc@tmp}{\ekvcSplit@build@argspec@{1}}}%
    \ekvc@tmp
  }
\def\ekvcSplit@build@argspec@#1%
  {%
    \ifnum#1>\ekvc@keycount
      \ekv@fi@gobble
    \fi
    \@firstofone
      {%
        \unexpanded\expandafter{\csname ekvc@splitmark@#1\endcsname####}#1%
        \expandafter\ekvcSplit@build@argspec@\expandafter{\the\numexpr#1+1}%
      }%
  }
%    \end{macrocode}
% \end{macro}
%
% \begin{macro}[internal]
%   {
%     \ekvc@SetupSplitKeys, \ekvc@SetupSplitKeys@a, \ekvc@SetupSplitKeys@b,
%     \ekvc@SetupSplitKeys@c
%   }
%   These macros parse the list of keys and set up the key macros. First we need
%   to initialise some macros and start |\ekvparse|.
%    \begin{macrocode}
\protected\long\def\ekvc@SetupSplitKeys
  {%
    \ekvc@keycount=0
    \let\ekvc@any@long\ekv@empty
    \let\ekvc@initials\ekv@empty
    \ekvparse\ekvc@err@value@required\ekvc@SetupSplitKeys@a
  }
%    \end{macrocode}
%   Then we need to step the key counter for each key. Also we have to check
%   whether this key has a |long| prefix so we initialise |\ekvc@long|.
%    \begin{macrocode}
\protected\def\ekvc@SetupSplitKeys@a#1%
  {%
    \advance\ekvc@keycount1
    \let\ekvc@long\ekv@empty
    \ekvc@ifspace{#1}%
      {\ekvc@SetupSplitKeys@b#1\ekvc@stop}%
      {\ekvc@SetupSplitKeys@c{#1}}%
  }
%    \end{macrocode}
%   If there was a space, there might be a prefix. If so call the prefix macro,
%   else call the next step |\ekvc@SetupSplitKeys@c| which will define the key
%   macro and add the key's value to the initials list.
%    \begin{macrocode}
\protected\def\ekvc@SetupSplitKeys@b#1 #2\ekvc@stop
  {%
    \ekv@ifdefined{ekvc@split@p@#1}%
      {\csname ekvc@split@p@#1\endcsname{#2}}%
      {\ekvc@SetupSplitKeys@c{#1 #2}}%
  }
%    \end{macrocode}
%   The inner definition is grouped, because we don't want to actually define
%   the marks we build with |\csname|. We have to append the value to the
%   |\ekvc@initials| list here with the correct split mark. The key macro will
%   read everything up to those split marks and change the value following it to
%   the value given to the key. Additionally we'll need a sorting macro for each
%   key count in use so we set it up with |\ekvc@setup@splitmacro|.
%    \begin{macrocode}
\protected\long\def\ekvc@SetupSplitKeys@c#1#2%
  {%
    \begingroup
    \edef\ekvc@tmp
      {%
        \endgroup
        \long\def\unexpanded{\ekvc@tmp}####1####2%
            \unexpanded\expandafter
            {\csname ekvc@splitmark@\the\ekvc@keycount\endcsname}####3%
          {%
            ####2%
            \unexpanded\expandafter
              {\csname ekvc@splitmark@\the\ekvc@keycount\endcsname}{####1}%
          }%
%    \end{macrocode}
%   The short variant needs a bit of special treatment. The key macro will be
%   short to throw the correct error, but since there might be long macros
%   somewhere the reordering of arguments needs to be long, so for short keys we
%   use a two step approach, first grabbing only the short argument, then
%   reordering.
%    \begin{macrocode}
        \unless\ifx\ekvc@long\long
          \let\unexpanded\expandafter
            {\csname ekvc@\ekvc@set(#1)\endcsname\ekvc@tmp}%
          \def\unexpanded{\ekvc@tmp}####1%
            {%
              \unexpanded\expandafter{\csname ekvc@\ekvc@set(#1)\endcsname}%
                {####1}%
            }%
        \fi
        \def\unexpanded{\ekvc@initials}%
          {%
            \unexpanded\expandafter{\ekvc@initials}%
            \unexpanded\expandafter
              {\csname ekvc@splitmark@\the\ekvc@keycount\endcsname{#2}}%
          }%
      }%
    \ekvc@tmp
    \ekvlet\ekvc@set{#1}\ekvc@tmp
    \ekvc@helpers@needed
      {\expandafter\ekvc@setup@splitmacro\expandafter{\the\ekvc@keycount}}%
      {}%
  }
%    \end{macrocode}
% \end{macro}
%
% \begin{macro}[internal]{\ekvc@split@p@long}
%   The |long| prefix lets the internals |\ekvc@long| and |\ekvc@any@long| to
%   |\long| so that the key macro will be long.
%    \begin{macrocode}
\protected\def\ekvc@split@p@long
  {%
    \let\ekvc@long\long
    \let\ekvc@any@long\long
    \ekvc@SetupSplitKeys@c
  }
%    \end{macrocode}
% \end{macro}
%
% \begin{macro}[internal]{\ekvc@defarggobbler}
%   This is needed to define a macro with 1-9 parameters programmatically.
%   \LaTeX's \cs[no-index]{newcommand} does something similar for example.
%    \begin{macrocode}
\protected\def\ekvc@defarggobbler#1{\def\ekvc@tmp##1#1##2##{##1#1}}
%    \end{macrocode}
% \end{macro}
%
% \begin{macro}[internal]
%   {
%     \ekvc@setup@splitmacro,
%     \ekvc@split@1, \ekvc@split@2, \ekvc@split@3,
%     \ekvc@split@4, \ekvc@split@5, \ekvc@split@6,
%     \ekvc@split@7
%   }
%   Since the first few split macros are different from the others we manually
%   set those up now. All the others will be defined as needed (always
%   globally).  The split macros just read up until the correct split mark, move
%   that argument into a list and reinsert the rest, calling the next split
%   macro afterwards.
%    \begin{macrocode}
\begingroup
\edef\ekvc@tmp
  {%
    \long\gdef\unexpanded\expandafter{\csname ekvc@split@1\endcsname}%
        \unexpanded\expandafter{\csname ekvc@splitmark@1\endcsname}%
        ##1##2##3%
      {##3{##1}##2}%
    \long\gdef\unexpanded\expandafter{\csname ekvc@split@2\endcsname}%
        \unexpanded\expandafter{\csname ekvc@splitmark@1\endcsname}##1%
        \unexpanded\expandafter{\csname ekvc@splitmark@2\endcsname}##2%
        ##3##4%
      {##4{##1}{##2}##3}%
    \long\gdef\unexpanded\expandafter{\csname ekvc@split@3\endcsname}%
        \unexpanded\expandafter{\csname ekvc@splitmark@1\endcsname}##1%
        \unexpanded\expandafter{\csname ekvc@splitmark@2\endcsname}##2%
        \unexpanded\expandafter{\csname ekvc@splitmark@3\endcsname}##3%
        ##4##5%
      {##5{##1}{##2}{##3}##4}%
    \long\gdef\unexpanded\expandafter{\csname ekvc@split@4\endcsname}%
        \unexpanded\expandafter{\csname ekvc@splitmark@1\endcsname}##1%
        \unexpanded\expandafter{\csname ekvc@splitmark@2\endcsname}##2%
        \unexpanded\expandafter{\csname ekvc@splitmark@3\endcsname}##3%
        \unexpanded\expandafter{\csname ekvc@splitmark@4\endcsname}##4%
        ##5##6%
      {##6{##1}{##2}{##3}{##4}##5}%
    \long\gdef\unexpanded\expandafter{\csname ekvc@split@5\endcsname}%
        \unexpanded\expandafter{\csname ekvc@splitmark@1\endcsname}##1%
        \unexpanded\expandafter{\csname ekvc@splitmark@2\endcsname}##2%
        \unexpanded\expandafter{\csname ekvc@splitmark@3\endcsname}##3%
        \unexpanded\expandafter{\csname ekvc@splitmark@4\endcsname}##4%
        \unexpanded\expandafter{\csname ekvc@splitmark@5\endcsname}##5%
        ##6##7%
      {##7{##1}{##2}{##3}{##4}{##5}##6}%
    \long\gdef\unexpanded\expandafter{\csname ekvc@split@6\endcsname}%
        \unexpanded\expandafter{\csname ekvc@splitmark@1\endcsname}##1%
        \unexpanded\expandafter{\csname ekvc@splitmark@2\endcsname}##2%
        \unexpanded\expandafter{\csname ekvc@splitmark@3\endcsname}##3%
        \unexpanded\expandafter{\csname ekvc@splitmark@4\endcsname}##4%
        \unexpanded\expandafter{\csname ekvc@splitmark@5\endcsname}##5%
        \unexpanded\expandafter{\csname ekvc@splitmark@6\endcsname}##6%
        ##7##8%
      {##8{##1}{##2}{##3}{##4}{##5}{##6}##7}%
    \long\gdef\unexpanded\expandafter{\csname ekvc@split@7\endcsname}%
        \unexpanded\expandafter{\csname ekvc@splitmark@1\endcsname}##1%
        \unexpanded\expandafter{\csname ekvc@splitmark@2\endcsname}##2%
        \unexpanded\expandafter{\csname ekvc@splitmark@3\endcsname}##3%
        \unexpanded\expandafter{\csname ekvc@splitmark@4\endcsname}##4%
        \unexpanded\expandafter{\csname ekvc@splitmark@5\endcsname}##5%
        \unexpanded\expandafter{\csname ekvc@splitmark@6\endcsname}##6%
        \unexpanded\expandafter{\csname ekvc@splitmark@7\endcsname}##7%
        ##8##9%
      {##9{##1}{##2}{##3}{##4}{##5}{##6}{##7}##8}%
  }
\ekvc@tmp
\endgroup
\protected\def\ekvc@setup@splitmacro#1%
  {%
    \ekv@ifdefined{ekvc@split@#1}{}%
      {%
        \begingroup
          \edef\ekvc@tmp
            {%
              \long\gdef
                  \unexpanded\expandafter{\csname ekvc@split@#1\endcsname}%
                  ####1%
                  \unexpanded\expandafter{\csname ekvc@splitmark@#1\endcsname}%
                  ####2####3%
                {%
                  \unexpanded\expandafter
                    {\csname ekvc@split@\the\numexpr#1-1\relax\endcsname}%
                  ####1{{####2}####3}%
                }%
            }%
          \ekvc@tmp
        \endgroup
      }%
  }
%    \end{macrocode}
% \end{macro}
%
% \begin{macro}{\ekvcHashAndUse}
%   |\ekvcHashAndUse| works just like |\ekvcSplitAndUse|.
%    \begin{macrocode}
\protected\long\def\ekvcHashAndUse#1#2%
  {%
    \let\ekvc@helpers@needed\@firstoftwo
    \ekvc@ifdefined#1%
      {\ekvc@err@already@defined#1}%
      {\ekvcHashAndUse@#1{}{#2}}%
  }
%    \end{macrocode}
% \end{macro}
%
% \begin{macro}[internal]{\ekvcHashAndUse@}
%   This is more or less the same as |\ekvcSplitAndUse@|. Instead of an empty
%   group we place a marker after the initials, we don't use the sorting macros
%   of |split|, but instead pack all the values in one argument.
%    \begin{macrocode}
\protected\long\def\ekvcHashAndUse@#1#2#3%
  {%
    \edef\ekvc@set{\string#1}%
    \ekvc@SetupHashKeys{#3}%
    \ekvc@helpers@needed
      {%
        \ekvc@any@long\edef#1##1%
          {%
            \expandafter\ekvc@ekvset@pre@expander\expandafter{\ekvc@set}%
            \unexpanded{\ekvc@hash@pack@argument}%
            \unexpanded\expandafter{\ekvc@initials\ekvc@stop#2}%
          }%
      }%
      {%
        \ekvc@any@long\edef#1##1%
          {%
            \expandafter\ekvc@ekvset@pre@expander\expandafter{\ekvc@set}%
            \unexpanded{#2}%
            \unexpanded\expandafter{\ekvc@initials\ekvc@stop}%
          }%
      }%
  }
%    \end{macrocode}
% \end{macro}
%
% \begin{macro}{\ekvcHashAndForward}
%   |\ekvcHashAndForward| works just like |\ekvcSplitAndForward|.
%    \begin{macrocode}
\protected\long\def\ekvcHashAndForward#1#2#3%
  {%
    \let\ekvc@helpers@needed\@firstoftwo
    \ekvc@ifdefined#1%
      {\ekvc@err@already@defined#1}%
      {\ekvcHashAndUse@#1{{#2}}{#3}}%
  }
%    \end{macrocode}
% \end{macro}
%
% \begin{macro}{\ekvcHash}
%   |\ekvcHash| does the same as |\ekvcSplit|, but has the advantage of not
%   needing to count arguments, so the definition of the internal macro is a bit
%   more straight forward.
%    \begin{macrocode}
\protected\long\def\ekvcHash#1#2#3%
  {%
    \let\ekvc@helpers@needed\@secondoftwo
    \ekvc@ifdefined#1%
      {\ekvc@err@already@defined#1}%
      {%
        \expandafter
        \ekvcHashAndUse@\expandafter#1\csname ekvc@\string#1\endcsname{#2}%
        \ekvc@any@long\expandafter\def\csname ekvc@\string#1\endcsname
            ##1\ekvc@stop
          {#3}%
      }%
  }
%    \end{macrocode}
% \end{macro}
%
% \begin{macro}[internal]{\ekvc@hash@pack@argument}
%   All this macro does is pack the values into one argument and forward that to
%   the next macro.
%    \begin{macrocode}
\long\def\ekvc@hash@pack@argument#1\ekvc@stop#2{#2{#1}}
%    \end{macrocode}
% \end{macro}
%
% \begin{macro}[internal]
%   {\ekvc@SetupHashKeys,\ekvc@SetupHashKeys@a,\ekvc@SetupHashKeys@b}
%   This should look awfully familiar as well, since it's just the same as for
%   the split keys with a few other names here and there.
%    \begin{macrocode}
\protected\long\def\ekvc@SetupHashKeys#1%
  {%
    \let\ekvc@any@long\ekv@empty
    \let\ekvc@initials\ekv@empty
    \ekvparse\ekvc@err@value@required\ekvc@SetupHashKeys@a{#1}%
  }
\protected\def\ekvc@SetupHashKeys@a#1%
  {%
    \let\ekvc@long\ekv@empty
    \ekvc@ifspace{#1}%
      {\ekvc@SetupHashKeys@b#1\ekvc@stop}%
      {\ekvc@SetupHashKeys@c{#1}}%
  }
\protected\def\ekvc@SetupHashKeys@b#1 #2\ekvc@stop
  {%
    \ekv@ifdefined{ekvc@hash@p@#1}%
      {\csname ekvc@hash@p@#1\endcsname{#2}}%
      {\ekvc@SetupHashKeys@c{#1 #2}}%
  }
%    \end{macrocode}
%   Yes, even the defining macro looks awfully familiar. Instead of numbered we
%   have named marks. Still the key macros grab everything up to their
%   respective mark and reorder the arguments. The same quirk is applied for
%   short keys. And instead of the |\ekvc@setup@splitmacro| we use
%   |\ekvc@setup@hashmacro|.
%    \begin{macrocode}
\protected\long\def\ekvc@SetupHashKeys@c#1#2%
  {%
    \begingroup
    \edef\ekvc@tmp
      {%
        \endgroup
        \long\def\unexpanded{\ekvc@tmp}####1####2%
            \unexpanded\expandafter{\csname ekvc@hashmark@#1\endcsname}####3%
          {%
            ####2%
            \unexpanded\expandafter{\csname ekvc@hashmark@#1\endcsname}{####1}%
          }%
        \unless\ifx\ekvc@long\long
          \let\unexpanded\expandafter
            {\csname ekvc@\ekvc@set(#1)\endcsname\ekvc@tmp}%
          \def\unexpanded{\ekvc@tmp}####1%
            {%
              \unexpanded\expandafter{\csname ekvc@\ekvc@set(#1)\endcsname}%
                {####1}%
            }%
        \fi
        \def\unexpanded{\ekvc@initials}%
          {%
            \unexpanded\expandafter{\ekvc@initials}%
            \unexpanded\expandafter{\csname ekvc@hashmark@#1\endcsname{#2}}%
          }%
      }%
    \ekvc@tmp
    \ekvlet\ekvc@set{#1}\ekvc@tmp
    \ekvc@setup@hashmacro{#1}%
  }
%    \end{macrocode}
% \end{macro}
%
% \begin{macro}[internal]{\ekvc@hash@p@long}
%   Nothing astonishing here either.
%    \begin{macrocode}
\protected\def\ekvc@hash@p@long
  {%
    \let\ekvc@long\long
    \let\ekvc@any@long\long
    \ekvc@SetupHashKeys@c
  }
%    \end{macrocode}
% \end{macro}
%
% \begin{macro}[internal]{\ekvc@setup@hashmacro}
%   The safe hash macros will be executed inside of an |\unexpanded| expansion
%   context, so they have to insert braces for that once they are done.
%   Most of the tests which have to be executed will already be done, but we
%   have to play safe if the hash doesn't show up in the hash list. Therefore we
%   use some |\ekvc@mark|s and |\ekvc@stop| to throw errors if the hash isn't
%   found in the right place. The fast variants have an easier life and just
%   return the correct value.
%    \begin{macrocode}
\protected\def\ekvc@setup@hashmacro#1%
  {%
    \ekv@ifdefined{ekvc@fasthash@#1}{}%
      {%
        \begingroup
          \edef\ekvc@tmp
            {%
              \long\gdef
                \unexpanded\expandafter{\csname ekvc@fasthash@#1\endcsname}%
                  ####1%
                  \unexpanded\expandafter{\csname ekvc@hashmark@#1\endcsname}%
                  ####2####3\unexpanded{\ekvc@stop}%
                {####2}%
              \long\gdef
                \unexpanded\expandafter{\csname ekvc@safehash@#1\endcsname}%
                  ####1%
                {%
                  \unexpanded\expandafter{\csname ekvc@@safehash@#1\endcsname}%
                  ####1\unexpanded{\ekvc@mark}{}%
                  \unexpanded\expandafter
                    {%
                      \csname ekvc@hashmark@#1\endcsname{}%
                      \ekvc@mark{\ekvc@err@missing@hash{#1}}\ekvc@stop
                    }%
                }%
              \long\gdef
                \unexpanded\expandafter{\csname ekvc@@safehash@#1\endcsname}%
                  ####1%
                  \unexpanded\expandafter{\csname ekvc@hashmark@#1\endcsname}%
                  ####2####3\unexpanded{\ekvc@mark}####4####5%
                  \unexpanded{\ekvc@stop}%
                {%
                  ####4{####2}%
                }%
              \long\gdef\unexpanded\expandafter
                {\csname ekvc@fastsplithash@#1\endcsname}%
                  ####1%
                  \unexpanded\expandafter{\csname ekvc@hashmark@#1\endcsname}%
                  ####2####3\unexpanded{\ekvc@stop}####4%
                {%
                  ####4{####1####3}{####2}%
                }%
              \long\gdef\unexpanded\expandafter
                {\csname ekvc@safesplithash@#1\endcsname}####1%
                {%
                  \unexpanded\expandafter
                    {\csname ekvc@@safesplithash@#1\endcsname}%
                  ####1\unexpanded{\ekvc@mark\ekvc@safe@found@hash}%
                  \unexpanded\expandafter
                    {%
                      \csname ekvc@hashmark@#1\endcsname{}%
                      \ekvc@mark{\ekvc@err@missing@hash{#1}\ekvc@safe@no@hash}%
                      \ekvc@stop
                    }%
                }%
              \long\gdef\unexpanded\expandafter
                {\csname ekvc@@safesplithash@#1\endcsname}%
                  ####1%
                  \unexpanded\expandafter{\csname ekvc@hashmark@#1\endcsname}%
                  ####2####3\unexpanded{\ekvc@mark}####4####5%
                  \unexpanded{\ekvc@stop}%
                {%
                  ####4{####2}####1####3\unexpanded{\ekvc@stop}%
                }%
            }%
          \ekvc@tmp
        \endgroup
      }%
  }
%    \end{macrocode}
% \end{macro}
%
% \begin{macro}{\ekvcValue}
%   All this does is a few consistency checks on the first argument (not empty,
%   hash macro exists) and then call that hash-grabbing macro that will also
%   test whether the hash is inside of |#2| or not.
%    \begin{macrocode}
\long\def\ekvcValue#1%
  {%
    \unexpanded
    \ekv@ifdefined{ekvc@safehash@#1}%
      {\csname ekvc@safehash@#1\endcsname}%
      {\ekvc@err@unknown@hash{#1}\@firstoftwo{{}}}%
  }
%    \end{macrocode}
% \end{macro}
%
% \begin{macro}{\ekvcValueFast}
%   To be as fast as possible, this doesn't test for anything, assuming the user
%   knows best.
%    \begin{macrocode}
\long\def\ekvcValueFast#1#2{\csname ekvc@fasthash@#1\endcsname#2\ekvc@stop}
%    \end{macrocode}
% \end{macro}
%
% \begin{macro}{\ekvcValueSplit,\ekvcValueSplit@recover}
%   This splits off a single value.
%    \begin{macrocode}
\long\def\ekvcValueSplit#1%
  {%
    \ekv@ifdefined{ekvc@safesplithash@#1}%
      {\csname ekvc@safesplithash@#1\endcsname}%
      {\ekvc@err@unknown@hash{#1}\ekvcValueSplit@recover}%
  }
\long\def\ekvcValueSplit@recover#1#2%
  {%
    #2{#1}{}%
  }
%    \end{macrocode}
% \end{macro}
%
% \begin{macro}[internal]{\ekvc@safe@found@hash, \ekvc@safe@no@hash}
%    \begin{macrocode}
\long\def\ekvc@safe@found@hash#1#2\ekvc@stop#3%
  {%
    #3{#2}{#1}%
  }
\long\def\ekvc@safe@no@hash#1#2\ekvc@mark\ekvc@safe@found@hash\ekvc@stop#3%
  {%
    #3{#2}{}%
  }
%    \end{macrocode}
% \end{macro}
%
% \begin{macro}{\ekvcValueSplitFast}
%   Again a fast approach which doesn't provide too many safety measurements.
%   This needs to build the hash function and expand it before passing the
%   results to the next control sequence. The first step only builds the control
%   sequence.
%    \begin{macrocode}
\long\def\ekvcValueSplitFast#1#2%
  {%
    \csname ekvc@fastsplithash@#1\endcsname#2\ekvc@stop
  }
%    \end{macrocode}
% \end{macro}
%
% \begin{macro}[internal]
%   {
%     \ekvc@safehash@,\ekvc@fasthash@,
%     \ekvc@safesplithash@,\ekvc@fastsplithash@
%   }
%   At least in the empty hash case we can provide a meaningful error message
%   without affecting performance by just defining the macro that would be build
%   in that case. There is of course a downside to this, the error will not be
%   thrown by |\ekvcValueFast| in three expansion steps. The safe hash variant
%   has to also stop the |\unexpanded| expansion.
%    \begin{macrocode}
\long\def\ekvc@safehash@#1{\ekvc@err@empty@hash{}}
\long\def\ekvc@fasthash@#1\ekvc@stop{\ekvc@err@empty@hash}
\long\def\ekvc@safesplithash@#1#2{\ekvc@err@empty@hash#2{#1}{}}
\long\def\ekvc@fastsplithash@#1\ekvc@stop#2{\ekvc@err@empty@hash#2{#1}{}}
%    \end{macrocode}
% \end{macro}
%
% \begin{macro}{\ekvcSecondaryKeys}
%   The secondary keys are defined pretty similar to the way the originals are,
%   but here we also introduce some key types (those have a |@t@| in their name)
%   additionally to the prefixes.
%    \begin{macrocode}
\protected\long\def\ekvcSecondaryKeys#1#2%
  {%
    \edef\ekvc@set{\string#1}%
    \ekvparse\ekvc@err@value@required\ekvcSecondaryKeys@a{#2}%
  }
\protected\def\ekvcSecondaryKeys@a#1%
  {%
    \let\ekvc@long\ekv@empty
    \ekvc@ifspace{#1}%
      {\ekvcSecondaryKeys@b#1\ekvc@stop}%
      {\ekvc@err@missing@type{#1}\@gobble}%
  }
\protected\def\ekvcSecondaryKeys@b#1 #2\ekvc@stop
  {%
    \ekv@ifdefined{ekvc@p@#1}%
      {\csname ekvc@p@#1\endcsname}%
      {%
        \ekv@ifdefined{ekvc@t@#1}%
          {\csname ekvc@t@#1\endcsname}%
          {\ekvc@err@unknown@keytype{#1}\@firstoftwo\@gobble}%
      }%
      {#2}%
  }
%    \end{macrocode}
% \end{macro}
%
% \begin{macro}{\ekvcChange}
%   This can be used to change the defaults of an \expkvc\ defined macro. It
%   checks whether there is a set with the correct name and that the macro is
%   defined. If both is true the real work is done by |\ekvc@change|.
%    \begin{macrocode}
\protected\long\def\ekvcChange#1%
  {%
    \ekvifdefinedset{\string#1}%
      {%
        \ekvc@ifdefined#1%
          {\ekvc@change#1}%
          {\ekvc@err@no@key@macro#1\@gobble}%
      }%
      {\ekvc@err@no@key@macro#1\@gobble}%
  }
%    \end{macrocode}
% \end{macro}
%
% \begin{macro}[internal]
%   {\ekvc@change,\ekvc@change@a,\ekvc@change@b,\ekvc@change@c}
%   First we need to see whether the macro is currently |\long|. For this we get
%   the meaning and will parse it. |#1| is the macro name in which we want to
%   change the defaults.
%    \begin{macrocode}
\protected\def\ekvc@change#1%
  {\expandafter\ekvc@change@a\meaning#1\ekv@stop#1}
%    \end{macrocode}
%   A temporary definition to get the stringified |macro:|. |##1| will be the
%   list of prefixes, we don't care for the exact contents of |##2| and |##3|.
%    \begin{macrocode}
\def\ekvc@change@a#1%
  {%
    \protected\def\ekvc@change@a##1#1##2->##3\ekv@stop
      {%
        \ekvc@change@iflong{##1}%
          {\ekvc@change@b{}}%
          {\ekvc@change@b{\long}}%
      }%
  }
\expandafter\ekvc@change@a\expandafter{\detokenize{macro:}}
%    \end{macrocode}
%   Next we expand the macro once to get its contents (including the current
%   default values with their markers). |#1| is either |\long| or empty, |#2| is
%   the macro.
%    \begin{macrocode}
\protected\def\ekvc@change@b#1#2%
  {\expandafter\ekvc@change@c\expandafter{#2{##1}}{#1}#2}
%    \end{macrocode}
%   Here we place an unbalanced closing brace after the expansion of the macro.
%   Then we just parse the \kv-list with |\ekvset|, that will exchange the
%   values behind the markers. Once those are changed we give control to
%   |\ekvc@change@d|. The |\ekvset| step might horribly fail if the user defined
%   some keys that don't behave nice. |#1| is the expansion of the macro, |#2|
%   is either |\long| or empty, |#3| is the macro, and |#4| is the \kv-list
%   containing the new defaults.
%    \begin{macrocode}
\ekv@exparg{\protected\long\def\ekvc@change@c#1#2#3#4}%
  {%
    \expandafter\iffalse\expandafter{\expandafter\fi
      \ekvset{\string#3}{#4}%
      \ekvc@change@d{#2}{#3}%
      #1%
    }%
  }
%    \end{macrocode}
%   The final step needs to put an unbalanced opening brace after |\def|. We do
%   that with the help of a temporary macro which stores everything necessary
%   for |\def| and expand an |\iffalse}\fi| construct to leave a single
%   opening brace. |#1| will be either empty or |\long| and |#2| is the macro.
%   Each of the macros defined with \expkvc\ takes exactly one parameter, so we
%   put that here as |####1|.
%    \begin{macrocode}
\protected\def\ekvc@change@d#1#2%
  {%
    \def\ekvc@tmp{#1\def#2####1}%
    \expandafter\ekvc@tmp\expandafter{\iffalse}\fi
  }
%    \end{macrocode}
% \end{macro}
%
% \begin{macro}[internal]{\ekvc@change@iflong,\ekvc@change@iflong@}
%   Checking whether a string contains the string representation of |\long| can
%   be done by gobbling everything up to the first |\long| and checking whether
%   the result is completely empty. We need a temporary macro to get the result
%   of |\string\long| inside the definitions.
%    \begin{macrocode}
\def\ekvc@change@iflong#1%
  {%
    \protected\def\ekvc@change@iflong##1%
      {\expandafter\ekv@ifempty\expandafter{\ekvc@change@iflong@##1#1}}%
    \def\ekvc@change@iflong@##1#1{}
  }
\expandafter\ekvc@change@iflong\expandafter{\string\long}
%    \end{macrocode}
% \end{macro}
%
% \begin{macro}{\ekvcPass}
%   This macro can be used to pass a value to a key of some macro (this way
%   more complicated key codes are possible that in the end pass processed
%   values on to some macro). The implemantation is pretty straight forward.
%    \begin{macrocode}
\long\def\ekvcPass#1#2%
  {%
    \ekvifdefined{\string#1}{#2}%
      {\csname\ekv@name{\string#1}{#2}\endcsname}%
      {\ekvc@err@unknown@key@or@macro{#1}{#2}\@gobble}%
  }
%    \end{macrocode}
% \end{macro}
%
%
% \subsubsection{Secondary Key Types}
%
% \begin{macro}[internal]{\ekvc@p@long,\ekvc@after@ptype}
%   The prefixes are pretty straight forward again. Just set |\ekvc@long| and
%   forward to the |@t@| type.
%    \begin{macrocode}
\protected\def\ekvc@p@long#1%
  {%
    \ekvc@ifspace{#1}%
      {%
        \let\ekvc@long\long
        \ekvc@after@ptype#1\ekvc@stop
      }%
      {\ekvc@err@missing@type{long #1}\@gobble}%
  }
\protected\def\ekvc@after@ptype#1 #2\ekvc@stop
  {%
    \ekv@ifdefined{ekvc@t@#1}%
      {\csname ekvc@t@#1\endcsname{#2}}%
      {\ekvc@err@unknown@keytype{#1}\@gobble}%
  }
%    \end{macrocode}
% \end{macro}
%
% \begin{macro}[internal]
%   {
%     \ekvc@t@meta, \ekvc@t@nmeta,
%     \ekvc@type@meta, \ekvc@type@meta@a, \ekvc@type@meta@b
%   }
%   The |meta| and |nmeta| key types use a nested |\ekvset| to set other keys in
%   the same macro's \set.
%    \begin{macrocode}
\protected\def\ekvc@t@meta
  {%
    \edef\ekvc@tmp{\ekvc@set}%
    \expandafter\ekvc@type@meta\expandafter{\ekvc@tmp}\ekvc@long{##1}\ekvlet
  }
\protected\def\ekvc@t@nmeta#1%
  {%
    \ekvc@assert@not@long{nmeta #1}%
    \edef\ekvc@tmp{\ekvc@set}%
    \expandafter\ekvc@type@meta\expandafter{\ekvc@tmp}{}{}\ekvletNoVal{#1}%
  }
\protected\long\def\ekvc@type@meta#1#2#3#4#5#6%
  {%
    \expandafter\ekvc@type@meta@a\expandafter{\ekvset{#1}{#6}}{#2}{#3}%
    #4\ekvc@set{#5}\ekvc@tmp
  }
\protected\def\ekvc@type@meta@a
  {%
    \expandafter\ekvc@type@meta@b\expandafter
  }
\protected\long\def\ekvc@type@meta@b#1#2#3%
  {%
    #2\def\ekvc@tmp#3{#1}%
  }
%    \end{macrocode}
% \end{macro}
%
% \begin{macro}[internal]{\ekvc@t@alias}
%   |alias| just checks whether there is a key and/or |NoVal| key defined with
%   the target name and |\let| the key to those.
%    \begin{macrocode}
\protected\def\ekvc@t@alias#1#2%
  {%
    \ekvc@assert@not@long{alias #1}%
    \let\ekvc@tmp\@firstofone
    \ekvifdefined\ekvc@set{#2}%
      {%
        \ekvletkv\ekvc@set{#1}\ekvc@set{#2}%
        \let\ekvc@tmp\@gobble
      }%
      {}%
    \ekvifdefinedNoVal\ekvc@set{#2}%
      {%
        \ekvletkvNoVal\ekvc@set{#1}\ekvc@set{#2}%
        \let\ekvc@tmp\@gobble
      }%
      {}%
    \ekvc@tmp{\ekvc@err@unknown@key{#2}}%
  }
%    \end{macrocode}
% \end{macro}
%
% \begin{macro}[internal]{\ekvc@t@default}
%   The |default| key can be used to set a |NoVal| key for an existing key. It
%   will just pass the \val\ to the key macro of that other key.
%    \begin{macrocode}
\protected\long\def\ekvc@t@default#1#2%
  {%
    \ekvifdefined\ekvc@set{#1}%
      {%
        \ekvc@assert@not@long{default #1}%
        \edef\ekvc@tmp
          {%
            \unexpanded\expandafter
              {\csname\ekv@name\ekvc@set{#1}\endcsname{#2}}%
          }%
        \ekvletNoVal\ekvc@set{#1}\ekvc@tmp
      }%
      {\ekvc@err@unknown@key{#1}}%
  }
%    \end{macrocode}
% \end{macro}
%
% \begin{macro}[internal]{\ekvc@t@flag-bool}
%    \begin{macrocode}
\protected\expandafter\def\csname ekvc@t@flag-bool\endcsname#1#2%
  {%
    \ekvc@assert@not@long{flag-bool #1}%
    \unless\ifdefined#2\ekvcFlagNew#2\fi
    \ekvdef\ekvc@set{#1}%
      {%
        \ekv@ifdefined{ekvc@flag@set@##1}%
          {%
            \csname ekvc@flag@set@##1\expandafter\endcsname
              \ekvcFlagHeight#2\ekv@stop#2%
          }%
          {\ekvc@err@invalid@bool{##1}}%
      }%
  }
%    \end{macrocode}
% \end{macro}
%
% \begin{macro}[internal]
%   {\ekvc@t@flag-true,\ekvc@t@flag-false,\ekvc@t@flag-raise,\ekvc@type@flag}
%    \begin{macrocode}
\protected\def\ekvc@type@flag#1#2#3#4%
  {%
    \ekvc@assert@not@long{flag-#1 #3}%
    \unless\ifdefined#4\ekvcFlagNew#4\fi
    \ekv@exparg{\ekvdefNoVal\ekvc@set{#3}}{#2#4}%
  }
\protected\expandafter\def\csname ekvc@t@flag-true\endcsname
  {\ekvc@type@flag{true}\ekvcFlagSetTrue}
\protected\expandafter\def\csname ekvc@t@flag-false\endcsname
  {\ekvc@type@flag{false}\ekvcFlagSetFalse}
\protected\expandafter\def\csname ekvc@t@flag-raise\endcsname
  {\ekvc@type@flag{raise}\ekvcFlagRaise}
%    \end{macrocode}
% \end{macro}
%
%
% \subsubsection{Flags}
%
% The basic idea of flags is to store information by the fact that \TeX\
% expandably assigns the meaning |\relax| to undefined control sequences which
% were built with |\csname|. This mechanism is borrowed from \pkg{expl3}.
%
% \begin{macro}[internal]{\ekvc@flag@name,\ekvc@flag@namescheme}
%   Flags follow a simple naming scheme which we define here. |\ekvc@flag@name|
%   will store the name of an internal function that is used to build names of
%   the second naming scheme defined by |\ekvc@flag@namescheme|.
%    \begin{macrocode}
\def\ekvc@flag@name{ekvcf\string}
\def\ekvc@flag@namescheme#1#2{ekvch#2#1}
%    \end{macrocode}
% \end{macro}
%
% \begin{macro}{\ekvcFlagHeight}
%   For semantic reasons we use |\number| with another name.
%    \begin{macrocode}
\let\ekvcFlagHeight\number
%    \end{macrocode}
% \end{macro}
%
% \begin{macro}{\ekvcFlagNew}
%   This macro defines a new flag. It stores the function build with the
%   |\ekvc@flag@name| naming scheme after the internal function
%   |\ekvc@flag@height| that'll determine the current flag height. It'll also
%   define the macro named via |\ekvc@flag@name| to build names according to
%   |\ekvc@flag@namescheme|.
%    \begin{macrocode}
\protected\def\ekvcFlagNew#1%
  {%
    \edef#1%
      {%
        \unexpanded{\ekvc@flag@height}%
          \unexpanded\expandafter{\csname\ekvc@flag@name#1\endcsname}%
      }%
    \ekv@expargtwice
      {\expandafter\def\csname\ekvc@flag@name#1\endcsname##1}%
      {\expandafter\ekvc@flag@namescheme\expandafter{\string#1}{##1}}%
  }
%    \end{macrocode}
% \end{macro}
%
% \begin{macro}[internal]{\ekvc@flag@height,\ekvc@flag@height@}
%   This macro gets the height of a flag by a simple loop. The first loop
%   iteration differs a bit from the following in that it doesn't have to get
%   the current iteration count. The space at the end of |\ekvc@flag@height|
%   ends the |\number| evaluation.
%    \begin{macrocode}
\def\ekvc@flag@height#1%
  {%
    \ifcsname#10\endcsname
      \ekvc@flag@height@1\ekv@stop#1%
    \fi
    \@firstofone{0} % leave this space
  }
\def\ekvc@flag@height@#1\ekv@stop#2\fi\@firstofone#3%
  {%
    \fi
    \ifcsname#2{#1}\endcsname
      \expandafter\ekvc@flag@height@\the\numexpr#1+1\relax\ekv@stop#2%
    \fi
    \@firstofone{#1}%
  }
%    \end{macrocode}
% \end{macro}
%
% \begin{macro}{\ekvcFlagRaise}
%   Raising a flag simply means letting the |\ekvc@flag@namescheme| macro for
%   the current height to relax. The result of raising a flag is that its height
%   is bigger by $1$.
%    \begin{macrocode}
\ekv@exparg{\def\ekvcFlagRaise#1}%
  {%
    \expandafter\expandafter\expandafter\@gobble\expandafter
      \csname\ekvc@flag@namescheme{\string#1}{\ekvcFlagHeight#1}\endcsname
  }
%    \end{macrocode}
% \end{macro}
%
% \begin{macro}{\ekvcFlagSetTrue,\ekvcFlagSetFalse}
% \begin{macro}[internal]{\ekvc@flag@set@true,\ekvc@flag@set@false}
%   A flag is considered true if its current height is odd, and as false if it
%   is even. Therefore |\ekvcFlagSetTrue| and |\ekvcFlagSetFalse| only need to
%   raise the flag if the opposing boolean value is the current one.
%    \begin{macrocode}
\def\ekvcFlagSetTrue#1%
  {\expandafter\ekvc@flag@set@true\ekvcFlagHeight#1\ekv@stop#1}
\def\ekvcFlagSetFalse#1%
  {\expandafter\ekvc@flag@set@false\ekvcFlagHeight#1\ekv@stop#1}
%    \end{macrocode}
%   We can expand |\ekvc@flag@namescheme| at definition time here, which is why
%   we're using a temporary definition to set up |\ekvc@flag@set@true| and
%   |\ekvc@flag@set@false|.
%    \begin{macrocode}
\def\ekvc@flag@set@true#1%
  {%
    \def\ekvc@flag@set@true##1\ekv@stop##2%
      {%
        \ifodd##1
          \ekv@fi@gobble
        \fi
        \@firstofone{\expandafter\@gobble\csname#1\endcsname}%
      }%
    \def\ekvc@flag@set@false##1\ekv@stop##2%
      {%
        \ifodd##1
          \ekv@fi@firstofone
        \fi
        \@gobble{\expandafter\@gobble\csname#1\endcsname}%
      }%
  }
\expandafter\ekvc@flag@set@true\expandafter
  {\ekvc@flag@namescheme{\string#2}{#1}}
%    \end{macrocode}
% \end{macro}
% \end{macro}
%
% \begin{macro}{\ekvcFlagIf}
%   As already explained, truthiness is defined as a flag's height is odd, so we
%   just branch accordingly here.
%    \begin{macrocode}
\def\ekvcFlagIf#1%
  {%
    \ifodd#1%
      \ekv@fi@firstoftwo
    \fi
    \@secondoftwo
  }
%    \end{macrocode}
% \end{macro}
%
% \begin{macro}{\ekvcFlagIfRaised}
%   This macro uses flags as a switch, if a flag's current height is bigger than
%   $0$ this test yields true.
%    \begin{macrocode}
\ekv@exparg{\def\ekvcFlagIfRaised#1}%
  {%
    \expandafter\ifcsname\ekvc@flag@namescheme{\string#1}0\endcsname
      \ekv@fi@firstoftwo
    \fi
    \@secondoftwo
  }
%    \end{macrocode}
% \end{macro}
%
% \begin{macro}{\ekvcFlagReset}
% \begin{macro}[internal]{\ekvc@flag@reset,\ekvc@flag@reset@}
%   Resetting works by locally letting all the defined internal macros named
%   after |\ekvc@flag@namescheme| to undefined.
%    \begin{macrocode}
\protected\def\ekvcFlagReset#1%
  {\expandafter\ekvc@flag@reset\csname\ekvc@flag@name#1\endcsname}
\protected\def\ekvc@flag@reset#1%
  {%
    \ifcsname#10\endcsname
      \expandafter\let\csname#10\endcsname\ekvc@undefined
      \ekvc@flag@reset@1\ekv@stop#1%
    \fi
  }
\protected\def\ekvc@flag@reset@#1\ekv@stop#2\fi
  {%
    \fi
    \ifcsname#2{#1}\endcsname
      \expandafter\let\csname#2{#1}\endcsname\ekvc@undefined
      \expandafter\ekvc@flag@reset@\the\numexpr#1+1\relax\ekv@stop#2%
    \fi
  }
%    \end{macrocode}
% \end{macro}
% \end{macro}
%
% \begin{macro}{\ekvcFlagGetHeight}
% \begin{macro}[internal]{\ekvc@flag@get@height@single}
%   These are just small helpers, first getting the height of the flag and then
%   passing it on to the user supplied code.
%    \begin{macrocode}
\def\ekvcFlagGetHeight#1%
  {\expandafter\ekvc@flag@get@height@single\ekvcFlagHeight#1\ekv@stop}
\long\def\ekvc@flag@get@height@single#1\ekv@stop#2{#2{#1}}
%    \end{macrocode}
% \end{macro}
% \end{macro}
%
% \begin{macro}{\ekvcFlagGetHeights}
% \begin{macro}[internal]
%   {\ekvc@flag@get@heights,\ekvc@flag@get@heights@,\ekvc@flag@get@heights@done}
%   This works by a simple loop that stops at |\ekv@stop|. As long as that
%   marker isn't hit, get the next flags height and put it into a list after
%   |\ekv@stop|. |\ekvc@flag@get@heights@| uses the same marker name for the
%   end of the height, which shouldn't clash in any case. Once we're done we
%   remove the remainder of the current iteration and leave the user supplied
%   code in the input stream with all the flags' heights as a single argument.
%    \begin{macrocode}
\def\ekvcFlagGetHeights#1%
  {%
    \ekvc@flag@get@heights#1\ekv@stop{}%
  }
\def\ekvc@flag@get@heights#1%
  {%
    \ekv@gobbleto@stop#1\ekvc@flag@get@heights@done\ekv@stop
    \expandafter\ekvc@flag@get@heights@\ekvcFlagHeight#1\ekv@stop
  }
\def\ekvc@flag@get@heights@#1\ekv@stop#2\ekv@stop#3%
  {\ekvc@flag@get@heights#2\ekv@stop{#3{#1}}}
\long\def\ekvc@flag@get@heights@done
    \ekv@stop
    \expandafter\ekvc@flag@get@heights@\ekvcFlagHeight\ekv@stop\ekv@stop#1#2%
  {#2{#1}}
%    \end{macrocode}
% \end{macro}
% \end{macro}
%
%
% \subsubsection{Helper Macros}
%
% \begin{macro}[internal]{\ekvc@ifspace,\ekvc@ifspace@}
% A test which can be reduced to an if-empty by gobbling everything up to the
% first space.
%    \begin{macrocode}
\long\def\ekvc@ifspace#1%
  {%
    \ekvc@ifspace@#1 \ekv@ifempty@B
      \ekv@ifempty@false\ekv@ifempty@A\ekv@ifempty@B\@firstoftwo
  }
\long\def\ekvc@ifspace@#1 % keep this space
  {%
    \ekv@ifempty@\ekv@ifempty@A
  }
%    \end{macrocode}
% \end{macro}
%
%
% \subsubsection{Assertions}
% \begin{macro}[internal]{\ekvc@assert@not@long}
%   Some keys don't want to be |long| and we have to educate the user, so let's
%   throw an error if someone wanted these to be long.
%    \begin{macrocode}
\long\def\ekvc@assert@not@long#1{\ifx\ekvc@long\long\ekvc@err@no@long{#1}\fi}
%    \end{macrocode}
% \end{macro}
%
%
% \subsubsection{Messages}
%
% \begin{macro}[internal]
%   {
%     \ekvc@err@toomany,\ekvc@err@value@required,\ekvc@err@missing@type,
%     \ekvc@err@already@defined,\ekvc@err@no@key@macro
%   }
%   Boring unexpandable error messages.
%    \begin{macrocode}
\protected\def\ekvc@err@toomany#1%
  {%
    \errmessage{expkv-cs Error: Too many keys for macro `\string#1'}%
  }
\protected\def\ekvc@err@value@required#1%
  {%
    \errmessage{expkv-cs Error: Missing value for key `\unexpanded{#1}'}%
  }
\protected\def\ekvc@err@missing@type#1%
  {%
    \errmessage
      {expkv-cs Error: Missing type for secondary key `\unexpanded{#1}'}%
  }
\protected\def\ekvc@err@no@long#1%
  {%
    \errmessage
      {expkv-cs Error: prefix `long' not accepted for `\unexpanded{#1}'}%
  }
\protected\def\ekvc@err@already@defined#1%
  {%
    \errmessage{expkv-cs Error: Macro `\string#1' already defined}%
  }
\protected\def\ekvc@err@unknown@keytype#1%
  {%
    \errmessage{expkv-cs Error: Unknown key type `\unexpanded{#1}'}%
  }
\protected\def\ekvc@err@unknown@key#1%
  {%
    \errmessage
      {expkv-cs Error: Unknown key `\unexpanded{#1}' for macro `\ekvc@set'}%
  }
\long\def\ekvc@err@no@key@macro#1%
  {\errmessage{expkv-cs Error: \string#1 is no key=val macro}}
%    \end{macrocode}
% \end{macro}
%
% \begin{macro}[internal]{\ekvc@err}
%   We need a way to throw error messages expandably in some contexts.
%    \begin{macrocode}
\ekv@exparg{\long\def\ekvc@err#1}{\ekverr{expkv-cs}{#1}}
%    \end{macrocode}
% \end{macro}
%
% \begin{macro}[internal]
%   {
%     \ekvc@err@unknown@hash,\ekvc@err@empty@hash,\ekvc@err@missing@hash,
%     \ekvc@err@invalid@bool
%   }
%   And here are the expandable error messages.
%    \begin{macrocode}
\long\def\ekvc@err@unknown@hash#1{\ekvc@err{unknown hash `#1'}}
\long\def\ekvc@err@missing@hash#1{\ekvc@err{hash `#1' not found}}
\long\def\ekvc@err@empty@hash{\ekvc@err{empty hash}}
\def\ekvc@err@invalid@bool#1{\ekvc@err{invalid boolean value `#1'}}
\long\def\ekvc@err@unknown@key@or@macro#1#2%
  {\ekvc@err{unknown key `#2' for macro #1}}
%    \end{macrocode}
% \end{macro}
%
%
% Now everything that's left is to reset the category code of |@|.
%    \begin{macrocode}
\catcode`\@=\ekvc@tmp
%    \end{macrocode}
%
% \gobbledocstriptag
%</tex>
%^^A=<<
%
%^^A=<<
%
% \end{implementation}^^A=<<
%
% \clearpage
% \PrintIndex
%
\endinput
%
^^A vim: ft=tex fdm=marker fmr=>>=,=<<

\ProvidesPackage{expkv-cs}%
  [%
    \ekvcDate\space v\ekvcVersion\space
    define expandable key=val macros using expkv%
  ]
%    \end{macrocode}
% \gobbledocstriptag
%</pkg>
%^^A=<<
%^^A main file >>=
% \subsection{The Generic Code}
% The rest of this implementation will be the generic code.
% \gobbledocstriptag
%<*tex>
%
% Load \expkv\ if the package didn't already do so -- since \expkv\ has
% safeguards against being loaded twice this does no harm and the overhead
% isn't that big. Also we reuse some of the internals of \expkv\ to save us from
% retyping them.
%    \begin{macrocode}
\input expkv
%    \end{macrocode}
%
% We make sure that \file{expkv-cs.tex} is only input once:
%    \begin{macrocode}
\expandafter\ifx\csname ekvcVersion\endcsname\relax
\else
  \expandafter\endinput
\fi
%    \end{macrocode}
%
% \begin{macro}{\ekvcVersion,\ekvcDate}
% We're on our first input, so lets store the version and date in a macro.
%    \begin{macrocode}
\def\ekvcVersion{0.10}
\def\ekvcDate{2021-06-03}
%    \end{macrocode}
% \end{macro}
%
% If the \LaTeX\ format is loaded we want to be a good file and report back who
% we are, for this the package will have defined |\ekvc@tmp| to use
% |\ProvidesFile|, else this will expand to a |\relax| and do no harm.
%    \begin{macrocode}
\csname ekvc@tmp\endcsname
%    \end{macrocode}
%
% Store the category code of |@| to later be able to reset it and change it to
% 11 for now.
%    \begin{macrocode}
\expandafter\chardef\csname ekvc@tmp\endcsname=\catcode`\@
\catcode`\@=11
%    \end{macrocode}
% |\ekvc@tmp| will be reused later, but we don't need it to ever store
% information long-term after \expkvc\ was initialized.
%
% \begin{macro}[internal]{\ekvc@keycount}
%   We'll need to keep count how many keys must be defined for each macro in the
%   |split| variants.
%    \begin{macrocode}
\newcount\ekvc@keycount
%    \end{macrocode}
% \end{macro}
%
% \begin{macro}[internal]{\ekvc@long,\ekvc@any@long}
%   Some macros will have to be defined long. These two will be let to |\long|
%   when this should be the case.
%    \begin{macrocode}
\let\ekvc@long\ekv@empty
\let\ekvc@any@long\ekv@empty
%    \end{macrocode}
% \end{macro}
%
% \begin{macro}[internal]{\ekvc@ifdefined}
%   We want to test whether a macro is already defined. This test checks for a
%   defined macro that isn't |\relax|.
%    \begin{macrocode}
\long\def\ekvc@ifdefined#1%
  {%
    \ifdefined#1%
      \ifx\relax#1%
        \ekv@fi@gobble
      \fi
      \@firstofone
      \ekv@fi@firstoftwo
    \fi
    \@secondoftwo
  }
%    \end{macrocode}
% \end{macro}
%
% \begin{macro}[internal]
%   {
%     \ekvc@ekvset@pre@expander,
%     \ekvc@ekvset@pre@expander@a,
%     \ekvc@ekvset@pre@expander@b
%   }
%   This macro expands |\ekvset| twice so that the first two steps of expansion
%   don't have to be made every time the \expkvc\ macros are used. We have to do
%   a little magic trick to get the macro parameter |#1| for the macro
%   definition this is used in, even though we're calling |\unexpanded|. We do
%   that by splitting the expanded |\ekvset| at some marks and place |##1| in
%   between. At this spot we also add |\ekv@alignsafe| and |\ekv@endalignsafe|
%   to ensure that macros created with \expkvc\ are alignment safe.
%    \begin{macrocode}
\def\ekvc@ekvset@pre@expander#1%
  {%
    \expandafter\ekvc@ekvset@pre@expander@a\ekvset{#1}\ekvc@stop\ekvc@stop
  }
\def\ekvc@ekvset@pre@expander@a
  {%
    \expandafter\ekvc@ekvset@pre@expander@b
  }
\def\ekvc@ekvset@pre@expander@b#1\ekvc@stop#2\ekvc@stop
  {%
    \unexpanded\expandafter{\ekv@alignsafe}%
    \unexpanded{#1}##1\unexpanded{#2}%
    \unexpanded\expandafter{\ekv@endalignsafe}%
  }
%    \end{macrocode}
% \end{macro}
%
% \begin{macro}{\ekvcSplitAndUse}
%   The first user macro we want to set up can be reused for
%   \cs[no-index]{ekvcSplitAndForward} and \cs[no-index]{ekvcSplit}. We'll split
%   this one up so that the test whether the macro is already defined doesn't
%   run twice.
%    \begin{macrocode}
\protected\long\def\ekvcSplitAndUse#1#2%
  {%
    \let\ekvc@helpers@needed\@firstoftwo
    \ekvc@ifdefined#1%
      {\ekvc@err@already@defined#1}%
      {\ekvcSplitAndUse@#1{}{#2}}%
  }
%    \end{macrocode}
% \end{macro}
%
% \begin{macro}[internal]{\ekvcSplitAndUse@}
%   The actual macro setting up things. We need to set some variables, forward
%   the key list to |\ekvc@SetupSplitKeys|, and afterwards define the front
%   facing macro to call |\ekvset| and put the initials and the argument sorting
%   macro behind it. The internals |\ekvc@any@long|, |\ekvc@initials| and
%   |\ekvc@keycount| will be set correctly by |\ekvc@SetupSplitKeys|.
%    \begin{macrocode}
\protected\long\def\ekvcSplitAndUse@#1#2#3%
  {%
    \edef\ekvc@set{\string#1}%
    \ekvc@SetupSplitKeys{#3}%
    \ekvc@helpers@needed
      {%
        \ekvc@any@long\edef#1##1%
          {%
            \expandafter\ekvc@ekvset@pre@expander\expandafter{\ekvc@set}%
            \unexpanded\expandafter
              {\csname ekvc@split@\the\ekvc@keycount\endcsname}%
            \unexpanded\expandafter{\ekvc@initials{}#2}%
          }%
      }%
      {%
        \ekvc@any@long\edef#1##1%
          {%
            \expandafter\ekvc@ekvset@pre@expander\expandafter{\ekvc@set}%
            \unexpanded{#2}%
            \unexpanded\expandafter{\ekvc@initials}%
          }%
      }%
  }
%    \end{macrocode}
% \end{macro}
%
%
% \begin{macro}{\ekvcSplitAndForward}
%   This just reuses |\ekvcSplitAndUse@| with a non-empty second argument,
%   resulting in that argument to be called after the splitting.
%    \begin{macrocode}
\protected\long\def\ekvcSplitAndForward#1#2#3%
  {%
    \let\ekvc@helpers@needed\@firstoftwo
    \ekvc@ifdefined#1%
      {\ekvc@err@already@defined#1}%
      {\ekvcSplitAndUse@#1{{#2}}{#3}}%
  }
%    \end{macrocode}
% \end{macro}
%
% \begin{macro}{\ekvcSplit}
%   The first half is just |\ekvcSplitAndForward| then we define the macro to
%   which the parsed key list is forwarded. There we need to allow for up to
%   nine arguments.
%    \begin{macrocode}
\protected\long\def\ekvcSplit#1#2#3%
  {%
    \let\ekvc@helpers@needed\@secondoftwo
    \ekvc@ifdefined#1%
      {\ekvc@err@already@defined#1}%
      {%
        \expandafter
        \ekvcSplitAndUse@\expandafter#1\csname ekvc@\string#1\endcsname{#2}%
        \ifnum\ekvc@keycount<1
          \ekvc@any@long\expandafter\def\csname ekvc@\string#1\endcsname{#3}%
        \else
          \ifnum\ekvc@keycount>9
            \ekvc@err@toomany{#1}%
            \let#1\ekvc@undefined
          \else
            \ekvcSplit@build@argspec
            \ekvc@any@long\expandafter
            \def\csname ekvc@\string#1\expandafter\endcsname\ekvc@tmp{#3}%
          \fi
        \fi
      }%
  }
%    \end{macrocode}
% \end{macro}
%
% \begin{macro}[internal]{\ekvcSplit@build@argspec,\ekvcSplit@build@argspec@}
%    \begin{macrocode}
\protected\def\ekvcSplit@build@argspec
  {%
    \begingroup
      \edef\ekvc@tmp
        {\endgroup\def\unexpanded{\ekvc@tmp}{\ekvcSplit@build@argspec@{1}}}%
    \ekvc@tmp
  }
\def\ekvcSplit@build@argspec@#1%
  {%
    \ifnum#1>\ekvc@keycount
      \ekv@fi@gobble
    \fi
    \@firstofone
      {%
        \unexpanded\expandafter{\csname ekvc@splitmark@#1\endcsname####}#1%
        \expandafter\ekvcSplit@build@argspec@\expandafter{\the\numexpr#1+1}%
      }%
  }
%    \end{macrocode}
% \end{macro}
%
% \begin{macro}[internal]
%   {
%     \ekvc@SetupSplitKeys, \ekvc@SetupSplitKeys@a, \ekvc@SetupSplitKeys@b,
%     \ekvc@SetupSplitKeys@c
%   }
%   These macros parse the list of keys and set up the key macros. First we need
%   to initialise some macros and start |\ekvparse|.
%    \begin{macrocode}
\protected\long\def\ekvc@SetupSplitKeys
  {%
    \ekvc@keycount=0
    \let\ekvc@any@long\ekv@empty
    \let\ekvc@initials\ekv@empty
    \ekvparse\ekvc@err@value@required\ekvc@SetupSplitKeys@a
  }
%    \end{macrocode}
%   Then we need to step the key counter for each key. Also we have to check
%   whether this key has a |long| prefix so we initialise |\ekvc@long|.
%    \begin{macrocode}
\protected\def\ekvc@SetupSplitKeys@a#1%
  {%
    \advance\ekvc@keycount1
    \let\ekvc@long\ekv@empty
    \ekvc@ifspace{#1}%
      {\ekvc@SetupSplitKeys@b#1\ekvc@stop}%
      {\ekvc@SetupSplitKeys@c{#1}}%
  }
%    \end{macrocode}
%   If there was a space, there might be a prefix. If so call the prefix macro,
%   else call the next step |\ekvc@SetupSplitKeys@c| which will define the key
%   macro and add the key's value to the initials list.
%    \begin{macrocode}
\protected\def\ekvc@SetupSplitKeys@b#1 #2\ekvc@stop
  {%
    \ekv@ifdefined{ekvc@split@p@#1}%
      {\csname ekvc@split@p@#1\endcsname{#2}}%
      {\ekvc@SetupSplitKeys@c{#1 #2}}%
  }
%    \end{macrocode}
%   The inner definition is grouped, because we don't want to actually define
%   the marks we build with |\csname|. We have to append the value to the
%   |\ekvc@initials| list here with the correct split mark. The key macro will
%   read everything up to those split marks and change the value following it to
%   the value given to the key. Additionally we'll need a sorting macro for each
%   key count in use so we set it up with |\ekvc@setup@splitmacro|.
%    \begin{macrocode}
\protected\long\def\ekvc@SetupSplitKeys@c#1#2%
  {%
    \begingroup
    \edef\ekvc@tmp
      {%
        \endgroup
        \long\def\unexpanded{\ekvc@tmp}####1####2%
            \unexpanded\expandafter
            {\csname ekvc@splitmark@\the\ekvc@keycount\endcsname}####3%
          {%
            ####2%
            \unexpanded\expandafter
              {\csname ekvc@splitmark@\the\ekvc@keycount\endcsname}{####1}%
          }%
%    \end{macrocode}
%   The short variant needs a bit of special treatment. The key macro will be
%   short to throw the correct error, but since there might be long macros
%   somewhere the reordering of arguments needs to be long, so for short keys we
%   use a two step approach, first grabbing only the short argument, then
%   reordering.
%    \begin{macrocode}
        \unless\ifx\ekvc@long\long
          \let\unexpanded\expandafter
            {\csname ekvc@\ekvc@set(#1)\endcsname\ekvc@tmp}%
          \def\unexpanded{\ekvc@tmp}####1%
            {%
              \unexpanded\expandafter{\csname ekvc@\ekvc@set(#1)\endcsname}%
                {####1}%
            }%
        \fi
        \def\unexpanded{\ekvc@initials}%
          {%
            \unexpanded\expandafter{\ekvc@initials}%
            \unexpanded\expandafter
              {\csname ekvc@splitmark@\the\ekvc@keycount\endcsname{#2}}%
          }%
      }%
    \ekvc@tmp
    \ekvlet\ekvc@set{#1}\ekvc@tmp
    \ekvc@helpers@needed
      {\expandafter\ekvc@setup@splitmacro\expandafter{\the\ekvc@keycount}}%
      {}%
  }
%    \end{macrocode}
% \end{macro}
%
% \begin{macro}[internal]{\ekvc@split@p@long}
%   The |long| prefix lets the internals |\ekvc@long| and |\ekvc@any@long| to
%   |\long| so that the key macro will be long.
%    \begin{macrocode}
\protected\def\ekvc@split@p@long
  {%
    \let\ekvc@long\long
    \let\ekvc@any@long\long
    \ekvc@SetupSplitKeys@c
  }
%    \end{macrocode}
% \end{macro}
%
% \begin{macro}[internal]{\ekvc@defarggobbler}
%   This is needed to define a macro with 1-9 parameters programmatically.
%   \LaTeX's \cs[no-index]{newcommand} does something similar for example.
%    \begin{macrocode}
\protected\def\ekvc@defarggobbler#1{\def\ekvc@tmp##1#1##2##{##1#1}}
%    \end{macrocode}
% \end{macro}
%
% \begin{macro}[internal]
%   {
%     \ekvc@setup@splitmacro,
%     \ekvc@split@1, \ekvc@split@2, \ekvc@split@3,
%     \ekvc@split@4, \ekvc@split@5, \ekvc@split@6,
%     \ekvc@split@7
%   }
%   Since the first few split macros are different from the others we manually
%   set those up now. All the others will be defined as needed (always
%   globally).  The split macros just read up until the correct split mark, move
%   that argument into a list and reinsert the rest, calling the next split
%   macro afterwards.
%    \begin{macrocode}
\begingroup
\edef\ekvc@tmp
  {%
    \long\gdef\unexpanded\expandafter{\csname ekvc@split@1\endcsname}%
        \unexpanded\expandafter{\csname ekvc@splitmark@1\endcsname}%
        ##1##2##3%
      {##3{##1}##2}%
    \long\gdef\unexpanded\expandafter{\csname ekvc@split@2\endcsname}%
        \unexpanded\expandafter{\csname ekvc@splitmark@1\endcsname}##1%
        \unexpanded\expandafter{\csname ekvc@splitmark@2\endcsname}##2%
        ##3##4%
      {##4{##1}{##2}##3}%
    \long\gdef\unexpanded\expandafter{\csname ekvc@split@3\endcsname}%
        \unexpanded\expandafter{\csname ekvc@splitmark@1\endcsname}##1%
        \unexpanded\expandafter{\csname ekvc@splitmark@2\endcsname}##2%
        \unexpanded\expandafter{\csname ekvc@splitmark@3\endcsname}##3%
        ##4##5%
      {##5{##1}{##2}{##3}##4}%
    \long\gdef\unexpanded\expandafter{\csname ekvc@split@4\endcsname}%
        \unexpanded\expandafter{\csname ekvc@splitmark@1\endcsname}##1%
        \unexpanded\expandafter{\csname ekvc@splitmark@2\endcsname}##2%
        \unexpanded\expandafter{\csname ekvc@splitmark@3\endcsname}##3%
        \unexpanded\expandafter{\csname ekvc@splitmark@4\endcsname}##4%
        ##5##6%
      {##6{##1}{##2}{##3}{##4}##5}%
    \long\gdef\unexpanded\expandafter{\csname ekvc@split@5\endcsname}%
        \unexpanded\expandafter{\csname ekvc@splitmark@1\endcsname}##1%
        \unexpanded\expandafter{\csname ekvc@splitmark@2\endcsname}##2%
        \unexpanded\expandafter{\csname ekvc@splitmark@3\endcsname}##3%
        \unexpanded\expandafter{\csname ekvc@splitmark@4\endcsname}##4%
        \unexpanded\expandafter{\csname ekvc@splitmark@5\endcsname}##5%
        ##6##7%
      {##7{##1}{##2}{##3}{##4}{##5}##6}%
    \long\gdef\unexpanded\expandafter{\csname ekvc@split@6\endcsname}%
        \unexpanded\expandafter{\csname ekvc@splitmark@1\endcsname}##1%
        \unexpanded\expandafter{\csname ekvc@splitmark@2\endcsname}##2%
        \unexpanded\expandafter{\csname ekvc@splitmark@3\endcsname}##3%
        \unexpanded\expandafter{\csname ekvc@splitmark@4\endcsname}##4%
        \unexpanded\expandafter{\csname ekvc@splitmark@5\endcsname}##5%
        \unexpanded\expandafter{\csname ekvc@splitmark@6\endcsname}##6%
        ##7##8%
      {##8{##1}{##2}{##3}{##4}{##5}{##6}##7}%
    \long\gdef\unexpanded\expandafter{\csname ekvc@split@7\endcsname}%
        \unexpanded\expandafter{\csname ekvc@splitmark@1\endcsname}##1%
        \unexpanded\expandafter{\csname ekvc@splitmark@2\endcsname}##2%
        \unexpanded\expandafter{\csname ekvc@splitmark@3\endcsname}##3%
        \unexpanded\expandafter{\csname ekvc@splitmark@4\endcsname}##4%
        \unexpanded\expandafter{\csname ekvc@splitmark@5\endcsname}##5%
        \unexpanded\expandafter{\csname ekvc@splitmark@6\endcsname}##6%
        \unexpanded\expandafter{\csname ekvc@splitmark@7\endcsname}##7%
        ##8##9%
      {##9{##1}{##2}{##3}{##4}{##5}{##6}{##7}##8}%
  }
\ekvc@tmp
\endgroup
\protected\def\ekvc@setup@splitmacro#1%
  {%
    \ekv@ifdefined{ekvc@split@#1}{}%
      {%
        \begingroup
          \edef\ekvc@tmp
            {%
              \long\gdef
                  \unexpanded\expandafter{\csname ekvc@split@#1\endcsname}%
                  ####1%
                  \unexpanded\expandafter{\csname ekvc@splitmark@#1\endcsname}%
                  ####2####3%
                {%
                  \unexpanded\expandafter
                    {\csname ekvc@split@\the\numexpr#1-1\relax\endcsname}%
                  ####1{{####2}####3}%
                }%
            }%
          \ekvc@tmp
        \endgroup
      }%
  }
%    \end{macrocode}
% \end{macro}
%
% \begin{macro}{\ekvcHashAndUse}
%   |\ekvcHashAndUse| works just like |\ekvcSplitAndUse|.
%    \begin{macrocode}
\protected\long\def\ekvcHashAndUse#1#2%
  {%
    \let\ekvc@helpers@needed\@firstoftwo
    \ekvc@ifdefined#1%
      {\ekvc@err@already@defined#1}%
      {\ekvcHashAndUse@#1{}{#2}}%
  }
%    \end{macrocode}
% \end{macro}
%
% \begin{macro}[internal]{\ekvcHashAndUse@}
%   This is more or less the same as |\ekvcSplitAndUse@|. Instead of an empty
%   group we place a marker after the initials, we don't use the sorting macros
%   of |split|, but instead pack all the values in one argument.
%    \begin{macrocode}
\protected\long\def\ekvcHashAndUse@#1#2#3%
  {%
    \edef\ekvc@set{\string#1}%
    \ekvc@SetupHashKeys{#3}%
    \ekvc@helpers@needed
      {%
        \ekvc@any@long\edef#1##1%
          {%
            \expandafter\ekvc@ekvset@pre@expander\expandafter{\ekvc@set}%
            \unexpanded{\ekvc@hash@pack@argument}%
            \unexpanded\expandafter{\ekvc@initials\ekvc@stop#2}%
          }%
      }%
      {%
        \ekvc@any@long\edef#1##1%
          {%
            \expandafter\ekvc@ekvset@pre@expander\expandafter{\ekvc@set}%
            \unexpanded{#2}%
            \unexpanded\expandafter{\ekvc@initials\ekvc@stop}%
          }%
      }%
  }
%    \end{macrocode}
% \end{macro}
%
% \begin{macro}{\ekvcHashAndForward}
%   |\ekvcHashAndForward| works just like |\ekvcSplitAndForward|.
%    \begin{macrocode}
\protected\long\def\ekvcHashAndForward#1#2#3%
  {%
    \let\ekvc@helpers@needed\@firstoftwo
    \ekvc@ifdefined#1%
      {\ekvc@err@already@defined#1}%
      {\ekvcHashAndUse@#1{{#2}}{#3}}%
  }
%    \end{macrocode}
% \end{macro}
%
% \begin{macro}{\ekvcHash}
%   |\ekvcHash| does the same as |\ekvcSplit|, but has the advantage of not
%   needing to count arguments, so the definition of the internal macro is a bit
%   more straight forward.
%    \begin{macrocode}
\protected\long\def\ekvcHash#1#2#3%
  {%
    \let\ekvc@helpers@needed\@secondoftwo
    \ekvc@ifdefined#1%
      {\ekvc@err@already@defined#1}%
      {%
        \expandafter
        \ekvcHashAndUse@\expandafter#1\csname ekvc@\string#1\endcsname{#2}%
        \ekvc@any@long\expandafter\def\csname ekvc@\string#1\endcsname
            ##1\ekvc@stop
          {#3}%
      }%
  }
%    \end{macrocode}
% \end{macro}
%
% \begin{macro}[internal]{\ekvc@hash@pack@argument}
%   All this macro does is pack the values into one argument and forward that to
%   the next macro.
%    \begin{macrocode}
\long\def\ekvc@hash@pack@argument#1\ekvc@stop#2{#2{#1}}
%    \end{macrocode}
% \end{macro}
%
% \begin{macro}[internal]
%   {\ekvc@SetupHashKeys,\ekvc@SetupHashKeys@a,\ekvc@SetupHashKeys@b}
%   This should look awfully familiar as well, since it's just the same as for
%   the split keys with a few other names here and there.
%    \begin{macrocode}
\protected\long\def\ekvc@SetupHashKeys#1%
  {%
    \let\ekvc@any@long\ekv@empty
    \let\ekvc@initials\ekv@empty
    \ekvparse\ekvc@err@value@required\ekvc@SetupHashKeys@a{#1}%
  }
\protected\def\ekvc@SetupHashKeys@a#1%
  {%
    \let\ekvc@long\ekv@empty
    \ekvc@ifspace{#1}%
      {\ekvc@SetupHashKeys@b#1\ekvc@stop}%
      {\ekvc@SetupHashKeys@c{#1}}%
  }
\protected\def\ekvc@SetupHashKeys@b#1 #2\ekvc@stop
  {%
    \ekv@ifdefined{ekvc@hash@p@#1}%
      {\csname ekvc@hash@p@#1\endcsname{#2}}%
      {\ekvc@SetupHashKeys@c{#1 #2}}%
  }
%    \end{macrocode}
%   Yes, even the defining macro looks awfully familiar. Instead of numbered we
%   have named marks. Still the key macros grab everything up to their
%   respective mark and reorder the arguments. The same quirk is applied for
%   short keys. And instead of the |\ekvc@setup@splitmacro| we use
%   |\ekvc@setup@hashmacro|.
%    \begin{macrocode}
\protected\long\def\ekvc@SetupHashKeys@c#1#2%
  {%
    \begingroup
    \edef\ekvc@tmp
      {%
        \endgroup
        \long\def\unexpanded{\ekvc@tmp}####1####2%
            \unexpanded\expandafter{\csname ekvc@hashmark@#1\endcsname}####3%
          {%
            ####2%
            \unexpanded\expandafter{\csname ekvc@hashmark@#1\endcsname}{####1}%
          }%
        \unless\ifx\ekvc@long\long
          \let\unexpanded\expandafter
            {\csname ekvc@\ekvc@set(#1)\endcsname\ekvc@tmp}%
          \def\unexpanded{\ekvc@tmp}####1%
            {%
              \unexpanded\expandafter{\csname ekvc@\ekvc@set(#1)\endcsname}%
                {####1}%
            }%
        \fi
        \def\unexpanded{\ekvc@initials}%
          {%
            \unexpanded\expandafter{\ekvc@initials}%
            \unexpanded\expandafter{\csname ekvc@hashmark@#1\endcsname{#2}}%
          }%
      }%
    \ekvc@tmp
    \ekvlet\ekvc@set{#1}\ekvc@tmp
    \ekvc@setup@hashmacro{#1}%
  }
%    \end{macrocode}
% \end{macro}
%
% \begin{macro}[internal]{\ekvc@hash@p@long}
%   Nothing astonishing here either.
%    \begin{macrocode}
\protected\def\ekvc@hash@p@long
  {%
    \let\ekvc@long\long
    \let\ekvc@any@long\long
    \ekvc@SetupHashKeys@c
  }
%    \end{macrocode}
% \end{macro}
%
% \begin{macro}[internal]{\ekvc@setup@hashmacro}
%   The safe hash macros will be executed inside of an |\unexpanded| expansion
%   context, so they have to insert braces for that once they are done.
%   Most of the tests which have to be executed will already be done, but we
%   have to play safe if the hash doesn't show up in the hash list. Therefore we
%   use some |\ekvc@mark|s and |\ekvc@stop| to throw errors if the hash isn't
%   found in the right place. The fast variants have an easier life and just
%   return the correct value.
%    \begin{macrocode}
\protected\def\ekvc@setup@hashmacro#1%
  {%
    \ekv@ifdefined{ekvc@fasthash@#1}{}%
      {%
        \begingroup
          \edef\ekvc@tmp
            {%
              \long\gdef
                \unexpanded\expandafter{\csname ekvc@fasthash@#1\endcsname}%
                  ####1%
                  \unexpanded\expandafter{\csname ekvc@hashmark@#1\endcsname}%
                  ####2####3\unexpanded{\ekvc@stop}%
                {####2}%
              \long\gdef
                \unexpanded\expandafter{\csname ekvc@safehash@#1\endcsname}%
                  ####1%
                {%
                  \unexpanded\expandafter{\csname ekvc@@safehash@#1\endcsname}%
                  ####1\unexpanded{\ekvc@mark}{}%
                  \unexpanded\expandafter
                    {%
                      \csname ekvc@hashmark@#1\endcsname{}%
                      \ekvc@mark{\ekvc@err@missing@hash{#1}}\ekvc@stop
                    }%
                }%
              \long\gdef
                \unexpanded\expandafter{\csname ekvc@@safehash@#1\endcsname}%
                  ####1%
                  \unexpanded\expandafter{\csname ekvc@hashmark@#1\endcsname}%
                  ####2####3\unexpanded{\ekvc@mark}####4####5%
                  \unexpanded{\ekvc@stop}%
                {%
                  ####4{####2}%
                }%
              \long\gdef\unexpanded\expandafter
                {\csname ekvc@fastsplithash@#1\endcsname}%
                  ####1%
                  \unexpanded\expandafter{\csname ekvc@hashmark@#1\endcsname}%
                  ####2####3\unexpanded{\ekvc@stop}####4%
                {%
                  ####4{####1####3}{####2}%
                }%
              \long\gdef\unexpanded\expandafter
                {\csname ekvc@safesplithash@#1\endcsname}####1%
                {%
                  \unexpanded\expandafter
                    {\csname ekvc@@safesplithash@#1\endcsname}%
                  ####1\unexpanded{\ekvc@mark\ekvc@safe@found@hash}%
                  \unexpanded\expandafter
                    {%
                      \csname ekvc@hashmark@#1\endcsname{}%
                      \ekvc@mark{\ekvc@err@missing@hash{#1}\ekvc@safe@no@hash}%
                      \ekvc@stop
                    }%
                }%
              \long\gdef\unexpanded\expandafter
                {\csname ekvc@@safesplithash@#1\endcsname}%
                  ####1%
                  \unexpanded\expandafter{\csname ekvc@hashmark@#1\endcsname}%
                  ####2####3\unexpanded{\ekvc@mark}####4####5%
                  \unexpanded{\ekvc@stop}%
                {%
                  ####4{####2}####1####3\unexpanded{\ekvc@stop}%
                }%
            }%
          \ekvc@tmp
        \endgroup
      }%
  }
%    \end{macrocode}
% \end{macro}
%
% \begin{macro}{\ekvcValue}
%   All this does is a few consistency checks on the first argument (not empty,
%   hash macro exists) and then call that hash-grabbing macro that will also
%   test whether the hash is inside of |#2| or not.
%    \begin{macrocode}
\long\def\ekvcValue#1%
  {%
    \unexpanded
    \ekv@ifdefined{ekvc@safehash@#1}%
      {\csname ekvc@safehash@#1\endcsname}%
      {\ekvc@err@unknown@hash{#1}\@firstoftwo{{}}}%
  }
%    \end{macrocode}
% \end{macro}
%
% \begin{macro}{\ekvcValueFast}
%   To be as fast as possible, this doesn't test for anything, assuming the user
%   knows best.
%    \begin{macrocode}
\long\def\ekvcValueFast#1#2{\csname ekvc@fasthash@#1\endcsname#2\ekvc@stop}
%    \end{macrocode}
% \end{macro}
%
% \begin{macro}{\ekvcValueSplit,\ekvcValueSplit@recover}
%   This splits off a single value.
%    \begin{macrocode}
\long\def\ekvcValueSplit#1%
  {%
    \ekv@ifdefined{ekvc@safesplithash@#1}%
      {\csname ekvc@safesplithash@#1\endcsname}%
      {\ekvc@err@unknown@hash{#1}\ekvcValueSplit@recover}%
  }
\long\def\ekvcValueSplit@recover#1#2%
  {%
    #2{#1}{}%
  }
%    \end{macrocode}
% \end{macro}
%
% \begin{macro}[internal]{\ekvc@safe@found@hash, \ekvc@safe@no@hash}
%    \begin{macrocode}
\long\def\ekvc@safe@found@hash#1#2\ekvc@stop#3%
  {%
    #3{#2}{#1}%
  }
\long\def\ekvc@safe@no@hash#1#2\ekvc@mark\ekvc@safe@found@hash\ekvc@stop#3%
  {%
    #3{#2}{}%
  }
%    \end{macrocode}
% \end{macro}
%
% \begin{macro}{\ekvcValueSplitFast}
%   Again a fast approach which doesn't provide too many safety measurements.
%   This needs to build the hash function and expand it before passing the
%   results to the next control sequence. The first step only builds the control
%   sequence.
%    \begin{macrocode}
\long\def\ekvcValueSplitFast#1#2%
  {%
    \csname ekvc@fastsplithash@#1\endcsname#2\ekvc@stop
  }
%    \end{macrocode}
% \end{macro}
%
% \begin{macro}[internal]
%   {
%     \ekvc@safehash@,\ekvc@fasthash@,
%     \ekvc@safesplithash@,\ekvc@fastsplithash@
%   }
%   At least in the empty hash case we can provide a meaningful error message
%   without affecting performance by just defining the macro that would be build
%   in that case. There is of course a downside to this, the error will not be
%   thrown by |\ekvcValueFast| in three expansion steps. The safe hash variant
%   has to also stop the |\unexpanded| expansion.
%    \begin{macrocode}
\long\def\ekvc@safehash@#1{\ekvc@err@empty@hash{}}
\long\def\ekvc@fasthash@#1\ekvc@stop{\ekvc@err@empty@hash}
\long\def\ekvc@safesplithash@#1#2{\ekvc@err@empty@hash#2{#1}{}}
\long\def\ekvc@fastsplithash@#1\ekvc@stop#2{\ekvc@err@empty@hash#2{#1}{}}
%    \end{macrocode}
% \end{macro}
%
% \begin{macro}{\ekvcSecondaryKeys}
%   The secondary keys are defined pretty similar to the way the originals are,
%   but here we also introduce some key types (those have a |@t@| in their name)
%   additionally to the prefixes.
%    \begin{macrocode}
\protected\long\def\ekvcSecondaryKeys#1#2%
  {%
    \edef\ekvc@set{\string#1}%
    \ekvparse\ekvc@err@value@required\ekvcSecondaryKeys@a{#2}%
  }
\protected\def\ekvcSecondaryKeys@a#1%
  {%
    \let\ekvc@long\ekv@empty
    \ekvc@ifspace{#1}%
      {\ekvcSecondaryKeys@b#1\ekvc@stop}%
      {\ekvc@err@missing@type{#1}\@gobble}%
  }
\protected\def\ekvcSecondaryKeys@b#1 #2\ekvc@stop
  {%
    \ekv@ifdefined{ekvc@p@#1}%
      {\csname ekvc@p@#1\endcsname}%
      {%
        \ekv@ifdefined{ekvc@t@#1}%
          {\csname ekvc@t@#1\endcsname}%
          {\ekvc@err@unknown@keytype{#1}\@firstoftwo\@gobble}%
      }%
      {#2}%
  }
%    \end{macrocode}
% \end{macro}
%
% \begin{macro}{\ekvcChange}
%   This can be used to change the defaults of an \expkvc\ defined macro. It
%   checks whether there is a set with the correct name and that the macro is
%   defined. If both is true the real work is done by |\ekvc@change|.
%    \begin{macrocode}
\protected\long\def\ekvcChange#1%
  {%
    \ekvifdefinedset{\string#1}%
      {%
        \ekvc@ifdefined#1%
          {\ekvc@change#1}%
          {\ekvc@err@no@key@macro#1\@gobble}%
      }%
      {\ekvc@err@no@key@macro#1\@gobble}%
  }
%    \end{macrocode}
% \end{macro}
%
% \begin{macro}[internal]
%   {\ekvc@change,\ekvc@change@a,\ekvc@change@b,\ekvc@change@c}
%   First we need to see whether the macro is currently |\long|. For this we get
%   the meaning and will parse it. |#1| is the macro name in which we want to
%   change the defaults.
%    \begin{macrocode}
\protected\def\ekvc@change#1%
  {\expandafter\ekvc@change@a\meaning#1\ekv@stop#1}
%    \end{macrocode}
%   A temporary definition to get the stringified |macro:|. |##1| will be the
%   list of prefixes, we don't care for the exact contents of |##2| and |##3|.
%    \begin{macrocode}
\def\ekvc@change@a#1%
  {%
    \protected\def\ekvc@change@a##1#1##2->##3\ekv@stop
      {%
        \ekvc@change@iflong{##1}%
          {\ekvc@change@b{}}%
          {\ekvc@change@b{\long}}%
      }%
  }
\expandafter\ekvc@change@a\expandafter{\detokenize{macro:}}
%    \end{macrocode}
%   Next we expand the macro once to get its contents (including the current
%   default values with their markers). |#1| is either |\long| or empty, |#2| is
%   the macro.
%    \begin{macrocode}
\protected\def\ekvc@change@b#1#2%
  {\expandafter\ekvc@change@c\expandafter{#2{##1}}{#1}#2}
%    \end{macrocode}
%   Here we place an unbalanced closing brace after the expansion of the macro.
%   Then we just parse the \kv-list with |\ekvset|, that will exchange the
%   values behind the markers. Once those are changed we give control to
%   |\ekvc@change@d|. The |\ekvset| step might horribly fail if the user defined
%   some keys that don't behave nice. |#1| is the expansion of the macro, |#2|
%   is either |\long| or empty, |#3| is the macro, and |#4| is the \kv-list
%   containing the new defaults.
%    \begin{macrocode}
\ekv@exparg{\protected\long\def\ekvc@change@c#1#2#3#4}%
  {%
    \expandafter\iffalse\expandafter{\expandafter\fi
      \ekvset{\string#3}{#4}%
      \ekvc@change@d{#2}{#3}%
      #1%
    }%
  }
%    \end{macrocode}
%   The final step needs to put an unbalanced opening brace after |\def|. We do
%   that with the help of a temporary macro which stores everything necessary
%   for |\def| and expand an |\iffalse}\fi| construct to leave a single
%   opening brace. |#1| will be either empty or |\long| and |#2| is the macro.
%   Each of the macros defined with \expkvc\ takes exactly one parameter, so we
%   put that here as |####1|.
%    \begin{macrocode}
\protected\def\ekvc@change@d#1#2%
  {%
    \def\ekvc@tmp{#1\def#2####1}%
    \expandafter\ekvc@tmp\expandafter{\iffalse}\fi
  }
%    \end{macrocode}
% \end{macro}
%
% \begin{macro}[internal]{\ekvc@change@iflong,\ekvc@change@iflong@}
%   Checking whether a string contains the string representation of |\long| can
%   be done by gobbling everything up to the first |\long| and checking whether
%   the result is completely empty. We need a temporary macro to get the result
%   of |\string\long| inside the definitions.
%    \begin{macrocode}
\def\ekvc@change@iflong#1%
  {%
    \protected\def\ekvc@change@iflong##1%
      {\expandafter\ekv@ifempty\expandafter{\ekvc@change@iflong@##1#1}}%
    \def\ekvc@change@iflong@##1#1{}
  }
\expandafter\ekvc@change@iflong\expandafter{\string\long}
%    \end{macrocode}
% \end{macro}
%
% \begin{macro}{\ekvcPass}
%   This macro can be used to pass a value to a key of some macro (this way
%   more complicated key codes are possible that in the end pass processed
%   values on to some macro). The implemantation is pretty straight forward.
%    \begin{macrocode}
\long\def\ekvcPass#1#2%
  {%
    \ekvifdefined{\string#1}{#2}%
      {\csname\ekv@name{\string#1}{#2}\endcsname}%
      {\ekvc@err@unknown@key@or@macro{#1}{#2}\@gobble}%
  }
%    \end{macrocode}
% \end{macro}
%
%
% \subsubsection{Secondary Key Types}
%
% \begin{macro}[internal]{\ekvc@p@long,\ekvc@after@ptype}
%   The prefixes are pretty straight forward again. Just set |\ekvc@long| and
%   forward to the |@t@| type.
%    \begin{macrocode}
\protected\def\ekvc@p@long#1%
  {%
    \ekvc@ifspace{#1}%
      {%
        \let\ekvc@long\long
        \ekvc@after@ptype#1\ekvc@stop
      }%
      {\ekvc@err@missing@type{long #1}\@gobble}%
  }
\protected\def\ekvc@after@ptype#1 #2\ekvc@stop
  {%
    \ekv@ifdefined{ekvc@t@#1}%
      {\csname ekvc@t@#1\endcsname{#2}}%
      {\ekvc@err@unknown@keytype{#1}\@gobble}%
  }
%    \end{macrocode}
% \end{macro}
%
% \begin{macro}[internal]
%   {
%     \ekvc@t@meta, \ekvc@t@nmeta,
%     \ekvc@type@meta, \ekvc@type@meta@a, \ekvc@type@meta@b
%   }
%   The |meta| and |nmeta| key types use a nested |\ekvset| to set other keys in
%   the same macro's \set.
%    \begin{macrocode}
\protected\def\ekvc@t@meta
  {%
    \edef\ekvc@tmp{\ekvc@set}%
    \expandafter\ekvc@type@meta\expandafter{\ekvc@tmp}\ekvc@long{##1}\ekvlet
  }
\protected\def\ekvc@t@nmeta#1%
  {%
    \ekvc@assert@not@long{nmeta #1}%
    \edef\ekvc@tmp{\ekvc@set}%
    \expandafter\ekvc@type@meta\expandafter{\ekvc@tmp}{}{}\ekvletNoVal{#1}%
  }
\protected\long\def\ekvc@type@meta#1#2#3#4#5#6%
  {%
    \expandafter\ekvc@type@meta@a\expandafter{\ekvset{#1}{#6}}{#2}{#3}%
    #4\ekvc@set{#5}\ekvc@tmp
  }
\protected\def\ekvc@type@meta@a
  {%
    \expandafter\ekvc@type@meta@b\expandafter
  }
\protected\long\def\ekvc@type@meta@b#1#2#3%
  {%
    #2\def\ekvc@tmp#3{#1}%
  }
%    \end{macrocode}
% \end{macro}
%
% \begin{macro}[internal]{\ekvc@t@alias}
%   |alias| just checks whether there is a key and/or |NoVal| key defined with
%   the target name and |\let| the key to those.
%    \begin{macrocode}
\protected\def\ekvc@t@alias#1#2%
  {%
    \ekvc@assert@not@long{alias #1}%
    \let\ekvc@tmp\@firstofone
    \ekvifdefined\ekvc@set{#2}%
      {%
        \ekvletkv\ekvc@set{#1}\ekvc@set{#2}%
        \let\ekvc@tmp\@gobble
      }%
      {}%
    \ekvifdefinedNoVal\ekvc@set{#2}%
      {%
        \ekvletkvNoVal\ekvc@set{#1}\ekvc@set{#2}%
        \let\ekvc@tmp\@gobble
      }%
      {}%
    \ekvc@tmp{\ekvc@err@unknown@key{#2}}%
  }
%    \end{macrocode}
% \end{macro}
%
% \begin{macro}[internal]{\ekvc@t@default}
%   The |default| key can be used to set a |NoVal| key for an existing key. It
%   will just pass the \val\ to the key macro of that other key.
%    \begin{macrocode}
\protected\long\def\ekvc@t@default#1#2%
  {%
    \ekvifdefined\ekvc@set{#1}%
      {%
        \ekvc@assert@not@long{default #1}%
        \edef\ekvc@tmp
          {%
            \unexpanded\expandafter
              {\csname\ekv@name\ekvc@set{#1}\endcsname{#2}}%
          }%
        \ekvletNoVal\ekvc@set{#1}\ekvc@tmp
      }%
      {\ekvc@err@unknown@key{#1}}%
  }
%    \end{macrocode}
% \end{macro}
%
% \begin{macro}[internal]{\ekvc@t@flag-bool}
%    \begin{macrocode}
\protected\expandafter\def\csname ekvc@t@flag-bool\endcsname#1#2%
  {%
    \ekvc@assert@not@long{flag-bool #1}%
    \unless\ifdefined#2\ekvcFlagNew#2\fi
    \ekvdef\ekvc@set{#1}%
      {%
        \ekv@ifdefined{ekvc@flag@set@##1}%
          {%
            \csname ekvc@flag@set@##1\expandafter\endcsname
              \ekvcFlagHeight#2\ekv@stop#2%
          }%
          {\ekvc@err@invalid@bool{##1}}%
      }%
  }
%    \end{macrocode}
% \end{macro}
%
% \begin{macro}[internal]
%   {\ekvc@t@flag-true,\ekvc@t@flag-false,\ekvc@t@flag-raise,\ekvc@type@flag}
%    \begin{macrocode}
\protected\def\ekvc@type@flag#1#2#3#4%
  {%
    \ekvc@assert@not@long{flag-#1 #3}%
    \unless\ifdefined#4\ekvcFlagNew#4\fi
    \ekv@exparg{\ekvdefNoVal\ekvc@set{#3}}{#2#4}%
  }
\protected\expandafter\def\csname ekvc@t@flag-true\endcsname
  {\ekvc@type@flag{true}\ekvcFlagSetTrue}
\protected\expandafter\def\csname ekvc@t@flag-false\endcsname
  {\ekvc@type@flag{false}\ekvcFlagSetFalse}
\protected\expandafter\def\csname ekvc@t@flag-raise\endcsname
  {\ekvc@type@flag{raise}\ekvcFlagRaise}
%    \end{macrocode}
% \end{macro}
%
%
% \subsubsection{Flags}
%
% The basic idea of flags is to store information by the fact that \TeX\
% expandably assigns the meaning |\relax| to undefined control sequences which
% were built with |\csname|. This mechanism is borrowed from \pkg{expl3}.
%
% \begin{macro}[internal]{\ekvc@flag@name,\ekvc@flag@namescheme}
%   Flags follow a simple naming scheme which we define here. |\ekvc@flag@name|
%   will store the name of an internal function that is used to build names of
%   the second naming scheme defined by |\ekvc@flag@namescheme|.
%    \begin{macrocode}
\def\ekvc@flag@name{ekvcf\string}
\def\ekvc@flag@namescheme#1#2{ekvch#2#1}
%    \end{macrocode}
% \end{macro}
%
% \begin{macro}{\ekvcFlagHeight}
%   For semantic reasons we use |\number| with another name.
%    \begin{macrocode}
\let\ekvcFlagHeight\number
%    \end{macrocode}
% \end{macro}
%
% \begin{macro}{\ekvcFlagNew}
%   This macro defines a new flag. It stores the function build with the
%   |\ekvc@flag@name| naming scheme after the internal function
%   |\ekvc@flag@height| that'll determine the current flag height. It'll also
%   define the macro named via |\ekvc@flag@name| to build names according to
%   |\ekvc@flag@namescheme|.
%    \begin{macrocode}
\protected\def\ekvcFlagNew#1%
  {%
    \edef#1%
      {%
        \unexpanded{\ekvc@flag@height}%
          \unexpanded\expandafter{\csname\ekvc@flag@name#1\endcsname}%
      }%
    \ekv@expargtwice
      {\expandafter\def\csname\ekvc@flag@name#1\endcsname##1}%
      {\expandafter\ekvc@flag@namescheme\expandafter{\string#1}{##1}}%
  }
%    \end{macrocode}
% \end{macro}
%
% \begin{macro}[internal]{\ekvc@flag@height,\ekvc@flag@height@}
%   This macro gets the height of a flag by a simple loop. The first loop
%   iteration differs a bit from the following in that it doesn't have to get
%   the current iteration count. The space at the end of |\ekvc@flag@height|
%   ends the |\number| evaluation.
%    \begin{macrocode}
\def\ekvc@flag@height#1%
  {%
    \ifcsname#10\endcsname
      \ekvc@flag@height@1\ekv@stop#1%
    \fi
    \@firstofone{0} % leave this space
  }
\def\ekvc@flag@height@#1\ekv@stop#2\fi\@firstofone#3%
  {%
    \fi
    \ifcsname#2{#1}\endcsname
      \expandafter\ekvc@flag@height@\the\numexpr#1+1\relax\ekv@stop#2%
    \fi
    \@firstofone{#1}%
  }
%    \end{macrocode}
% \end{macro}
%
% \begin{macro}{\ekvcFlagRaise}
%   Raising a flag simply means letting the |\ekvc@flag@namescheme| macro for
%   the current height to relax. The result of raising a flag is that its height
%   is bigger by $1$.
%    \begin{macrocode}
\ekv@exparg{\def\ekvcFlagRaise#1}%
  {%
    \expandafter\expandafter\expandafter\@gobble\expandafter
      \csname\ekvc@flag@namescheme{\string#1}{\ekvcFlagHeight#1}\endcsname
  }
%    \end{macrocode}
% \end{macro}
%
% \begin{macro}{\ekvcFlagSetTrue,\ekvcFlagSetFalse}
% \begin{macro}[internal]{\ekvc@flag@set@true,\ekvc@flag@set@false}
%   A flag is considered true if its current height is odd, and as false if it
%   is even. Therefore |\ekvcFlagSetTrue| and |\ekvcFlagSetFalse| only need to
%   raise the flag if the opposing boolean value is the current one.
%    \begin{macrocode}
\def\ekvcFlagSetTrue#1%
  {\expandafter\ekvc@flag@set@true\ekvcFlagHeight#1\ekv@stop#1}
\def\ekvcFlagSetFalse#1%
  {\expandafter\ekvc@flag@set@false\ekvcFlagHeight#1\ekv@stop#1}
%    \end{macrocode}
%   We can expand |\ekvc@flag@namescheme| at definition time here, which is why
%   we're using a temporary definition to set up |\ekvc@flag@set@true| and
%   |\ekvc@flag@set@false|.
%    \begin{macrocode}
\def\ekvc@flag@set@true#1%
  {%
    \def\ekvc@flag@set@true##1\ekv@stop##2%
      {%
        \ifodd##1
          \ekv@fi@gobble
        \fi
        \@firstofone{\expandafter\@gobble\csname#1\endcsname}%
      }%
    \def\ekvc@flag@set@false##1\ekv@stop##2%
      {%
        \ifodd##1
          \ekv@fi@firstofone
        \fi
        \@gobble{\expandafter\@gobble\csname#1\endcsname}%
      }%
  }
\expandafter\ekvc@flag@set@true\expandafter
  {\ekvc@flag@namescheme{\string#2}{#1}}
%    \end{macrocode}
% \end{macro}
% \end{macro}
%
% \begin{macro}{\ekvcFlagIf}
%   As already explained, truthiness is defined as a flag's height is odd, so we
%   just branch accordingly here.
%    \begin{macrocode}
\def\ekvcFlagIf#1%
  {%
    \ifodd#1%
      \ekv@fi@firstoftwo
    \fi
    \@secondoftwo
  }
%    \end{macrocode}
% \end{macro}
%
% \begin{macro}{\ekvcFlagIfRaised}
%   This macro uses flags as a switch, if a flag's current height is bigger than
%   $0$ this test yields true.
%    \begin{macrocode}
\ekv@exparg{\def\ekvcFlagIfRaised#1}%
  {%
    \expandafter\ifcsname\ekvc@flag@namescheme{\string#1}0\endcsname
      \ekv@fi@firstoftwo
    \fi
    \@secondoftwo
  }
%    \end{macrocode}
% \end{macro}
%
% \begin{macro}{\ekvcFlagReset}
% \begin{macro}[internal]{\ekvc@flag@reset,\ekvc@flag@reset@}
%   Resetting works by locally letting all the defined internal macros named
%   after |\ekvc@flag@namescheme| to undefined.
%    \begin{macrocode}
\protected\def\ekvcFlagReset#1%
  {\expandafter\ekvc@flag@reset\csname\ekvc@flag@name#1\endcsname}
\protected\def\ekvc@flag@reset#1%
  {%
    \ifcsname#10\endcsname
      \expandafter\let\csname#10\endcsname\ekvc@undefined
      \ekvc@flag@reset@1\ekv@stop#1%
    \fi
  }
\protected\def\ekvc@flag@reset@#1\ekv@stop#2\fi
  {%
    \fi
    \ifcsname#2{#1}\endcsname
      \expandafter\let\csname#2{#1}\endcsname\ekvc@undefined
      \expandafter\ekvc@flag@reset@\the\numexpr#1+1\relax\ekv@stop#2%
    \fi
  }
%    \end{macrocode}
% \end{macro}
% \end{macro}
%
% \begin{macro}{\ekvcFlagGetHeight}
% \begin{macro}[internal]{\ekvc@flag@get@height@single}
%   These are just small helpers, first getting the height of the flag and then
%   passing it on to the user supplied code.
%    \begin{macrocode}
\def\ekvcFlagGetHeight#1%
  {\expandafter\ekvc@flag@get@height@single\ekvcFlagHeight#1\ekv@stop}
\long\def\ekvc@flag@get@height@single#1\ekv@stop#2{#2{#1}}
%    \end{macrocode}
% \end{macro}
% \end{macro}
%
% \begin{macro}{\ekvcFlagGetHeights}
% \begin{macro}[internal]
%   {\ekvc@flag@get@heights,\ekvc@flag@get@heights@,\ekvc@flag@get@heights@done}
%   This works by a simple loop that stops at |\ekv@stop|. As long as that
%   marker isn't hit, get the next flags height and put it into a list after
%   |\ekv@stop|. |\ekvc@flag@get@heights@| uses the same marker name for the
%   end of the height, which shouldn't clash in any case. Once we're done we
%   remove the remainder of the current iteration and leave the user supplied
%   code in the input stream with all the flags' heights as a single argument.
%    \begin{macrocode}
\def\ekvcFlagGetHeights#1%
  {%
    \ekvc@flag@get@heights#1\ekv@stop{}%
  }
\def\ekvc@flag@get@heights#1%
  {%
    \ekv@gobbleto@stop#1\ekvc@flag@get@heights@done\ekv@stop
    \expandafter\ekvc@flag@get@heights@\ekvcFlagHeight#1\ekv@stop
  }
\def\ekvc@flag@get@heights@#1\ekv@stop#2\ekv@stop#3%
  {\ekvc@flag@get@heights#2\ekv@stop{#3{#1}}}
\long\def\ekvc@flag@get@heights@done
    \ekv@stop
    \expandafter\ekvc@flag@get@heights@\ekvcFlagHeight\ekv@stop\ekv@stop#1#2%
  {#2{#1}}
%    \end{macrocode}
% \end{macro}
% \end{macro}
%
%
% \subsubsection{Helper Macros}
%
% \begin{macro}[internal]{\ekvc@ifspace,\ekvc@ifspace@}
% A test which can be reduced to an if-empty by gobbling everything up to the
% first space.
%    \begin{macrocode}
\long\def\ekvc@ifspace#1%
  {%
    \ekvc@ifspace@#1 \ekv@ifempty@B
      \ekv@ifempty@false\ekv@ifempty@A\ekv@ifempty@B\@firstoftwo
  }
\long\def\ekvc@ifspace@#1 % keep this space
  {%
    \ekv@ifempty@\ekv@ifempty@A
  }
%    \end{macrocode}
% \end{macro}
%
%
% \subsubsection{Assertions}
% \begin{macro}[internal]{\ekvc@assert@not@long}
%   Some keys don't want to be |long| and we have to educate the user, so let's
%   throw an error if someone wanted these to be long.
%    \begin{macrocode}
\long\def\ekvc@assert@not@long#1{\ifx\ekvc@long\long\ekvc@err@no@long{#1}\fi}
%    \end{macrocode}
% \end{macro}
%
%
% \subsubsection{Messages}
%
% \begin{macro}[internal]
%   {
%     \ekvc@err@toomany,\ekvc@err@value@required,\ekvc@err@missing@type,
%     \ekvc@err@already@defined,\ekvc@err@no@key@macro
%   }
%   Boring unexpandable error messages.
%    \begin{macrocode}
\protected\def\ekvc@err@toomany#1%
  {%
    \errmessage{expkv-cs Error: Too many keys for macro `\string#1'}%
  }
\protected\def\ekvc@err@value@required#1%
  {%
    \errmessage{expkv-cs Error: Missing value for key `\unexpanded{#1}'}%
  }
\protected\def\ekvc@err@missing@type#1%
  {%
    \errmessage
      {expkv-cs Error: Missing type for secondary key `\unexpanded{#1}'}%
  }
\protected\def\ekvc@err@no@long#1%
  {%
    \errmessage
      {expkv-cs Error: prefix `long' not accepted for `\unexpanded{#1}'}%
  }
\protected\def\ekvc@err@already@defined#1%
  {%
    \errmessage{expkv-cs Error: Macro `\string#1' already defined}%
  }
\protected\def\ekvc@err@unknown@keytype#1%
  {%
    \errmessage{expkv-cs Error: Unknown key type `\unexpanded{#1}'}%
  }
\protected\def\ekvc@err@unknown@key#1%
  {%
    \errmessage
      {expkv-cs Error: Unknown key `\unexpanded{#1}' for macro `\ekvc@set'}%
  }
\long\def\ekvc@err@no@key@macro#1%
  {\errmessage{expkv-cs Error: \string#1 is no key=val macro}}
%    \end{macrocode}
% \end{macro}
%
% \begin{macro}[internal]{\ekvc@err}
%   We need a way to throw error messages expandably in some contexts.
%    \begin{macrocode}
\ekv@exparg{\long\def\ekvc@err#1}{\ekverr{expkv-cs}{#1}}
%    \end{macrocode}
% \end{macro}
%
% \begin{macro}[internal]
%   {
%     \ekvc@err@unknown@hash,\ekvc@err@empty@hash,\ekvc@err@missing@hash,
%     \ekvc@err@invalid@bool
%   }
%   And here are the expandable error messages.
%    \begin{macrocode}
\long\def\ekvc@err@unknown@hash#1{\ekvc@err{unknown hash `#1'}}
\long\def\ekvc@err@missing@hash#1{\ekvc@err{hash `#1' not found}}
\long\def\ekvc@err@empty@hash{\ekvc@err{empty hash}}
\def\ekvc@err@invalid@bool#1{\ekvc@err{invalid boolean value `#1'}}
\long\def\ekvc@err@unknown@key@or@macro#1#2%
  {\ekvc@err{unknown key `#2' for macro #1}}
%    \end{macrocode}
% \end{macro}
%
%
% Now everything that's left is to reset the category code of |@|.
%    \begin{macrocode}
\catcode`\@=\ekvc@tmp
%    \end{macrocode}
%
% \gobbledocstriptag
%</tex>
%^^A=<<
%
%^^A=<<
%
% \end{implementation}^^A=<<
%
% \clearpage
% \PrintIndex
%
\endinput
%
^^A vim: ft=tex fdm=marker fmr=>>=,=<<

\ProvidesPackage{expkv-cs}%
  [%
    \ekvcDate\space v\ekvcVersion\space
    define expandable key=val macros using expkv%
  ]
%    \end{macrocode}
% \gobbledocstriptag
%</pkg>
%^^A=<<
%^^A main file >>=
% \subsection{The Generic Code}
% The rest of this implementation will be the generic code.
% \gobbledocstriptag
%<*tex>
%
% Load \expkv\ if the package didn't already do so -- since \expkv\ has
% safeguards against being loaded twice this does no harm and the overhead
% isn't that big. Also we reuse some of the internals of \expkv\ to save us from
% retyping them.
%    \begin{macrocode}
\input expkv
%    \end{macrocode}
%
% We make sure that \file{expkv-cs.tex} is only input once:
%    \begin{macrocode}
\expandafter\ifx\csname ekvcVersion\endcsname\relax
\else
  \expandafter\endinput
\fi
%    \end{macrocode}
%
% \begin{macro}{\ekvcVersion,\ekvcDate}
% We're on our first input, so lets store the version and date in a macro.
%    \begin{macrocode}
\def\ekvcVersion{0.10}
\def\ekvcDate{2021-06-03}
%    \end{macrocode}
% \end{macro}
%
% If the \LaTeX\ format is loaded we want to be a good file and report back who
% we are, for this the package will have defined |\ekvc@tmp| to use
% |\ProvidesFile|, else this will expand to a |\relax| and do no harm.
%    \begin{macrocode}
\csname ekvc@tmp\endcsname
%    \end{macrocode}
%
% Store the category code of |@| to later be able to reset it and change it to
% 11 for now.
%    \begin{macrocode}
\expandafter\chardef\csname ekvc@tmp\endcsname=\catcode`\@
\catcode`\@=11
%    \end{macrocode}
% |\ekvc@tmp| will be reused later, but we don't need it to ever store
% information long-term after \expkvc\ was initialized.
%
% \begin{macro}[internal]{\ekvc@keycount}
%   We'll need to keep count how many keys must be defined for each macro in the
%   |split| variants.
%    \begin{macrocode}
\newcount\ekvc@keycount
%    \end{macrocode}
% \end{macro}
%
% \begin{macro}[internal]{\ekvc@long,\ekvc@any@long}
%   Some macros will have to be defined long. These two will be let to |\long|
%   when this should be the case.
%    \begin{macrocode}
\let\ekvc@long\ekv@empty
\let\ekvc@any@long\ekv@empty
%    \end{macrocode}
% \end{macro}
%
% \begin{macro}[internal]{\ekvc@ifdefined}
%   We want to test whether a macro is already defined. This test checks for a
%   defined macro that isn't |\relax|.
%    \begin{macrocode}
\long\def\ekvc@ifdefined#1%
  {%
    \ifdefined#1%
      \ifx\relax#1%
        \ekv@fi@gobble
      \fi
      \@firstofone
      \ekv@fi@firstoftwo
    \fi
    \@secondoftwo
  }
%    \end{macrocode}
% \end{macro}
%
% \begin{macro}[internal]
%   {
%     \ekvc@ekvset@pre@expander,
%     \ekvc@ekvset@pre@expander@a,
%     \ekvc@ekvset@pre@expander@b
%   }
%   This macro expands |\ekvset| twice so that the first two steps of expansion
%   don't have to be made every time the \expkvc\ macros are used. We have to do
%   a little magic trick to get the macro parameter |#1| for the macro
%   definition this is used in, even though we're calling |\unexpanded|. We do
%   that by splitting the expanded |\ekvset| at some marks and place |##1| in
%   between. At this spot we also add |\ekv@alignsafe| and |\ekv@endalignsafe|
%   to ensure that macros created with \expkvc\ are alignment safe.
%    \begin{macrocode}
\def\ekvc@ekvset@pre@expander#1%
  {%
    \expandafter\ekvc@ekvset@pre@expander@a\ekvset{#1}\ekvc@stop\ekvc@stop
  }
\def\ekvc@ekvset@pre@expander@a
  {%
    \expandafter\ekvc@ekvset@pre@expander@b
  }
\def\ekvc@ekvset@pre@expander@b#1\ekvc@stop#2\ekvc@stop
  {%
    \unexpanded\expandafter{\ekv@alignsafe}%
    \unexpanded{#1}##1\unexpanded{#2}%
    \unexpanded\expandafter{\ekv@endalignsafe}%
  }
%    \end{macrocode}
% \end{macro}
%
% \begin{macro}{\ekvcSplitAndUse}
%   The first user macro we want to set up can be reused for
%   \cs[no-index]{ekvcSplitAndForward} and \cs[no-index]{ekvcSplit}. We'll split
%   this one up so that the test whether the macro is already defined doesn't
%   run twice.
%    \begin{macrocode}
\protected\long\def\ekvcSplitAndUse#1#2%
  {%
    \let\ekvc@helpers@needed\@firstoftwo
    \ekvc@ifdefined#1%
      {\ekvc@err@already@defined#1}%
      {\ekvcSplitAndUse@#1{}{#2}}%
  }
%    \end{macrocode}
% \end{macro}
%
% \begin{macro}[internal]{\ekvcSplitAndUse@}
%   The actual macro setting up things. We need to set some variables, forward
%   the key list to |\ekvc@SetupSplitKeys|, and afterwards define the front
%   facing macro to call |\ekvset| and put the initials and the argument sorting
%   macro behind it. The internals |\ekvc@any@long|, |\ekvc@initials| and
%   |\ekvc@keycount| will be set correctly by |\ekvc@SetupSplitKeys|.
%    \begin{macrocode}
\protected\long\def\ekvcSplitAndUse@#1#2#3%
  {%
    \edef\ekvc@set{\string#1}%
    \ekvc@SetupSplitKeys{#3}%
    \ekvc@helpers@needed
      {%
        \ekvc@any@long\edef#1##1%
          {%
            \expandafter\ekvc@ekvset@pre@expander\expandafter{\ekvc@set}%
            \unexpanded\expandafter
              {\csname ekvc@split@\the\ekvc@keycount\endcsname}%
            \unexpanded\expandafter{\ekvc@initials{}#2}%
          }%
      }%
      {%
        \ekvc@any@long\edef#1##1%
          {%
            \expandafter\ekvc@ekvset@pre@expander\expandafter{\ekvc@set}%
            \unexpanded{#2}%
            \unexpanded\expandafter{\ekvc@initials}%
          }%
      }%
  }
%    \end{macrocode}
% \end{macro}
%
%
% \begin{macro}{\ekvcSplitAndForward}
%   This just reuses |\ekvcSplitAndUse@| with a non-empty second argument,
%   resulting in that argument to be called after the splitting.
%    \begin{macrocode}
\protected\long\def\ekvcSplitAndForward#1#2#3%
  {%
    \let\ekvc@helpers@needed\@firstoftwo
    \ekvc@ifdefined#1%
      {\ekvc@err@already@defined#1}%
      {\ekvcSplitAndUse@#1{{#2}}{#3}}%
  }
%    \end{macrocode}
% \end{macro}
%
% \begin{macro}{\ekvcSplit}
%   The first half is just |\ekvcSplitAndForward| then we define the macro to
%   which the parsed key list is forwarded. There we need to allow for up to
%   nine arguments.
%    \begin{macrocode}
\protected\long\def\ekvcSplit#1#2#3%
  {%
    \let\ekvc@helpers@needed\@secondoftwo
    \ekvc@ifdefined#1%
      {\ekvc@err@already@defined#1}%
      {%
        \expandafter
        \ekvcSplitAndUse@\expandafter#1\csname ekvc@\string#1\endcsname{#2}%
        \ifnum\ekvc@keycount<1
          \ekvc@any@long\expandafter\def\csname ekvc@\string#1\endcsname{#3}%
        \else
          \ifnum\ekvc@keycount>9
            \ekvc@err@toomany{#1}%
            \let#1\ekvc@undefined
          \else
            \ekvcSplit@build@argspec
            \ekvc@any@long\expandafter
            \def\csname ekvc@\string#1\expandafter\endcsname\ekvc@tmp{#3}%
          \fi
        \fi
      }%
  }
%    \end{macrocode}
% \end{macro}
%
% \begin{macro}[internal]{\ekvcSplit@build@argspec,\ekvcSplit@build@argspec@}
%    \begin{macrocode}
\protected\def\ekvcSplit@build@argspec
  {%
    \begingroup
      \edef\ekvc@tmp
        {\endgroup\def\unexpanded{\ekvc@tmp}{\ekvcSplit@build@argspec@{1}}}%
    \ekvc@tmp
  }
\def\ekvcSplit@build@argspec@#1%
  {%
    \ifnum#1>\ekvc@keycount
      \ekv@fi@gobble
    \fi
    \@firstofone
      {%
        \unexpanded\expandafter{\csname ekvc@splitmark@#1\endcsname####}#1%
        \expandafter\ekvcSplit@build@argspec@\expandafter{\the\numexpr#1+1}%
      }%
  }
%    \end{macrocode}
% \end{macro}
%
% \begin{macro}[internal]
%   {
%     \ekvc@SetupSplitKeys, \ekvc@SetupSplitKeys@a, \ekvc@SetupSplitKeys@b,
%     \ekvc@SetupSplitKeys@c
%   }
%   These macros parse the list of keys and set up the key macros. First we need
%   to initialise some macros and start |\ekvparse|.
%    \begin{macrocode}
\protected\long\def\ekvc@SetupSplitKeys
  {%
    \ekvc@keycount=0
    \let\ekvc@any@long\ekv@empty
    \let\ekvc@initials\ekv@empty
    \ekvparse\ekvc@err@value@required\ekvc@SetupSplitKeys@a
  }
%    \end{macrocode}
%   Then we need to step the key counter for each key. Also we have to check
%   whether this key has a |long| prefix so we initialise |\ekvc@long|.
%    \begin{macrocode}
\protected\def\ekvc@SetupSplitKeys@a#1%
  {%
    \advance\ekvc@keycount1
    \let\ekvc@long\ekv@empty
    \ekvc@ifspace{#1}%
      {\ekvc@SetupSplitKeys@b#1\ekvc@stop}%
      {\ekvc@SetupSplitKeys@c{#1}}%
  }
%    \end{macrocode}
%   If there was a space, there might be a prefix. If so call the prefix macro,
%   else call the next step |\ekvc@SetupSplitKeys@c| which will define the key
%   macro and add the key's value to the initials list.
%    \begin{macrocode}
\protected\def\ekvc@SetupSplitKeys@b#1 #2\ekvc@stop
  {%
    \ekv@ifdefined{ekvc@split@p@#1}%
      {\csname ekvc@split@p@#1\endcsname{#2}}%
      {\ekvc@SetupSplitKeys@c{#1 #2}}%
  }
%    \end{macrocode}
%   The inner definition is grouped, because we don't want to actually define
%   the marks we build with |\csname|. We have to append the value to the
%   |\ekvc@initials| list here with the correct split mark. The key macro will
%   read everything up to those split marks and change the value following it to
%   the value given to the key. Additionally we'll need a sorting macro for each
%   key count in use so we set it up with |\ekvc@setup@splitmacro|.
%    \begin{macrocode}
\protected\long\def\ekvc@SetupSplitKeys@c#1#2%
  {%
    \begingroup
    \edef\ekvc@tmp
      {%
        \endgroup
        \long\def\unexpanded{\ekvc@tmp}####1####2%
            \unexpanded\expandafter
            {\csname ekvc@splitmark@\the\ekvc@keycount\endcsname}####3%
          {%
            ####2%
            \unexpanded\expandafter
              {\csname ekvc@splitmark@\the\ekvc@keycount\endcsname}{####1}%
          }%
%    \end{macrocode}
%   The short variant needs a bit of special treatment. The key macro will be
%   short to throw the correct error, but since there might be long macros
%   somewhere the reordering of arguments needs to be long, so for short keys we
%   use a two step approach, first grabbing only the short argument, then
%   reordering.
%    \begin{macrocode}
        \unless\ifx\ekvc@long\long
          \let\unexpanded\expandafter
            {\csname ekvc@\ekvc@set(#1)\endcsname\ekvc@tmp}%
          \def\unexpanded{\ekvc@tmp}####1%
            {%
              \unexpanded\expandafter{\csname ekvc@\ekvc@set(#1)\endcsname}%
                {####1}%
            }%
        \fi
        \def\unexpanded{\ekvc@initials}%
          {%
            \unexpanded\expandafter{\ekvc@initials}%
            \unexpanded\expandafter
              {\csname ekvc@splitmark@\the\ekvc@keycount\endcsname{#2}}%
          }%
      }%
    \ekvc@tmp
    \ekvlet\ekvc@set{#1}\ekvc@tmp
    \ekvc@helpers@needed
      {\expandafter\ekvc@setup@splitmacro\expandafter{\the\ekvc@keycount}}%
      {}%
  }
%    \end{macrocode}
% \end{macro}
%
% \begin{macro}[internal]{\ekvc@split@p@long}
%   The |long| prefix lets the internals |\ekvc@long| and |\ekvc@any@long| to
%   |\long| so that the key macro will be long.
%    \begin{macrocode}
\protected\def\ekvc@split@p@long
  {%
    \let\ekvc@long\long
    \let\ekvc@any@long\long
    \ekvc@SetupSplitKeys@c
  }
%    \end{macrocode}
% \end{macro}
%
% \begin{macro}[internal]{\ekvc@defarggobbler}
%   This is needed to define a macro with 1-9 parameters programmatically.
%   \LaTeX's \cs[no-index]{newcommand} does something similar for example.
%    \begin{macrocode}
\protected\def\ekvc@defarggobbler#1{\def\ekvc@tmp##1#1##2##{##1#1}}
%    \end{macrocode}
% \end{macro}
%
% \begin{macro}[internal]
%   {
%     \ekvc@setup@splitmacro,
%     \ekvc@split@1, \ekvc@split@2, \ekvc@split@3,
%     \ekvc@split@4, \ekvc@split@5, \ekvc@split@6,
%     \ekvc@split@7
%   }
%   Since the first few split macros are different from the others we manually
%   set those up now. All the others will be defined as needed (always
%   globally).  The split macros just read up until the correct split mark, move
%   that argument into a list and reinsert the rest, calling the next split
%   macro afterwards.
%    \begin{macrocode}
\begingroup
\edef\ekvc@tmp
  {%
    \long\gdef\unexpanded\expandafter{\csname ekvc@split@1\endcsname}%
        \unexpanded\expandafter{\csname ekvc@splitmark@1\endcsname}%
        ##1##2##3%
      {##3{##1}##2}%
    \long\gdef\unexpanded\expandafter{\csname ekvc@split@2\endcsname}%
        \unexpanded\expandafter{\csname ekvc@splitmark@1\endcsname}##1%
        \unexpanded\expandafter{\csname ekvc@splitmark@2\endcsname}##2%
        ##3##4%
      {##4{##1}{##2}##3}%
    \long\gdef\unexpanded\expandafter{\csname ekvc@split@3\endcsname}%
        \unexpanded\expandafter{\csname ekvc@splitmark@1\endcsname}##1%
        \unexpanded\expandafter{\csname ekvc@splitmark@2\endcsname}##2%
        \unexpanded\expandafter{\csname ekvc@splitmark@3\endcsname}##3%
        ##4##5%
      {##5{##1}{##2}{##3}##4}%
    \long\gdef\unexpanded\expandafter{\csname ekvc@split@4\endcsname}%
        \unexpanded\expandafter{\csname ekvc@splitmark@1\endcsname}##1%
        \unexpanded\expandafter{\csname ekvc@splitmark@2\endcsname}##2%
        \unexpanded\expandafter{\csname ekvc@splitmark@3\endcsname}##3%
        \unexpanded\expandafter{\csname ekvc@splitmark@4\endcsname}##4%
        ##5##6%
      {##6{##1}{##2}{##3}{##4}##5}%
    \long\gdef\unexpanded\expandafter{\csname ekvc@split@5\endcsname}%
        \unexpanded\expandafter{\csname ekvc@splitmark@1\endcsname}##1%
        \unexpanded\expandafter{\csname ekvc@splitmark@2\endcsname}##2%
        \unexpanded\expandafter{\csname ekvc@splitmark@3\endcsname}##3%
        \unexpanded\expandafter{\csname ekvc@splitmark@4\endcsname}##4%
        \unexpanded\expandafter{\csname ekvc@splitmark@5\endcsname}##5%
        ##6##7%
      {##7{##1}{##2}{##3}{##4}{##5}##6}%
    \long\gdef\unexpanded\expandafter{\csname ekvc@split@6\endcsname}%
        \unexpanded\expandafter{\csname ekvc@splitmark@1\endcsname}##1%
        \unexpanded\expandafter{\csname ekvc@splitmark@2\endcsname}##2%
        \unexpanded\expandafter{\csname ekvc@splitmark@3\endcsname}##3%
        \unexpanded\expandafter{\csname ekvc@splitmark@4\endcsname}##4%
        \unexpanded\expandafter{\csname ekvc@splitmark@5\endcsname}##5%
        \unexpanded\expandafter{\csname ekvc@splitmark@6\endcsname}##6%
        ##7##8%
      {##8{##1}{##2}{##3}{##4}{##5}{##6}##7}%
    \long\gdef\unexpanded\expandafter{\csname ekvc@split@7\endcsname}%
        \unexpanded\expandafter{\csname ekvc@splitmark@1\endcsname}##1%
        \unexpanded\expandafter{\csname ekvc@splitmark@2\endcsname}##2%
        \unexpanded\expandafter{\csname ekvc@splitmark@3\endcsname}##3%
        \unexpanded\expandafter{\csname ekvc@splitmark@4\endcsname}##4%
        \unexpanded\expandafter{\csname ekvc@splitmark@5\endcsname}##5%
        \unexpanded\expandafter{\csname ekvc@splitmark@6\endcsname}##6%
        \unexpanded\expandafter{\csname ekvc@splitmark@7\endcsname}##7%
        ##8##9%
      {##9{##1}{##2}{##3}{##4}{##5}{##6}{##7}##8}%
  }
\ekvc@tmp
\endgroup
\protected\def\ekvc@setup@splitmacro#1%
  {%
    \ekv@ifdefined{ekvc@split@#1}{}%
      {%
        \begingroup
          \edef\ekvc@tmp
            {%
              \long\gdef
                  \unexpanded\expandafter{\csname ekvc@split@#1\endcsname}%
                  ####1%
                  \unexpanded\expandafter{\csname ekvc@splitmark@#1\endcsname}%
                  ####2####3%
                {%
                  \unexpanded\expandafter
                    {\csname ekvc@split@\the\numexpr#1-1\relax\endcsname}%
                  ####1{{####2}####3}%
                }%
            }%
          \ekvc@tmp
        \endgroup
      }%
  }
%    \end{macrocode}
% \end{macro}
%
% \begin{macro}{\ekvcHashAndUse}
%   |\ekvcHashAndUse| works just like |\ekvcSplitAndUse|.
%    \begin{macrocode}
\protected\long\def\ekvcHashAndUse#1#2%
  {%
    \let\ekvc@helpers@needed\@firstoftwo
    \ekvc@ifdefined#1%
      {\ekvc@err@already@defined#1}%
      {\ekvcHashAndUse@#1{}{#2}}%
  }
%    \end{macrocode}
% \end{macro}
%
% \begin{macro}[internal]{\ekvcHashAndUse@}
%   This is more or less the same as |\ekvcSplitAndUse@|. Instead of an empty
%   group we place a marker after the initials, we don't use the sorting macros
%   of |split|, but instead pack all the values in one argument.
%    \begin{macrocode}
\protected\long\def\ekvcHashAndUse@#1#2#3%
  {%
    \edef\ekvc@set{\string#1}%
    \ekvc@SetupHashKeys{#3}%
    \ekvc@helpers@needed
      {%
        \ekvc@any@long\edef#1##1%
          {%
            \expandafter\ekvc@ekvset@pre@expander\expandafter{\ekvc@set}%
            \unexpanded{\ekvc@hash@pack@argument}%
            \unexpanded\expandafter{\ekvc@initials\ekvc@stop#2}%
          }%
      }%
      {%
        \ekvc@any@long\edef#1##1%
          {%
            \expandafter\ekvc@ekvset@pre@expander\expandafter{\ekvc@set}%
            \unexpanded{#2}%
            \unexpanded\expandafter{\ekvc@initials\ekvc@stop}%
          }%
      }%
  }
%    \end{macrocode}
% \end{macro}
%
% \begin{macro}{\ekvcHashAndForward}
%   |\ekvcHashAndForward| works just like |\ekvcSplitAndForward|.
%    \begin{macrocode}
\protected\long\def\ekvcHashAndForward#1#2#3%
  {%
    \let\ekvc@helpers@needed\@firstoftwo
    \ekvc@ifdefined#1%
      {\ekvc@err@already@defined#1}%
      {\ekvcHashAndUse@#1{{#2}}{#3}}%
  }
%    \end{macrocode}
% \end{macro}
%
% \begin{macro}{\ekvcHash}
%   |\ekvcHash| does the same as |\ekvcSplit|, but has the advantage of not
%   needing to count arguments, so the definition of the internal macro is a bit
%   more straight forward.
%    \begin{macrocode}
\protected\long\def\ekvcHash#1#2#3%
  {%
    \let\ekvc@helpers@needed\@secondoftwo
    \ekvc@ifdefined#1%
      {\ekvc@err@already@defined#1}%
      {%
        \expandafter
        \ekvcHashAndUse@\expandafter#1\csname ekvc@\string#1\endcsname{#2}%
        \ekvc@any@long\expandafter\def\csname ekvc@\string#1\endcsname
            ##1\ekvc@stop
          {#3}%
      }%
  }
%    \end{macrocode}
% \end{macro}
%
% \begin{macro}[internal]{\ekvc@hash@pack@argument}
%   All this macro does is pack the values into one argument and forward that to
%   the next macro.
%    \begin{macrocode}
\long\def\ekvc@hash@pack@argument#1\ekvc@stop#2{#2{#1}}
%    \end{macrocode}
% \end{macro}
%
% \begin{macro}[internal]
%   {\ekvc@SetupHashKeys,\ekvc@SetupHashKeys@a,\ekvc@SetupHashKeys@b}
%   This should look awfully familiar as well, since it's just the same as for
%   the split keys with a few other names here and there.
%    \begin{macrocode}
\protected\long\def\ekvc@SetupHashKeys#1%
  {%
    \let\ekvc@any@long\ekv@empty
    \let\ekvc@initials\ekv@empty
    \ekvparse\ekvc@err@value@required\ekvc@SetupHashKeys@a{#1}%
  }
\protected\def\ekvc@SetupHashKeys@a#1%
  {%
    \let\ekvc@long\ekv@empty
    \ekvc@ifspace{#1}%
      {\ekvc@SetupHashKeys@b#1\ekvc@stop}%
      {\ekvc@SetupHashKeys@c{#1}}%
  }
\protected\def\ekvc@SetupHashKeys@b#1 #2\ekvc@stop
  {%
    \ekv@ifdefined{ekvc@hash@p@#1}%
      {\csname ekvc@hash@p@#1\endcsname{#2}}%
      {\ekvc@SetupHashKeys@c{#1 #2}}%
  }
%    \end{macrocode}
%   Yes, even the defining macro looks awfully familiar. Instead of numbered we
%   have named marks. Still the key macros grab everything up to their
%   respective mark and reorder the arguments. The same quirk is applied for
%   short keys. And instead of the |\ekvc@setup@splitmacro| we use
%   |\ekvc@setup@hashmacro|.
%    \begin{macrocode}
\protected\long\def\ekvc@SetupHashKeys@c#1#2%
  {%
    \begingroup
    \edef\ekvc@tmp
      {%
        \endgroup
        \long\def\unexpanded{\ekvc@tmp}####1####2%
            \unexpanded\expandafter{\csname ekvc@hashmark@#1\endcsname}####3%
          {%
            ####2%
            \unexpanded\expandafter{\csname ekvc@hashmark@#1\endcsname}{####1}%
          }%
        \unless\ifx\ekvc@long\long
          \let\unexpanded\expandafter
            {\csname ekvc@\ekvc@set(#1)\endcsname\ekvc@tmp}%
          \def\unexpanded{\ekvc@tmp}####1%
            {%
              \unexpanded\expandafter{\csname ekvc@\ekvc@set(#1)\endcsname}%
                {####1}%
            }%
        \fi
        \def\unexpanded{\ekvc@initials}%
          {%
            \unexpanded\expandafter{\ekvc@initials}%
            \unexpanded\expandafter{\csname ekvc@hashmark@#1\endcsname{#2}}%
          }%
      }%
    \ekvc@tmp
    \ekvlet\ekvc@set{#1}\ekvc@tmp
    \ekvc@setup@hashmacro{#1}%
  }
%    \end{macrocode}
% \end{macro}
%
% \begin{macro}[internal]{\ekvc@hash@p@long}
%   Nothing astonishing here either.
%    \begin{macrocode}
\protected\def\ekvc@hash@p@long
  {%
    \let\ekvc@long\long
    \let\ekvc@any@long\long
    \ekvc@SetupHashKeys@c
  }
%    \end{macrocode}
% \end{macro}
%
% \begin{macro}[internal]{\ekvc@setup@hashmacro}
%   The safe hash macros will be executed inside of an |\unexpanded| expansion
%   context, so they have to insert braces for that once they are done.
%   Most of the tests which have to be executed will already be done, but we
%   have to play safe if the hash doesn't show up in the hash list. Therefore we
%   use some |\ekvc@mark|s and |\ekvc@stop| to throw errors if the hash isn't
%   found in the right place. The fast variants have an easier life and just
%   return the correct value.
%    \begin{macrocode}
\protected\def\ekvc@setup@hashmacro#1%
  {%
    \ekv@ifdefined{ekvc@fasthash@#1}{}%
      {%
        \begingroup
          \edef\ekvc@tmp
            {%
              \long\gdef
                \unexpanded\expandafter{\csname ekvc@fasthash@#1\endcsname}%
                  ####1%
                  \unexpanded\expandafter{\csname ekvc@hashmark@#1\endcsname}%
                  ####2####3\unexpanded{\ekvc@stop}%
                {####2}%
              \long\gdef
                \unexpanded\expandafter{\csname ekvc@safehash@#1\endcsname}%
                  ####1%
                {%
                  \unexpanded\expandafter{\csname ekvc@@safehash@#1\endcsname}%
                  ####1\unexpanded{\ekvc@mark}{}%
                  \unexpanded\expandafter
                    {%
                      \csname ekvc@hashmark@#1\endcsname{}%
                      \ekvc@mark{\ekvc@err@missing@hash{#1}}\ekvc@stop
                    }%
                }%
              \long\gdef
                \unexpanded\expandafter{\csname ekvc@@safehash@#1\endcsname}%
                  ####1%
                  \unexpanded\expandafter{\csname ekvc@hashmark@#1\endcsname}%
                  ####2####3\unexpanded{\ekvc@mark}####4####5%
                  \unexpanded{\ekvc@stop}%
                {%
                  ####4{####2}%
                }%
              \long\gdef\unexpanded\expandafter
                {\csname ekvc@fastsplithash@#1\endcsname}%
                  ####1%
                  \unexpanded\expandafter{\csname ekvc@hashmark@#1\endcsname}%
                  ####2####3\unexpanded{\ekvc@stop}####4%
                {%
                  ####4{####1####3}{####2}%
                }%
              \long\gdef\unexpanded\expandafter
                {\csname ekvc@safesplithash@#1\endcsname}####1%
                {%
                  \unexpanded\expandafter
                    {\csname ekvc@@safesplithash@#1\endcsname}%
                  ####1\unexpanded{\ekvc@mark\ekvc@safe@found@hash}%
                  \unexpanded\expandafter
                    {%
                      \csname ekvc@hashmark@#1\endcsname{}%
                      \ekvc@mark{\ekvc@err@missing@hash{#1}\ekvc@safe@no@hash}%
                      \ekvc@stop
                    }%
                }%
              \long\gdef\unexpanded\expandafter
                {\csname ekvc@@safesplithash@#1\endcsname}%
                  ####1%
                  \unexpanded\expandafter{\csname ekvc@hashmark@#1\endcsname}%
                  ####2####3\unexpanded{\ekvc@mark}####4####5%
                  \unexpanded{\ekvc@stop}%
                {%
                  ####4{####2}####1####3\unexpanded{\ekvc@stop}%
                }%
            }%
          \ekvc@tmp
        \endgroup
      }%
  }
%    \end{macrocode}
% \end{macro}
%
% \begin{macro}{\ekvcValue}
%   All this does is a few consistency checks on the first argument (not empty,
%   hash macro exists) and then call that hash-grabbing macro that will also
%   test whether the hash is inside of |#2| or not.
%    \begin{macrocode}
\long\def\ekvcValue#1%
  {%
    \unexpanded
    \ekv@ifdefined{ekvc@safehash@#1}%
      {\csname ekvc@safehash@#1\endcsname}%
      {\ekvc@err@unknown@hash{#1}\@firstoftwo{{}}}%
  }
%    \end{macrocode}
% \end{macro}
%
% \begin{macro}{\ekvcValueFast}
%   To be as fast as possible, this doesn't test for anything, assuming the user
%   knows best.
%    \begin{macrocode}
\long\def\ekvcValueFast#1#2{\csname ekvc@fasthash@#1\endcsname#2\ekvc@stop}
%    \end{macrocode}
% \end{macro}
%
% \begin{macro}{\ekvcValueSplit,\ekvcValueSplit@recover}
%   This splits off a single value.
%    \begin{macrocode}
\long\def\ekvcValueSplit#1%
  {%
    \ekv@ifdefined{ekvc@safesplithash@#1}%
      {\csname ekvc@safesplithash@#1\endcsname}%
      {\ekvc@err@unknown@hash{#1}\ekvcValueSplit@recover}%
  }
\long\def\ekvcValueSplit@recover#1#2%
  {%
    #2{#1}{}%
  }
%    \end{macrocode}
% \end{macro}
%
% \begin{macro}[internal]{\ekvc@safe@found@hash, \ekvc@safe@no@hash}
%    \begin{macrocode}
\long\def\ekvc@safe@found@hash#1#2\ekvc@stop#3%
  {%
    #3{#2}{#1}%
  }
\long\def\ekvc@safe@no@hash#1#2\ekvc@mark\ekvc@safe@found@hash\ekvc@stop#3%
  {%
    #3{#2}{}%
  }
%    \end{macrocode}
% \end{macro}
%
% \begin{macro}{\ekvcValueSplitFast}
%   Again a fast approach which doesn't provide too many safety measurements.
%   This needs to build the hash function and expand it before passing the
%   results to the next control sequence. The first step only builds the control
%   sequence.
%    \begin{macrocode}
\long\def\ekvcValueSplitFast#1#2%
  {%
    \csname ekvc@fastsplithash@#1\endcsname#2\ekvc@stop
  }
%    \end{macrocode}
% \end{macro}
%
% \begin{macro}[internal]
%   {
%     \ekvc@safehash@,\ekvc@fasthash@,
%     \ekvc@safesplithash@,\ekvc@fastsplithash@
%   }
%   At least in the empty hash case we can provide a meaningful error message
%   without affecting performance by just defining the macro that would be build
%   in that case. There is of course a downside to this, the error will not be
%   thrown by |\ekvcValueFast| in three expansion steps. The safe hash variant
%   has to also stop the |\unexpanded| expansion.
%    \begin{macrocode}
\long\def\ekvc@safehash@#1{\ekvc@err@empty@hash{}}
\long\def\ekvc@fasthash@#1\ekvc@stop{\ekvc@err@empty@hash}
\long\def\ekvc@safesplithash@#1#2{\ekvc@err@empty@hash#2{#1}{}}
\long\def\ekvc@fastsplithash@#1\ekvc@stop#2{\ekvc@err@empty@hash#2{#1}{}}
%    \end{macrocode}
% \end{macro}
%
% \begin{macro}{\ekvcSecondaryKeys}
%   The secondary keys are defined pretty similar to the way the originals are,
%   but here we also introduce some key types (those have a |@t@| in their name)
%   additionally to the prefixes.
%    \begin{macrocode}
\protected\long\def\ekvcSecondaryKeys#1#2%
  {%
    \edef\ekvc@set{\string#1}%
    \ekvparse\ekvc@err@value@required\ekvcSecondaryKeys@a{#2}%
  }
\protected\def\ekvcSecondaryKeys@a#1%
  {%
    \let\ekvc@long\ekv@empty
    \ekvc@ifspace{#1}%
      {\ekvcSecondaryKeys@b#1\ekvc@stop}%
      {\ekvc@err@missing@type{#1}\@gobble}%
  }
\protected\def\ekvcSecondaryKeys@b#1 #2\ekvc@stop
  {%
    \ekv@ifdefined{ekvc@p@#1}%
      {\csname ekvc@p@#1\endcsname}%
      {%
        \ekv@ifdefined{ekvc@t@#1}%
          {\csname ekvc@t@#1\endcsname}%
          {\ekvc@err@unknown@keytype{#1}\@firstoftwo\@gobble}%
      }%
      {#2}%
  }
%    \end{macrocode}
% \end{macro}
%
% \begin{macro}{\ekvcChange}
%   This can be used to change the defaults of an \expkvc\ defined macro. It
%   checks whether there is a set with the correct name and that the macro is
%   defined. If both is true the real work is done by |\ekvc@change|.
%    \begin{macrocode}
\protected\long\def\ekvcChange#1%
  {%
    \ekvifdefinedset{\string#1}%
      {%
        \ekvc@ifdefined#1%
          {\ekvc@change#1}%
          {\ekvc@err@no@key@macro#1\@gobble}%
      }%
      {\ekvc@err@no@key@macro#1\@gobble}%
  }
%    \end{macrocode}
% \end{macro}
%
% \begin{macro}[internal]
%   {\ekvc@change,\ekvc@change@a,\ekvc@change@b,\ekvc@change@c}
%   First we need to see whether the macro is currently |\long|. For this we get
%   the meaning and will parse it. |#1| is the macro name in which we want to
%   change the defaults.
%    \begin{macrocode}
\protected\def\ekvc@change#1%
  {\expandafter\ekvc@change@a\meaning#1\ekv@stop#1}
%    \end{macrocode}
%   A temporary definition to get the stringified |macro:|. |##1| will be the
%   list of prefixes, we don't care for the exact contents of |##2| and |##3|.
%    \begin{macrocode}
\def\ekvc@change@a#1%
  {%
    \protected\def\ekvc@change@a##1#1##2->##3\ekv@stop
      {%
        \ekvc@change@iflong{##1}%
          {\ekvc@change@b{}}%
          {\ekvc@change@b{\long}}%
      }%
  }
\expandafter\ekvc@change@a\expandafter{\detokenize{macro:}}
%    \end{macrocode}
%   Next we expand the macro once to get its contents (including the current
%   default values with their markers). |#1| is either |\long| or empty, |#2| is
%   the macro.
%    \begin{macrocode}
\protected\def\ekvc@change@b#1#2%
  {\expandafter\ekvc@change@c\expandafter{#2{##1}}{#1}#2}
%    \end{macrocode}
%   Here we place an unbalanced closing brace after the expansion of the macro.
%   Then we just parse the \kv-list with |\ekvset|, that will exchange the
%   values behind the markers. Once those are changed we give control to
%   |\ekvc@change@d|. The |\ekvset| step might horribly fail if the user defined
%   some keys that don't behave nice. |#1| is the expansion of the macro, |#2|
%   is either |\long| or empty, |#3| is the macro, and |#4| is the \kv-list
%   containing the new defaults.
%    \begin{macrocode}
\ekv@exparg{\protected\long\def\ekvc@change@c#1#2#3#4}%
  {%
    \expandafter\iffalse\expandafter{\expandafter\fi
      \ekvset{\string#3}{#4}%
      \ekvc@change@d{#2}{#3}%
      #1%
    }%
  }
%    \end{macrocode}
%   The final step needs to put an unbalanced opening brace after |\def|. We do
%   that with the help of a temporary macro which stores everything necessary
%   for |\def| and expand an |\iffalse}\fi| construct to leave a single
%   opening brace. |#1| will be either empty or |\long| and |#2| is the macro.
%   Each of the macros defined with \expkvc\ takes exactly one parameter, so we
%   put that here as |####1|.
%    \begin{macrocode}
\protected\def\ekvc@change@d#1#2%
  {%
    \def\ekvc@tmp{#1\def#2####1}%
    \expandafter\ekvc@tmp\expandafter{\iffalse}\fi
  }
%    \end{macrocode}
% \end{macro}
%
% \begin{macro}[internal]{\ekvc@change@iflong,\ekvc@change@iflong@}
%   Checking whether a string contains the string representation of |\long| can
%   be done by gobbling everything up to the first |\long| and checking whether
%   the result is completely empty. We need a temporary macro to get the result
%   of |\string\long| inside the definitions.
%    \begin{macrocode}
\def\ekvc@change@iflong#1%
  {%
    \protected\def\ekvc@change@iflong##1%
      {\expandafter\ekv@ifempty\expandafter{\ekvc@change@iflong@##1#1}}%
    \def\ekvc@change@iflong@##1#1{}
  }
\expandafter\ekvc@change@iflong\expandafter{\string\long}
%    \end{macrocode}
% \end{macro}
%
% \begin{macro}{\ekvcPass}
%   This macro can be used to pass a value to a key of some macro (this way
%   more complicated key codes are possible that in the end pass processed
%   values on to some macro). The implemantation is pretty straight forward.
%    \begin{macrocode}
\long\def\ekvcPass#1#2%
  {%
    \ekvifdefined{\string#1}{#2}%
      {\csname\ekv@name{\string#1}{#2}\endcsname}%
      {\ekvc@err@unknown@key@or@macro{#1}{#2}\@gobble}%
  }
%    \end{macrocode}
% \end{macro}
%
%
% \subsubsection{Secondary Key Types}
%
% \begin{macro}[internal]{\ekvc@p@long,\ekvc@after@ptype}
%   The prefixes are pretty straight forward again. Just set |\ekvc@long| and
%   forward to the |@t@| type.
%    \begin{macrocode}
\protected\def\ekvc@p@long#1%
  {%
    \ekvc@ifspace{#1}%
      {%
        \let\ekvc@long\long
        \ekvc@after@ptype#1\ekvc@stop
      }%
      {\ekvc@err@missing@type{long #1}\@gobble}%
  }
\protected\def\ekvc@after@ptype#1 #2\ekvc@stop
  {%
    \ekv@ifdefined{ekvc@t@#1}%
      {\csname ekvc@t@#1\endcsname{#2}}%
      {\ekvc@err@unknown@keytype{#1}\@gobble}%
  }
%    \end{macrocode}
% \end{macro}
%
% \begin{macro}[internal]
%   {
%     \ekvc@t@meta, \ekvc@t@nmeta,
%     \ekvc@type@meta, \ekvc@type@meta@a, \ekvc@type@meta@b
%   }
%   The |meta| and |nmeta| key types use a nested |\ekvset| to set other keys in
%   the same macro's \set.
%    \begin{macrocode}
\protected\def\ekvc@t@meta
  {%
    \edef\ekvc@tmp{\ekvc@set}%
    \expandafter\ekvc@type@meta\expandafter{\ekvc@tmp}\ekvc@long{##1}\ekvlet
  }
\protected\def\ekvc@t@nmeta#1%
  {%
    \ekvc@assert@not@long{nmeta #1}%
    \edef\ekvc@tmp{\ekvc@set}%
    \expandafter\ekvc@type@meta\expandafter{\ekvc@tmp}{}{}\ekvletNoVal{#1}%
  }
\protected\long\def\ekvc@type@meta#1#2#3#4#5#6%
  {%
    \expandafter\ekvc@type@meta@a\expandafter{\ekvset{#1}{#6}}{#2}{#3}%
    #4\ekvc@set{#5}\ekvc@tmp
  }
\protected\def\ekvc@type@meta@a
  {%
    \expandafter\ekvc@type@meta@b\expandafter
  }
\protected\long\def\ekvc@type@meta@b#1#2#3%
  {%
    #2\def\ekvc@tmp#3{#1}%
  }
%    \end{macrocode}
% \end{macro}
%
% \begin{macro}[internal]{\ekvc@t@alias}
%   |alias| just checks whether there is a key and/or |NoVal| key defined with
%   the target name and |\let| the key to those.
%    \begin{macrocode}
\protected\def\ekvc@t@alias#1#2%
  {%
    \ekvc@assert@not@long{alias #1}%
    \let\ekvc@tmp\@firstofone
    \ekvifdefined\ekvc@set{#2}%
      {%
        \ekvletkv\ekvc@set{#1}\ekvc@set{#2}%
        \let\ekvc@tmp\@gobble
      }%
      {}%
    \ekvifdefinedNoVal\ekvc@set{#2}%
      {%
        \ekvletkvNoVal\ekvc@set{#1}\ekvc@set{#2}%
        \let\ekvc@tmp\@gobble
      }%
      {}%
    \ekvc@tmp{\ekvc@err@unknown@key{#2}}%
  }
%    \end{macrocode}
% \end{macro}
%
% \begin{macro}[internal]{\ekvc@t@default}
%   The |default| key can be used to set a |NoVal| key for an existing key. It
%   will just pass the \val\ to the key macro of that other key.
%    \begin{macrocode}
\protected\long\def\ekvc@t@default#1#2%
  {%
    \ekvifdefined\ekvc@set{#1}%
      {%
        \ekvc@assert@not@long{default #1}%
        \edef\ekvc@tmp
          {%
            \unexpanded\expandafter
              {\csname\ekv@name\ekvc@set{#1}\endcsname{#2}}%
          }%
        \ekvletNoVal\ekvc@set{#1}\ekvc@tmp
      }%
      {\ekvc@err@unknown@key{#1}}%
  }
%    \end{macrocode}
% \end{macro}
%
% \begin{macro}[internal]{\ekvc@t@flag-bool}
%    \begin{macrocode}
\protected\expandafter\def\csname ekvc@t@flag-bool\endcsname#1#2%
  {%
    \ekvc@assert@not@long{flag-bool #1}%
    \unless\ifdefined#2\ekvcFlagNew#2\fi
    \ekvdef\ekvc@set{#1}%
      {%
        \ekv@ifdefined{ekvc@flag@set@##1}%
          {%
            \csname ekvc@flag@set@##1\expandafter\endcsname
              \ekvcFlagHeight#2\ekv@stop#2%
          }%
          {\ekvc@err@invalid@bool{##1}}%
      }%
  }
%    \end{macrocode}
% \end{macro}
%
% \begin{macro}[internal]
%   {\ekvc@t@flag-true,\ekvc@t@flag-false,\ekvc@t@flag-raise,\ekvc@type@flag}
%    \begin{macrocode}
\protected\def\ekvc@type@flag#1#2#3#4%
  {%
    \ekvc@assert@not@long{flag-#1 #3}%
    \unless\ifdefined#4\ekvcFlagNew#4\fi
    \ekv@exparg{\ekvdefNoVal\ekvc@set{#3}}{#2#4}%
  }
\protected\expandafter\def\csname ekvc@t@flag-true\endcsname
  {\ekvc@type@flag{true}\ekvcFlagSetTrue}
\protected\expandafter\def\csname ekvc@t@flag-false\endcsname
  {\ekvc@type@flag{false}\ekvcFlagSetFalse}
\protected\expandafter\def\csname ekvc@t@flag-raise\endcsname
  {\ekvc@type@flag{raise}\ekvcFlagRaise}
%    \end{macrocode}
% \end{macro}
%
%
% \subsubsection{Flags}
%
% The basic idea of flags is to store information by the fact that \TeX\
% expandably assigns the meaning |\relax| to undefined control sequences which
% were built with |\csname|. This mechanism is borrowed from \pkg{expl3}.
%
% \begin{macro}[internal]{\ekvc@flag@name,\ekvc@flag@namescheme}
%   Flags follow a simple naming scheme which we define here. |\ekvc@flag@name|
%   will store the name of an internal function that is used to build names of
%   the second naming scheme defined by |\ekvc@flag@namescheme|.
%    \begin{macrocode}
\def\ekvc@flag@name{ekvcf\string}
\def\ekvc@flag@namescheme#1#2{ekvch#2#1}
%    \end{macrocode}
% \end{macro}
%
% \begin{macro}{\ekvcFlagHeight}
%   For semantic reasons we use |\number| with another name.
%    \begin{macrocode}
\let\ekvcFlagHeight\number
%    \end{macrocode}
% \end{macro}
%
% \begin{macro}{\ekvcFlagNew}
%   This macro defines a new flag. It stores the function build with the
%   |\ekvc@flag@name| naming scheme after the internal function
%   |\ekvc@flag@height| that'll determine the current flag height. It'll also
%   define the macro named via |\ekvc@flag@name| to build names according to
%   |\ekvc@flag@namescheme|.
%    \begin{macrocode}
\protected\def\ekvcFlagNew#1%
  {%
    \edef#1%
      {%
        \unexpanded{\ekvc@flag@height}%
          \unexpanded\expandafter{\csname\ekvc@flag@name#1\endcsname}%
      }%
    \ekv@expargtwice
      {\expandafter\def\csname\ekvc@flag@name#1\endcsname##1}%
      {\expandafter\ekvc@flag@namescheme\expandafter{\string#1}{##1}}%
  }
%    \end{macrocode}
% \end{macro}
%
% \begin{macro}[internal]{\ekvc@flag@height,\ekvc@flag@height@}
%   This macro gets the height of a flag by a simple loop. The first loop
%   iteration differs a bit from the following in that it doesn't have to get
%   the current iteration count. The space at the end of |\ekvc@flag@height|
%   ends the |\number| evaluation.
%    \begin{macrocode}
\def\ekvc@flag@height#1%
  {%
    \ifcsname#10\endcsname
      \ekvc@flag@height@1\ekv@stop#1%
    \fi
    \@firstofone{0} % leave this space
  }
\def\ekvc@flag@height@#1\ekv@stop#2\fi\@firstofone#3%
  {%
    \fi
    \ifcsname#2{#1}\endcsname
      \expandafter\ekvc@flag@height@\the\numexpr#1+1\relax\ekv@stop#2%
    \fi
    \@firstofone{#1}%
  }
%    \end{macrocode}
% \end{macro}
%
% \begin{macro}{\ekvcFlagRaise}
%   Raising a flag simply means letting the |\ekvc@flag@namescheme| macro for
%   the current height to relax. The result of raising a flag is that its height
%   is bigger by $1$.
%    \begin{macrocode}
\ekv@exparg{\def\ekvcFlagRaise#1}%
  {%
    \expandafter\expandafter\expandafter\@gobble\expandafter
      \csname\ekvc@flag@namescheme{\string#1}{\ekvcFlagHeight#1}\endcsname
  }
%    \end{macrocode}
% \end{macro}
%
% \begin{macro}{\ekvcFlagSetTrue,\ekvcFlagSetFalse}
% \begin{macro}[internal]{\ekvc@flag@set@true,\ekvc@flag@set@false}
%   A flag is considered true if its current height is odd, and as false if it
%   is even. Therefore |\ekvcFlagSetTrue| and |\ekvcFlagSetFalse| only need to
%   raise the flag if the opposing boolean value is the current one.
%    \begin{macrocode}
\def\ekvcFlagSetTrue#1%
  {\expandafter\ekvc@flag@set@true\ekvcFlagHeight#1\ekv@stop#1}
\def\ekvcFlagSetFalse#1%
  {\expandafter\ekvc@flag@set@false\ekvcFlagHeight#1\ekv@stop#1}
%    \end{macrocode}
%   We can expand |\ekvc@flag@namescheme| at definition time here, which is why
%   we're using a temporary definition to set up |\ekvc@flag@set@true| and
%   |\ekvc@flag@set@false|.
%    \begin{macrocode}
\def\ekvc@flag@set@true#1%
  {%
    \def\ekvc@flag@set@true##1\ekv@stop##2%
      {%
        \ifodd##1
          \ekv@fi@gobble
        \fi
        \@firstofone{\expandafter\@gobble\csname#1\endcsname}%
      }%
    \def\ekvc@flag@set@false##1\ekv@stop##2%
      {%
        \ifodd##1
          \ekv@fi@firstofone
        \fi
        \@gobble{\expandafter\@gobble\csname#1\endcsname}%
      }%
  }
\expandafter\ekvc@flag@set@true\expandafter
  {\ekvc@flag@namescheme{\string#2}{#1}}
%    \end{macrocode}
% \end{macro}
% \end{macro}
%
% \begin{macro}{\ekvcFlagIf}
%   As already explained, truthiness is defined as a flag's height is odd, so we
%   just branch accordingly here.
%    \begin{macrocode}
\def\ekvcFlagIf#1%
  {%
    \ifodd#1%
      \ekv@fi@firstoftwo
    \fi
    \@secondoftwo
  }
%    \end{macrocode}
% \end{macro}
%
% \begin{macro}{\ekvcFlagIfRaised}
%   This macro uses flags as a switch, if a flag's current height is bigger than
%   $0$ this test yields true.
%    \begin{macrocode}
\ekv@exparg{\def\ekvcFlagIfRaised#1}%
  {%
    \expandafter\ifcsname\ekvc@flag@namescheme{\string#1}0\endcsname
      \ekv@fi@firstoftwo
    \fi
    \@secondoftwo
  }
%    \end{macrocode}
% \end{macro}
%
% \begin{macro}{\ekvcFlagReset}
% \begin{macro}[internal]{\ekvc@flag@reset,\ekvc@flag@reset@}
%   Resetting works by locally letting all the defined internal macros named
%   after |\ekvc@flag@namescheme| to undefined.
%    \begin{macrocode}
\protected\def\ekvcFlagReset#1%
  {\expandafter\ekvc@flag@reset\csname\ekvc@flag@name#1\endcsname}
\protected\def\ekvc@flag@reset#1%
  {%
    \ifcsname#10\endcsname
      \expandafter\let\csname#10\endcsname\ekvc@undefined
      \ekvc@flag@reset@1\ekv@stop#1%
    \fi
  }
\protected\def\ekvc@flag@reset@#1\ekv@stop#2\fi
  {%
    \fi
    \ifcsname#2{#1}\endcsname
      \expandafter\let\csname#2{#1}\endcsname\ekvc@undefined
      \expandafter\ekvc@flag@reset@\the\numexpr#1+1\relax\ekv@stop#2%
    \fi
  }
%    \end{macrocode}
% \end{macro}
% \end{macro}
%
% \begin{macro}{\ekvcFlagGetHeight}
% \begin{macro}[internal]{\ekvc@flag@get@height@single}
%   These are just small helpers, first getting the height of the flag and then
%   passing it on to the user supplied code.
%    \begin{macrocode}
\def\ekvcFlagGetHeight#1%
  {\expandafter\ekvc@flag@get@height@single\ekvcFlagHeight#1\ekv@stop}
\long\def\ekvc@flag@get@height@single#1\ekv@stop#2{#2{#1}}
%    \end{macrocode}
% \end{macro}
% \end{macro}
%
% \begin{macro}{\ekvcFlagGetHeights}
% \begin{macro}[internal]
%   {\ekvc@flag@get@heights,\ekvc@flag@get@heights@,\ekvc@flag@get@heights@done}
%   This works by a simple loop that stops at |\ekv@stop|. As long as that
%   marker isn't hit, get the next flags height and put it into a list after
%   |\ekv@stop|. |\ekvc@flag@get@heights@| uses the same marker name for the
%   end of the height, which shouldn't clash in any case. Once we're done we
%   remove the remainder of the current iteration and leave the user supplied
%   code in the input stream with all the flags' heights as a single argument.
%    \begin{macrocode}
\def\ekvcFlagGetHeights#1%
  {%
    \ekvc@flag@get@heights#1\ekv@stop{}%
  }
\def\ekvc@flag@get@heights#1%
  {%
    \ekv@gobbleto@stop#1\ekvc@flag@get@heights@done\ekv@stop
    \expandafter\ekvc@flag@get@heights@\ekvcFlagHeight#1\ekv@stop
  }
\def\ekvc@flag@get@heights@#1\ekv@stop#2\ekv@stop#3%
  {\ekvc@flag@get@heights#2\ekv@stop{#3{#1}}}
\long\def\ekvc@flag@get@heights@done
    \ekv@stop
    \expandafter\ekvc@flag@get@heights@\ekvcFlagHeight\ekv@stop\ekv@stop#1#2%
  {#2{#1}}
%    \end{macrocode}
% \end{macro}
% \end{macro}
%
%
% \subsubsection{Helper Macros}
%
% \begin{macro}[internal]{\ekvc@ifspace,\ekvc@ifspace@}
% A test which can be reduced to an if-empty by gobbling everything up to the
% first space.
%    \begin{macrocode}
\long\def\ekvc@ifspace#1%
  {%
    \ekvc@ifspace@#1 \ekv@ifempty@B
      \ekv@ifempty@false\ekv@ifempty@A\ekv@ifempty@B\@firstoftwo
  }
\long\def\ekvc@ifspace@#1 % keep this space
  {%
    \ekv@ifempty@\ekv@ifempty@A
  }
%    \end{macrocode}
% \end{macro}
%
%
% \subsubsection{Assertions}
% \begin{macro}[internal]{\ekvc@assert@not@long}
%   Some keys don't want to be |long| and we have to educate the user, so let's
%   throw an error if someone wanted these to be long.
%    \begin{macrocode}
\long\def\ekvc@assert@not@long#1{\ifx\ekvc@long\long\ekvc@err@no@long{#1}\fi}
%    \end{macrocode}
% \end{macro}
%
%
% \subsubsection{Messages}
%
% \begin{macro}[internal]
%   {
%     \ekvc@err@toomany,\ekvc@err@value@required,\ekvc@err@missing@type,
%     \ekvc@err@already@defined,\ekvc@err@no@key@macro
%   }
%   Boring unexpandable error messages.
%    \begin{macrocode}
\protected\def\ekvc@err@toomany#1%
  {%
    \errmessage{expkv-cs Error: Too many keys for macro `\string#1'}%
  }
\protected\def\ekvc@err@value@required#1%
  {%
    \errmessage{expkv-cs Error: Missing value for key `\unexpanded{#1}'}%
  }
\protected\def\ekvc@err@missing@type#1%
  {%
    \errmessage
      {expkv-cs Error: Missing type for secondary key `\unexpanded{#1}'}%
  }
\protected\def\ekvc@err@no@long#1%
  {%
    \errmessage
      {expkv-cs Error: prefix `long' not accepted for `\unexpanded{#1}'}%
  }
\protected\def\ekvc@err@already@defined#1%
  {%
    \errmessage{expkv-cs Error: Macro `\string#1' already defined}%
  }
\protected\def\ekvc@err@unknown@keytype#1%
  {%
    \errmessage{expkv-cs Error: Unknown key type `\unexpanded{#1}'}%
  }
\protected\def\ekvc@err@unknown@key#1%
  {%
    \errmessage
      {expkv-cs Error: Unknown key `\unexpanded{#1}' for macro `\ekvc@set'}%
  }
\long\def\ekvc@err@no@key@macro#1%
  {\errmessage{expkv-cs Error: \string#1 is no key=val macro}}
%    \end{macrocode}
% \end{macro}
%
% \begin{macro}[internal]{\ekvc@err}
%   We need a way to throw error messages expandably in some contexts.
%    \begin{macrocode}
\ekv@exparg{\long\def\ekvc@err#1}{\ekverr{expkv-cs}{#1}}
%    \end{macrocode}
% \end{macro}
%
% \begin{macro}[internal]
%   {
%     \ekvc@err@unknown@hash,\ekvc@err@empty@hash,\ekvc@err@missing@hash,
%     \ekvc@err@invalid@bool
%   }
%   And here are the expandable error messages.
%    \begin{macrocode}
\long\def\ekvc@err@unknown@hash#1{\ekvc@err{unknown hash `#1'}}
\long\def\ekvc@err@missing@hash#1{\ekvc@err{hash `#1' not found}}
\long\def\ekvc@err@empty@hash{\ekvc@err{empty hash}}
\def\ekvc@err@invalid@bool#1{\ekvc@err{invalid boolean value `#1'}}
\long\def\ekvc@err@unknown@key@or@macro#1#2%
  {\ekvc@err{unknown key `#2' for macro #1}}
%    \end{macrocode}
% \end{macro}
%
%
% Now everything that's left is to reset the category code of |@|.
%    \begin{macrocode}
\catcode`\@=\ekvc@tmp
%    \end{macrocode}
%
% \gobbledocstriptag
%</tex>
%^^A=<<
%
%^^A=<<
%
% \end{implementation}^^A=<<
%
% \clearpage
% \PrintIndex
%
\endinput
%
^^A vim: ft=tex fdm=marker fmr=>>=,=<<
